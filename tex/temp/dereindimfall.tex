%!TEX root = ../main.tex

\section{Der eindimensionale Fall} % (fold)
\label{sec:der_eindimensionale_fall}

Wir betrachten zunächst den eindimensionalen Fall, das heißt $\Omega \subset \mathbb{R}$ sei ein Intervall, ohne Einschränkung sei $\Omega = [0, 1]$.

\subsubsection{Entwicklung von $\omega$} % (fold)
\label{ssub:entwicklung_von_}

Um die Aussagen von \cite{Kunoth:2013ef} zu verwenden, wollen wir den Operator $A \eta = - c \Delta \eta + \omega \eta$ als affin parametrischen Operator der Form
\begin{equation}
    A(\sigma) = A_{0} + \sum_{j \geq 1} \sigma_{j} A_{j}
\end{equation}
auffassen.
Dazu schreiben wir den Reaktionsterm $\omega$ als Reihenentwicklung auf, das heißt
\begin{equation}
    \label{eq:omega_reihenentwicklung}
    \omega \colon \Omega \to \mathbb{R}, \quad x \mapsto \omega(x) = \sum_{j \geq 1} \sigma_{j} \varphi_{j}(x).
\end{equation}
Auf die Wahl der Funktionen $\{ \varphi_{j} \}_{j \geq 1}$ gehen wir später genauer ein.

Eine offensichtliche Aufteilung des Operators $A$ ist damit gegeben durch
\begin{equation}
    A_{0} = - c \Delta, \qquad
    A_{j} = \varphi_{j}, \quad j \geq 1.
\end{equation}
Die zugehörigen Bilinearformen lassen sich ebenfalls direkt angeben, es gilt
\begin{equation}
    a_{0}(\eta, \zeta) = \skprod{\grad \eta}{\grad \zeta}_{L_{2}(\Omega)}, \qquad a_{j}(\eta, \zeta) = \skprod{\varphi_{j} \eta}{\zeta}_{L_{2}(\Omega)}, \quad j \geq 1.
\end{equation}

% subsubsection entwicklung_von_ (end)

\subsubsection{Wahl der Funktionen $\varphi_{j}$} % (fold)
\label{ssub:wahl_der_funktionen_}

Die Wahl der Funktionen $\varphi_{j}$ entscheidet über die Konvergenz und weitere Eigenschaften der Reihenentwicklung \eqref{eq:omega_reihenentwicklung}.

Als ersten Ansatz wählen wir
\begin{equation}
    \varphi_{j} = \frac{1}{(j \pi)^{1 + \epsilon}} \sin(j \pi \cdot).
\end{equation}
Da wir damit stets $\varphi_{j}(0) = \varphi_{j}(1) = 0$ für alle $j \geq 1$ erhalten, bei $\omega$ aber nicht nur homogene Randvorgaben betrachten wollen, wählen wir zusätzlich
\begin{equation}
    \varphi_{0} = c \in \mathbb{R}_{+}.
\end{equation}

Wir weisen nun einige Eigenschaften für diese Wahl der $\varphi_{j}$ nach.

\begin{Lemma}[Orthogonalität]
    Es gilt für $\varphi_{j}$, $j \geq 1$,
    \begin{equation}
        \skprod{\varphi_{j}}{\varphi_{k}}_{H^{1}(\Omega)} = \begin{cases}
            \frac{\sqrt{1 + (j \pi)^{2}}}{\sqrt 2 (j \pi)^{1 + \epsilon}},   &j = k \\
            0,          &j \neq k,
        \end{cases}
    \end{equation}
    das heißt $\{ \varphi_{j} \}_{j \geq 1}$ bilden ein Orthogonalsystem in $H^{1}(\Omega)$.
\end{Lemma}

\begin{Lemma}[Normen]
    Es gilt
    \begin{equation}
        \norm{\varphi_{0}}_{L_{\infty(\Omega)}} = c \quad \text{und} \quad \norm{\varphi_{j}}_{L_{\infty(\Omega)}} = \frac{1}{(j \pi)^{1 + \epsilon}}, \quad j \geq 1,
    \end{equation}
    sowie
    \begin{equation}
        \begin{aligned}
            \norm{\varphi_{j}}_{L_{2}(\Omega)}  = \frac{1}{\sqrt{2}(j \pi)^{1+ \epsilon}}, \quad
            \norm{\varphi'_{j}}_{L_{2}(\Omega)} = \frac{1}{\sqrt{2}(j \pi)^{\epsilon}}, \quad
            \norm{\varphi_{j}}_{H^{1}(\Omega)}  = \frac{\sqrt{1 + (j \pi)^{2}}}{\sqrt 2 (j \pi)^{1 + \epsilon}},
        \end{aligned}
    \end{equation}
    für $j \geq 1$, und
    \begin{equation}
        \begin{aligned}
            \norm{\varphi_{0}}_{L_{2}(\Omega)}  = c, \quad
            \norm{\varphi'_{0}}_{L_{2}(\Omega)} = 0, \quad
            \norm{\varphi_{0}}_{H^{1}(\Omega)}  = c.
        \end{aligned}
    \end{equation}
\end{Lemma}

\begin{Lemma}[Konvergenz der Reihenentwicklung von $\omega$]
    Unter gewissen, noch zu klärenden, Voraussetzungen konvergiert $\omega(x) = \sum_{j \geq 1} \sigma_{j} \varphi_{j}(x)$ in $H^{1}(\Omega)$.

    \begin{Beweis}

    \end{Beweis}
\end{Lemma}


% subsubsection wahl_der_funktionen_ (end)

\subsubsection{Nachrechnen der Eigenschaften für die Regularität und so} % (fold)
\label{ssub:nachrechnen_von_thref_thm_kunoth_assumption2}

% subsubsection nachrechnen_von_thref_thm_kunoth_assumption2 (end)

% section der_eindimensionale_fall (end)
