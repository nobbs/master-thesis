%!TEX root = ../main.tex

\chapter{Propagator-Differentialgleichung} % (fold)
\label{chapter:propagator_differentialgleichung}

Wir greifen nun die in \cref{chapter:einleitung} eingeführten parabolischen Differentialgleichungen \cref{eq:forward_propagator,eq:backward_propagator} auf, welche von den beiden Propagatoren $q$ und $q^{\dagger}$ erfüllt werden.
Im Weiteren Verlauf der Arbeit verwenden wir den Begriff \emph{Propagator"=Differentialgleichung}, den wir noch genauer definieren werden, wenn wir uns
auf die genannten partiellen Differentialgleichungen beziehen.

In diesem Kapitel konkretisieren wir zunächst diese Propagator"=Differentialgleichungen, schaffen geeignete Rahmenbedingungen und leiten eine schwache Raum"=Zeit"=Formulierung her.
Anschließend werden die in den Propagator"=Differentialgleichungen auftretenden Felder $\omega_{i}$ parametrisiert und als Grundlage für eine parametrische schwache Formulierung verwendet.
Für diese weisen wir abschließend nach, dass sie sachgemäß gestellt ist und unter zusätzlichen Bedingungen eine gewisse Regularität bezüglich der Parameter aufweist.


\section{Eine Raum-Zeit-Variationsformulierung}
\label{section:raum_zeit_variationsformulierung}

Wir wollen nun zunächst die Rahmenbedingungen festlegen und die aus der Einführung bekannten Propagator"=Differentialgleichungen in einem allgemeineren Rahmen formulieren.
Dabei halten wir an dem Fall zweier Felder fest, wobei diese Einschränkung nicht notwendig ist, da die nachfolgenden Ergebnisse in gleicher Weise auch für jede andere endliche Felderanzahl nachgewiesen werden können.
Weiter sei an dieser Stelle angemerkt, dass es ausreicht sich auf die Betrachtung des Vorwärts"=Propagators \cref{eq:forward_propagator} zu beschränken, da der Rückwärts"=Propagator \cref{eq:backward_propagator} durch die einfache Transformation $s \mapsto 1 - s$ auf die selbe Form, lediglich mit vertauschen Rollen bei den Feldern, gebracht werden kann.

Seien nun $0 < T_{f} < T < \infty$ reelle Konstanten und $I = [0, T]$ ein reelles Intervall, welches wir in die beiden disjunkten Teilintervalle $I_{1} = [0, T_{f})$ und $I_{2} = [T_{f}, T]$ zerlegen.
Wir interpretieren die Größe $t \in I$ als Zeit, wobei dies einen rein notationellen und keinen physikalischen Hintergrund hat.
Weiter sei $\Omega \subset \mathbb{R}^{n}$, $n \in \mathbb{N}$, ein beschränktes Gebiet, das heißt offen, nichtleer, zusammenhängend und beschränkt, welches einen Lipschitz"=Rand besitzt.

Zunächst wollen wir den Begriff der Felder und der Propagator-Differentialgleichung konkretisieren.
Dies dient vor allem der einfacheren Notation und Benennung, weswegen wir den definierten Abbildungen an dieser Stelle möglichst wenige einschränkende Bedingungen auferlegen wollen.

\begin{Definition}
\label{definition:feld_raum_zeit_feld}
    Unter einem \emph{Feld} verstehen wir eine $L_{\infty}(\Omega)$-Abbildung $w$.
    Seien $w_{1}, w_{2}$ Felder und $\chi_{I_{1}}, \chi_{I_{2}}$ charakteristische Funktionen.
    Wir bezeichnen die Abbildung
    \begin{equation}
    \label{eq:raum_zeit_feld}
        \omega \colon I \times \Omega \to \mathbb{R}, \quad \omega(t, \vec{x}) =
        w_{1}(\vec{x}) \chi_{I_{1}}(t) + w_{2}(\vec{x}) \chi_{I_{2}}(t)
        =
        \begin{cases}
            w_{1}(\vec{x}), & t < T_{f}, \\
            w_{2}(\vec{x}), & t \geq T_{f}.
        \end{cases}
    \end{equation}
    als \emph{Raum"=Zeit"=Feld}.
\end{Definition}

\begin{Definition}
\label{definition:propagator_differentialgleichung}
    Seien $\omega$ ein Raum"=Zeit"=Feld wie in \cref{eq:raum_zeit_feld}, $u_{0} \colon \Omega \to \mathbb{R}$ eine Anfangsbedingung, $g \colon I \times \Omega \to \mathbb{R}$ ein Quellterm sowie $c \in \mathbb{R}_{+}$ und $\mu \in \mathbb{R}$ Konstanten.
    Als \emph{Propagator"=Differentialgleichung} bezeichnen wir die parabolische partielle Differentialgleichung
    \begin{equation}
    \label{eq:propagator_differentialgleichung}
        \left\{
        \begin{aligned}
            u_{t}(t, \vec{x}) - c \Delta u(t, \vec{x}) + \omega(t, \vec{x}) u(t, \vec{x}) + \mu u(t, \vec{x}) &= g(t, \vec{x}) \quad &&\text{auf}~I \times \Omega,\\
            u(0, \vec{x}) &= u_{0}(\vec{x}) \quad &&\text{auf}~\Omega, \\
            u(t, \vec{x})~\text{erfüllt eine vorgegebene Randbe}&\text{dingung} &&\text{auf}~I \times \partial \Omega.
        \end{aligned}
        \right.
    \end{equation}
\end{Definition}

Wie in der Einleitung erwähnt, hat der Mittelwert der Felder keinen Einfluss auf das Ergebnis des dort beschriebenen Iterationsverfahrens, weswegen wir den zusätzlichen Term $\mu u(t, \vec{x})$ einführen können.
Dieser wird sich bei den folgenden theoretischen Untersuchungen und numerischen Umsetzung als nützlich erweisen.

Unser Ziel ist es nun, eine Raum"=Zeit"=Variationsformulierung der Propagator"=Differentialgleichung aus obiger Definition herzuleiten.
Diese wird uns als Ausgangspunkt für unsere numerischen Verfahren dienen, weswegen wir uns auch bei der theoretischen Arbeit auf diese konzentrieren werden.

Zunächst führen wir aber eine Einschränkung der möglichen Randbedingungen durch.
Von größtem Interesse sind für uns, bedingt durch die Motivation der parabolischen Differentialgleichung in \cref{chapter:einleitung}, der Fall homogener Dirichlet"=Randbedingungen, also $u(t, \vec{x}) = 0$ auf $I \times \partial \Omega$, sowie der Fall periodischer Randbedingungen.
Letztere werden am Ende dieses Kapitels wieder aufgegriffen, während wir uns im Rest der Ausführungen auf den Fall homogener Dirichlet"=Randbedingungen beschränken.

Bei der Herleitung der Raum"=Zeit"=Variationsformulierung der Propagator"=Differentialgleichung werden wir schrittweise vorgehen und zunächst den stationären Fall betrachten, bevor wir darauf aufbauend die schwache Raum"=Zeit"=Variationsformulierung erhalten.
Als Grundlage für die schwache Formulierung im stationären Fall verwenden wir die wohlbekannten Sobolev"=Räume.
Die folgende Bemerkung führt die notwendigen Notationen in diesem Zusammenhang ein.

\begin{Bemerkung}
\label{bemerkung:raeume_und_gelfand_tripel}
    Wir schreiben kurz $V = H^{1}_{0}(\Omega)$ und $H = L_{2}(\Omega)$ für den bekannten Sobolev- respektive Lebesgue"=Raum auf $\Omega$.
    Dabei handelt es sich jeweils um einen Hilbertraum; wir kennzeichnen durch den jeweiligen Index die entsprechenden Skalarprodukte $\skp{\blank}{\blank}{}$ und Normen $\norm{\blank}$.
    Als $V$-Norm wählen wir $\norm{\eta}_{V} = (\norm{\eta}^{2}_{H} + \norm{\grad \eta}^{2}_{H})^{1/2}$.
    Da $V$ dicht in $H$ eingebettet werden kann, definiert
    \begin{equation}
        V \denseinclusion H \cong H' \denseinclusion V' = (H^{1}_{0}(\Omega))' = H^{-1}(\Omega)
    \end{equation}
    ein Gelfand"=Tripel, vergleiche \cref{definition:gelfand_tripel}.
    Motiviert durch \cref{bemerkung:skalarprodukte_und_duality_pairing} verwenden wir die Schreibweise $\skp{\blank}{\blank}{V' \times V}$ auch für die duale Paarung auf $V' \times V$.
\end{Bemerkung}

Damit können wir nun die folgende Operatorfamilie und die zugehörigen Bilinearform definieren.

\begin{Definition}
\label{definition:operator_bilinearform_zeit}
    Seien $\omega$ ein Raum"=Zeit"=Feld sowie $c \in \mathbb{R}_{+}$ und $\mu \in \mathbb{R}$ Konstanten.
    Wir definieren für $t \in I$ eine Familie von Operatoren
    \begin{equation}
        \label{eq:operator_zeit}
        A(t) \colon V \to V', \quad  A(t) \eta = - c \Delta \eta + \omega(t, \blank) \eta + \mu \eta
    \end{equation}
    und eine zugehörige Familie von Bilinearformen
    \begin{equation}
        \label{eq:bilinearform_zeit}
        a(\blank, \blank; t) \colon V \times V \to \mathbb{R}, \quad  a(\eta, \zeta; t) = \skp{A(t)\eta}{\zeta}{V' \times V}.
    \end{equation}
\end{Definition}

\begin{Bemerkung}
\label{bemerkung:operator_bilinearform_riesz}
    Die Existenz der Bilinearform $a(\blank, \blank; t)$ zum jeweiligen Operator $A(t)$ lässt sich durch den Rieszschen Darstellungssatz begründen, siehe beispielsweise \cite[Theorem \S{}22.1]{Halmos:1957vd}.
\end{Bemerkung}

Aufgrund der gewählten Rahmenbedingungen können wir die Bilinearform $a(\blank, \blank; t)$ explizit angeben.
Dazu greifen wir unter anderem auf die wohlbekannte schwache Formulierung des Laplace"=Operators $- \Delta \colon V \to V'$ zurück.
Diese lässt sich durch die Wahl von $V$ als eine Bilinearform auf $V \times V$ der Form
\begin{equation}
    (\eta, \zeta) \mapsto \skp{- \Delta \eta}{\zeta}{V' \times V} = \skp{\grad \eta}{\grad \zeta}{H}
\end{equation}
schreiben.
Damit ergibt sich zusammen mit der Linearität der dualen Paarung
\begin{equation}
\label{eq:bilinearform_formel}
    \begin{aligned}
        a(\eta, \zeta; t)
            &= \skp{A(t) \eta}{\zeta}{V' \times V}
          \\&= \skp{-c\Delta \eta + \omega(t, \blank) \eta + \mu \eta}{\zeta}{V' \times V}
          \\&= - c\skp{\Delta \eta}{\zeta}{V' \times V} + \skp{\omega(t, \blank) \eta}{\zeta}{V' \times V} + \mu \skp{\eta}{\zeta}{V' \times V}
          \\&= c\skp{\grad \eta}{\grad \zeta}{H} + \skp{\omega(t, \blank) \eta}{\zeta}{H} + \mu \skp{\eta}{\zeta}{H},
    \end{aligned}
\end{equation}
wobei bei der Wechsel von dualer Paarung auf $V' \times V$ zum $H$-Skalarprodukt beim zweiten und dritten Summanden durch \cref{bemerkung:skalarprodukte_und_duality_pairing} und die Tatsache, dass mit $\eta \in V \subset H$ und der \cref{definition:feld_raum_zeit_feld} von $\omega$ auch $\omega(t, \blank) \eta \in H$ gilt, gerechtfertigt ist.

\begin{Bemerkung}
\label{bemerkung:raum_zeit_feld_norm_zeitunabhaengig}
    Nach Definition von $\omega$ in \cref{eq:raum_zeit_feld} gilt offenbar
    \begin{equation}
        \norm{\omega(t, \blank)}_{L_{\infty}(\Omega)} = \norm{w_{1}}_{L_{\infty}(\Omega)} \chi_{I_{1}}(t) + \norm{w_{2}}_{L_{\infty}(\Omega)} \chi_{I_{2}}(t),
    \end{equation}
    damit gilt wegen der Zerlegung $I = I_{1} \cupdot I_{2}$ insbesondere
    \begin{equation}
        \norm{\omega}_{L_{\infty}(I; L_{\infty}(\Omega))} = \esssup_{t \in I} \norm{\omega(t, \blank)}_{L_{\infty}(\Omega)} = \max\Set{ \norm{w_{1}}_{L_{\infty}(\Omega)}, \norm{w_{2}}_{L_{\infty}(\Omega)} } < \infty.
    \end{equation}
\end{Bemerkung}

Wir weisen nun zwei Eigenschaften für diesen Operator respektive die zugehörige Bilinearform nach, welche eine wichtige Rolle für die spätere Raum"=Zeit"=Variationsformulierung spielen werden.

\begin{Satz}
\label{satz:bilinearform_messbar_stetig_garding}
    Sei $a(\blank, \blank; t)$, $t \in I$, die Familie von Bilinearformen aus \cref{eq:bilinearform_formel}.
    Diese Bilinearformen erfüllen die folgenden Eigenschaften:
    \begin{thmenumerate}
        \item\label{satz:bilinearform_messbar_stetig_garding:messbar}
        \emph{Messbarkeit}: Die Abbildung $I \ni t \mapsto a(\eta, \zeta; t)$ ist messbar für alle $\eta, \zeta \in V$.
        \item\label{satz:bilinearform_messbar_stetig_garding:stetig}
        \emph{Stetigkeit:} Es gilt
        \begin{equation}
            \label{eq:bilinearform_messbar_stetig_garding:stetig}
            \abs{a(\eta, \zeta; t)} \leq \gamma_{a} \norm{\eta}_{V} \norm{\zeta}_{V}, \quad \fa \eta, \zeta \in V,
        \end{equation}
        mit Stetigkeitskonstante $\gamma_{a} = \max\Set{c, \norm{\omega}_{L_{\infty}(I; L_{\infty}(\Omega))} + \abs{\mu}} < \infty$.
        \item\label{satz:bilinearform_messbar_stetig_garding:garding}
        \emph{G\aa{}rding"=Ungleichung:} Es gilt
        \begin{equation}
            \label{eq:bilinearform_messbar_stetig_garding:garding}
            a(\eta, \eta; t) + \lambda \norm{\eta}_{H}^{2} \geq \alpha \norm{\eta}_{V}^{2}, \quad \fa \eta \in V,
        \end{equation}
        mit $\alpha = c \gamma_{\Omega}^{2} > 0$ und $\lambda = \max\Set{\norm{\omega}_{L_{\infty}(I; L_{\infty}(\Omega))} - \mu, 0} \geq 0$, wobei $\gamma_{\Omega}$ die Poincaré"=Friedrichs"=Konstante ist.
    \end{thmenumerate}

    \begin{Beweis}
        Für den Nachweis der Messbarkeit stützen wir uns auf die Aussagen in \cite[177]{fattorini2005infinite} und \cite[Theorem 2.7.9, Corollary 2.7.10, Lemma 8.1.1]{Andreev:2012ep}.
        Demnach ist für $\xi \in L_{\infty}(I \times \Omega)$ und $\psi \in L_{1}(\Omega)$ die Abbildung $I \ni t \mapsto \skp{\xi(t, \blank)}{\psi}{L_{2}(\Omega)}$ messbar.

        Wegen $\eta, \zeta \in V = H^{1}_{0}(\Omega)$ ist sowohl die Abbildung $\Omega \ni \vec x \mapsto \skp{\grad \eta(\vec{x})}{\grad \zeta (\vec x)}{H}$ als auch $\Omega \ni \vec x \mapsto \eta(\vec{x})\zeta (\vec x)$ in $L_{1}(\Omega)$.
        Es bleibt also lediglich $\omega \in L_{\infty}(I \times \Omega)$ zu zeigen.
        Dies ist aber aufgrund von $\chi_{I_{1}}, \chi_{I_{2}} \in L_{\infty}(I)$ und $w_{1}, w_{2} \in L_{\infty}(\Omega)$ ebenfalls gegeben.
        Damit erfüllt jeder der Summanden von $a(\blank, \blank; t)$ und folglich auch die Bilinearform selbst die geforderte Messbarkeit.

        Als nächstes zeigen wir die Stetigkeit.
        Seien dazu $\eta, \zeta \in V$ beliebig, dann erhalten wir unter Verwendung der Dreiecks- und der Cauchy"=Schwarz"=Ungleichung für beliebiges $t \in I$ die Abschätzung
        \begin{align}
            \abs{a(\eta, \zeta; t)}
            &= \abs{c \skp{\grad \eta}{\grad \zeta}{H} + \skp{\omega(t, \blank) \eta}{\zeta}{H} + \mu \skp{\eta}{\zeta}{H} }
            \\&\leq c \abs{\skp{\grad \eta}{\grad \zeta}{H}} + \abs{\skp{\omega(t, \blank) \eta}{\zeta}{H}} + \abs{\mu} \abs{\skp{\eta}{\zeta}{H}}
            \\&\leq c \norm{\grad \eta}_{H} \norm{\grad \zeta}_{H} + \left( \norm{\omega}_{L_{\infty}(I; L_{\infty}(\Omega))} + \abs{\mu} \right) \norm{\eta}_{H} \norm{\zeta}_{H}
            \\&\leq \max \Set{ c, \norm{\omega}_{L_{\infty}(I; L_{\infty}(\Omega))} + \abs{\mu}} \norm{\eta}_{V} \norm{\zeta}_{V}.
        \end{align}

        Für die G\aa{}rding"=Ungleichung seien nun $\eta \in V$ und $\lambda \in \mathbb{R}$.
        Es gilt
        \begin{align}
            a(\eta, \eta; t) + \lambda \norm{\eta}^{2}_{H}
            &= c \norm{\grad \eta}^{2}_{H} + \skprod{\omega(t, \blank) \eta}{\eta}_{H} + \mu \skprod{\eta}{\eta}_{H} + \lambda \skprod{\eta}{\eta}_{H}
            \\&= c \norm{\grad \eta}^{2}_{H} + \skprod{(\omega(t, \blank) + \mu + \lambda) \eta}{\eta}_{H}.
        \end{align}
        Wählen wir nun $\lambda = \max\Set{\norm{\omega}_{L_{\infty}(I; L_{\infty}(\Omega))} - \mu, 0} \geq 0$, dann gilt $\omega + \mu + \lambda \geq 0$ fast überall in $\Omega$ für alle $t \in I$ und wir erhalten die Abschätzung
        \begin{align}
            a(\eta, \eta; t) + \lambda \norm{\eta}^{2}_{H}
            &\geq c \norm{\grad \eta}^{2}_{H},
            \intertext{woraus wir durch Anwenden der Poincaré"=Friedrichs"=Ungleichung \cite[Lemma 89.4]{HankeBourgeois:2009fk}}
            a(\eta, \eta;t ) + \lambda \norm{\eta}^{2}_{H}
            &\geq c \gamma_{\Omega}^{2} \norm{\eta}^{2}_{V}
        \end{align}
        folgern können.
    \end{Beweis}
\end{Satz}

\begin{Korollar}
\label{korollar:bilinearform_elliptisch}
    Ist $\mu \geq \norm{\omega}_{L_{\infty}(I; L_{\infty}(\Omega))}$, dann ist die Bilinearform $a(\blank, \blank; t)$ elliptisch.
\end{Korollar}

Mit diesem Satz sind die notwendigen theoretischen Grundlagen abgeschlossen, so dass im Folgenden die Raum"=Zeit"=Variationsformulierung der Propagator-Differentialgleichung formuliert werden kann.
Diese können wir informell durch Multiplikation der parabolischen Differentialgleichung \cref{eq:propagator_differentialgleichung} mit einer Raum"=Zeit"=Testfunktion $v_{1}$ und Integration über $\Omega$ und $I$ sowie Addition der Anfangsbedingung, welche mit einer Raum"=Testfunktion $v_{2}$ multipliziert und anschließend über $\Omega$ integriert wird, herleiten.

Um eine exakte Definition zu ermöglichen, benötigen wir zunächst zwei Raum"=Zeit"=Hilberträume, welche als Ansatz- respektive Testraum dienen werden.
Es bietet sich an, die aus \cref{definition:ansatz_und_testraum} bereits bekannten Räume
\begin{equation}
    \label{eq:raum_zeit_ansatzraum_testraum}
    \mathcal X = L_{2}(I; V) \cap H^{1}(I; V')
    \quad \text{und} \quad
    \mathcal Y = L_{2}(I; V) \times H,
\end{equation}
zu verwenden.
Damit können wir die schwache Formulierung nun wie folgt definieren.

\begin{Definition}[Schwache Formulierung]
\label{definition:raum_zeit_variationsformulierung}
    Seien $g \in L_{2}(I; V')$ ein Quellterm und $u_{0} \in H$ eine Anfangsbedingung.
    Als \emph{schwache Formulierung} oder \emph{Raum"=Zeit"=Variationsformulierung} der Propagator"=Differentialgleichung \cref{eq:propagator_differentialgleichung} bezeichnen wir das folgende Variationsproblem:
    \begin{equation}
    \label{eq:raum_zeit_variationsformulierung}
        \text{Finde}~u \in \mathcal X \colon \quad  b(u, v) = f(v) \quad \fa v = (v_{1}, v_{2}) \in \mathcal Y.
    \end{equation}
    Dabei sei die Bilinearform $b(\blank, \blank) \colon \mathcal X \times \mathcal Y \to \mathbb{R}$ durch
    \begin{equation}
        \label{eq:raum_zeit_variationsformulierung:lhs}
        b(u, v)
            = \int_{I} \left[ \skp{u_{t}(t)}{v_{1}(t)}{V' \times V} + a(u(t), v_{1}(t); t) \right]  \diff t + \skprod{u(0)}{v_{2}}_{H}
    \end{equation}
    und das stetige lineare Funktional $f \colon \mathcal Y \to \mathbb{R}$ auf der rechten Seite durch
    \begin{equation}
        \label{eq:raum_zeit_variationsformulierung:rhs}
        f(v) = \int_{I} \skp{g(t)}{v_{1}(t)}{V' \times V} \diff t + \skprod{u_{0}}{v_{2}}_{H}
    \end{equation}
    gegeben.
\end{Definition}

\begin{Bemerkung}
\label{bemerkung:raum_zeit_variationsformulierung_rhs_stetig}
    Die Linearität von $f$ ist direkt ersichtlich; die Stetigkeit aber wollen wir an dieser Stelle nachweisen.
    Durch Anwendung der Cauchy"=Schwarz- und der Hölder"=Ungleichung erhalten wir
    \begin{equation}
        \begin{aligned}
            f(v)
            % &= \int_{I} \skprod{g(t)}{v_{1}(t)}_{V' \times V} \diff t + \skprod{u_{0}}{v_{2}}_{H}
            &\leq \int_{I} \norm{g(t)}_{V'} \norm{v_{1}(t)}_{V} \diff t + \norm{u_{0}}_{H} \norm{v_{2}}_{H}
            \\&\leq \left( \int_{I} \norm{g(t)}^{2}_{V'} \diff t \right)^{1/2} \left( \int_{I} \norm{v_{1}(t)}^{2}_{V} \diff t \right)^{1/2} + \norm{u_{0}}_{H} \norm{v_{2}}_{H}
            \\&= \norm{g}_{L_{2}(I; V')} \norm{v_{1}}_{L_{2}(I; V)} + \norm{u_{0}}_{H} \norm{v_{2}}_{H}
            \\&\leq \max\Set*{\norm{g}_{L_{2}(I; V')}, \norm{u_{0}}_{H}} \left( \norm{v_{1}}_{L_{2}(I; V)} + \norm{v_{2}}_{H} \right)
            \\&\leq \sqrt{2} \max\Set*{\norm{g}_{L_{2}(I; V')}, \norm{u_{0}}_{H}} \left( \norm{v_{1}}_{L_{2}(I; V)}^{2} + \norm{v_{2}}_{H}^{2} \right)^{1/2}
            \\&= \sqrt{2} \max\Set*{\norm{g}_{L_{2}(I; V')}, \norm{u_{0}}_{H}} \norm{v}_{\mathcal Y}
        \end{aligned}
    \end{equation}
    und damit die Stetigkeit, wobei die Ungleichung $x + y \leq \sqrt{2} \sqrt{x^2 + y^2}$, welche für alle $x, y \in \mathbb{R}$ gilt, für die letzte Abschätzung verwendet wurde.
\end{Bemerkung}

Der nächste Schritt ist nun, nachzuweisen, dass obige Raum"=Zeit"=Variationsformulierung im Sinne von \cref{definition:sachgemaess_gestellt_nach_hadamard} sachgemäß gestellt ist, also eine eindeutige Lösung besitzt, welche stetig von dem Funktional $f \in \mathcal Y'$ abhängt.
Hierzu werden wir mit \cref{satz:ss09:theorem51} ansetzen, welcher unter den gegebenen Rahmenbedingungen die zu prüfenden Bedingungen auf die in \cref{satz:bilinearform_messbar_stetig_garding} bereits nachgewiesenen reduziert.

\begin{Korollar}
\label{satz:raum_zeit_variationsformulierung_sachgemaess_gestellt}
    Seien Ansatz- und Testraum $\mathcal X$ und $\mathcal Y$ wie in \cref{eq:raum_zeit_ansatzraum_testraum}.
    Dann ist die Raum"=Zeit"=Variationsformulierung \cref{eq:raum_zeit_variationsformulierung} sachgemäß gestellt.

    \begin{Beweis}
        Ist nach Definition der Raum"=Zeit"=Variationsformulierung eine unmittelbare Folgerung aus \cref{satz:ss09:theorem51} und \cref{satz:bilinearform_messbar_stetig_garding}.
    \end{Beweis}
\end{Korollar}

Weiter erhalten wir als Nebenprodukt aus \cref{korrolar:ss09:theorem51_abschaetzungen} auch Schranken für die Stetigkeitskonstante $\gamma_{b}$ sowie die inf-sup"=Konstante $\beta$ der Bilinearform $b(\blank, \blank)$.

\begin{Korollar}
\label{korollar:rz_variationsformulierung_stetig_infsup_schranken}
    Seien die Voraussetzungen von \cref{satz:raum_zeit_variationsformulierung_sachgemaess_gestellt} gegeben.
    Dann erfüllt die Bilinearform $b(\blank, \blank)$ die folgenden Eigenschaften:
    \begin{thmenumerate}
        \item \emph{Stetigkeit:} Es gilt
            \begin{equation}
                \gamma_{b} = \sup_{u \in \mathcal X} \sup_{v \in \mathcal Y} \frac{b(u, v)}{\norm{u}_{\mathcal X} \norm{v}_{\mathcal Y}} < \infty.
            \end{equation}
        \item \emph{inf-sup-Bedingung:} Es gilt
            \begin{equation}
                \beta = \inf_{u \in \mathcal X} \sup_{v \in \mathcal Y} \frac{b(u, v)}{\norm{u}_{\mathcal X} \norm{v}_{\mathcal Y}} > 0.
            \end{equation}
    \end{thmenumerate}

    Erfüllt die Bilinearform $a(\blank, \blank; t)$ die G\aa{}rding"=Ungleichung \cref{eq:bilinearform_messbar_stetig_garding:garding} mit $\lambda = 0$, dann gilt
    \begin{align}
        \gamma_{b}  &\leq \sqrt{2 \max\Set{ 1, c, \norm{\omega}_{L_{\infty}(I; L_{\infty}(\Omega))} + \abs{\mu} }^{2} + \gamma_{e}^{2} }, \\
        \beta &\geq \frac{\gamma_{\Omega}^{2} \min\Set{c, c^{-1}, c (\norm{\omega}_{L_{\infty}(I; L_{\infty}(\Omega))} + \abs{\mu})^{-2}}}{\sqrt{2 \max\Set{c^{-2}\gamma_{\Omega}^{-4}, 1} + \gamma_{e}^{2}}}.
    \end{align}
    Ist dagegen $\lambda > 0$, dann erhalten wir stattdessen die erweiterten Abschätzungen
    \begin{align}
        \gamma_{b} &\leq \frac{\gamma'_{b}}{\max\Set{\sqrt{1 + 2 (\norm{\omega}_{L_{\infty}(I; L_{\infty}(\Omega))} - \mu)^{2} \rho^{4} }, \sqrt{2}}}, \\
        \beta &\geq \frac{e^{-2(\norm{\omega}_{L_{\infty}(I; L_{\infty}(\Omega))} - \mu) T}}{\max\Set{\sqrt{1 + 2 (\norm{\omega}_{L_{\infty}(I; L_{\infty}(\Omega))} - \mu)^{2} \rho^{4} }, \sqrt{2}}} \beta',
    \end{align}
    wobei $\gamma'_{b}$ und $\beta'$ den Größen $\gamma_{b}$ und $\beta$ des ersten Falles entsprechen.
    Die Größen $\gamma_{e}$ und $\rho$ entsprechen dabei denen aus \cref{korrolar:ss09:theorem51_abschaetzungen}.

    \begin{Beweis}
        Die Schranken ergeben sich durch Einsetzen der Größen $\gamma_{a}, \lambda$ und $\alpha$ aus \cref{satz:bilinearform_messbar_stetig_garding} in die Schranken von \cref{korrolar:ss09:theorem51_abschaetzungen}.
    \end{Beweis}
\end{Korollar}


\section{Parametrische Formulierung} % (fold)
\label{section:parametrische_formulierung}

Nachdem nun eine erste schwache Formulierung der Propagator"=Differentialgleichung eingeführt wurde, welche in dieser Form bereits als Grundlage für eine numerische Umsetzung verwendet werden kann, wollen wir nun als nächsten Schritt eine Parametrisierung dieser vornehmen.
Motiviert durch das Iterationsverfahren aus \cref{chapter:einleitung}, in dem die Propagator"=Differentialgleichung immer wieder für leicht variierte Raum-Zeit-Felder $\omega$ berechnet wird, werden wir diese als Ausgangspunkt der Parametrisierung verwenden.

Dazu kehren wir nun zunächst zum Operator $A(t)$ aus \cref{definition:operator_bilinearform_zeit} zurück und betrachten diesen zunächst unabhängig von der Zeit $t \in I$, aber in Abhängigkeit von einem Feld $w \in L_{\infty}(\Omega)$.
Wir definieren für $w \in L_{\infty}(\Omega)$ eine Familie von Operatoren $A(w)$ als
\begin{equation}
    \label{eq:operator_feld}
    A(w) \colon V \to V', \quad A(w) \eta = - c \Delta \eta + w \eta + \mu \eta.
\end{equation}
Wie zuvor sei auch eine Familie von zugehörigen Bilinearformen $a(\blank, \blank; w)$ gegeben durch
\begin{equation}
    \label{eq:bilinearform_feld}
    \begin{aligned}
        a(\blank, \blank; w) \colon V \times V \to \mathbb{R}, \quad
        a(\eta, \zeta; w) = c\skp{\grad \eta}{\grad \zeta}{H} + \skp{w \eta}{\zeta}{H} + \mu \skp{\eta}{\zeta}{H}.
    \end{aligned}
\end{equation}

Um die Abhängigkeit der obigen Operatoren respektive Bilinearformen vom Feld auch für die nachfolgende numerische Umsetzung verwendbar zu machen, wollen wir die Abhängigkeit von einer Abbildung $w \in L_{\infty}(\Omega)$ durch eine Abhängigkeit von einer diskreten Größe, beispielsweise einer Koeffizientenfolge aus dem $\ell_{1}(\mathbb{N})$ oder ähnlichen Folgenräumen, ersetzen.
Dies erreichen wir durch folgende Einschränkung der verwendeten Felder $w \in L_{\infty}(\Omega)$ auf Abbildungen, die wir als Reihenentwicklung von hier noch nicht näher spezifizierten Abbildungen $\varphi_{j}$ darstellen können.
Auf die Wahl dieser $\varphi_{j}$ werden wir bei der numerischen Untersuchung in \cref{chapter:galerkin} erneut eingehen.

\begin{Definition}
\label{definition:feld_entwickelbar}
    Sei $\Set{\varphi_{j}}_{j \in \mathbb{N}} \subset L_{\infty}(\Omega)$ ein System von Funktionen und sei weiter ein Parameterraum $\mathcal P \subset \mathbb{R}^{\mathbb{N}}$ gegeben.
    Wir nennen ein Feld $w \in L_{\infty}(\Omega)$ \emph{darstellbar} durch $\Set{\varphi_{j}}_{j \in \mathbb{N}}$, wenn ein $\bm{\sigma} \in \mathcal P$ existiert, so dass $w$ mit
    \begin{equation}
        w(\blank; \bm\sigma) = \sum_{j = 1}^{\infty} \sigma_{j} \varphi_{j}
    \end{equation}
    übereinstimmt.
\end{Definition}

Wir können mit einem festen System $\Set{\varphi_{j}}_{j \in \mathbb{N}}$ im Allgemeinen nicht alle möglichen $L_{\infty}(\Omega)$-Funktionen darstellen.
Dies ist aber auch nicht nötig, da, wie man in der Einführung bereits gesehen hat, die während des Iterationsverfahrens tatsächlich auftretenden Felder gewisse Eigenschaften, beispielsweise Symmetrien, aufweisen, und wir diese in die Wahl des Systems $\Set{\varphi_{j}}_{j \in \mathbb{N}}$ einfließen lassen können.

\begin{Bemerkung}
    Für den Rest dieses Kapitels beschränken wir uns bei der Wahl des Parameterraums auf $\mathcal P = [-1, 1]^{\mathbb{N}}$.
    Dies dient hauptsächlich der Notation und stellt keine Einschränkung dar, da die Funktionen $\Set{ \varphi_{j} }_{j \in \mathbb{N}}$ entsprechend umskaliert werden können.
\end{Bemerkung}

Um sicherzustellen, dass derartige Felder $w(\bm\sigma)$ wohldefinierte Operatoren $A(w(\bm\sigma))$ liefern, fordern wir die in der folgenden Annahme festgehaltene Eigenschaft von der Funktionenfolge $\Set{\varphi_{j}}_{j \in \mathbb{N}}$.

\begin{Annahme}
\label{annahme:system_l1_summierbar}
    Das Funktionensystem $\Set{ \varphi_{j} }_{j \in \mathbb{N}} \subset L_{\infty}(\Omega)$ sei einfach summierbar in der $L_{\infty}$-Norm, das heißt, es gelte $\Set{ \norm{\varphi_{j}}_{L_{\infty}(\Omega) } }_{j \in \mathbb{N}} \in \ell_{1}(\mathbb{N})$.
\end{Annahme}

Im Folgenden bezeichnen wir die obige $\ell_{1}(\mathbb{N})$-Norm der Kürze wegen als
\begin{equation}
\label{eq:system_l1_norm}
    c_{\varphi} = \sum_{j = 1}^{\infty} \norm{\varphi_{j}}_{L_{\infty}(\Omega)}.
\end{equation}
Diese Annahme stellt insbesondere die gleichmäßige Konvergenz von $w(\bm\sigma)$ für alle $\bm\sigma \in \mathcal P$ sicher, denn es gilt
\begin{equation}
\label{eq:rz_feld_norm_schranke}
    \sup_{\bm\sigma \in \mathcal P} \norm{w(\bm\sigma)}_{L_{\infty}(\Omega)} \leq \sum_{j = 1}^{\infty} \norm{\varphi_{j}}_{L_{\infty}(\Omega)} = c_{\varphi} < \infty.
\end{equation}

Legen wir uns auf ein konkretes Funktionensystem $\Set{\varphi_{j}}_{j \in \mathbb{N}}$, welches die \cref{annahme:system_l1_summierbar} erfüllt, fest, dann können wir die Operatoren $A(\omega)$ nun auch als Familie von Operatoren $A(\bm\sigma)$ betrachten, denn durch Einsetzen von $w(\bm\sigma)$ in \cref{eq:operator_feld} erhalten wir
\begin{equation}
\label{eq:operator_parameter}
    A(\bm\sigma) \colon V \to V', \quad A(\bm\sigma) \eta = -c \Delta \eta + \sum_{j = 1}^{\infty} \sigma_{j} \varphi_{j} \eta + \mu \eta.
\end{equation}
Weiter können wir auch die zugehörige Bilinearform $a(\blank, \blank; \bm\sigma)$ angeben als
\begin{equation}
\label{eq:bilinearform_parameter}
    \begin{aligned}
    a(\blank, \blank; \bm\sigma) \colon V \times V \to \mathbb{R},
    \quad a(\eta, \zeta; \bm\sigma) = c\skp{\grad \eta}{\grad \zeta}{H} + \sum_{j = 1}^{\infty} \sigma_{j} \skp{\varphi_{j} \eta}{\zeta}{H} + \mu \skp{\eta}{\zeta}{H}.
    \end{aligned}
\end{equation}

Die bei der Bilinearform vorgenommene Vertauschung von Summe und $H$"=Skalarprodukt ist durch \cref{annahme:system_l1_summierbar} respektive Ungleichung \cref{eq:rz_feld_norm_schranke} und den Satz von Lebesgue gerechtfertigt.
Wie auch für den nicht"=parametrischen Fall werden wir nachweisen, dass es sich hierbei um Operatoren beziehungsweise Bilinearformen handelt, welche die für uns wichtigen Eigenschaften der Stetigkeit und die Gültigkeit einer G\aa{}rding"=Ungleichung besitzen.
Zunächst wollen wir an dieser Stelle noch ein parametrisches Äquivalent der Raum"=Zeit"=Variationsformulierung aus \cref{definition:raum_zeit_variationsformulierung} formulieren.

Dies Bedarf, wie zuvor, einen zeitlichen Wechsel zwischen mehreren Feldern $w_{i}$.
Wir beschränken uns auf den bereits bekannten Fall zweier Felder und erweitern die obige Operator"=Definition um die zeitliche Abhängigkeit.
Zunächst definieren wir analog zu \cref{eq:raum_zeit_feld} ein parametrisches Raum"=Zeit"=Feld.
Dies geschieht auf Basis der Darstellung von $w_{i}$ aus \cref{definition:feld_entwickelbar}.
Sei dazu $\bm\sigma \in \mathcal P$, dann definieren wir die folgenden Teilfolgen der ungeraden respektive geraden Indizes als $\bm\sigma_{\odd} = (\sigma_{2j-1})_{j \in \mathbb{N}}$ und $\bm\sigma_{\even} = (\sigma_{2j})_{j \in \mathbb{N}}$.

\begin{Definition}
\label{definition:parametrisches_raum_zeit_feld}
    Sei $\bm\sigma \in \mathcal P$.
    Unter einem \emph{parametrischen Raum-Zeit-Feld} verstehen wir die Abbildung
    $\omega(\blank, \blank; \bm\sigma) \colon I \times \Omega \to \mathbb{R}$, die durch
    \begin{equation}
    \label{eq:parametrisches_raum_zeit_feld}
        \begin{aligned}
            \omega(t, \vec{x}; \bm\sigma)
            &= w(\vec{x}; \bm\sigma_{\odd}) \chi_{I_{1}}(t) + w(\vec{x}; \bm\sigma_{\even}) \chi_{I_{2}}(t)\\
            &= \sum_{j = 1}^{\infty} (\sigma_{2j-1} \chi_{I_{1}}(t) + \sigma_{2j} \chi_{I_{2}}(t)) \varphi_{j}(\vec x)
        \end{aligned}
    \end{equation}
    gegeben ist.
\end{Definition}

Wegen der Zerlegung $I = I_{1} \cupdot I_{2}$ ist direkt ersichtlich, dass analog zu \cref{eq:rz_feld_norm_schranke} die Abschätzung
\begin{equation}
\label{eq:parametrisches_raum_zeit_feld_schranke}
    \sup_{\bm\sigma \in \mathcal P} \norm{\omega(\bm\sigma)}_{L_{\infty}(I; L_{\infty}(\Omega))}
    = \sup_{\bm\sigma \in \mathcal P} \norm{w(\bm\sigma)}_{L_{\infty}(\Omega)} \leq \sum_{j = 1}^{\infty} \norm{\varphi_{j}}_{L_{\infty}(\Omega)} = c_{\varphi}
\end{equation}
gilt.

Erweitern wir nun die Operator"=Definition \cref{eq:operator_parameter} um die Zeitabhängigkeit; definieren wir also für $t \in I$ und $\bm\sigma \in \mathcal P$ die Operatorfamilie
\begin{equation}
    \label{eq:operator_zeit_parameter}
    A(t, \bm\sigma) \colon V \to V', \quad A(t, \bm\sigma) \eta = -c \Delta \eta + \omega(t, \blank; \bm\sigma) \eta + \mu \eta,
\end{equation}
dann hat die zugehörige Familie von Bilinearformen $a(\blank, \blank; t, \bm\sigma) \colon V \times V \to \mathbb{R}$ nach \cref{eq:parametrisches_raum_zeit_feld} die Form
\begin{equation}
    \label{eq:bilinearform_zeit_parameter}
    \begin{aligned}
        a(\eta, \zeta; t, \bm\sigma) = c\skp{\grad \eta}{\grad \zeta}{H} + \sum_{j = 1}^{\infty} \left[ \sigma_{2j-1} \chi_{I_{1}}(t) + \sigma_{2j} \chi_{I_{2}}(t)  \right] \skp{\varphi_{j} \eta}{\zeta}{H} + \mu \skp{\eta}{\zeta}{H}.
    \end{aligned}
\end{equation}
Mit dieser Vorarbeit können wir analog zu \cref{definition:raum_zeit_variationsformulierung} nun die folgende parametrische schwache Formulierung definieren.

\begin{Definition}[Parametrische schwache Formulierung]
\label{definition:parametrische_rz_variationsformulierung}
    Seien $g \in L_{2}(I; V')$ ein Quellterm und $u_{0} \in H$ eine Anfangsbedingung.
    Als \emph{parametrische schwache Formulierung} oder \emph{parametrische Raum"=Zeit"=Variationsformulierung} der Propagator"=Differentialgleichung \cref{eq:propagator_differentialgleichung} bezeichnen wir das folgende Variationsproblem:
    \begin{equation}
    \label{eq:parametrisches_rz_variationsproblem}
        \text{Sei}~\bm\sigma \in \mathcal P,~\text{finde}~u(\bm\sigma) \in \mathcal X : \quad b(u(\bm\sigma), v; \bm\sigma) = f(v) \quad \fa v \in \mathcal Y.
    \end{equation}
    Dabei sei die Familie von Bilinearformen $b(\blank, \blank; \bm\sigma) \colon \mathcal X \times \mathcal Y \to \mathbb{R}$ gegeben durch
     \begin{equation}
     \label{eq:parametrisches_rz_variationsproblem:lhs}
         b(u, v; \bm\sigma)
             = \int_{I} \left[ \skp{u_{t}(t)}{v_{1}(t)}{V' \times V} + a(u(t), v_{1}(t); t, \bm\sigma) \right] \diff t + \skp{u(0)}{v_{2}}{H},
     \end{equation}
     wobei $a(\blank, \blank; t, \bm\sigma)$ wie in \cref{eq:bilinearform_zeit_parameter} definiert sei.
     Das stetige lineare Funktional $f \colon \mathcal Y \to \mathbb{R}$ sei wie zuvor
     \begin{equation}
     \label{eq:parametrisches_rz_variationsproblem:rhs}
         f(v) = \int_{I} \skp{g(t)}{v_{1}(t)}{V' \times V} \diff t + \skp{u_{0}}{v_{2}}{H}.
     \end{equation}
\end{Definition}

Hierfür weisen wir nun erneut nach, dass die schwache Formulierung sachgemäß gestellt ist.
Wie zuvor wollen wir \cref{satz:ss09:theorem51} verwenden, müssen also die Bedingungen an $a(\blank, \blank;t, \bm\sigma)$ nachweisen.

\begin{Satz}
\label{satz:parametrische_bilinearform_messbar_stetig_garding}
    Sei $a(\blank, \blank; t; \bm\sigma)$, $t \in I$ und $\bm\sigma \in \mathcal P$, die Familie von Bilinearformen aus \cref{eq:bilinearform_zeit_parameter}.
    Dann gelten für alle $t \in I$ und $\bm\sigma \in \mathcal P$ die folgenden Eigenschaften:
    \begin{thmenumerate}
        \item\label{satz:parametrische_bilinearform_messbar_stetig_garding:messbar}
        \emph{Messbarkeit}: Die Abbildung $I \ni t \mapsto a(\eta, \zeta; t, \bm\sigma)$ ist messbar für alle $\eta, \zeta \in V$.
        \item\label{satz:parametrische_bilinearform_messbar_stetig_garding:stetig}
        \emph{Stetigkeit:} Es gilt
        \begin{equation}
            \label{eq:parametrische_bilinearform_messbar_stetig_garding:stetig}
            \abs{a(\eta, \zeta; t; \bm\sigma)} \leq \gamma_{a} \norm{\eta}_{V} \norm{\zeta}_{V}, \quad \fa \eta, \zeta \in V,
        \end{equation}
        mit Stetigkeitskonstante $\gamma_{a} = \max\Set{c, c_{\varphi} + \abs{\mu}} < \infty$.
        \item\label{satz:parametrische_bilinearform_messbar_stetig_garding:garding}
        \emph{G\aa{}rding"=Ungleichung:} Es gilt
        \begin{equation}
            \label{eq:parametrische_bilinearform_messbar_stetig_garding:garding}
            a(\eta, \eta; t; \bm\sigma) + \lambda \norm{\eta}_{H}^{2} \geq \alpha \norm{\eta}_{V}^{2}, \quad \fa \eta \in V,
        \end{equation}
        mit $\alpha = c \gamma_{\Omega}^{2} > 0$ und $\lambda = \max\Set{c_{\varphi} - \mu, 0} \geq 0$.
    \end{thmenumerate}
    Dabei ist $c_{\varphi}$ die Konstante aus \cref{eq:system_l1_norm} und die Konstanten $\gamma_{a}$, $\lambda$ und $\alpha$ sind sowohl unabhängig von $t \in I$, als auch von $\bm\sigma \in \mathcal P$.

    \begin{Beweis}
        Der Nachweis erfolgt analog zum nicht"=parametrischen Fall in \cref{satz:bilinearform_messbar_stetig_garding}.
        Wir müssen zusätzlich lediglich die $L_{\infty}(\Omega)$-Norm von $\omega(t, \blank; \bm\sigma)$ mit Hilfe von \cref{eq:parametrisches_raum_zeit_feld_schranke} weiter durch $c_{\varphi}$ abschätzen.
    \end{Beweis}
\end{Satz}

\begin{Korollar}
\label{korollar:parametrisches_rz_variationsproblem_sachgemaess}
    Die parametrische schwache Formulierung \cref{eq:parametrisches_rz_variationsproblem} ist für alle $\bm\sigma \in \mathcal P$ sachgemäß gestellt.
    Ferner existieren analog zu \cref{korollar:rz_variationsformulierung_stetig_infsup_schranken} Schranken für die Stetigkeitskonstante $\gamma_{b}$ und die inf-sup"=Konstante $\beta$, welche unabhängig von $\bm \sigma \in \mathcal P$ sind.
\end{Korollar}


\section{Regularität bezüglich der Parameter} % (fold)
\label{section:regularitaet_bezueglich_der_parameter}

Wir interessieren uns nun für die Regularität der Abhängigkeit der Lösung $u(\bm\sigma)$ der parametrischen schwachen Formulierung vom Parameter $\bm \sigma \in \mathcal P$.
Konkret werden wir nachweisen, dass die Lösung unter gewissen Annahmen an das Funktionensystem $\Set{\varphi_{j}}_{j \in \mathbb{N}}$ analytisch vom Parameter $\bm \sigma$ abhängt.
Dabei orientieren wir uns an den Arbeiten von \textcite{Cohen:2010kz,Cohen:2011jp,Kunoth:2013ef}, weisen diese Eigenschaften aber direkt für die Raum"=Zeit"=Variationsformulierung nach, statt wie in den genannten Arbeiten den Umweg über den stationären Fall zu gehen.

Zunächst folgen einige notationelle Vorbemerkungen.
\begin{Bemerkung}
    Als Multiindexmenge $\mathcal F$ definieren wir $\mathcal F = \Set{ \bm\nu \in \mathbb{N}^{\mathbb{N}}_{0} \given \abs{\bm\nu} < \infty }$, wobei
    \begin{equation}
        \abs{\bm\nu} = \sum_{k = 1}^{\infty} \abs{\nu_{k}}
    \end{equation}
    die $\ell_{1}(\mathbb{N})$-Norm sei.
    Anders formuliert, besteht $\mathcal F$ gerade aus denjenigen Folgen in $\mathbb{N}_{0}$, welche nur endliche viele Einträge ungleich Null aufweisen.

    Sei $\bm\nu \in \mathcal F$ und $\vec{b} \in \ell_{p}(\mathbb{N})$, $p > 0$, dann schreiben wir
    \begin{equation}
        \vec{b}^{\bm\nu} = \prod_{j = 1}^{\infty} b_{j}^{\nu_{j}}
    \end{equation}
    mit der Konvention $0^{0} = 1$.
    Wegen $\abs{\bm\nu} < \infty$ ist dieses Produkt stets endlich.
\end{Bemerkung}

Um die Notation für die nachfolgenden Beweise zu vereinfachen, ordnen wir die Darstellung der parametrischen Raum"=Zeit"=Felder um.
\begin{Bemerkung}
    Wir definieren neue charakteristische Funktionen und Entwicklungsfunktionen für $j \in \mathbb{N}$ durch
    \begin{equation}
        \tilde{\chi}_{j} = \begin{cases}
            \chi_{I_{1}}, & j~\text{ungerade},\\
            \chi_{I_{2}}, & j~\text{gerade},
        \end{cases} \quad
        \tilde{\varphi}_{j} = \begin{cases}
            \varphi_{(j+1)/{2}}, & j~\text{ungerade},\\
            \varphi_{j / 2}, & j~\text{gerade}.
        \end{cases}
    \end{equation}
    Damit können wir \cref{eq:parametrisches_raum_zeit_feld} auch schreiben als
    \begin{equation}
        w(t, \vec{x}; \bm \sigma) = \sum_{j = 1}^{\infty} \sigma_{j} \tilde{\chi}_{j}(t) \tilde{\varphi}_{j}(\vec{x}).
    \end{equation}
\end{Bemerkung}

Für die nachfolgenden Beweise fixieren wir die rechte Seite $f \in \mathcal Y'$ der schwachen Formulierungen \cref{eq:raum_zeit_variationsformulierung,eq:parametrisches_rz_variationsproblem} und beginnen den Nachweis der Regularität mit einer Stabilitätsaussage, welche in den folgenden Beweisen nützlich sein wird.

\begin{Lemma}
\label{lemma:stabilitaetsaussage}
    Seien $\omega, \tilde{\omega}$ zwei Raum"=Zeit"=Felder wie in \cref{eq:raum_zeit_feld} und $u, \tilde{u}$ die zugehörigen Lösungen der schwachen Formulierung \cref{section:raum_zeit_variationsformulierung}.
    Dann gilt
    \begin{equation}
        \norm{u - \tilde{u}}_{\mathcal X} \leq \frac{\norm{f}_{\mathcal Y'}}{\beta^{2}} \norm{\omega - \tilde{\omega}}_{L_{\infty}(I; L_{\infty}(\Omega))},
    \end{equation}
    wobei $\beta$ eine Feld-unabhängige inf-sup"=Konstante ist.

    \begin{Beweis}
        Wir vernachlässigen der Kürze wegen im Folgenden die explizite Angabe der Zeitabhängigkeit der jeweiligen Funktionen.
        Weiter setzen wir $\theta = u - \tilde{u}$.
        Subtraktion der Variationsformulierung für die beiden Lösungen $u$ und $\tilde{u}$ liefert für beliebige Testfunktionen $v = (v_{1}, v_{2}) \in \mathcal Y$ die Gleichung
        \begin{align}
            0
            &= f(v) - f(v) = b(u, v; \omega) - b(\tilde{u}, v; \tilde{\omega})
           \\&= \int_{I} \left[ \skp{u_{t} - \tilde{u}_{t}}{v_{1}}{V' \times V} + a(u, v; \omega) - a(\tilde{u}, v; \tilde{\omega}) \right] \diff t + \skp{u(0) - \tilde{u}(0)}{v_{2}}{H}
           \\&= \int_{I} \left[ \skp{\theta_{t}}{v_{1}}{V' \times V} + c \skp{\grad \theta}{\grad v_{1}}{H} + \mu \skp{\theta}{v_{1}}{H} + \skp{\omega u - \tilde{\omega}\tilde{u}}{v_{1}}{H} \right] \diff t + \skp{\theta(0)}{v_{2}}{H}
           \\&= \int_{I} \left[ \skp{\theta_{t}}{v_{1}}{V' \times V} + a(\theta, v; \omega) \right] \diff t + \skp{\theta(0)}{v_{2}}{H} + \int_{I} \skp{(w - \tilde{w})\tilde{u}}{v_{1}}{H} \diff t
           \\&= b(\theta, v; \omega) + \int_{I} \skp{(w - \tilde{w})\tilde{u}}{v_{1}}{H} \diff t,
        \end{align}
        welche wir auch als
        \begin{equation}
            \label{eq:stabilitaetsaussage:beweis:variationsproblem}
            b(\theta, v; \omega) = h(v)
        \end{equation}
        mit der Abbildung
        \begin{equation}
            h \colon \mathcal Y \to \mathbb{R}, \quad h(v) = - \int_{I} \skp{(w - \tilde{w})\tilde{u}}{v_{1}}{H} \diff t
        \end{equation}
        auffassen können.
        Die Linearität von $h$ ist klar.
        Wir weisen nun die Stetigkeit nach; betrachten also
        \begin{align}
            \norm{h}_{\mathcal Y'}
            &= \sup_{\norm{v}_{\mathcal Y} = 1} \abs{\int_{I} \skp{(w - \tilde{w})\tilde{u}}{v_{1}}{H} \diff t}
            \\&\leq \norm{w - \tilde{w}}_{L_{\infty}(I; L_{\infty}(\Omega))} \sup_{\norm{v}_{\mathcal Y} = 1} \abs{\skp{\tilde{u}}{v_{1}}{L_{2}(I; H)}}
            \\&\leq \norm{w - \tilde{w}}_{L_{\infty}(I; L_{\infty}(\Omega))} \sup_{\norm{v}_{\mathcal Y} = 1} \norm{\tilde{u}}_{L_{2}(I; H)} \norm{v_{1}}_{L_{2}(I; H)}
            \\&\leq \norm{w - \tilde{w}}_{L_{\infty}(I; L_{\infty}(\Omega))} \sup_{\norm{v}_{\mathcal Y} = 1} \norm{\tilde{u}}_{\mathcal X} \norm{v}_{\mathcal Y}
            \\&= \norm{w - \tilde{w}}_{L_{\infty}(I; L_{\infty}(\Omega))} \norm{\tilde{u}}_{\mathcal X}
            < \infty.
        \end{align}
        Damit ist $h \in \mathcal Y'$, wir können \cref{eq:stabilitaetsaussage:beweis:variationsproblem} also selbst als Raum"=Zeit"=Variationsproblem der Form \cref{eq:raum_zeit_variationsformulierung} auffassen und erhalten damit durch \cref{satz:raum_zeit_variationsformulierung_sachgemaess_gestellt} die Abschätzung
        \begin{equation}
            \norm{\theta}_{\mathcal X} \leq \frac{1}{\beta} \norm{h}_{\mathcal Y'}.
        \end{equation}
        Wenden wir \cref{satz:raum_zeit_variationsformulierung_sachgemaess_gestellt} weiter auf die schwache Formulierung zu $\tilde{\omega}$ an, dann gilt
        \begin{equation}
            \norm{\tilde{u}}_{\mathcal X} \leq \frac{1}{\beta} \norm{f}_{\mathcal Y'}.
        \end{equation}
        Die Unabhängigkeit von $\beta$ von den Feldern kann durch das Minimum der beiden Feld-abhängigen inf-sup-Konstanten sichergestellt werden.

        Durch Zusammenfassen dieser drei Abschätzungen erhalten wir die Behauptung
        \begin{equation}
            \norm{u - \tilde{u}}_{\mathcal X} = \norm{\theta}_{\mathcal X} \leq \frac{\norm{f}_{\mathcal Y'}}{\beta^{2}} \norm{\omega - \tilde{\omega}}_{L_{\infty}(I; L_{\infty}(\Omega))}.
        \end{equation}
    \end{Beweis}
\end{Lemma}

Als erster Schritt des Regularitätsnachweises wird zunächst die Existenz beliebiger partieller Ableitungen gezeigt, bevor dann nachfolgend schrittweise die Konvergenz der Taylorreihe der Lösung $u(\bm \sigma) \in \mathcal X$ nachgewiesen wird.

\begin{Satz}
\label{satz:existenz_partieller_ableitungen}
    Die Abbildung $\mathcal P \ni \bm\sigma \mapsto u(\bm\sigma) \in \mathcal X$, welche einem Parameter $\bm\sigma$ die eindeutige Lösung $u(\bm\sigma)$ der parametrischen schwachen Formulierung \cref{eq:parametrisches_rz_variationsproblem} zuordnet, besitzt für alle $\bm\nu \in \mathcal F$ eine partielle Ableitung $\partial^{\bm\nu}_{\bm\sigma} u(\bm\sigma)$.

    \begin{Beweis}
        Erneut verzichten wir auf die explizite Notation der Zeitabhängigkeit.
        Wir beschränken uns darauf, die Behauptung exemplarisch für die partiellen Ableitungen erster Ordnung für ein festes $\bm\sigma \in \mathcal P$ nachzuweisen.
        Ohne Einschränkung sei nun $\bm\nu = \vec{e}_{j} \in \mathcal F$ für ein $j \in \mathbb{N}$ und ferner sei $h \in \mathbb{R} \setminus \Set{ 0 }$.
        Wir definieren $\bm\sigma_{h} = \bm\sigma + h \bm\nu = \bm\sigma + h \vec e_{j}$ und
        \begin{equation}
            \theta_{h} = \frac{u(\bm\sigma_{h}) - u(\bm\sigma)}{h},
        \end{equation}
        wobei $u(\blank)$ die Lösungen der parametrischen schwachen Formulierung \cref{eq:parametrisches_rz_variationsproblem} für die entsprechenden Parameter ist.
        Ist $\abs{h}$ klein genug, dann existieren diese auch im Fall $\bm\sigma_{h} \not\in \mathcal P$, da analog zu \cref{eq:parametrisches_raum_zeit_feld_schranke} die schwache Formulierung nach wie vor sachgemäß gestellt ist.
        Dies ergibt sich durch die Abschätzung
        \begin{equation}
            \norm{\omega(\bm \sigma_{h})}_{L_{\infty}(I; L_{\infty}(\Omega))} \leq \sum_{j' = 1}^{\infty} \norm{\varphi_{j'}}_{L_{\infty}(\Omega)} + \abs{h} \norm{\varphi_{j}}_{L_{\infty}(\Omega)}\leq c_{\varphi} + \abs{h} \norm{\varphi_{j}}_{L_{\infty}(\Omega)} < \infty.
        \end{equation}

        Zunächst schreiben wir die Differenz der parametrischen Raum-Zeit-Felder um zu
        \begin{equation}
            \omega(t, \vec{x}; \bm\sigma_{h}) - \omega(t, \vec{x}; \bm\sigma)
            = \sum_{j = 1}^{\infty} (\sigma_{h,j} - \sigma_{j} ) \tilde{\chi}_{j}(t) \tilde{\varphi}_{j}(\vec{x})
            = h \tilde{\chi}_{j}(t) \tilde{\varphi}_{j}(\vec{x}).
        \end{equation}
        Unter diesen Gegebenheiten betrachten wir nun die Differenz der zu $u(\bm\sigma_{h})$ und $u(\bm\sigma)$ zugehörigen Variationsprobleme.
        Für $v = (v_{1}, v_{2}) \in \mathcal Y$ gilt dann
        \begin{align}
            0
            &= b(u(\bm\sigma_{h}), v; \bm\sigma_{h}) - b(u(\bm\sigma), v; \bm\sigma)
            \\&= \int_{I} \left[ \skp{u_{t}(\bm\sigma_{h}) - u_{t}(\bm\sigma)}{v_{1}}{V' \times V} + a(u(\bm\sigma_{h}), v_{1}; \bm\sigma_{h}) - a(u(\bm\sigma), v_{1}; \bm\sigma) \right] \diff t
            \\&\qquad + \skp{u(0; \bm\sigma_{h}) - u(0; \bm\sigma)}{v_{2}}{H}
            \\&=  h \int_{I} \left[ \skp{(\theta_{h})_{t}}{v_{1}}{V' \times V} + c\skp{\grad \theta_{h}}{\grad v_{1}}{H}  +  \mu \skp{\theta_{h}}{v_{1}}{H} \right] \diff t
            \\&\qquad + \int_{I} \left[ \skp{\omega(\bm\sigma_{h}) u(\bm\sigma_{h})}{v_{1}}{H} - \skp{\omega(\bm\sigma) u(\bm\sigma)}{v_{1}}{H}  \right] \diff t + h \skp{\theta_{h}(0)}{v_{2}}{H}
            \\&= h \int_{I} \left[ \skp{(\theta_{h})_{t}}{v_{1}}{V' \times V} + a(\theta_{h}, v_{1}; \bm\sigma)  \right] \diff t + h \skp{\theta_{h}(0)}{v_{2}}{H}
            \\&\qquad +\int_{I} \skp{(\omega(\bm\sigma_{h}) - \omega(\bm\sigma))u(\bm\sigma_{h})}{v_{1}}{H} \diff t
            \\&= h \cdot b(\theta_{h}, v; \bm\sigma) + h \int_{I} \tilde{\chi}_{j} \skp{\tilde{\varphi}_{j} u(\bm\sigma_{h})}{v_{1}}{H} \diff t.
        \end{align}
        Dies schreiben wir erneut in Form der Gleichung
        \begin{equation}
            \label{eq:existenz_partieller_ableitungen:beweis:variationsproblem}
            b(\theta_{h}, v; \bm\sigma) = F_{h}(v)
        \end{equation}
        mit der Abbildung
        \begin{equation}
            F_{h} \colon \mathcal Y \to \mathbb{R}, \quad F_{h}(v) = - \int_{I} \tilde{\chi}_{j} \skp{\tilde{\varphi}_{j}  u(\bm\sigma_{h})}{v_{1}}{H} \diff t.
        \end{equation}
        Vollkommen analog zum Beweis des vorherigen Lemmas kann gezeigt werden, dass $F_{h}$ ein stetiges lineares Funktional auf $\mathcal Y$ definiert, das heißt, $\theta_{h}$ ist die eindeutige Lösung des Variationsproblems \cref{eq:existenz_partieller_ableitungen:beweis:variationsproblem}.
        Weiter ist $F_{h}(\blank)$ stetig in $h = 0$, denn für festes $v \in \mathcal Y$ gilt unter Verwendung der Cauchy-Schwarz-Ungleichung die Abschätzung
        \begin{align}
            \abs{F_{h}(v) - F_{0}(v)}
            &= \abs{\int_{I} \tilde{\chi}_{j} \skp{\tilde{\varphi}_{j}  (u(\bm\sigma_{h}) - u(\bm\sigma))}{v_{1}}{H} \diff t }
            \\&\leq \norm{\tilde{\varphi}_{j}}_{L_{\infty}(\Omega)} \abs{\skp{u(\bm\sigma_{h}) - u(\bm\sigma)}{v_{1}}{L_{2}(I; H)} }
            \\&\leq \norm{\tilde{\varphi}_{j}}_{L_{\infty}(\Omega)} \norm{u(\bm\sigma_{h}) - u(\bm\sigma)}_{L_{2}(I; H)} \norm{v_{1}}_{L_{2}(I; H)}
            \\&\leq \norm{\tilde{\varphi}_{j}}_{L_{\infty}(\Omega)} \norm{u(\bm\sigma_{h}) - u(\bm\sigma)}_{\mathcal X} \norm{v}_{\mathcal Y}.
        \end{align}
        Hier setzen wir mit der Stabilitätsaussage \cref{lemma:stabilitaetsaussage} an, um $\norm{u(\bm\sigma_{h}) - u(\bm\sigma)}_{\mathcal X}$ weiter abzuschätzen und erhalten
        \begin{equation}
            \begin{aligned}
                \norm{u(\bm\sigma_{h}) - u(\bm\sigma)}_{\mathcal X}
                &\leq \frac{\norm{f}_{\mathcal Y'}}{\beta^{2}} \norm{w(\bm\sigma_{h}) - \omega(\bm\sigma)}_{L_{\infty}(I; L_{\infty}(\Omega))}
                = \frac{\norm{f}_{\mathcal Y'}}{\beta^{2}} \norm{h \tilde{\chi}_{j} \tilde{\varphi}_{j} }_{L_{\infty}(I; L_{\infty}(\Omega))}
                \\&\leq \frac{\norm{f}_{\mathcal Y'}}{\beta^{2}} \abs{h} \norm{\tilde{\varphi}_{j}}_{L_{\infty}(\Omega)}.
            \end{aligned}
        \end{equation}
        Zusammen mit obiger Ungleichung liefert dies
        \begin{equation}
            \abs{F_{h}(v) - F_{0}(v)}
            \leq \norm{\tilde{\varphi}_{j}}^{2}_{L_{\infty}(\Omega)} \norm{v}_{\mathcal Y} \frac{\norm{f}_{\mathcal Y'}}{\beta^{2}} \abs{h} \to 0 \quad \text{für}~h \to 0.
        \end{equation}
        Das bedeutet, dass $F_{h} \to F_{0}$ in $\mathcal Y'$ für $h \to 0$ gilt,
        was insbesondere $\theta_{h} \to \theta_{0}$ in $\mathcal X$ für $h \to 0$ impliziert, da $\theta_{h}$ als Lösung des Variationsproblems \cref{eq:existenz_partieller_ableitungen:beweis:variationsproblem} nach \cref{korollar:parametrisches_rz_variationsproblem_sachgemaess} stetig von $F_{h}$ abhängt.
        Ferner ist durch $\partial^{\bm\nu}_{\bm\sigma} u(\bm\sigma) = \theta_{0}$ als Lösung von
        \begin{equation}
            \label{eq:existenz_partieller_ableitungen:beweis:variationsproblem_ableitung}
            \text{Finde}~\theta_{0} \in \mathcal X \colon \quad b(\theta_{0}, v; \bm\sigma) = - \int_{I} \tilde{\chi}_{j} \skp{\tilde{\varphi}_{j}  u(\bm\sigma)}{v_{1}}{H} \diff t \quad \fa v \in \mathcal Y;
        \end{equation}
        die Existenz und Wohldefiniertheit der gesuchten partiellen Ableitung gegeben.

        Die Ableitungen höherer Ordnung lassen sich auf gleiche Weise durch Anwendung der beschriebenen Schritte auf die Variationsformulierung \cref{eq:existenz_partieller_ableitungen:beweis:variationsproblem_ableitung} et cetera konstruieren.
    \end{Beweis}
\end{Satz}

\begin{Bemerkung}
\label{bemerkung:alternative_herleitung_variationsproblem_ableitung}
    Sei erneut ohne Einschränkung $\bm\nu = \vec e_{j} \in \mathcal{F}$ für ein $j \in \mathbb{N}$.
    Alternativ erhält man das Variationsproblem \cref{eq:existenz_partieller_ableitungen:beweis:variationsproblem_ableitung} für die partielle Ableitung $\partial^{\bm\nu}_{\bm\sigma} u(\bm\sigma)$ auch durch formales Differenzieren der Variationsformulierung \cref{eq:parametrisches_rz_variationsproblem} nach $\sigma_{j}$, denn es gilt
    \begin{align}
        \partial^{\bm\nu}_{\bm\sigma} b(u(\bm\sigma), v; \bm\sigma)
        &= \partial^{\bm\nu}_{\bm\sigma} \left( \int_{I} \left[ \skp{u_{t}(\bm\sigma)}{v_{1}}{V' \times V} + c \skp{\grad u(\bm\sigma)}{\grad v_{1}}{H} + \mu \skp{u(\bm\sigma)}{v_{1}}{H} \right.\right.
        \\&\qquad\qquad \left.\vphantom{\int_{I}}\left. + \skp{\omega(\bm\sigma) u(\bm\sigma)}{v_{1}}{H} \right] \diff t + \skp{u(0; \bm\sigma)}{v_{2}}{H} \right)
        \\&= \int_{I} \left[ \skp{\partial^{\bm\nu}_{\bm\sigma} u_{t}(\bm\sigma)}{v_{1}}{V' \times V}
            + \skp{\grad \partial^{\bm\nu}_{\bm\sigma} u(\bm\sigma)}{\grad v_{1}}{H} + \mu \skp{\partial^{\bm\nu}_{\bm\sigma} u(\bm\sigma)}{v_{1}}{H} \right.
        \\&\qquad \quad \left.+ \skp{\partial^{\bm\nu}_{\bm\sigma} \omega(\bm\sigma) u(\bm\sigma) + \omega(\bm\sigma) \partial^{\bm\nu}_{\bm\sigma} u(\bm\sigma)}{v_{1}}{H} \right] \diff t + \skp{\partial^{\bm\nu}_{\bm\sigma} u(0; \bm\sigma)}{v_{2}}{H}
        \\&= b(\partial^{\bm\nu}_{\bm\sigma} u(\bm\sigma), v; \bm\sigma) + \int_{I} \skp{\partial^{\bm\nu}_{\bm\sigma} \omega(\bm\sigma) u(\bm\sigma)}{v_{1}}{H} \diff t
        \\&= b(\partial^{\bm\nu}_{\bm\sigma} u(\bm\sigma), v; \bm\sigma) + \int_{I} \tilde{\chi}_{j} \skp{\tilde{\varphi}_{j} u(\bm\sigma)}{v_{1}}{H} \diff t
    \end{align}
    und ferner $\partial^{\bm\nu}_{\bm\sigma} f(v) = 0$.
    Daraus erhält man insgesamt erneut das Variationsproblem \cref{eq:existenz_partieller_ableitungen:beweis:variationsproblem_ableitung}
    \begin{equation}
        \text{Finde}~\partial^{\bm\nu}_{\bm\sigma} u(\bm\sigma) \in \mathcal X \colon \quad b(\partial^{\bm\nu}_{\bm\sigma} u(\bm\sigma), v; \bm\sigma) = - \int_{I} \tilde{\chi}_{j} \skp{\tilde{\varphi}_{j}  u(\bm\sigma)}{v_{1}}{H} \diff t \quad \fa v \in \mathcal Y.
    \end{equation}
\end{Bemerkung}

\begin{Satz}
\label{satz:abschaetzung_norm_partieller_ableitungen}
    Sei $\vec b = (b_j)_{j \in \mathbb{N}} \in \mathbb{R}^{\mathbb{N}}$ die durch $b_{j} = \beta^{-1} \norm{\tilde{\varphi}_{j}}_{L_{\infty}(\Omega)}$ gegebene Folge, wobei $\beta$ die nach \cref{satz:raum_zeit_variationsformulierung_sachgemaess_gestellt} existierende parameterunabhängige inf-sup"=Konstante ist.
    Dann gilt
    \begin{equation}
        \label{eq:abschaetzung_norm_partieller_ableitungen}
        \sup_{\bm\sigma \in \mathcal P} \norm{\partial^{\bm\nu}_{\bm\sigma} u(\bm\sigma)}_{\mathcal X} \leq \frac{\norm{f}_{\mathcal Y'}}{\beta} \abs{\bm\nu}! \vec b^{\bm\nu} \quad \fa \bm \nu \in \mathcal F.
    \end{equation}

    \begin{Beweis}
        Wir beginnen damit eine Darstellung der Variationsprobleme, welche von den partiellen Ableitungen erfüllt werden, herzuleiten.
        Diese lassen sich durch
        \begin{equation}
        \label{eq:abschaetzung_norm_partieller_ableitungen:rekursiv}
            b(\partial^{\bm\nu}_{\bm\sigma} u(\bm\sigma), v; \bm\sigma)
            = - \sum_{\Set{j \given \nu_{j} \neq 0}} \nu_{j} \int_{I} \tilde{\chi}_{j} \skp{\tilde{\varphi}_{j} \partial^{\bm\nu - \vec e_{j}}_{\bm\sigma} u(\bm\sigma)}{v_{1}}{H} \diff t.
        \end{equation}
        rekursiv darstellen, was wir im Folgenden induktiv zeigen werden.

        Den Fall $\abs{\bm\nu} = 1$ haben wir in \cref{bemerkung:alternative_herleitung_variationsproblem_ableitung} bereits gezeigt.
        Sei nun also $\abs{\bm\nu} > 1$.
        Sei weiter $k \in \mathbb{N}$ ein Index mit $\nu_{k} > 0$, dann definieren wir $\tilde{\bm\nu} := \bm\nu - \vec e_{k}$ und es gilt offenbar $\abs{\tilde{\bm\nu}} = \abs{\bm\nu} - 1$.
        Nach Induktionsvoraussetzung gilt damit
        \begin{equation}
            b(\partial^{\tilde{\bm\nu}}_{\bm\sigma} u(\bm\sigma), v; \bm\sigma) + \sum_{\Set{j \given \tilde{\nu}_{j} \neq 0}} \tilde{\nu}_{j} \int_{I} \tilde{\chi}_{j} \skp{\tilde{\varphi}_{j} \partial^{\tilde{\bm\nu} - \vec e_{j}}_{\bm\sigma} u(\bm\sigma)}{v_{1}}{H} \diff t = 0,
        \end{equation}
        wobei nach Definition $\nu_{j} = \tilde{\nu}_{j}$ für $j \neq k$ und $\tilde{\nu}_{k} = \nu_{k} - 1$ ist.
        Partielles Differenzieren dieser Gleichung nach $\sigma_{k}$, analog zu \cref{bemerkung:alternative_herleitung_variationsproblem_ableitung}, liefert dann die Gleichung
        \begin{align}
            0 &=
                b(\partial^{\bm\nu}_{\bm\sigma} u(\bm\sigma), v; \bm\sigma)
           \\&\qquad          + \int_{I} \tilde{\chi}_{k} \skp{\tilde{\varphi}_{k} \partial^{\bm\nu - \vec e_{k}}_{\bm\sigma} u(\bm\sigma)}{v_{1}}{H} \diff t
                + (\nu_{k} - 1) \int_{I} \tilde{\chi}_{k} \skp{\tilde{\varphi}_{k} \partial^{\bm\nu - \vec e_{k}}_{\bm\sigma} u(\bm\sigma) }{v_{1}}{H} \diff t
           \\&\qquad     + \sum_{\Set{j \neq k \given \nu_{j} \neq 0}} \nu_{j} \int_{I} \tilde{\chi}_{j} \skp{\tilde{\varphi}_{j} \partial^{\bm\nu - \vec e_{j}}_{\bm\sigma} u(\bm\sigma)}{v_{1}}{H} \diff t,
        \end{align}
        welche nach Zusammenfassen der Summanden Gleichung \cref{eq:abschaetzung_norm_partieller_ableitungen:rekursiv} entspricht.

        Für die rechte Seite von \cref{eq:abschaetzung_norm_partieller_ableitungen:rekursiv} können wir wie zuvor nachweisen, dass es sich um ein stetiges lineares Funktional auf $\mathcal Y$ handelt.
        Wir können also \cref{satz:raum_zeit_variationsformulierung_sachgemaess_gestellt} verwenden um die Abschätzung
        \begin{equation}
            \norm{\partial^{\bm\nu}_{\bm\sigma} u(\bm\sigma)}_{\mathcal X} \leq \frac{1}{\beta} \norm{\sum_{\Set{j \given \nu_{j} \neq 0}} \nu_{j} \int_{I} \tilde{\chi}_{j} \skp{\tilde{\varphi}_{j} \partial^{\bm\nu - \vec e_{j}}_{\bm\sigma} u(\bm\sigma)}{v_{1}}{H} \diff t}_{\mathcal Y'}
        \end{equation}
        zu erhalten.
        Wir wollen nun die $\mathcal Y'$-Norm der rechten Seite weiter abschätzen.
        Dazu verwenden wir erneut die Cauchy"=Schwarz"=Ungleichung und erhalten wegen
        \begin{align}
            &\abs{\sum_{\Set{j \given \nu_{j} \neq 0}} \nu_{j} \int_{I} \tilde{\chi}_{j} \skp{\tilde{\varphi}_{j} \partial^{\bm\nu - \vec e_{j}}_{\bm\sigma} u(\bm\sigma)}{v_{1}}{H} \diff t}
            \\\leq~
            &\sum_{\Set{j \given \nu_{j} \neq 0}} \nu_{j} \abs {\int_{I} \tilde{\chi}_{j} \skp{\tilde{\varphi}_{j} \partial^{\bm\nu - \vec e_{j}}_{\bm\sigma} u(\bm\sigma)}{v_{1}}{H} \diff t}
            \\\leq~
            &\sum_{\Set{j \given \nu_{j} \neq 0}} \nu_{j} \norm{\tilde{\varphi}_{j}}_{L_{\infty}(\Omega)} \norm{\partial^{\bm\nu - \vec e_{j}}_{\bm\sigma} u(\bm\sigma)}_{L_{2}(I; H)} \norm{v_{1}}_{L_{2}(I; H)}
            \\\leq~
            &\sum_{\Set{j \given \nu_{j} \neq 0}} \nu_{j} \norm{\tilde{\varphi}_{j}}_{L_{\infty}(\Omega)} \norm{\partial^{\bm\nu - \vec e_{j}}_{\bm\sigma} u(\bm\sigma)}_{\mathcal X} \norm{v}_{\mathcal Y},
        \end{align}
        weiter die Abschätzung
        \begin{equation}
            \label{eq:abschaetzung_norm_partieller_ableitungen:rekursive_schranke}
            \norm{\partial^{\bm\nu}_{\bm\sigma} u(\bm\sigma)}_{\mathcal X} \leq \sum_{\Set{j \given \nu_{j} \neq 0}} \nu_{j} \frac{\norm{\tilde{\varphi}_{j}}_{L_{\infty}(\Omega)}}{\beta} \norm{\partial^{\bm\nu - \vec e_{j}}_{\bm\sigma} u(\bm\sigma)}_{\mathcal X}.
        \end{equation}

        Um nun die eigentliche Behauptung zu beweisen, verfolgen wir erneut einen Induktionsansatz.
        Sei zunächst $\abs{\bm\nu} = 0$, dann entspricht
        \begin{equation}
            \sup_{\bm \sigma \in \mathcal P} \norm{u(\bm\sigma)}_{\mathcal X} \leq \frac{\norm{f}_{\mathcal Y'}}{\beta}
        \end{equation}
        Ungleichung \cref{eq:abschaetzung_norm_partieller_ableitungen} und ist nach \cref{satz:raum_zeit_variationsformulierung_sachgemaess_gestellt} erfüllt.
        Sei also weiter $\abs{\bm\nu} > 0$, dann gilt für die rekursive Darstellung \cref{eq:abschaetzung_norm_partieller_ableitungen:rekursive_schranke} unter Verwendung der Induktionsvoraussetzung \cref{eq:abschaetzung_norm_partieller_ableitungen} für $\norm{\partial^{\bm\nu - \vec e_{j}}_{\bm\sigma} u(\bm\sigma)}_{\mathcal X}$ die Abschätzung
        \begin{align}
            \norm{\partial^{\bm\nu}_{\bm\sigma} u(\bm\sigma)}_{\mathcal X}
            &\leq
            \sum_{\Set{j \given \nu_{j} \neq 0}} \nu_{j} b_{j} \norm{\partial^{\bm\nu - \vec e_{j}}_{\bm\sigma} u(\bm\sigma)}_{\mathcal X}
            \\&\leq
            \sum_{\Set{j \given \nu_{j} \neq 0}} \nu_{j} b_{j} \frac{\norm{f}_{\mathcal Y'}}{\beta} \abs{\bm\nu - \vec e_{j}}! \vec b^{\bm\nu - \vec e_{j}}
            \\&=
            \bigg( \sum_{\Set{j \given \nu_{j} \neq 0}} \nu_{j} \bigg) \bigg( \frac{\norm{f}_{\mathcal Y'}}{\beta} (\abs{\bm\nu} - 1)! \vec b^{\bm\nu} \bigg)
            \\&=
            \frac{\norm{f}_{\mathcal Y'}}{\beta} \abs{\bm\nu}! \vec b^{\bm\nu}
         \end{align}
         und damit die Behauptung.
    \end{Beweis}
\end{Satz}

Bevor zusätzliche Annahmen notwendig werden, um die benötigten Aussagen zu beweisen, wird an dieser Stelle zunächst erläutert, wie die obigen Aussagen zum Nachweis der analytischen Abhängigkeit der Lösung vom Parameter beitragen.

\begin{Definition}
\label{definition:analytisch}
    Wir nennen die Abbildung $\mathcal P \ni \bm \sigma \mapsto u(\bm \sigma) \in \mathcal X$ \emph{analytisch}, wenn sie in jedem $\bm\sigma_{0} \in \mathcal P$ als lokal gleichmäßig konvergente Potenzreihe
    \begin{equation}
        \sum_{\bm \nu \in \mathcal F} t_{\bm \nu} (\bm \sigma - \bm \sigma_{0})^{\bm \nu}
    \end{equation}
    dargestellt werden kann.
\end{Definition}

Ist die Abbildung analytisch, dann entspricht die Potenzreihe gerade ihrer Taylorreihe.
Diese Eigenschaft wollen wir ausnutzen, denn in diesem Fall sind die Koeffizienten $t_{\bm \nu}$ gerade durch
\begin{equation}
    t_{\bm \nu} = \frac{1}{\bm\nu!} \partial^{\bm \nu}_{\bm \sigma} u(\bm \sigma_{0})
\end{equation}
gegeben.

Betrachten wir beispielsweise die Taylorreihe von $u(\blank)$ um den Nullpunkt $\vec 0 \in \mathcal P$, dann erhalten wir wegen $\mathcal P = [-1, 1]^{\mathbb{N}}$ die Abschätzung
\begin{equation}
    \sup_{\bm\sigma \in \mathcal P} \norm{\sum_{\bm\nu \in \mathcal F} t_{\bm\nu} \bm\sigma^{\bm\nu}}_{\mathcal X}
    \leq \sup_{\bm\sigma \in \mathcal P} \sum_{\bm\nu \in \mathcal F} \norm{t_{\bm\nu} \bm\sigma^{\bm\nu}}_{\mathcal X}
    \leq \sum_{\bm\nu \in \mathcal F} \norm{t_{\bm\nu}}_{\mathcal X}
    \leq \sum_{\bm \nu \in \mathcal F} \frac{1}{\bm\nu!} \norm{ \partial^{\bm \nu}_{\bm \sigma} u(\vec 0)}_{\mathcal X},
\end{equation}
beziehungsweise nach \cref{satz:abschaetzung_norm_partieller_ableitungen} weiter
\begin{equation}
    \sup_{\bm\sigma \in \mathcal P} \norm{\sum_{\bm\nu \in \mathcal F} t_{\bm\nu} \bm\sigma^{\bm\nu}}_{\mathcal X}
    \leq \frac{\norm{f}_{\mathcal Y'}}{\beta} \sum_{\bm \nu \in \mathcal F} \frac{\abs{\bm\nu}!}{\bm\nu!} \vec b^{\bm\nu}.
\end{equation}
Die Frage, ob die Abbildung $\mathcal P \ni \bm \sigma \mapsto u(\bm \sigma) \in \mathcal X$ analytisch ist, hat sich somit auf jene, unter welchen Bedingungen $\sum_{\bm \nu \in \mathcal F} (\abs{\bm\nu}!)(\bm\nu!)^{-1} \vec b^{\bm\nu}$ konvergiert, reduziert.
Wann dies der Fall ist, wurde beispielsweise in \cite[Theorem 7.2]{Cohen:2010kz} untersucht.
Wir geben die Aussage hier ohne Beweis wieder.

\begin{Satz}
\label{satz:cohen2010kz:theorem72}
    Sei $0 < p \leq 1$.
    Die Folge $(\frac{\abs{\bm\nu}!}{\bm\nu!} \vec b^{\bm\nu})_{\bm\nu \in \mathcal F}$ liegt genau dann in $\ell_{p}(\mathcal F)$, wenn $\norm{\vec b}_{\ell_{1}(\mathbb{N})} < 1$ und $\vec b \in \ell_{p}(\mathbb{N})$ gilt.
\end{Satz}

\begin{Satz}
\label{satz:loesungen_analytisch}
    Das Funktionensystem $\Set{\varphi_{j}}_{j \in \mathbb{N}}$ sei so gewählt, dass $\vec b \in \ell_{1}(\mathbb{N})$, definiert als $b_{j} = \beta^{-1} \norm{\varphi_{j}}_{L_{\infty}(\Omega)}$, die Bedingung $\norm{\vec b}_{\ell_{1}(\mathbb{N})} < 1$ erfüllt.
    Dann hängt die Lösung $u(\bm \sigma)$ des parametrischen Raum"=Zeit"=Variationsproblems \cref{eq:parametrisches_rz_variationsproblem} analytisch vom Parameter $\bm \sigma \in \mathcal P$ ab.

    \begin{Beweis}
        Der vorherige Satz und vorangegangene Überlegungen liefern unter diesen Voraussetzungen die behauptete Aussage.
    \end{Beweis}
\end{Satz}

Die Bedingung $\norm{\vec b}_{\ell_{1}(\mathbb{N})} < 1$ kann dabei so interpretiert werden, dass der feld-abhängige Teil der partiellen Differentialgleichung den feld-unabhängigen Anteil nicht zu stark stören darf, um analytische Abhängigkeit garantieren zu können.
Dies verträgt sich allerdings nur bedingt mit der Motivation der Propagator-Differentialgleichung, da bei dieser die Felder einen starken Einfluss haben.

In \cref{chapter:galerkin} werden wir auf diesen Punkt noch einmal eingehen und anhand der numerischen Ergebnisse deutlich machen, inwiefern die Bedingung eingehalten wird respektive eingehalten werden kann.

\section{Periodische Randbedingungen} % (fold)
\label{section:periodische_randbedingungen}

Da wir in diesem Kapitel bisher ausschließlich mit homogenen Dirichlet"=Randbedingungen gearbeitet haben, wollen wir an dieser Stelle auf den Fall periodischer Randbedingungen eingehen.
Dabei werden wir feststellen, dass Aufgrund der Struktur der Propagator-Differentialgleichung nur sehr geringe Unterschiede zum betrachteten homogenen Fall bestehen.

Zunächst müssen wir die Rahmenbedingungen für die Betrachtung periodischer Randbedingungen festlegen.
Dazu beschränken wir uns auf den Fall, dass $\Omega = \bigtimes_{i = 1}^{n} (0, l_{i}) \subset \mathbb{R}^{n}$ ein beschränkter offener Quader ist, wobei $l_{i} \in \mathbb{R}_{+}$ für $i = 1 \dots n$ sei.
Weiter führen wir nun die Äquivalente des Lebesgue-Raums $L_{2}(\Omega)$ und des Sobolev-Raums $H^{1}(\Omega)$ für periodische Funktionen ein.
Da dies für die diese Arbeit von untergeordneter Bedeutung ist, wird an dieser Stelle nur ein Überblick über die benötigten Ergebnisse gegeben, und kann beispielsweise ausführlicher bei \textcite{Han2009} gefunden werden.

\begin{Definition}
\label{definition:periodische_sobolev_raeume}
    Sei $\mathcal C_{\text{per}}^{\infty}(\Omega) \subset \mathcal C^{\infty}(\mathbb{R}^{n})$ die Teilmenge der glatten $\Omega$-periodischen Funktionen.
    Als den Lebesgue-Raum $\Omega$-periodischer Funktionen $L_{2,\text{per}}(\Omega)$  definieren wir den Abschluss von $C_{\text{per}}^{\infty}(\Omega)$ bezüglich der $L_{2}$-Norm.
    Weiter definieren wir den Sobolev-Raum $\Omega$-periodischer Funktionen $H^{1}_{\text{per}}(\Omega)$ als den Abschluss $C_{\text{per}}^{\infty}(\Omega)$ bezüglich der $H^{1}$-Norm.
\end{Definition}

Diese Räume verwenden wir nun, um die Räume $V$ und $H$, vergleiche \cref{bemerkung:raeume_und_gelfand_tripel}, zu definieren.
Nach Konstruktion handelt es sich bei $H^{1}_{\text{per}}(\Omega)$ und $L_{2,\text{per}}(\Omega)$ um Hilberträume und desweiteren ist $H^{1}_{\text{per}}(\Omega)$ ein dichter Unterraum von $L_{2,\text{per}}(\Omega)$.
Wählen wir also $V = H^{1}_{\text{per}}(\Omega)$ und $H = L_{2,\text{per}}(\Omega)$, dann erhalten wir nach \cref{definition:gelfand_tripel} wie zuvor ein Gelfand-Tripel der Form
\begin{equation}
    V \denseinclusion H \cong H' \denseinclusion V' = (H^{1}_{\text{per}}(\Omega))'.
\end{equation}

Untersucht man die Ausführungen dieses Kapitels für den Fall homogener Dirichlet-Randbedingungen, dann stellt man fest, dass lediglich die aus \cref{annahme:eigenschaften_der_bilinearform_a} stammenden Bedingungen an die Bilinearform $a(\blank, \blank)$, deren Darstellung durch den Wechsel zu periodischen Randbedingungen unverändert bleibt, nachgewiesen werden müssen.

\begin{Lemma}
\label{lemma:bilinearform_periodisch_messbar_stetig_garding}
    Sei $a(\blank, \blank; t)$, $t \in I$, die Familie von Bilinearformen aus \cref{definition:operator_bilinearform_zeit}, wobei die Hilberträume als $V = H^{1}_{\text{per}}(\Omega)$ und $H = L_{2,\text{per}}(\Omega)$ gegeben seien.
    Diese Bilinearformen erfüllen die folgenden Eigenschaften:
    \begin{thmenumerate}
        \item \label{lemma:bilinearform_periodisch_messbar_stetig_garding:messbar}
        \emph{Messbarkeit:} Die Abbildung $I \ni t \mapsto a(\eta, \zeta; t)$ ist messbar für alle $\eta, \zeta \in V$.
        \item\label{lemma:bilinearform_periodisch_messbar_stetig_garding:stetig}
        \emph{Stetigkeit:} Es gilt
        \begin{equation}
            \label{eq:bilinearform_periodisch_messbar_stetig_garding:stetig}
            \abs{a(\eta, \zeta; t)} \leq \gamma_{a} \norm{\eta}_{V} \norm{\zeta}_{V}, \quad \fa \eta, \zeta \in V,
        \end{equation}
        mit Stetigkeitskonstante $\gamma_{a} = \max\Set{c, \norm{\omega}_{L_{\infty}(I; L_{\infty}(\Omega))} + \abs{\mu}} < \infty$.
        \item\label{lemma:bilinearform_periodisch_messbar_stetig_garding:garding}
        \emph{G\aa{}rding-Ungleichung:} Es gilt
        \begin{equation}
            \label{eq:bilinearform_periodisch_messbar_stetig_garding:garding}
            a(\eta, \eta; t) + \lambda \norm{\eta}_{H}^{2} \geq \alpha \norm{\eta}_{V}^{2}, \quad \fa \eta \in V,
        \end{equation}
        mit $\alpha = c > 0$ und $\lambda = \max\Set{\norm{\omega}_{L_{\infty}(I;L_{\infty}(\Omega))} - \mu + c, c} \geq c > 0$.
    \end{thmenumerate}

    \begin{Beweis}
        Die Nachweise der Messbarkeit und Stetigkeit bleiben unverändert wie in \cref{satz:bilinearform_messbar_stetig_garding} und werden hier nicht wiederholt.

        Für die G\aa{}rding-Ungleichung sei zunächst $\eta \in V$ beliebig und $\lambda \in \mathbb{R}$, dann gilt
        \begin{align}
            a(\eta, \eta; t) + \lambda \norm{\eta}^{2}_{H}
            &= c \norm{\grad \eta}^{2}_{H} + \skprod{\omega(t, \blank) \eta}{\eta}_{H} + \mu \skprod{\eta}{\eta}_{H} + \lambda \skprod{\eta}{\eta}_{H}
            \\&= c \norm{\grad \eta}^{2}_{H} + \skprod{\omega(t, \blank) + \mu + \lambda) \eta}{\eta}_{H}.
        \end{align}
        Wählen wir $\lambda = \max\Set{\norm{\omega}_{L_{\infty}(I; L_{\infty}(\Omega))} - \mu + c, c} \geq c > 0$, dann gilt $\omega(t, \blank) + \mu + \lambda - c \geq 0$ fast überall in $\Omega$ und wir erhalten die Abschätzung
        \begin{align}
            a(\eta, \eta; t) + \lambda \norm{\eta}^{2}_{H}
            &= c \norm{\grad \eta}^{2}_{H} + \skprod{(\omega(t, \blank) + \mu + \lambda - c) \eta}{\eta}_{H} + c \norm{\eta}^{2}_{H} \\
            &\geq c \norm{\grad \eta}^{2}_{H} + c \norm{\eta}^{2}_{H}
            \\&= c \norm{\eta}^{2}_{V}.
        \end{align}
    \end{Beweis}
\end{Lemma}

Auf dieses Lemma aufbauend können die restlichen Ergebnisse des Kapitels analog auch auf den periodischen Fall übertragen werden.

% section periodische_randbedingungen (end)
