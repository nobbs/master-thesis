% -*- root: main.tex -*-

%%% Grundlegendes

% Disable the koma-tocstyle warning...
\usepackage{silence}
\WarningFilter{tocstyle}{THIS IS AN ALPHA VERSION!}

% Verwende etex als zugrundeliegende Implementierung (-> mehr Speicher für Tex!)
\usepackage{etex}
% Encoding
\usepackage[utf8]{inputenc}

% subfile-System
\usepackage{subfiles}
% etoolbox?
\usepackage{etoolbox}

%%% Schriften, Lokalisierung und Typographie

% T1 Schriftsystem verwenden
\usepackage[T1]{fontenc}
% LModern als Standard-Schrift
\usepackage{lmodern}
% Typographische Kleinigkeiten
\usepackage[final]{microtype}
% Fraktur-Schriftsatz
% \usepackage{eufrak}
% Fette Griechische
\usepackage{bm}

%% Lokalisierung

% Deutsche Silbentrennung
\usepackage[english,ngerman]{babel}
% Deutsche Anführungszeichen
\usepackage[babel,german=quotes]{csquotes}

% Zeilenabstand anpassen (MUST NOT USE!)
% \usepackage{setspace}
% \setstretch{1.1}

% Textsatzbereich nach unten vergrößern
% TODO: richtig konfigurieren
% \usepackage[a4paper,twoside,bottom=4cm]{geometry}

%%% Graphiken und Farben

% Graphiken und Farben
\usepackage{xcolor}
\usepackage{graphicx}

% Vordefinierte Farben laden
\definecolor{jgu_rot}{RGB}{193,0,42}
\definecolor{jgu_hellgrau}{RGB}{172,172,172}
\definecolor{jgu_dunkelgrau}{RGB}{99,99,99}
\definecolor{matlab_blau}{rgb}{0.00000,0.44700,0.74100}
\definecolor{matlab_orange}{rgb}{0.85000,0.32500,0.09800}


% Caption anpassen und Subfigures erlauben
\usepackage[style=base,font+=small,labelfont+=bf,margin=1em]{caption}
\usepackage{subcaption}

% Tikz ist kein Zeichenprogramm!
\usepackage{tikz}
\usepackage{pgfplots}
\pgfplotsset{compat=1.12}
\usetikzlibrary{patterns}

% Standalone-Paket, damit die Tikz Bilder extern auch funktionieren
\usepackage{standalone}


%%% Bibliographie

\usepackage[%
    backend=biber,
    style=alphabetic,
    firstinits=true
]{biblatex}

% Fix für doi2bib generierte Bibliography-Einträge
\newcommand\mathplus{+}

% et al statt u.a.
\DefineBibliographyStrings{ngerman}{
   andothers = {{et~al\adddot}},
}

% kein In bei article
\renewbibmacro{in:}{%
  \ifentrytype{article}{}{\printtext{\bibstring{in}\intitlepunct}}}

% Volume und Issue no. angeben
\DeclareFieldFormat[article]{volume}{\bibstring{volume}\addnbspace #1}
\DeclareFieldFormat[article]{number}{\bibstring{number}\addnbspace #1}
\renewbibmacro*{volume+number+eid}{%
  \printfield{volume}%
  \setunit{\addcomma\space}%<---- was \setunit*{\adddot}%
  \printfield{number}%
  \setunit{\addcomma\space}%
  \printfield{eid}}


% \renewbibmacro*{volume+number+eid}{%
%   \printfield{volume}%
% %  \setunit*{\adddot}% DELETED
%   \setunit*{\addnbspace}% NEW (optional); there's also \addnbthinspace
%   \printfield{number}%
%   \setunit{\addcomma\space}%
%   \printfield{eid}}
% % \DeclareFieldFormat[article]{number}{\mkbibparens{#1}}



%%% Grundlegende Mathematik-Pakete

% Standardpackages
\usepackage{amsmath}
\usepackage{amssymb}
\usepackage{stmaryrd}
\usepackage{mathtools}

% Bessere Ausrichtung von : vor =
\mathtoolsset{centercolon}

% Erlaube Seitenumbrüche in Gleichungen
\allowdisplaybreaks

% "Bessere" Theorem-Umgebungen
\usepackage[
    amsmath,
    thmmarks,
    hyperref,
]{ntheorem}


%%% Hyperref

\usepackage[
	colorlinks=false,
	linkcolor=jgu_rot,          % color of internal links (change box color with linkbordercolor)
	citecolor=jgu_rot,        % color of links to bibliography
	filecolor=jgu_rot,      % color of file links
	urlcolor=jgu_rot,
    hypertexnames=false
]{hyperref}


%%% Clevere Referenzen

\usepackage[german,noabbrev,nameinlink,capitalise]{cleveref}
\crefname{equation}{}{}

% Nummern nur für referenzierte Gleichungen
\usepackage{autonum}


%%% Layout und Stil

% Fancyhdr-Ersatz für scrbook
\usepackage[
    automark,
    % headsepline,
    draft=false
]{scrlayer-scrpage}

% Inhaltsverzeichnis anpassen (Schriften angleichen!)
\usepackage{tocstyle}
\usetocstyle{KOMAlike}

%% Chapter-Style und mehr

%% BEGIN STYLE
% Schrift vordefinieren
\newcommand{\altfont}{\fontfamily{qpl}}

% Alle Überschriften in TeX Gyre Pagella
\addtokomafont{disposition}{\normalfont\altfont\selectfont}
% Alle Überschriften in Dunkelgrau
% \addtokomafont{disposition}{\color{jgu_dunkelgrau}}

% Einzelne Optionen bei Bedarf
% \addtokomafont{chapter}{\normalfont\altfont\selectfont}
% \addtokomafont{chapter}{\color{black}}
% \addtokomafont{section}{\normalfont\altfont\selectfont}
% \addtokomafont{subsection}{\normalfont\altfont\selectfont}
% \addtokomafont{subsubsection}{\normalfont\altfont\selectfont}
% \addtokomafont{minisec}{\normalfont\altfont\selectfont}
% \addtokomafont{paragraph}{\normalfont\altfont\selectfont}
\addtokomafont{paragraph}{\bfseries}

% Kopfzeile anpassen
% \addtokomafont{pageheadfoot}{\normalfont\altfont\selectfont}
% \addtokomafont{pageheadfoot}{\color{jgu_dunkelgrau}\normalfont\altfont\selectfont}
% Seitenzahl anpassen
% \addtokomafont{pagenumber}{\normalfont\altfont\selectfont}
% \addtokomafont{pagenumber}{\color{jgu_dunkelgrau}\normalfont\altfont\selectfont}

% Kapitel-Stil
\renewcommand*{\chapterheadstartvskip}{\vspace*{3\baselineskip}}
\renewcommand*{\chapterheadendvskip}{\vspace*{2\baselineskip}}
\renewcommand*{\chapterformat}{%
                \raggedright
                \color{jgu_rot}
                \altfont\fontsize{60}{30}\selectfont\thechapter
                \fontsize{20}{30}\scshape\selectfont\enskip\chapappifchapterprefix
                }
\renewcommand*{\raggedchapter}{\raggedleft}
%% END STYLE

% Zitatblock für Kapitelanfang
\renewcommand*\dictumwidth{.35\linewidth}
\renewcommand*\dictumrule{}
\renewcommand*\dictumauthorformat[1]{--- #1}
\addtokomafont{dictumtext}{\normalfont\selectfont\itshape}
\addtokomafont{dictumauthor}{\normalfont\fontsize{8}{11}\selectfont}

% Inhaltsverzeichnis
% \addtokomafont{sectionentrypagenumber}{\color{black}}
\addtokomafont{chapterentrypagenumber}{\color{black}\rmfamily}
\addtokomafont{chapterentry}{\rmfamily\bfseries}


%%% Verzeichnisse

% Akronymverzeichnis
\usepackage{acro}
\acsetup{
    page-ref=paren,
    pages=first,
    page-name={siehe S.\@\,},
}

% ntheorem-Umgebungen laden
%%
%% This is file `ntheorem.std',
%% generated with the docstrip utility.
%%
%% The original source files were:
%%
%% ntheorem.dtx  (with options: `standard')
%%
%% IMPORTANT NOTICE:
%%
%% For the copyright see the source file.
%%
%% Any modified versions of this file must be renamed
%% with new filenames distinct from ntheorem.std.
%%
%% For distribution of the original source see the terms
%% for copying and modification in the file ntheorem.dtx.
%%
%% This generated file may be distributed as long as the
%% original source files, as listed above, are part of the
%% same distribution. (The sources need not necessarily be
%% in the same archive or directory.)
\def\filedate{2011/08/15}
\def\docdate{2011/08/15}
\def\fileversion{1.33}
\def\basename{ntheorem}
%% This file may be distributed and/or modified under the
%% conditions of the LaTeX Project Public License, either version 1.2
%% of this license or (at your option) any later version.
%% The latest version of this license is in
%%    http://www.latex-project.org/lppl.txt
%% and version 1.2 or later is part of all distributions of LaTeX
%% version 1999/12/01 or later.
%% \CharacterTable
%%  {Upper-case    \A\B\C\D\E\F\G\H\I\J\K\L\M\N\O\P\Q\R\S\T\U\V\W\X\Y\Z
%%   Lower-case    \a\b\c\d\e\f\g\h\i\j\k\l\m\n\o\p\q\r\s\t\u\v\w\x\y\z
%%   Digits        \0\1\2\3\4\5\6\7\8\9
%%   Exclamation   \!     Double quote  \"     Hash (number) \#
%%   Dollar        \$     Percent       \%     Ampersand     \&
%%   Acute accent  \'     Left paren    \(     Right paren   \)
%%   Asterisk      \*     Plus          \+     Comma         \,
%%   Minus         \-     Point         \.     Solidus       \/
%%   Colon         \:     Semicolon     \;     Less than     \<
%%   Equals        \=     Greater than  \>     Question mark \?
%%   Commercial at \@     Left bracket  \[     Backslash     \\
%%   Right bracket \]     Circumflex    \^     Underscore    \_
%%   Grave accent  \`     Left brace    \{     Vertical bar  \|
%%   Right brace   \}     Tilde         \~}

\theoremnumbering{arabic}
\theoremstyle{plain}
\RequirePackage{latexsym}
% \theoremsymbol{\ensuremath{_\Box}}
\theorembodyfont{\itshape}
\theoremheaderfont{\normalfont\bfseries}
\theoremseparator{}
\newtheorem{Theorem}{Theorem}[chapter]
\newtheorem{theorem}[Theorem]{Theorem}
\newtheorem{Satz}[Theorem]{Satz}
\newtheorem{satz}[Theorem]{Satz}
\newtheorem{Proposition}[Theorem]{Proposition}
\newtheorem{proposition}[Theorem]{Proposition}
\newtheorem{Lemma}[Theorem]{Lemma}
\newtheorem{lemma}[Theorem]{Lemma}
\newtheorem{Korollar}[Theorem]{Korollar}
\newtheorem{korollar}[Theorem]{Korollar}
\newtheorem{Corollary}[Theorem]{Corollary}
\newtheorem{corollary}[Theorem]{Corollary}

\theorembodyfont{\upshape}
\newtheorem{Example}[Theorem]{Example}
\newtheorem{example}[Theorem]{Example}
\newtheorem{Beispiel}[Theorem]{Beispiel}
\newtheorem{beispiel}[Theorem]{Beispiel}
\newtheorem{Bemerkung}[Theorem]{Bemerkung}
\newtheorem{bemerkung}[Theorem]{Bemerkung}
\newtheorem{Anmerkung}[Theorem]{Anmerkung}
\newtheorem{anmerkung}[Theorem]{Anmerkung}
\newtheorem{Remark}[Theorem]{Remark}
\newtheorem{remark}[Theorem]{Remark}
\newtheorem{Definition}[Theorem]{Definition}
\newtheorem{definition}[Theorem]{Definition}
\newtheorem{Annahme}[Theorem]{Annahme}
\newtheorem{annahme}[Theorem]{Annahme}

\theoremstyle{nonumberplain}
\theoremheaderfont{\scshape}
\theorembodyfont{\normalfont}
\theoremseparator{.}
\theoremsymbol{\ensuremath{_\square}}
\RequirePackage{amssymb}
\newtheorem{Proof}{Proof}
\newtheorem{proof}{Proof}
\newtheorem{Beweis}{Beweis}
\newtheorem{beweis}{Beweis}
\qedsymbol{\ensuremath{_\square}}

\theoremstyle{nonumberplain}
\theoremheaderfont{\normalfont\bfseries}
\theoremseparator{}
\theorembodyfont{\normalfont}
\theoremsymbol{}
\RequirePackage{amssymb}
\newtheorem{Problem}{Problem}
\newtheorem{problem}{Problem}
\qedsymbol{}

\theoremclass{LaTeX}
\endinput
%%
%% End of file `ntheorem.std'.



%%% Alles andere

% Querformatseiten
% \usepackage{pdflscape}

% Enumerates anpassen
\usepackage{enumitem}
% Numerierte Umgebung für Theoreme definieren
\newlist{thmenumerate}{enumerate}{1}
\setlist[thmenumerate]{label={\upshape(\roman*)}, align=left, widest=iii, leftmargin=*}
\crefname{thmenumeratei}{}{}

% Konturen um Text zeichnen
\usepackage[outline]{contour}
\contourlength{0.09em}

% Gepunktete Linien
\usepackage{dashrule}


%%% "Debugging"

% Labels am Seitenrand anzeigen
% \usepackage[
%     % inline,
%     % final
% ]{showlabels}

% todos
% \usepackage[%
%     german,
%     colorinlistoftodos,
%     % disable
% ]{todonotes}

% % Todos vordefinieren
% \newcommand{\mdo}[1]{\todo[color=yellow!50]{#1}}
% \newcommand{\mfix}[1]{\todo[color=red!50]{#1}}
% \newcommand{\mwarn}[1]{\todo[color=blue!50]{#1}}

% Blindtext
% \usepackage{blindtext}
% \blindmathtrue{}

% \usepackage{showframe}

% Verzeichnisstruktur setzen
\usepackage{dirtree}

% Symbolverzeichnis
\usepackage[nopostdot,toc]{glossaries}
\newglossary[slg]{symbolslist}{syi}{syg}{Symbolverzeichnis}
\makeglossaries

% Algorithm
\usepackage[german,algochapter,ruled,vlined]{algorithm2e}
\crefname{algocf}{Algorithmus}{Algorithmen}
\Crefname{algocf}{Algorithmus}{Algorithmen}
