%!TEX root = ../main.tex

\chapter{Petrov-Galerkin-Verfahren} % (fold)
\label{chapter:galerkin}

In diesem Kapitel wiederholen wir das in der Numerik wohlbekannte Galerkin-Verfahren, welches als Grundlage für die Reduzierte-Basis-Methoden des nächsten Kapitels dienen wird.
Dies wird, wie es auch bereits bei den funktionalanalytischen Grundlagen in \cref{chapter:grundlagen} der Fall war, möglichst allgemein gehalten.
Als Quellen für dieses Kapitel dienten vorwiegend die Arbeiten von \textcite{Braess:2007wm,Patera:2007un,Quarteroni:2011jm}, weitere werden an den entsprechenden Stellen angegeben.

Als Setting für diese Ausführungen soll das folgende abstrakte Variationsproblem dienen, wobei wir uns an die bereits bekannten Notationen halten: Seien $\mathcal X$ und $\mathcal Y$ zwei Hilberträume mit Skalarprodukten $\skp{\blank}{\blank}{\bullet}$ und Normen $\norm{\blank}_{\bullet}$, welche durch entsprechende Indizes markiert werden.
Wir betrachten für eine stetige Bilinearform $b \colon \mathcal X \times \mathcal Y \to \mathbb{R}$ und ein lineares Funktional $f \colon \mathcal Y \to \mathbb{R}$ das folgende Variationsproblem:
\begin{equation}
    \label{eq:gal:abstraktes_variationsproblem}
    \text{Finde}~u \in \mathcal X \colon \quad  b(u, v) = f(v) \quad \fa v \in \mathcal Y.
\end{equation}
Unter welchen Bedingungen dieses Variationsproblem lösbar ist, haben wir bereits in \cref{chapter:grundlagen} diskutiert.
Für den Rest dieses Kapitels nehmen wir an, dass \cref{eq:gal:abstraktes_variationsproblem} sachgemäß gestellt ist und insbesondere die inf-sup-Konstante $\beta$ die Bedingung
\begin{equation}
    \label{eq:gal:abstraktes_variationsproblem_infsup}
    \beta = \inf_{u \in \mathcal X} \sup_{v \in \mathcal Y} \frac{b(u, v)}{\norm{u}_{\mathcal X} \norm{v}_{\mathcal Y}} > 0
\end{equation}
erfüllt.

Im nun folgenden ersten Abschnitt werden einige grundlegende Aussagen zum Petrov-Galerkin-Verfahren behandelt.
Der Name Petrov-Galerkin-Verfahren wird dabei für die Untermenge der Galerkin-Verfahren verwendet, welche auf Variationsprobleme, bei denen Ansatz- und Testraum $\mathcal X$, $\mathcal Y$ ungleich sind, zugeschnitten sind.


\section{Grundlagen des Petrov-Galerkin-Verfahrens} % (fold)
\label{sub:grb:gv:grundlagen}

Die grundlegende Idee hinter den Galerkin-Verfahren ist die Diskretisierung des Variationsproblems \cref{eq:gal:abstraktes_variationsproblem} durch Approximation der im Allgemeinen unendlichdimensionalen Hilberträume $\mathcal X$ und $\mathcal Y$ mittels endlichdimensionaler Unterräume $\mathcal X^{\mathcal N} \subset \mathcal X$ und $\mathcal Y^{\mathcal N} \subset \mathcal Y$.
Wir beschränken uns auf den Fall, dass Ansatz- und Testraum die gleiche Dimension $\mathcal N$ haben.
% Dies ist im Allgemeinen nicht notwendig; denkbar ist auch der Fall $\dim \mathcal Y^{M} > \dim \mathcal X^{N}$, dies erfordert aber eine Reformulierung der nachfolgenden Aussagen im Sinne einer Residuum-Minimierung.

Dieser Abschnitt baut vor allem auf die Ausführungen von \textcite[Section 3.1]{Nochetto:2009il} auf.

Zunächst wollen wir definieren, was wir unter einer Petrov-Galerkin-Lösung, im Folgenden kurz Lösung genannt, verstehen wollen.

\begin{Definition}[Petrov-Galerkin-Lösung]
    \label{def:gal:disrekte_loesung}
    Seien durch $\mathcal X^{\mathcal N} \subset \mathcal X$ und $\mathcal Y^{\mathcal N} \subset \mathcal Y$ Unterräume der Dimension $\mathcal N \in \mathbb{N}$ gegeben.
    Als Petrov-Galerkin-Lösung von \cref{eq:gal:abstraktes_variationsproblem} bezeichnen wir eine Lösung $u^{\mathcal N} \in \mathcal X^{\mathcal N}$ des Variationsproblems:
    \begin{equation}
        \label{eq:gal:pg_variationsproblem}
        \text{Finde}~u^{\mathcal N} \in \mathcal X^{\mathcal N} \colon \quad  b(u^{\mathcal N}, v) = f(v) \quad \fa v \in \mathcal Y^{\mathcal N}.
    \end{equation}
\end{Definition}

Zur einfacheren Unterscheidung bezeichnen wir \cref{eq:gal:abstraktes_variationsproblem} im Weiteren als stetiges und \cref{eq:gal:pg_variationsproblem} als diskretes Variationsproblem.

\begin{Bemerkung}[Zur Wohldefiniertheit]
    \label{bem:gal:zur_wohldefiniertheit}
    Anders als bei Galerkin-Verfahren für koerzive Variationsprobleme vererbt sich die Wohldefiniertheit des stetigen nicht automatisch auf das diskrete Variationsproblem.
    Dies wird deutlich durch
    \begin{equation}
        \sup_{v \in \mathcal Y} \frac{b(u, v)}{\norm{v}_{\mathcal Y}} \geq \sup_{v \in \mathcal Y^{\mathcal N}} \frac{b(u, v)}{\norm{v}_{\mathcal Y}} \quad \fa u \in \mathcal X.
    \end{equation}
    Dadurch ist, selbst wenn die stetige inf-sup-Bedingung \cref{eq:gal:abstraktes_variationsproblem_infsup} gilt, nicht sichergestellt, dass die diskrete inf-sup-Konstante $\beta^{\mathcal N}$ von \cref{eq:gal:pg_variationsproblem} die entsprechende Bedingung
    \begin{equation}
        \label{eq:gal:diskrete_infsup}
        \beta^{\mathcal N} = \inf_{u \in \mathcal X^{\mathcal N}} \sup_{v \in \mathcal Y^{\mathcal N}} \frac{b(u, v)}{\norm{u}_{\mathcal X} \norm{v}_{\mathcal Y}} = \inf_{v \in \mathcal Y^{\mathcal N}} \sup_{u \in \mathcal X^{\mathcal N}} \frac{b(u, v)}{\norm{u}_{\mathcal X} \norm{v}_{\mathcal Y}} > 0
    \end{equation}
    erfüllt.

   Lediglich die Stetigkeit muss dafür nicht explizit nachgewiesen werden, da die diskrete Stetigkeitskonstante $\gamma^{\mathcal N}$ stets durch die des stetigen Variationsproblems von oben beschränkt wird.

   Weiterhin gilt, anders im stetigen Fall, beim diskreten Variationsproblem stets die Gleichheit der beiden inf-sup-Konstanten in \cref{eq:gal:diskrete_infsup}.
\end{Bemerkung}

Der folgende Satz fasst einige äquivalente Bedingungen zusammen, unter denen die Wohldefiniertheit des diskreten Variationsproblems sichergestellt ist:

\begin{Satz}[Wohldefiniertheit des diskreten Variationsproblems]
    \label{satz:gal:wohldefiniertheit}
    Es seien $\mathcal N$-di"-mensionale Unterräume $\mathcal X^{\mathcal N} \subset \mathcal X$ respektive $\mathcal Y^{\mathcal N} \subset \mathcal Y$ gegeben und weiter sei $f \in (\mathcal Y^{\mathcal N})'$.
    Das diskrete Variationsproblem \cref{eq:gal:pg_variationsproblem} besitzt eine eindeutige Lösung $u^{\mathcal N} \in \mathcal X^{\mathcal N}$ genau dann, wenn eine der folgenden äquivalenten Bedingungen erfüllt ist:
    \begin{thmenumerate}
        \item \label{punkt:gal:tfae_diskrete_inf_sup} Es gilt die diskrete inf-sup-Bedingung \cref{eq:gal:diskrete_infsup}.
        \item \label{punkt:gal:tfae_infsup_x} Es gilt
            \begin{equation}
                \inf_{u \in \mathcal X^{\mathcal N}} \sup_{v \in \mathcal Y^{\mathcal N}} \frac{b(u, v)}{\norm{u}_{\mathcal X} \norm{v}_{\mathcal Y}} > 0.
            \end{equation}
        \item \label{punkt:gal:tfae_infsup_y} Es gilt
            \begin{equation}
                \inf_{v \in \mathcal Y^{\mathcal N}} \sup_{u \in \mathcal X^{\mathcal N}} \frac{b(u, v)}{\norm{u}_{\mathcal X} \norm{v}_{\mathcal Y}} > 0.
            \end{equation}
        \item \label{punkt:gal:tfae_injektiv1} Für jedes $0 \neq u \in \mathcal X^{\mathcal N}$ existiert ein $v \in \mathcal Y^{\mathcal N}$ mit $b(u, v) \neq 0$.
        \item \label{punkt:gal:tfae_injektiv2} Für jedes $0 \neq v \in \mathcal Y^{\mathcal N}$ existiert ein $u \in \mathcal X^{\mathcal N}$ mit $b(u, v) \neq 0$.
    \end{thmenumerate}

    \begin{Beweis}
        Siehe \cite[Theorem 3.1, Proposition 3.1]{Nochetto:2009il}.
    \end{Beweis}
\end{Satz}

Obwohl die Bedingungen \cref{punkt:gal:tfae_infsup_x,punkt:gal:tfae_infsup_y,punkt:gal:tfae_injektiv1,punkt:gal:tfae_injektiv2} augenscheinlich angenehmer zu Handhaben sind, hat Bedingung \cref{punkt:gal:tfae_diskrete_inf_sup} herausragende Bedeutung, wie die folgende Stabilitätsaussage zeigt:

\begin{Satz}[Stabilität der diskreten Lösung]
    \label{satz:gal:stabilitaet}
    Gilt die diskrete inf-sup-Bedingung \cref{eq:gal:diskrete_infsup}, dann erfüllt die Petrov-Galerkin-Lösung $u^{\mathcal N} \in \mathcal X^{\mathcal N}$ die Abschätzung
    \begin{equation}
        \label{eq:gal:statibilitaet}
        \norm{u^{\mathcal N}}_{\mathcal X} \leq \frac{1}{\beta^{\mathcal N}} \norm{f}_{\mathcal Y'}.
    \end{equation}

    \begin{Beweis}
        Direkte Folgerung aus dem \ac{bnb}, \cref{satz:gl:le:bnb_theorem}.
    \end{Beweis}
\end{Satz}

Eine wichtige Eigenschaft der Galerkin-Verfahren ist die sogenannte Galerkin-Orthogonalität, mit welcher wir aus obiger Stabilitätsaussage eine Fehlerabschätzung ableiten können.

\begin{Lemma}[Galerkin-Orthogonalität]
    \label{lemma:gal:galerkin_ortho}
    Sei $u \in \mathcal X$ die Lösung des stetigen Variationsproblems \cref{eq:gal:abstraktes_variationsproblem} und $u^{\mathcal N} \in \mathcal X^{\mathcal N}$ die Lösung eines zugehörigen diskreten Variationsproblems.
    Dann gilt
    \begin{equation}
        \label{eq:gal:galerkin_ortho}
        b(u - u^{\mathcal N}, v) = 0 \quad \fa v \in \mathcal Y^{\mathcal N},
    \end{equation}
    das heißt, der Fehler $u - u^{\mathcal N}$ ist orthogonal zu $\mathcal Y^{\mathcal N}$.

    \begin{Beweis}
        Ausnutzen der Bilinearität liefert für beliebiges $v \in \mathcal Y^{\mathcal N}$ die Gleichung
        \begin{equation}
            b(u - u^{\mathcal N}, v) = b(u, v) - b(u^{\mathcal N}, v) = f(v) - f(v) = 0,
        \end{equation}
        wobei die zweite Gleichheit durch $\mathcal Y^{\mathcal N} \subset \mathcal Y$ gerechtfertigt wird.
    \end{Beweis}
\end{Lemma}

\begin{Satz}[Lemma von Céa]
    \label{satz:gal:cea}
    Sei $u \in \mathcal X$ die Lösung des stetigen Variationsproblems \cref{eq:gal:abstraktes_variationsproblem} und $u^{\mathcal N} \in \mathcal X^{\mathcal N}$ die diskrete Lösung von \cref{eq:gal:pg_variationsproblem}.
    Der Fehler $u - u^{\mathcal N}$ erfüllt die Ungleichung
    \begin{equation}
        \label{eq:gal:cea}
        \norm{u - u^{\mathcal N}}_{\mathcal X} \leq \frac{\gamma^{\mathcal N}}{\beta^{\mathcal N}} \inf_{w \in \mathcal X^{\mathcal N}} \norm{u - w}_{\mathcal X}.
    \end{equation}

    \begin{Beweis}
        Siehe \cite[Theorem 3.2]{Nochetto:2009il}.
    \end{Beweis}
\end{Satz}

Anhand von \cref{satz:gal:stabilitaet} und \cref{satz:gal:cea} wird die Wichtigkeit der diskreten inf-sup-Konstante $\beta^{\mathcal N}$ deutlich.
Dies motiviert insbesondere die folgende Stabilitätseigenschaft:

\begin{Definition}[Stabile Diskretisierungen]
    Sei $\Set{(\mathcal X^{\mathcal N}, \mathcal Y^{\mathcal N})}_{\mathcal N \geq 1}$ eine Folge von endlichdimensionalen Unterräumen mit zugehörigen diskreten inf-sup-Konstanten $\Set{\beta^{\mathcal N}}_{\mathcal N \geq 1}$.
    Diese Folge von Diskretisierungen wird stabil genannt, wenn ein $\beta > 0$ mit
    \begin{equation}
        \inf_{\mathcal N \geq 1} \beta^{\mathcal N} \geq \beta > 0
    \end{equation}
    existiert.
\end{Definition}

\begin{Bemerkung}
    Die Annahme, dass die diskreten Ansatz- und Testräume die selbe Dimension $\mathcal N$ haben, ist für Petrov-Galerkin-Verfahren nicht notwendig.
    Erlaubt man $\dim \mathcal Y^{\mathcal N} \geq \dim \mathcal X^{\mathcal N} = \mathcal N$, so erfordert dies, da \cref{def:gal:disrekte_loesung} dann nicht mehr wohldefiniert ist, eine Reformulierung als Residuum-Minimierendes Petrov-Galerkin-Verfahren.

    Zwar lassen sich dafür analoge Aussagen zu denen in diesem Abschnitt herleiten, siehe beispielsweise \cite{Andreev:2012ep,Andreev:2013gk}, aber
    die Verwendung als Grundlage für die Reduzierte-Basis-Methode in der klassischen Form ist nicht möglich, da hierzu die Definition der diskreten Lösung in der Form von \cref{eq:gal:pg_variationsproblem} notwendig ist.
\end{Bemerkung}

% subsection grundlagen (end)

\section{Raum-Zeit-Diskretisierung} % (fold)
\label{sub:raum_zeit_diskretisierung}

Wir kehren nun wieder zu der Problemstellung aus \cref{chapter:propagator_differentialgleichung} zurück und wollen eine Raum-Zeit-Diskretisierung in Form eines Petrov-Galerkin-Verfahrens durchführen.
\mdo{quelle}

Dazu greifen wir auf \cref{satz:bochner_sobolev_raum_als_tensorprodukt} zurück um eine Zerlegung der verwendeten Bochnerräume in Tensorprodukte von nur raum- beziehungsweise zeitabhängigen Räumen zu erreichen.

\begin{Korollar}
    Der Ansatzraum $\mathcal X$ aus \cref{eq:gl:le:ansatzraum_X} lässt sich schreiben als
    \begin{equation}
        \label{eq:gal:ansatzraum_tensor}
        \mathcal X = L_{2}(0, T; V) \cap H^{1}(0, T; V')
            = (L_2(0, T) \otimes V) \cap (H^{1}(0, T) \otimes V'),
    \end{equation}
    für den Testraum $\mathcal Y$ aus \cref{eq:gl:le:testraum_Y} gilt weiter
    \begin{equation}
        \label{eq:gal:testraum_tensor}
        \mathcal Y = L_{2}(0, T; V) \times H = (L_{2}(0, T) \otimes V) \times H.
    \end{equation}
\end{Korollar}

An dieser Stelle wollen wir das parametrische Variationsproblem, für welches wir die Diskretisierung bestimmen wollen, noch einmal kurz zusammenfassen.

\begin{Bemerkung}
    \mdo{Zusammenfassen und so!}
\end{Bemerkung}

Nach diesen kurzen Wiederholungen widmen wir uns nun der Konstruktion der diskreten Räume.
Dabei orientieren wir uns an den Ausführungen von \textcite{Andreev:2012ep}, woraus auch die nachfolgenden Stabilitätsaussagen zu dieser Diskretisierung entnommen werden können.
Die diskreten Äquivalente zu $\mathcal X$ und $\mathcal Y$ werden die Form
\begin{equation}
    \label{eq:gal:diskrete_tensor_raueme}
    \mathcal X^{\mathcal N} = E^{\mathcal K} \otimes V^{\mathcal J}, \qquad \mathcal Y^{\mathcal N} = (F^{\mathcal K} \otimes V^{\mathcal J}) \times V^{\mathcal J}
\end{equation}
haben, wobei mit $E^{\mathcal K}$, $F^{\mathcal K}$ die zeitlichen Unterräume und mit $V^{\mathcal J}$ der räumliche Unterraum bezeichnet werden.
Da wir Ansatz- und Testräume gleicher Dimension $\mathcal N$ konstruieren wollen, müssen die Dimensionen der einzelnen vorkommen Komponentenräume entsprechend gewählt werden.
Eine von vielen denkbaren Varianten, insbesondere die, welche wir verwenden werden, ist,
\begin{equation}
    \dim E^{\mathcal K} = \mathcal K + 1, \qquad \dim F^{\mathcal K} = \mathcal K, \qquad \dim V^{\mathcal J} = \mathcal J.
\end{equation}
Damit ergibt sich insgesamt, wie gewünscht, die Gleichheit
\begin{equation}
    \mathcal N = \dim \mathcal X^{\mathcal N} = (\mathcal K + 1) \mathcal J = \mathcal K \mathcal J + \mathcal J = \dim \mathcal Y^{\mathcal N}.
\end{equation}

\paragraph{Zeitliche Komponente} % (fold)
\label{par:zeitabh_ngige_unterr_ume}

Zunächst wollen wir uns mit den rein zeitabhängigen Räumen befassen.
Hierfür benötigen wir eine Diskretisierung des Zeitintervalls $I = [0, T]$, welche durch ein im Allgemeinen nicht äquidistantes Gitter der Form
\begin{equation}
    \mathcal T^{\mathcal K}_{t} = \Set{0 = t_0 < t_1 < \dots < t_{\mathcal K - 1} < t_{\mathcal K} = T} \subset I
\end{equation}
gegeben sei.

Als Basis für den zeitlichen Anteil $E^{\mathcal K}$ des Ansatzraumes verwenden wir stetige, stückweise affine Funktionen, konkret die klassischen Hutfunktionen $\tau_{k}$ auf den Gitterpunkten $t_{k}$, $k = 0, \dots, \mathcal K$, von $\mathcal T^{\mathcal K}_{t}$, das heißt, es gilt $\tau_{k}(t_{\tilde k}) = \delta_{k \tilde k}$, wobei $\delta_{k \tilde k}$ das bekannte Kronecker-Delta sei.
Kurz gefasst setzen wir also
\begin{equation}
    E^{\mathcal K} = \spn\Set{\tau_{k} \given k = 0, \dots, \mathcal K}.
\end{equation}

Für den Testraum-Anteil $F^{\mathcal K}$ verwenden wir stattdessen stückweise konstante Funktionen $\xi_{k} = \chi_{(t_{k-1}, t_{k})}$ auf den Teilintervallen $(t_{k-1}, t_{k})$ mit den Gitterpunkten aus $\mathcal T^{\mathcal K}_{t}$, also
\begin{equation}
    F^{\mathcal K} = \spn\Set{\xi_{k} \given k = 1, \dots, \mathcal K}.
\end{equation}

Diese Wahl führt, wie man beispielsweise \cite{Andreev:2013gk} entnehmen kann, auf ein Crank-Nicolson-ähnliches Verfahren, welches auch als Time-Stepping-Verfahren interpretiert werden kann.
Insbesondere erfüllt es $\dim E^{\mathcal K} = \dim F^{\mathcal K} + 1$.

\paragraph{Räumliche Komponente} % (fold)
\label{par:r_umliche_unterraeume}

Auf die Diskretisierung der räumlichen Komponente des Variationsproblems so wie es in \cref{bla} gegeben ist, wollen wir an dieser Stelle nur kurz eingehen, da wir uns im nachfolgenden Abschnitt zu Gunsten einer realisierbaren Implementierung auf den Fall einer Raumdimension beschränken werden.

Im Allgemeinen stehen einem viele Möglichkeiten zur Diskretisierung des räumlichen Anteils $V^{\mathcal J}$ zur Verfügung.
So bietet sich ein Finite-Elemente-Ansatz, wie er beispielsweise auch bei der zeitlichen Komponente verwendet wird, an.
Wir werden im Weiteren Verlauf auf globale Basisfunktionen, welche eine orthogonale Basis bilden, zurückgreifen.

Der Einfachheit halber nehmen wir an, dass der räumliche Anteil durch eine Basis $\Sigma$ aufgespannt wird, folglich also
\begin{equation}
    V^{\mathcal J} = \spn \Sigma = \spn\Set{\sigma_{j} \given j = 1, \dots, \mathcal J}
\end{equation}
sei.

\mdo{mehr?}

% paragraph r_umliche_unterraeume (end)

\subsection{Stabilität der Diskretisierung} % (fold)
\label{sub:stabilit_t_der_diskretisierung}

% subsection stabilit_t_der_diskretisierung (end)

\paragraph{Raum-Zeit-Diskretisierung} % (fold)
\label{par:raum_zeit_diskretisierung}

Wir kombinieren nun die zuvor eingeführten rein zeitlichen beziehungsweise räumlichen Basen unter Verwendung der Tensor-Produkt-Darstellung \cref{eq:gal:diskrete_tensor_raueme} zu Raum-Zeit-Basen.
Dazu setzen wir
\begin{equation}
    \Phi = \Set{\theta \otimes \sigma \given \theta \in \Theta, \sigma \in \Sigma}, \qquad \Psi_{1} = \Set{\xi \otimes \sigma \given \xi \in \Xi, \sigma \in \Sigma}
\end{equation}
und weiter
\begin{equation}
    \Psi = (\Psi_{1} \times \Set{ 0 }) \cup (\Set{ 0 } \times \Sigma).
\end{equation}
Es gilt offenbar $\Phi \subset \mathcal X^{\mathcal N}$ sowie $\Psi \subset \mathcal Y^{\mathcal N}$ und weiter bilden $\Phi$ und $\Psi$ eine Basis des diskreten Ansatzraums $\mathcal X^{\mathcal N}$ respektive Testraums $\mathcal Y^{\mathcal N}$.

\mdo{quelle?}

% paragraph raum_zeit_diskretisierung (end)


% paragraph zeitabh_ngige_unterr_ume (end)

% subsection raum_zeit_diskretisierung (end)

\section{Numerische Umsetzung} % (fold)
\label{sub:grb:gv:numerische_umsetzung}

\subsection{Sammlung zur numerischen Umsetzung} % (fold)
\label{sub:sammlung_zur_numerischen_umsetzung}

\begin{Lemma}[Stetige Fortsetzung der dualen Paarung]
\label{lemma:stetige_fortstetung_der_dualen_paarung}
    Sei $(V, H, V')$ ein Gelfand-Tripel.
    Sei weiter $v \in H$, dann gilt $\skp{v}{w}{V' \times V} = \skp{v}{w}{H}$ für alle $w \in V$.
\end{Lemma}

\begin{Lemma}[Berechnung der Rieszchen Darstellung]
\label{lemma:berechnung_rieszsche_darstellung}
    Sei $X$ ein endlichdimensionaler Hilbertraum mit Basis ${\phi_i}_{i=1}^{N}$.
    Sei weiter $g \in X'$.
    Der Koeffizientenvektor $\vec{v} \in \mathbb{R}^{N}$ der Rieszschen Darstellung $v_g = \sum_{i=1}^{N} v_{i} \phi_{i} \in X$ von $g$, das heißt, es gilt $\skp{v_g}{w}{X} = \skp{g}{w}{X' \times X}$ für alle $w \in X$, lässt sich durch das Gleichungssystem $\mat{K}\vec{v} = \vec{g}$ berechnen, wobei $\mat{K} = [\skp{\phi_{i}}{\phi_{j}}{X}]_{i,j}$ die Massematrix zum inneren Produkt auf $X$ sei und weiter $\vec{g} = [\skp{g}{\phi_{i}}{X' \times X}]_{i}$ sei.
\end{Lemma}

\begin{Lemma}[Diskrete $\mathcal{X}$-Norm]
% todo: fixme
    Sei $\mathcal{X}^{\mathcal{N}} = \spn\Set{ \phi_{i} \given i = 1 \dots \mathcal N } \subset \mathcal X$ ein endlichdimensionaler Unterraum.
    Die $\mathcal X$-Norm aus \eqref{eq:gl:le:ansatzraum_X_norm}
    wird für den Unterraum $\mathcal{X}^{\mathcal N}$ durch den folgenden Operator
    \begin{equation}
        \mat{M} = \mat{M}_{t} \otimes (\mat{M}_{x} + \mat{A}_{x}) + \mat{A}_{t} \otimes (\mat{M}_{x} (\mat{M}_{x} + \mat{A}_{x})^{-1} \mat{M}_{x})
    \end{equation}
    induziert.
\end{Lemma}

\begin{Lemma}[Diskrete $\mathcal{Y}$-Norm]
% todo: fixme
    Sei bla.
    Die $\mathcal{Y}^{\mathcal N}$-Norm aus \eqref{eq:gl:le:testraum_Y_norm} wird durch den folgenden Operator
    \begin{equation}
        \mat{N} = \begin{pmatrix}
            \mat{M}_{t} \otimes (\mat{M}_{x} + \mat{A}_{x}) & \mat{0} \\
            \mat{0} & \mat{M}_{x}.
        \end{pmatrix}
    \end{equation}
    induziert.
\end{Lemma}

% subsubsection sammlung_zur_numerischen_umsetzung (end)

% subsection numerische_umsetzung (end)
