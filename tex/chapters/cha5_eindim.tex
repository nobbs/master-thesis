% -*- root: ../main.tex -*-

\iftoggle{dictum}{
    \setchapterpreamble[ul][0.6\textwidth]{%
        \dictum[Robert Heinlein, \textit{Time Enough For Love}]{\enquote{Progress isn't made by early risers. It's made by lazy men trying to find easier ways to do something.}}
        \vspace*{2\baselineskip}
    }
}{}
\chapter{Der eindimensionale Fall}
\label{sec:der_eindimensionale_fall}
\label{cha:der_eindimensionale_fall}

\todo[inline]{Kapitel ordentlich überarbeiten. Mittlerweile besser, aber noch deutlich verbesserungswürdig!}
\todo[inline]{Sind ein paar notationelle Kleinigkeiten zu korrigieren, also mal ordentlich durchlesen!}
\todo[inline]{Die Abschätzung in Satz 4.4. ist schon heftig. Da bleibt ja so gut wie kein Spielraum für den Faktor K...}

In diesem Kapitel beschränken wir uns zunächst auf den vereinfachten Fall einer Raumdimension.
Es sei also $\Omega \subset \mathbb{R}$ ein Intervall.
Ohne Beschränkung der Allgemeinheit wählen wir $\Omega = [0, L]$ für ein $0 < L < \infty$.

\todo[inline]{Eventuell wäre, insbesondere mit der gewünschten Anwendung bei SCFT-Verfahren, eine Entwicklung in Cosinus-Funktionen (oder direkt Fourier-Reihen) sinnvoller.
Insbesondere wären Cosinus-Funktionen achsensymmetrisch und automatisch nicht-homogen; beides Eigenschaften, welche die Felder $\omega$ bei der SCFT oftmals aufweisen.}


Als Ansatzfunktionen für eine geeignete Entwicklung des Operators $A$ in eine affin parametrische Darstellung wählen wir Sinusfunktionen, welche zusätzlich gewichtet werden, so dass wir die gewünschten Konvergenzeigenschaften erhalten.
Zusätzlich zu den Sinusfunktionen fügen wir eine konstante Funktion $\varphi_{0}$ hinzu, um so inhomogene Randbedingungen für $\omega$ zuzulassen.

Da die Parameter $\sigma$ weiterhin aus der Menge $\mathcal S = [-1, 1]^{\mathbb{N}}$ kommen, skalieren wir die Ansatzfunktionen mit einem Faktor $K \in \mathbb{R}_{+}$.
Dieser Faktor wird später durch die Anforderungen, die wir durch die gewünschte analytische Abhängigkeit vom Parameter erhalten, genauer bestimmt.

Konkret kommt nun
\begin{equation}
    \label{eq:sinusfunktionen_ansatz}
    \varphi_{0} = K, \qquad
    \varphi_{j} = \frac{K}{(\pi j)^{1 + \epsilon}} \sin(\tfrac{\pi j}{L} \blank), \quad j \geq 1,
\end{equation}
als Funktionensystem $\Set{ \varphi_{j} }_{j \geq 0}$ zum Einsatz.
Damit schreiben wir den parametrischen Faktor $\omega$ der linearen Evolutionsgleichung aus dem vorherigen Kapitel als
\begin{equation}
    w(\blank; \sigma) \colon \Omega \to \mathbb{R}, \quad w(x; \sigma) = \sum_{j = 0}^{\infty} \sigma_{j} \varphi_{j}(x)
\end{equation}
mit $\sigma \in \mathcal S$.
Damit schreibt sich die affine Zerlegung des Differentialoperators $A = - c \Delta + \omega$ als
\begin{equation}
    A = \hat A + \sum_{j = 0}^{\infty} \sigma_{j} A_{j}
\end{equation}
mit
\begin{equation}
    \label{eq:1d:affine_zerlegung}
    \hat A = - c \Delta, \qquad A_{j} = \varphi_{j}
\end{equation}
und den zugehörigen Bilinearformen
\begin{equation}
    \hat a(\eta, \zeta) = c \skprod{\grad \eta}{\grad \zeta}_{L_{2}(\Omega)}, \qquad a_{j}(\eta, \zeta) = \skprod{\varphi_{j} \eta}{\zeta}_{L_{2}(\Omega)}.
\end{equation}


Zunächst rechnen wir nun verschiedene, später benötigte, Normen nach.
\begin{Lemma}
    Es gilt
    \begin{alignat}{2}
        \norm{\varphi_{0}}_{L_{\infty}(\Omega)} &= K,
        \qquad&
        \norm{\varphi_{j}}_{L_{\infty}(\Omega)} &= \frac{K}{(\pi j)^{1 + \epsilon}} , \quad j \geq 1,
    \intertext{sowie}
        \norm{\varphi_{0}}_{H^{1}(\Omega)}  &= K \sqrt{L},
        \qquad&
        \norm{\varphi_{j}}_{H^{1}(\Omega)}  &= \frac{K \sqrt{L^{2} + (\pi j)^{2}}}{\sqrt{2L} (\pi j)^{1 + \epsilon}}
        , \quad j \geq 1.
    \end{alignat}
\end{Lemma}

\begin{Lemma}
    Die Funktionen $\Set{ \varphi_{j} }_{j \geq 1}$ bilden ein Orthogonalsystem in $H^{1}(\Omega)$, denn es gilt
    \begin{equation}
        \skprod{\varphi_{j}}{\varphi_{k}}_{H^{1}(\Omega)} = \begin{cases}
            \frac{K^{2}}{2L} \frac{L^{2} + (\pi j)^{2}}{(\pi j)^{2(1 + \epsilon)}}
            ,   &j = k \\
            0,          &j \neq k.
        \end{cases}
    \end{equation}
\end{Lemma}

\begin{Lemma}
    Sei $\sigma \in \mathcal S$ und $\epsilon > 0$, dann konvergiert obiges $\omega(\blank; \sigma)$ in $L_{\infty}(\Omega)$.
    Ist $\epsilon > 1$, dann gilt auch Konvergenz in $H^{1}_{0}(\Omega)$.

    \begin{Beweis}
        Sei zunächst $\epsilon > 0$.
        Da $\mathcal S = [0, 1]^{\mathbb{N}}$ ist, erhalten wir die Konvergenz in $L_{\infty}(\Omega)$ nach dem Weierstraßschen Majorantenkriterium via
        \begin{align}
            \sum_{j = 0}^{\infty} \norm{\sigma_{j} \varphi_{j}}_{L_{\infty}(\Omega)}
            &= \sum_{j = 0}^{\infty} \abs{\sigma_{j}} \norm{\varphi_{j}}_{L_{\infty}(\Omega)}
             \leq \norm{\varphi_{0}}_{L_{\infty}(\Omega)} + \sum_{j = 1}^{\infty}  \norm{\varphi_{j}}_{L_{\infty}(\Omega)}
            \\&= K + \sum_{j = 1}^{\infty} \frac{K}{(\pi j)^{1 + \epsilon}}
            = K + \frac{K}{\pi^{1 + \epsilon}} \sum_{j = 1}^{\infty} \frac{1}{j^{1+\epsilon}}
        \end{align}
        Diese Reihe konvergiert bekanntlich für alle $\epsilon > 0$, womit wir bereits die Konvergenz von $\omega$ in $L_{\infty}(\Omega)$ erhalten.

        Sei nun $\epsilon > 1$.
        Betrachte
        \begin{align}
            \sum_{j = 0}^{\infty} \norm{\sigma_{j} \varphi_{j}}_{H^{1}(\Omega)}
            &= \sum_{j = 0}^{\infty} \abs{\sigma_{j}} \norm{\varphi_{j}}_{H^{1}(\Omega)}
            \leq  \norm{\varphi_{0}}_{H^{1}(\Omega)} + \sum_{j = 1}^{\infty} \norm{\varphi_{j}}_{H^{1}(\Omega)}
            \\&= K \sqrt{L} + \sum_{j = 1}^{\infty} \frac{K \sqrt{L^{2} + (\pi j)^{2}}}{\sqrt{2L} (\pi j)^{1 + \epsilon}}
            \\&= K \sqrt{L} + \frac{K}{\sqrt{2L}} \sum_{j = 1}^{\infty} \frac{\sqrt{L^{2} + (\pi j)^{2}}}{(\pi j)^{1 + \epsilon}}
            \\&\leq K \sqrt{L} + \frac{K}{\sqrt{2L}} \sum_{j = 1}^{\infty} \frac{L + \pi j}{(\pi j)^{1 + \epsilon}}
            \\&= K \sqrt{L} + \frac{K}{\sqrt{2L}} \sum_{j = 1}^{\infty} \frac{L}{(\pi j)^{1 + \epsilon}} + \frac{K}{\sqrt{2L}} \sum_{j = 1}^{\infty} \frac{1}{(\pi j)^{\epsilon}}
        \end{align}
        Wegen $\epsilon > 1$ konvergiert sowohl die erste als auch die zweite Reihe.
        Zusammen liefert dies die Konvergenz in $H^{1}_{0}(\Omega)$.
    \end{Beweis}
\end{Lemma}

Wir wollen nun die Regularität von $\omega(\blank; \sigma)$ in Abhängigkeit vom Parameter $\sigma$ nachweisen.

\todo[inline]{Anpassen! Lässt sich die Schranke verbessern?}
\begin{Satz}
\label{satz:regularitaet_nachrechnen}
    Seien $\epsilon > 0$ und $0 < \kappa < 1$ so gewählt, dass
    \begin{equation}
        \sum_{j = 1}^{\infty} \frac{1}{j^{2 + \epsilon}} \leq \frac{(\kappa c (\tfrac{\pi}{L})^{2} - K) \pi^{2 + \epsilon}}{4 C_{\infty} K L^{3/2}}
    \end{equation}
    gilt,
    wobei $c$ und $K$ die Konstanten aus \eqref{eq:def_op_A} respektive \eqref{eq:sinusfunktionen_ansatz} und $C_{\infty}$ die Einbettungskonstante von $H^{1}_{0}(\Omega) \hookrightarrow L_{\infty}(\Omega)$ sind.
    Dann erfüllt die affine Zerlegung $\Set{\hat A, A_{j} \given j \in \mathbb{N}_{0}}$ \thref{thm:kunoth:assumption2}.

    \begin{Beweis}
        Wir weisen zunächst die inf-sup-Bedingungen \eqref{eq:kunoth:ass2_gamma_0} für $\hat a(\blank, \blank)$ nach und bestimmen die Konstante $\gamma_{0}$.
        Da $\hat a(\blank, \blank)$ symmetrisch ist, genügt es, die inf-sup-Bedingung \eqref{eq:kunoth:ass2_gamma_0_a} nachzuweisen. Die zweite inf-sup-Bedingung \eqref{eq:kunoth:ass2_gamma_0_b} folgt dann analog mit dem selben $\gamma_{0}$.

        Nach \thref{lem:sauter:2.1.48} reicht es, für alle $\eta \in H^{1}_{0}(\Omega)$ ein $\zeta = \zeta(\eta) \in H^{1}_{0}(\Omega)$ und von $\eta$ und $\zeta$ unabhängige Konstanten $C_{1}, C_{2} > 0$ mit
        \begin{equation}
            \hat a(\eta, \zeta) \geq C_{1} \norm{\eta}_{H^{1}(\Omega)}^{2} \quad \text{und} \quad \norm{\zeta}_{H^{1}(\Omega)} \leq C_{2} \norm{\eta}_{H^{1}(\Omega)}
        \end{equation}
        zu finden.
        Dann ist die inf-sup-Bedingung \eqref{eq:kunoth:ass2_gamma_0_a} mit $\gamma_{0} = \frac{C_{1}}{C_{2}}$ erfüllt.

        Sei nun also $\eta \in H^{1}_{0}(\Omega)$ beliebig.
        Wir wählen $\zeta = \eta \in H^{1}_{0}(\Omega)$, das heißt, es gilt $C_{2} = 1$.
        Es ergibt sich
        \begin{align}
            \hat a(\eta, \zeta) = \hat a(\eta, \eta) = c \skprod{\grad \eta}{\grad \eta}_{L_{2}(\Omega)} = c \norm{\grad \eta}_{L_{2}(\Omega)}^{2} \geq c \gamma_{\Omega}^{2} \norm{\eta}_{H^{1}(\Omega)}^{2},
        \end{align}
        wobei die letzte Abschätzung aus der Poincaré-Friedrichs-Ungleichung \eqref{eq:gl:poincare_friedrichs_ungleichung} folgt.
        Zusammen liefert dies $\gamma_{0} = c \gamma_{\Omega}^{2}$ als inf-sup-Konstante.

        Für den vorliegenden Fall können wir $\gamma_{\Omega}^{2}$ exakt bestimmen.
        Nach \cite[Chapter 11]{Strauss:2007vz} entspricht das Quadrat der optimalen Poincaré-Friedrichs-Konstante $\gamma_{\Omega}^{2}$ gerade dem kleinsten Eigenwert des Laplace-Operators auf $\Omega$ mit Dirichlet-Randbedingung.
        Dieser hat für $\Omega = [0, L]$ den Wert $\frac{\pi^{2}}{L^{2}}$.
        Wir erhalten damit also $\gamma_{0} = c \frac{\pi^{2}}{L^{2}}$.

        Seien nun $\eta, \zeta \in H^{1}_{0}(\Omega)$.
        Für $j = 0$ gilt die simple Abschätzung
        \begin{equation}
            \begin{aligned}
                a_{0}(\eta, \zeta)
                &= \skprod{\varphi_{0} \eta}{\zeta}_{L_{2}(\Omega)}
                = K \skprod{\eta}{\zeta}_{L_{2}(\Omega)}
                \\&\leq K \norm{\eta}_{L_{2}(\Omega)} \norm{\zeta}_{L_{2}(\Omega)}
                \leq K \norm{\eta}_{H^{1}(\Omega)} \norm{\zeta}_{H^{1}(\Omega)}
            \end{aligned}
        \end{equation}
        Betrachte für $j \geq 1$
        \begin{align}
            a_{j}(\eta, \zeta)
            &= \skprod{\varphi_{j} \eta}{\zeta}_{L_{2}(\Omega)}
            = \int_{0}^{L} \varphi_{j} \eta \zeta \diff x
            \intertext{da $\varphi_{j}$ integrierbar ist und $\varphi_{j}(0) = 0$, können wir dies umschreiben zu}
            a_{j}(\eta, \zeta)
            &= \int_{0}^{L} \frac{\diff}{\diff x} \left( \int_{0}^{x} \varphi_{j}(y) \diff y \right) \eta \zeta \diff x
            \intertext{woraus wir mittels partieller Integration und $\eta, \zeta \in H^{1}_{0}(\Omega)$ folgenden Ausdruck erhalten}
            a_{j}(\eta, \zeta)
            &= - \int_{0}^{L} \left( \int_{0}^{x} \varphi_{j}(y) \diff y \right) (\eta \zeta)' \diff x
            \leq \norm*{\left( \int_{0}^{x} \varphi_{j}(y) \diff y \right) (\eta \zeta)'}_{L_{1}(\Omega)}
            \\&\leq \norm*{\int_{0}^{x} \varphi_{j}(y) \diff y }_{L_{\infty}(\Omega)} \norm{(\eta \zeta)'}_{L_{1}(\Omega)}.
        \end{align}
        Die erste Norm können wir weiter abschätzen mit
        \begin{equation}
            \begin{aligned}
                \norm*{\int_{0}^{x} \varphi_{j}(y) \diff y }_{L_{\infty}(\Omega)}
                &= \norm*{\int_{0}^{x} \frac{K}{(\pi j)^{1 + \epsilon}} \sin(\tfrac{\pi j}{L} y) \diff y}_{L_{\infty}(\Omega)}
                \\&= \norm*{\frac{K L}{(\pi j)^{2 + \epsilon}} \left( 1 - \cos(\tfrac{\pi j}{L} x) \right) }_{L_{\infty}(\Omega)}
                \leq \frac{2 K L}{(\pi j)^{2 + \epsilon}}
            \end{aligned}
        \end{equation}
        Aus der zweiten Norm erhalten wir mittels Minkowski- und Hölderungleichung sowie der Einbettung $H^{1}_{0}(\Omega) \hookrightarrow L_{\infty}(\Omega)$ die Abschätzung
        \begin{equation}
            \begin{aligned}
                \norm{(\eta \zeta)'}_{L_{1}(\Omega)}
                &= \norm{\eta' \zeta + \eta \zeta'}_{L_{1}(\Omega)}
                \leq \norm{\eta' \zeta}_{L_{1}(\Omega)} + \norm{\eta \zeta'}_{L_{1}(\Omega)}
                \\&\leq \norm{\eta'}_{L_{1}(\Omega)} \norm{\zeta}_{L_{\infty}(\Omega)} + \norm{\eta}_{L_{\infty}(\Omega)} \norm{\zeta'}_{L_{1}(\Omega)}
                \\&\leq \norm{1}_{L_{2}(\Omega)} \norm{\eta'}_{L_{2}(\Omega)} \norm{\zeta}_{L_{\infty}(\Omega)} + \norm{\eta}_{L_{\infty}(\Omega)} \norm{1}_{L_{2}(\Omega)} \norm{\zeta'}_{L_{2}(\Omega)}
                \\&\leq \sqrt{L} C_{\infty} \norm{\eta'}_{L_{2}(\Omega)} \norm{\zeta}_{H^{1}(\Omega)} + \sqrt{L} C_{\infty} \norm{\eta}_{H^{1}(\Omega)} \norm{\zeta'}_{L_{2}(\Omega)}
                % \\&\leq \norm{\eta'}_{L_{2}(\Omega)} \norm{\zeta}_{L_{2}(\Omega)} + \norm{\eta}_{L_{2}(\Omega)} \norm{\zeta'}_{L_{2}(\Omega)}
                \\&\leq 2 \sqrt{L} C_{\infty} \norm{\eta}_{H^{1}(\Omega)} \norm{\zeta}_{H^{1}(\Omega)}
            \end{aligned}
        \end{equation}
        Zusammen also
        \begin{align}
            a_{j}(u, v)
            &\leq \norm*{\int_{0}^{x} \varphi_{j}(y) \diff y }_{L_{\infty}(\Omega)} \norm{(\eta \zeta)'}_{L_{1}(\Omega)}
            \\&\leq \frac{4 K L^{3 / 2} C_{\infty}}{(\pi j)^{2 + \epsilon}} \norm{\eta}_{H^{1}(\Omega)} \norm{\zeta}_{H^{1}(\Omega)}.
        \end{align}
        Betrachte nun
        \begin{align}
                    \sum_{j = 0}^{\infty} \norm{A_{j}}_{\mathcal L(V, V')}
            &= \norm{A_{0}}_{\mathcal L(V, V')} + \sum_{j = 1}^{\infty} \norm{A_{j}}_{\mathcal L(V, V')}
            \\&\leq K + \sum_{j = 1}^{\infty} \frac{4 K L^{3 / 2} C_{\infty}}{(\pi j)^{2 + \epsilon}}
            \\&\leq K + \frac{4 K L^{3 / 2} C_{\infty}}{\pi^{2+ \epsilon}} \sum_{j = 1}^{\infty} \frac{1}{j^{2 + \epsilon}}
        \end{align}
        Fordern wir nun die Gültigkeit von \eqref{eq:kunoth:ass2_abs_reihe}, also
        \begin{equation}
            \sum_{j \geq 0} \norm{A_{j}}_{\mathcal L(V, V')} \leq \kappa \gamma_{0}
        \end{equation}
        für ein $0 < \kappa < 1$, dann ist damit also
        \begin{equation}
            \sum_{j = 1}^{\infty} \frac{1}{j^{2 + \epsilon}} \leq \frac{(\kappa (\tfrac{\pi}{L})^{2} - K) \pi^{2+ \epsilon}}{4 K L^{3/2} C_{\infty}}
        \end{equation}
        mit $\epsilon > 0$ hinreichend.
    \end{Beweis}
\end{Satz}

Zusammenfassend erhalten wir damit die folgende Aussage.

\todo[inline]{Besser ausformulieren}
\begin{Satz}
    Seien $\mathcal X$ und $\mathcal Y$ gegeben wie in~\eqref{eq:ps:rzvp:ansatzraum_testraum}.
    Weiter sei $\Set{\hat A, A_{j} \given j \in \mathbb{N}_{0}}$ die affine Zerlegung von $A = -c \Delta + \omega$ wie in \eqref{eq:1d:affine_zerlegung} und es gelte \thref{satz:regularitaet_nachrechnen}.
    Für jedes $\sigma \in \mathcal S$ sei $B(\sigma) \in \mathcal L(\mathcal X, \mathcal Y')$ definiert durch
    \begin{equation}
        \label{eq:ps:rg:theorem21_variationsproblem_als_operatorgleichung_parametrisch}
        \skprod{B(\sigma) u}{v}_{\mathcal Y' \times \mathcal Y} = b(u, v; \sigma), \quad u \in \mathcal X,~y \in \mathcal Y,
    \end{equation}
    mit $b(\blank, \blank; \sigma)$ wie in~\eqref{eq:ps:rzvp:schwache_formulierung_lhs_b_2}.
    Dann ist $B(\sigma)$ für jedes $\sigma \in \mathcal S$ stetig invertierbar und die parametrische Familie von Lösungen $u(\sigma)$ des parametrischen Raum-Zeit-Variationsproblems \eqref{eq:ps:rzvp:schwache_formulierung_2} hängt analytisch von $\sigma$ ab.

    \begin{Beweis}
        Direkte Folgerung aus \thref{satz:regularitaet_nachrechnen} und \thref{satz:ps:rg:kunoth13_theorem21}.
    \end{Beweis}
\end{Satz}

% subsection nachrechnen_von_thref_thm_kunoth_assumption2 (end)

% section der_eindimensionale_fall (end)

\clearpage
\section{Zu klärende Fragen} % (fold)
\label{sub:zu_kl_rende_fragen}

\begin{enumerate}
    \item Wohldefiniertheit der PDE \eqref{eq:parabolische_pde}, das heißt die Voraussetzungen von \thref{satz:gl:le:ss09_theorem51} nachweisen. Weiterhin lassen sich damit die inf-sup-Bedingung von \eqref{eq:ps:rzvp:schwache_formulierung} nachrechnen und damit die Schranken für $B$ und $B^{-1}$ bestimmen.
    \item Parametrische Variante des Variationsproblems herleiten.
    Dazu Ansetzen mit Entwicklung des Parameters $\omega$ in eine Reihe
    \begin{equation}
        \omega = \sum_{j = 0}^{\infty} \sigma_{j} \varphi_{j}.
    \end{equation}
    Dabei ergeben sich folgende Fragen:
    \begin{enumerate}
        \item Konvergenz der Reihe? Notwendig ist Konvergenz in $L_{\infty}(\Omega)$, da die Norm $\norm{\omega}_{L_{\infty}(\Omega)}$ mehrfach in Abschätzungen verwendet wird.
        \item Weiterhin ist eventuell Konvergenz in einem Unterraum $Z \hookrightarrow L_{\infty}(\Omega)$ wünschenswert.
        Zum Beispiel in $H^{1}(\Omega)$?
        \item Welche Bedingungen ergeben sich an $\sigma_{j}$ und $\varphi_{j}$?
        \item Welches Funktionensystem $\Set{ \varphi_{j} }_{j}$ ist überhaupt sinnvoll?
        Die Wahl der $\varphi_{j}$ entscheidet maßgeblich über Konvergenz der Reihenentwicklung.
        Welche Randvorgaben sind angestrebt?
        Dies wird ebenfalls durch die $\varphi_{j}$ geregelt.
    \end{enumerate}
    \item Welche affine Zerlegung $A(\sigma) = A_{0} + \sum_{j} \sigma_{j} A_{j}$ ist brauchbar?
    Wie genau sehen die $A_{j}$ aus?
    \item Nachweisen, dass $A(\sigma)$ \thref{ann:ps:rg:kunoth13_assumption1} oder \thref{thm:kunoth:assumption2} erfüllt und mittels \thref{satz:ps:rg:kunoth13_theorem21} die gewünschte Regularität von $B(\sigma)$ bezüglich $\sigma$ gewinnen.
    \item Die Abschätzungen in \thref{satz:regularitaet_nachrechnen} lassen sich noch deutlich verbessern.
    Das gilt wahrscheinlich auch für andere Abschätzungen!
\end{enumerate}

