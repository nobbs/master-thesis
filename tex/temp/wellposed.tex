%!TEX root = ../main.tex

\chapter{Einmal bunt gemischt}

In diesem Kapitel führen wir die in \cite{Stasiak:2011ba} motivierte parabolische partielle Differentialgleichung ein und betrachten danach eine parametrisierte Variante dieser.
Anschließend nutzen wir die Aussagen aus Kapitel \ref{cha:einfuehrung} um die Eigenschaften dieses Problems zu diskutieren.

\section*{Zu klärende Fragen} % (fold)
\label{sub:zu_kl_rende_fragen}

\begin{enumerate}
    \item Well-posedness von
    \begin{equation}
        u_{t} = \Delta u - \omega u + f, \quad u(0) = u_{0}
    \end{equation}
    für $\omega \in L_{\infty}(\Omega)$, $\Omega$ beschränkt mit Lipschitz-Rand.
    Genauer: \cite[Theorem 5.1]{Schwab:2009ec}  nachweisen für obiges Problem, d.h.
    \begin{itemize}
        \item inf-sup-Bedingungen und so weiter nachrechnen \todo{noch offen \dots}
        \item Konstanten (in Abhängigkeit von $\omega$) bestimmen \todo{noch offen \dots}
    \end{itemize}
    \item \todo{Alles noch offen \dots}Ab hier Bezug auf \cite{Kunoth:2013ef}.
        Entwicklung des Parameters
    \begin{equation}
        \omega = \sum_{j = 0}^{\infty} \sigma_{j} \varphi_{j},
    \end{equation}
    Zu klären:
    \begin{itemize}
        \item Konvergenz? In welchem Raum? Zum Beispiel $Z \hookrightarrow L_{\infty}(\Omega)$, $Z = H^{1}(\Omega)$.
        \item Bedingungen an $\sigma_{j}$? Zum Beispiel $\abs{\sigma_{j}} \leq 1$.
        \item Welche $\varphi_{j}$? Möglich mit $\varphi_{j} = \sin(j \pi \cdot)$ oder $\sin(j \pi \cdot) / j \pi$ oder $\sin(j \pi \cdot) / (j \pi)^{1 + \epsilon}$? Je nach verwendetem Raum anders normalisiert etc.
        \item Randwerte? Homogene machbar mit obigen $\sin$-Funktionen und $\sigma_{0} = 0$. Periodische mit passender Wahl von $\sigma_{0}$. Sonst?
    \end{itemize}
    \item Wie sieht der Operator $G$ aus und wie sieht eine brauchbare Zerlegung in $G = G_{0} + \sum_{j} \sigma_{j} G_{j}$ aus?
    \item Nachweisen, dass $G_{j}$ beschränkt bzw. die Konvergenz erfüllt.
    \item Ist \cite[Assumption 2]{Kunoth:2013ef} erfüllt?
\end{enumerate}



\section{Problemstellung} % (fold)
\label{sub:problemstellung}

Es sei $0 < T < \infty$ und $I = [0, T]$.
Wir bezeichnen mit $\Omega$ eine beschränkte Teilmenge des $\mathbb{R}^{n}$ mit Lipschitz-Rand.

Wir setzen $V = H^{1}_{0}(\Omega)$ und $H = L_{2}(\Omega)$ und verwenden diese Abkürzungen im Folgenden der notationellen Einfachheit halber.
Bekanntlich handelt es sich dabei um separable Hilberträume und es existiert eine dichte stetige Einbettung $V \hookrightarrow H$.
Durch Identifikation von $H$ mit seinem Dualraum $H'$ erhalten wir das Gelfand-Tripel $V \hookrightarrow H \hookrightarrow V'$, wobei $V' = (H^{1}_{0}(\Omega))' = H^{-1}(\Omega)$.
Wie zuvor verwenden wir $\skprod{\cdot}{\cdot}$ mit entsprechendem Index sowohl für die inneren Produkte als auch für die duale Paarung.

Es seien $c \in \mathbb{R}_{+}$ und $\omega \in L_{\infty}(\Omega)$ gegeben, dann definieren wir einen linearen Operator
\begin{equation}
    \label{eq:def_op_A}
    A \colon V \to V', \quad \eta \mapsto A \eta = - c \Delta \eta + \omega \eta,
\end{equation}
zudem sei ein $g \in L_{2}(I; V')$ und ein $u_{0} \in H$ geben.
Wir interessieren uns nun für Lösungen der parabolischen partiellen Differentialgleichung \eqref{eq:allgemeine_parabolische_pde} mit $A$ aus \eqref{eq:def_op_A}, das heißt
\begin{equation}
    \label{eq:parabolische_pde}
    u_{t}(t) - c \Delta u(t) + \omega u(t) = g(t) \quad \text{in}~V',
    \qquad
    u(0) = u_{0} \quad \text{in}~H.
\end{equation}

Bevor wir analog zu \autoref{sec:raum_zeit_variationsformulierung} eine Raum-Zeit-Variationsformulierung für \eqref{eq:parabolische_pde} angeben, betrachten wir zunächst den Operator $A$ genauer.

Bezeichne mit $a(\cdot, \cdot)$ die zu $A$ zugehörige Bilinearform, das heißt es gilt $\skprod{A \eta}{\zeta}_{V' \times V} = a(\eta, \zeta)$ für alle $\eta, \zeta \in V$.
Die Form von $a(\cdot, \cdot)$ lässt sich explizit angeben, denn es gilt\todo{ordentlich formatieren}
\begin{equation}
    \label{eq:bf_a}
    \begin{aligned}
        a(\eta, \zeta)
        &= \skprod{A \eta}{\zeta}_{V' \times V}
        = \skprod{A \eta}{\zeta}_{H}
        \\&= \skprod{- c \Delta \eta + \omega \eta}{\zeta}_{H}
        = - c \skprod{\Delta \eta}{\zeta}_{H} + \skprod{\omega \eta}{\zeta}_{H}
        \\&= -c \int_{\Omega} \Delta \eta \zeta \diff x + \skprod{\omega \eta}{\zeta}_{H}
        = c \int_{\Omega} \skprod{\grad \eta}{\grad \zeta}_{\mathbb{R}^{n}} \diff x + \skprod{\omega \eta}{\zeta}_{H}
        \\&= c \skprod{\grad \eta}{\grad \zeta}_{L_{2}(\Omega, \mathbb{R}^{n})} + \skprod{\omega \eta}{\zeta}_{L_{2}(\Omega)}.
    \end{aligned}
\end{equation}


\begin{Lemma}
    \label{lemma:a_bf_bounded_garding}
    Sei $a \colon V \times V \to \mathbb{R}$ wie in \eqref{eq:bf_a}, also
    \begin{equation}
        a(\eta, \zeta) = c \skprod{\grad \eta}{\grad \zeta}_{L_{2}(\Omega, \mathbb{R}^{n})} + \skprod{\omega \eta}{\zeta}_{L_{2}(\Omega)}
    \end{equation}
    mit $c \in \mathbb{R}_{+}$ und $\omega \in L_{\infty}(\Omega)$.
    Dann erfüllt $a(\cdot, \cdot)$ die folgenden Eigenschaften:
    \begin{thmenumerate}
        \item \emph{Beschränktheit:} es existiert eine Konstante $M_{a} > 0$ mit
        \begin{equation}
            \abs{a(\eta, \zeta)} \leq M_{a} \norm{\eta}_{V} \norm{\zeta}_{V}
        \end{equation}
        für alle $\eta, \zeta \in V$.
        \label{lemma:a_bf_bounded_garding:1}
        \item \emph{G\aa{}rding-Ungleichung:} es existieren Konstanten $\alpha > 0$ und $\lambda \geq 0$ mit
        \begin{equation}
                a(\eta, \eta) + \lambda \norm{\eta}_{H}^{2} \geq \alpha \norm{\eta}_{V}^{2}
        \end{equation}
        für alle $\eta \in V$.
        \label{lemma:a_bf_bounded_garding:2}
    \end{thmenumerate}

    \begin{Beweis}
    Wir zeigen zunächst die Beschränktheit.
    Seien dazu $\eta, \zeta \in V = H^{1}_{0}(\Omega)$ beliebig.
    Unter Verwendung der Dreiecks- und der Cauchy-Schwarz-Ungleichung erhalten wir
    \begin{align}
        \abs{a(\eta, \zeta)}
        &= \abs{c \skprod{\grad \eta}{\grad \zeta}_{L_{2(\Omega, \mathbb{R}^{n})}} + \skprod{\omega \eta}{\zeta}_{L_{2}(\Omega)}}
        \\&\leq c \abs{\skprod{\grad \eta}{\grad \zeta}_{L_{2(\Omega, \mathbb{R}^{n})}}} + \abs{\skprod{\omega \eta}{\zeta}_{L_{2}(\Omega)}}
        \\&\leq c \norm{\grad \eta}_{L_{2}(\Omega)} \norm{\grad \zeta}_{L_{2}(\Omega)} + \norm{\omega}_{L_{\infty}(\Omega)} \norm{\eta}_{L_{2}(\Omega)} \norm{\zeta}_{L_{2}(\Omega)}
        \\&\leq \max \left\{ c, \norm{\omega}_{L_{\infty}(\Omega)} \right\} \norm{\eta}_{H^{1}(\Omega)} \norm{\zeta}_{H^{1}(\Omega)}
        \\&= M_{a} \norm{\eta}_{H^{1}(\Omega)} \norm{\zeta}_{H^{1}(\Omega)}
    \end{align}
    mit $M_{a} = \max \left\{ c, \norm{\omega}_{L_{\infty}(\Omega)} \right\}$.

    Für die G\aa{}rding-Ungleichung seien nun $\eta \in V$ und $\lambda \in \mathbb{R}$.
    Wir betrachten
    \begin{align}
        a(\eta, \eta) + \lambda \norm{\eta}^{2}_{L_{2}(\Omega)}
        &= c \norm{\grad \eta}^{2}_{L_{2}(\Omega)} + \skprod{\omega \eta}{\eta}_{L_{2}(\Omega)} + \lambda \skprod{\eta}{\eta}_{L_{2}(\Omega)}
        \\&= c \norm{\grad \eta}^{2}_{L_{2}(\Omega)} + \skprod{(\omega + \lambda) \eta}{\eta}_{L_{2}(\Omega)}.
        \intertext{Wählen wir nun $\lambda = \norm{\omega}_{L_{\infty}(\Omega)} \geq 0$, dann gilt $\omega + \lambda \geq 0$ fast überall in $\Omega$ und wir erhalten die Abschätzung}
        a(\eta, \eta) + \lambda \norm{\eta}^{2}_{L_{2}(\Omega)}
        &\geq c \norm{\grad \eta}^{2}_{L_{2}(\Omega)},
        \intertext{woraus wir durch Anwenden der Poincaré-Friedrichs-Ungleichung\todo{zitieren}}
        a(\eta, \eta) + \lambda \norm{\eta}^{2}_{L_{2}(\Omega)}
        &\geq c \gamma_{\Omega}^{2} \norm{\eta}^{2}_{H^{1}_{0}(\Omega)}
        = \alpha \norm{\eta}^{2}_{H^{1}_{0}(\Omega)},
    \end{align}
    mit $\alpha = c \gamma_{\Omega}^{2}$, gewinnen.
    \end{Beweis}
\end{Lemma}

Für die Raum-Zeit-Variationsformulierung verwenden wir die aus \eqref{eq:var_all_ansatzraum_x} und \eqref{eq:var_all_testraum_y} bekannten Ansatz- und Testfunktionenräume.
Mit unserer konkreten Wahl $V = H^{1}_{0}(\Omega)$ und $H = L_{2}(\Omega)$ sind dies also
\begin{equation}
    \label{eq:var_ansatzraum_testraum}
    \mathcal X = L_{2}(I; H^{1}_{0}(\Omega)) \cap H^{1}(I; H^{-1}(\Omega))
    \quad \text{und} \quad
    \mathcal Y = L_{2}(I; H^{1}_{0}(\Omega)) \times L_{2}(\Omega).
\end{equation}

Damit ergibt sich analog zu \autoref{sec:raum_zeit_variationsformulierung} das folgende Variationsproblem:

Gegeben ein $g \in L_{2}(I; V')$ und ein $u_{0} \in H$. Finde ein $u \in \mathcal X$ mit
\begin{equation}
    b(u, v) = f(v) \quad \text{für alle}~v \in \mathcal Y,
\end{equation}
wobei $b(\cdot, \cdot) \colon \mathcal X \times \mathcal Y \to \mathbb{R}$ eine durch
\begin{equation}
    b(u, v)
        = \int_{I} \skprod{u_{t}(t)}{v_{1}(t)}_{V' \times V} + a(u(t), v_{1}(t)) \diff t + \skprod{u(0)}{v_{2}}_{H}
\end{equation}
gegebene Bilinearform und $f(\cdot) \colon \mathcal Y \to \mathbb{R}$ definiert ist durch
\begin{equation}
    \label{eq:var_all_f_wiederholung}
    f(v) = \int_{I} \skprod{g(t)}{v_{1}(t)}_{V' \times V} \diff t + \skprod{u_{0}}{v_{2}}_{H}.
\end{equation}


\subsection{Well-posedness} % (fold)
\label{ssub:well_posedness}
Um zu klären, ob und unter welchen Bedingungen das obige Variationsproblem \emph{well-posed} ist, das heißt eine eindeutige Lösung existiert, verweisen wir auf \cite{Schwab:2009ec}.
Es reicht aus, Beschränktheit und eine G\aa{}rding-Ungleichung für die Bilinearform $a(\cdot, \cdot)$ nachzuweisen.

Zusätzlich zur eindeutigen Lösbarkeit liefert uns \cite{Schwab:2009ec} auch Schranken für die Stetigkeits- beziehungsweise inf-sup-Konstanten.
Dazu definieren wir zunächst
\begin{equation}
    M_{e} = \sup_{0 \neq u \in \mathcal X} \frac{\norm{u(0)}_{L_{2}(\Omega)}}{\norm{u}_{\mathcal X}}
\end{equation}
und
\begin{equation}
    \rho = \sup_{0 \neq \eta \in H^{1}_{0}(\Omega)} \frac{\norm{\eta}_{L_{2}(\Omega)}}{\norm{\eta}_{H^{1}_{0}(\Omega)}}.
\end{equation}

\begin{Lemma}
    \label{lemma:schranken_an_b}
    Es gilt\todo{nochmal ordentlich nachrechnen}
    \begin{equation}
        \sup_{0 \neq u \in \mathcal X} \sup_{0 \neq v \in \mathcal Y} \frac{\abs{b(u, v)}}{\norm{u}_{\mathcal X}\norm{v}_{\mathcal Y}}
        \leq M_{b} = \frac{\sqrt{2 \max\{1, \norm{\omega}_{L_{\infty}(\Omega)}^{2}\} + M_{e}^{2}}}{\max\{ \sqrt{1 + 2 \norm{\omega}_{L_{\infty}(\Omega)}^{2} \rho^{4}}, \sqrt{2} \}}
    \end{equation}
    und
    \begin{equation}
        \inf_{0 \neq u \in \mathcal X} \sup_{0 \neq v \in \mathcal Y} \frac{\abs{b(u, v)}}{\norm{u}_{\mathcal X}\norm{v}_{\mathcal Y}}
        \geq \beta = \frac{e^{-2 \norm{\omega}_{L_{\infty}(\Omega)} T}}{\max\{\sqrt{ 1 + 2 \norm{\omega}_{L_{\infty}(\Omega)} \rho^{4}}, \sqrt{2} \}} \frac{\min\{ \sigma \gamma_{\Omega}^{2} \norm{\omega}_{L_{\infty}(\Omega)}^{-2}, \sigma \gamma_{\Omega}^{2} \}}{\sqrt{2 \max\{ 1, \sigma^{-2} \gamma_{\Omega}^{-4} \} + M_{e}^{2}}}
    \end{equation}
\end{Lemma}
