%!TEX root = ../main.tex

\chapter{Einführung} % (fold)
\label{cha:einfuehrung}

Wir führen nun zunächst die parabolische partielle Differentialgleichung ganz allgemein, also noch ohne konkrete Ausrichtung auf die Ziele dieser Arbeit, ein.
Anschließend leiten wir für diese eine Raum-Zeit-Variationsformulierung und behandeln Existenz und Eindeutigkeit von Lösungen des daraus gewonnenen Variationsproblems.
Dieses Kapitel orientiert sich an den Ausführungen in~\cite{Schwab:2009ec, Urban:2014kg}.

\section{Allgemeine Problemstellung} % (fold)
\label{sec:allgemeine_problemstellung}

Zunächst einige notationelle Vorbereitungen.
Seien $V$ und $H$ zwei separable Hilberträume, wobei eine dichte stetige Einbettung $V \hookrightarrow H$ existiere.
% Später werden wir oftmals mit $H = L_{2}(\Omega)$ und einem dichten Unterhilbertraum $V \subset H$, $\Omega \subset \mathbb{R}^{n}$, arbeiten
Durch die Identifikation $H \cong H'$ von $H$ mit seinem Dualraum $H'$ erhalten wir ein Gelfand-Tripel $V \hookrightarrow H \hookrightarrow V'$.
Wir verwenden die Schreibweise $\skprod{\blank}{\blank}$ mit entsprechendem Index sowohl für die inneren Produkte der Hilberträume, als auch für die duale Paarung auf $V' \times V$.

Es sei $0 < T < \infty$ und $I = [0, T]$.
Weiterhin sei $A \in \mathcal L(V, V')$ ein stetiger linearer Operator und $a(\blank, \blank) \colon V \times V \to \mathbb{R}$ die zugehörige Bilinearform, das heißt es gilt $\skprod{A \eta}{\zeta}_{V' \times V} = a(\eta, \zeta)$ für $\eta, \zeta \in V$.
Seien außerdem $g \in L_{2}(I; V')$ und $u_{0} \in H$ gegeben.
Wir sind nun an Lösungen der parabolischen partiellen Differentialgleichung
\begin{equation}
    \label{eq:allgemeine_parabolische_pde}
    u_{t}(t) + A u(t) = g(t) \quad \text{in}~V',
    \qquad
    u(0) = u_{0} \quad \text{in}~H
\end{equation}
interessiert.

Von der Bilinearform $a(\blank, \blank)$ fordern wir außerdem Stetigkeit, das heißt die Existenz einer Konstante $0 < M_{a} < \infty$, so dass die Ungleichung
\begin{equation}
    \label{eq:allgemeine_parabolische_pde:bf_stetig}
    \abs{a(\eta, \zeta)} \leq M_{a} \norm{\eta}_{V} \norm{\zeta}_{V} \quad \text{für alle}~\eta, \zeta \in V
\end{equation}
erfüllt ist, und eine G\r{a}rding-Ungleichung, das heißt es existieren Konstanten $\alpha > 0$ und $\lambda \geq 0$ mit
\begin{equation}
    \label{eq:allgemeine_parabolische_pde:bf_garding}
    a(\eta, \eta) + \lambda \norm{\eta}_{H}^{2} \geq \alpha \norm{\eta}_{V}^{2} \quad \text{für alle}~\eta \in V.
\end{equation}

% section allgemeine_problemstellung (end)

\section{Raum-Zeit-Variationsformulierung} % (fold)
\label{sec:raum_zeit_variationsformulierung}

\subsection{Herleitung} % (fold)
\label{sub:herleitung}

Da, wie so oft, klassische Lösungen von partiellen Differentialgleichungen in der Numerik nur von geringem Interesse sind, leiten wir nun eine Variationsformulierung für das Problem~\eqref{eq:allgemeine_parabolische_pde} her.
% TODO: Quelle
Hierbei werden Raum- und Zeitkoordinaten oftmals verschieden behandelt, das heißt die Variationsformulierung wird nur bezüglich der Raumkoordinaten hergeleitet und die Zeitkoordinate wird separat behandelt.
Dies wollen wir vermeiden und führen eine Raum-Zeit-Variationsformulierung ein, bei der die Raum- und Zeitkoordinaten gleichwertig behandelt werden.

Wir benötigen für die Variationsformulierung zunächst einen Ansatz- und einen Testfunktionenraum.
Der Ansatzraum sei gegeben durch
\begin{equation}
    \label{eq:var_all_ansatzraum_x}
    \mathcal X = L_{2}(I; V) \cap H^{1}(I; V') = \Set*{ u \given u \in L_{2}(I; V),~u_{t} \in L_{2}(I; V') }
\end{equation}
ausgestattet mit der Graphnorm
\begin{equation}
    \label{eq:var_all_ansatzraum_x_norm}
    \norm{u}_{\mathcal X} = \left( \norm{u}_{L_{2}{(I; V)}}^{2} + \norm{u_{t}}_{L_{2}{(I; V')}}^{2} \right)^{\frac 12}, \quad u \in \mathcal X.
\end{equation}
Als Testfunktionenraum verwenden wir
\begin{equation}
    \label{eq:var_all_testraum_y}
    \mathcal Y = L_{2}(I; V) \times H,
\end{equation}
wobei hierbei die Norm
\begin{equation}
    \label{eq:var_all_testraum_y_norm}
    \norm{v}_{\mathcal Y} = \left( \norm{v_{1}}_{L_{2}(I; V)}^{2} + \norm{v_{2}}_{H}^{2} \right)^{\frac 12}, \quad v = (v_{1}, v_{2}) \in \mathcal Y,
\end{equation}
zum Einsatz kommt.

% TODO: Herleitung aus-x-en?
Multiplizieren wir nun die partielle Differentialgleichung~\eqref{eq:allgemeine_parabolische_pde} mit $v = (v_{1}, v_{2}) \in \mathcal Y$ und integrieren über den Zeitintervall $I$, dann erhalten wir daraus das Variationsproblem:

Gegeben ein $g \in L_{2}(I, V')$ und $u_{0} \in H$. Finde ein $u \in \mathcal X$ mit
\begin{equation}
    \label{eq:var_all_problem}
    b(u, v) = f(v) \quad \text{für alle}~v \in \mathcal Y,
\end{equation}
wobei $f$ von $g$ und $u_{0}$ abhängt.
Dabei ist $b(\blank, \blank) \colon \mathcal X \times \mathcal Y \to \mathbb{R}$ eine Bilinearform definiert durch
\begin{equation}
    \label{eq:var_all_bf_b}
    b(u, v) = \int_{I} \skprod{u_{t}(t)}{v_{1}(t)}_{V' \times V} + a(u(t), v_{1}(t)) \diff t + \skprod{u(0)}{v_{2}}_{H},
\end{equation}
und $f(\blank) \colon \mathcal Y \to \mathbb{R}$ ein Funktional gegeben durch
\begin{equation}
    \label{eq:var_all_f}
    f(v) = \int_{I} \skprod{g(t)}{v_{1}(t)}_{V' \times V} \diff t + \skprod{u_{0}}{v_{2}}_{H}.
\end{equation}

% subsection herleitung (end)

\subsection{Existenz und Eindeutigkeit von Lösungen} % (fold)
\label{sub:existenz_und_eindeutigkeit_von_l_sungen}

Um Existenz und Eindeutigkeit von Lösungen für das Variationsproblem~\eqref{eq:var_all_problem} zu erhalten, zeigen wir, dass die Bilinearform $b(\blank, \blank)$ einen stetig invertierbaren Operator $B \in \mathcal L(\mathcal X, \mathcal Y')$ definiert.
Folgender Satz fasst das Ganze zusammen:

\begin{Satz}[{{\cite[Theorem 5.1]{Schwab:2009ec}}}]
\label{thm:schwab09:theorem51}
    Seien $\mathcal X$ und $\mathcal Y$ gegeben wie in~\eqref{eq:var_all_ansatzraum_x} respektive~\eqref{eq:var_all_testraum_y} und sei $B \colon \mathcal X \to \mathcal Y'$ definiert durch
    \begin{equation}
        \label{eq:var_all_gross_b}
        \skprod{B u}{v}_{\mathcal Y' \times \mathcal Y} = b(u, v), \quad u \in \mathcal X,~y \in \mathcal Y,
    \end{equation}
    mit $b(\blank, \blank)$ wie in~\eqref{eq:var_all_bf_b}.
    Dann ist $B$ stetig invertierbar und es gilt
    \begin{equation}
        \label{eq:var_all_norm_B}
        \norm{B}_{\mathcal L(\mathcal X, \mathcal Y')} \leq \frac{\sqrt{2\max\Set{1, M_{a}^{2}} + M_{e}^{2}}}{\max\Set{\sqrt{1 + 2 \lambda^{2} \rho^{4}}, \sqrt{2}}}
    \end{equation}
    sowie
    \begin{equation}
        \label{eq:var_all_norm_B_inv}
        \norm{B^{-1}}_{\mathcal L(\mathcal Y', \mathcal X)} \leq \frac{e^{2 \lambda T} \max\Set{\sqrt{1 + 2 \lambda^{2} \rho^{4}}, \sqrt{2}} \sqrt{2 \max\Set{ \alpha^{-2}, 1} + M_{e}^{2}}}{\min\Set{\alpha M_{a}^{-2}, \alpha}}.
    \end{equation}

    \begin{Beweis}
        Siehe~\cite[Appendix A]{Schwab:2009ec}.
        % \begin{Beweis}
        %     Wir weisen die Bedingungen von \thref{satz:babuska-aziz} nach.

        %     Zunächst sei anzumerken, dass wir in \eqref{eq:garding-inequality} ohne Einschränkung $\lambda = 0$ wählen können.
        %     Wähle
        %     \begin{equation}
        %         u(t) = \hat u(t) e^{\lambda t}, \quad v_{1}(t) = \hat v_{1}(t) e^{- \lambda t}, \quad g(t) = \hat g(t) e^{\lambda t},
        %     \end{equation}
        %     dann sieht man, dass $u$ die Gleichung \eqref{eq:bilinearform} genau dann löst, wenn $\hat u$ die Gleichung
        %     \begin{equation}
        %         \label{eq:bilinearform_tmp}
        %         \begin{gathered}
        %             \int_{I} \skprod{\hat{u}_{t}(t)}{\hat{v}_{1}(t)}_{H} + \lambda \skprod{\hat{u}(t)}{\hat{v}_{1}(t)}_{H} + a(t; \hat{u}(t), \hat{v}_{1}(t)) \diff t + \skprod{\hat{u}(0)}{v_{2}}_{H}
        %                 \\= \int_{I} \skprod{\hat{g}(t)}{\hat{v}_{1}(t)}_{H} \diff t + \skprod{u_{0}}{v_{2}}_{H}
        %         \end{gathered}
        %     \end{equation}
        %     für alle $\hat{v} = (\hat{v}_{1}, v) \in \mathcal Y$ löst.

        %     \paragraph{Stetigkeit} % (fold)
        %     \label{par:stetigkeit}
        %     Betrachte für $u \in \mathcal X$ und $v = (v_{1}, v_{2}) \in \mathcal Y$ die Bilinearform $b(u, v)$.
        %     Nach Anwenden der Dreiecksungleichung erhalten wir
        %     \begin{equation}
        %         \label{eq:stetigkeit_zweiter_term}
        %         \abs{b(u, v)} = \int_{I} \abs{\skprod{u_{t}(t)}{v_{1}(t)}_{H}} + \abs{a(u(t), v_{1}(t))} \diff t + \abs{\skprod{u(0)}{v_{2}}_{H}}.
        %     \end{equation}
        %     Betrachten wir zunächst den hinteren Term, dann erhalten wir unter Verwendung der Cauchy-Schwarz-Ungleichung und der Einbettungs-Konstante $M_{e}$ die Abschätzung
        %     \begin{equation}
        %         \abs{\skprod{u(0)}{v_{2}}_{H}} \leq \norm{u(0)}_{H} \norm{v_{2}}_{H} \leq M_{e} \norm{u}_{X} \norm{v_{2}}_{H}.
        %     \end{equation}
        %     Widmen wir uns nun dem ersten Term und wenden ebenfalls die Cauchy-Schwarz-Ungleichung sowie die Stetigkeit von $a$ an, dann erhalten wir
        %     \begin{align}
        %         &\int_{I} \abs{\skprod{u_{t}(t)}{v_{1}(t)}_{H}} + \abs{a(u(t), v_{1}(t))} \diff t
        %         \\&\qquad
        %         \leq \int_{I} \norm{u_{t}(t)}_{H} \norm{v_{1}(t)}_{H} + M_{a} \norm{u(t)}_{H} \norm{v_{1}(t)}_{H} \diff t
        %         \\&\qquad
        %         \leq \int_{I} \max\{1, M_{a}\} \norm{v_{1}(t)}_{H} \left(  \norm{u_{t}(t)}_{H} + \norm{u(t)}_{H} \right) \diff t
        %         \intertext{mittels Hölder-Ungleichung lässt sich dies weiter abschätzen zu}
        %         &\qquad
        %         \leq \left( \int_{I} \max\{1, M_{a}\}^{2} \norm{v_{1}(t)}_{H}^{2} \diff t \right)^{\frac 12} \left( \int_{I} \left( \norm{u_{t}(t)}_{H} + \norm{u(t)}_{H} \right)^{2} \diff t \right)^{\frac 12},
        %         \intertext{und unter Verwendung der Youngschen-Ungleichung zu}
        %         &\qquad
        %         \leq \left( \int_{I} \max\{1, M_{a}\}^{2} \norm{v_{1}(t)}_{H}^{2} \diff t \right)^{\frac 12} \left( \int_{I} 2 \left( \norm{u_{t}(t)}_{H}^{2} + \norm{u(t)}_{H}^{2} \right) \diff t \right)^{\frac 12}
        %         \intertext{was nach Definition der verwendeten Normen auch geschrieben werden kann als}
        %         &\qquad
        %         = \sqrt{2 \max\{1, M_{a}^{2}\}} \norm{u}_{\mathcal X} \norm{v_{1}}_{L_{2}(I; V)}
        %     \end{align}
        %     Zusammen mit \eqref{eq:stetigkeit_zweiter_term} liefert dies nach einer erneuten Anwendung der Cauchy-Schwarz-Ungleichung
        %     \begin{align}
        %     \abs{b(u, v)}
        %     &\leq \sqrt{2 \max\{1, M_{a}\}^{2}} \norm{u}_{\mathcal X} \norm{v_{1}}_{L_{2}(I; V)} + M_{e} \norm{u}_{X} \norm{v_{2}}_{H}
        %     \\
        %     &\leq \norm{u}_{\mathcal X} \left( \norm{v_{1}}_{L_{2}(I; V)}^{2} + \norm{v_{2}}_{H}^{2} \right)^{\frac 12} \left( 2 \max\{1, M_{a}\}^{2} + M_{e}^{2} \right)^{\frac 12}
        %     \\
        %     &= \sqrt{2 \max\{1, M_{a}^{2}\} + M_{e}^{2}} \norm{u}_{\mathcal X} \norm{v}_{\mathcal Y}.
        %     \end{align}
        %     Damit folgt die Stetigkeit.
        %     % paragraph stetigkeit (end)

        %     \paragraph{Inf-Sup-Bedingung} % (fold)
        %     \label{par:inf_sup_bedingung}

        %     % paragraph inf_sup_bedingung (end)
    \end{Beweis}
\end{Satz}

Die Größen $M_{a}$, $\alpha$ und $\lambda$ stammen aus der Stetigkeit~\eqref{eq:allgemeine_parabolische_pde:bf_stetig} beziehungsweise der G\r{a}rding-Ungleichung~\eqref{eq:allgemeine_parabolische_pde:bf_garding} der Bilinearform $a(\blank, \blank)$.
Die Konstante $M_{e}$ ist definiert durch
% TODO: genauer?
\begin{equation}
    \label{eq:var_all_M_e}
    M_{e} = \sup_{0 \neq u \in \mathcal X} \frac{\norm{u(0)}_{H}}{\norm{u}_{\mathcal X}}
\end{equation}
und $\rho$ durch
\begin{equation}
    \label{eq:var_all_rho}
    \rho = \sup_{0 \neq \eta \in V} \frac{\norm{\eta}_{H}}{\norm{\eta}_{V}}.
\end{equation}

% subsection existenz_und_eindeutigkeit_von_l_sungen (end)

% section raum_zeit_variationsformulierung (end)

\section{Parametrisches Problem} % (fold)
\label{sec:parametrisches_problem}

Wir beschäftigen uns nun mit einer parametrischen Variante der parabolischen partiellen Differentialgleichung~\eqref{eq:allgemeine_parabolische_pde}, folgern erneut Existenz und Eindeutigkeit von Lösungen und zeigen die analytische Abhängigkeit der Lösung vom Parameter.
Dieser Abschnitt basiert auf~\cite{Kunoth:2013ef}.

Bevor wir die gewünschten Aussagen für die später vorgestellte parametrische Variante der parabolischen partiellen Differentialgleichung~\eqref{eq:allgemeine_parabolische_pde} folgern können, müssen wir uns zunächst mit einer parametrischen Operatorgleichung beschäftigen.

Es seien $X$ und $Y$ zwei reflexive Banachräume und $\mathcal S \subset \mathbb{R}^{\mathbb{N}}$ bezeichne den sogenannten Parameterraum, ohne Einschränkung wählen wir $\mathcal S = {[-1, 1]}^{\mathbb{N}}$.
Für alle $\sigma \in \mathcal S$ sei nun durch $A(\sigma) \in \mathcal L(X, Y')$ ein stetiger linearer Operator gegeben.
Wir betrachten nun für $g \in Y'$ die parametrische Operatorgleichung
\begin{equation}
    \label{eq:allgemeine_parametrische_elliptische_pde}
    A(\sigma) u(\sigma) = g \quad \text{in}~Y'.
\end{equation}
Um Aussagen über die Operatorgleichung~\eqref{eq:allgemeine_parametrische_elliptische_pde} treffen zu können, müssen wir zunächst die Abhängigkeit des Operators $A(\sigma)$ vom Parameter $\sigma$ konkretisieren.
Zunächst aber einige Notationen.

\begin{Bemerkung}
    Wir bezeichnen mit $\mathfrak F = \Set{ \nu \in \mathbb{N}^{\mathbb{N}}_{0} \given \abs{\nu} < \infty }$ die Menge aller Folgen nichtnegativer ganzer Zahlen mit endlichem Träger, das heißt nur endlich vielen Einträgen ungleich Null.
    % NOTE: Eventuell mehr definieren, siehe $\mathfrak n$ und $\mathfrak m$

    Sei $\nu \in \mathfrak F$ und $b \in \ell^{p}$, $p > 0$, dann schreiben wir
    \begin{equation}
        b^{\nu} = \prod_{j = 1}^{\infty} b_{j}^{\nu_{j}}
    \end{equation}
    mit der Konvention $0^{0} = 1$.
    Wegen $\abs{\nu} < \infty$ ist dieses Produkt endlich.
\end{Bemerkung}


\begin{Annahme}[{{\cite[Assumption 1]{Kunoth:2013ef}}}]
\label{thm:kunoth:assumption1}
    Die parametrische Familie von Operatoren
    $\Set{ A(\sigma) \in \mathcal L(X, Y') \given \sigma \in \mathcal S }$ sei $\mathfrak p$-regulär für ein $0 < \mathfrak p \leq 1$, das heißt
    \begin{thmenumerate}
        \item $A(\sigma) \in \mathcal L(X, Y')$ sei stetig invertierbar für alle $\sigma \in \mathcal S$ mit gleichmäßig beschränktem Inversen $A{(\sigma)}^{-1} \in \mathcal L(Y', X)$, das heißt es existiert ein $C_{0} > 0$ mit
        \begin{equation}
            \sup_{\sigma \in \mathcal S} \norm{A{(\sigma)}^{-1}}_{\mathcal L(Y', X)} \leq C_{0},
        \end{equation}
        \item für jedes feste $\sigma \in \mathcal S$ seien die Operatoren $A(\sigma)$ analytisch bezüglich $\sigma$, konkret existiert eine nichtnegative Folge $b = (b_{j})_{j \geq 1} \in \ell^{\mathfrak p}$ so dass
        \begin{equation}
            \sup_{\sigma \in \mathcal S} \norm{(A{(0)}^{-1})(\partial^{\nu}_{\sigma} A(\sigma))}_{\mathcal L(X, X)} \leq C_{0} b^{\nu}
        \end{equation}
        für alle $\nu \in \mathfrak F \setminus \{ 0 \}$ gilt, wobei $\partial^{\nu}_{\sigma} A(\sigma) = \frac{\partial^{\nu_{1}}}{\partial \sigma_{1}} \frac{\partial^{\nu_{2}}}{\partial \sigma_{2}} \cdots A(\sigma)$ sei.
    \end{thmenumerate}
\end{Annahme}

Der für uns wichtigste Fall der parametrischen Abhängigkeit von $A(\sigma)$ ist die affine Abhängigkeit, das heißt es existiert eine Familie von Operatoren $\Set{ A_{j} }_{j \geq 0} \subset \mathcal L(X, Y')$, so dass
\begin{equation}
    \label{eq:all_affiner_operator}
    A(\sigma) = A_{0} + \sum_{j = 1}^{\infty} \sigma_{j} A_{j} \qquad\text{für alle}~\sigma \in \mathcal S
\end{equation}
gilt.
Wir bezeichnen mit $a_{j}(\blank, \blank) \colon X \times Y \to \mathbb{R}$ die zu $A_{j}$, $j \geq 0$, zugehörigen Bilinearformen, also
\begin{equation}
    \label{eq:allg_affine_bf}
    a_{j}(\eta, \zeta) = \skprod{A_{j} \eta}{\zeta}_{Y' \times Y}, \quad \eta \in X,~\zeta \in Y.
\end{equation}

Um die Konvergenz von~\eqref{eq:all_affiner_operator} sicherzustellen, fordern wir:

\begin{Annahme}[{{\cite[Assumption 2]{Kunoth:2013ef}}}]
\label{thm:kunoth:assumption2}
    Die Operatorfamilie $\Set{ A_{j} }_{j \geq 0}$ erfülle folgende Eigenschaften:
    \begin{thmenumerate}
        \item Der \emph{Mean Field}-Operator $A_{0} \in \mathcal L(X, Y')$ sei stetig invertierbar, das heißt es existiert ein $\gamma_{0} > 0$ mit
        \begin{subequations}\label{eq:kunoth:ass2_gamma_0}
            \begin{align}
                \label{eq:kunoth:ass2_gamma_0_a}
                \inf_{0 \neq u \in X} \sup_{0 \neq v \in Y} \frac{a_{0}(u, v)}{\norm{u}_{X} \norm{v}_{Y}} \geq \gamma_{0}
                \intertext{und}
                \label{eq:kunoth:ass2_gamma_0_b}
                \inf_{0 \neq v \in Y} \sup_{0 \neq u \in X} \frac{a_{0}(u, v)}{\norm{u}_{X} \norm{v}_{Y}} \geq \gamma_{0}.
            \end{align}
        \end{subequations}
        \item Die \emph{Fluctuation}-Operatoren $\Set{ A_{j} }_{j \geq 1}$ seien \emph{klein} relativ zu $A_{0}$ im folgenden Sinne: es existiert eine Konstante $0 < \kappa < 1$ so dass
        \begin{equation}
            \label{eq:kunoth:ass2_abs_reihe}
            \sum_{j = 1}^{\infty} \norm{A_{j}}_{\mathcal L(X, Y')} \leq \kappa \gamma_{0}
        \end{equation}
        gilt.
    \end{thmenumerate}
\end{Annahme}

\begin{Korollar}[{{\cite[Corollary 3]{Kunoth:2013ef}}}]
\label{thm:kunoth:corollary3}
    Die affin parametrische Operatorfamilie $\{ A_{j} \}_{j \geq 0}$ erfülle \thref{thm:kunoth:assumption2}, dann wird auch \thref{thm:kunoth:assumption1} mit $\mathfrak p = 1$ und
    \begin{equation}
        C_{0} = \frac{1}{(1 - \kappa) \gamma_{0}}, \qquad b_{j} = \frac{\norm{A_{j}}_{\mathcal L(X, Y')}}{(1 - \kappa) \gamma_{0}}, \quad \text{für alle}~j \geq 1,
    \end{equation}
    erfüllt.
\end{Korollar}

\begin{Satz}[{{\cite[Theorem 4]{Kunoth:2013ef}}}]
\label{thm:kunoth:theorem4}
    Die parametrische Familie $\Set{ A(\sigma) \in \mathcal L(X, Y') \given \sigma \in \mathcal S }$ erfülle \thref{thm:kunoth:assumption1} für ein $0 < \mathfrak p \leq 1$.
    Dann existiert für jedes $g \in Y'$ und jedes $\sigma \in \mathcal S$ eine eindeutige Lösung $u(\sigma) \in X$ der parametrischen Operatorgleichung~\eqref{eq:allgemeine_parametrische_elliptische_pde}
    \begin{equation}
        A(\sigma) u(\sigma) = g \quad \text{in}~Y'.
    \end{equation}
    Die parametrische Familie von Lösungen $u(\sigma)$ hängt analytisch vom Parameter $\sigma$ ab und die partiellen Ableitungen von $u(\sigma)$ erfüllen
    \begin{equation}
        \label{eq:kunoth:schranke_part_abl}
        \sup_{\sigma \in \mathcal S} \norm{(\partial^{\nu}_{\sigma} u)(\sigma)}_{X} \leq C_{0} \norm{g}_{Y'} \abs{\nu}! \tilde{b}^{\nu}
    \end{equation}
    für alle $\nu \in \mathfrak F$, wobei die Folge $\tilde{b} = (\tilde{b}_{j})_{j \geq 1} \in \ell^{\mathfrak p}$ definiert ist durch
    \begin{equation}
        \tilde{b}_{j} = \frac{b_{j}}{\ln 2} \qquad \text{für alle j} \in \mathbb{N}.
    \end{equation}
\end{Satz}

Aus diesen Ergebnissen erhalten wir nun die gewünschten Aussagen für die parametrische Variante von~\eqref{eq:allgemeine_parabolische_pde}, diese müssen wir aber nun zunächst konkretisieren.
Wir arbeiten nun wieder im Setting aus Abschnitt~\ref{sec:allgemeine_problemstellung}.

Für $\sigma \in \mathcal S$ sei $A(\sigma) \in \mathcal L(V, V')$ ein stetiger linearer Operator und $a(\blank, \blank; \sigma) \colon V \times V \to \mathbb{R}$ die zugehörige Bilinearform, also $\skprod{A(\sigma) \eta}{\zeta}_{V' \times V} = a(\eta, \zeta; \sigma)$ für $\eta, \zeta \in V$.
Die Bilinearform $a(\blank, \blank; \sigma)$ sei stetig und erfülle eine G\r{a}rding-Ungleichung gleichmäßig in $\sigma$, das heißt es existieren von $\sigma$ unabhängige Konstanten $0 < M_{a} < \infty$, $\alpha > 0$ und $\lambda \geq 0$ mit
\begin{equation}
    \label{eq:allgemeine_parabolische_pde:bf_stetig_parametrisch}
    \abs{a(\eta, \zeta; \sigma)} \leq M_{a} \norm{\eta}_{V} \norm{\zeta}_{V} \quad \text{für alle}~\eta, \zeta \in V
\end{equation}
und
\begin{equation}
    \label{eq:allgemeine_parabolische_pde:bf_garding_parametrisch}
    a(\eta, \eta; \sigma) + \lambda \norm{\eta}_{H}^{2} \geq \alpha \norm{\eta}_{V}^{2} \quad \text{für alle}~\eta \in V.
\end{equation}

Analog zu der Herleitung in Abschnitt~\ref{sec:raum_zeit_variationsformulierung} erhalten wir das parametrische Variationsproblem:

Gegeben ein $g \in L_{2}(I; V')$. Finde für alle $\sigma \in \mathcal S$ ein $u(\sigma) \in \mathcal X$ mit
\begin{equation}
    \label{eq:var_all_problem_parametrisch}
    b(u(\sigma), v; \sigma) = f(v) \quad \text{für alle}~v \in \mathcal Y,
\end{equation}
wobei $f$ von $g$ abhängt.
Dabei ist $b(\blank, \blank; \sigma) \colon \mathcal X \times \mathcal Y \to \mathbb{R}$ eine Bilinearform definiert durch
\begin{equation}
    \label{eq:var_all_bf_b_parametrisch}
    b(u, v; \sigma) = \int_{I} \skprod{u_{t}(t)}{v_{1}(t)}_{V' \times V} + a(u(t), v_{1}(t); \sigma) \diff t + \skprod{u(0)}{v_{2}}_{H},
\end{equation}
und $f(\blank) \colon \mathcal Y \to \mathbb{R}$ ein Funktional gegeben durch
\begin{equation}
    \label{eq:var_all_f_parametrisch}
    f(v) = \int_{I} \skprod{g(t)}{v_{1}(t)}_{V' \times V} \diff t + \skprod{u_{0}}{v_{2}}_{H}.
\end{equation}

\begin{Satz}[{{\cite[Theorem 21]{Kunoth:2013ef}}}]
\label{thm:kunoth:theorem21}
    Seien $\mathcal X$ und $\mathcal Y$ gegeben wie in~\eqref{eq:var_all_ansatzraum_x} respektive~\eqref{eq:var_all_testraum_y} und die Familie von Operatoren $\Set{ A(\sigma) \given \sigma \in \mathcal S }$ erfülle \thref{thm:kunoth:assumption1} für ein $0 < \mathfrak p \leq 1$.
    Für jedes $\sigma \in \mathcal S$ sei $B(\sigma) \in \mathcal L(\mathcal X, \mathcal Y')$ definiert durch
    \begin{equation}
        \label{eq:var_all_gross_b_parametrisch}
        \skprod{B(\sigma) u}{v}_{\mathcal Y' \times \mathcal Y} = b(u, v; \sigma), \quad u \in \mathcal X,~y \in \mathcal Y,
    \end{equation}
    mit $b(\blank, \blank; \sigma)$ wie in~\eqref{eq:var_all_bf_b_parametrisch}.
    Dann ist $B(\sigma)$ für jedes $\sigma \in \mathcal S$ stetig invertierbar und es existieren Konstanten $0 < \beta_{1} \leq \beta_{2} < \infty$ mit
    \begin{equation}
        \label{eq:var_all_norm_B_und_B_inv_parametrisch}
        \sup_{\sigma \in \mathcal S} \norm{B(\sigma)} \leq \beta_{2} \quad \text{und} \quad  \sup_{\sigma \in \mathcal S} \norm{B(\sigma)^{-1}} \leq \frac{1}{\beta_{1}}.
    \end{equation}
    Die parametrische Familie von Operatoren $\{ B(\sigma) : \sigma \in \mathcal S \}$ erfüllt \thref{thm:kunoth:assumption1} mit dem selben Regularitätsparameter $\mathfrak p$, die parametrische Familie von Lösungen $u(\sigma)$ hängt analytisch von $\sigma$ ab und erfüllt die \emph{a priori}-Abschätzung
    \begin{equation}
        \label{eq:var_all_a_priori_schranke}
        \sup_{\sigma \in \mathcal S} \norm{(\partial^{\nu}_{\sigma} u)(\sigma)}_{\mathcal X} \leq C_{0} \norm{f}_{\mathcal Y'} \abs{\nu}! \tilde{b}^{\nu}
    \end{equation}
    für alle $\nu \in \mathfrak F$, wobei $f$ wie in~\eqref{eq:var_all_f_parametrisch} gegeben ist.

    \begin{Beweis}
        TODO:\@ Bedingungen von \thref{thm:kunoth:assumption1} nachrechnen.
        Zu (i): Folgt aus \thref{thm:schwab09:theorem51}, da $M_{a}, \alpha, \lambda$ unabhänging von $\sigma$.
        Zu (ii): Folgt aus nachfolgendem \thref{lemma:norm_B_beschraenkt_durch_norm_A}.
    \end{Beweis}
\end{Satz}

\begin{Lemma}
\label{lemma:norm_B_beschraenkt_durch_norm_A}
    Es gilt
    \begin{equation}
        \norm{\partial^{\nu}_{\sigma} B(\sigma)}_{\mathcal L(\mathcal X, \mathcal Y')}
        \leq
        \norm{\partial^{\nu}_{\sigma} A(\sigma)}_{\mathcal L(V, V')}
        \quad
        \text{für alle}~\sigma \in \mathcal S, \nu \in \mathfrak F \setminus \Set{ 0 }.
    \end{equation}

    % \begin{Beweis}
        % TODO: Moo.
    % \end{Beweis}
\end{Lemma}

% subsubsection aus_cite_kunoth_2013ef (end)

% section parametrisches_problem (end)

% chapter einfuehrung (end)
