%!TEX root = ../main.tex

\chapter{Grundlagen} % (fold)
\label{cha:grundlagen}

In diesem ersten Kapitel werden die für diese Arbeit benötigten Grundlagen aus der Funktionalanalysis und der Numerik zusammengefasst und wiederholt.
Wir orientieren uns dabei maßgeblich an \textcite{Dautray:1992by,Lions:1972tg}.

\section{Bochner-Räume} % (fold)
\label{sec:bochner_r_ume}

Wir beginnen mit der Einführung sogenannter Bochner-Räume.
Dabei handelt es sich um Verallgemeinerungen der Lebesgue-Räume $L_{p}$ auf Banachraum-wertige Funktionen.
Bochner-Räume treten bei der Betrachtung parabolischer partieller Differentialgleichungen in natürlicher Weise auf.

\begin{Definition}[{{{\cite[Definition XVIII.1.1]{Dautray:1992by}}}}]
\label{definition:gl:bochner_raum}
    Sei $X$ ein Banachraum.
    Weiter seien $- \infty \leq a < b \leq \infty$ und $1 \leq p < \infty$.
    Als \emph{Bochner-Raum} $L_{p}(a, b; X)$ bezeichnen wir die Menge (der Äquivalenzklassen) $L_{p}$-integrierbarer Funktionen $f \colon [a, b] \to X$, das heißt, aller Lebesgue-messbarer Funktionen auf $[a, b]$ mit
    \begin{equation}
        \norm{f}_{L_{p}(a, b; X)} \deq \left( \int_{a}^{b} \norm{f(t)}_{X}^{p} \diff t \right)^{1 / p} < \infty.
    \end{equation}
    Weiterhin ist der \emph{Bochner-Raum} $L_{\infty}(a, b; X)$ definiert als die Menge (der Äquivalenzklassen) der für fast alle $t \in [a, b]$ wesentlich beschränkten Funktionen mit
    \begin{equation}
        \norm{f}_{L_{\infty}(a, b; X)} \deq \esssup_{t \in [a, b]} \norm{f(t)}_{X} < \infty.
    \end{equation}
\end{Definition}
\nomenclature{$X, Y$}{Banachräume}

\begin{Lemma}[{{\cite[Proposition XVIII.1.1]{Dautray:1992by}}}, {{\cite[Abschnitt 1.1.3]{Lions:1972tg}}}]
\label{lemma:gl:bochner_ist_banach}
    Für alle $1 \leq p \leq \infty$ ist $L_{p}(a, b; X)$ ein Banachraum.
    Falls $H$ ein Hilbertraum ist, so auch $L_{2}(a, b; H)$.
\end{Lemma}

\begin{Definition}[Schwache Zeitableitung, {{{\cite{Dautray:1992by}}}}]
\label{definition:gl:schwache_zeitableitung}
    Seien $X$ und $Y$ Banachräume mit $X \hookrightarrow Y$ und $u \in L_{2}(a, b; X)$.
    Die distributionelle Ableitung $\frac{\partial}{\partial t} u \in L_{2}(a, b; Y)$ sei definiert als das $v \in L_{2}(a, b; Y)$, welches
    \begin{equation}
        \int_{a}^{b} v(t) \varphi(t) \diff t = - \int_{a}^{b} u(t) \frac{\partial}{\partial t} \varphi(t) \diff t \quad \fa \phi \in C^{\infty}_{0}((a, b), \mathbb{R})
    \end{equation}
    erfüllt, falls ein solches $v$ existiert.
\end{Definition}

\begin{Bemerkung}
    Je nach Situation werden wir der Einfachheit halber eine der Schreibweisen $\frac{\partial}{\partial t} u = u' = u_{t}$ verwenden.
\end{Bemerkung}

Im Zuge dieser Arbeit werden wir es stets mit Hilberträumen zu tun haben.
Dabei werden wir oft auf folgendes Konstrukt zurückgreifen.

\begin{Definition}
\label{definition:gl:gelfand_tripel}
    Seien $V$ und $H$ separable Hilberträume mit den Dualräumen $V'$ und $H'$.
    Weiter sei $V$ ein dichter Unterraum von $H$.
    Durch Identifikation von $H$ mit seinem Dualraum $H'$ erhalten wir das sogenannte \emph{Gelfand-Tripel} $(V, H, V')$, oder auch
    \begin{equation}
        V \denseinclusion H \simeq H' \denseinclusion V',
    \end{equation}
    wobei die Inklusionen jeweils dichte stetige Einbettungen sind.
\end{Definition}
\nomenclature{$V, H$}{Hilberträume}

\begin{Bemerkung}
    Mit $\skprod{\blank}{\blank}_{V}$ und $\skprod{\blank}{\blank}_{H}$ bezeichnen wir das Skalarprodukt auf $V$ respektive $H$.
    Weiterhin $\skprod{\blank}{\blank}_{V' \times V}$ wird auch für die duale Paarung auf $V' \times V$, die als die eindeutige stetige Fortsetzung von $\skprod{\blank}{\blank}_{H}$ definiert ist, verwendet.
    Insbesondere gilt für $u \in H \subset V'$ und $v \in V$ die Gleichheit
    \begin{equation}
        \skprod{u}{v}_{V' \times V} = \skprod{u}{v}_{H}.
    \end{equation}
\end{Bemerkung}
\nomenclature{$\skprod{\blank}{\blank}$}{Je nach Index Skalarprodukt oder duale Paarung}

\begin{Definition}[{{{\cite[Definition XVIII.2.4]{Dautray:1992by}}}}]
\label{definition:gl:bochner_raum_W}
    Sei $V \denseinclusion H \denseinclusion V'$ ein Gelfand-Tripel.
    Definiere den Raum $W(a, b; V, V')$ als
    \begin{equation}
        W(a, b; V, V') \deq \Set{u \in L_{2}(a, b; V) \given u' \in L_{2}(a, b; V')}
    \end{equation}
    wobei $u'$ im Sinne von \thref{definition:gl:schwache_zeitableitung} zu verstehen ist.
\end{Definition}
\nomenclature{$\denseinclusion$}{Dichte stetige Einbettung}

\begin{Bemerkung}
\label{bemerkung:gl:alternative_darstellung_bochner}
    Eine alternative Darstellung von $W(a, b; V, V')$ ist
    \begin{equation}
        W(a, b; V, V') = L_{2}(a, b; V) \cap H^{1}(a, b; V').
    \end{equation}
\end{Bemerkung}

\begin{Lemma}[{{{\cite[Proposition XVIII.2.6]{Dautray:1992by}}}}]
\label{lemma:gl:w_a_v_v_ist_hilbertraum}
    Mit dem Skalarprodukt
    \begin{equation}
        \skprod{u}{v}_{W(a, b; V, V')} \deq \skprod{u'}{v'}_{L_{2}(a, b; V')} +  \skprod{u}{v}_{L_{2}(a, b; V)}
    \end{equation}
    und der induzierten Norm
    \begin{equation}
        \begin{aligned}
            \norm{u}_{W(a, b; V, V')}
            \deq& \left( \int_{a}^{b} \norm{u'(t)}_{V'}^{2} \diff t + \int_{a}^{b} \norm{u(t)}_{V}^{2} \diff t \right)^{1/2}
            \\=& \left( \norm{u'}_{L_{2}(a, b; V')}^{2} + \norm{u}_{L_{2}(a, b; V)}^{2} \right)^{1/2}
        \end{aligned}
    \end{equation}
    ist $W(a, b; V, V')$ ein Hilbertraum.
\end{Lemma}

\begin{Definition}
\label{definition:gl:stetige_funktionen}
    Mit $\mathcal C([a, b]; X)$ bezeichnen wir die Menge aller bezüglich der Norm $\norm{f} = \sup_{t \in [a, b]} \norm{f(t)}_{X}$ stetigen Funktionen $f \colon [a, b] \to X$.
\end{Definition}

\begin{Satz}[{{{\cite[Theorem XVIII.2.1]{Dautray:1992by}}}}]
\label{satz:gl:einbettung_bochner_stetig}
    Seien $a, b \in \mathbb{R}$, dann stimmt jedes $u \in W(a, b)$ fast überall mit einer stetigen Funktion von $[a, b]$ nach $H$ überein.
    Genauer gilt
    \begin{equation}
        W(a, b; V, V') \hookrightarrow \mathcal C([a, b]; H).
    \end{equation}
\end{Satz}
\nomenclature{$\hookrightarrow$}{Stetige Einbettung}

\begin{Korollar}[{{{\cite[Remark XVIII.2.4]{Dautray:1992by}}}}]
\label{korollar:gl:spur_wohldefiniert}
    Sei $a, b \in \mathbb{R}$ und $u \in W(a, b; V, V')$.
    Dann sind $u(a), u(b) \in H$ wohldefiniert.
\end{Korollar}

% TODO: überprüfen, ob's so stimmt!
% TODO: Die Norm sieht grauenhaft aus...
\begin{Korollar}
\label{korollar:gl:einbettungskonstante_M_e}
    Seien $a, b \in \mathbb{R}$.
    Die Einbettungskonstante
    \begin{equation}
        \label{eq:gl:einbettungskonst_M_e}
        M_{e} \deq \sup_{\substack{u\in W(a, b; V, V')\\u \neq 0}} \frac{\norm{u(0)}_{H}}{\norm{u}_{W(a, b; V, V')}}
    \end{equation}
    ist gleichmäßig beschränkt in der Wahl $V \hookrightarrow H$ und hängt nur im Fall $T \to 0$ von $T$ ab.

    \begin{Beweis}
        Siehe \textcite[Beweis zu Theorem XVIII.2.1]{Dautray:1992by}.
    \end{Beweis}
\end{Korollar}

% section bochner_r_ume (end)

\section{Lineare Evolutionsgleichungen} % (fold)
\label{sec:lineare_evolutionsgleichungen}
\label{sec:allgemeine_problemstellung}
\label{sec:raum_zeit_variationsformulierung}

In diesem Abschnitt werden lineare Evolutionsgleichungen, eine bestimmte Unterart parabolischer partieller Differentialgleichungen, eingeführt.
Diese Einführung orientiert sich an \textcite{Lions:1971wp,Schwab:2009ec,Urban:2014kg}.

Zunächst definieren wir, was wir unter dem Begriff einer linearen Evolutionsgleichungen verstehen wollen.
Anschließend leiten wir eine Raum-Zeit-Variationsformulierung her und geben einen Satz an, der unter geeigneten Voraussetzungen Existenz und Eindeutigkeit einer Lösung dieser Variationsformulierung garantiert.

Wir befinden uns nun im Folgenden Setting:
wie in \thref{definition:gl:gelfand_tripel} seien $V$ und $H$ zwei separable Hilberträume und $(V, H, V')$ das daraus resultierende Gelfand-Tripel.
Wie zuvor verwenden wir $\skprod{\blank}{\blank}$ mit entsprechendem Index sowohl für die Skalarprodukte auf $V$ und $H$, als auch für die duale Paarung auf $V' \times V$.

Es sei $0 < T < \infty$ und damit $[0, T]$ ein endliches Zeitintervall.
Weiterhin sei für fast alle $t \in [0, T]$ eine Familie $a(\blank, \blank; t)$ von Bilinearformen
\begin{equation}
    \label{eq:gl:familie_bf_a}
    a(\blank, \blank; t) \colon V \times V \to \mathbb{R}, \quad (\eta, \zeta) \mapsto a(\eta, \zeta; t)
\end{equation}
gegeben.
Für alle $\eta, \zeta \in V$ sei die Abbildung $t \mapsto a(\eta, \zeta; t)$ messbar auf $[0, T]$ und erfülle die folgenden Bedingungen:

\begin{Annahme}
\label{annahme:eigenschaften_bf_a}
    \leavevmode
    \begin{thmenumerate}
        \item \emph{Stetigkeit.}
        Es existiert eine Konstante $0 < M_{a} < \infty$ mit
        \begin{equation}
            \label{eq:allgemeine_parabolische_pde:bf_stetig}
            \abs{a(\eta, \zeta; t)} \leq M_{a} \norm{\eta}_{V} \norm{\zeta}_{V} \quad \fa \eta, \zeta \in V
        \end{equation}
        für fast alle $t \in [0, T]$.
        \item \emph{G\r{a}rding-Ungleichung}.
        Es existieren Konstanten $\alpha > 0$ und $\lambda \geq 0$ mit
        \begin{equation}
            \label{eq:allgemeine_parabolische_pde:bf_garding}
            a(\eta, \eta; t) + \lambda \norm{\eta}_{H}^{2} \geq \alpha \norm{\eta}_{V}^{2} \quad \fa \eta \in V
        \end{equation}
        für fast alle $t \in [0, T]$.
    \end{thmenumerate}
\end{Annahme}

Nach dem Rieszschen Darstellungssatz, vergleiche \cite[Theorem \S{}22.1]{Halmos:1957vd}, wird durch die Familie $a(\blank, \blank; t)$ für fast alle $t \in [0, T]$ durch
\begin{equation}
    \skprod{A(t)\eta}{\zeta}_{H} = a(\eta, \zeta; t)
\end{equation}
eine Familie stetiger linearer Operatoren $A(t) \in \mathcal L(V, V')$ induziert.

Mit dieser Vorarbeit können wir nun definieren, was wir unter einer linearen Evolutionsgleichung verstehen wollen.

% TODO: Quelle?
\begin{Definition}
\label{definition:lineare_evolutionsgleichung}
    Seien $a(\blank, \blank; t)$ und $A(t)$ wie oben gegeben.
    Weiterhin sei ein \emph{Quellterm} $g \in L_{2}(0, T; V')$ und ein \emph{Anfangswert} $u_{0} \in H$ gegeben.
    Als \emph{lineare Evolutionsgleichung} bezeichnen wir die parabolische partielle Differentialgleichung
    \begin{equation}
        \label{eq:allgemeine_parabolische_pde}
        \begin{cases}
            u_{t}(t) + A(t) u(t) = g(t)     &\text{in}~V', \quad \text{für fast alle}~t \in I, \\
            u(0) = u_{0}                    &\text{in}~H.
        \end{cases}
    \end{equation}
\end{Definition}

\begin{Bemerkung}
\leavevmode
\begin{thmenumerate}
    \item Die Anfangswertbedingung $u(0) = u_{0}$ in $H$ ist wegen \thref{korollar:gl:spur_wohldefiniert} wohldefiniert.
    \item Ist die Bilinearform $a(\blank, \blank; t)$ respektive der zugehörige stetige lineare Operator $A(t)$ unabhängig von $t \in [0, T]$, dann sprechen wir von einer \emph{autonomen} linearen Evolutionsgleichung.
\end{thmenumerate}
\end{Bemerkung}

Als nächstes leiten wir eine Raum-Zeit-Variationsformulierung für~\eqref{eq:allgemeine_parabolische_pde} her.
Dazu werden geeignete Ansatz- und Testfunktionenräume benötigt.
Hier kommen nun die in \autoref{sec:bochner_r_ume} definierten Bochner-Räume zum Einsatz.

\begin{Definition}
    Als Ansatzfunktionenraum $\mathcal X$ bezeichnen wir den Raum $W(0, T; V, V')$ aus \thref{definition:gl:bochner_raum_W}.
    Es ist also
    \begin{equation}
        \label{eq:var_all_ansatzraum_x}
        \begin{aligned}
            \mathcal X &= L_{2}(0, T; V) \cap H^{1}(0, T; V')
            \\&= \Set*{ u \in L_{2}(0, T; V) \given u_{t} \in L_{2}(0, T; V') }
        \end{aligned}
    \end{equation}
    ausgestattet mit der Norm
    \begin{equation}
        \label{eq:var_all_ansatzraum_x_norm}
        \norm{u}_{\mathcal X} = \left( \norm{u}_{L_{2}{(0, T; V)}}^{2} + \norm{u_{t}}_{L_{2}{(0, T; V')}}^{2} \right)^{1 / 2}, \quad u \in \mathcal X.
    \end{equation}
    Der Testfunktionenraum $\mathcal Y$ sei gegeben durch
    \begin{equation}
        \label{eq:var_all_testraum_y}
        \mathcal Y = L_{2}(0, T; V) \oplus H
    \end{equation}
    mit der Norm
    \begin{equation}
        \label{eq:var_all_testraum_y_norm}
        \norm{v}_{\mathcal Y} = \left( \norm{v_{1}}_{L_{2}(0, T; V)}^{2} + \norm{v_{2}}_{H}^{2} \right)^{1 / 2}, \quad v = (v_{1}, v_{2}) \in \mathcal Y.
    \end{equation}
\end{Definition}

Beide Räume sind Hilberträume, $\mathcal X$ nach \thref{lemma:gl:w_a_v_v_ist_hilbertraum} und $\mathcal Y$ als direkte Summe zweier Hilberträume.

Um aus~\eqref{eq:allgemeine_parabolische_pde} eine Variationsformulierung zu erhalten, multiplizieren wir die lineare Evolutionsgleichung mit $v = (v_{1}, v_{2}) \in \mathcal Y$ und integrieren anschließend über das Zeitintervall $[0, T]$.
Dadurch ergibt sich folgende schwache Formulierung der linearen Evolutionsgleichung aus \thref{definition:lineare_evolutionsgleichung}:

\begin{Definition}
\label{definition:variationsformulierung}
    Seien $\mathcal X$ und $\mathcal Y$ wie in~\eqref{eq:var_all_ansatzraum_x} respektive~\eqref{eq:var_all_testraum_y}.
    Weiter sei ein Quellterm $g \in L_{2}(0, T; V')$ und ein Anfangswert $u_{0} \in H$ gegeben.
    Als \emph{Raum-Zeit-Variationsfor"-mu"-lie"-rung} der linearen Evolutionsgleichung~\eqref{eq:allgemeine_parabolische_pde} bezeichnen wir das folgende Problem:
    Finde ein $u \in \mathcal X$ mit
    \begin{equation}
        \label{eq:var_all_problem}
        b(u, v) = f(v) \quad \fa v \in \mathcal Y.
    \end{equation}
    Dabei ist $b \colon \mathcal X \times \mathcal Y \to \mathbb{R}$ die durch
    \begin{equation}
        \label{eq:var_all_bf_b}
        b(u, v) = \int_{0}^{T} \skprod{u_{t}(t)}{v_{1}(t)}_{H} + a(u(t), v_{1}(t); t) \diff t + \skprod{u(0)}{v_{2}}_{H}
    \end{equation}
    definierte Bilinearform und $f \colon \mathcal Y \to \mathbb{R}$ das durch
    \begin{equation}
        \label{eq:var_all_f}
        f(v) = \int_{0}^{T} \skprod{g(t)}{v_{1}(t)}_{H} \diff t + \skprod{u_{0}}{v_{2}}_{H}
    \end{equation}
    gegebene Funktional.
\end{Definition}

Es bleibt nun zu zeigen, dass obige Raum-Zeit-Variationsformulierung \emph{sachgemäß gestellt} (nach Hadamard [TODO]) ist, das heißt, das eine Lösung existiert, diese eindeutig ist und zudem stetig von den Eingangsdaten, also $f$, abhängt.
Dazu wird der folgende wichtige Satz, welcher in dieser oder ähnlicher Form bei \textcites[Theorem 2.1]{Babuska:1971fx}[Theorem 5.2.1]{Aziz:2014wf}[Theorem \S{}3.3.6]{Braess:2007wm} zu finden ist, verwendet.

\begin{Satz}[Banach-Ne{\v c}as-Babu{\v s}ka-Theorem]
    Seien $H_{1}$ und $H_{2}$ Hilberträume.
    Eine lineare Abbildung $A \colon H_{1} \to H_{2}'$ ist genau dann ein Isomorphismus, das heißt stetig invertierbar, wenn die zugehörige Bilinearform $a \colon H_{1} \times H_{2} \to \mathbb{R}$ die folgenden Bedingungen erfüllt:
    \begin{thmenumerate}
        \item \emph{Stetigkeit}.
        Es existiert ein $0 < C < \infty$ mit
        \begin{equation}
            \abs{a(u, v)} \leq C \norm{u}_{H_{1}} \norm{v}_{H_{2}} \quad \fa u \in H_{1},~v\in H_{2}.
        \end{equation}
        \item \emph{Inf-sup-Bedingung}.
        Es existiert ein $\alpha > 0$ mit
        \begin{equation}
            \inf_{u \in H_{1}} \sup_{v \in H_{2}} \frac{a(u, v)}{\norm{u}_{H_{1}} \norm{v}_{H_{2}}} \geq \alpha.
        \end{equation}
        \item Zu jedem $v \in H_{2}$, $v \neq 0$, existiert ein $u \in H_{1}$ mit
        \begin{equation}
            a(u, v) \neq 0.
        \end{equation}
    \end{thmenumerate}
    Ist dies der Fall und ist weiter ein Funktional $f \in H_{2}'$ gegeben, dann existiert eine eindeutige Lösung $\hat u \in H_{1}$ mit
    \begin{equation}
        a(\hat u, v) = f(v) \quad \fa v \in H_{2}
    \end{equation}
    und es gilt
    \begin{equation}
        \norm{\hat u}_{H_{1}} \leq \frac{1}{\alpha} \norm{f}_{H_{2}'}.
    \end{equation}
\end{Satz}

Für das Raum-Zeit-Variationsproblem aus \thref{definition:variationsformulierung} lässt sich die Tatsache, dass es sachgemäß gestellt ist, zu folgendem Satz zusammenfassen.

\begin{Satz}[{{\cite[Theorem 5.1]{Schwab:2009ec}}}]
\label{thm:schwab09:theorem51}
    Seien $\mathcal X$ und $\mathcal Y$ wie in~\eqref{eq:var_all_ansatzraum_x} respektive~\eqref{eq:var_all_testraum_y}.
    Sei weiter $B \colon \mathcal X \to \mathcal Y'$ definiert durch
    \begin{equation}
        \label{eq:var_all_gross_b}
        \skprod{B u}{v}_{\mathcal Y' \times \mathcal Y} = b(u, v), \quad u \in \mathcal X,~v \in \mathcal Y,
    \end{equation}
    wobei $b$ die Bilinearform aus~\eqref{eq:var_all_bf_b} sei.
    Dann ist $B$ stetig invertierbar.

    \begin{Beweis}
        Ein ausführlicher Beweis in dem die Bedingungen des Banach-Ne\v{c}as-Babu\v{s}ka-Theorems nachgewiesen werden, ist bei \textcite[Appendix A]{Schwab:2009ec} zu finden.
        % \begin{Beweis}
        %     Wir weisen die Bedingungen von \thref{satz:babuska-aziz} nach.

        %     Zunächst sei anzumerken, dass wir in~\eqref{eq:garding-inequality} ohne Einschränkung $\lambda = 0$ wählen können.
        %     Wähle
        %     \begin{equation}
        %         u(t) = \hat u(t) e^{\lambda t}, \quad v_{1}(t) = \hat v_{1}(t) e^{- \lambda t}, \quad g(t) = \hat g(t) e^{\lambda t},
        %     \end{equation}
        %     dann sieht man, dass $u$ die Gleichung~\eqref{eq:bilinearform} genau dann löst, wenn $\hat u$ die Gleichung
        %     \begin{equation}
        %         \label{eq:bilinearform_tmp}
        %         \begin{gathered}
        %             \int_{I} \skprod{\hat{u}_{t}(t)}{\hat{v}_{1}(t)}_{H} + \lambda \skprod{\hat{u}(t)}{\hat{v}_{1}(t)}_{H} + a(t; \hat{u}(t), \hat{v}_{1}(t)) \diff t + \skprod{\hat{u}(0)}{v_{2}}_{H}
        %                 \\= \int_{I} \skprod{\hat{g}(t)}{\hat{v}_{1}(t)}_{H} \diff t + \skprod{u_{0}}{v_{2}}_{H}
        %         \end{gathered}
        %     \end{equation}
        %     für alle $\hat{v} = (\hat{v}_{1}, v) \in \mathcal Y$ löst.

        %     \paragraph{Stetigkeit} % (fold)
        %     \label{par:stetigkeit}
        %     Betrachte für $u \in \mathcal X$ und $v = (v_{1}, v_{2}) \in \mathcal Y$ die Bilinearform $b(u, v)$.
        %     Nach Anwenden der Dreiecksungleichung erhalten wir
        %     \begin{equation}
        %         \label{eq:stetigkeit_zweiter_term}
        %         \abs{b(u, v)} = \int_{I} \abs{\skprod{u_{t}(t)}{v_{1}(t)}_{H}} + \abs{a(u(t), v_{1}(t))} \diff t + \abs{\skprod{u(0)}{v_{2}}_{H}}.
        %     \end{equation}
        %     Betrachten wir zunächst den hinteren Term, dann erhalten wir unter Verwendung der Cauchy-Schwarz-Ungleichung und der Einbettungs-Konstante $M_{e}$ die Abschätzung
        %     \begin{equation}
        %         \abs{\skprod{u(0)}{v_{2}}_{H}} \leq \norm{u(0)}_{H} \norm{v_{2}}_{H} \leq M_{e} \norm{u}_{X} \norm{v_{2}}_{H}.
        %     \end{equation}
        %     Widmen wir uns nun dem ersten Term und wenden ebenfalls die Cauchy-Schwarz-Ungleichung sowie die Stetigkeit von $a$ an, dann erhalten wir
        %     \begin{align}
        %         &\int_{I} \abs{\skprod{u_{t}(t)}{v_{1}(t)}_{H}} + \abs{a(u(t), v_{1}(t))} \diff t
        %         \\&\qquad
        %         \leq \int_{I} \norm{u_{t}(t)}_{H} \norm{v_{1}(t)}_{H} + M_{a} \norm{u(t)}_{H} \norm{v_{1}(t)}_{H} \diff t
        %         \\&\qquad
        %         \leq \int_{I} \max\{1, M_{a}\} \norm{v_{1}(t)}_{H} \left(  \norm{u_{t}(t)}_{H} + \norm{u(t)}_{H} \right) \diff t
        %         \intertext{mittels Hölder-Ungleichung lässt sich dies weiter abschätzen zu}
        %         &\qquad
        %         \leq \left( \int_{I} \max\{1, M_{a}\}^{2} \norm{v_{1}(t)}_{H}^{2} \diff t \right)^{\frac 12} \left( \int_{I} \left( \norm{u_{t}(t)}_{H} + \norm{u(t)}_{H} \right)^{2} \diff t \right)^{\frac 12},
        %         \intertext{und unter Verwendung der Youngschen-Ungleichung zu}
        %         &\qquad
        %         \leq \left( \int_{I} \max\{1, M_{a}\}^{2} \norm{v_{1}(t)}_{H}^{2} \diff t \right)^{\frac 12} \left( \int_{I} 2 \left( \norm{u_{t}(t)}_{H}^{2} + \norm{u(t)}_{H}^{2} \right) \diff t \right)^{\frac 12}
        %         \intertext{was nach Definition der verwendeten Normen auch geschrieben werden kann als}
        %         &\qquad
        %         = \sqrt{2 \max\{1, M_{a}^{2}\}} \norm{u}_{\mathcal X} \norm{v_{1}}_{L_{2}(I; V)}
        %     \end{align}
        %     Zusammen mit~\eqref{eq:stetigkeit_zweiter_term} liefert dies nach einer erneuten Anwendung der Cauchy-Schwarz-Ungleichung
        %     \begin{align}
        %     \abs{b(u, v)}
        %     &\leq \sqrt{2 \max\{1, M_{a}\}^{2}} \norm{u}_{\mathcal X} \norm{v_{1}}_{L_{2}(I; V)} + M_{e} \norm{u}_{X} \norm{v_{2}}_{H}
        %     \\
        %     &\leq \norm{u}_{\mathcal X} \left( \norm{v_{1}}_{L_{2}(I; V)}^{2} + \norm{v_{2}}_{H}^{2} \right)^{\frac 12} \left( 2 \max\{1, M_{a}\}^{2} + M_{e}^{2} \right)^{\frac 12}
        %     \\
        %     &= \sqrt{2 \max\{1, M_{a}^{2}\} + M_{e}^{2}} \norm{u}_{\mathcal X} \norm{v}_{\mathcal Y}.
        %     \end{align}
        %     Damit folgt die Stetigkeit.
        %     % paragraph stetigkeit (end)

        %     \paragraph{Inf-Sup-Bedingung} % (fold)
        %     \label{par:inf_sup_bedingung}

        %     % paragraph inf_sup_bedingung (end)
    \end{Beweis}
\end{Satz}

Aus dem Beweis des vorherigen Satzes ergeben sich zugleich auch Abschätzungen für die Operatornormen von $B$ und $B^{-1}$.

% TODO: Ungleichung der stetigen Abhängigkeit einbauen
\begin{Korollar}
    Unter den Gegebenheiten aus \thref{thm:schwab09:theorem51} gilt ferner
    \begin{equation}
        \label{eq:var_all_norm_B}
        \norm{B}_{\mathcal L(\mathcal X, \mathcal Y')} \leq \frac{\sqrt{2\max\Set{1, M_{a}^{2}} + M_{e}^{2}}}{\max\Set{\sqrt{1 + 2 \lambda^{2} \rho^{4}}, \sqrt{2}}}
    \end{equation}
    und
    \begin{equation}
        \label{eq:var_all_norm_B_inv}
        \norm{B^{-1}}_{\mathcal L(\mathcal Y', \mathcal X)} \leq \frac{e^{2 \lambda T} \max\Set{\sqrt{1 + 2 \lambda^{2} \rho^{4}}, \sqrt{2}} \sqrt{2 \max\Set{ \alpha^{-2}, 1} + M_{e}^{2}}}{\min\Set{\alpha M_{a}^{-2}, \alpha}}.
    \end{equation}
    Die Größen $M_{a}$, $\alpha$ und $\lambda$ stammen aus \thref{annahme:eigenschaften_bf_a},
    während die Konstanten $M_{e}$ und $\rho$ als die Einbettungskonstanten
    % TODO: genauer?
    \begin{equation}
        \label{eq:var_all_M_e}
        M_{e} = \sup_{0 \neq u \in \mathcal X} \frac{\norm{u(0)}_{H}}{\norm{u}_{\mathcal X}},
    \end{equation}
    vergleiche \thref{korollar:gl:einbettungskonstante_M_e}, beziehungsweise
    \begin{equation}
        \label{eq:var_all_rho}
        \rho = \sup_{0 \neq \eta \in V} \frac{\norm{\eta}_{H}}{\norm{\eta}_{V}}
    \end{equation}
    definiert sind.
    % Nach \thref{korollar:gl:einbettungskonst_M_e} ist $M_{e}$ gleichmäßig beschränkt in der Wahl von $V \hookrightarrow H$ und nur im Fall $T \to 0$ von $T$ abhängig.
\end{Korollar}
% subsection existenz_und_eindeutigkeit_von_l_sungen (end)


% section lineare_evolutionsgleichungen (end)

% section raum_zeit_variationsformulierung (end)

\section{Parametrisches Problem} % (fold)
\label{sec:parametrisches_problem}

TODO: Anpassen an Zeitabhängige lineare Operatoren bzw. Bilinearformen!

Wir beschäftigen uns nun mit einer parametrischen Variante der parabolischen partiellen Differentialgleichung~\eqref{eq:allgemeine_parabolische_pde}, folgern erneut Existenz und Eindeutigkeit von Lösungen und zeigen die analytische Abhängigkeit der Lösung vom Parameter.
Dieser Abschnitt basiert auf~\cite{Kunoth:2013ef}.

Bevor wir die gewünschten Aussagen für die später vorgestellte parametrische Variante der parabolischen partiellen Differentialgleichung~\eqref{eq:allgemeine_parabolische_pde} folgern können, müssen wir uns zunächst mit einer parametrischen Operatorgleichung beschäftigen.

Es seien $X$ und $Y$ zwei reflexive Banachräume und $\mathcal S \subset \mathbb{R}^{\mathbb{N}}$ bezeichne den sogenannten Parameterraum, ohne Einschränkung wählen wir $\mathcal S = {[-1, 1]}^{\mathbb{N}}$.
Für alle $\sigma \in \mathcal S$ sei nun durch $A(\sigma) \in \mathcal L(X, Y')$ ein stetiger linearer Operator gegeben.
Wir betrachten nun für $g \in Y'$ die parametrische Operatorgleichung
\begin{equation}
    \label{eq:allgemeine_parametrische_elliptische_pde}
    A(\sigma) u(\sigma) = g \quad \text{in}~Y'.
\end{equation}
Um Aussagen über diese Operatorgleichung treffen zu können, müssen wir zunächst die Abhängigkeit des Operators $A(\sigma)$ vom Parameter $\sigma$ konkretisieren.
Zunächst aber einige Notationen.

\begin{Bemerkung}
    Wir bezeichnen mit $\mathfrak F = \Set{ \nu \in \mathbb{N}^{\mathbb{N}}_{0} \given \abs{\nu} < \infty }$ die Menge aller Folgen nichtnegativer ganzer Zahlen mit endlichem Träger, das heißt nur endlich vielen Einträgen ungleich Null.
    % NOTE: Eventuell mehr definieren, siehe $\mathfrak n$ und $\mathfrak m$

    Sei $\nu \in \mathfrak F$ und $b \in \ell_{p}(\mathbb{N})$, $p > 0$, dann schreiben wir
    \begin{equation}
        b^{\nu} = \prod_{j = 1}^{\infty} b_{j}^{\nu_{j}}
    \end{equation}
    mit der Konvention $0^{0} = 1$.
    Wegen $\abs{\nu} < \infty$ ist dieses Produkt stets endlich.
\end{Bemerkung}


\begin{Annahme}[{{\cite[Assumption 1]{Kunoth:2013ef}}}]
\label{thm:kunoth:assumption1}
    Die parametrische Familie von Operatoren
    $\Set{ A(\sigma) \in \mathcal L(X, Y') \given \sigma \in \mathcal S }$ sei $\mathfrak p$-regulär für ein $0 < \mathfrak p \leq 1$, das heißt
    \begin{thmenumerate}
        \item $A(\sigma) \in \mathcal L(X, Y')$ sei stetig invertierbar für alle $\sigma \in \mathcal S$ mit gleichmäßig beschränktem Inversen $A{(\sigma)}^{-1} \in \mathcal L(Y', X)$, das heißt es existiert ein $C_{0} > 0$ mit
        \begin{equation}
            \sup_{\sigma \in \mathcal S} \norm{A{(\sigma)}^{-1}}_{\mathcal L(Y', X)} \leq C_{0},
        \end{equation}
        \item für jedes feste $\sigma \in \mathcal S$ seien die Operatoren $A(\sigma)$ analytisch bezüglich $\sigma$, konkret existiert eine nichtnegative Folge $b = (b_{j})_{j \geq 1} \in \ell_{\mathfrak p}(\mathbb{N})$ so dass
        \begin{equation}
            \sup_{\sigma \in \mathcal S} \norm{(A{(0)}^{-1})(\partial^{\nu}_{\sigma} A(\sigma))}_{\mathcal L(X, X)} \leq C_{0} b^{\nu}
        \end{equation}
        für alle $\nu \in \mathfrak F \setminus \{ 0 \}$ gilt, wobei $\partial^{\nu}_{\sigma} A(\sigma) = \frac{\partial^{\nu_{1}}}{\partial \sigma_{1}} \frac{\partial^{\nu_{2}}}{\partial \sigma_{2}} \cdots A(\sigma)$ sei.
    \end{thmenumerate}
\end{Annahme}

Der für uns wichtigste Fall der parametrischen Abhängigkeit von $A(\sigma)$ ist die affine Abhängigkeit, das heißt es existiert eine Familie von Operatoren $\Set{\hat A} \cup \Set{ A_{j} }_{j \geq 1} \subset \mathcal L(X, Y')$, so dass
\begin{equation}
    \label{eq:all_affiner_operator}
    A(\sigma) = \hat A + \sum_{j = 1}^{\infty} \sigma_{j} A_{j} \qquad\fa \sigma \in \mathcal S
\end{equation}
gilt.
Wir bezeichnen mit $\hat a, a_{j} \colon X \times Y \to \mathbb{R}$ die zu $\hat A$ respektive $A_{j}$, $j \geq 1$, zugehörigen Bilinearformen, also
\begin{equation}
    \label{eq:allg_affine_bf}
    \begin{aligned}
    \hat a(\eta, \zeta) &= \skprod{\hat A \eta}{\zeta}_{Y' \times Y}
    \\a_{j}(\eta, \zeta) &= \skprod{A_{j} \eta}{\zeta}_{Y' \times Y}
    \end{aligned}
\end{equation}
für $\eta \in X$, $\zeta \in Y$.

Um die Konvergenz von~\eqref{eq:all_affiner_operator} sicherzustellen, fordern wir:

\begin{Annahme}[{{\cite[Assumption 2]{Kunoth:2013ef}}}]
\label{thm:kunoth:assumption2}
    Die Operatorfamilie $\Set{\hat A} \cup \Set{ A_{j} }_{j \geq 1}$ erfülle folgende Eigenschaften:
    \begin{thmenumerate}
        \item Der \emph{Mean Field}-Operator $\hat A \in \mathcal L(X, Y')$ sei stetig invertierbar, das heißt es existiert ein $\gamma_{0} > 0$ mit
        \begin{subequations}\label{eq:kunoth:ass2_gamma_0}
            \begin{align}
                \label{eq:kunoth:ass2_gamma_0_a}
                \inf_{0 \neq u \in X} \sup_{0 \neq v \in Y} \frac{\hat a(u, v)}{\norm{u}_{X} \norm{v}_{Y}} \geq \gamma_{0}
                \intertext{und}
                \label{eq:kunoth:ass2_gamma_0_b}
                \inf_{0 \neq v \in Y} \sup_{0 \neq u \in X} \frac{\hat a(u, v)}{\norm{u}_{X} \norm{v}_{Y}} \geq \gamma_{0}.
            \end{align}
        \end{subequations}
        \item Die \emph{Fluctuation}-Operatoren $\Set{ A_{j} }_{j \geq 1}$ seien \emph{klein} relativ zu $\hat A$ im folgenden Sinne: es existiert eine Konstante $0 < \kappa < 1$ so dass
        \begin{equation}
            \label{eq:kunoth:ass2_abs_reihe}
            \sum_{j = 1}^{\infty} \norm{A_{j}}_{\mathcal L(X, Y')} \leq \kappa \gamma_{0}
        \end{equation}
        gilt.
    \end{thmenumerate}
\end{Annahme}

\begin{Korollar}[{{\cite[Corollary 3]{Kunoth:2013ef}}}]
\label{thm:kunoth:corollary3}
    Die affin parametrische Operatorfamilie $\Set{\hat A} \cup \Set{ A_{j} }_{j \geq 1}$ erfülle \thref{thm:kunoth:assumption2}, dann wird auch \thref{thm:kunoth:assumption1} mit $\mathfrak p = 1$ und
    \begin{equation}
        C_{0} = \frac{1}{(1 - \kappa) \gamma_{0}}, \qquad b_{j} = \frac{\norm{A_{j}}_{\mathcal L(X, Y')}}{(1 - \kappa) \gamma_{0}}, \quad \fa j \geq 1,
    \end{equation}
    erfüllt.
\end{Korollar}

\begin{Satz}[{{\cite[Theorem 4]{Kunoth:2013ef}}}]
\label{thm:kunoth:theorem4}
    Die parametrische Familie $\Set{ A(\sigma) \in \mathcal L(X, Y') \given \sigma \in \mathcal S }$ erfülle \thref{thm:kunoth:assumption1} für ein $0 < \mathfrak p \leq 1$.
    Dann existiert für jedes $g \in Y'$ und jedes $\sigma \in \mathcal S$ eine eindeutige Lösung $u(\sigma) \in X$ der parametrischen Operatorgleichung
    \begin{equation}
        A(\sigma) u(\sigma) = g \quad \text{in}~Y'.
    \end{equation}
    Die parametrische Familie von Lösungen $u(\sigma)$ hängt analytisch vom Parameter $\sigma$ ab und die partiellen Ableitungen von $u(\sigma)$ erfüllen
    \begin{equation}
        \label{eq:kunoth:schranke_part_abl}
        \sup_{\sigma \in \mathcal S} \norm{(\partial^{\nu}_{\sigma} u)(\sigma)}_{X} \leq C_{0} \norm{g}_{Y'} \abs{\nu}! \tilde{b}^{\nu}
    \end{equation}
    für alle $\nu \in \mathfrak F$, wobei die Folge $\tilde{b} = (\tilde{b}_{j})_{j \geq 1} \in \ell_{\mathfrak p}(\mathbb{N})$ definiert ist durch
    \begin{equation}
        \tilde{b}_{j} = \frac{b_{j}}{\ln 2} \qquad \text{für alle j} \in \mathbb{N}.
    \end{equation}
\end{Satz}

Aus diesen Ergebnissen erhalten wir nun die gewünschten Aussagen für die parametrische Variante von~\eqref{eq:allgemeine_parabolische_pde}, diese müssen wir aber nun zunächst konkretisieren.
Wir arbeiten nun wieder im Setting aus \autoref{sec:allgemeine_problemstellung}.

Für $\sigma \in \mathcal S$ sei $A(\sigma) \in \mathcal L(V, V')$ ein stetiger linearer Operator und $a(\blank, \blank; \sigma) \colon V \times V \to \mathbb{R}$ die zugehörige Bilinearform, also $a(\eta, \zeta; \sigma) = \skprod{A(\sigma) \eta}{\zeta}_{V' \times V}$ für $\eta, \zeta \in V$.
Die Bilinearform $a(\blank, \blank; \sigma)$ erfülle \thref{annahme:eigenschaften_bf_a} gleichmäßig in $\sigma$, sie sei also stetig und erfülle eine G\r{a}rding-Ungleichung, das heißt es existieren von $\sigma$ unabhängige Konstanten $0 < M_{a} < \infty$, $\alpha > 0$ und $\lambda \geq 0$ mit
\begin{equation}
    \label{eq:allgemeine_parabolische_pde:bf_stetig_parametrisch}
    \abs{a(\eta, \zeta; \sigma)} \leq M_{a} \norm{\eta}_{V} \norm{\zeta}_{V} \quad \fa \eta, \zeta \in V
\end{equation}
und
\begin{equation}
    \label{eq:allgemeine_parabolische_pde:bf_garding_parametrisch}
    a(\eta, \eta; \sigma) + \lambda \norm{\eta}_{H}^{2} \geq \alpha \norm{\eta}_{V}^{2} \quad \fa \eta \in V.
\end{equation}

Analog zu der Herleitung in \autoref{sec:raum_zeit_variationsformulierung} erhalten wir das parametrische Variationsproblem:

Gegeben ein $g \in L_{2}(I; V')$ und $u_{0} \in H$. Finde für alle $\sigma \in \mathcal S$ ein $u(\sigma) \in \mathcal X$ mit
\begin{equation}
    \label{eq:var_all_problem_parametrisch}
    b(u(\sigma), v; \sigma) = f(v) \quad \fa v \in \mathcal Y,
\end{equation}
wobei $f$ von $g$ und $u_{0}$ abhängt.
Dabei ist $b(\blank, \blank; \sigma) \colon \mathcal X \times \mathcal Y \to \mathbb{R}$ eine Bilinearform definiert durch
\begin{equation}
    \label{eq:var_all_bf_b_parametrisch}
    b(u, v; \sigma) = \int_{I} \skprod{u_{t}(t)}{v_{1}(t)}_{H} + a(u(t), v_{1}(t); \sigma) \diff t + \skprod{u(0)}{v_{2}}_{H},
\end{equation}
und $f(\blank) \colon \mathcal Y \to \mathbb{R}$ das durch
\begin{equation}
    \label{eq:var_all_f_parametrisch}
    f(v) = \int_{I} \skprod{g(t)}{v_{1}(t)}_{H} \diff t + \skprod{u_{0}}{v_{2}}_{H}
\end{equation}
gegebene Funktional.

\begin{Satz}[{{\cite[Theorem 21]{Kunoth:2013ef}}}]
\label{thm:kunoth:theorem21}
    Seien $\mathcal X$ und $\mathcal Y$ gegeben wie in~\eqref{eq:var_all_ansatzraum_x} respektive~\eqref{eq:var_all_testraum_y} und die Familie von Operatoren $\Set{ A(\sigma) \in \mathcal L(V, V') \given \sigma \in \mathcal S }$ erfülle \thref{thm:kunoth:assumption1} für ein $0 < \mathfrak p \leq 1$.
    Für jedes $\sigma \in \mathcal S$ sei $B(\sigma) \in \mathcal L(\mathcal X, \mathcal Y')$ definiert durch
    \begin{equation}
        \label{eq:var_all_gross_b_parametrisch}
        \skprod{B(\sigma) u}{v}_{\mathcal Y' \times \mathcal Y} = b(u, v; \sigma), \quad u \in \mathcal X,~y \in \mathcal Y,
    \end{equation}
    mit $b(\blank, \blank; \sigma)$ wie in~\eqref{eq:var_all_bf_b_parametrisch}.
    Dann ist $B(\sigma)$ für jedes $\sigma \in \mathcal S$ stetig invertierbar und es existieren Konstanten $0 < \beta_{1} \leq \beta_{2} < \infty$ mit
    \begin{equation}
        \label{eq:var_all_norm_B_und_B_inv_parametrisch}
        \sup_{\sigma \in \mathcal S} \norm{B(\sigma)}_{\mathcal L(\mathcal X, \mathcal Y')} \leq \beta_{2} \quad \text{und} \quad  \sup_{\sigma \in \mathcal S} \norm{B(\sigma)^{-1}}_{\mathcal L(\mathcal Y', \mathcal X)} \leq \frac{1}{\beta_{1}}.
    \end{equation}
    Die parametrische Familie von Operatoren $\Set{ B(\sigma) \in \mathcal L(\mathcal X, \mathcal Y') \given \sigma \in \mathcal S }$ erfüllt \thref{thm:kunoth:assumption1} mit dem selben Regularitätsparameter $\mathfrak p$, die parametrische Familie von Lösungen $u(\sigma)$ hängt analytisch von $\sigma$ ab und erfüllt die \emph{a priori}-Abschätzung
    \begin{equation}
        \label{eq:var_all_a_priori_schranke}
        \sup_{\sigma \in \mathcal S} \norm{(\partial^{\nu}_{\sigma} u)(\sigma)}_{\mathcal X} \leq C_{0} \norm{f}_{\mathcal Y'} \abs{\nu}! \tilde{b}^{\nu}
    \end{equation}
    für alle $\nu \in \mathfrak F$, wobei $f$ wie in~\eqref{eq:var_all_f_parametrisch} gegeben ist.

    \begin{Beweis}
        TODO:\@ Bedingungen von \thref{thm:kunoth:assumption1} nachrechnen.
        Zu (i): Folgt aus \thref{thm:schwab09:theorem51}, da $M_{a}, \alpha, \lambda$ unabhänging von $\sigma$.
        Zu (ii): Folgt aus nachfolgendem \thref{lemma:norm_B_beschraenkt_durch_norm_A}.
    \end{Beweis}
\end{Satz}

\begin{Lemma}
\label{lemma:norm_B_beschraenkt_durch_norm_A}
    Sei $\sigma \in \mathcal S$ und $\nu \in \mathfrak F \setminus \Set{ 0 }$, dann gilt
    \begin{equation}
        \norm{\partial^{\nu}_{\sigma} B(\sigma)}_{\mathcal L(\mathcal X, \mathcal Y')}
        \leq
        \norm{\partial^{\nu}_{\sigma} A(\sigma)}_{\mathcal L(V, V')}
    \end{equation}

    % \begin{Beweis}
        % TODO: Moo.
    % \end{Beweis}
\end{Lemma}

% subsubsection aus_cite_kunoth_2013ef (end)

% section parametrisches_problem (end)


% chapter grundlagen (end)


