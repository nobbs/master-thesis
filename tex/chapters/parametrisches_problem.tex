%!TEX root = ../main.tex

\chapter{Parametrisches Problem} % (fold)
\label{cha:parametrisches_problem}

% TODO: Anpassen an Zeitabhängige lineare Operatoren bzw. Bilinearformen!

In diesem Kapitel liegt das Augenmerk erneut auf der linearen Evolutionsgleichung \eqref{eq:allgemeine_parabolische_pde}, diesmal aber mit der Erweiterung, dass der lineare Operator $A(t)$ zusätzlich von einem Parameter $\sigma$ abhängt.

Zunächst konkretisieren wir diese Parameterabhängigkeit für einen linearen Operator $A$, betrachten dann eine parametrische lineare Operatorgleichung, leiten Regularitätsergebnisse für diese her und übertragen diese anschließend auf die Raum-Zeit-Variationsformulierung einer parametrischen linearen Evolutionsgleichung.
Dabei orientieren wir uns hauptsächlich an den Arbeiten von \textcite{Kunoth:2013ef,Cohen:2010kz}.

\section{Parametrische Operatorgleichung} % (fold)
\label{sec:parametrische_operatorgleichung}

Seien $X$ und $Y$ zwei reflexive Banachräume.
Weiter sei $\mathcal S \subset \mathbb{R}^{\mathbb{N}}$ der sogenannte Parameterraum.
Der Einfachheit halber wählen wir $\mathcal S = [-1, 1]^{\mathbb{N}}$. % TODO: warum?

Wir betrachten parametrische Familien stetiger linearer Operatoren $A(\sigma) \in \mathcal L(X, Y')$ mit $\sigma \in \mathcal S$.
Folgende lineare Operatorgleichung ist für uns von Interesse:
Sei ein $g \in Y'$ gegeben.
Finde für alle $\sigma \in \mathcal S$ eine Lösung $u(\sigma) \in X$ von
\begin{equation}
    \label{eq:allgemeine_parametrische_elliptische_pde}
    A(\sigma) u(\sigma) = g \quad \text{in}~Y'.
\end{equation}
Wie zuvor sei $a(\blank, \blank; \sigma) \colon X \times Y \to \mathbb{R}$ die zugehörige Bilinearform.

Zunächst einige notationelle Vorbemerkungen.
\begin{Bemerkung}
    Wir bezeichnen mit $\mathfrak F = \Set{ \nu \in \mathbb{N}^{\mathbb{N}}_{0} \given \abs{\nu} < \infty }$ die Menge aller Folgen nichtnegativer ganzer Zahlen mit endlichem Träger, das heißt nur endlich vielen Einträgen ungleich Null.
    % NOTE: Eventuell mehr definieren, siehe $\mathfrak n$ und $\mathfrak m$

    Sei $\nu \in \mathfrak F$ und $b \in \ell_{p}(\mathbb{N})$, $p > 0$, dann schreiben wir
    \begin{equation}
        b^{\nu} = \prod_{j = 1}^{\infty} b_{j}^{\nu_{j}}
    \end{equation}
    mit der Konvention $0^{0} = 1$.
    Wegen $\abs{\nu} < \infty$ ist dieses Produkt stets endlich.
\end{Bemerkung}

Für die nachfolgenden Regularitätsaussagen über die Lösung $u(\sigma)$ von \eqref{eq:allgemeine_parametrische_elliptische_pde} benötigen wir Regularität der Operatorfamilie $A(\sigma)$ bezüglich $\sigma \in \mathcal S$.
Konkret fordern wir:
\begin{Annahme}[{{\cite[Assumption 1]{Kunoth:2013ef}}}]
\label{thm:kunoth:assumption1}
    Die parametrische Familie von Operatoren
    $\Set{ A(\sigma) \in \mathcal L(X, Y') \given \sigma \in \mathcal S }$ sei eine $\mathfrak p$-reguläre Operatorfamilie für ein $0 < \mathfrak p \leq 1$, das heißt,
    \begin{thmenumerate}
        \item $A(\sigma) \in \mathcal L(X, Y')$ sei stetig invertierbar für alle $\sigma \in \mathcal S$ mit gleichmäßig beschränktem Inversen $A{(\sigma)}^{-1} \in \mathcal L(Y', X)$, das heißt, es existiert ein $C_{0} > 0$ mit
        \begin{equation}
            \sup_{\sigma \in \mathcal S} \norm{A{(\sigma)}^{-1}}_{\mathcal L(Y', X)} \leq C_{0},
        \end{equation}
        \item für jedes feste $\sigma \in \mathcal S$ seien die Operatoren $A(\sigma)$ analytisch bezüglich $\sigma$.
        Konkret existiert eine nichtnegative Folge $b = (b_{j})_{j \in \mathbb{N}} \in \ell_{\mathfrak p}(\mathbb{N})$, so dass
        \begin{equation}
            \sup_{\sigma \in \mathcal S} \norm{(A{(0)})^{-1}(\partial^{\nu}_{\sigma} A(\sigma))}_{\mathcal L(X, X)} \leq C_{0} b^{\nu}
        \end{equation}
        für alle $\nu \in \mathfrak F \setminus \{ 0 \}$ gilt.
        Dabei sei $\partial^{\nu}_{\sigma} A(\sigma) \deq \frac{\partial^{\nu_{1}}}{\partial \sigma_{1}} \frac{\partial^{\nu_{2}}}{\partial \sigma_{2}} \cdots A(\sigma)$.
    \end{thmenumerate}
\end{Annahme}

Die bisherigen Anforderungen an $A(\sigma)$ decken einen noch sehr weiten Bereich ab.
Wir beschränken uns in dieser Arbeit aus praktischen Gründen ausschließlich auf den folgenden Fall, der affin parametrischen Operatoren.

\begin{Definition}
    Sei $\Set{ A(\sigma) \in \mathcal L(X, Y') \given \sigma \in \cal S }$ eine parametrische Operatorfamilie.
    Wir nennen $A(\sigma)$ einen \emph{affin parametrischen Operator}, falls eine Familie von Operatoren $\Set{ \hat A, A_{j} \given j \in \mathbb{N} } \subset \cal L(X, Y')$ existiert, so dass
    \begin{equation}
        \label{eq:all_affiner_operator}
        A(\sigma) = \hat A + \sum_{j = 1}^{\infty} \sigma_{j} A_{j} \qquad\fa \sigma \in \mathcal S
    \end{equation}
    gilt.
\end{Definition}

Seien $\hat a, a_{j} \colon X \times Y \to \mathbb{R}$ die durch den Rieszschen Darstellungssatz von $\hat A$ respektive $A_{j}$ induzierten Bilinearformen, das heißt also,
\begin{equation}
    \label{eq:allg_affine_bf}
    \begin{aligned}
    \hat a(\eta, \zeta) &= \skprod{\hat A \eta}{\zeta}_{Y' \times Y}
    \\
    a_{j}(\eta, \zeta) &= \skprod{A_{j} \eta}{\zeta}_{Y' \times Y}, \quad j \in \mathbb{N},
    \end{aligned}
\end{equation}
für $\eta \in X$, $\zeta \in Y$.

Um die Wohldefiniertheit von $A(\sigma)$, das heißt Konvergenz von \eqref{eq:all_affiner_operator}, sicherzustellen, stellen wir folgende Bedingungen:
\begin{Annahme}[{{\cite[Assumption 2]{Kunoth:2013ef}}}]
\label{thm:kunoth:assumption2}
    Die Operatorfamilie $\Set{\hat A, A_{j} \given j \in \mathbb{N}}$ erfülle folgende Eigenschaften:
    \begin{thmenumerate}
        \item Der \emph{Mean Field}-Operator $\hat A \in \mathcal L(X, Y')$ sei stetig invertierbar, das heißt, es existiert ein $\gamma_{0} > 0$ mit
        \begin{subequations}\label{eq:kunoth:ass2_gamma_0}
            \begin{align}
                \label{eq:kunoth:ass2_gamma_0_a}
                \inf_{0 \neq u \in X} \sup_{0 \neq v \in Y} \frac{\hat a(u, v)}{\norm{u}_{X} \norm{v}_{Y}} \geq \gamma_{0}
                \intertext{und}
                \label{eq:kunoth:ass2_gamma_0_b}
                \inf_{0 \neq v \in Y} \sup_{0 \neq u \in X} \frac{\hat a(u, v)}{\norm{u}_{X} \norm{v}_{Y}} \geq \gamma_{0}.
            \end{align}
        \end{subequations}
        \item Die \emph{Fluctuation}-Operatoren $\Set{ A_{j} }_{j \geq 1}$ seien \emph{klein} relativ zu $\hat A$ im folgenden Sinne: es existiert eine Konstante $0 < \kappa < 1$ so dass
        \begin{equation}
            \label{eq:kunoth:ass2_abs_reihe}
            \sum_{j = 1}^{\infty} \norm{A_{j}}_{\mathcal L(X, Y')} \leq \kappa \gamma_{0}
        \end{equation}
        gilt.
    \end{thmenumerate}
\end{Annahme}

Unter diesen Bedingungen liefert das Banach-Ne{\v c}as-Babu{\v s}ka-Theorem, \thref{satz:gl:bnb_theorem}, die stetige Invertierbarkeit von $A(\sigma)$ aus \eqref{eq:all_affiner_operator} gleichmäßig in $\sigma$.

\begin{Satz}[{{\cite[Theorem 2]{Kunoth:2013ef}}}]
    Der affin parametrische Operator $A(\sigma)$ erfülle \thref{thm:kunoth:assumption2}.
    Dann ist $A(\sigma)$ für alle $\sigma \in \mathcal S$ stetig invertierbar.

    Konkret gilt
    \begin{equation}
        \inf_{u \in H_{1}} \sup_{v \in H_{2}} \frac{a(u, v)}{\norm{u}_{H_{1}} \norm{v}_{H_{2}}} \geq (1 - \kappa) \gamma_{0} > 0 \quad \fa \sigma \in \mathcal S
    \end{equation}
    und
    \begin{equation}
        \inf_{v \in H_{2}} \sup_{u \in H_{1}} \frac{a(u, v)}{\norm{u}_{H_{1}} \norm{v}_{H_{2}}} \geq (1 - \kappa) \gamma_{0} > 0 \quad \fa \sigma \in \mathcal S.
    \end{equation}

    Ist ferner ein $g \in Y'$ gegeben, dann existiert für jedes $\sigma \in \mathcal S$ ein $\hat u(\sigma) \in X$ mit
    \begin{equation}
        a(\hat u(\sigma), v; \sigma) = \skprod{g}{v}_{Y' \times Y} \quad \fa v \in Y
    \end{equation}
    und es gilt die A-Priori-Abschätzung
    \begin{equation}
        \sup_{\sigma \in \mathcal S} \norm{\hat u(\sigma)}_{X} \leq \frac{\norm{g}_{Y'}}{(1 - \kappa) \gamma_{0}}.
    \end{equation}

    \begin{Beweis}
        Nachrechnen der beiden inf-sup-Bedingungen unter Verwendung der affinen Zerlegung von $A(\sigma)$ und anschließendes Anwenden des Banach-Ne{\v c}as-Babu{\v s}ka-Theorems liefert die gewünschten Aussagen.
    \end{Beweis}
\end{Satz}

\begin{Korollar}[{{\cite[Corollary 3]{Kunoth:2013ef}}}]
\label{thm:kunoth:corollary3}
    Die affin parametrische Operatorfamilie $\Set{\hat A, A_{j} \given j \in \mathbb{N}}$ erfülle \thref{thm:kunoth:assumption2}, dann wird auch \thref{thm:kunoth:assumption1} mit $\mathfrak p = 1$ und
    \begin{equation}
        C_{0} = \frac{1}{(1 - \kappa) \gamma_{0}}, \qquad b_{j} = \frac{\norm{A_{j}}_{\mathcal L(X, Y')}}{(1 - \kappa) \gamma_{0}} \quad \fa j \in \mathbb{N},
    \end{equation}
    erfüllt.
\end{Korollar}

Weiter erhält man unter den Bedingungen aus \thref{thm:kunoth:assumption1} folgendes Regularitätsergebnis bezüglich des Parameters $\sigma$.

\begin{Satz}[{{\cite[Theorem 4]{Kunoth:2013ef}}}]
\label{thm:kunoth:theorem4}
    Die parametrische Familie $\Set{ A(\sigma) \in \mathcal L(X, Y') \given \sigma \in \mathcal S }$ erfülle \thref{thm:kunoth:assumption1} für ein $0 < \mathfrak p \leq 1$.
    Dann existiert für jedes $g \in Y'$ und jedes $\sigma \in \mathcal S$ eine eindeutige Lösung $u(\sigma) \in X$ der parametrischen Operatorgleichung
    \begin{equation}
        A(\sigma) u(\sigma) = g \quad \text{in}~Y'.
    \end{equation}

    Die parametrische Familie von Lösungen $u(\sigma)$ hängt analytisch vom Parameter $\sigma$ ab und die partiellen Ableitungen von $u(\sigma)$ erfüllen
    \begin{equation}
        \label{eq:kunoth:schranke_part_abl}
        \sup_{\sigma \in \mathcal S} \norm{(\partial^{\nu}_{\sigma} u)(\sigma)}_{X} \leq C_{0} \norm{g}_{Y'} \abs{\nu}! \tilde{b}^{\nu}
    \end{equation}
    für alle $\nu \in \mathfrak F$, wobei die Folge $\tilde{b} = (\tilde{b}_{j})_{j \geq 1} \in \ell_{\mathfrak p}(\mathbb{N})$ definiert ist durch
    \begin{equation}
        \tilde{b}_{j} = \frac{b_{j}}{\ln 2} \qquad \text{für alle j} \in \mathbb{N}.
    \end{equation}

    \begin{Beweis}
        TODO: etwas dazu sagen.
    \end{Beweis}
\end{Satz}

% section parametrische_operatorgleichung (end)

\section{Parametrische lineare Evolutionsgleichung} % (fold)
\label{sec:parametrische_lineare_evolutionsgleichung}

Dieser Abschnitt soll nun dazu dienen, aufbauend auf \autoref{sec:lineare_evolutionsgleichungen} eine parametrische lineare Evolutionsgleichung zu definieren und anschließend die Regularitätsergebnisse aus dem vorherigen Abschnitt auf diese zu übertragen.

Wir wiederholen kurz das Setting aus \autoref{sec:lineare_evolutionsgleichungen}, in dem wir hier erneut arbeiten.
Seien $V$ und $H$ separable Hilberträume mit einer dichten stetigen Einbettung von $V$ in $H$ und $(V, H, V')$ sei das zugehörige Gelfand-Tripel.
Weiter seien ein $0 < T < \infty$ und ein endliches Zeitintervall $[0, T]$ gegeben.

Wir bezeichnen $\mathcal S = [-1, 1]^{\mathbb{N}}$ weiterhin als Parameterraum.
Es sei für fast alle $t \in [0, T]$ und für alle $\sigma \in \mathcal S$ eine Familie von Bilinearformen
\begin{equation}
    a(\blank, \blank; \sigma, t) \colon V \times V \to \mathbb{R}, \quad (\eta, \zeta) \mapsto a(\eta, \zeta; \sigma, t)
\end{equation}
gegeben, so dass $t \mapsto a(\eta, \zeta; \sigma, t)$ für alle $\sigma \in \mathcal S$ messbar auf $[0, T]$ ist.
Analog zu \thref{annahme:eigenschaften_bf_a} fordern wir diesmal für den Rest dieses Abschnitts:
\begin{Annahme}
\label{annahme:pp:eigenschaften_bf_a}
    \leavevmode
    \begin{thmenumerate}
        \item \emph{Stetigkeit.}
        Es existiert eine Konstante $0 < M_{a} < \infty$, so dass
        \begin{equation}
            \label{eq:allgemeine_parabolische_pde:bf_stetig}
            \abs{a(\eta, \zeta; \sigma, t)} \leq M_{a} \norm{\eta}_{V} \norm{\zeta}_{V} \quad \fa \eta, \zeta \in V
        \end{equation}
        für fast alle $t \in [0, T]$ und alle $\sigma \in \mathcal S$ gilt.
        \item \emph{G\r{a}rding-Ungleichung}.
        Es existieren Konstanten $\alpha > 0$ und $\lambda \geq 0$ mit
        \begin{equation}
            \label{eq:allgemeine_parabolische_pde:bf_garding}
            a(\eta, \eta; \sigma, t) + \lambda \norm{\eta}_{H}^{2} \geq \alpha \norm{\eta}_{V}^{2} \quad \fa \eta \in V
        \end{equation}
        für fast alle $t \in [0, T]$ und alle $\sigma \in \mathcal S$.
    \end{thmenumerate}
\end{Annahme}

Unter diesen Voraussetzungen existiert nach dem Rieszschen Darstellungssatz für jedes $\sigma \in \mathcal S$ und fast alle $t \in [0, T]$ ein stetiger linearer Operator $A(\sigma, t) \in \mathcal L(V, V')$ und es gilt für alle $\sigma \in \mathcal S$ die Gleichheit
\begin{equation}
    \skprod{A(\sigma, t) \eta}{\zeta} = a(\eta, \zeta; \sigma, t) \quad \eta, \zeta \in V.
\end{equation}

Vollkommen analog zur Herleitung der Raum-Zeit-Variationsformulierung in \autoref{sec:raum_zeit_variationsformulierung} erhalten wir damit das folgende parametrische Raum-Zeit-Variationsproblem:

\begin{Definition}
\label{definition:pp:variationsformulierung}
    Seien $\mathcal X$ und $\mathcal Y$ wie in \thref{definition:gl:ansatz_und_testraum}.
    Als \emph{parametrische Raum-Zeit-Variationsfor"-mu"-lie"-rung}
    %der linearen Evolutionsgleichung~\eqref{eq:allgemeine_parabolische_pde}
    bezeichnen wir das folgende Problem:

    Seien ein Quellterm $g \in L_{2}(0, T; V')$ und ein Anfangswert $u_{0} \in H$ gegeben.
    Finde für alle $\sigma \in \mathcal S$ ein $u(\sigma) \in \mathcal X$ mit
    \begin{equation}
        \label{eq:pp:var_all_problem}
        b(u(\sigma), v; \sigma) = f(v) \quad \fa v \in \mathcal Y.
    \end{equation}
    Dabei ist $b \colon \mathcal X \times \mathcal Y \to \mathbb{R}$ die durch
    \begin{equation}
        \label{eq:pp:var_all_bf_b}
        b(u, v; \sigma) = \int_{0}^{T} \skprod{u_{t}(t)}{v_{1}(t)}_{H} + a(u(t), v_{1}(t); \sigma, t) \diff t + \skprod{u(0)}{v_{2}}_{H}
    \end{equation}
    definierte Bilinearform und $f \colon \mathcal Y \to \mathbb{R}$ das durch
    \begin{equation}
        \label{eq:pp:var_all_f}
        f(v) = \int_{0}^{T} \skprod{g(t)}{v_{1}(t)}_{H} \diff t + \skprod{u_{0}}{v_{2}}_{H}
    \end{equation}
    gegebene Funktional.
\end{Definition}

Als nächstes wollen wir nachweisen, dass obiges Raum-Zeit-Variationsproblem sachgemäß gestellt ist und zudem die Lösungen $u(\sigma)$ analytisch vom Parameter $\sigma \in \mathcal S$ abhängen.
Ersteres erhalten wir analog zu \thref{thm:schwab09:theorem51} für den nichtparametrischen Fall.
Bezüglich der Regularität stellt sich heraus, dass wir lediglich Bedingungen an die Familie von stetigen linearen Operatoren $\Set{ A(\sigma, t) \in \mathcal L(V, V') \given \sigma \in \mathcal S, t \in [0, T] }$ stellen müssen, wie folgender Satz zeigt:

\begin{Satz}[{{\cite[Theorem 21]{Kunoth:2013ef}}}]
\label{thm:kunoth:theorem21}
    Seien $\mathcal X$ und $\mathcal Y$ gegeben wie in~\eqref{eq:var_all_ansatzraum_x} respektive~\eqref{eq:var_all_testraum_y}.
    Weiter erfülle die Familie von Operatoren $\Set{ A(\sigma, t) \in \mathcal L(V, V') \given \sigma \in \mathcal S, t \in [0, T] }$ \thref{thm:kunoth:assumption1} für ein $0 < \mathfrak p \leq 1$.
    Für jedes $\sigma \in \mathcal S$ sei $B(\sigma) \in \mathcal L(\mathcal X, \mathcal Y')$ definiert durch
    \begin{equation}
        \label{eq:var_all_gross_b_parametrisch}
        \skprod{B(\sigma) u}{v}_{\mathcal Y' \times \mathcal Y} = b(u, v; \sigma), \quad u \in \mathcal X,~y \in \mathcal Y,
    \end{equation}
    mit $b(\blank, \blank; \sigma)$ wie in~\eqref{eq:pp:var_all_bf_b}.
    Dann ist $B(\sigma)$ für jedes $\sigma \in \mathcal S$ stetig invertierbar und es existieren Konstanten $0 < \beta_{1} \leq \beta_{2} < \infty$ mit
    \begin{equation}
        \label{eq:var_all_norm_B_und_B_inv_parametrisch}
        \sup_{\sigma \in \mathcal S} \norm{B(\sigma)}_{\mathcal L(\mathcal X, \mathcal Y')} \leq \beta_{2} \quad \text{und} \quad  \sup_{\sigma \in \mathcal S} \norm{B(\sigma)^{-1}}_{\mathcal L(\mathcal Y', \mathcal X)} \leq \frac{1}{\beta_{1}}.
    \end{equation}

    Zudem erfüllt die parametrische Familie von Operatoren $\Set{ B(\sigma) \in \mathcal L(\mathcal X, \mathcal Y') \given \sigma \in \mathcal S }$ \thref{thm:kunoth:assumption1} mit dem gleichen Regularitätsparameter $\mathfrak p$, die parametrische Familie von Lösungen $u(\sigma)$ des parametrischen Raum-Zeit-Variationsproblems \eqref{eq:pp:var_all_problem} hängt analytisch von $\sigma$ ab und erfüllt die A-Priori-Abschätzung
    \begin{equation}
        \label{eq:var_all_a_priori_schranke}
        \sup_{\sigma \in \mathcal S} \norm{(\partial^{\nu}_{\sigma} u)(\sigma)}_{\mathcal X} \leq C_{0} \norm{f}_{\mathcal Y'} \abs{\nu}! \tilde{b}^{\nu}
    \end{equation}
    für alle $\nu \in \mathfrak F$, wobei $f$ wie in~\eqref{eq:pp:var_all_f} gegeben ist.
\end{Satz}

\begin{Lemma}
\label{lemma:norm_B_beschraenkt_durch_norm_A}
    Sei $\sigma \in \mathcal S$ und $\nu \in \mathfrak F \setminus \Set{ 0 }$, dann gilt
    \begin{equation}
        \norm{\partial^{\nu}_{\sigma} B(\sigma)}_{\mathcal L(\mathcal X, \mathcal Y')}
        \leq
        \norm{\partial^{\nu}_{\sigma} A(\sigma)}_{\mathcal L(V, V')}
    \end{equation}

    \begin{Beweis}
        TODO: Moo.
    \end{Beweis}
\end{Lemma}

\begin{Beweis}[\thref{thm:kunoth:theorem21}]
TODO:\@ Bedingungen von \thref{thm:kunoth:assumption1} nachrechnen.
Zu (i): Folgt aus \thref{thm:schwab09:theorem51}, da $M_{a}, \alpha, \lambda$ unabhänging von $\sigma$.
Zu (ii): Folgt aus nachfolgendem \thref{lemma:norm_B_beschraenkt_durch_norm_A}.
\end{Beweis}

% section parametrische_lineare_evolutionsgleichung (end)

% chapter parametrisches_problem (end)
