%!TEX root = ../main.tex

\setchapterpreamble[ul][0.6\textwidth]{%
    \dictum[Andy Weir, \textit{The Martian}]{\enquote{I guess you could call it a \enquote{failure}, but I prefer the term \enquote{learning
    experience}.}}
    \vspace*{2\baselineskip}
}
\chapter{Problemstellung} % (fold)
\label{cha:parametrische_problem_neuer_versuch}

In diesem Kapitel greifen wir die in \cref{cha:einleitung} als Propagatoren bezeichneten parabolischen partiellen Differentialgleichungen \cref{eq:einleitung:fpropagator,eq:einleitung:bpropagator} auf, konkretisieren die Rahmenbedingungen und reinterpretieren diese Propagatoren anschließend als parametrische Differentialgleichungen.

Zudem leiten wir Variationsformulierungen her und weisen anhand dieser nach, dass diese sachgemäß gestellt sind und dass die Lösungen der parametrischen partiellen Differentialgleichung eine gewisse Regularität bezüglich der Parameter aufweisen.

\todo[inline]{Quellen}

\section{Die parabolische partielle Differentialgleichung} % (fold)
\label{sec:die_parabolische_partielle_differentialgleichung}

Ohne Einschränkung reicht es, nur den Vorwärts-Propagator \cref{eq:einleitung:fpropagator} zu betrachten, da der Rückwärts-Propagator \cref{eq:einleitung:bpropagator} durch die simple Transformation $s \mapsto 1 - s$ auf die selbe Form, lediglich mit vertauschten Rollen bei den Feldern $\omega_{\mathrm{A}}$ und $\omega_{\mathrm{B}}$, gebracht werden kann.

Wir wollen nun zunächst das Setting festlegen und dabei die aus der Einleitung bekannte partielle Differentialgleichung etwas allgemeiner auffassen.

Seien also $0 < T < \infty$ eine Konstante und $I = [0, T]$ ein reelles Intervall.
Weiter sei $\Omega \subset \mathbb{R}^{n}$ ein Gebiet, das heißt, eine offene, nichtleere, zusammenhängende Teilmenge, mit Lipschitz-Rand.
Wir betrachten nun die partielle Differentialgleichung
\begin{equation}
\label{eq:cha3:pde_erste}
    u_{t}(t, \vec{x}) = c \Delta_{\vec{x}} u(t, \vec{x}) - \omega(t, \vec{x}) u(t, \vec{x}) - \mu u(t, \vec{x}) \quad \text{auf}~I \times \Omega,
\end{equation}
wobei $c \in \mathbb{R}_{+}$ und $\mu \in \mathbb{R}$ Konstanten seien.
Weiter sei
\begin{equation}
\label{eq:pp:def_omega}
    \omega \colon I \times \Omega \to \mathbb{R}, \quad (t, \vec{x}) \mapsto
    \begin{cases}
        \omega_{1}(\vec{x}), & t \leq f, \\
        \omega_{2}(\vec{x}), & t > f,
    \end{cases}
\end{equation}
mit $L_{\infty}(\Omega)$-Abbildungen $\omega_{1}, \omega_{2}$ und einer Konstante $f \in I$.

Da, wie in \cref{cha:einleitung} erwähnt, konstante Verschiebungen der Felder $\omega_{i}$ keinen Einfluss auf das Ergebnis des dort beschriebenen Iterationsverfahrens haben führen wir den zusätzlichen Term $\mu u(t, x)$ ein.
Dies wird sich später noch als hilfreich beim Nachweis einiger Eigenschaften der Differentialgleichung erweisen.

Weiter beschränken wir uns an dieser Stelle auf den Fall homogener Randbedingungen in $\Omega$, das heißt, es gilt $\restr{u(t, \blank)}{\partial \Omega} = 0$ für alle $t \in I$.

Unter diesen Gegebenheiten können wir die obige partielle Differentialgleichung \cref{eq:cha3:pde_erste} weiter konkretisieren; es bietet sich an, dazu die Räume $V = H^{1}_{0}(\Omega)$ und $H = L_{2}(\Omega)$ zu verwenden.
Dabei handelt es sich jeweils um einen separablen Hilbertraum und es existiert eine dichte stetige Einbettung $H^{1}_{0}(\Omega) \denseinclusion L_{2}(\Omega)$.
Nach \cref{definition:gl:gelfand_tripel} erhalten wir daraus das Gelfand-Tripel
\begin{equation}
\label{eq:cha3:gelfand_triple}
    H^{1}_{0}(\Omega) \denseinclusion L_{2}(\Omega) \denseinclusion (H^{1}_{0}(\Omega))' = H^{-1}(\Omega).
\end{equation}

Mit diesen Vorbemerkungen können wir unter Hinzunahme eines Quellterms und einer Anfangsbedingung unser Problem wie folgt definieren:

\todo[inline]{Sind die Voraussetzungen so passend?}
\begin{Definition}[Starke Formulierung]
\label{definition:pp:starke_formulierung}
    Seien eine \emph{Anfangsbedingung} $u_{0} \in L_{2}(\Omega)$, ein \emph{Quellterm} $g \in L_{2}(I) \otimes H^{-1}(\Omega)$ und $\omega$ wie in \cref{eq:pp:def_omega}, sowie Konstanten $c \in \mathbb{R}_{+}$ und $\mu \in \mathbb{R}$ gegeben.
    Als \emph{starke Formulierung} bezeichnen wir die parabolische partielle Differentialgleichung
    \begin{equation}
    \label{eq:pp:starke_formulierung}
        \left\{
        \begin{aligned}
            u_{t}(t, x) - c \Delta u(t, x) + \omega(t, x) u(t, x) + \mu u(t, x) &= g(t, x) \quad &&\text{auf}~I \times \Omega,\\
            u(0, x) &= u_{0}(x) \quad &&\text{auf}~\Omega.
        \end{aligned}
        \right.
    \end{equation}
\end{Definition}

\begin{Bemerkung}
\label{bemerkung:pp:definition_operator_A}
    Der notationellen Einfachheit halber definieren wir für $t \in I$ den Operator
    \begin{equation}
    \label{eq:pp:definition_operator_A}
        A(t) \colon H^{1}_{0}(\Omega) \to H^{-1}(\Omega), \quad \eta \mapsto A(t) \eta = - c \Delta \eta + \omega(t, \blank) \eta + \mu \eta.
    \end{equation}
    Damit lässt sich die starke Formulierung \cref{eq:pp:starke_formulierung} auch schreiben als
    \begin{equation}
        \left\{
        \begin{aligned}
            u_t(t) + A(t)u(t) &= g(t) \quad &&\text{in}~H^{-1}(\Omega)~\fa t \in I,\\
            u(0) &= u_{0} \quad &&\text{in}~L_{2}(\Omega).
        \end{aligned}
        \right.
    \end{equation}
\end{Bemerkung}

\subsection{Raum-Zeit-Variationsformulierung} % (fold)
\label{sub:variationsformulierungen}

% subsection variationsformulierungen (end)

Als nächsten Schritt wollen wir nun eine Raum-Zeit-Variationsformulierung, auch schwache Formulierung genannt, für das Problem  \cref{eq:pp:starke_formulierung} herleiten.
Dazu ist es nützlich, den Operator $A(t)$ in Form einer Bilinearform zu schreiben.
Nach dem Rieszschen Darstellungssatz, vergleiche \cite[Theorem \S{}22.1]{Halmos:1957vd}, existiert eine Bilinearform
\begin{equation}
    a \colon H^{1}_{0}(\Omega) \times H^{1}_{0}(\Omega) \to \mathbb{R},
\end{equation}
so dass
\begin{equation}
    (A(t)\eta)\zeta = \skp{A(t)\eta}{\zeta}{H^{-1}(\Omega) \times H^{1}_{0}(\Omega)} = a(\eta, \zeta; t) \quad \fa \eta, \zeta \in H^{1}_{0}(\Omega)
\end{equation}
gilt.

\todo[inline]{Nochmal genauer anschauen, ob das mit der dualen Paarung und dem Skalarprodukt auch so einfach geht.}

Für den gegebenen Fall können wir diese Bilinearform explizit angeben, denn es gilt für $\eta, \zeta \in H^{1}_{0}(\Omega)$ nach dem Gaußschen Integralsatz die Gleichheit
\begin{equation}
    \begin{aligned}
        a(\eta, \zeta; t)
        &=    \skp{A(t)\eta}{\zeta}{H^{-1}(\Omega) \times H^{1}_{0}(\Omega)}
        =  \skp{A(t)\eta}{\zeta}{L_{2}(\Omega)}
        \\&= - c \skp{\Delta \eta}{\zeta}{L_{2}(\Omega)}
                + \skp{\omega(t, \blank) \eta}{\zeta}{L_{2}(\Omega)}
                + \mu \skp{\eta}{\zeta}{L_{2}(\Omega)}
        \\&= c \skp{\grad \eta}{\grad \zeta}{L_{2}(\Omega)}
                + \skp{\omega(t, \blank) \eta}{\zeta}{L_{2}(\Omega)}
                + \mu \skp{\eta}{\zeta}{L_{2}(\Omega)}.
    \end{aligned}
\end{equation}

Mit Hilfe dieser Darstellung erhalten wir nun für die Bilinearform $a(\blank, \blank; t)$ und damit auch für den Operator $A(t)$ die folgenden Aussagen für ein festes $t \in I$.
%
\begin{Satz}
\label{satz:pp:a_bf_bounded_garding}
    Seien $c \in \mathbb{R}_{+}$, $\mu \in \mathbb{R}$, $\omega \in L_{\infty}(\Omega)$ und
    \begin{equation}
    \label{eq:bf_a}
        \begin{aligned}
            &a(\blank, \blank) \colon H^{1}_{0}(\Omega) \times H^{1}_{0}(\Omega) \to \mathbb{R}, \\
            &(\eta, \zeta) \mapsto c\skp{\grad \eta}{\grad \zeta}{L_{2}(\Omega)} + \skp{\omega \eta}{\zeta}{L_{2}(\Omega)} + \mu \skp{\eta}{\zeta}{L_{2}(\Omega)}.
        \end{aligned}
    \end{equation}
    Dann erfüllt $a$ die folgenden Eigenschaften:
    \begin{thmenumerate}
        \item\label{satz:pp:a_bf_bounded_garding:1}
        \emph{Stetigkeit:} es gilt
        \begin{equation}
            \abs{a(\eta, \zeta)} \leq M_{a} \norm{\eta}_{H^{1}(\Omega)} \norm{\zeta}_{H^{1}(\Omega)} \quad \text{für alle}~\eta, \zeta \in H^{1}_{0}(\Omega)
        \end{equation}
        mit $M_{a} = \max\Set{c, \norm{\omega}_{L_{\infty}(\Omega)} + \abs{\mu}} \geq 0$.
        \item\label{satz:pp:a_bf_bounded_garding:2}
        \emph{G\aa{}rding-Ungleichung:} es gilt
        \begin{equation}
                a(\eta, \eta) + \lambda \norm{\eta}_{L_{2}(\Omega)}^{2} \geq \alpha \norm{\eta}_{H^{1}(\Omega)}^{2} \quad \text{für alle}~\eta \in H^{1}_{0}(\Omega)
        \end{equation}
        mit $\alpha = c \gamma_{\Omega}^{2} > 0$ und $\lambda = \min\Set{\norm{\omega}_{L_{\infty}(\Omega)} - \mu, 0} \geq 0$, wobei $\gamma_{\Omega}$ die Poincaré-Friedrichs-Konstante ist.
    \end{thmenumerate}

    \todo[inline]{Beweis prüfen!}
    \begin{Beweis}
    Wir zeigen zunächst die Stetigkeit.
    Seien dazu $\eta, \zeta \in H^{1}_{0}(\Omega)$ beliebig.
    Unter Verwendung der Dreiecks- und der Cauchy-Schwarz-Ungleichung erhalten wir
    \begin{align}
        \abs{a(\eta, \zeta)}
        &= \abs{c \skprod{\grad \eta}{\grad \zeta}_{L_{2}(\Omega)} + \skprod{\omega \eta}{\zeta}_{L_{2}(\Omega)} + \mu \skp{\eta}{\zeta}{L_{2}(\Omega)} }
        \\&\leq c \abs{\skprod{\grad \eta}{\grad \zeta}_{L_{2}(\Omega)}} + \abs{\skprod{\omega \eta}{\zeta}_{L_{2}(\Omega)}} + \abs{\mu} \abs{\skp{\eta}{\zeta}{L_{2}(\Omega)}}
        \\&\leq c \norm{\grad \eta}_{L_{2}(\Omega)} \norm{\grad \zeta}_{L_{2}(\Omega)} + (\norm{\omega}_{L_{\infty}(\Omega)} + \abs{\mu}) \norm{\eta}_{L_{2}(\Omega)} \norm{\zeta}_{L_{2}(\Omega)}
        \\&\leq \max \Set{ c, \norm{\omega}_{L_{\infty}(\Omega)} + \abs{\mu}} \norm{\eta}_{H^{1}(\Omega)} \norm{\zeta}_{H^{1}(\Omega)}.
    \end{align}

    Für die G\aa{}rding-Ungleichung seien nun $\eta \in H^{1}_{0}(\Omega)$ und $\lambda \in \mathbb{R}$.
    Wir betrachten
    \begin{align}
        a(\eta, \eta) + \lambda \norm{\eta}^{2}_{L_{2}(\Omega)}
        &= c \norm{\grad \eta}^{2}_{L_{2}(\Omega)} + \skprod{\omega \eta}{\eta}_{L_{2}(\Omega)} + \mu \skprod{\eta}{\eta}_{L_{2}(\Omega)} + \lambda \skprod{\eta}{\eta}_{L_{2}(\Omega)}
        \\&= c \norm{\grad \eta}^{2}_{L_{2}(\Omega)} + \skprod{(\omega + \mu + \lambda) \eta}{\eta}_{L_{2}(\Omega)}.
    \end{align}
    Wählen wir nun $\lambda = \min\Set{\norm{\omega}_{L_{\infty}(\Omega)} - \mu, 0} \geq 0$, dann gilt $\omega + \mu + \lambda \geq 0$ fast überall in $\Omega$ und wir erhalten die Abschätzung
    \begin{align}
        a(\eta, \eta) + \lambda \norm{\eta}^{2}_{L_{2}(\Omega)}
        &\geq c \norm{\grad \eta}^{2}_{L_{2}(\Omega)},
        \intertext{woraus wir durch Anwenden der Poincaré-Friedrichs-Ungleichung \cref{satz:grundlagen:poincare_friedrichs_ungleichung}}
        a(\eta, \eta) + \lambda \norm{\eta}^{2}_{L_{2}(\Omega)}
        &\geq c \gamma_{\Omega}^{2} \norm{\eta}^{2}_{H^{1}(\Omega)}
    \end{align}
    folgern.
    \end{Beweis}
\end{Satz}

\begin{Korollar}
\label{korollar:cha3:bf_elliptisch}
    Ist $\mu \geq \norm{\omega}_{L_{\infty}(\Omega)}$, dann ist die Bilinearform $a$ aus \cref{satz:pp:a_bf_bounded_garding} elliptisch.
\end{Korollar}

Unter diesen Gegebenheiten können wir nun mit den theoretischen Grundlagen aus \cref{sec:lineare_evolutionsgleichungen} eine sachgemäß gestellte Raum-Zeit-Variationsformulierung herleiten.
Als Ansatz- und Testfunktionenraum erhalten wir mit den konkret gewählten Hilberträumen
\begin{equation}
    \label{eq:var_ansatzraum_testraum}
    \mathcal X = L_{2}(I; H^{1}_{0}(\Omega)) \cap H^{1}(I; H^{-1}(\Omega))
    \quad \text{und} \quad
    \mathcal Y = L_{2}(I; H^{1}_{0}(\Omega)) \times L_{2}(\Omega).
\end{equation}
Das Raum-Zeit-Variationsproblem lautet damit:
\begin{Definition}[Schwache Formulierung]
\label{definition:cha3:schwache_formulierung}
    Als \emph{schwache Formulierung}, oder auch \emph{Raum"=Zeit"=Va"-ri"-a"-ti"-ons"-for"-mu"-lie"-rung}, der parabolischen partiellen Differentialgleichung aus \cref{definition:pp:starke_formulierung} bezeichnen wir das folgende Variationsproblem:
    Gegeben einen Quellterm $g \in L_{2}(I; H^{-1}(\Omega))$ und eine Anfangsbedingung $u_{0} \in L_{2}(\Omega)$.
    Finde ein $u \in \mathcal X$ mit
    \begin{equation}
        \label{eq:varprob}
        b(u, v) = f(v) \quad \text{für alle}~v = (v_{1}, v_{2}) \in \mathcal Y,
    \end{equation}
    wobei $b(\blank, \blank) \colon \mathcal X \times \mathcal Y \to \mathbb{R}$ die durch
    \begin{equation}
        \label{eq:buv}
        b(u, v)
            = \int_{I} \skprod{u_{t}(t)}{v_{1}(t)}_{L_{2}(\Omega)} + a(u(t), v_{1}(t); t) \diff t + \skprod{u(0)}{v_{2}}_{L_{2}(\Omega)}
    \end{equation}
    gegebene Bilinearform und $f(\blank) \colon \mathcal Y \to \mathbb{R}$ definiert ist durch
    \begin{equation}
        \label{eq:var_all_f_wiederholung}
        f(v) = \int_{I} \skprod{g(t)}{v_{1}(t)}_{L_{2}(\Omega)} \diff t + \skprod{u_{0}}{v_{2}}_{L_{2}(\Omega)}.
    \end{equation}
\end{Definition}

Diese Variationsformulierung wird uns in den nachfolgenden Kapiteln als Ausgangspunkt für die von uns angestrebten numerischen Verfahren dienen.

\subsection{Existenz und Eindeutigkeit von Lösungen} % (fold)
\label{sub:existenz_und_eindeutigkeit_von_l_sungen}

Wir weisen nun, aufbauend auf \cref{sec:lineare_evolutionsgleichungen}, nach, dass das Raum-Zeit-Variationsproblem aus \cref{definition:cha3:schwache_formulierung} sachgemäß gestellt ist, und bestimmen zudem Abschätzungen für die Norm des Operators und dessen Inverse.
Die Hauptarbeit dazu wurde bereits durch das \cref{satz:gl:bnb_theorem} und \cref{thm:schwab09:theorem51} geleistet und muss hier nur noch angewandt werden.

\todo[inline]{Anpassen an das neu vorkommende Mu!}

\begin{Korollar}
\label{korollar:2.2}
    Seien $\mathcal X$ und $\mathcal Y$ gegeben wie in \cref{eq:var_ansatzraum_testraum} und sei $B \colon \mathcal X \to \mathcal Y'$ definiert durch
    \begin{equation}
        \skprod{Bu}{v}_{\mathcal Y' \times \mathcal Y}  = b(u, v), \quad u \in \mathcal X,~ v \in \mathcal Y,
    \end{equation}
    mit $b(\blank, \blank)$ wie in \cref{eq:buv}.
    Dann ist $B$ stetig invertierbar und es gilt
    \begin{equation}
        \norm{B}_{\mathcal L(\mathcal X, \mathcal Y')}
        \leq
        \frac{\sqrt{2 \max\Set{1, c^{2}, \norm{\omega}_{L_{\infty}(\Omega)}^{2}} + M_{e}^{2}}}{\max\Set{\sqrt{1 + 2 \norm{\omega}_{L_{\infty}(\Omega)}^{2} \rho^{4}}, \sqrt{2} }}
    \end{equation}
    und
    \begin{equation}
        \norm{B^{-1}}_{\mathcal L( \mathcal Y', \mathcal X)}
        \leq \frac{e^{2 T \norm{\omega}_{L_{\infty}(\Omega)}} \max\Set{\sqrt{1 + 2 \norm{\omega}_{L_{\infty}(\Omega)}^{2} \rho^{4}}, \sqrt{2}} \sqrt{2 \max\Set{c^{-2} \gamma_{\Omega}^{-4}, 1} + M_{e}^{2}}}{\min\Set{c^{-1} \gamma_{\Omega}^{2}, c \gamma_{\Omega}^{2} \norm{\omega}_{L_{\infty}(\Omega)}^{-2}, c \gamma_{\Omega}^{2} }}.
        % \leq
        % \frac{\max\{\sqrt{ 1 + 2 \norm{\omega}_{L_{\infty}(\Omega)} \rho^{4}}, \sqrt{2} \}}{e^{-2 \norm{\omega}_{L_{\infty}(\Omega)} T}}
        % \frac{\sqrt{2 \max\{ 1, \sigma^{-2} \gamma_{\Omega}^{-4} \} + M_{e}^{2}}}{\min\{ \sigma \gamma_{\Omega}^{2} \norm{\omega}_{L_{\infty}(\Omega)}^{-2}, \sigma \gamma_{\Omega}^{2} \}}
    \end{equation}
    mit $M_{e}$ und $\rho$ wie in \cref{eq:var_all_M_e} respektive \cref{eq:var_all_rho}.
\end{Korollar}

% subsection existenz_und_eindeutigkeit_von_l_sungen (end)

% section die_parabolische_partielle_differentialgleichung (end)

\section{Parametrische Formulierung} % (fold)
\label{sec:parametrische_formulierung}

Nachdem wir nun die Rahmenbedingungen geklärt haben, fahren wir fort mit der Einführung einer parametrischen Variante der in diesem Kapitel bisher betrachteten parabolischen partiellen Differentialgleichung.
Diese ermöglicht es uns in den folgenden Kapiteln erst, eine Reduzierte-Basis-Methode zu entwickeln.

Dazu greifen wir den aus \cref{bemerkung:pp:definition_operator_A} bekannten Operator $A(t)$ auf und betrachten diesen zunächst zwar für einen festen Zeitpunkt $t \in I$, dafür aber in Abhängigkeit von einem $\omega \in L_{\infty}(\Omega)$.
Sei also $\omega \in L_{\infty}(\Omega)$, dann definieren wir eine Abbildung
\begin{equation}
    \label{eq:pp:op_a}
    A(\omega) \colon H^{1}_{0}(\Omega) \to H^{-1}(\Omega), \quad A(\omega) \eta = - c \Delta \eta + \omega \eta + \mu \eta
\end{equation}
und erhalten wie zuvor die zugehörige Bilinearform $a(\blank, \blank; \omega)$ durch
\begin{equation}
    \label{eq:pp:bf_a}
    \begin{aligned}
        &a(\blank, \blank; \omega) \colon H^{1}_{0}(\Omega) \times H^{1}_{0}(\Omega) \to \mathbb{R}, \\
        &(\eta, \zeta) \mapsto c\skp{\grad \eta}{\grad \zeta}{L_{2}(\Omega)} + \skp{\omega \eta}{\zeta}{L_{2}(\Omega)} + \mu \skp{\eta}{\zeta}{L_{2}(\Omega)}.
    \end{aligned}
\end{equation}

Als nächstes konkretisieren wir die Abhängigkeit des Operators $A(\omega)$ beziehungsweise der Bilinearform $a(\blank, \blank; \omega)$ von $\omega$.
Dazu stellen wir die folgende Bedingung an den Parameter $\omega$, welche sich später wiederum bei der Reduzierte-Basis-Methode als nützlich erweisen wird.

\begin{Definition}
\label{definition:pp:omega_affin}
    Die Funktionen $\omega \in L_{\infty}(\Omega)$ seien \emph{affin} darstellbar.
    Genauer sei $\mathcal S \subset \mathbb{R}^{\mathbb{N}}$ ein Parameterraum und $\Set{ \varphi_{j} }_{j \in \mathbb{N}} \in L_{\infty}(\Omega)$ eine Folge von Funktion, so dass $\omega$ sich für $\sigma \in \mathcal S$ als Reihe der Form
    \begin{equation}
    \label{eq:pp:affine_zerlegung_omega}
        w(\blank; \sigma) \colon \Omega \to \mathbb{R}, \quad w(x; \sigma) = \sum_{j = 1}^{\infty} \sigma_{j} \varphi_{j}(x)
    \end{equation}
    schreiben lässt.
\end{Definition}

\begin{Bemerkung}
    Wir wählen für den Rest der Arbeit $\mathcal S = [-1, 1]^{\mathbb{N}}$.
    Dies stellt keine Einschränkung dar, da die Funktionen $\Set{ \varphi_{j} }_{j \in \mathbb{N}}$ beliebig umskaliert werden können.
\end{Bemerkung}

Setzen wir diese affine Darstellung nun zunächst in den Operator $A(\omega)$ ein, dann erhalten wir die Darstellung
\begin{equation}
\label{eq:pp:op_a_sigma}
    A(\omega(\sigma)) \colon H^{1}_{0}(\Omega) \to H^{-1}(\Omega), \quad A(\omega(\sigma)) u = -c \Delta u + \sum_{j = 1}^{\infty} \sigma_{j} \varphi_{j} u + \mu u,
\end{equation}
und als zugehörige Bilinearform $a(\blank, \blank; \omega(\sigma))$ ergibt sich wie zuvor
\begin{equation}
\label{eq:pp:bf_a_sigma}
    \begin{aligned}
    &a(\blank, \blank; \omega(\sigma)) \colon H^{1}_{0}(\Omega) \times H^{1}_{0}(\Omega) \to \mathbb{R}, \\
    &(u, v) \mapsto c\skp{\grad u}{\grad v}{L_{2}(\Omega)} + \sum_{j = 1}^{\infty} \sigma_{j} \skp{\varphi_{j} u}{v}{L_{2}(\Omega)} + \mu \skp{u}{v}{L_{2}(\Omega)}.
    \end{aligned}
\end{equation}

\begin{Bemerkung}
    Um die Schreibweisen kurz zu halten, schreiben wir meist $\omega(\sigma)$ statt $\omega(\blank; \sigma)$.
    Weiter schreiben wir, vor allem um zwischen dem allgemeinen parametrischen und dem affinen Fall zu unterscheiden, kurz $A(\sigma)$ statt $A(\omega(\sigma))$ für \cref{eq:pp:op_a_sigma} und $a(\blank, \blank; \sigma)$ statt $a(\blank, \blank; \omega(\sigma))$ für \cref{eq:pp:bf_a_sigma}.
\end{Bemerkung}

Damit der Operator $A(\sigma)$ sowie die Bilinearform $a(\blank, \blank; \sigma)$ für $\sigma \in \mathcal S$ wohldefiniert sind, müssen wir Wohldefiniertheit, dass heißt gleichmäßige Konvergenz, der obigen affinen Zerlegung \cref{eq:pp:affine_zerlegung_omega} von $\omega$ aus \cref{definition:pp:omega_affin} fordern.
Dies wird durch folgende Bedingung sichergestellt.
%%
\begin{Annahme}
    Das Funktionensystem $\Set{ \varphi_{j} }_{j \in \mathbb{N}} \in L_{\infty}(\Omega)$ sei einfach summierbar in der $L_{\infty}$-Norm, das heißt es gelte
    \begin{equation}
        \Set{ \norm{\varphi_{j}}_{L_{\infty}(\Omega) } }_{j \in \mathbb{N}} \in \ell_{1}(\mathbb{N}).
    \end{equation}
\end{Annahme}

Hieraus folgt wegen $\mathcal S = [-1, 1]^{\mathbb{N}}$ insbesondere
\begin{equation}
    \sup_{\sigma \in \mathcal S} \norm{\omega(\sigma)}_{L_{\infty}(\Omega)} \leq \sum_{j = 1}^{\infty} \norm{\varphi_{j}}_{L_{\infty}(\Omega)} < \infty.
\end{equation}

Mit dieser Vorarbeit können wir die schwache Formulierung aus \cref{definition:cha3:schwache_formulierung} nun zu der folgenden parametrischen schwachen Formulierung ausweiten.

\todo[inline]{Die Zeitabhängigkeit des Operators $A$ und der Bilinearform $a$ muss noch eingebaut werden!}

\begin{Definition}[Parametrische schwache Formulierung]
\label{definition:cha3:param_schwache_formulierung}
    Als \emph{parametrisches schwache Formulierung}, oder auch \emph{parametrische Raum"=Zeit"=Va"-ri"-a"-ti"-ons"-for"-mu"-lie"-rung} von \cref{eq:pp:starke_formulierung} bezeichnen wir das folgende Variationsproblem:
    Gegeben einen Quellterm $g \in L_{2}(I; H^{-1}(\Omega))$ und eine Anfangsbedingung $u_{0} \in L_{2}(\Omega)$.
    Finde für alle $\sigma \in \mathcal S$ ein $u(\sigma) \in \mathcal X$ mit
    \begin{equation}
        \label{eq:varprob}
        b(u(\sigma), v; \sigma) = f(v) \quad \text{für alle}~v = (v_{1}, v_{2}) \in \mathcal Y,
    \end{equation}
    wobei $b(\blank, \blank; \sigma) \colon \mathcal X \times \mathcal Y \to \mathbb{R}$ die durch
    \begin{equation}
        \label{eq:buv}
        b(u, v; \sigma)
            = \int_{I} \skprod{u_{t}(t)}{v_{1}(t)}_{L_{2}(\Omega)} + a(u(t), v_{1}(t); t, \sigma) \diff t + \skprod{u(0)}{v_{2}}_{L_{2}(\Omega)}
    \end{equation}
    gegebene Bilinearform und $f(\blank) \colon \mathcal Y \to \mathbb{R}$ definiert ist durch
    \begin{equation}
        \label{eq:var_all_f_wiederholung}
        f(v) = \int_{I} \skprod{g(t)}{v_{1}(t)}_{L_{2}(\Omega)} \diff t + \skprod{u_{0}}{v_{2}}_{L_{2}(\Omega)}.
    \end{equation}
\end{Definition}

% section parametrische_formulierung (end)


\section{Regularität bezüglich der Parameter} % (fold)
\label{sec:regularit_t_bez_glich_der_parameter}

In diesem Abschnitt wollen wir nun nachweisen, dass die im vorherigen Abschnitt hergeleitete parametrische schwache Formulierung eine Regularität bezüglich des Parameters aufweist, konkret werden wir nachweisen, dass die Lösung $u(\sigma)$ analytisch im Parameter $\sigma$ ist.
Dabei orientieren wir uns an den Arbeiten von \textcite{Cohen:2010kz,Kunoth:2013ef} und weisen diese Eigenschaften zunächst für den stationären Fall nach, das heißt, wir verlieren die Zeitabhängigkeit und betrachten stattdessen die parametrische Operatorgleichung
\begin{equation}
\label{eq:pp:operatorgleichung_param}
    A(\omega) \eta(\omega) = f \quad \text{in}~H^{-1}(\Omega).
\end{equation}

Wir beginnen damit, die Existenz der partiellen Ableitung $\partial^{\nu}_{\sigma} \eta$ in jedem Punkt $\sigma \in \mathcal S$ nachzuweisen.

Zunächst einige notationelle Vorbemerkungen.
\begin{Bemerkung}
    Bezeichne mit $\mathfrak F = \Set{ \nu \in \mathbb{N}^{\mathbb{N}}_{0} \given \norm{\nu}_{\ell_{1}(\mathbb{N})} < \infty }$ die Menge aller Folgen nichtnegativer ganzer Zahlen, wobei
    \begin{equation}
        \norm{\nu}_{\ell_{1}}(\mathbb{N}) = \sum_{k = 1}^{\infty} \abs{\nu_{k}}
    \end{equation}
    die $\ell_{1}(\mathbb{N})$-Norm sei.
    Damit ergibt sich, dass $\mathfrak F$ gerade diejenigen Folgen enthält, die nur endlich viele Einträgen ungleich Null enthalten.

    Sei $\nu \in \mathfrak F$ und $b \in \ell_{p}(\mathbb{N})$, $p > 0$, dann schreiben wir
    \begin{equation}
        b^{\nu} = \prod_{j = 1}^{\infty} b_{j}^{\nu_{j}}
    \end{equation}
    mit der Konvention $0^{0} = 1$.
    Wegen $\norm{\nu}_{\ell_{1}(\mathbb{N})} < \infty$ ist dieses Produkt stets endlich.
\end{Bemerkung}

Bevor wir uns an die partiellen Ableitungen wagen, beweisen wir eine Stabilitätsaussage für die obige parametrische Operatorgleichung \cref{eq:pp:operatorgleichung_param}, welche sich dabei als nützlich erweisen wird.
Für diese benötigen wir, dass die obige Operatorgleichung sachgemäß gestellt ist.

\begin{Satz}
\label{satz:pp:lax_auf_elliptisch}
    Seien $\omega \in L_{\infty}(\Omega)$, $\mu \geq \norm{\omega}_{L_{\infty}(\Omega)}$ und weiter $g \in H^{-1}(\Omega)$ und $A(\omega)$ wie in \cref{eq:pp:op_a}, dann besitzt die Operatorgleichung
    \begin{equation}
        A(\omega) u(\omega) = g
    \end{equation}
    eine eindeutige Lösung $u(\omega) \in H^{1}_{0}(\Omega)$ und diese erfüllt
    \begin{equation}
        \norm{u(\omega)}_{H^{1}(\Omega)} \leq \frac{\norm{g}_{H^{-1}(\Omega)}}{\alpha}
    \end{equation}
    mit $\alpha$ aus \cref{satz:pp:a_bf_bounded_garding}.

    \begin{Beweis}
        Folgt direkt aus dem Lemma vom Lax-Milgram, \cref{lemma:gl:lax_milgram}.
    \end{Beweis}
\end{Satz}



\begin{Lemma}
\label{lemma:pp:norm_abschaetzung}
    Seien $\omega_{1}, \omega_{2} \in L_{\infty}(\Omega)$ und $u_{1}, u_{2}$ die zugehörigen Lösungen von \cref{eq:pp:operatorgleichung_param}, dann gilt
    \begin{equation}
        \norm{u_{1} - u_{2}}_{H^{1}(\Omega)} \leq \frac{\norm{f}_{H^{-1}(\Omega)}}{\gamma_{0}^{2}} \norm{\omega_{1} - \omega_{2}}_{L_{\infty}}.
    \end{equation}

    \todo[inline]{Beweis prüfen!}

    \begin{Beweis}
        Durch Subtraktion der beiden Variationsformulierungen erhalten wir für $v \in V$ die Gleichung
        \begin{align}
            0 &= a(u_{1}, v; \omega_{1}) - a(u_{2}, v; \omega_{2})
            \\&= c \skp{\grad u_{1} - \grad u_{2}}{\grad v}{H} + \skp{\omega_{1}u_{1} - \omega_{2} u_{2}}{v}{H} + \mu \skp{u_{1} - u_{2}}{v}{H},
            \intertext{durch setzen von $z = u_{1} - u_{2}$ erhalten wir weiter}
            0 &= c \skp{\grad z}{\grad v}{H} + \skp{\omega_{1} z}{v}{H} + \mu \skp{z}{v}{H} + \skp{(\omega_{1} - \omega_{2}) u_{2}}{v}{H}
            \\&= a(z, v; \omega_{1}) + \skp{(\omega_{1} - \omega_{2}) u_{2}}{v}{H}.
        \end{align}
        Dies lässt sich nun wieder in Form des Variationsproblems schreiben, konkret
        \begin{equation}
            a(z, v; \omega_{1}) = g(v) \quad \fa v \in V,
        \end{equation}
        mit
        \begin{equation}
            g(v) = - \skp{(\omega_{1} - \omega_{2}) u_{2}}{v}{H}.
        \end{equation}

        Nach \cref{satz:pp:lax_auf_elliptisch} ist die Lösung $z = u_{1} - u_{2} \in V$ eindeutig und erfüllt
        \begin{equation}
            \norm{z}_{V} \leq \frac{\norm{g}_{V'}}{\gamma_{0}}.
        \end{equation}

        Die Operatornorm von $g$ lässt sich mittels der Cauchy-Schwarz-Ungleichung bestimmen zu
        \begin{equation}
            \begin{aligned}
                \norm{g}_{V'}
                  &=    \sup_{\norm{v}_{V} = 1} \abs{g(v)}
                   =    \sup_{\norm{v}_{V} = 1} \abs{\skp{(\omega_{1} - \omega_{2}) u_{2}}{v}{H}}
                \\&\leq \sup_{\norm{v}_{V} = 1} \norm{\omega_{1} - \omega_{2}}_{L_{\infty}(\Omega)} \norm{u_{1}}_{H} \norm{v}_{H}
                   \leq \sup_{\norm{v}_{V} = 1} \norm{\omega_{1} - \omega_{2}}_{L_{\infty}(\Omega)} \norm{u_{1}}_{V} \norm{v}_{V}
                \\&=    \norm{\omega_{1} - \omega_{2}}_{L_{\infty}(\Omega)} \norm{u_{1}}_{V}
                   \leq \norm{\omega_{1} - \omega_{2}}_{L_{\infty}(\Omega)} \frac{\norm{f}_{V'}}{\gamma_{0}}.
            \end{aligned}
        \end{equation}
        Zusammen liefert dies die Ungleichung
        \begin{equation}
            \norm{u_{1} - u_{2}}_{V}
            = \norm{z}_{V} \leq \norm{\omega_{1} - \omega_{2}}_{L_{\infty}(\Omega)} \frac{\norm{f}_{V'}}{\gamma_{0}^{2}}
        \end{equation}
        und damit die Behauptung.
    \end{Beweis}
\end{Lemma}

\begin{Satz}
\label{satz:cha3:existenz_part_ableitung}
    Die Abbildung $\mathcal S \ni \sigma \mapsto u(\sigma) \in V$ besitzt für alle $\nu \in \mathfrak F$ die partielle Ableitung $\partial^{\nu}_{\sigma} u(\sigma)$.

    \todo[inline]{Beweis prüfen!}

    \begin{Beweis}
        Wir zeigen die Behauptung exemplarisch für für die partiellen Ableitungen erster Ordnung für ein festes $\sigma \in \mathcal S$.
        Seien dazu $\nu = e_{j}$ für ein $j \in \mathbb{N}$ und sei weiter $h \in \mathbb{R} \setminus \Set{ 0 }$.
        Definiere
        \begin{equation}
            u_{h} := \frac{u(\sigma + h e_{j}) - u(\sigma)}{h}.
        \end{equation}

        \todo[inline]{wohldfiniertheit?}
        Ist $\abs{h}$ klein genug, so dass $\sigma + h e_{j} \in \mathcal S$ gilt, dann existieren die eindeutigen Lösungen $u(\sigma + h e_{j})$ und $u(\sigma)$ des Variationsproblems (??) zu den jeweiligen Parametern.
        Betrachte nun die Differenz der beiden Variationsformulierungen, dann gilt
        \begin{align}
            0 &= a(u(\sigma + h e_{j}), v; \sigma + h e_{j}) - a(u(\sigma), v; \sigma)
            %
            \\ &= c \skp{u(\sigma + h e_{j})}{v}{L_{2}(\Omega)} + \skp{\omega(\sigma + h e_{j}) u(\sigma + h e_{j})}{v}{L_{2}(\Omega)} + \mu \skp{u(\sigma + h e_{j})}{v}{L_{2}(\Omega)}
            \\&\qquad - c \skp{u(\sigma)}{v}{L_{2}(\Omega)} - \skp{\omega(\sigma) u(\sigma)}{v}{L_{2}(\Omega)} - \mu \skp{u(\sigma)}{v}{L_{2}(\Omega)}
            %
            \\&= c \skp{u(\sigma + h e_{j}) - u(\sigma)}{v}{L_{2}(\Omega)} + \skp{\omega(\sigma + h e_{j}) u(\sigma + h e_{j}) - \omega(\sigma) u(\sigma)}{v}{L_{2}(\Omega)}
            \\&\qquad  + \mu \skp{u(\sigma + h e_{j}) - u(\sigma)}{v}{L_{2}(\Omega)}
            %
            \\&= c \skp{u(\sigma + h e_{j}) - u(\sigma)}{v}{L_{2}(\Omega)} + \skp{\omega(\sigma) (u(\sigma + h e_{j}) - u(\sigma))}{v}{L_{2}(\Omega)}
            \\&\qquad + \skp{(\omega(\sigma + h e_{j}) - \omega(\sigma)) u(\sigma + h e_{j})}{v}{L_{2}(\Omega)} + \mu \skp{u(\sigma + h e_{j}) - u(\sigma)}{v}{L_{2}(\Omega)}
            %
            \\&= h \left(
                c \skp{u_{h}}{v}{L_{2}(\Omega)} + \skp{\omega(\sigma) u_{h}}{v}{L_{2}(\Omega)} + \mu \skp{u_{h}}{v}{L_{2}(\Omega)}
             \right) + h \skp{\varphi_{j} u(\sigma + h e_{j})}{v}{L_{2}(\Omega)}
            \\&= h a(u_{h}, v; \sigma) + h \skp{\varphi_{j} u(\sigma + h e_{j})}{v}{L_{2}(\Omega)}
        \end{align}
        Dies lässt sich zu folgendem Variationsproblem
        \begin{equation}
            a(u_{h}, v; \sigma) = F_{h}(v) \quad \fa v \in V.
        \end{equation}
        Dabei ist
        \begin{equation}
            F_{h} \colon V \to \mathbb{R}, \quad v \mapsto - \skp{\varphi_{j} u(\sigma + h e_{j})}{v}{L_{2}(\Omega)}
        \end{equation}
        ein lineares stetiges Funktional.
        Weiter ist $F_{h}(\blank)$ stetig in $h = 0$, denn es gilt für festes $v \in V$ die Abschätzung
        \begin{align}
            \abs{F_{h}(v) - F_{0}(v)}
            &= \abs{\skp{\varphi_{j} (u(\sigma + h e_{j}) - u(\sigma))}{v}{L_{2}(\Omega)}}
            \\&\leq \norm{\varphi_{j}}_{L_{\infty}(\Omega)} \abs{\skp{u(\sigma + h e_{j}) - u(\sigma)}{v}{L_{2}(\Omega)}}
            \\&\leq \norm{\varphi_{j}}_{L_{\infty}(\Omega)} \norm{u(\sigma + h e_{j}) - u(\sigma)}_{L_{2}(\Omega)} \norm{v}_{L_{2}(\Omega)}
            \\&\leq \norm{\varphi_{j}}_{L_{\infty}(\Omega)} \norm{u(\sigma + h e_{j}) - u(\sigma)}_{H^{1}(\Omega)} \norm{v}_{H^{1}(\Omega)}.
        \end{align}
        Weiter liefert \cref{lemma:pp:norm_abschaetzung} die Ungleichung
        \begin{align}
            \norm{u(\sigma + h e_{j}) - u(\sigma)}_{H^{1}(\Omega)}
            &\leq \frac{\norm{f}_{V'}}{\gamma_{0}^{2}} \norm{\omega(\sigma + h e_{j}) - \omega(\sigma)}_{L_{\infty}(\Omega)}
            \\&\leq \abs{h} \norm{\varphi_{j}}_{L_{\infty}(\Omega)} \frac{\norm{f}_{V'}}{\gamma_{0}^{2}}.
        \end{align}
        Zusammen liefern die beiden Abschätzungen
        \begin{equation}
            \abs{F_{h}(v) - F_{0}(v)} \leq \abs{h} \norm{\varphi_{j}}^{2}_{L_{\infty}(\Omega)} \norm{v}_{H^{1}(\Omega)} \frac{\norm{f}_{V'}}{\gamma_{0}^{2}} \to 0 \quad \text{für}~h \to 0,
        \end{equation}
        das heißt es gilt $F_{h} \to F_{0}$ in $V'$ für $h \to 0$.

        Dies impliziert insbesondere $u_{h} \to u_{0}$ in $V$ für $h \to 0$.
        Weiter erfüllt $u_{0}$ die Gleichung
        \begin{equation}
            a(u_{0}, v; \sigma) = F_{0}(v) \quad \fa v \in V.
        \end{equation}
        Damit existiert $\partial_{\sigma_{j}} u(\sigma) = u_{0}$ in $V$ und ist die eindeutige Lösung des Variationsproblems
        \begin{equation}
            a(\partial_{\sigma_{j}} u(\sigma), v; \sigma) = - \skp{\varphi_{j} u(\sigma)}{v}{L_{2}(\Omega)} \quad \fa v \in V.
        \end{equation}

        Alternativ erhält man dieses Variationsproblem auch durch formales Differenzieren der eigentlichen Variationsformulierung nach $\sigma_{j}$, denn dann gilt
        \begin{align}
            \partial_{\sigma_{j}} a(u(\sigma), v; \sigma)
            &= \partial_{\sigma_{j}} \left( c \skp{\grad u(\sigma)}{\grad v}{L_{2}(\Omega)} + \skp{w(\sigma)u(\sigma)}{v}{L_{2}(\Omega)} + \mu \skp{u(\sigma)}{v}{L_{2}(\Omega)} \right)
            \\&= c \skp{\grad \partial_{\sigma_{j}} u(\sigma)}{\grad v}{L_{2}(\Omega)} + \skp{ \partial_{\sigma_{j}} w(\sigma)u(\sigma) + w(\sigma) u(\sigma) }{v}{L_{2}(\Omega)} + \mu \skp{\partial_{\sigma_{j}}u(\sigma)}{v}{L_{2}(\Omega)}
            \\&= a(\partial_{\sigma_{j}} u(\sigma), v; \sigma) + \skp{\partial_{\sigma_{j}}\omega(\sigma)u(\sigma)}{v}{L_{2}(\Omega)}
            \\&= a(\partial_{\sigma_{j}} u(\sigma), v; \sigma) + \skp{\varphi_{j} u(\sigma)}{v}{L_{2}(\Omega)}
        \end{align}
        und
        \begin{equation}
            \partial_{\sigma_{j}} \skp{f}{v}{L_{2}(\Omega)} = 0,
        \end{equation}
        woraus man ingesamt erneut das Variationsproblem
        \begin{equation}
            a(\partial_{\sigma_{j}} u(\sigma), v; \sigma) = - \skp{\varphi u(\sigma)}{v}{L_{2}(\Omega)} \quad \fa v \in V
        \end{equation}
        erhält.

        Durch wiederholtes Anwenden des beschriebenen Vorgehens erhält man damit auch die Existenz der höheren partiellen Ableitungen.
    \end{Beweis}
\end{Satz}

Seien $b := (b_{j})_{j} \in \mathbb{R}$ und $b_{j} := \frac{\norm{\varphi_{j}}_{L_{\infty}}}{\gamma_{0}}$.

\begin{Satz}
\label{satz:cha3:part_ableitung_schranke}
    Es gilt
    \begin{equation}
        \sup_{\sigma \in \mathcal S} \norm{\partial^{\nu}_{\sigma} u(\sigma)} \leq B \abs{\nu}! b^{\nu}.
    \end{equation}

    \begin{Beweis}
        Wir betrachten zunächst die Variationsprobleme, welche von den partiellen Ableitungen $\partial^{\nu}_{\sigma} u(\sigma)$ erfüllt werden.
        Diese Variationsprobleme haben die folgende rekurisve Darstellung
        \begin{equation}
            a(\partial^{\nu}_{\sigma} u(\sigma), v; \sigma)
            = - \sum_{\Set{j \given \nu_{j} \neq 0}} \nu_{j} \skp{\varphi_{j} \partial^{\nu - e_{j}}_{\sigma} u(\sigma)}{v}{L_{2}(\Omega)}.
        \end{equation}
        Die Gültigkeit dieser Formel beweisen wir per Induktion.
        Der Fall $\abs{\nu} = 1$ ergibt sich aus (??).
        Betrachte also $abs{\nu} > 1$.
        Sei $k \in \mathbb{N}$ ein Index mit $\nu_{k} \neq 0$, dann definieren wir $\tilde{\nu} := \nu - e_{k}$ und es gilt offenbar $\abs{\tilde{\nu}} = \abs{\nu} - 1$.
        Nach Induktionsvoraussetzung gilt weiter
        \begin{equation}
            a(\partial^{\tilde{\nu}}_{\sigma} u(\sigma), v; \sigma) + \sum_{\Set{j \given \tilde{\nu}_{j} \neq 0}} \tilde{\nu}_{j} \skp{\varphi_{j} \partial^{\tilde{\nu} - e_{j}}_{\sigma} u(\sigma)}{v}{L_{2}(\Omega)} = 0,
        \end{equation}
        wobei $\nu_{j} = \tilde{\nu}_{j}$ für $j \neq k$ und $\tilde{\nu}_{k} = \nu_{k} - 1$ gilt.

        Partielles Differenzieren dieser Gleichung nach $\sigma_{k}$ liefert
        \begin{align}
            0 &=
                a(\partial^{\nu}_{\sigma} u(\sigma), v; \sigma)
                + \skp{\varphi_{k} \partial^{\nu - e_{k}}_{\sigma} u(\sigma)}{v}{L_{2}(\Omega)}
                + \sum_{\Set{j \neq k \given \nu_{j} \neq 0}} \nu_{j} \skp{\varphi_{j} \partial^{\nu - e_{j}}_{\sigma} u(\sigma)}{v}{L_{2}(\Omega)}
            \\&\qquad     + (\nu_{k} - 1) \skp{\varphi_{k} \partial^{\nu - e_{k}}_{\sigma} u(\sigma) }{v}{L_{2}(\Omega)}
        \end{align}

        Wählt man nun $v = \partial^{\nu}_{\sigma} u(\sigma)$, dann gilt damit aufgrund der Koerzivität von $a(\blank, \blank; \sigma)$ die Ungleichung
        \begin{equation}
            a(\partial^{\nu}_{\sigma} u(\sigma), \partial^{\nu}_{\sigma} u(\sigma); \sigma) \geq \alpha \norm{\partial^{\nu}_{\sigma} u(\sigma)}_{H^{1}(\Omega)}^{2}.
        \end{equation}
        Außerdem erhalten wir mit Hilfe er Cauchy-Schwarz-Ungleichung die Abschätzung
        \begin{align}
            a(\partial^{\nu}_{\sigma} u(\sigma), \partial^{\nu}_{\sigma} u(\sigma); \sigma)
            &= - \sum_{\Set{j \given \nu_{j} \neq 0}} \nu_{j} \skp{\varphi_{j} \partial^{\nu - e_{j}}_{\sigma} u(\sigma) }{\partial^{\nu}_{\sigma} u(\sigma)}{L_{2}(\Omega)}
            \\&\leq \sum_{\Set{j \given \nu_{j} \neq 0}} \nu_{j} \norm{\varphi_{j}}_{L_{\infty}(\Omega)} \norm{\partial^{\nu - e_{j}}_{\sigma} u(\sigma)}_{H^{1}(\Omega)} \norm{\partial^{\nu}_{\sigma} u(\sigma)}_{H^{1}(\Omega)}.
        \end{align}
        Beide Ungleichungen zusammen ergeben
        \begin{equation}
            \norm{\partial^{\nu}_{\sigma} u(\sigma)}_{H^{1}(\Omega)} \leq \sum_{\Set{j \given \nu_{j} \neq 0}} \nu_{j} \frac{\norm{\varphi_{j}}_{L_{\infty}(\Omega)}}{\alpha} \norm{\partial^{\nu - e_{j}}_{\sigma} u(\sigma)}_{H^{1}(\Omega)}.
        \end{equation}

        Um nun die eigentliche Behauptung zu beweisen, verfolgen wir erneut einen Induktionsansatz.
        Sei zunächst $\abs{\nu} = 0$, dann haben wir die Ungleichung
        \begin{equation}
            \norm{u(\sigma)}_{H^{1}(\Omega)} \leq \frac{\norm{\varphi_{j}}_{L_{\infty}(\Omega)}}{\alpha},
        \end{equation}
        welche nach Satz (??) erfüllt ist.
        Sei also weiter $\abs{\nu} > 0$, dann gilt mit obiger Ungleichung unter Verwendung der Induktionsvoraussetzung für $\norm{\partial^{\nu - e_{j}}_{\sigma} u(\sigma)}_{H^{1}(\Omega)}$ die Abschätzung
        \begin{align}
            \norm{\partial^{\nu}_{\sigma} u(\sigma)}_{H^{1}(\Omega)}
            &\leq
            \sum_{\Set{j \given \nu_{j} \neq 0}} \nu_{j} \frac{\norm{\varphi_{j}}_{L_{\infty}(\Omega)}}{\alpha} \norm{\partial^{\nu - e_{j}}_{\sigma} u(\sigma)}_{H^{1}(\Omega)}
            \\&\leq
            \sum_{\Set{j \given \nu_{j} \neq 0}} \nu_{j} \frac{\norm{\varphi_{j}}_{L_{\infty}(\Omega)}}{\alpha} B \abs{\nu - e_{j}}! b^{\nu - e_{j}}
            \\&\leq
            \left( \sum_{\Set{j \given \nu_{j} \neq 0}} \nu_{j} \right)  \frac{\norm{\varphi_{j}}_{L_{\infty}(\Omega)}}{\alpha} B (\abs{\nu} - 1)! b^{\nu}
            \\&\leq
            \frac{\norm{\varphi_{j}}_{L_{\infty}(\Omega)}}{\alpha} B \abs{\nu}! b^{\nu}
        \end{align}
    \end{Beweis}
\end{Satz}

\todo[inline]{Daraus folgern, dass es für den parabolischen auch gilt.}

Die in den Beweisen von \cref{satz:cha3:existenz_part_ableitung} und \cref{satz:cha3:part_ableitung_schranke} angewandten Schritte lassen sich, solange die Wohldefiniertheit aller Ausdrücke gegeben ist, analog auch für allgemeinere parametrisch affine Operatoren respektive Bilinearformen nachweisen.
Dies findet man beispielsweise bei \textcite{Kunoth:2013ef}.
Wir entnehmen daraus die folgende Annahme und den folgenden Satz, welcher uns als Motivation dienen soll.

\todo[inline]{Genauer ausarbeiten!}

\begin{Annahme}[{{\cite[Assumption 1]{Kunoth:2013ef}}}]
\label{thm:kunoth:assumption1}
    Seien $X$ und $Y$ zwei reflexive Banachräume.
    Die parametrische Familie von Operatoren
    $\Set{ A(\sigma) \in \mathcal L(X, Y') \given \sigma \in \mathcal S }$ sei eine $\mathfrak p$-reguläre Operatorfamilie für ein $0 < \mathfrak p \leq 1$, das heißt,
    \begin{thmenumerate}
        \item $A(\sigma) \in \mathcal L(X, Y')$ sei stetig invertierbar für alle $\sigma \in \mathcal S$ mit gleichmäßig beschränktem Inversen $A{(\sigma)}^{-1} \in \mathcal L(Y', X)$, das heißt, es existiert ein $C_{0} > 0$ mit
        \begin{equation}
            \sup_{\sigma \in \mathcal S} \norm{A{(\sigma)}^{-1}}_{\mathcal L(Y', X)} \leq C_{0},
        \end{equation}
        \item für jedes feste $\sigma \in \mathcal S$ seien die Operatoren $A(\sigma)$ analytisch bezüglich $\sigma$.
        Konkret existiert eine nichtnegative Folge $b = (b_{j})_{j \in \mathbb{N}} \in \ell_{\mathfrak p}(\mathbb{N})$, so dass
        \begin{equation}
            \sup_{\sigma \in \mathcal S} \norm{(A{(0)})^{-1}(\partial^{\nu}_{\sigma} A(\sigma))}_{\mathcal L(X, X)} \leq C_{0} b^{\nu}
        \end{equation}
        für alle $\nu \in \mathfrak F \setminus \{ 0 \}$ gilt.
        Dabei sei $\partial^{\nu}_{\sigma} A(\sigma) \deq \frac{\partial^{\nu_{1}}}{\partial \sigma_{1}} \frac{\partial^{\nu_{2}}}{\partial \sigma_{2}} \cdots A(\sigma)$.
    \end{thmenumerate}
\end{Annahme}

Als nächstes wollen wir nachweisen, dass obiges Raum-Zeit-Variationsproblem sachgemäß gestellt ist und zudem die Lösungen $u(\sigma)$ analytisch vom Parameter $\sigma \in \mathcal S$ abhängen.
Ersteres erhalten wir analog zu \cref{thm:schwab09:theorem51} für den nichtparametrischen Fall.
Bezüglich der Regularität stellt sich heraus, dass wir lediglich Bedingungen an die Familie von stetigen linearen Operatoren $\Set{ A(\sigma, t) \in \mathcal L(V, V') \given \sigma \in \mathcal S, t \in [0, T] }$ stellen müssen, wie folgender Satz zeigt:

\begin{Satz}[{{\cite[Theorem 21]{Kunoth:2013ef}}}]
\label{thm:kunoth:theorem21}
    Seien $\mathcal X$ und $\mathcal Y$ gegeben wie in~\cref{eq:var_all_ansatzraum_x} respektive~\cref{eq:var_all_testraum_y}.
    Weiter erfülle die Familie von Operatoren $\Set{ A(\sigma, t) \in \mathcal L(V, V') \given \sigma \in \mathcal S, t \in [0, T] }$ \cref{thm:kunoth:assumption1} für ein $0 < \mathfrak p \leq 1$.
    Für jedes $\sigma \in \mathcal S$ sei $B(\sigma) \in \mathcal L(\mathcal X, \mathcal Y')$ definiert durch
    \begin{equation}
        \label{eq:var_all_gross_b_parametrisch}
        \skprod{B(\sigma) u}{v}_{\mathcal Y' \times \mathcal Y} = b(u, v; \sigma), \quad u \in \mathcal X,~y \in \mathcal Y,
    \end{equation}
    mit $b(\blank, \blank; \sigma)$ wie in~\cref{eq:pp:var_all_bf_b}.
    Dann ist $B(\sigma)$ für jedes $\sigma \in \mathcal S$ stetig invertierbar und es existieren Konstanten $0 < \beta_{1} \leq \beta_{2} < \infty$ mit
    \begin{equation}
        \label{eq:var_all_norm_B_und_B_inv_parametrisch}
        \sup_{\sigma \in \mathcal S} \norm{B(\sigma)}_{\mathcal L(\mathcal X, \mathcal Y')} \leq \beta_{2} \quad \text{und} \quad  \sup_{\sigma \in \mathcal S} \norm{B(\sigma)^{-1}}_{\mathcal L(\mathcal Y', \mathcal X)} \leq \frac{1}{\beta_{1}}.
    \end{equation}

    Zudem erfüllt die parametrische Familie von Operatoren $\Set{ B(\sigma) \in \mathcal L(\mathcal X, \mathcal Y') \given \sigma \in \mathcal S }$ \cref{thm:kunoth:assumption1} mit dem gleichen Regularitätsparameter $\mathfrak p$, die parametrische Familie von Lösungen $u(\sigma)$ des parametrischen Raum-Zeit-Variationsproblems \cref{eq:pp:var_all_problem} hängt analytisch von $\sigma$ ab und erfüllt die A-Priori-Abschätzung
    \begin{equation}
        \label{eq:var_all_a_priori_schranke}
        \sup_{\sigma \in \mathcal S} \norm{(\partial^{\nu}_{\sigma} u)(\sigma)}_{\mathcal X} \leq C_{0} \norm{f}_{\mathcal Y'} \abs{\nu}! \tilde{b}^{\nu}
    \end{equation}
    für alle $\nu \in \mathfrak F$, wobei $f$ wie in~\cref{eq:pp:var_all_f} gegeben ist.
\end{Satz}

\begin{Lemma}
\label{lemma:norm_B_beschraenkt_durch_norm_A}
    Sei $\sigma \in \mathcal S$ und $\nu \in \mathfrak F \setminus \Set{ 0 }$, dann gilt
    \begin{equation}
        \norm{\partial^{\nu}_{\sigma} B(\sigma)}_{\mathcal L(\mathcal X, \mathcal Y')}
        \leq
        \norm{\partial^{\nu}_{\sigma} A(\sigma)}_{\mathcal L(V, V')}
    \end{equation}

    \begin{Beweis}
        \todo[inline]{Beweis}
    \end{Beweis}
\end{Lemma}

\begin{Beweis}[\cref{thm:kunoth:theorem21}]
\todo[inline]{Soll der ausgeführt werden?}
Bedingungen von \cref{thm:kunoth:assumption1} nachrechnen.
Zu (i): Folgt aus \cref{thm:schwab09:theorem51}, da $M_{a}, \alpha, \lambda$ unabhänging von $\sigma$.
Zu (ii): Folgt aus nachfolgendem \cref{lemma:norm_B_beschraenkt_durch_norm_A}.
\end{Beweis}


% % section regularit_t_bez_glich_der_parameter (end)


% % \section{Der betrachtete Fall} % (fold)
% % \label{sec:der_betrachtete_fall}

% % Wir betrachten in diesem Abschnitt zunächst eine vereinfachte Variante der vorgestellten Differentialgleichung.
% % Zunächst ignorieren wir den Wechsel des Feldes $\omega$ ab einem bestimmten Zeitpunkt und erhalten dadurch einen autonomen linearen Differentialoperator $A$.
% % Weiter schränken wir uns auf homogene Dirichlet- statt periodischen Randbedingungen ein.

% % Unter diesen Gegebenheiten bietet es sich an, die Hilberträume als $V = H^{1}_{0}(\Omega)$ und $H = L_{2}(\Omega)$ zu wählen.
% % Bekanntlich sind diese separabel und es existiert eine dichte stetige Einbettung von $H^{1}_{0}(\Omega)$ in $L_{2}(\Omega)$.
% % Wegen $(H^{1}_{0}(\Omega))' = H^{-1}(\Omega)$ ergibt dies das Gelfand-Tripel
% % \begin{equation}
% %     H^{1}_{0}(\Omega) \denseinclusion L_{2}(\Omega) \denseinclusion H^{-1}(\Omega).
% % \end{equation}
% % Wie zuvor verwenden wir $\skprod{\blank}{\blank}$ mit entsprechendem Index sowohl für die Skalarprodukte als auch für die duale Paarung auf $H^{-1}(\Omega) \times H^{1}_{0}(\Omega)$.

% % Um obige partielle Differentialgleichung in das Setting aus \cref{sec:lineare_evolutionsgleichungen} zu übertragen, definieren wir einen linearen Operator $A$ als
% % \begin{equation}
% %     \label{eq:def_op_A}
% %     A \colon H^{1}_{0}(\Omega) \to H^{-1}(\Omega), \quad \eta \mapsto A \eta = - c \Delta \eta + \omega \eta
% % \end{equation}
% % und weiter die zugehörige Bilinearform
% % \begin{equation}
% %     a \colon H^{1}_{0}(\Omega) \times H^{1}_{0}(\Omega) \to \mathbb{R}, \quad a(\eta, \zeta) = \skprod{A \eta}{\zeta}_{L_{2}(\Omega)}.
% % \end{equation}
% % Diese lässt sich unter Verwendung der Greenschen Formeln (TODO!) auch schreiben als
% % \begin{equation}
% %     \begin{aligned}
% %         a(\eta, \zeta)
% %         &= \skprod{- c \Delta \eta + \omega \eta}{\zeta}_{L_{2}(\Omega)}
% %         = - c \skprod{\Delta \eta}{\zeta}_{L_{2}(\Omega)} + \skprod{\omega \eta}{\zeta}_{L_{2}(\Omega)}
% %         \\&= c \skprod{\grad \eta}{\grad \zeta}_{L_{2}(\Omega)} + \skprod{\omega \eta}{\zeta}_{L_{2}(\Omega)}.
% %     \end{aligned}
% % \end{equation}

% % Diese Bilinearform ist stetig und erfüllt eine G\aa{}rding-Ungleichung, wie das folgende Lemma zeigt.

% % \begin{Lemma}
% % \label{lemma:a_bf_bounded_garding}
% %     Seien $c \in \mathbb{R}_{+}$, $\omega \in L_{\infty}(\Omega)$ und
% %     \begin{equation}
% %     \label{eq:bf_a}
% %         a \colon H^{1}_{0}(\Omega) \times H^{1}_{0}(\Omega) \to \mathbb{R}, \quad a(\eta, \zeta) = c \skprod{\grad \eta}{\grad \zeta}_{L_{2}(\Omega)} + \skprod{\omega \eta}{\zeta}_{L_{2}(\Omega)}.
% %     \end{equation}
% %     Dann erfüllt $a$ die Eigenschaften aus \cref{annahme:eigenschaften_bf_a}:
% %     \begin{thmenumerate}
% %         \item\label{lemma:a_bf_bounded_garding:1}
% %         \emph{Stetigkeit:} es gilt
% %         \begin{equation}
% %             \abs{a(\eta, \zeta)} \leq M_{a} \norm{\eta}_{H^{1}(\Omega)} \norm{\zeta}_{H^{1}(\Omega)} \quad \text{für alle}~\eta, \zeta \in H^{1}_{0}(\Omega)
% %         \end{equation}
% %         mit $M_{a} = \max\Set{c, \norm{\omega}_{L_{\infty}(\Omega)} } \geq 0$.
% %         \item\label{lemma:a_bf_bounded_garding:2}
% %         \emph{G\aa{}rding-Ungleichung:} es gilt
% %         \begin{equation}
% %                 a(\eta, \eta) + \lambda \norm{\eta}_{L_{2}(\Omega)}^{2} \geq \alpha \norm{\eta}_{H^{1}(\Omega)}^{2} \quad \text{für alle}~\eta \in H^{1}_{0}(\Omega)
% %         \end{equation}
% %         mit $\alpha = c \gamma_{\Omega}^{2} > 0$ und $\lambda = \norm{\omega}_{L_{\infty}(\Omega)} \geq 0$, wobei $\gamma_{\Omega}$ die Poincaré-Friedrichs-Konstante ist.
% %     \end{thmenumerate}

% %     \begin{Beweis}
% %     Wir zeigen zunächst die Stetigkeit.
% %     Seien dazu $\eta, \zeta \in H^{1}_{0}(\Omega)$ beliebig.
% %     Unter Verwendung der Dreiecks- und der Cauchy-Schwarz-Ungleichung erhalten wir
% %     \begin{align}
% %         \abs{a(\eta, \zeta)}
% %         &= \abs{c \skprod{\grad \eta}{\grad \zeta}_{L_{2}(\Omega)} + \skprod{\omega \eta}{\zeta}_{L_{2}(\Omega)}}
% %         \\&\leq c \abs{\skprod{\grad \eta}{\grad \zeta}_{L_{2}(\Omega)}} + \abs{\skprod{\omega \eta}{\zeta}_{L_{2}(\Omega)}}
% %         \\&\leq c \norm{\grad \eta}_{L_{2}(\Omega)} \norm{\grad \zeta}_{L_{2}(\Omega)} + \norm{\omega}_{L_{\infty}(\Omega)} \norm{\eta}_{L_{2}(\Omega)} \norm{\zeta}_{L_{2}(\Omega)}
% %         \\&\leq \max \Set{ c, \norm{\omega}_{L_{\infty}(\Omega)} } \norm{\eta}_{H^{1}(\Omega)} \norm{\zeta}_{H^{1}(\Omega)}.
% %     \end{align}

% %     Für die G\aa{}rding-Ungleichung seien nun $\eta \in H^{1}_{0}(\Omega)$ und $\lambda \in \mathbb{R}$.
% %     Wir betrachten
% %     \begin{align}
% %         a(\eta, \eta) + \lambda \norm{\eta}^{2}_{L_{2}(\Omega)}
% %         &= c \norm{\grad \eta}^{2}_{L_{2}(\Omega)} + \skprod{\omega \eta}{\eta}_{L_{2}(\Omega)} + \lambda \skprod{\eta}{\eta}_{L_{2}(\Omega)}
% %         \\&= c \norm{\grad \eta}^{2}_{L_{2}(\Omega)} + \skprod{(\omega + \lambda) \eta}{\eta}_{L_{2}(\Omega)}.
% %     \end{align}
% %     Wählen wir nun $\lambda = \norm{\omega}_{L_{\infty}(\Omega)} \geq 0$, dann gilt $\omega + \lambda \geq 0$ fast überall in $\Omega$ und wir erhalten die Abschätzung
% %     \begin{align}
% %         a(\eta, \eta) + \lambda \norm{\eta}^{2}_{L_{2}(\Omega)}
% %         &\geq c \norm{\grad \eta}^{2}_{L_{2}(\Omega)},
% %         \intertext{woraus wir durch Anwenden der Poincaré-Friedrichs-Ungleichung \cref{satz:grundlagen:poincare_friedrichs_ungleichung}}
% %         a(\eta, \eta) + \lambda \norm{\eta}^{2}_{L_{2}(\Omega)}
% %         &\geq c \gamma_{\Omega}^{2} \norm{\eta}^{2}_{H^{1}(\Omega)}
% %     \end{align}
% %     folgern.
% %     \end{Beweis}
% % \end{Lemma}

% % Unter diesen Gegebenheiten erhalten wir nach \cref{sec:lineare_evolutionsgleichungen} eine sachgemäß gestellte Raum-Zeit-Variationsformulierung.
% % Ansatz- und Testfunktionenraum ergeben sich mit den konkret gewählten Hilberträumen zu
% % \begin{equation}
% %     \label{eq:var_ansatzraum_testraum}
% %     \mathcal X = L_{2}(I; H^{1}_{0}(\Omega)) \cap H^{1}(I; H^{-1}(\Omega))
% %     \quad \text{und} \quad
% %     \mathcal Y = L_{2}(I; H^{1}_{0}(\Omega)) \times L_{2}(\Omega).
% % \end{equation}
% % Das Variationsproblem lautet damit:
% %     Gegeben ein $g \in L_{2}(I; H^{-1}(\Omega))$ und ein $u_{0} \in L_{2}(\Omega)$. Finde ein $u \in \mathcal X$ mit
% %     \begin{equation}
% %         \label{eq:varprob}
% %         b(u, v) = f(v) \quad \text{für alle}~v \in \mathcal Y,
% %     \end{equation}
% %     wobei $b(\blank, \blank) \colon \mathcal X \times \mathcal Y \to \mathbb{R}$ die durch
% %     \begin{equation}
% %         \label{eq:buv}
% %         b(u, v)
% %             = \int_{I} \skprod{u_{t}(t)}{v_{1}(t)}_{L_{2}(\Omega)} + a(u(t), v_{1}(t)) \diff t + \skprod{u(0)}{v_{2}}_{L_{2}(\Omega)}
% %     \end{equation}
% %     gegebene Bilinearform und $f(\blank) \colon \mathcal Y \to \mathbb{R}$ definiert ist durch
% %     \begin{equation}
% %         \label{eq:var_all_f_wiederholung}
% %         f(v) = \int_{I} \skprod{g(t)}{v_{1}(t)}_{L_{2}(\Omega)} \diff t + \skprod{u_{0}}{v_{2}}_{L_{2}(\Omega)}.
% %     \end{equation}

% % Aus \cref{thm:schwab09:theorem51} und \cref{thm:schwab09:theorem51:ungleichungen} erhalten wir nun die Wohldefiniertheit des obigen Variationsproblems und zugleich Schranken für die Operatoren.

% % \begin{Korollar}
% % \label{korollar:2.2}
% %     Seien $\mathcal X$ und $\mathcal Y$ gegeben wie in \cref{eq:var_ansatzraum_testraum} und sei $B \colon \mathcal X \to \mathcal Y'$ definiert durch
% %     \begin{equation}
% %         \skprod{Bu}{v}_{\mathcal Y' \times \mathcal Y}  = b(u, v), \quad u \in \mathcal X,~ v \in \mathcal Y,
% %     \end{equation}
% %     mit $b(\blank, \blank)$ wie in \cref{eq:buv}.
% %     Dann ist $B$ stetig invertierbar und es gilt
% %     \begin{equation}
% %         \norm{B}_{\mathcal L(\mathcal X, \mathcal Y')}
% %         \leq
% %         \frac{\sqrt{2 \max\Set{1, c^{2}, \norm{\omega}_{L_{\infty}(\Omega)}^{2}} + M_{e}^{2}}}{\max\Set{\sqrt{1 + 2 \norm{\omega}_{L_{\infty}(\Omega)}^{2} \rho^{4}}, \sqrt{2} }}
% %     \end{equation}
% %     und
% %     \begin{equation}
% %         \norm{B^{-1}}_{\mathcal L( \mathcal Y', \mathcal X)}
% %         \leq \frac{e^{2 T \norm{\omega}_{L_{\infty}(\Omega)}} \max\Set{\sqrt{1 + 2 \norm{\omega}_{L_{\infty}(\Omega)}^{2} \rho^{4}}, \sqrt{2}} \sqrt{2 \max\Set{c^{-2} \gamma_{\Omega}^{-4}, 1} + M_{e}^{2}}}{\min\Set{c^{-1} \gamma_{\Omega}^{2}, c \gamma_{\Omega}^{2} \norm{\omega}_{L_{\infty}(\Omega)}^{-2}, c \gamma_{\Omega}^{2} }}.
% %         % \leq
% %         % \frac{\max\{\sqrt{ 1 + 2 \norm{\omega}_{L_{\infty}(\Omega)} \rho^{4}}, \sqrt{2} \}}{e^{-2 \norm{\omega}_{L_{\infty}(\Omega)} T}}
% %         % \frac{\sqrt{2 \max\{ 1, \sigma^{-2} \gamma_{\Omega}^{-4} \} + M_{e}^{2}}}{\min\{ \sigma \gamma_{\Omega}^{2} \norm{\omega}_{L_{\infty}(\Omega)}^{-2}, \sigma \gamma_{\Omega}^{2} \}}
% %     \end{equation}
% %     mit $M_{e}$ und $\rho$ wie in \cref{eq:var_all_M_e} respektive \cref{eq:var_all_rho}.
% % \end{Korollar}

% % section der_betrachtete_fall (end)

% % \section{Parametrische Formulierung} % (fold)
% % \label{sec:parametrische_formulierung}

% % \todo[inline]{Ordentlich aufschreiben}

% % Dieser Abschnitt dient der Einführung einer parametrischen Variante der zuvor vorgestellten parabolischen partiellen Differentialgleichung.

% % Für den weiteren Verlauf der Arbeit wählen wir $V = H^{1}_{0}(\Omega)$, $H = L_{2}(\Omega)$ und $V' = H^{-1}(\Omega)$.
% % % Diese separablen Hilberträume bilden ein Gelfand-Tripel $V \denseinclusion H \denseinclusion V'$.

% % Wir betrachten nun zunächst die folgende parametrische Operatorgleichung: Sei ein $g \in V'$ gegeben, finde für alle $\omega$ ein $u(\omega) \in V$, so dass
% % \begin{equation}
% %     A(\omega) u(\omega) = g \in V'
% % \end{equation}
% % gilt.
% % Durch die Wahl $V = H^{1}_{0}(\Omega)$ sind die Randbedingungen $\restr{u(\omega)}{\partial \Omega} = 0$ bereits implizit gegeben.
% % Dabei sei der Operator $A(\omega)$ gegeben durch
% % \begin{equation}
% %     \label{eq:pp:op_a}
% %     A(\omega) \colon V \to V', \quad A(\omega) u = - c \Delta u + \omega u + \mu u.
% % \end{equation}
% % Die zugehörige Bilinearform $a(\blank, \blank; \omega)$ ergibt sich damit zu
% % \begin{equation}
% %     \label{eq:pp:bf_a}
% %     a(\blank, \blank; \omega) \colon V \times V \to \mathbb{R},
% %     \quad (u, v) \mapsto c\skp{\grad u}{\grad v}{H} + \skp{\omega u}{v}{H} + \mu \skp{u}{v}{H}.
% % \end{equation}

% % Unter diesen Bedingungen erhalten wir für die Bilinearform aus \cref{eq:pp:bf_a} respektive \cref{eq:pp:bf_a_sigma} die folgenden Eigenschaften.
% % %
% % \begin{Satz}
% % \label{satz:pp:a_bf_bounded_garding}
% %     Seien $c \in \mathbb{R}_{+}$, $\mu \in \mathbb{R}$, $\omega \in L_{\infty}(\Omega)$ und
% %     \begin{equation}
% %     \label{eq:bf_a}
% %         \begin{aligned}
% %             &a(\blank, \blank) \colon H^{1}_{0}(\Omega) \times H^{1}_{0}(\Omega) \to \mathbb{R}, \\
% %             &(u, v) \mapsto c\skp{\grad u}{\grad v}{L_{2}(\Omega)} + \skp{\omega u}{v}{L_{2}(\Omega)} + \mu \skp{u}{v}{L_{2}(\Omega)}.
% %         \end{aligned}
% %     \end{equation}
% %     Dann erfüllt $a$ die Eigenschaften aus \cref{annahme:eigenschaften_bf_a}:
% %     \begin{thmenumerate}
% %         \item\label{satz:pp:a_bf_bounded_garding:1}
% %         \emph{Stetigkeit:} es gilt
% %         \begin{equation}
% %             \abs{a(\eta, \zeta)} \leq M_{a} \norm{\eta}_{H^{1}(\Omega)} \norm{\zeta}_{H^{1}(\Omega)} \quad \text{für alle}~\eta, \zeta \in H^{1}_{0}(\Omega)
% %         \end{equation}
% %         mit $M_{a} = \max\Set{c, \norm{\omega}_{L_{\infty}(\Omega)} + \abs{\mu}} \geq 0$.
% %         \item\label{satz:pp:a_bf_bounded_garding:2}
% %         \emph{G\aa{}rding-Ungleichung:} es gilt
% %         \begin{equation}
% %                 a(\eta, \eta) + \lambda \norm{\eta}_{L_{2}(\Omega)}^{2} \geq \alpha \norm{\eta}_{H^{1}(\Omega)}^{2} \quad \text{für alle}~\eta \in H^{1}_{0}(\Omega)
% %         \end{equation}
% %         mit $\alpha = c \gamma_{\Omega}^{2} > 0$ und $\lambda = \min\Set{\norm{\omega}_{L_{\infty}(\Omega)} - \mu, 0} \geq 0$, wobei $\gamma_{\Omega}$ die Poincaré-Friedrichs-Konstante ist.
% %     \end{thmenumerate}

% %     \begin{Beweis}
% %     Wir zeigen zunächst die Stetigkeit.
% %     Seien dazu $\eta, \zeta \in H^{1}_{0}(\Omega)$ beliebig.
% %     Unter Verwendung der Dreiecks- und der Cauchy-Schwarz-Ungleichung erhalten wir
% %     \begin{align}
% %         \abs{a(\eta, \zeta)}
% %         &= \abs{c \skprod{\grad \eta}{\grad \zeta}_{L_{2}(\Omega)} + \skprod{\omega \eta}{\zeta}_{L_{2}(\Omega)} + \mu \skp{\eta}{\zeta}{L_{2}(\Omega)} }
% %         \\&\leq c \abs{\skprod{\grad \eta}{\grad \zeta}_{L_{2}(\Omega)}} + \abs{\skprod{\omega \eta}{\zeta}_{L_{2}(\Omega)}} + \abs{\mu} \abs{\skp{\eta}{\zeta}{L_{2}(\Omega)}}
% %         \\&\leq c \norm{\grad \eta}_{L_{2}(\Omega)} \norm{\grad \zeta}_{L_{2}(\Omega)} + (\norm{\omega}_{L_{\infty}(\Omega)} + \abs{\mu}) \norm{\eta}_{L_{2}(\Omega)} \norm{\zeta}_{L_{2}(\Omega)}
% %         \\&\leq \max \Set{ c, \norm{\omega}_{L_{\infty}(\Omega)} + \abs{\mu}} \norm{\eta}_{H^{1}(\Omega)} \norm{\zeta}_{H^{1}(\Omega)}.
% %     \end{align}

% %     Für die G\aa{}rding-Ungleichung seien nun $\eta \in H^{1}_{0}(\Omega)$ und $\lambda \in \mathbb{R}$.
% %     Wir betrachten
% %     \begin{align}
% %         a(\eta, \eta) + \lambda \norm{\eta}^{2}_{L_{2}(\Omega)}
% %         &= c \norm{\grad \eta}^{2}_{L_{2}(\Omega)} + \skprod{\omega \eta}{\eta}_{L_{2}(\Omega)} + \mu \skprod{\eta}{\eta}_{L_{2}(\Omega)} + \lambda \skprod{\eta}{\eta}_{L_{2}(\Omega)}
% %         \\&= c \norm{\grad \eta}^{2}_{L_{2}(\Omega)} + \skprod{(\omega + \mu + \lambda) \eta}{\eta}_{L_{2}(\Omega)}.
% %     \end{align}
% %     Wählen wir nun $\lambda = \min\Set{\norm{\omega}_{L_{\infty}(\Omega)} - \mu, 0} \geq 0$, dann gilt $\omega + \mu + \lambda \geq 0$ fast überall in $\Omega$ und wir erhalten die Abschätzung
% %     \begin{align}
% %         a(\eta, \eta) + \lambda \norm{\eta}^{2}_{L_{2}(\Omega)}
% %         &\geq c \norm{\grad \eta}^{2}_{L_{2}(\Omega)},
% %         \intertext{woraus wir durch Anwenden der Poincaré-Friedrichs-Ungleichung \cref{satz:grundlagen:poincare_friedrichs_ungleichung}}
% %         a(\eta, \eta) + \lambda \norm{\eta}^{2}_{L_{2}(\Omega)}
% %         &\geq c \gamma_{\Omega}^{2} \norm{\eta}^{2}_{H^{1}(\Omega)}
% %     \end{align}
% %     folgern.
% %     \end{Beweis}
% % \end{Satz}

% % \begin{Korollar}
% %     Ist $\mu \geq \norm{\omega}_{L_{\infty}(\Omega)}$, dann ist die Bilinearform koerziv.
% % \end{Korollar}

% % \begin{Satz}
% % \label{satz:pp:lax_auf_elliptisch}
% %     Seien $\omega \in L_{\infty}(\Omega)$, $\mu \geq \norm{\omega}_{L_{\infty}(\Omega)}$ und weiter $g \in H^{-1}(\Omega)$ und $A(\omega)$ wie in \cref{eq:pp:op_a}, dann besitzt die Operatorgleichung
% %     \begin{equation}
% %         A(\omega) u(\omega) = g
% %     \end{equation}
% %     eine eindeutige Lösung $u(\omega) \in H^{1}_{0}(\Omega)$ und diese erfüllt
% %     \begin{equation}
% %         \norm{u(\omega)}_{H^{1}(\Omega)} \leq \frac{\norm{g}_{H^{-1}(\Omega)}}{\alpha}
% %     \end{equation}
% %     mit $\alpha$ aus \cref{satz:pp:a_bf_bounded_garding}.

% %     \begin{Beweis}
% %         Folgt aus dem Banach-Ne\v{c}as-Babu\v{s}ka-Theorem, \cref{satz:gl:bnb_theorem}.
% %     \end{Beweis}
% % \end{Satz}

% % \todo[inline]{Ab hier parametrisch}

% % Wir wollen nun die Abhängigkeit des Operators $A(\omega)$ von dem Parameter $\omega$ konkretisieren.

% % \begin{Definition}
% % \label{definition:pp:omega_affin}
% %     Die Funktion $\omega$ ist affin darstellbar.
% %     Genauer sei $\mathcal S \subset \mathbb{R}^{\mathbb{N}}$ ein Parameterraum und $\Set{ \varphi_{j} }_{j \in \mathbb{N}} \in L_{\infty}(\Omega)$ eine Folge von Funktion, so dass $\omega$ sich für $\sigma \in \mathcal S$ schreiben lässt als
% %     \begin{equation}
% %         w(\blank; \sigma) \colon \Omega \to \mathbb{R}, \quad w(x; \sigma) = \sum_{j = 1}^{\infty} \sigma_{j} \varphi_{j}(x).
% %     \end{equation}
% % \end{Definition}

% % \begin{Bemerkung}
% %     Wir wählen für den Rest der Arbeit $\mathcal S = [-1, 1]^{\mathbb{N}}$.
% %     Dies stellt keine Einschränkung dar, da die Funktionen $\Set{ \varphi_{j} }_{j \in \mathbb{N}}$ beliebig umskaliert werden können.
% % \end{Bemerkung}

% % Setzen wir diese affine Darstellung nun zunächst in den Operator $A(\omega)$ ein, dann erhalten wir die Darstellung
% % \begin{equation}
% %     A(\omega(\sigma)) \colon V \to V', \quad A(\omega(\sigma)) u = -c \Delta u + \sum_{j = 1}^{\infty} \sigma_{j} \varphi_{j} u + \mu u,
% % \end{equation}
% % und als zugehörige Bilinearform $a(\blank, \blank; \omega(\sigma))$ ergibt sich
% % \begin{equation}
% % \label{eq:pp:bf_a_sigma}
% %     a(\blank, \blank; \omega(\sigma)) \colon V \times V \to \mathbb{R}, \quad a(u, v) \mapsto c\skp{\grad u}{\grad v}{H} + \sum_{j = 1}^{\infty} \sigma_{j} \skp{\varphi_{j} u}{v}{H} + \mu \skp{u}{v}{H}.
% % \end{equation}

% % \begin{Bemerkung}
% %     Um die Schreibweisen zu verkürzen, verwenden wir meist $\omega(\sigma)$ statt $w(\blank; \sigma)$, sowie $A(\sigma)$ und $a(\blank, \blank; \sigma)$ statt $A(\omega(\sigma))$ respektive $a(\blank, \blank; \omega(\sigma))$.
% % \end{Bemerkung}

% % Damit der Operator $A(\sigma)$ sowie die Bilinearform $a(\blank, \blank; \sigma)$ wohldefiniert sind, müssen wir Wohldefiniertheit, dass heißt gleichmäßige Konvergenz, der obigen affinen Zerlegung von $\omega$ aus \cref{definition:pp:omega_affin} fordern.
% % Dies wird durch folgende Bedingung sichergestellt.
% % %%
% % \begin{Annahme}
% %     Das Funktionensystem $\Set{ \varphi_{j} }_{j \in \mathbb{N}} \in L_{\infty}(\Omega)$ sei einfach summierbar in der $L_{\infty}$-Norm, das heißt es gelte
% %     \begin{equation}
% %         \Set{ \norm{\varphi_{j}}_{L_{\infty}(\Omega) } }_{j \in \mathbb{N}} \in \ell_{1}(\mathbb{N}).
% %     \end{equation}
% % \end{Annahme}
% % %%
% % Hieraus folgt wegen $\mathcal S = [-1, 1]^{\mathbb{N}}$ insbesondere
% % \begin{equation}
% %     \sup_{\sigma \in \mathcal S} \norm{\omega(\sigma)}_{L_{\infty}(\Omega)} \leq \sum_{j = 1}^{\infty} \norm{\varphi_{j}}_{L_{\infty}(\Omega)} < \infty.
% % \end{equation}

% % section parametrische_formulierung (end)

% % \section{Regularität bezüglich des Parameters} % (fold)
% % \label{sec:regularit_t_bez_glich_des_parameters}

% % In diesem Abschnitt wollen wir nun unter geeigneten, noch näher zu bestimmenden Bedingungen, die analytische Abhängigkeit der Lösung $u(\sigma)$ der zuvor eingeführten PPDE vom Parameter $\sigma \in \mathcal S$ nachweisen.
% % Dabei orientieren wir uns vor allem an den Arbeiten von \textcite{Cohen:2010kz,Kunoth:2013ef}.

% % \begin{Lemma}
% % \label{lemma:pp:norm_abschaetzung}
% %     Seien $\omega_{1}, \omega_{2} \in L_{\infty}(\Omega)$ und $u_{1}, u_{2}$ die zugehörigen Lösungen, dann gilt
% %     \begin{equation}
% %         \norm{u_{1} - u_{2}}_{V} \leq \frac{\norm{f}_{V'}}{\gamma_{0}^{2}} \norm{\omega_{1} - \omega_{2}}_{L_{\infty}}.
% %     \end{equation}

% %     \begin{Beweis}
% %         Durch Subtraktion der beiden Variationsformulierungen erhalten wir für $v \in V$ die Gleichung
% %         \begin{align}
% %             0 &= a(u_{1}, v; \omega_{1}) - a(u_{2}, v; \omega_{2})
% %             \\&= c \skp{\grad u_{1} - \grad u_{2}}{\grad v}{H} + \skp{\omega_{1}u_{1} - \omega_{2} u_{2}}{v}{H} + \mu \skp{u_{1} - u_{2}}{v}{H},
% %             \intertext{durch setzen von $z = u_{1} - u_{2}$ erhalten wir weiter}
% %             0 &= c \skp{\grad z}{\grad v}{H} + \skp{\omega_{1} z}{v}{H} + \mu \skp{z}{v}{H} + \skp{(\omega_{1} - \omega_{2}) u_{2}}{v}{H}
% %             \\&= a(z, v; \omega_{1}) + \skp{(\omega_{1} - \omega_{2}) u_{2}}{v}{H}.
% %         \end{align}
% %         Dies lässt sich nun wieder in Form des Variationsproblems schreiben, konkret
% %         \begin{equation}
% %             a(z, v; \omega_{1}) = g(v) \quad \fa v \in V,
% %         \end{equation}
% %         mit
% %         \begin{equation}
% %             g(v) = - \skp{(\omega_{1} - \omega_{2}) u_{2}}{v}{H}.
% %         \end{equation}

% %         Nach \cref{satz:pp:lax_auf_elliptisch} ist die Lösung $z = u_{1} - u_{2} \in V$ eindeutig und erfüllt
% %         \begin{equation}
% %             \norm{z}_{V} \leq \frac{\norm{g}_{V'}}{\gamma_{0}}.
% %         \end{equation}

% %         Die Operatornorm von $g$ lässt sich mittels der Cauchy-Schwarz-Ungleichung bestimmen zu
% %         \begin{equation}
% %             \begin{aligned}
% %                 \norm{g}_{V'}
% %                   &=    \sup_{\norm{v}_{V} = 1} \abs{g(v)}
% %                    =    \sup_{\norm{v}_{V} = 1} \abs{\skp{(\omega_{1} - \omega_{2}) u_{2}}{v}{H}}
% %                 \\&\leq \sup_{\norm{v}_{V} = 1} \norm{\omega_{1} - \omega_{2}}_{L_{\infty}(\Omega)} \norm{u_{1}}_{H} \norm{v}_{H}
% %                    \leq \sup_{\norm{v}_{V} = 1} \norm{\omega_{1} - \omega_{2}}_{L_{\infty}(\Omega)} \norm{u_{1}}_{V} \norm{v}_{V}
% %                 \\&=    \norm{\omega_{1} - \omega_{2}}_{L_{\infty}(\Omega)} \norm{u_{1}}_{V}
% %                    \leq \norm{\omega_{1} - \omega_{2}}_{L_{\infty}(\Omega)} \frac{\norm{f}_{V'}}{\gamma_{0}}.
% %             \end{aligned}
% %         \end{equation}
% %         Zusammen liefert dies die Ungleichung
% %         \begin{equation}
% %             \norm{u_{1} - u_{2}}_{V}
% %             = \norm{z}_{V} \leq \norm{\omega_{1} - \omega_{2}}_{L_{\infty}(\Omega)} \frac{\norm{f}_{V'}}{\gamma_{0}^{2}}
% %         \end{equation}
% %         und damit die Behauptung.
% %     \end{Beweis}
% % \end{Lemma}

% % \begin{Satz}
% %     Die Abbildung $\mathcal S \ni \sigma \mapsto u(\sigma) \in V$ besitzt für alle $\nu \in \mathfrak F$ die partielle Ableitung $\partial^{\nu}_{\sigma} u(\sigma)$.

% %     \begin{Beweis}
% %         Wir zeigen die Behauptung exemplarisch für für die partiellen Ableitungen erster Ordnung für ein festes $\sigma \in \mathcal S$.
% %         Seien dazu $\nu = e_{j}$ für ein $j \in \mathbb{N}$ und sei weiter $h \in \mathbb{R} \setminus \Set{ 0 }$.
% %         Definiere
% %         \begin{equation}
% %             u_{h} := \frac{u(\sigma + h e_{j}) - u(\sigma)}{h}.
% %         \end{equation}

% %         \todo[inline]{wohldfiniertheit?}
% %         Ist $\abs{h}$ klein genug, so dass $\sigma + h e_{j} \in \mathcal S$ gilt, dann existieren die eindeutigen Lösungen $u(\sigma + h e_{j})$ und $u(\sigma)$ des Variationsproblems (??) zu den jeweiligen Parametern.
% %         Betrachte nun die Differenz der beiden Variationsformulierungen, dann gilt
% %         \begin{align}
% %             0 &= a(u(\sigma + h e_{j}), v; \sigma + h e_{j}) - a(u(\sigma), v; \sigma)
% %             %
% %             \\ &= c \skp{u(\sigma + h e_{j})}{v}{L_{2}(\Omega)} + \skp{\omega(\sigma + h e_{j}) u(\sigma + h e_{j})}{v}{L_{2}(\Omega)} + \mu \skp{u(\sigma + h e_{j})}{v}{L_{2}(\Omega)}
% %             \\&\qquad - c \skp{u(\sigma)}{v}{L_{2}(\Omega)} - \skp{\omega(\sigma) u(\sigma)}{v}{L_{2}(\Omega)} - \mu \skp{u(\sigma)}{v}{L_{2}(\Omega)}
% %             %
% %             \\&= c \skp{u(\sigma + h e_{j}) - u(\sigma)}{v}{L_{2}(\Omega)} + \skp{\omega(\sigma + h e_{j}) u(\sigma + h e_{j}) - \omega(\sigma) u(\sigma)}{v}{L_{2}(\Omega)}
% %             \\&\qquad  + \mu \skp{u(\sigma + h e_{j}) - u(\sigma)}{v}{L_{2}(\Omega)}
% %             %
% %             \\&= c \skp{u(\sigma + h e_{j}) - u(\sigma)}{v}{L_{2}(\Omega)} + \skp{\omega(\sigma) (u(\sigma + h e_{j}) - u(\sigma))}{v}{L_{2}(\Omega)}
% %             \\&\qquad + \skp{(\omega(\sigma + h e_{j}) - \omega(\sigma)) u(\sigma + h e_{j})}{v}{L_{2}(\Omega)} + \mu \skp{u(\sigma + h e_{j}) - u(\sigma)}{v}{L_{2}(\Omega)}
% %             %
% %             \\&= h \left(
% %                 c \skp{u_{h}}{v}{L_{2}(\Omega)} + \skp{\omega(\sigma) u_{h}}{v}{L_{2}(\Omega)} + \mu \skp{u_{h}}{v}{L_{2}(\Omega)}
% %              \right) + h \skp{\varphi_{j} u(\sigma + h e_{j})}{v}{L_{2}(\Omega)}
% %             \\&= h a(u_{h}, v; \sigma) + h \skp{\varphi_{j} u(\sigma + h e_{j})}{v}{L_{2}(\Omega)}
% %         \end{align}
% %         Dies lässt sich zu folgendem Variationsproblem
% %         \begin{equation}
% %             a(u_{h}, v; \sigma) = F_{h}(v) \quad \fa v \in V.
% %         \end{equation}
% %         Dabei ist
% %         \begin{equation}
% %             F_{h} \colon V \to \mathbb{R}, \quad v \mapsto - \skp{\varphi_{j} u(\sigma + h e_{j})}{v}{L_{2}(\Omega)}
% %         \end{equation}
% %         ein lineares stetiges Funktional.
% %         Weiter ist $F_{h}(\blank)$ stetig in $h = 0$, denn es gilt für festes $v \in V$ die Abschätzung
% %         \begin{align}
% %             \abs{F_{h}(v) - F_{0}(v)}
% %             &= \abs{\skp{\varphi_{j} (u(\sigma + h e_{j}) - u(\sigma))}{v}{L_{2}(\Omega)}}
% %             \\&\leq \norm{\varphi_{j}}_{L_{\infty}(\Omega)} \abs{\skp{u(\sigma + h e_{j}) - u(\sigma)}{v}{L_{2}(\Omega)}}
% %             \\&\leq \norm{\varphi_{j}}_{L_{\infty}(\Omega)} \norm{u(\sigma + h e_{j}) - u(\sigma)}_{L_{2}(\Omega)} \norm{v}_{L_{2}(\Omega)}
% %             \\&\leq \norm{\varphi_{j}}_{L_{\infty}(\Omega)} \norm{u(\sigma + h e_{j}) - u(\sigma)}_{H^{1}(\Omega)} \norm{v}_{H^{1}(\Omega)}.
% %         \end{align}
% %         Weiter liefert \cref{lemma:pp:norm_abschaetzung} die Ungleichung
% %         \begin{align}
% %             \norm{u(\sigma + h e_{j}) - u(\sigma)}_{H^{1}(\Omega)}
% %             &\leq \frac{\norm{f}_{V'}}{\gamma_{0}^{2}} \norm{\omega(\sigma + h e_{j}) - \omega(\sigma)}_{L_{\infty}(\Omega)}
% %             \\&\leq \abs{h} \norm{\varphi_{j}}_{L_{\infty}(\Omega)} \frac{\norm{f}_{V'}}{\gamma_{0}^{2}}.
% %         \end{align}
% %         Zusammen liefern die beiden Abschätzungen
% %         \begin{equation}
% %             \abs{F_{h}(v) - F_{0}(v)} \leq \abs{h} \norm{\varphi_{j}}^{2}_{L_{\infty}(\Omega)} \norm{v}_{H^{1}(\Omega)} \frac{\norm{f}_{V'}}{\gamma_{0}^{2}} \to 0 \quad \text{für}~h \to 0,
% %         \end{equation}
% %         das heißt es gilt $F_{h} \to F_{0}$ in $V'$ für $h \to 0$.

% %         Dies impliziert insbesondere $u_{h} \to u_{0}$ in $V$ für $h \to 0$.
% %         Weiter erfüllt $u_{0}$ die Gleichung
% %         \begin{equation}
% %             a(u_{0}, v; \sigma) = F_{0}(v) \quad \fa v \in V.
% %         \end{equation}
% %         Damit existiert $\partial_{\sigma_{j}} u(\sigma) = u_{0}$ in $V$ und ist die eindeutige Lösung des Variationsproblems
% %         \begin{equation}
% %             a(\partial_{\sigma_{j}} u(\sigma), v; \sigma) = - \skp{\varphi_{j} u(\sigma)}{v}{L_{2}(\Omega)} \quad \fa v \in V.
% %         \end{equation}

% %         Alternativ erhält man dieses Variationsproblem auch durch formales Differenzieren der eigentlichen Variationsformulierung nach $\sigma_{j}$, denn dann gilt
% %         \begin{align}
% %             \partial_{\sigma_{j}} a(u(\sigma), v; \sigma)
% %             &= \partial_{\sigma_{j}} \left( c \skp{\grad u(\sigma)}{\grad v}{L_{2}(\Omega)} + \skp{w(\sigma)u(\sigma)}{v}{L_{2}(\Omega)} + \mu \skp{u(\sigma)}{v}{L_{2}(\Omega)} \right)
% %             \\&= c \skp{\grad \partial_{\sigma_{j}} u(\sigma)}{\grad v}{L_{2}(\Omega)} + \skp{ \partial_{\sigma_{j}} w(\sigma)u(\sigma) + w(\sigma) u(\sigma) }{v}{L_{2}(\Omega)} + \mu \skp{\partial_{\sigma_{j}}u(\sigma)}{v}{L_{2}(\Omega)}
% %             \\&= a(\partial_{\sigma_{j}} u(\sigma), v; \sigma) + \skp{\partial_{\sigma_{j}}\omega(\sigma)u(\sigma)}{v}{L_{2}(\Omega)}
% %             \\&= a(\partial_{\sigma_{j}} u(\sigma), v; \sigma) + \skp{\varphi_{j} u(\sigma)}{v}{L_{2}(\Omega)}
% %         \end{align}
% %         und
% %         \begin{equation}
% %             \partial_{\sigma_{j}} \skp{f}{v}{L_{2}(\Omega)} = 0,
% %         \end{equation}
% %         woraus man ingesamt erneut das Variationsproblem
% %         \begin{equation}
% %             a(\partial_{\sigma_{j}} u(\sigma), v; \sigma) = - \skp{\varphi u(\sigma)}{v}{L_{2}(\Omega)} \quad \fa v \in V
% %         \end{equation}
% %         erhält.

% %         Durch wiederholtes Anwenden des beschriebenen Vorgehens erhält man damit auch die Existenz der höheren partiellen Ableitungen.
% %     \end{Beweis}
% % \end{Satz}

% % Seien $b := (b_{j})_{j} \in \mathbb{R}$ und $b_{j} := \frac{\norm{\varphi_{j}}_{L_{\infty}}}{\gamma_{0}}$.

% % \begin{Satz}
% %     Es gilt
% %     \begin{equation}
% %         \sup_{\sigma \in \mathcal S} \norm{\partial^{\nu}_{\sigma} u(\sigma)} \leq B \abs{\nu}! b^{\nu}.
% %     \end{equation}

% %     \begin{Beweis}
% %         Wir betrachten zunächst die Variationsprobleme, welche von den partiellen Ableitungen $\partial^{\nu}_{\sigma} u(\sigma)$ erfüllt werden.
% %         Diese Variationsprobleme haben die folgende rekurisve Darstellung
% %         \begin{equation}
% %             a(\partial^{\nu}_{\sigma} u(\sigma), v; \sigma)
% %             = - \sum_{\Set{j \given \nu_{j} \neq 0}} \nu_{j} \skp{\varphi_{j} \partial^{\nu - e_{j}}_{\sigma} u(\sigma)}{v}{L_{2}(\Omega)}.
% %         \end{equation}
% %         Die Gültigkeit dieser Formel beweisen wir per Induktion.
% %         Der Fall $\abs{\nu} = 1$ ergibt sich aus (??).
% %         Betrachte also $abs{\nu} > 1$.
% %         Sei $k \in \mathbb{N}$ ein Index mit $\nu_{k} \neq 0$, dann definieren wir $\tilde{\nu} := \nu - e_{k}$ und es gilt offenbar $\abs{\tilde{\nu}} = \abs{\nu} - 1$.
% %         Nach Induktionsvoraussetzung gilt weiter
% %         \begin{equation}
% %             a(\partial^{\tilde{\nu}}_{\sigma} u(\sigma), v; \sigma) + \sum_{\Set{j \given \tilde{\nu}_{j} \neq 0}} \tilde{\nu}_{j} \skp{\varphi_{j} \partial^{\tilde{\nu} - e_{j}}_{\sigma} u(\sigma)}{v}{L_{2}(\Omega)} = 0,
% %         \end{equation}
% %         wobei $\nu_{j} = \tilde{\nu}_{j}$ für $j \neq k$ und $\tilde{\nu}_{k} = \nu_{k} - 1$ gilt.

% %         Partielles Differenzieren dieser Gleichung nach $\sigma_{k}$ liefert
% %         \begin{align}
% %             0 &=
% %                 a(\partial^{\nu}_{\sigma} u(\sigma), v; \sigma)
% %                 + \skp{\varphi_{k} \partial^{\nu - e_{k}}_{\sigma} u(\sigma)}{v}{L_{2}(\Omega)}
% %                 + \sum_{\Set{j \neq k \given \nu_{j} \neq 0}} \nu_{j} \skp{\varphi_{j} \partial^{\nu - e_{j}}_{\sigma} u(\sigma)}{v}{L_{2}(\Omega)}
% %             \\&\qquad     + (\nu_{k} - 1) \skp{\varphi_{k} \partial^{\nu - e_{k}}_{\sigma} u(\sigma) }{v}{L_{2}(\Omega)}
% %         \end{align}

% %         Wählt man nun $v = \partial^{\nu}_{\sigma} u(\sigma)$, dann gilt damit aufgrund der Koerzivität von $a(\blank, \blank; \sigma)$ die Ungleichung
% %         \begin{equation}
% %             a(\partial^{\nu}_{\sigma} u(\sigma), \partial^{\nu}_{\sigma} u(\sigma); \sigma) \geq \alpha \norm{\partial^{\nu}_{\sigma} u(\sigma)}_{H^{1}(\Omega)}^{2}.
% %         \end{equation}
% %         Außerdem erhalten wir mit Hilfe er Cauchy-Schwarz-Ungleichung die Abschätzung
% %         \begin{align}
% %             a(\partial^{\nu}_{\sigma} u(\sigma), \partial^{\nu}_{\sigma} u(\sigma); \sigma)
% %             &= - \sum_{\Set{j \given \nu_{j} \neq 0}} \nu_{j} \skp{\varphi_{j} \partial^{\nu - e_{j}}_{\sigma} u(\sigma) }{\partial^{\nu}_{\sigma} u(\sigma)}{L_{2}(\Omega)}
% %             \\&\leq \sum_{\Set{j \given \nu_{j} \neq 0}} \nu_{j} \norm{\varphi_{j}}_{L_{\infty}(\Omega)} \norm{\partial^{\nu - e_{j}}_{\sigma} u(\sigma)}_{H^{1}(\Omega)} \norm{\partial^{\nu}_{\sigma} u(\sigma)}_{H^{1}(\Omega)}.
% %         \end{align}
% %         Beide Ungleichungen zusammen ergeben
% %         \begin{equation}
% %             \norm{\partial^{\nu}_{\sigma} u(\sigma)}_{H^{1}(\Omega)} \leq \sum_{\Set{j \given \nu_{j} \neq 0}} \nu_{j} \frac{\norm{\varphi_{j}}_{L_{\infty}(\Omega)}}{\alpha} \norm{\partial^{\nu - e_{j}}_{\sigma} u(\sigma)}_{H^{1}(\Omega)}.
% %         \end{equation}

% %         Um nun die eigentliche Behauptung zu beweisen, verfolgen wir erneut einen Induktionsansatz.
% %         Sei zunächst $\abs{\nu} = 0$, dann haben wir die Ungleichung
% %         \begin{equation}
% %             \norm{u(\sigma)}_{H^{1}(\Omega)} \leq \frac{\norm{\varphi_{j}}_{L_{\infty}(\Omega)}}{\alpha},
% %         \end{equation}
% %         welche nach Satz (??) erfüllt ist.
% %         Sei also weiter $\abs{\nu} > 0$, dann gilt mit obiger Ungleichung unter Verwendung der Induktionsvoraussetzung für $\norm{\partial^{\nu - e_{j}}_{\sigma} u(\sigma)}_{H^{1}(\Omega)}$ die Abschätzung
% %         \begin{align}
% %             \norm{\partial^{\nu}_{\sigma} u(\sigma)}_{H^{1}(\Omega)}
% %             &\leq
% %             \sum_{\Set{j \given \nu_{j} \neq 0}} \nu_{j} \frac{\norm{\varphi_{j}}_{L_{\infty}(\Omega)}}{\alpha} \norm{\partial^{\nu - e_{j}}_{\sigma} u(\sigma)}_{H^{1}(\Omega)}
% %             \\&\leq
% %             \sum_{\Set{j \given \nu_{j} \neq 0}} \nu_{j} \frac{\norm{\varphi_{j}}_{L_{\infty}(\Omega)}}{\alpha} B \abs{\nu - e_{j}}! b^{\nu - e_{j}}
% %             \\&\leq
% %             \left( \sum_{\Set{j \given \nu_{j} \neq 0}} \nu_{j} \right)  \frac{\norm{\varphi_{j}}_{L_{\infty}(\Omega)}}{\alpha} B (\abs{\nu} - 1)! b^{\nu}
% %             \\&\leq
% %             \frac{\norm{\varphi_{j}}_{L_{\infty}(\Omega)}}{\alpha} B \abs{\nu}! b^{\nu}
% %         \end{align}
% %     \end{Beweis}
% % \end{Satz}

% % \todo[inline]{Daraus folgern, dass es für den parabolischen auch gilt.}

% % section regularit_t_bez_glich_des_parameters (end)

% % chapter parametrische_problem_neuer_versuch (end)


% \chapter{Ab hier geht der Müll los} % (fold)
% \label{cha:ab_hier_geht_der_m_ll_los}

% % chapter ab_hier_geht_der_m_ll_los (end)

% \todo[inline]{Alten Kram entfernen}

% \todo[inline]{Anpassen an Zeitabhängige lineare Operatoren bzw. Bilinearformen!}

% In diesem Kapitel liegt das Augenmerk erneut auf der linearen Evolutionsgleichung \cref{eq:allgemeine_parabolische_pde}, diesmal aber mit der Erweiterung, dass der lineare Operator $A(t)$ zusätzlich von einem Parameter $\sigma$ abhängt.

% Zunächst konkretisieren wir diese Parameterabhängigkeit für einen linearen Operator $A$, betrachten dann eine parametrische lineare Operatorgleichung, leiten Regularitätsergebnisse für diese her und übertragen diese anschließend auf die Raum-Zeit-Variationsformulierung einer parametrischen linearen Evolutionsgleichung.
% Dabei orientieren wir uns hauptsächlich an den Arbeiten von \textcite{Kunoth:2013ef,Cohen:2010kz}.

% \section{Parametrische Operatorgleichung} % (fold)
% \label{sec:parametrische_operatorgleichung}

% Seien $X$ und $Y$ zwei reflexive Banachräume.
% Weiter sei $\mathcal S \subset \mathbb{R}^{\mathbb{N}}$ der sogenannte Parameterraum.
% Der Einfachheit halber wählen wir $\mathcal S = [-1, 1]^{\mathbb{N}}$.
% \todo[inline]{Warum reicht das?}

% Wir betrachten parametrische Familien stetiger linearer Operatoren $A(\sigma) \in \mathcal L(X, Y')$ mit $\sigma \in \mathcal S$.
% Folgende lineare Operatorgleichung ist für uns von Interesse:
% Sei ein $g \in Y'$ gegeben.
% Finde für alle $\sigma \in \mathcal S$ eine Lösung $u(\sigma) \in X$ von
% \begin{equation}
%     \label{eq:allgemeine_parametrische_elliptische_pde}
%     A(\sigma) u(\sigma) = g \quad \text{in}~Y'.
% \end{equation}
% Wie zuvor sei $a(\blank, \blank; \sigma) \colon X \times Y \to \mathbb{R}$ die zugehörige Bilinearform.

% Zunächst einige notationelle Vorbemerkungen.
% \begin{Bemerkung}
%     Wir bezeichnen mit $\mathfrak F = \Set{ \nu \in \mathbb{N}^{\mathbb{N}}_{0} \given \abs{\nu} < \infty }$ die Menge aller Folgen nichtnegativer ganzer Zahlen mit endlichem Träger, das heißt nur endlich vielen Einträgen ungleich Null.
%     % NOTE: Eventuell mehr definieren, siehe $\mathfrak n$ und $\mathfrak m$

%     Sei $\nu \in \mathfrak F$ und $b \in \ell_{p}(\mathbb{N})$, $p > 0$, dann schreiben wir
%     \begin{equation}
%         b^{\nu} = \prod_{j = 1}^{\infty} b_{j}^{\nu_{j}}
%     \end{equation}
%     mit der Konvention $0^{0} = 1$.
%     Wegen $\abs{\nu} < \infty$ ist dieses Produkt stets endlich.
% \end{Bemerkung}

% Für die nachfolgenden Regularitätsaussagen über die Lösung $u(\sigma)$ von \cref{eq:allgemeine_parametrische_elliptische_pde} benötigen wir Regularität der Operatorfamilie $A(\sigma)$ bezüglich $\sigma \in \mathcal S$.
% Konkret fordern wir:
% \begin{Annahme}[{{\cite[Assumption 1]{Kunoth:2013ef}}}]
% \label{thm:kunoth:assumption1}
%     Die parametrische Familie von Operatoren
%     $\Set{ A(\sigma) \in \mathcal L(X, Y') \given \sigma \in \mathcal S }$ sei eine $\mathfrak p$-reguläre Operatorfamilie für ein $0 < \mathfrak p \leq 1$, das heißt,
%     \begin{thmenumerate}
%         \item $A(\sigma) \in \mathcal L(X, Y')$ sei stetig invertierbar für alle $\sigma \in \mathcal S$ mit gleichmäßig beschränktem Inversen $A{(\sigma)}^{-1} \in \mathcal L(Y', X)$, das heißt, es existiert ein $C_{0} > 0$ mit
%         \begin{equation}
%             \sup_{\sigma \in \mathcal S} \norm{A{(\sigma)}^{-1}}_{\mathcal L(Y', X)} \leq C_{0},
%         \end{equation}
%         \item für jedes feste $\sigma \in \mathcal S$ seien die Operatoren $A(\sigma)$ analytisch bezüglich $\sigma$.
%         Konkret existiert eine nichtnegative Folge $b = (b_{j})_{j \in \mathbb{N}} \in \ell_{\mathfrak p}(\mathbb{N})$, so dass
%         \begin{equation}
%             \sup_{\sigma \in \mathcal S} \norm{(A{(0)})^{-1}(\partial^{\nu}_{\sigma} A(\sigma))}_{\mathcal L(X, X)} \leq C_{0} b^{\nu}
%         \end{equation}
%         für alle $\nu \in \mathfrak F \setminus \{ 0 \}$ gilt.
%         Dabei sei $\partial^{\nu}_{\sigma} A(\sigma) \deq \frac{\partial^{\nu_{1}}}{\partial \sigma_{1}} \frac{\partial^{\nu_{2}}}{\partial \sigma_{2}} \cdots A(\sigma)$.
%     \end{thmenumerate}
% \end{Annahme}

% Die bisherigen Anforderungen an $A(\sigma)$ decken einen noch sehr weiten Bereich ab.
% Wir beschränken uns in dieser Arbeit aus praktischen Gründen ausschließlich auf den folgenden Fall, der affin parametrischen Operatoren.

% \begin{Definition}
%     Sei $\Set{ A(\sigma) \in \mathcal L(X, Y') \given \sigma \in \cal S }$ eine parametrische Operatorfamilie.
%     Wir nennen $A(\sigma)$ einen \emph{affin parametrischen Operator}, falls eine Familie von Operatoren $\Set{ \hat A, A_{j} \given j \in \mathbb{N} } \subset \cal L(X, Y')$ existiert, so dass
%     \begin{equation}
%         \label{eq:all_affiner_operator}
%         A(\sigma) = \hat A + \sum_{j = 1}^{\infty} \sigma_{j} A_{j} \qquad\fa \sigma \in \mathcal S
%     \end{equation}
%     gilt.
% \end{Definition}

% Seien $\hat a, a_{j} \colon X \times Y \to \mathbb{R}$ die durch den Rieszschen Darstellungssatz von $\hat A$ respektive $A_{j}$ induzierten Bilinearformen, das heißt also,
% \begin{equation}
%     \label{eq:allg_affine_bf}
%     \begin{aligned}
%     \hat a(\eta, \zeta) &= \skprod{\hat A \eta}{\zeta}_{Y' \times Y}
%     \\
%     a_{j}(\eta, \zeta) &= \skprod{A_{j} \eta}{\zeta}_{Y' \times Y}, \quad j \in \mathbb{N},
%     \end{aligned}
% \end{equation}
% für $\eta \in X$, $\zeta \in Y$.

% Um die Wohldefiniertheit von $A(\sigma)$, das heißt Konvergenz von \cref{eq:all_affiner_operator}, sicherzustellen, stellen wir folgende Bedingungen:
% \begin{Annahme}[{{\cite[Assumption 2]{Kunoth:2013ef}}}]
% \label{thm:kunoth:assumption2}
%     Die Operatorfamilie $\Set{\hat A, A_{j} \given j \in \mathbb{N}}$ erfülle folgende Eigenschaften:
%     \begin{thmenumerate}
%         \item Der \emph{Mean Field}-Operator $\hat A \in \mathcal L(X, Y')$ sei stetig invertierbar, das heißt, es existiert ein $\gamma_{0} > 0$ mit
%         \begin{subequations}\label{eq:kunoth:ass2_gamma_0}
%             \begin{align}
%                 \label{eq:kunoth:ass2_gamma_0_a}
%                 \inf_{0 \neq u \in X} \sup_{0 \neq v \in Y} \frac{\hat a(u, v)}{\norm{u}_{X} \norm{v}_{Y}} \geq \gamma_{0}
%                 \intertext{und}
%                 \label{eq:kunoth:ass2_gamma_0_b}
%                 \inf_{0 \neq v \in Y} \sup_{0 \neq u \in X} \frac{\hat a(u, v)}{\norm{u}_{X} \norm{v}_{Y}} \geq \gamma_{0}.
%             \end{align}
%         \end{subequations}
%         \item Die \emph{Fluctuation}-Operatoren $\Set{ A_{j} }_{j \geq 1}$ seien \emph{klein} relativ zu $\hat A$ im folgenden Sinne: es existiert eine Konstante $0 < \kappa < 1$ so dass
%         \begin{equation}
%             \label{eq:kunoth:ass2_abs_reihe}
%             \sum_{j = 1}^{\infty} \norm{A_{j}}_{\mathcal L(X, Y')} \leq \kappa \gamma_{0}
%         \end{equation}
%         gilt.
%     \end{thmenumerate}
% \end{Annahme}

% Unter diesen Bedingungen liefert das Banach-Ne{\v c}as-Babu{\v s}ka-Theorem, \cref{satz:gl:bnb_theorem}, die stetige Invertierbarkeit von $A(\sigma)$ aus \cref{eq:all_affiner_operator} gleichmäßig in $\sigma$.

% \begin{Satz}[{{\cite[Theorem 2]{Kunoth:2013ef}}}]
%     Der affin parametrische Operator $A(\sigma)$ erfülle \cref{thm:kunoth:assumption2}.
%     Dann ist $A(\sigma)$ für alle $\sigma \in \mathcal S$ stetig invertierbar.

%     Konkret gilt
%     \begin{equation}
%         \inf_{u \in H_{1}} \sup_{v \in H_{2}} \frac{a(u, v)}{\norm{u}_{H_{1}} \norm{v}_{H_{2}}} \geq (1 - \kappa) \gamma_{0} > 0 \quad \fa \sigma \in \mathcal S
%     \end{equation}
%     und
%     \begin{equation}
%         \inf_{v \in H_{2}} \sup_{u \in H_{1}} \frac{a(u, v)}{\norm{u}_{H_{1}} \norm{v}_{H_{2}}} \geq (1 - \kappa) \gamma_{0} > 0 \quad \fa \sigma \in \mathcal S.
%     \end{equation}

%     Ist ferner ein $g \in Y'$ gegeben, dann existiert für jedes $\sigma \in \mathcal S$ ein $\hat u(\sigma) \in X$ mit
%     \begin{equation}
%         a(\hat u(\sigma), v; \sigma) = \skprod{g}{v}_{Y' \times Y} \quad \fa v \in Y
%     \end{equation}
%     und es gilt die A-Priori-Abschätzung
%     \begin{equation}
%         \sup_{\sigma \in \mathcal S} \norm{\hat u(\sigma)}_{X} \leq \frac{\norm{g}_{Y'}}{(1 - \kappa) \gamma_{0}}.
%     \end{equation}

%     \begin{Beweis}
%         Nachrechnen der beiden inf-sup-Bedingungen unter Verwendung der affinen Zerlegung von $A(\sigma)$ und anschließendes Anwenden des Banach-Ne{\v c}as-Babu{\v s}ka-Theorems liefert die gewünschten Aussagen.
%     \end{Beweis}
% \end{Satz}

% \begin{Korollar}[{{\cite[Corollary 3]{Kunoth:2013ef}}}]
% \label{thm:kunoth:corollary3}
%     Die affin parametrische Operatorfamilie $\Set{\hat A, A_{j} \given j \in \mathbb{N}}$ erfülle \cref{thm:kunoth:assumption2}, dann wird auch \cref{thm:kunoth:assumption1} mit $\mathfrak p = 1$ und
%     \begin{equation}
%         C_{0} = \frac{1}{(1 - \kappa) \gamma_{0}}, \qquad b_{j} = \frac{\norm{A_{j}}_{\mathcal L(X, Y')}}{(1 - \kappa) \gamma_{0}} \quad \fa j \in \mathbb{N},
%     \end{equation}
%     erfüllt.
% \end{Korollar}

% Weiter erhält man unter den Bedingungen aus \cref{thm:kunoth:assumption1} folgendes Regularitätsergebnis bezüglich des Parameters $\sigma$.

% \begin{Satz}[{{\cite[Theorem 4]{Kunoth:2013ef}}}]
% \label{thm:kunoth:theorem4}
%     Die parametrische Familie $\Set{ A(\sigma) \in \mathcal L(X, Y') \given \sigma \in \mathcal S }$ erfülle \cref{thm:kunoth:assumption1} für ein $0 < \mathfrak p \leq 1$.
%     Dann existiert für jedes $g \in Y'$ und jedes $\sigma \in \mathcal S$ eine eindeutige Lösung $u(\sigma) \in X$ der parametrischen Operatorgleichung
%     \begin{equation}
%         A(\sigma) u(\sigma) = g \quad \text{in}~Y'.
%     \end{equation}

%     Die parametrische Familie von Lösungen $u(\sigma)$ hängt analytisch vom Parameter $\sigma$ ab und die partiellen Ableitungen von $u(\sigma)$ erfüllen
%     \begin{equation}
%         \label{eq:kunoth:schranke_part_abl}
%         \sup_{\sigma \in \mathcal S} \norm{(\partial^{\nu}_{\sigma} u)(\sigma)}_{X} \leq C_{0} \norm{g}_{Y'} \abs{\nu}! \tilde{b}^{\nu}
%     \end{equation}
%     für alle $\nu \in \mathfrak F$, wobei die Folge $\tilde{b} = (\tilde{b}_{j})_{j \geq 1} \in \ell_{\mathfrak p}(\mathbb{N})$ definiert ist durch
%     \begin{equation}
%         \tilde{b}_{j} = \frac{b_{j}}{\ln 2} \qquad \text{für alle j} \in \mathbb{N}.
%     \end{equation}

%     \begin{Beweis}
%         \todo[inline]{Beweis?}
%     \end{Beweis}
% \end{Satz}

% % section parametrische_operatorgleichung (end)

% \section{Parametrische lineare Evolutionsgleichung} % (fold)
% \label{sec:parametrische_lineare_evolutionsgleichung}

% Dieser Abschnitt soll nun dazu dienen, aufbauend auf \cref{sec:lineare_evolutionsgleichungen} eine parametrische lineare Evolutionsgleichung zu definieren und anschließend die Regularitätsergebnisse aus dem vorherigen Abschnitt auf diese zu übertragen.

% Wir wiederholen kurz das Setting aus \cref{sec:lineare_evolutionsgleichungen}, in dem wir hier erneut arbeiten.
% Seien $V$ und $H$ separable Hilberträume mit einer dichten stetigen Einbettung von $V$ in $H$ und $(V, H, V')$ sei das zugehörige Gelfand-Tripel.
% Weiter seien ein $0 < T < \infty$ und ein endliches Zeitintervall $[0, T]$ gegeben.

% Wir bezeichnen $\mathcal S = [-1, 1]^{\mathbb{N}}$ weiterhin als Parameterraum.
% Es sei für fast alle $t \in [0, T]$ und für alle $\sigma \in \mathcal S$ eine Familie von Bilinearformen
% \begin{equation}
%     a(\blank, \blank; \sigma, t) \colon V \times V \to \mathbb{R}, \quad (\eta, \zeta) \mapsto a(\eta, \zeta; \sigma, t)
% \end{equation}
% gegeben, so dass $t \mapsto a(\eta, \zeta; \sigma, t)$ für alle $\sigma \in \mathcal S$ messbar auf $[0, T]$ ist.
% Analog zu \cref{annahme:eigenschaften_bf_a} fordern wir diesmal für den Rest dieses Abschnitts:
% \begin{Annahme}
% \label{annahme:pp:eigenschaften_bf_a}
%     \leavevmode
%     \begin{thmenumerate}
%         \item \emph{Stetigkeit.}
%         Es existiert eine Konstante $0 < M_{a} < \infty$, so dass
%         \begin{equation}
%             \label{eq:allgemeine_parabolische_pde:bf_stetig}
%             \abs{a(\eta, \zeta; \sigma, t)} \leq M_{a} \norm{\eta}_{V} \norm{\zeta}_{V} \quad \fa \eta, \zeta \in V
%         \end{equation}
%         für fast alle $t \in [0, T]$ und alle $\sigma \in \mathcal S$ gilt.
%         \item \emph{G\r{a}rding-Ungleichung}.
%         Es existieren Konstanten $\alpha > 0$ und $\lambda \geq 0$ mit
%         \begin{equation}
%             \label{eq:allgemeine_parabolische_pde:bf_garding}
%             a(\eta, \eta; \sigma, t) + \lambda \norm{\eta}_{H}^{2} \geq \alpha \norm{\eta}_{V}^{2} \quad \fa \eta \in V
%         \end{equation}
%         für fast alle $t \in [0, T]$ und alle $\sigma \in \mathcal S$.
%     \end{thmenumerate}
% \end{Annahme}

% Unter diesen Voraussetzungen existiert nach dem Rieszschen Darstellungssatz für jedes $\sigma \in \mathcal S$ und fast alle $t \in [0, T]$ ein stetiger linearer Operator $A(\sigma, t) \in \mathcal L(V, V')$ und es gilt für alle $\sigma \in \mathcal S$ die Gleichheit
% \begin{equation}
%     \skprod{A(\sigma, t) \eta}{\zeta} = a(\eta, \zeta; \sigma, t) \quad \eta, \zeta \in V.
% \end{equation}

% Vollkommen analog zur Herleitung der Raum-Zeit-Variationsformulierung in \cref{sec:raum_zeit_variationsformulierung} erhalten wir damit das folgende parametrische Raum-Zeit-Variationsproblem:

% \begin{Definition}
% \label{definition:pp:variationsformulierung}
%     Seien $\mathcal X$ und $\mathcal Y$ wie in \cref{definition:gl:ansatz_und_testraum}.
%     Als \emph{parametrische Raum-Zeit-Variationsfor"-mu"-lie"-rung}
%     %der linearen Evolutionsgleichung~\cref{eq:allgemeine_parabolische_pde}
%     bezeichnen wir das folgende Problem:

%     Seien ein Quellterm $g \in L_{2}(0, T; V')$ und ein Anfangswert $u_{0} \in H$ gegeben.
%     Finde für alle $\sigma \in \mathcal S$ ein $u(\sigma) \in \mathcal X$ mit
%     \begin{equation}
%         \label{eq:pp:var_all_problem}
%         b(u(\sigma), v; \sigma) = f(v) \quad \fa v \in \mathcal Y.
%     \end{equation}
%     Dabei ist $b \colon \mathcal X \times \mathcal Y \to \mathbb{R}$ die durch
%     \begin{equation}
%         \label{eq:pp:var_all_bf_b}
%         b(u, v; \sigma) = \int_{0}^{T} \skprod{u_{t}(t)}{v_{1}(t)}_{H} + a(u(t), v_{1}(t); \sigma, t) \diff t + \skprod{u(0)}{v_{2}}_{H}
%     \end{equation}
%     definierte Bilinearform und $f \colon \mathcal Y \to \mathbb{R}$ das durch
%     \begin{equation}
%         \label{eq:pp:var_all_f}
%         f(v) = \int_{0}^{T} \skprod{g(t)}{v_{1}(t)}_{H} \diff t + \skprod{u_{0}}{v_{2}}_{H}
%     \end{equation}
%     gegebene Funktional.
% \end{Definition}

% Als nächstes wollen wir nachweisen, dass obiges Raum-Zeit-Variationsproblem sachgemäß gestellt ist und zudem die Lösungen $u(\sigma)$ analytisch vom Parameter $\sigma \in \mathcal S$ abhängen.
% Ersteres erhalten wir analog zu \cref{thm:schwab09:theorem51} für den nichtparametrischen Fall.
% Bezüglich der Regularität stellt sich heraus, dass wir lediglich Bedingungen an die Familie von stetigen linearen Operatoren $\Set{ A(\sigma, t) \in \mathcal L(V, V') \given \sigma \in \mathcal S, t \in [0, T] }$ stellen müssen, wie folgender Satz zeigt:

% \begin{Satz}[{{\cite[Theorem 21]{Kunoth:2013ef}}}]
% \label{thm:kunoth:theorem21}
%     Seien $\mathcal X$ und $\mathcal Y$ gegeben wie in~\cref{eq:var_all_ansatzraum_x} respektive~\cref{eq:var_all_testraum_y}.
%     Weiter erfülle die Familie von Operatoren $\Set{ A(\sigma, t) \in \mathcal L(V, V') \given \sigma \in \mathcal S, t \in [0, T] }$ \cref{thm:kunoth:assumption1} für ein $0 < \mathfrak p \leq 1$.
%     Für jedes $\sigma \in \mathcal S$ sei $B(\sigma) \in \mathcal L(\mathcal X, \mathcal Y')$ definiert durch
%     \begin{equation}
%         \label{eq:var_all_gross_b_parametrisch}
%         \skprod{B(\sigma) u}{v}_{\mathcal Y' \times \mathcal Y} = b(u, v; \sigma), \quad u \in \mathcal X,~y \in \mathcal Y,
%     \end{equation}
%     mit $b(\blank, \blank; \sigma)$ wie in~\cref{eq:pp:var_all_bf_b}.
%     Dann ist $B(\sigma)$ für jedes $\sigma \in \mathcal S$ stetig invertierbar und es existieren Konstanten $0 < \beta_{1} \leq \beta_{2} < \infty$ mit
%     \begin{equation}
%         \label{eq:var_all_norm_B_und_B_inv_parametrisch}
%         \sup_{\sigma \in \mathcal S} \norm{B(\sigma)}_{\mathcal L(\mathcal X, \mathcal Y')} \leq \beta_{2} \quad \text{und} \quad  \sup_{\sigma \in \mathcal S} \norm{B(\sigma)^{-1}}_{\mathcal L(\mathcal Y', \mathcal X)} \leq \frac{1}{\beta_{1}}.
%     \end{equation}

%     Zudem erfüllt die parametrische Familie von Operatoren $\Set{ B(\sigma) \in \mathcal L(\mathcal X, \mathcal Y') \given \sigma \in \mathcal S }$ \cref{thm:kunoth:assumption1} mit dem gleichen Regularitätsparameter $\mathfrak p$, die parametrische Familie von Lösungen $u(\sigma)$ des parametrischen Raum-Zeit-Variationsproblems \cref{eq:pp:var_all_problem} hängt analytisch von $\sigma$ ab und erfüllt die A-Priori-Abschätzung
%     \begin{equation}
%         \label{eq:var_all_a_priori_schranke}
%         \sup_{\sigma \in \mathcal S} \norm{(\partial^{\nu}_{\sigma} u)(\sigma)}_{\mathcal X} \leq C_{0} \norm{f}_{\mathcal Y'} \abs{\nu}! \tilde{b}^{\nu}
%     \end{equation}
%     für alle $\nu \in \mathfrak F$, wobei $f$ wie in~\cref{eq:pp:var_all_f} gegeben ist.
% \end{Satz}

% \begin{Lemma}
% \label{lemma:norm_B_beschraenkt_durch_norm_A}
%     Sei $\sigma \in \mathcal S$ und $\nu \in \mathfrak F \setminus \Set{ 0 }$, dann gilt
%     \begin{equation}
%         \norm{\partial^{\nu}_{\sigma} B(\sigma)}_{\mathcal L(\mathcal X, \mathcal Y')}
%         \leq
%         \norm{\partial^{\nu}_{\sigma} A(\sigma)}_{\mathcal L(V, V')}
%     \end{equation}

%     \begin{Beweis}
%         \todo[inline]{Beweis}
%     \end{Beweis}
% \end{Lemma}

% \begin{Beweis}[\cref{thm:kunoth:theorem21}]
% \todo[inline]{Soll der ausgeführt werden?}
% Bedingungen von \cref{thm:kunoth:assumption1} nachrechnen.
% Zu (i): Folgt aus \cref{thm:schwab09:theorem51}, da $M_{a}, \alpha, \lambda$ unabhänging von $\sigma$.
% Zu (ii): Folgt aus nachfolgendem \cref{lemma:norm_B_beschraenkt_durch_norm_A}.
% \end{Beweis}

% section parametrische_lineare_evolutionsgleichung (end)

% chapter parametrisches_problem (end)

% \newpage

% \todo[inline]{Ebenfalls verarbeiten}

% \todo[inline]{Kapitel komplett überarbeiten und am besten nochmal nachrechnen.}

% In diesem Kapitel konzentrieren wir uns nun auf die in der Polymerchemie motivierte parabolische partielle Differentialgleichung.
% Eine ausführliche Herleitung findet sich bei \textcite{Fredrickson:2006th}.

% \section{Motivation} % (fold)
% \label{sec:motivation}

% \todo[inline]{schreiben!}

% % section motivation (end)

% \section{Vereinfachte Variante} % (fold)
% \label{sec:vereinfachte_variante}

% Wir betrachten in diesem Abschnitt zunächst eine vereinfachte Variante der vorgestellten Differentialgleichung.
% Zunächst ignorieren wir den Wechsel des Feldes $\omega$ ab einem bestimmten Zeitpunkt und erhalten dadurch einen autonomen linearen Differentialoperator $A$.
% Weiter schränken wir uns auf homogene Dirichlet- statt periodischen Randbedingungen ein.

% Unter diesen Gegebenheiten bietet es sich an, die Hilberträume als $V = H^{1}_{0}(\Omega)$ und $H = L_{2}(\Omega)$ zu wählen.
% Bekanntlich sind diese separabel und es existiert eine dichte stetige Einbettung von $H^{1}_{0}(\Omega)$ in $L_{2}(\Omega)$.
% Wegen $(H^{1}_{0}(\Omega))' = H^{-1}(\Omega)$ ergibt dies das Gelfand-Tripel
% \begin{equation}
%     H^{1}_{0}(\Omega) \denseinclusion L_{2}(\Omega) \denseinclusion H^{-1}(\Omega).
% \end{equation}
% Wie zuvor verwenden wir $\skprod{\blank}{\blank}$ mit entsprechendem Index sowohl für die Skalarprodukte als auch für die duale Paarung auf $H^{-1}(\Omega) \times H^{1}_{0}(\Omega)$.

% Um obige partielle Differentialgleichung in das Setting aus \cref{sec:lineare_evolutionsgleichungen} zu übertragen, definieren wir einen linearen Operator $A$ als
% \begin{equation}
%     \label{eq:def_op_A}
%     A \colon H^{1}_{0}(\Omega) \to H^{-1}(\Omega), \quad \eta \mapsto A \eta = - c \Delta \eta + \omega \eta
% \end{equation}
% und weiter die zugehörige Bilinearform
% \begin{equation}
%     a \colon H^{1}_{0}(\Omega) \times H^{1}_{0}(\Omega) \to \mathbb{R}, \quad a(\eta, \zeta) = \skprod{A \eta}{\zeta}_{L_{2}(\Omega)}.
% \end{equation}
% Diese lässt sich unter Verwendung der Greenschen Formeln (TODO!) auch schreiben als
% \begin{equation}
%     \begin{aligned}
%         a(\eta, \zeta)
%         &= \skprod{- c \Delta \eta + \omega \eta}{\zeta}_{L_{2}(\Omega)}
%         = - c \skprod{\Delta \eta}{\zeta}_{L_{2}(\Omega)} + \skprod{\omega \eta}{\zeta}_{L_{2}(\Omega)}
%         \\&= c \skprod{\grad \eta}{\grad \zeta}_{L_{2}(\Omega)} + \skprod{\omega \eta}{\zeta}_{L_{2}(\Omega)}.
%     \end{aligned}
% \end{equation}

% Diese Bilinearform ist stetig und erfüllt eine G\aa{}rding-Ungleichung, wie das folgende Lemma zeigt.

% \begin{Lemma}
% \label{lemma:a_bf_bounded_garding}
%     Seien $c \in \mathbb{R}_{+}$, $\omega \in L_{\infty}(\Omega)$ und
%     \begin{equation}
%     \label{eq:bf_a}
%         a \colon H^{1}_{0}(\Omega) \times H^{1}_{0}(\Omega) \to \mathbb{R}, \quad a(\eta, \zeta) = c \skprod{\grad \eta}{\grad \zeta}_{L_{2}(\Omega)} + \skprod{\omega \eta}{\zeta}_{L_{2}(\Omega)}.
%     \end{equation}
%     Dann erfüllt $a$ die Eigenschaften aus \cref{annahme:eigenschaften_bf_a}:
%     \begin{thmenumerate}
%         \item\label{lemma:a_bf_bounded_garding:1}
%         \emph{Stetigkeit:} es gilt
%         \begin{equation}
%             \abs{a(\eta, \zeta)} \leq M_{a} \norm{\eta}_{H^{1}(\Omega)} \norm{\zeta}_{H^{1}(\Omega)} \quad \text{für alle}~\eta, \zeta \in H^{1}_{0}(\Omega)
%         \end{equation}
%         mit $M_{a} = \max\Set{c, \norm{\omega}_{L_{\infty}(\Omega)} } \geq 0$.
%         \item\label{lemma:a_bf_bounded_garding:2}
%         \emph{G\aa{}rding-Ungleichung:} es gilt
%         \begin{equation}
%                 a(\eta, \eta) + \lambda \norm{\eta}_{L_{2}(\Omega)}^{2} \geq \alpha \norm{\eta}_{H^{1}(\Omega)}^{2} \quad \text{für alle}~\eta \in H^{1}_{0}(\Omega)
%         \end{equation}
%         mit $\alpha = c \gamma_{\Omega}^{2} > 0$ und $\lambda = \norm{\omega}_{L_{\infty}(\Omega)} \geq 0$, wobei $\gamma_{\Omega}$ die Poincaré-Friedrichs-Konstante ist.
%     \end{thmenumerate}

%     \begin{Beweis}
%     Wir zeigen zunächst die Stetigkeit.
%     Seien dazu $\eta, \zeta \in H^{1}_{0}(\Omega)$ beliebig.
%     Unter Verwendung der Dreiecks- und der Cauchy-Schwarz-Ungleichung erhalten wir
%     \begin{align}
%         \abs{a(\eta, \zeta)}
%         &= \abs{c \skprod{\grad \eta}{\grad \zeta}_{L_{2}(\Omega)} + \skprod{\omega \eta}{\zeta}_{L_{2}(\Omega)}}
%         \\&\leq c \abs{\skprod{\grad \eta}{\grad \zeta}_{L_{2}(\Omega)}} + \abs{\skprod{\omega \eta}{\zeta}_{L_{2}(\Omega)}}
%         \\&\leq c \norm{\grad \eta}_{L_{2}(\Omega)} \norm{\grad \zeta}_{L_{2}(\Omega)} + \norm{\omega}_{L_{\infty}(\Omega)} \norm{\eta}_{L_{2}(\Omega)} \norm{\zeta}_{L_{2}(\Omega)}
%         \\&\leq \max \Set{ c, \norm{\omega}_{L_{\infty}(\Omega)} } \norm{\eta}_{H^{1}(\Omega)} \norm{\zeta}_{H^{1}(\Omega)}.
%     \end{align}

%     Für die G\aa{}rding-Ungleichung seien nun $\eta \in H^{1}_{0}(\Omega)$ und $\lambda \in \mathbb{R}$.
%     Wir betrachten
%     \begin{align}
%         a(\eta, \eta) + \lambda \norm{\eta}^{2}_{L_{2}(\Omega)}
%         &= c \norm{\grad \eta}^{2}_{L_{2}(\Omega)} + \skprod{\omega \eta}{\eta}_{L_{2}(\Omega)} + \lambda \skprod{\eta}{\eta}_{L_{2}(\Omega)}
%         \\&= c \norm{\grad \eta}^{2}_{L_{2}(\Omega)} + \skprod{(\omega + \lambda) \eta}{\eta}_{L_{2}(\Omega)}.
%     \end{align}
%     Wählen wir nun $\lambda = \norm{\omega}_{L_{\infty}(\Omega)} \geq 0$, dann gilt $\omega + \lambda \geq 0$ fast überall in $\Omega$ und wir erhalten die Abschätzung
%     \begin{align}
%         a(\eta, \eta) + \lambda \norm{\eta}^{2}_{L_{2}(\Omega)}
%         &\geq c \norm{\grad \eta}^{2}_{L_{2}(\Omega)},
%         \intertext{woraus wir durch Anwenden der Poincaré-Friedrichs-Ungleichung \cref{satz:grundlagen:poincare_friedrichs_ungleichung}}
%         a(\eta, \eta) + \lambda \norm{\eta}^{2}_{L_{2}(\Omega)}
%         &\geq c \gamma_{\Omega}^{2} \norm{\eta}^{2}_{H^{1}(\Omega)}
%     \end{align}
%     folgern.
%     \end{Beweis}
% \end{Lemma}

% Unter diesen Gegebenheiten erhalten wir nach \cref{sec:lineare_evolutionsgleichungen} eine sachgemäß gestellte Raum-Zeit-Variationsformulierung.
% Ansatz- und Testfunktionenraum ergeben sich mit den konkret gewählten Hilberträumen zu
% \begin{equation}
%     \label{eq:var_ansatzraum_testraum}
%     \mathcal X = L_{2}(I; H^{1}_{0}(\Omega)) \cap H^{1}(I; H^{-1}(\Omega))
%     \quad \text{und} \quad
%     \mathcal Y = L_{2}(I; H^{1}_{0}(\Omega)) \times L_{2}(\Omega).
% \end{equation}
% Das Variationsproblem lautet damit:
%     Gegeben ein $g \in L_{2}(I; H^{-1}(\Omega))$ und ein $u_{0} \in L_{2}(\Omega)$. Finde ein $u \in \mathcal X$ mit
%     \begin{equation}
%         \label{eq:varprob}
%         b(u, v) = f(v) \quad \text{für alle}~v \in \mathcal Y,
%     \end{equation}
%     wobei $b(\blank, \blank) \colon \mathcal X \times \mathcal Y \to \mathbb{R}$ die durch
%     \begin{equation}
%         \label{eq:buv}
%         b(u, v)
%             = \int_{I} \skprod{u_{t}(t)}{v_{1}(t)}_{L_{2}(\Omega)} + a(u(t), v_{1}(t)) \diff t + \skprod{u(0)}{v_{2}}_{L_{2}(\Omega)}
%     \end{equation}
%     gegebene Bilinearform und $f(\blank) \colon \mathcal Y \to \mathbb{R}$ definiert ist durch
%     \begin{equation}
%         \label{eq:var_all_f_wiederholung}
%         f(v) = \int_{I} \skprod{g(t)}{v_{1}(t)}_{L_{2}(\Omega)} \diff t + \skprod{u_{0}}{v_{2}}_{L_{2}(\Omega)}.
%     \end{equation}

% Aus \cref{thm:schwab09:theorem51} und \cref{thm:schwab09:theorem51:ungleichungen} erhalten wir nun die Wohldefiniertheit des obigen Variationsproblems und zugleich Schranken für die Operatoren.

% \begin{Korollar}
% \label{korollar:2.2}
%     Seien $\mathcal X$ und $\mathcal Y$ gegeben wie in \cref{eq:var_ansatzraum_testraum} und sei $B \colon \mathcal X \to \mathcal Y'$ definiert durch
%     \begin{equation}
%         \skprod{Bu}{v}_{\mathcal Y' \times \mathcal Y}  = b(u, v), \quad u \in \mathcal X,~ v \in \mathcal Y,
%     \end{equation}
%     mit $b(\blank, \blank)$ wie in \cref{eq:buv}.
%     Dann ist $B$ stetig invertierbar und es gilt
%     \begin{equation}
%         \norm{B}_{\mathcal L(\mathcal X, \mathcal Y')}
%         \leq
%         \frac{\sqrt{2 \max\Set{1, c^{2}, \norm{\omega}_{L_{\infty}(\Omega)}^{2}} + M_{e}^{2}}}{\max\Set{\sqrt{1 + 2 \norm{\omega}_{L_{\infty}(\Omega)}^{2} \rho^{4}}, \sqrt{2} }}
%     \end{equation}
%     und
%     \begin{equation}
%         \norm{B^{-1}}_{\mathcal L( \mathcal Y', \mathcal X)}
%         \leq \frac{e^{2 T \norm{\omega}_{L_{\infty}(\Omega)}} \max\Set{\sqrt{1 + 2 \norm{\omega}_{L_{\infty}(\Omega)}^{2} \rho^{4}}, \sqrt{2}} \sqrt{2 \max\Set{c^{-2} \gamma_{\Omega}^{-4}, 1} + M_{e}^{2}}}{\min\Set{c^{-1} \gamma_{\Omega}^{2}, c \gamma_{\Omega}^{2} \norm{\omega}_{L_{\infty}(\Omega)}^{-2}, c \gamma_{\Omega}^{2} }}.
%         % \leq
%         % \frac{\max\{\sqrt{ 1 + 2 \norm{\omega}_{L_{\infty}(\Omega)} \rho^{4}}, \sqrt{2} \}}{e^{-2 \norm{\omega}_{L_{\infty}(\Omega)} T}}
%         % \frac{\sqrt{2 \max\{ 1, \sigma^{-2} \gamma_{\Omega}^{-4} \} + M_{e}^{2}}}{\min\{ \sigma \gamma_{\Omega}^{2} \norm{\omega}_{L_{\infty}(\Omega)}^{-2}, \sigma \gamma_{\Omega}^{2} \}}
%     \end{equation}
%     mit $M_{e}$ und $\rho$ wie in \cref{eq:var_all_M_e} respektive \cref{eq:var_all_rho}.
% \end{Korollar}

% % section vereinfachte_variante (end)

% \section{Parametrische Variante} % (fold)
% \label{sec:parametrische_variante}

% Wir wollen nun aus dem gerade beschriebenen Variationsproblem eine parametrische Variante gewinnen und aufbauend auf \cref{sec:parametrisches_problem} Regularität bezüglich des Parameters folgern.
% Dazu müssen wir den Operator $A \in \mathcal L(V, V')$ aus \cref{eq:def_op_A} zunächst zu einem parametrischen Operator $A(\sigma)$ mit $\sigma \in \mathcal S$, wobei $\mathcal S \subset \mathbb{R}^{\mathbb{N}}$ ein geeigneter Parameterraum ist, umschreiben.
% Dabei beschränken wir uns auf den Fall affiner parametrischer Abhängigkeit \cref{eq:all_affiner_operator}.
% Der Einfachheit halber wählen wir $\mathcal S = [-1, 1]^{\mathbb{N}}$, das heißt $\mathcal S$ sei die Einheitskugel aus $\ell_{\infty}(\mathbb{N})$.

% Sei $\Set{ \varphi_{j} }_{j \in \mathbb{N}} \subset L_{\infty}(\Omega)$ ein noch näher zu bestimmendes, passend gewähltes Funktionensystem und $\sigma \in \mathcal S$.
% Wir entwickeln nun $\omega$ formal in eine Reihe der Form
% \begin{equation}
%     \label{eq:reihenentwicklung_omega}
%     \omega(\blank; \sigma) = \sum_{j = 1}^{\infty} \sigma_{j} \varphi_{j}.
% \end{equation}
% Offenbar ist für die Konvergenz der Reihe \cref{eq:reihenentwicklung_omega} hinreichend, dass $\Set{ \norm{\varphi_{j}}_{L_{\infty}(\Omega)} }_{j \in \mathbb{N}} \in \ell_{1}(\mathbb{N})$ gilt, insbesondere folgt daraus
% \begin{equation}
%     \norm{\omega(\blank; \sigma)}_{L_{\infty}(\Omega)} \leq \sum_{j = 1}^{\infty} \norm{\varphi_{j}}_{L_{\infty}(\Omega)} < \infty \quad \fa \sigma \in \mathcal S.
% \end{equation}
% % Diese Eigenschaft wird auch benötigt, denn dadurch erhalten wir aus \cref{lemma:2.2} die für \cref{thm:kunoth:theorem21} notwendigen, von $\sigma$ unabhängigen, Schranken $\beta_{1}$ und $\beta_{2}$.
% Damit ist die Wahl des Funktionensystems $\Set{ \varphi_{j} }_{j \in \mathbb{N}} \subset L_{\infty}(\Omega)$ ist entscheidend für die Konvergenz von \cref{eq:reihenentwicklung_omega}, aber auch für die Erfüllbarkeit von \cref{thm:kunoth:assumption1} respektive \cref{thm:kunoth:assumption2},
% und wird in den nächsten Abschnitten genauer behandelt.

% % \subsection{Affiner Operator} % (fold)
% % \label{ssub:entwicklung_von_}

% Wir wollen den Operator $A$ aus \cref{eq:def_op_A} als affin parametrischen Operator der Form
% \begin{equation}
%     \label{eq:aff_zerlegung_A}
%     A(\sigma) = \hat A + \sum_{j \geq 1} \sigma_{j} A_{j}
% \end{equation}
% auffassen, beziehungsweise als Bilinearformen
% \begin{equation}
%      \label{eq:aff_zerelgung_A_bf}
%      a(\eta, \zeta; \sigma) = \hat a(\eta, \zeta) + \sum_{j \geq 1} \sigma_{j} a_{j}(\eta, \zeta), \quad \eta, \zeta \in V.
%  \end{equation}
% Dazu entwickeln wir $\omega$ in eine Reihe der Form \cref{eq:reihenentwicklung_omega}, das heißt wir erhalten
% \begin{equation}
%     \label{eq:omega_reihenentwicklung}
%     \omega(\blank; \sigma) \colon \Omega \to \mathbb{R}, \quad x \mapsto \omega(x; \sigma) = \sum_{j \geq 1} \sigma_{j} \varphi_{j}(x)
% \end{equation}
% mit $\sigma \in \mathcal S$.
% Eine naheliegende affine Aufteilung des Operators $A$ erhalten wir damit durch die Wahl
% \begin{equation}
%     \label{eq:affine_zerlegung_A_def}
%     \hat A = - c \Delta, \qquad
%     A_{j} = \varphi_{j}, \quad j \geq 1.
% \end{equation}
% Die zugehörigen Bilinearformen lassen sich ebenfalls direkt angeben, denn es gilt
% \begin{equation}
%     \hat a(\eta, \zeta) = \skprod{\grad \eta}{\grad \zeta}_{L_{2}(\Omega)}, \qquad a_{j}(\eta, \zeta) = \skprod{\varphi_{j} \eta}{\zeta}_{L_{2}(\Omega)}, \quad j \geq 1.
% \end{equation}

% Die daraus resultierende Raum-Zeit-Variationsformulierung lautet nun:
% \begin{Problem}
%     Gegeben ein $g \in L_{2}(I; H^{-1}(\Omega))$ und ein $u_{0} \in L_{2}(\Omega)$.
%     Finde für alle $\sigma \in \mathcal S$ ein $u(\sigma) \in \mathcal X$ mit
%     \begin{equation}
%         \label{eq:varprob_2}
%         b(u, v; \sigma) = f(v) \quad \text{für alle}~v \in \mathcal Y,
%     \end{equation}
%     wobei $b(\blank, \blank; \sigma) \colon \mathcal X \times \mathcal Y \times \mathcal S \to \mathbb{R}$ die durch
%     \begin{equation}
%         \label{eq:buv_2}
%         b(u, v; \sigma)
%             = \int_{I} \skprod{u_{t}(t)}{v_{1}(t)}_{L_{2}(\Omega)} + a(u(t), v_{1}(t); \sigma) \diff t + \skprod{u(0)}{v_{2}}_{L_{2}(\Omega)}
%     \end{equation}
%     gegebene Bilinearform und $f(\blank) \colon \mathcal Y \to \mathbb{R}$ definiert ist durch
%     \begin{equation}
%         \label{eq:var_all_f_wiederholung_2}
%         f(v) = \int_{I} \skprod{g(t)}{v_{1}(t)}_{L_{2}(\Omega)} \diff t + \skprod{u_{0}}{v_{2}}_{L_{2}(\Omega)}.
%     \end{equation}
% \end{Problem}


% % \section{Periodische Randbedingungen} % (fold)
% % \label{sec:periodische_randbedingungen}


% \todo[inline]{besser platzieren}
% \begin{Satz}
% \label{satz:pp:lax_auf_elliptisch}
%     Seien $\omega \in L_{\infty}(\Omega)$, $\mu \geq \norm{\omega}_{L_{\infty}(\Omega)}$ und weiter $g \in H^{-1}(\Omega)$ und $A(\omega)$ wie in \cref{eq:pp:op_a}, dann besitzt die Operatorgleichung
%     \begin{equation}
%         A(\omega) u(\omega) = g
%     \end{equation}
%     eine eindeutige Lösung $u(\omega) \in H^{1}_{0}(\Omega)$ und diese erfüllt
%     \begin{equation}
%         \norm{u(\omega)}_{H^{1}(\Omega)} \leq \frac{\norm{g}_{H^{-1}(\Omega)}}{\alpha}
%     \end{equation}
%     mit $\alpha$ aus \cref{satz:pp:a_bf_bounded_garding}.

%     \begin{Beweis}
%         Folgt aus dem Banach-Ne\v{c}as-Babu\v{s}ka-Theorem, \cref{satz:gl:bnb_theorem}.
%     \end{Beweis}
% \end{Satz}


% % section periodische_randbedingungen (end)
