%!TEX root = ../main.tex

\setchapterpreamble[ul][0.6\textwidth]{%
    \dictum[Alfréd Rényi]{\enquote{A mathematician is a device for turning coffee into theorems.}}
    \vspace*{2\baselineskip}
}
\chapter{Reduzierte-Basis-Methode} % (fold)
\label{cha:reduzierte_basis_methode}

\todo[inline]{Erklärung schreiben}

\blindtext


\begin{figure}[tb]
    \centering
    \includestandalone[width=0.6\textwidth]{tikz/rb}
    \caption[%
    Skizzenhafte Darstellung der Funktionsweise der Reduzierte-Basis-Methode.
    ]{
        Skizzenhafte Darstellung der Funktionsweise der Reduzierte-Basis-Methode.
        Die Reduzierte-Basis-Approximation $u_{N}(\mu)$ ergibt sich als Linearkombination der Finite-Elemente-Snapshots $u^{\mathcal N}(\mu_{i})$, welche mutmaßlich eine glatte parametrische Manigfaltigkeit bilden.
        }
    \label{fig:figure1}
\end{figure}


% chapter reduzierte_basis_methode (end)
