%!TEX root = ../main.tex

\chapter{Funktionalanalytische Grundlagen} % (fold)
\label{chapter:grundlagen}

Um die in der Einleitung beschriebene parabolische partielle Differentialgleichungen theoretisch und numerisch untersuchen zu können, müssen wir zunächst ein robustes Grundgerüst schaffen.
Dies beginnen wir in diesem Kapitel mit der Einführung respektive der Wiederholung der benötigten Grundlagen aus der Funktionalanalysis.
Dabei orientieren wir uns maßgeblich an den Arbeiten von \textcite{Dautray:1992by,Schweizer2013}, in welchen die nachfolgenden Ausführungen weit detaillierter zu finden sind.


\section{Bochner"=Räume} % (fold)
\label{section:bochner_raeume}

Bevor wir uns an die Einführung einer Raum"=Zeit"=Variationsformulierung parabolischer partieller Differentialgleichungen begeben, müssen wir zunächst die zugrundeliegenden Funktionenräume einführen.
Hierbei konzentrieren wir uns auf die sogenannten \emph{Bochner"=Räume}, welche eine Verallgemeinerungen der bekannten Lebesgue"=Räume $L_{p}$ auf Banachraum"=wertige Funktionen darstellen, schränken uns dabei aber stets auf den für uns relevanten Fall eines endlichen Zeitintervalls ein.

Wir beginnen nun mit der folgenden Definition der Bochner"=Räume nach \cite[Definition XVIII.1.1]{Dautray:1992by}.

\begin{Definition}
\label{definition:bochner_raum}
    Seien $X$ ein Banachraum, $- \infty < a < b < \infty$ und $1 \leq p < \infty$.
    Als \emph{Bochner"=Raum} $L_{p}(a, b; X)$ bezeichnen wir die Menge (der Äquivalenzklassen) $L_{p}$-integrierbarer Funktionen $f \colon [a, b] \to X$, das heißt, aller messbarer Funktionen auf $[a, b]$ mit
    \begin{equation}
        \norm{f}_{L_{p}(a, b; X)} \deq \left( \int_{a}^{b} \norm{f(t)}_{X}^{p} \diff t \right)^{1 / p} < \infty.
    \end{equation}
    Ferner ist der \emph{Bochner"=Raum} $L_{\infty}(a, b; X)$ definiert als die Menge (der Äquivalenzklassen) der für fast alle $t \in [a, b]$ wesentlich beschränkten Funktionen, also aller messbaren Funktionen $f \colon [a, b] \to X$ mit
    \begin{equation}
        \norm{f}_{L_{\infty}(a, b; X)} \deq \esssup_{t \in [a, b]} \norm{f(t)}_{X} < \infty.
    \end{equation}
\end{Definition}

\begin{Lemma}
\label{lemma:bochnerraum_ist_banachraum_bzw_hilbertraum}
    Für alle $1 \leq p \leq \infty$ ist $L_{p}(a, b; X)$ ein Banachraum.
    Ist ferner $H$ ein Hilbertraum, so auch $L_{2}(a, b; H)$ mit dem Skalarprodukt
    \begin{equation}
        \skp{u}{v}{L_{2}(a, b; H)} = \int_{a}^{b} \skp{u(t)}{v(t)}{H} \diff t, \quad \text{für}~u,v \in L_{2}(a, b; H).
    \end{equation}

    \begin{Beweis}
        Die erste Aussage findet sich in \cite[Proposition XVIII.1.1]{Dautray:1992by}, die zweite in \cite[Abschnitt 1.1.3]{Lions:1972tg}.
    \end{Beweis}
\end{Lemma}

Weiter können wir für Funktionen aus einem Bochner"=Raum eine Zeitableitung definieren, hier nach \cites[471]{Dautray:1992by}[Definition 10.6]{Schweizer2013}.

\begin{Definition}%[Zeitableitung]
\label{definition:zeitableitung}
    Seien $X$ und $Y$ Banachräume mit stetiger Einbettung $X \hookrightarrow Y$ und sei $u \in L_{2}(a, b; X)$.
    Als \emph{Zeitableitung} bezeichnen wir die distributionelle Ableitung $\frac{\mathrm{d}}{\mathrm{d} t} u \in L_{2}(a, b; Y)$, welche als das $v \in L_{2}(a, b; Y)$ definiert ist, welches
    \begin{equation}
        \int_{a}^{b} v(t) \varphi(t) \diff t = - \int_{a}^{b} u(t) \frac{\mathrm{d}}{\mathrm{d} t} \varphi(t) \diff t \quad \fa \varphi \in \mathcal C^{\infty}_{0}((a, b), \mathbb{R})
    \end{equation}
    erfüllt, falls ein solches $v$ existiert.
\end{Definition}

\begin{Bemerkung}
    Je nach Situation werden wir der Einfachheit halber eine der Schreibweisen $\frac{\mathrm{d}}{\mathrm{d} t} u = u' = u_{t}$ verwenden.
\end{Bemerkung}

Obige Definition einer Zeitableitung ermöglicht die Definition der Bochner"=Sobolev"=Räume.
Wir beschränken uns hier auf den für uns relevanten Raum erster Ordnung, siehe \cite[Section 5.9.2]{evans2010partial}.

\begin{Definition}
\label{definition:bochner_sobolev_raum}
    Sei $X$ ein Banachraum.
    Als \emph{Bochner"=Sobolev"=Raum} erster Ordnung definieren wir den Raum
    \begin{equation}
        H^{1}(a, b; X) = \Set{u \in L_{2}(a, b; X) \given u' \in L_{2}(a, b; X)}.
    \end{equation}
\end{Definition}

Ferner gilt eine zu \cref{lemma:bochnerraum_ist_banachraum_bzw_hilbertraum} analoge Aussage auch für Bochner"=Sobolev"=Räume.
Da wir im Zuge dieser Arbeit ausschließlich mit Hilberträumen arbeiten werden, wird sich das folgende als Gelfand"=Tripel bekannte Konstrukt als grundlegend erweisen.
Die folgende Definition ist eine leichte Abwandlung von \cite[Abschnitt 10.2]{Schweizer2013}.

\begin{Definition}
\label{definition:gelfand_tripel}
    Seien $V$ und $H$ separable Hilberträume mit den Dualräumen $V'$ und $H'$.
    Weiter sei die Einbettung $V \denseinclusion H$ dicht und stetig.
    Durch die Identifikation $H \cong H'$ erhalten wir das sogenannte \emph{Gelfand"=Tripel} $V \denseinclusion H \denseinclusion V'$, wobei die Inklusionen jeweils dichte stetige Einbettungen sind.
\end{Definition}

\begin{Bemerkung}
\label{bemerkung:skalarprodukte_und_duality_pairing}
    Bezeichne mit $\skp{\blank}{\blank}{V}$ und $\skp{\blank}{\blank}{H}$ die Skalarprodukte auf $V$ respektive $H$.
    Weiter verwenden wir $\skp{\blank}{\blank}{V' \times V}$ auch für die duale Paarung auf $V' \times V$, welche die eindeutige stetige Fortsetzung von $\skp{\blank}{\blank}{H}$ ist.
    Dadurch gilt insbesondere
    \begin{equation}
        \skp{u}{v}{V' \times V} = \skp{u}{v}{H}, \quad \fa u \in H, v \in V.
    \end{equation}
\end{Bemerkung}

Mit Hilfe dieser Gelfand"=Tripel können wir nun die Räume definieren, welche wir letztendlich für die schwache Formulierung parabolischer partieller Differentialgleichungen benutzen werden.
Dies geschieht analog zu \cite[Definition XVIII.2.4]{Dautray:1992by}.

\begin{Definition}
\label{definition:bochner_raum_W}
    Sei $V \denseinclusion H \denseinclusion V'$ ein Gelfand"=Tripel.
    Wir definieren den Raum
    \begin{equation}
        W(a, b; V, V') \deq \Set{u \in L_{2}(a, b; V) \given u' \in L_{2}(a, b; V')}
    \end{equation}
    wobei $u'$ im Sinne von \cref{definition:zeitableitung} zu verstehen ist.
    Es gilt ferner
    \begin{equation}
        W(a, b; V, V') = L_{2}(a, b; V) \cap H^{1}(a, b; V').
    \end{equation}
\end{Definition}

Weiter können wir diesem Raum ein Skalarprodukt und damit auch eine Norm zuordnen, dies liefert die folgende Aussage.

\begin{Lemma}
\label{lemma:bochner_W_ist_hilbertraum}
    Versehen wir $W(a, b; V, V')$ mit dem Skalarprodukt
    \begin{equation}
        \skp{u}{v}{W(a, b; V, V')} \deq \skp{u}{v}{L_{2}(a, b; V)} + \skp{u'}{v'}{L_{2}(a, b; V')}
    \end{equation}
    und der dadurch induzierten Norm
    \begin{equation}
        \norm{u}_{W(a, b; V, V')} = \left( \norm{u}_{L_{2}(a, b; V)}^{2} + \norm{u'}_{L_{2}(a, b; V')}^{2} \right)^{1/2},
    \end{equation}
    so ist $W(a, b; V, V')$ ein Hilbertraum.

    \begin{Beweis}
        Siehe \cite[Proposition XVIII.2.6]{Dautray:1992by}.
    \end{Beweis}
\end{Lemma}

Da die von uns betrachteten parabolischen partiellen Differentialgleichungen jeweils mit Anfangsbedingungen versehen sein werden, müssen wir an dieser Stelle klären, in welchem Sinne diese zu verstehen sind.
Dies führt zum sogenannten Spursatz, welchen wir durch die folgende Einbettungsaussage erhalten.

\begin{Satz}
\label{satz:bochner_raum_W_stetige_einbettung_C0}
    Seien $V \denseinclusion H \denseinclusion V'$ ein Gelfand"=Tripel und $a, b \in \mathbb{R}$.
    Ferner sei $\mathcal C^{0}([a, b]; H)$ die Menge aller stetigen Funktionen $f \colon [a, b] \to H$.
    Dann ist die Einbettung
    \begin{equation}
        W(a, b; V, V') \hookrightarrow \mathcal C^{0}([a, b], H).
    \end{equation}
    stetig.
    Insbesondere stimmt jedes $u \in W(a, b; V, V')$ fast überall mit einer stetigen Funktion aus $\mathcal C^{0}([a, b], H)$ überein.

    \begin{Beweis}
        Siehe \cites[Theorem XVIII.2.1]{Dautray:1992by}[Theorem 10.9]{Schweizer2013}.
    \end{Beweis}
\end{Satz}

\begin{Korollar}[Spursatz]
\label{korollar:spursatz}
    Seien $a, b \in \mathbb{R}$ und $u \in W(a, b; V, V')$.
    Dann sind die \emph{Spuren} $u(a), u(b) \in H$ wohldefiniert.

    \begin{Beweis}
        Direkte Folgerung aus \cref{satz:bochner_raum_W_stetige_einbettung_C0}, siehe auch \cite[Remark XVIII.2.4]{Dautray:1992by}.
    \end{Beweis}
\end{Korollar}

Ferner erhalten wir aus obiger Einbettungsaussage \cref{satz:bochner_raum_W_stetige_einbettung_C0} auch das folgende Ergebnis, welches im Rahmen der Betrachtung linearer Evolutionsgleichungen benötigt wird.

\begin{Korollar}
\label{korollar:einbettungskonstante_M_e}
    Seien $a, b \in \mathbb{R}$.
    Die Einbettungskonstante
    \begin{equation}
        \label{eq:einbettungskonstante_M_e}
        \gamma_{e} \deq \sup_{\substack{u\in W(a, b; V, V')\\u \neq 0}} \frac{\norm{u(a)}_{H}}{\norm{u}_{W(a, b; V, V')}}
    \end{equation}
    ist gleichmäßig beschränkt in der Wahl $V \hookrightarrow H$ und hängt nur im Fall $b \to a$ von $b$ ab.

    \begin{Beweis}
        Siehe \cites[Section 5]{Schwab:2009ec}[Beweis zu Theorem XVIII.2.1]{Dautray:1992by}.
    \end{Beweis}
\end{Korollar}

Abschließen wollen wir diesen Abschnitt mit einer alternativen Charakterisierung der hier betrachteten Bochner"=Räume als Tensor"=Produkt, welche erst bei der numerischen Umsetzung in \cref{chapter:galerkin} relevant wird.

\begin{Satz}
\label{satz:bochner_sobolev_raum_als_tensorprodukt}
    Seien $V$ ein Hilbertraum und $a, b \in \mathbb{R}$ mit $a < b$.
    Dann ist der Bochner"=Sobolev"=Raum $H^{m}(a, b; V)$, $m \geq 0$, isometrisch isomorph zum Hilbertraum"=Tensorprodukt $H^{m}([a, b]) \otimes V$, kurz
    \begin{equation}
        H^{m}(a, b; V) \cong H^{m}([a, b]) \otimes V.
    \end{equation}

    \begin{Beweis}
        Siehe \cite[Theorem 12.6.1, Theorem 12.7.1]{Aubin:2000un}.
    \end{Beweis}
\end{Satz}


\section{Lineare Evolutionsgleichungen} % (fold)
\label{section:lineare_evolutionsgleichungen}

Nach der Einführung der benötigten Funktionenräume wenden wir uns nun den linearen Evolutionsgleichungen, einer bestimmten Unterart parabolischer partieller Differentialgleichungen, zu.
Wir orientieren uns hierbei an \textcite{Lions:1971wp,Schwab:2009ec,Urban:2014kg},
definieren den Begriff der linearen Evolutionsgleichungen, leiten eine schwache Formulierung her und weißen abschließend nach, dass diese sachgemäß gestellt ist.

Um den Begriff der linearen Evolutionsgleichungen definieren zu können, müssen wir zunächst die richtigen Rahmenbedingungen schaffen.
Sei $V \denseinclusion H \denseinclusion V'$ ein Gelfand-Tripel im Sinne von \cref{definition:gelfand_tripel}, das heißt, $V$ und $H$ seien separable Hilberträume und die Inklusionen seien jeweils dicht und stetig.
Weiter sei $0 < T < \infty$ und $I = [0, T]$.
Wir betrachten nun eine Familie von stetigen linearen Operatoren $A(t) \in \mathcal L(V, V')$ für $t \in I$.
Nach dem Rieszschen Darstellungssatz, genauer siehe \cite[Theorem \S{}22.1]{Halmos:1957vd}, können wir diesen Operatoren eine Familie von Bilinearformen $a(\blank, \blank; t) \colon V \times V \to \mathbb{R}$ zuordnen, wobei diese durch
\begin{equation}
    \skp{A(t)\eta}{\zeta}{V' \times V} = a(\eta, \zeta; t) \quad \fa \eta, \zeta \in V
\end{equation}
induziert werden.

Wie wir später sehen werden, stellt die nachfolgende Annahme sicher, dass die auf diese Operatoren aufbauenden linearen Evolutionsgleichungen gewisse wünschenswerte Eigenschaften, wie Lösbarkeit und Eindeutigkeit dieser Lösung, besitzen.

\begin{Annahme}
\label{annahme:eigenschaften_der_bilinearform_a}
    Die Familie von Bilinearformen $a(\blank, \blank; t)$, $t \in I$, erfülle die folgenden Eigenschaften:
    \leavevmode
    \begin{thmenumerate}
        \item \emph{Messbarkeit:} die Abbildung $I \ni t \mapsto a(\eta, \zeta; t)$ ist messbar für alle $\eta, \zeta \in V$.
        \item \emph{Stetigkeit:}
        es existiert ein $0 < \gamma_{a} < \infty$ mit
        \begin{equation}
            \label{eq:eigenschaften_der_bilinearform_a:stetigkeit}
            \abs{a(\eta, \zeta; t)} \leq \gamma_{a} \norm{\eta}_{V} \norm{\zeta}_{V}, \quad \fa \eta, \zeta \in V \text{ und fast alle } t \in I.
        \end{equation}
        \item \emph{G\r{a}rding-Ungleichung:}
        es existieren $\alpha > 0$ und $\lambda \geq 0$ mit
        \begin{equation}
            \label{eq:eigenschaften_der_bilinearform_a:garding}
            a(\eta, \eta; t) + \lambda \norm{\eta}_{H}^{2} \geq \alpha \norm{\eta}_{V}^{2}, \quad \fa \eta \in V \text{ und fast alle } t \in I.
        \end{equation}
    \end{thmenumerate}
\end{Annahme}

Für den Rest dieses Abschnitts nehmen wir stets an, dass die Bilinearform $a(\blank, \blank; t)$ obige Annahme erfüllt.
Mit dieser Vorarbeit definieren wir nun den Begriff der linearen Evolutionsgleichung.

\begin{Definition}
\label{definition:lineare_evolutionsgleichung}
    Sei $g \in L_{2}(I; V')$ ein \emph{Quellterm} und $u_{0} \in H$ ein \emph{Anfangswert}.
    Als \emph{lineare Evolutionsgleichung} bezeichnen wir die parabolische partielle Differentialgleichung
    \begin{equation}
        \label{eq:lineare_evolutionsgleichung}
        \left\{
        \begin{aligned}
            u_{t}(t) + A(t) u(t) &= g(t)     \quad&&\text{in}~V', \quad \text{für fast alle}~t \in I, \\
            u(0) &= u_{0}                    \quad&&\text{in}~H,
        \end{aligned}
        \right.
    \end{equation}
    wobei die  Operatorfamilie $A(t) \in \mathcal L(V, V')$ wie oben gegeben sei.
\end{Definition}

Darauf aufbauend leiten wir als nächstes eine Raum-Zeit-Variationsformulierung für~\cref{eq:lineare_evolutionsgleichung} her.
Dazu benötigen wir geeignete Ansatz- und Testfunktionenräume; hierzu verwenden wir die im vorherigen Abschnitt eingeführten Bochner-Räume.

\begin{Definition}
\label{definition:ansatz_und_testraum}
    Den Ansatzfunktionenraum $\mathcal X$ definieren wir als den Hilbertraum
    \begin{equation}
    \label{eq:ansatzraum_X}
        \mathcal X = W(I; V, V') = L_{2}(I; V) \cap H^{1}(I; V')
    \end{equation}
    mit dem Skalarprodukt
    \begin{equation}
    \label{eq:ansatzraum_skalarprodukt}
        \skp{u}{v}{\mathcal X} = \skp{u}{v}{L_{2}(I; V)} + \skp{u'}{v'}{L_{2}(I; V')}.
    \end{equation}
    Der Testfunktionenraum $\mathcal Y$ sei der Hilbertraum
    \begin{equation}
    \label{eq:testraum_Y}
        \mathcal Y = L_{2}(0, T; V) \times H
    \end{equation}
    mit Skalarprodukt
    \begin{equation}
        \label{eq:testraum_skalarprodukt}
        \skp{u}{v}{\mathcal Y} = \skp{u_{1}}{v_{1}}{L_{2}(I; V)} + \skp{u_{2}}{v_{2}}{H} \quad \text{für } u = (u_{1}, u_{2}), v = (v_{1}, v_{2}) \in \mathcal Y.
    \end{equation}
\end{Definition}

Um nun aus~\cref{eq:lineare_evolutionsgleichung} eine schwache Formulierung zu erhalten, \enquote{multiplizieren} wir die lineare Evolutionsgleichung mit $v = (v_{1}, v_{2}) \in \mathcal Y$ und integrieren anschließend über das Zeitintervall $I = [0, T]$.
Dadurch erhalten wir die folgende Raum"=Zeit"=Variationsformulierung:

\begin{Definition}
\label{definition:absktrakte_raum_zeit_formulierung}
    Seien $\mathcal X$ und $\mathcal Y$ wie in~\cref{eq:ansatzraum_X,eq:testraum_Y}, $g \in L_{2}(I; V')$ ein Quellterm und $u_{0} \in H$ ein Anfangswert.
    Als \emph{Raum"=Zeit"=Variationsformulierung} der linearen Evolutionsgleichung~\cref{eq:lineare_evolutionsgleichung} bezeichnen wir das folgende Problem:
    \begin{equation}
        \label{eq:absktrakte_raum_zeit_formulierung}
        \text{Finde } u \in \mathcal X \colon \quad b(u, v) = f(v) \quad \fa v \in \mathcal Y.
    \end{equation}
    Dabei sei die Bilinearform $b \colon \mathcal X \times \mathcal Y \to \mathbb{R}$ durch
    \begin{equation}
        \label{eq:absktrakte_raum_zeit_formulierung:lhs}
        b(u, v) = \int_{I} \left[   \skp{u_{t}(t)}{v_{1}(t)}{V' \times V} + a(u(t), v_{1}(t); t)  \right] \diff t + \skp{u(0)}{v_{2}}{H}
    \end{equation}
    definiert und das Funktional $f \colon \mathcal Y \to \mathbb{R}$ durch
    \begin{equation}
        \label{eq:absktrakte_raum_zeit_formulierung:rhs}
        f(v) = \int_{I} \skp{g(t)}{v_{1}(t)}{V' \times V} \diff t + \skp{u_{0}}{v_{2}}{H}.
    \end{equation}
\end{Definition}

\begin{Bemerkung}
    Die Anfangswertbedingung $u(0) = u_{0}$ in $H$ ist wegen \cref{korollar:spursatz} wohldefiniert.
\end{Bemerkung}

Es bleibt nun zu zeigen, welche Bedingungen hinreichend sind, damit obige Raum"=Zeit"=Variationsformulierung \emph{sachgemäß gestellt} ist.
Dazu definieren wir zunächst, was wir darunter verstehen wollen und greifen dazu auf die Definition nach \textcite{hadamard1902problemes} zurück.

\begin{Definition}[Hadamard]
\label{definition:sachgemaess_gestellt_nach_hadamard}
    Seien $X$ und $Y$ zwei Hilberträume, $a \in \mathcal L(X \times Y)$ eine stetige Bilinearform und $f \in Y'$ ein stetiges lineares Funktional.
    Wir nennen das abstrakte Variationsproblem
    \begin{equation}
    \label{eq:hadamard_variationsproblem}
        \text{Finde } u \in X \colon \quad a(u, v) = f(v) \quad \fa v \in Y.
    \end{equation}
    \emph{sachgemäß gestellt}, wenn eine eindeutige Lösung $u \in X$ existiert und diese eine A-priori-Abschätzung der Form
    \begin{equation}
    \label{eq:hadamard_abschaetzung}
        \norm{u}_{X} \leq c \norm{f}_{Y'}, \quad \fa f \in Y',
    \end{equation}
    mit einer von $f$ unabhängigen Konstante $c > 0$ erfüllt.
\end{Definition}

Um dies für die Raum"=Zeit"=Variationsformulierung \cref{eq:absktrakte_raum_zeit_formulierung} nachzuweisen, werden wir indirekt auf den nachfolgenden wichtigen Satz zurückgreifen.
Dieser findet sich in dieser oder ähnlicher Form bei \textcites[Theorem 2.1]{Babuska:1971fx}[Theorem 5.2.1]{Aziz:2014wf}[Theorem \S{}3.3.6]{Braess:2007wm} sowie vielen weiteren einschlägigen Quellen.

\begin{Satz}[Banach-Ne{\v c}as-Babu{\v s}ka-Theorem]
\label{satz:bnb_theorem}
    Seien $X$ und $Y$ zwei Hilberträume.
    Eine lineare Abbildung $A \colon X \to Y'$ ist genau dann ein Isomorphismus, das heißt stetig invertierbar, wenn die zugehörige Bilinearform $a \colon X \times Y \to \mathbb{R}$ die folgenden Bedingungen erfüllt:
    \begin{thmenumerate}
        \item \label{satz:bnb_theorem:stetig}
        \emph{Stetigkeit:}
        es existiert eine Konstante $0 < \gamma < \infty$ mit
        \begin{equation}
            \abs{a(u, v)} \leq \gamma \norm{u}_{X} \norm{v}_{Y} \quad \fa u \in X,~v\in Y.
        \end{equation}
        \item \label{satz:bnb_theorem:inf_sup_bedingung}
        \emph{Inf-sup-Bedingung:}
        es existiert eine Konstante $\beta > 0$ mit
        \begin{equation}
            \inf_{u \in X} \sup_{v \in Y} \frac{a(u, v)}{\norm{u}_{X} \norm{v}_{Y}} \geq \beta.
        \end{equation}
        \item \label{satz:bnb_theorem:surjektiv}
        \emph{Surjektivität:}
        zu jedem $v \in Y$, $v \neq 0$, existiert ein $u \in X$ mit
        \begin{equation}
            a(u, v) \neq 0.
        \end{equation}
    \end{thmenumerate}
    Gelten die drei Bedingungen und ist weiter ein Funktional $f \in Y'$ gegeben, dann existiert eine eindeutige Lösung $\hat u \in X$ mit
    \begin{equation}
        a(\hat u, v) = f(v) \quad \fa v \in Y
    \end{equation}
    und es gilt
    \begin{equation}
        \norm{\hat u}_{X} \leq \frac{1}{\beta} \norm{f}_{Y'}.
    \end{equation}
\end{Satz}

\begin{Bemerkung}
\label{bemerkung:bnb_theorem_inf_sup_statt_surjektiv}
    Die Surjektivitätsbedingung \cref{satz:bnb_theorem:surjektiv} kann durch eine weitere inf"=sup"=Bedingung ersetzt werden, denn existiert ein $\beta' > 0$ mit
    \begin{equation}
        \inf_{v \in Y} \sup_{u \in X} \frac{a(u, v)}{\norm{u}_{X} \norm{v}_{Y}} \geq \beta',
    \end{equation}
    dann gilt insbesondere auch \cref{satz:bnb_theorem:surjektiv}.
\end{Bemerkung}

Für das abstrakte Raum"=Zeit"=Variationsproblem \cref{eq:absktrakte_raum_zeit_formulierung} wurde für die hier vorliegenden Rahmenbedingungen bereits von \textcite[Section XVIII.3]{Dautray:1992by} nachgewiesen, dass es sich hierbei um ein sachgemäß gestelltes Problem handelt.
Ein alternativer Beweis, welcher die Bedingungen des \acl{bnb}s nachweist und weiter explizite Schranken für die Stetigkeitskonstante $\gamma_{b}$ und die Inf"=sup"=Konstante $\beta$ der Bilinearform $b$ aus \cref{eq:absktrakte_raum_zeit_formulierung:lhs} liefert, wurde von \textcite{Schwab:2009ec} geführt.
Wir wiederholen die Kernaussage \cite[Theorem 5.1]{Schwab:2009ec} und verweisen für einen ausführlichen Beweis auf \cite[Appendix A]{Schwab:2009ec}.

\begin{Satz}
\label{satz:ss09:theorem51}
    Seien $\mathcal X$ und $\mathcal Y$ durch \cref{eq:ansatzraum_X,eq:testraum_Y} gegeben und sei $a(\blank, \blank; t) \colon V \times V \to \mathbb{R}$ eine Familie von Bilinearformen, welche \cref{annahme:eigenschaften_der_bilinearform_a} erfüllt.
    Dann ist das Raum"=Zeit"=Variationsproblem \cref{eq:absktrakte_raum_zeit_formulierung} sachgemäß gestellt, das heißt, für alle $f \in \mathcal Y'$ existiert eine eindeutige Lösung $u \in \mathcal X$, so dass
    \begin{equation}
        b(u, v) = f(v) \quad \fa v \in \mathcal Y
    \end{equation}
    gilt und ferner existiert eine von $f$ unabhängige Konstante $\beta > 0$ mit
    \begin{equation}
        \norm{u}_{\mathcal X} \leq \frac{1}{\beta} \norm{f}_{\mathcal Y'}.
    \end{equation}
\end{Satz}

Abschließend wollen wir im nachfolgenden Korollar die bereits angesprochenen Schranken für die Stetigkeitskonstante $\gamma_{b}$ und die inf"=sup"=Konstante $\beta$ der Bilinearform $b$ angeben.

\begin{Korollar}
\label{korrolar:ss09:theorem51_abschaetzungen}
    Unter den gleichen Voraussetzungen wie in \cref{satz:ss09:theorem51} und der Bedingung, dass die Bilinearformen $a(\blank, \blank; t)$ die G\aa{}rding-Ungleichung mit $\lambda = 0$ erfüllen, gilt
    \begin{equation}
        \label{eq:ss09:theorem51_abschaetzungen:lambda_null:stetig}
        \gamma_{b}  \leq \sqrt{2 \max\Set{1, \gamma_{a}^{2}} + \gamma_{e}^{2}}
    \end{equation}
    und
    \begin{equation}
        \label{eq:ss09:theorem51_abschaetzungen:lambda_null:inf_sup}
        \beta  \geq \frac{\min\Set{\alpha \gamma_{a}^{-2}, \alpha}}{\sqrt{2 \max \Set{\alpha^{-2}, 1} + \gamma_{e}^{2}}}.
    \end{equation}
    Im Fall $\lambda \neq 0$ werden die Abschätzungen zu
    \begin{equation}
        \label{eq:ss09:theorem51_abschaetzungen:lambda_nicht_null:stetig}
        \gamma_{b}  \leq \frac{\sqrt{2\max\Set{1, \gamma_{a}^{2}} + \gamma_{e}^{2}}}{\max\Set{\sqrt{1 + 2 \lambda^{2} \rho^{4}}, \sqrt{2}}}
    \end{equation}
    und
    \begin{equation}
        \label{eq:ss09:theorem51_abschaetzungen:lambda_nicht_null:inf_sup}
        \beta  \geq \frac{ e^{-2 \lambda T}}{\max\Set{\sqrt{1 + 2 \lambda^{2} \rho^{4}}, \sqrt{2}}  } \cdot \frac{\min\Set{\alpha \gamma_{a}^{-2}, \alpha}}{\sqrt{2 \max\Set{ \alpha^{-2}, 1} + \gamma_{e}^{2}}}.
    \end{equation}
    Die Größen $\gamma_{a}$, $\alpha$ und $\lambda$ stammen aus \cref{annahme:eigenschaften_der_bilinearform_a},
    während die Konstanten $\rho$ und $\gamma_{e}$, für letztere siehe auch \cref{korollar:einbettungskonstante_M_e}, als die Einbettungskonstanten
    \begin{equation}
        \label{eq:ss09:einbettungskonstanten}
        \gamma_{e} = \sup_{0 \neq u \in \mathcal X} \frac{\norm{u(0)}_{H}}{\norm{u}_{\mathcal X}}, \qquad
        \rho = \sup_{0 \neq \eta \in V} \frac{\norm{\eta}_{H}}{\norm{\eta}_{V}}
    \end{equation}
    definiert sind.

    \begin{Beweis}
        Siehe \cite[Appendix A]{Schwab:2009ec}.
    \end{Beweis}
\end{Korollar}
