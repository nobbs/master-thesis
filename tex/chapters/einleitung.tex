%!TEX root = ../main.tex

\chapter*{Einleitung} % (fold)
\addcontentsline{toc}{chapter}{Einleitung}
\label{cha:einleitung}

Lineare Evolutiongsgleichungen treten in vielen Bereichen der Mathematik und der Naturwissenschaften auf.
Zu den prominentesten Vertreten gehören sicherlich die Wärmeleitungsgleichung, die Wellengleichung und viele lineare Reaktionsdiffusionsgleichungen.
Auch in der Polymerchemie lassen sie sich antreffen, so zum Beispiel bei der Modellierung von Copolymer-Melts [zitieren, Helfand oder Stasiak], wo auch die Motivation für diese Arbeit herrührt.

Konkret stammt das vorliegende Problem aus dem Bereich der \emph{Self Consistent Field Theory}.
Diese wird zum Beispiel verwendet um in obengenannten Copolymer-Melts Gleichgewichte, oder auch Konfigurationen, durch Minimierung eines Funktionals zu finden.
Dazu werden Iterationsverfahren verwendet, die darauf zurückgreifen in jedem Schritt eine Reaktionsdiffusionsgleichung zu lösen, wobei der Reaktionsterm durch das Iterationsverfahren in jedem Schritt variiert wird.
Dies motiviert dazu, die Reaktionsdiffusionsgleichung als parametrische partielle Differentialgleichung aufzufassen und bereits bekannte Methoden zur Dimensionsreduktion zu verwenden.

Eines dieser Verfahren ist die Reduzierte-Basis-Methode.
Ziel dieser ist es, die Eigenschaft, dass die Abhängigkeit der Lösung der parametrischen partiellen Differentialgleichung eine gewisse Regularität bezüglich des Parameters aufweist, auszunutzen.
Die Reduzierte-Basis-Methode beruht, wie zum Beispiel auch die Finite-Elemente-Methode, auf der Galerkin-Projektion auf einen endlichdimensionalen Unterraum.
Anders als bei der Finite-Elemente-Methode werden die Unterräume aber von Lösungen der parametrischen partiellen Differentialgleichung zu gewissen, clever gewählten, Parametern aufgespannt.
Zur Wahl dieser Parameter gibt es verschiedene Ansätze, zum Beispiel \emph{Greedy-Verfahren}, oder auch die \emph{Proper Orthogonal Decomposition}.
Für die meisten dieser Ansätze werden \emph{a priori}-Fehlerschätzer benötigt.


% chapter einleitung (end)
