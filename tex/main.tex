%!TEX root = main.tex

\documentclass[
  % draft,
  % final,
  a4paper,
  % 11pt,
  twoside=semi,
  % oneside,
  % parskip,
  toc=bibliography,
  toc=listof,
  chapterprefix=true,
]{scrbook}

%!TEX root = main.tex

%%%%%%%%%%%%%%%%%%%%%%%%%%%%%%%%%%%%%%%%%%%%%%%%%%%%%%%%%%%%%%%%%%%%%%%%%%%%%%%
%%% Allgemeines

% Mehr Speicher für latex
\usepackage{etex}

% Encoding und Schriftsystem
\usepackage[utf8]{inputenc}
\usepackage[T1]{fontenc}

% Deutsche Silbentrennung
\usepackage[ngerman]{babel}

% Bessere Standard-Schriftart
\usepackage{lmodern}

% Typographische Kleinigkeiten
\usepackage{microtype}

% Graphiken und Farben
\usepackage{graphicx, color}

% PDF-Verlinkungen und Metadaten
\usepackage[colorlinks=false]{hyperref}

% Bibliographie-Stil
\bibliographystyle{amsplain}

% Blindtext
\usepackage{blindtext}
\blindmathtrue

% Dashed Linien...
\usepackage{dashrule}

%%%%%%%%%%%%%%%%%%%%%%%%%%%%%%%%%%%%%%%%%%%%%%%%%%%%%%%%%%%%%%%%%%%%%%%%%%%%%%%
%%% Mathematik

% Standardpackages
\usepackage{amsmath, amssymb, stmaryrd, mathtools}

% "Bessere" Theorem-Umgebungen
\usepackage[standard,amsmath,thmmarks,hyperref]{ntheorem}

% Setze Label-Nummern nur, wenn diese auch referenziert werden
\mathtoolsset{showonlyrefs=true}

%!TEX root = main.tex

% griechisches Alphabet
\renewcommand{\epsilon}{\varepsilon}
\renewcommand{\theta}{\vartheta}

%%% Operatoren und Funktionen
\DeclarePairedDelimiter{\abs}{\lvert}{\rvert}
\DeclarePairedDelimiter{\norm}{\lVert}{\rVert}

\DeclarePairedDelimiter{\ceil}{\lceil}{\rceil}
\DeclarePairedDelimiter{\floor}{\lfloor}{\rfloor}

\newcommand{\skprod}[2]{\left\langle#1,#2\right\rangle}
\newcommand{\fracpart}[2]{\frac{\partial#1}{\partial#2}}

\newcommand{\Transp}{^{\mathrm{T}}}
\newcommand{\Stern}{^{*}}
\newcommand{\Int}[1]{#1^\circ}
\newcommand{\Ext}[1]{\overline{#1}}

\DeclareMathOperator{\spn}{span}
\DeclareMathOperator{\ee}{e}
\DeclareMathOperator{\ii}{i}

\newcommand{\grad}{\nabla}
\DeclareMathOperator{\divergenz}{div}
\newcommand{\hesse}{\nabla^2}

\newcommand\restr[2]{\ensuremath{\left.#1\right|_{#2}}}

\newcommand{\fa}{\text{für alle}~}

% fetter Vektor
\renewcommand{\vec}[1]{\mathbf{#1}}
\newcommand{\mat}[1]{\mathbf{#1}}

% Differential-d
\newcommand*\diff{\mathop{}\!\mathrm{d}}
\newcommand*\Diff[1]{\mathop{}\!\mathrm{d^#1}}

\DeclareMathOperator*{\esssup}{ess\,sup}
\newcommand\blank{{\mkern2mu\cdot\mkern2mu}}

% -*- root: main.tex -*-

\newcommand{\titel}{Numerische Behandlung von Self-Consistent Field Theory-Modellen mittels Reduzierter-Basis-Methoden}
\newcommand{\art}{Masterarbeit}
\newcommand{\autor}{Alexej Disterhoft}
\newcommand{\fach}{Mathematik}
\newcommand{\matrikelnr}{2669611}
\newcommand{\erstgutachter}{Prof. Dr. Thorsten Raasch}
\newcommand{\zweitgutachter}{Prof. Dr. Martin Hanke-Bourgeois}
% \newcommand{\monat}{Irgendwann}
% \newcommand{\jahr}{2015}
\newcommand{\ort}{Mainz}
\newcommand{\logo}{figures/title/logo_gross.eps}

\hypersetup{pdfinfo={
  Title=\titel,
  Author=\autor}
}

%!TEX root = main.tex

\DeclareAcronym{scft}{
	short            = SCFT,
	long             = selbstkonsistente Feldtheorie,
	long-plural-form = selbstkonsistenten Feldtheorie,
	short-plural     = {},
	foreign          = \foreign{engl.}{self-consistent field theory}
}

\DeclareAcronym{rbm}{
	short = RBM,
	long  = Reduzierte-Basis-Methode
}

\DeclareAcronym{fem}{
	short = FEM,
	long = Finite-Elemente-Methode
}

\DeclareAcronym{bnb}{
	short = BNB,
	long = Banach-Ne{\v c}as-Babu{\v s}ka-Theorem
}


\addbibresource{literature.bib}

\begin{document}
    % Deckblack / Titelseite
    %!TEX root = ../main.tex

\thispagestyle{plain}
\begin{titlepage}

\begin{center}

\huge{\textbf{\titel}}\\[1.5ex]
\LARGE{\textbf{\untertitel}}\\[6ex]
\LARGE{\textbf{\art}}\\[1.5ex]
\Large{im Fach \fach}\\[18ex]

\includegraphics[scale=0.5]{\logo}\\[6ex]

\normalsize
\begin{tabular}{p{5.4cm}p{6cm}}\\
vorgelegt von:  & \quad \autor\\[1.2ex]
Matrikelnummer: & \quad \matrikelnr\\[1.2ex]
Erstgutachter:  & \quad \erstgutachter\\[1.2ex]
Zweitgutachter: & \quad \zweitgutachter\\[3ex]
\end{tabular}

\end{center}
\end{titlepage}


    %%% Frontmatter-Teil
    \frontmatter{}

    % Danksagung
    % % -*- root: ../main.tex -*-

\thispagestyle{empty}
\vspace*{0.2\textheight}
\noindent\enquote{I guess you could call it a \enquote{failure}, but I prefer the term \enquote{learning
experience}.}\bigbreak

\hfill Andy Weir, \textit{The Martian}

\vfill{}
\begin{flushright}
\emph{Für meine Eltern.}
\end{flushright}
\cleardoublepage


    % Abstrakt
    % %!TEX root = ../main.tex

\chapter{Zusammenfassung} % (fold)
\label{cha:Zusammenfassung}

\blindtext

% chapter Zusammenfassung (end)


    % Inhaltsverzeichnis
    \tableofcontents

    % TODO-Liste
    \listoftodos

    %%% Hauptteil
    \mainmatter{}

    % Kapitel

    %!TEX root = ../main.tex

\setchapterpreamble[ul][0.6\textwidth]{%
    \dictum[Douglas Adams, \textit{The Hitchhiker’s Guide to the Galaxy}]{%
        \enquote{\foreignlanguage{english}{He attacked everything in life with a mixture of extraordinary genius and naive incompetence and it was often difficult to tell which was which.}}
    }
    \vspace*{2\baselineskip}
}
\chapter{Einleitung} % (fold)
\label{cha:el:einleitung}

Ein Polymer, oder auch Makromolekül, ist ein Molekül, welches sich aus vielen kleineren, sich wiederholenden Molekülen, sogenannten Monomeren, zusammensetzt.
Besteht ein Polymer aus nur einer Monomer-Art, dann spricht man von einem Homopolymer, sonst von einem Heteropolymer oder auch Copolymer.
Typischerweise besteht ein Polymer aus einer langen Kette von aneinanderhängenden Monomeren, es existieren aber auch weitere Konfigurationen, zum Beispiel stern- oder ringförmige Anordnungen.

Copolymere lassen sich anhand der Anordnung der Monomere weiter klassifizieren.
Bilden die verschiedenen Monomer-Gattungen homogene, zusammenhängende Gruppen, welche wiederum durch aneinanderreihen das Copolymer bilden, dann nennen wir dies ein Blockcopolymer, vergleiche \cref{fig:el:polymerketten}.

Es existieren unüberschaubar viele solcher Konfigurationen von Polymeren, und insbesondere Blockcopolymeren, weswegen man häufig auf das Studium vergleichsweise simpler Anordnungen zurückgreift.
Als besonders beliebt hat sich der Fall des kettenförmigen Blockcopolymers mit zwei Monomer-Typen, der Einfachheit halber A und B genannt, herausgestellt.
Diese Konfiguration wird auch als AB-Diblockcopolymer bezeichnet.

\begin{figure}[tb]
    \centering
    \begin{subfigure}[b]{\textwidth}
        \centering
        \includestandalone[width=0.8\textwidth]{tikz/einleitung/fig1}
    \end{subfigure}
    \\[1em]
    \begin{subfigure}[b]{\textwidth}
        \centering
        \includestandalone[width=0.8\textwidth]{tikz/einleitung/fig2}
    \end{subfigure}
    \\[1em]
    \begin{subfigure}[b]{\textwidth}
        \centering
        \includestandalone[width=0.8\textwidth]{tikz/einleitung/fig3}
    \end{subfigure}
    \caption[Skizzenhafte Darstellung verschiedener Polymerarten]{%
        Skizzenhafte Darstellung verschiedener Polymerarten.
        Von oben nach unten: Homopolymer, ein AB-Diblockcopolymer und ein sogenanntes statistisches AB-Copolymer, bei dem die beiden Monomer-Arten zufällig verteilt sind.
    }
    \label{fig:el:polymerketten}
\end{figure}

Von großem Interesse ist das Verhalten von Polymerschmelzen (\foreign{engl.}{polymer melt}), das heißt, des flüssigen Aggregatzustands eines Polymers, sowie das Verhalten von Gemischen verschiedener polymerer Stoffe.
So neigen die Gemische vieler Paare von Homopolymeren zu makroskopischer Phasenseparation, wie man es zum Beispiel von Öl und Essig kennt.
Eine ähnliche Tendenz findet man auch bei den Polymerschmelzen von Blockcopolymeren, hierbei kann aufgrund der Verbindung der verschiedenen Monomer-Blöcke aber keine makroskopische Phasenseparation auftreten, stattdessen kommt es zu einer periodischen, mikroskopischen Separation.
\cref{fig:el:phasen} zeigt einige mögliche Anordnungen, die bei Diblockcopolymeren tatsächlich experimentell beobachtet wurden.

\begin{figure}[tb]
    \centering
    \begin{subfigure}[b]{0.18\textwidth}
        \includegraphics[width=\textwidth]{figures/einleitung/fig1}
    \end{subfigure}
    \begin{subfigure}[b]{0.18\textwidth}
        \includegraphics[width=\textwidth]{figures/einleitung/fig2}
    \end{subfigure}
    \begin{subfigure}[b]{0.18\textwidth}
        \includegraphics[width=\textwidth]{figures/einleitung/fig3}
    \end{subfigure}
    \begin{subfigure}[b]{0.18\textwidth}
        \includegraphics[width=\textwidth]{figures/einleitung/fig4}
    \end{subfigure}
    \begin{subfigure}[b]{0.18\textwidth}
        \includegraphics[width=\textwidth]{figures/einleitung/fig5}
    \end{subfigure}
    \caption[Verschiedene Phasen bei Diblockcopolymeren]{%
        Verschiedene Phasen bei Diblockcopolymeren, welche experimentell beobachtet wurden, wobei hierbei nur eine der beiden Monomer-Gattungen dargestellt wird.
        Diese heißen von links nach rechts: Lamellar, Perforiert-Lamellar, Sphärisch, Zylindrisch, Gyroid.
        Diese Abbildung wurde \cite[Figure 1.18]{Matsen:2006ud} entnommen.
    }
    \label{fig:el:phasen}
\end{figure}

Da die experimentelle Bestimmung ohne Vorwissen über die möglichen, stabilen Anordnungen nur wenig erfolgversprechend ist, wird eine fundierte Theorie benötigt, auf Basis derer theoretische Vorhersagen getroffen werden können, die vorzugsweise wiederum experimentell belegbar sein sollten.
Da Diblockcopolymere einen relativ simplen Fall eines Copolymers darstellen, wurde sowohl in die experimentelle als auch theoretische Untersuchung dieser bereits vergleichsweise viel Arbeit investiert.

Als besonders nützliche und dennoch relativ einfache Theorie hat sich das auf der sogenannten \acp{scft} basierende Modell herausgestellt.

\subsection*{Mathematische Modellierung} % (fold)

Als Grundlage für die \acl{scft} dient eine Modellierung der Polymere als frei bewegliche Ketten (\foreign{engl.}{ideal chain}).
Dabei gibt es einige verschiedene Modelle, die hierfür verwendet werden.
Als relevante Modelle wollen wir hier ein diskretes, \enquote{grobkörniges} (\foreign{engl.}{coarse-grained}) Modell und das stetige Gaußsche Kettenmodell erwähnen, eine ausführliche Ausarbeitung findet man bei \textcites[Chapter 2]{Fredrickson:2006th}{rubinstein2003polymer}.
Im Folgenden sei $\vec{r}$ ein Vektor, der eine Position in einem Volumen angibt.

\begin{figure}[tb]
    \centering
        \includestandalone[width=0.6\textwidth]{tikz/einleitung/chains}
    \caption[%
        Polymerkette in diskretem und Gaußschen Kettenmodell
    ]{%
        Schematische Darstellung einer Polymerkette im diskreten Kettenmodell (links) und im stetigen Gaußschen Kettenmodell (rechts).
        Abbildung reproduziert nach \cite[Figure 2.1 und 2.5]{Fredrickson:2006th}.
    }
    \label{fig:el:kettenmodelle}
\end{figure}

Das \enquote{grobkörnige} Modell stellt die Polymerkette als eine diskrete Kette von Partikeln so dar, dass aneinanderhängende Monomere ähnlich einem Scharnier frei beweglich sind.
Dabei werden Wechselwirkung zwischen benachbarten Monomeren berücksichtigt, zwischen auf der Kette weit auseinanderliegenden Partikeln aber ignoriert.
Diese Wechselwirkungen können am Beispiel von \cref{fig:el:kettenmodelle} beispielsweise als Einschränkung des Winkels $\vartheta_9$ durch die gegenseitige Beeinflussung der Partikel $8, 9$ und $10$ auftreten.
Das diskrete Modell hängt stark mit den aus der Stochastik bekannten Random Walks zusammen und lässt sich deswegen auch ausführlich mit stochastischen und statistischen Methoden untersuchen.

Das stetige Gaußsche Kettenmodell, welches man unter anderem auch als stetigen Grenzfall des beschriebenen diskreten Modells erhält, hat sich als besonders nützlich erwiesen, sowohl bei analytischen als auch numerischen Betrachtungen.
Dabei wird die Polymerkette als stetige, linear elastische Faser aufgefasst und durch eine Kurve $\vec{r}(s)$ parametrisiert, wobei $s \in [0, 1]$ eine entlang der Kontur der Kette laufende Variable ist.
Ähnlich wie beim diskreten Modell findet man auch hier viele Zusammenhänge zu stochastischen Prozessen, hier vor allem zu Brownschen Bewegungen,
wodurch auch bei diesem Modell ein umfangreicher \enquote{Werkzeugkasten} zur Untersuchung zur Verfügung steht.

Da die benötigten stochastischen Ausführungen und Herleitungen für diese Arbeit nebensächlich sind, belassen wir es bei diesen informalen Beschreibungen und widmen uns nun der darauf aufbauenden \ac{scft}.

% subsection mathematische_modellierung (end)

\subsection*{Selbstkonsistente Feldtheorie} % (fold)

Die \acl{scft} ist ein weit verbreitetes theoretisches Modell der Physik um das Verhalten von Teilchen unter Einwirkung von Kräften, die durch Wechselwirkungen mit weiteren Teilchen auftreten, zu studieren und wird nicht nur im Zusammenhang mit Polymeren sondern zum Beispiel auch in der Thermodynamik oder Informatik verwendet.

Als Grundidee dient dabei, dass in einem System mit sehr vielen Objekten, welche miteinander wechselwirken, eine hinreichend gute Beschreibung der auf eines dieser Objekte wirkenden Kräfte durch Mitteln der Wechselwirkungen vorliegt.
Diese gemittelten Einwirkungen werden als externes Feld aufgefasst und ignorieren dabei Fluktuationen, das heißt, Veränderungen der wirkenden Kräfte durch das lokale Verhalten der einzelnen Teilchen.
Damit erreicht man effektiv die Reduktion eines Mehrkörperproblems auf ein Einkörperproblem und kann so das Verhalten eines solchen Systems untersuchen.

Dieses Prinzip lässt sich auch zur Untersuchung von Polymeren verwenden,
da ein einzelnes Polymer oftmals aus einer hohen vierstelligen Zahl von Atomen besteht und dadurch die Wechselwirkungen auf atomarem Level vernachlässigbar sind, weswegen auch die im vorherigen Abschnitt beschriebene Modellierung als Kette sinnvoll erscheint.
Aufbauend auf den beschriebenen Modellen, in diesem Fall dem Gaußschen Kettenmodell, kann man so die statistische räumliche Verteilung beziehungsweise Ausrichtung eines Polymers bestimmen.

Wir beschränken uns im Folgenden auf die Beschreibung der \ac{scft} für die inkompressible Schmelze eines AB-Diblockcopolymers aufbauend auf dem stetigen Gaußschen Kettenmodell und folgen dabei größtenteils den Ausführungen von \textcite{Matsen:1994bz,Stasiak:2011ba}.

Betrachte eine einzelne Volumenzelle, beispielsweise einen Würfel, welche selbst Teil eines größeren Systems sein kann.
Diese Zelle enthalte $n$ AB-Diblockcopolymere, welche jeweils aus einem A-Block und einem B-Block bestehen, wobei diese wiederum aus $N_{\mathrm{A}}$ Monomeren vom Typ A respektive aus $N_{\mathrm{B}}$ Monomeren vom Typ B bestehen.
Der Polymerisationsgrad, das heißt, die Gesamtanzahl an Monomeren in einem Polymer, ergibt sich damit zu $N = N_{\mathrm{A}} + N_{\mathrm{B}}$.
Weiter bezeichne $f = N_{\mathrm{A}} / N$ den Anteil an A-Monomeren im gesamten Polymer.

Als vereinfachende Annahmen sei die statistische Länge $a$ eines Monomers, auch Kuhn-Länge genannt, der beiden Monomer-Gattungen gleich und ein Monomer beider Gattungen nehme das selbe Volumen $1 / \rho_{0}$ ein.
Das Gesamtvolumen der Schmelze in dieser Zelle ist damit gegeben durch $V = n N / \rho_{0}$.

Wie bei der Beschreibung des Gaußschen Modells sei $s \in [0, 1]$ eine Distanz entlang der Kontur eines Polymers, wobei $s = 0$ und $s = 1$ den beiden Enden entspreche.

Die wichtigsten Größen bei der \ac{scft} sind nun die Konzentrationen $\phi_{\mathrm{A}}(\vec{r})$ und $\phi_{\mathrm{B}}(\vec{r})$ der A- und B-Monomere an einer Position $\vec{r}$ in der betrachteten Zelle und die externen Felder $\omega_{\mathrm{A}}(\vec{r})$ und $\omega_{\mathrm{B}}(\vec{r})$, welche auf die jeweiligen Monomer-Gattungen wirken.

Als Ausgangspunkt für die Bestimmungen möglicher stabiler Anordnungen der Polymere in der Schmelze betrachtet man das folgende Funktional, welches eine Approximation für die sogenannte freie Energie des Systems liefert und dessen physikalische Motivation den Rahmen dieser Einführung sprengen würde, siehe zum Beispiel \cite{Matsen:2006ud,Fredrickson:2006th}.
Die freie Energie $F$ eines einzelnen Polymers lässt sich bestimmen als
\begin{equation}
\label{eq:el:freie_energie_funktional}
    \frac{F}{nk_{\mathrm{B}}T} = - \ln \frac{Q}{V} + \frac{1}{V} \int \chi N \phi_{\mathrm{A}}(\vec{r}) \phi_{\mathrm{B}}(\vec{r}) - \omega_{\mathrm{A}}(\vec{r}) \phi_{\mathrm{A}}(\vec{r}) - \omega_{\mathrm{B}}(\vec{r}) \phi_{\mathrm{B}}(\vec{r}) \diff \vec{r},
\end{equation}
wobei $\chi$ der sogenannte Flory-Huggins-Wechselwirkungsparameter für die Wechselwirkungen zwischen den Monomeren vom Typ A und B und $k_{\mathrm{B}} T$ die thermische Energie ist.

Stabile Anordnungen entsprechen dabei Sattelpunkten von $F$ bezüglich der Konzentrationen $\phi_{\mathrm{A}}$, $\phi_{\mathrm{B}}$ und der Felder $\omega_{\mathrm{A}}$ und $\omega_{\mathrm{B}}$.
Betrachtet man die Funktionalableitungen von $F$ bezüglich dieser Größen, dann erhält man eine Menge von Gleichungen, welche oft als \ac{scft}-Gleichungen bezeichnet werden, da Anhand dieser die gesuchten Sattelpunkte bestimmt werden können.

Diese Gleichungen bestehen aus der Inkompressibilität der Schmelze, welche durch die Bedingung
\begin{equation}
\label{eq:el:inkompressibilitaet}
    \phi_{\mathrm{A}}(\vec{r}) + \phi_{\mathrm{B}}(\vec{r}) = 1
\end{equation}%
berücksichtigt wird, sowie der Kopplung der Felder und der Konzentrationen durch
\begin{equation}
\label{eq:el:felder}
    \begin{aligned}
        \omega_{\mathrm{A}}(\vec{r}) = \chi N \phi_{\mathrm{B}}(\vec{r}) + \xi(\vec{r}),\\
        \omega_{\mathrm{B}}(\vec{r}) = \chi N \phi_{\mathrm{A}}(\vec{r}) + \xi(\vec{r}),
    \end{aligned}
\end{equation}%
wobei mit dem Lagrange-Multiplikator $\xi(\vec{r})$ die Inkompressibilität \cref{eq:el:inkompressibilitaet} erzwungen wird.
Weiter erhält man eine Darstellung der Konzentrationen in Form von
\begin{equation}
\label{eq:el:konzentrationen}
    \begin{aligned}
        \phi_{\mathrm{A}}(\vec{r}) = \frac{V}{Q} \int_{0}^{f} q(\vec{r}, s) q^{\dagger}(\vec{r}, s) \diff s,\\
        \phi_{\mathrm{B}}(\vec{r}) = \frac{V}{Q} \int_{f}^{1} q(\vec{r}, s) q^{\dagger}(\vec{r}, s) \diff s,
    \end{aligned}
\end{equation}%
wobei $Q = Q(\omega_{\mathrm{A}}, \omega_{\mathrm{B}})$ die Partitionsfunktion (\foreign{engl.}{partition function}) eines einzelnen Polymers ist und durch
\begin{equation}
\label{eq:el:partitionsfunktion}
    Q = \int q(\vec{r}, 1) \diff \vec{r}
\end{equation}%
bestimmt wird.

Die in den Gleichungen \cref{eq:el:konzentrationen,eq:el:partitionsfunktion} auftretende Funktion $q(\vec{r}, s)$ wird als Vorwärts-Propagator bezeichnet und erfüllt die parabolische partielle Differentialgleichung
\begin{equation}
\label{eq:el:forward_propagator}
    \left\{
    \begin{aligned}
        \frac{\partial q}{\partial s}(\vec{r}, s) &= \frac{a^{2}N}{6} \Delta q(\vec{r}, s) - \omega_{\alpha(s)}(\vec{r}) q(\vec{r}, s),\\
        q(\vec{r}, 0) &= 1.
    \end{aligned}
    \right.
\end{equation}%
Analog bezeichnet man $q^{\dagger}(\vec{r}, s)$ als Rückwärts-Propagator, welcher eine ähnliche Differentialgleichung, nämlich
\begin{equation}
\label{eq:el:backward_propagator}
    \left\{
    \begin{aligned}
        -\frac{\partial q^{\dagger}}{\partial s}(\vec{r}, s) &= \frac{a^{2}N}{6} \Delta q^{\dagger}(\vec{r}, s) - \omega_{\alpha(s)}(\vec{r}) q^{\dagger}(\vec{r}, s),\\
        q^{\dagger}(\vec{r}, 1) &= 1,
    \end{aligned}
    \right.
\end{equation}%
erfüllt.
Dabei ist die Funktion $\omega_{\alpha(s)}$ gegeben durch
\begin{equation}
    \omega_{\alpha(s)}(\vec{r}) = \begin{cases}
        \omega_{\mathrm{A}}(\vec{r}), & 0 \leq s \leq f\\
        \omega_{\mathrm{B}}(\vec{r}), & f < s \leq 1.
    \end{cases}
\end{equation}

Der Propagator $q(\vec{r}, s)$ repräsentiert das statistische Gewicht, im Wesentlichen also eine nichtnormalisierte Wahrscheinlichkeit, dass man ein Polymer findet, welches irgendwo innerhalb des Volumens beginnt, und dessen Teilstück, das zur Konturvariable $s$ gehört, sich an der Position $\vec{r}$ befindet.
Die Anfangsbedingung $q(\vec{r}, 0) = 1$ lässt sich damit so interpretieren, dass ein Polymer der Länge Null von den externen Feldern nicht beeinflusst wird.
Der Rückwärts-Propagator hat die selbe Bedeutung, allerdings wird hierbei das andere Ende des Polymers festgehalten.

Die Tatsache, dass sich die Konzentrationen $\phi_{\mathrm{A}}$ und $\phi_{\mathrm{B}}$ aus Lösungen der obigen Differentialgleichungen bestimmen lassen, entstammt aus dem Gaußschen Kettenmodell und dessen stochastischen Hintergründen.

Je nachdem, welches Szenario man betrachtet, das heißt, eine Volumenzelle innerhalb eines größeren Systems, oder eine Zelle, die durch feste Wände begrenzt wird, erhält man verschiedene Randbedingungen an die beiden Differentialgleichungen, so entspricht ersteres beispielsweise periodischen Randbedingungen.

Ein weiterer Punkt, der hier noch erwähnt werden soll, da er sich im Rahmen dieser Arbeit als äußerst nützlich erweisen wird, ist, dass das Funktional $F$ der freien Energie \cref{eq:el:freie_energie_funktional} invariant ist bezüglich konstanter Verschiebungen der Felder $\omega_{\mathrm{A}}$ und $\omega_{\mathrm{B}}$, wie beispielsweise \cite{Ceniceros:2006is} entnommen werden kann.

Diese Gleichungen, genauer \cref{eq:el:inkompressibilitaet,eq:el:felder,eq:el:konzentrationen,eq:el:partitionsfunktion,eq:el:forward_propagator,eq:el:backward_propagator}, lassen sich nun numerisch lösen und so stabile Anordnungen über die Sattelpunkte von \cref{eq:el:freie_energie_funktional} bestimmen.

% subsection selbstkonsistente_feldtheorie (end)

\subsection*{Einsatz numerischer Methoden} % (fold)

Es gibt verschiedene Ansätze, die \ac{scft}-Gleichungen zu einem numerischen Verfahren zu verarbeiten, die meisten führen auf das folgende iterative, Newton-artige Schema:
\begin{enumerate}[label={\itshape\roman*.},ref={\itshape\roman*}]
    \item Zunächst werden zwei externe Felder $\omega^{(0)}_{\mathrm{A}}$ und $\omega^{(0)}_{\mathrm{B}}$ generiert, typischerweise zufällig, um von vornherein auftretende Verzerrungen zu verhindern.
    \item\label{eq:el:iterationsverfahren_punkt_2} Die beiden partiellen Differentialgleichungen \cref{eq:el:forward_propagator,eq:el:backward_propagator} werden für die Felder $\omega^{(k)}_{\mathrm{A}}$ und $\omega^{(k)}_{\mathrm{B}}$ gelöst.
    \item Die Konzentrationen $\phi^{(k)}_{\mathrm{A}}$ und $\phi^{(k)}_{\mathrm{B}}$ werden durch Auswerten der Gleichungen \cref{eq:el:partitionsfunktion,eq:el:konzentrationen} bestimmt.
    \item Diese werden nun benutzt um aus den Gleichungen \cref{eq:el:felder} die zu diesen Konzentrationen zugehörigen Felder zu bestimmen.
    Aus diesen Feldern werden nun mit einem sogenannten Mixing-Verfahren die neuen Felder $\omega^{(k+1)}_{\mathrm{A}}$ und $\omega^{(k+1)}_{\mathrm{B}}$ für die nächste Iteration erzeugt.
    Das Mixing dient dazu, die Konvergenz des Verfahrens sicherzustellen beziehungsweise zu verbessern, typischerweise gehen hier die Inkompressibilität \cref{eq:el:inkompressibilitaet} und zurückliegende Iterationen ein.
    \item Sind die neuen Konzentrationen und Felder noch kein Sattelpunkt von \cref{eq:el:freie_energie_funktional}, dann gehe zurück zu \cref{eq:el:iterationsverfahren_punkt_2}.
\end{enumerate}

Ein Beispiel für eine mögliche Anordnung, die durch dieses Iterationsverfahren für den Fall einer Raumdimension bestimmt wurde, ist in \cref{fig:el:felder_nach_iterationsverfahren} zu sehen.

\begin{figure}[tb]
    \centering
    \begin{subfigure}[b]{0.45\textwidth}
        \centering
        \includestandalone[width=1\textwidth]{tikz/einleitung/iter1}
        \caption{Felder}
    \end{subfigure}
    ~
    \begin{subfigure}[b]{0.45\textwidth}
        \centering
        \includestandalone[width=1\textwidth]{tikz/einleitung/iter2}
        \caption{Konzentrationen}
    \end{subfigure}
    \caption[%
    Eindimensionales Beispiel einer stabilen Anordnung eines Diblockcopolymers
    ]{%
        Eindimensionales Beispiel einer stabilen Anordnung eines Diblockcopolymers, welche mittels \ac{scft} bestimmt wurde.
        Simuliert wurde auf einem Intervall der Länge $L = 10$ mit den relevanten Größen $f = 1/2$, $\chi N = 25$ und $a^{2} N / 6 = 10 / 3$.
        Monomer-Typ A entspricht den blauen und B dementsprechend den orangen Graphen.
    }
    \label{fig:el:felder_nach_iterationsverfahren}
\end{figure}

Bei dem beschriebenen Verfahren stellen sich die folgenden beiden Probleme als besonders wichtig heraus, da sie massiv die Laufzeit des Iterationsverfahren beeinflussen:
\begin{enumerate}[label={\itshape\roman*.}]
    \item das Lösungsverfahren für die parabolischen partiellen Differentialgleichungen \cref{eq:el:forward_propagator} und \cref{eq:el:backward_propagator},
    \item das Mixing-Verfahren, mit dem iterativ neue Felder $\omega_{\mathrm{A}}$ und $\omega_{\mathrm{B}}$ bestimmt werden.
\end{enumerate}

Auf den zweiten Punkt, das Mixing-Verfahren, werden wir im Verlauf dieser Arbeit nicht weiter eingehen.
Trotz dessen, und insbesondere, da es viele, sehr verschiedene Verfahren gibt, die hierfür zum Einsatz kommen, wollen wir hier einige Ansätze nennen.
Da es sich beim Mixing-Schritt im Wesentlichen um die Suche nach einem Sattelpunkt eines nichtlinearen Funktionals handelt, lassen sich hierfür viele bekannte Verfahren der nichtlinearen Optimierung, aber auch aus anderen Bereichen, anwenden.

Dies reicht von einem Quasi-Newton-Verfahren bei \textcite{Matsen:1994bz}, welches direkt an den darin verwendeten Löser für die partielle Differentialgleichung gekoppelt ist, bis zu Integrationsverfahren, wie zum Beispiel Runge-Kutta-Verfahren oder Mehrschrittverfahren.
Ein auf einem solchen Integrationsverfahren basierendes Mixing, neben einem weiteren, welches auf einem Conjugated-Gradients-Verfahren aufbaut, findet sich bei \textcite{Ceniceros:2006is}.
Als besonders nützlicher, da effektiver, Ansatz hat sich das sogenannte Anderson-Mixing erwiesen, siehe die Arbeiten von \textcite{Thompson:2004um,Stasiak:2011ba}.
Dabei werden neue Felder durch Kombination der Felder vieler zurückliegender Iterationen gewonnen.
Ein weiteres, einer Picard-Iteration ähnliches, Verfahren findet man bei \textcite{Drolet:1999bs}.

Unser Hauptaugenmerk in dieser Arbeit liegt auf dem ersten Problem, dem wiederholten Lösen der parabolischen partiellen Differentialgleichung \cref{eq:el:forward_propagator}.
Da es, abhängig vom gewählten Mixing-Verfahren, oftmals eine Iterationsanzahl im dreistelligen Bereich oder höher benötigt, bis eine zufriedenstellende Genauigkeit bei der Sattelpunktgleichung erreicht ist, und damit insbesondere auch die partielle Differentialgleichung so oft gelöst werden muss, ist es wichtig, dass das Lösungsverfahren möglichst effizient ist.
Weiter darf zu Gunsten der Laufzeit aber auch nicht die Genauigkeit des Lösers vernachlässigt werden, da sich dies im Iterationsverfahren durch Instabilität und zusätzliche Iterationen niederschlagen kann.

Ähnlich wie beim Mixing-Schritt wurden bereits viele verschiedene Ansätze mit mehr oder weniger zufriedenstellenden Ergebnissen verfolgt.
Da es sich bei \cref{eq:el:forward_propagator} im Grunde um eine Diffusionsgleichung handelt, lassen sich gut bekannte Verfahren, zum Beispiel ein Finite-Differenzen-Verfahren, anwenden.
So wird in der Arbeit von \textcite{Drolet:1999bs} ein Crank-Nicolson-Verfahren eingesetzt, wobei hierbei explizit der Laufzeit Vorrang gegenüber der Genauigkeit gegeben wurde.

Als guter Kompromiss zwischen Laufzeit und Genauigkeit haben sich Spektral- und Pseudospektralverfahren herausgestellt.
Erstere wurden von \textcite{Matsen:1994bz} erfolgreich eingesetzt, wobei hier erst das explizite Berücksichtigen der Symmetrien der zu erwartenden resultierenden Anordnung bei der Konstruktion des Spektralverfahrens zu annehmbaren Laufzeiten führt.
Die verwendeten Pseudospektralverfahren kommen zwar nicht an die Genauigkeit der Spektralverfahren heran, können aber unter Ausnutzung der Struktur der partiellen Differentialgleichung massiv Laufzeit einsparen.
Dazu wird bei \textcite{Rasmussen:2002kt} der Differentialoperator mittels Operator-Splitting so zerlegt, dass man das Lösen der Differentialgleichung mittels schneller Fourier-Transformation im Wesentlichen auf komponentenweise Vektor-Multiplikationen zurückführt.
Das daraus resultierende Verfahren zweiter Ordnung wurde von \textcite{GarciaCervera:2006uu,Ranjan:2007kl} auf unterschiedliche Weisen zu Verfahren vierter Ordnung erweitert, ohne signifikant Laufzeit einzubüßen.
Eine gute Übersicht über die meisten der genannten Methoden findet man bei \textcites[Section 3.6]{Fredrickson:2006th}{Audus:2013ep}.

Obwohl man in der Literatur zur \ac{scft} für Polymere verschiedenste Verfahren für das Lösen der partiellen Differentialgleichung findet, ist ein Finite-Elemente-Ansatz basierend auf einer Raum-Zeit-Variationsformulierung der partiellen Differentialgleichung unseres Wissens nach bisher nicht verfolgt worden.
Hier wollen wir anknüpfen und zugleich die Tatsache, dass die Differentialgleichung wiederholt für verschiedene Felder gelöst werden muss, in Form eines Reduzierte-Basis-Ansatzes ausnutzen.
Dies ist in aller Kürze das angestrebte Ziel dieser Arbeit.

% subsection einsatz_numerischer_methoden (end)

\subsection*{Aufbau der restlichen Arbeit}

In \cref{cha:gl:grundlagen} werden zunächst die benötigten funktionalanalytischen und numerischen Grundlagen eingeführt, beziehungsweise wiederholt.
Einige weitere Aussagen, die sich thematisch nur bedingt in diesem Kapitel unterbringen lassen, werden in \cref{cha:funktionalanalytische_grundlagen} gesammelt.

Es folgt \cref{cha:ps:problemstellung}, in dem die parabolische partielle Differentialgleichung, welche im Rahmen dieser Einleitung bereits vorgestellt wurde, konkretisiert und als parametrische Differentialgleichung formalisiert wird.
Anschließend werden wünschenswerte Aussagen hergeleitet, die eine sinnvolle Bearbeitung durch numerische Methoden erst ermöglichen.

\cref{cha:der_eindimensionale_fall} dient dazu, die bis dahin ausgearbeitete Theorie auf den einfachen, eindimensionalen Fall anzuwenden und numerische Methoden für diesen zu entwickeln.

Im nachfolgenden \cref{cha:reduzierte_basis_methode} werden diese Methoden als Grundlage für eine Reduzierte-Basis-Methode verwendet und dadurch ein effizientes Verfahren für die Approximation der parametrischen Differentialgleichung entwickelt.

% subsection aufbau_der_restlichen_arbeit (end)

% chapter einleitung (end)

    %!TEX root = ../main.tex

\setchapterpreamble[ul][0.6\textwidth]{%
    \dictum[Iain M. Banks, \textit{The Algebraist}]{\enquote{Elegance is an algorithm.}}
    \vspace*{2\baselineskip}
}
\chapter{Grundlagen} % (fold)
\label{cha:gl:grundlagen}

\mdo{Bei Bedarf erweitern!}

In diesem ersten Kapitel werden die für diese Arbeit benötigten Grundlagen aus der Funktionalanalysis und der Numerik zusammengefasst und wiederholt.
Wir orientieren uns in den ersten beiden, funktionalanalytischen Abschnitten maßgeblich an \textcite{Dautray:1992by,Lions:1972tg}, während für die nachfolgenden Abschnitte [Quelle].

\mdo{Quellen ergänzen!}

\section{Bochner-Räume} % (fold)
\label{sec:gl:br:bochner_raeume}

Wir beginnen mit der Einführung sogenannter Bochner-Räume.
Dabei handelt es sich um Verallgemeinerungen der Lebesgue-Räume $L_{p}$ auf Banachraum-wertige Funktionen.
Bochner-Räume treten bei der Betrachtung parabolischer partieller Differentialgleichungen in natürlicher Weise auf.

\begin{Definition}[{{{\cite[Definition XVIII.1.1]{Dautray:1992by}}}}]
\label{def:gl:br:bochner_raum}
    Sei $X$ ein Banachraum.
    Weiter seien $- \infty \leq a < b \leq \infty$ und $1 \leq p < \infty$.
    Als \emph{Bochner-Raum} $L_{p}(a, b; X)$ bezeichnen wir die Menge (der Äquivalenzklassen) $L_{p}$-integrierbarer Funktionen $f \colon [a, b] \to X$, das heißt, aller Lebesgue-messbarer Funktionen auf $[a, b]$ mit
    \begin{equation}
        \norm{f}_{L_{p}(a, b; X)} \deq \left( \int_{a}^{b} \norm{f(t)}_{X}^{p} \diff t \right)^{1 / p} < \infty.
    \end{equation}
    Weiterhin ist der \emph{Bochner-Raum} $L_{\infty}(a, b; X)$ definiert als die Menge (der Äquivalenzklassen) der für fast alle $t \in [a, b]$ wesentlich beschränkten Funktionen mit
    \begin{equation}
        \norm{f}_{L_{\infty}(a, b; X)} \deq \esssup_{t \in [a, b]} \norm{f(t)}_{X} < \infty.
    \end{equation}
\end{Definition}
\nomenclature{$X, Y$}{Banachräume}

\begin{Lemma}[{{\cite[Proposition XVIII.1.1]{Dautray:1992by}}}, {{\cite[Abschnitt 1.1.3]{Lions:1972tg}}}]
\label{lem:gl:br:bochner_ist_banachraum}
    Für alle $1 \leq p \leq \infty$ ist $L_{p}(a, b; X)$ ein Banachraum.
    Falls $H$ ein Hilbertraum ist, so auch $L_{2}(a, b; H)$.
\end{Lemma}

\begin{Definition}[Schwache Zeitableitung, {{{\cite{Dautray:1992by}}}}]
\label{def:gl:br:schwache_zeitableitung}
    Seien $X$ und $Y$ Banachräume mit $X \hookrightarrow Y$ und $u \in L_{2}(a, b; X)$.
    Die distributionelle Ableitung $\frac{\partial}{\partial t} u \in L_{2}(a, b; Y)$ sei definiert als das $v \in L_{2}(a, b; Y)$, welches
    \begin{equation}
        \int_{a}^{b} v(t) \varphi(t) \diff t = - \int_{a}^{b} u(t) \frac{\partial}{\partial t} \varphi(t) \diff t \quad \fa \phi \in C^{\infty}_{0}((a, b), \mathbb{R})
    \end{equation}
    erfüllt, falls ein solches $v$ existiert.
\end{Definition}

\begin{Bemerkung}
    Je nach Situation werden wir der Einfachheit halber eine der Schreibweisen $\frac{\partial}{\partial t} u = u' = u_{t}$ verwenden.
\end{Bemerkung}

Im Zuge dieser Arbeit werden wir es stets mit Hilberträumen zu tun haben.
Dabei werden wir oft auf folgendes Konstrukt zurückgreifen.

\begin{Definition}
\label{def:gl:br:gelfand_tripel}
    Seien $V$ und $H$ separable Hilberträume mit den Dualräumen $V'$ und $H'$.
    Weiter sei $V$ ein dichter Unterraum von $H$.
    Durch Identifikation von $H$ mit seinem Dualraum $H'$ erhalten wir das sogenannte \emph{Gelfand-Tripel} $(V, H, V')$, oder auch
    \begin{equation}
        V \denseinclusion H \simeq H' \denseinclusion V',
    \end{equation}
    wobei die Inklusionen jeweils dichte stetige Einbettungen sind.
\end{Definition}
\nomenclature{$V, H$}{Hilberträume}

\begin{Bemerkung}
    Mit $\skprod{\blank}{\blank}_{V}$ und $\skprod{\blank}{\blank}_{H}$ bezeichnen wir das Skalarprodukt auf $V$ respektive $H$.
    Weiterhin $\skprod{\blank}{\blank}_{V' \times V}$ wird auch für die duale Paarung auf $V' \times V$, die als die eindeutige stetige Fortsetzung von $\skprod{\blank}{\blank}_{H}$ definiert ist, verwendet.
    Insbesondere gilt für $u \in H \subset V'$ und $v \in V$ die Gleichheit
    \begin{equation}
        \skprod{u}{v}_{V' \times V} = \skprod{u}{v}_{H}.
    \end{equation}
\end{Bemerkung}
\nomenclature{$\skprod{\blank}{\blank}$}{Je nach Index Skalarprodukt oder duale Paarung}

\begin{Definition}[{{{\cite[Definition XVIII.2.4]{Dautray:1992by}}}}]
\label{def:gl:br:bochner_raum_W}
    Sei $V \denseinclusion H \denseinclusion V'$ ein Gelfand-Tripel.
    Definiere den Raum $W(a, b; V, V')$ als
    \begin{equation}
        W(a, b; V, V') \deq \Set{u \in L_{2}(a, b; V) \given u' \in L_{2}(a, b; V')}
    \end{equation}
    wobei $u'$ im Sinne von \cref{def:gl:br:schwache_zeitableitung} zu verstehen ist.
\end{Definition}
\nomenclature{$\denseinclusion$}{Dichte stetige Einbettung}

\begin{Bemerkung}
\label{bem:gl:br:bochner_alternative_darstellung}
    Eine alternative Darstellung von $W(a, b; V, V')$ ist
    \begin{equation}
        W(a, b; V, V') = L_{2}(a, b; V) \cap H^{1}(a, b; V').
    \end{equation}
\end{Bemerkung}

Bochner-Räume lassen sich anhand folgender Charakterisierung auch als Tensor-Produkte auffassen.
\begin{Satz}[{{\cite[Theorem 12.6.1, Theorem 12.7.1]{Aubin:2000un}}}]
\label{satz:gl:br:bochner_als_tensorprodukt}
    Seien $H$ ein Hilbertraum und $I \subset \mathbb{R}$ eine offene Menge.
    Dann ist der Bochner-Sobolev-Raum $H^{m}(I; H)$, $m \geq 0$, isometrisch zum Hilbert-Tensor-Produkt $H^{m}(I) \otimes H$.
\end{Satz}

\begin{Lemma}[{{{\cite[Proposition XVIII.2.6]{Dautray:1992by}}}}]
\label{lem:gl:br:bochner_W_ist_hilbertraum}
    Mit dem Skalarprodukt
    \begin{equation}
        \skprod{u}{v}_{W(a, b; V, V')} \deq \skprod{u'}{v'}_{L_{2}(a, b; V')} +  \skprod{u}{v}_{L_{2}(a, b; V)}
    \end{equation}
    und der induzierten Norm
    \begin{equation}
        \begin{aligned}
            \norm{u}_{W(a, b; V, V')}
            \deq& \left( \int_{a}^{b} \norm{u'(t)}_{V'}^{2} \diff t + \int_{a}^{b} \norm{u(t)}_{V}^{2} \diff t \right)^{1/2}
            \\=& \left( \norm{u'}_{L_{2}(a, b; V')}^{2} + \norm{u}_{L_{2}(a, b; V)}^{2} \right)^{1/2}
        \end{aligned}
    \end{equation}
    ist $W(a, b; V, V')$ ein Hilbertraum.
\end{Lemma}

\begin{Definition}
\label{def:gl:br:stetige_funktionen}
    Mit $\mathcal C([a, b]; X)$ bezeichnen wir die Menge aller bezüglich der Norm $\norm{f} = \sup_{t \in [a, b]} \norm{f(t)}_{X}$ stetigen Funktionen $f \colon [a, b] \to X$.
\end{Definition}

\begin{Satz}[{{{\cite[Theorem XVIII.2.1]{Dautray:1992by}}}}]
\label{satz:gl:br:bochner_eingebettet_in_stetigen_funktionen}
    Seien $a, b \in \mathbb{R}$, dann stimmt jedes $u \in W(a, b)$ fast überall mit einer stetigen Funktion von $[a, b]$ nach $H$ überein.
    Genauer gilt
    \begin{equation}
        W(a, b; V, V') \hookrightarrow \mathcal C([a, b]; H).
    \end{equation}
\end{Satz}
\nomenclature{$\hookrightarrow$}{Stetige Einbettung}

\begin{Korollar}[{{{\cite[Remark XVIII.2.4]{Dautray:1992by}}}}]
\label{kor:gl:br:bochner_spur_wohldefiniert}
    Sei $a, b \in \mathbb{R}$ und $u \in W(a, b; V, V')$.
    Dann sind $u(a), u(b) \in H$ wohldefiniert.
\end{Korollar}

\mfix{Prüfen, ob das so auch stimmt.}
\begin{Korollar}
\label{kor:gl:br:einbettungskonstante_M_e}
    Seien $a, b \in \mathbb{R}$.
    Die Einbettungskonstante
    \begin{equation}
        \label{eq:gl:br:einbettungskonst_M_e}
        M_{e} \deq \sup_{\substack{u\in W(a, b; V, V')\\u \neq 0}} \frac{\norm{u(0)}_{H}}{\norm{u}_{W(a, b; V, V')}}
    \end{equation}
    ist gleichmäßig beschränkt in der Wahl $V \hookrightarrow H$ und hängt nur im Fall $T \to 0$ von $T$ ab.

    \begin{Beweis}
        Siehe \textcite[Beweis zu Theorem XVIII.2.1]{Dautray:1992by}.
    \end{Beweis}
\end{Korollar}

% section bochner_r_ume (end)

\section{Lineare Evolutionsgleichungen} % (fold)
\label{sec:gl:le:lineare_evolutionsgleichungen}

In diesem Abschnitt werden lineare Evolutionsgleichungen, eine bestimmte Unterart parabolischer partieller Differentialgleichungen, eingeführt.
Diese Einführung orientiert sich an \textcite{Lions:1971wp,Schwab:2009ec,Urban:2014kg}.

Zunächst definieren wir, was wir unter dem Begriff einer linearen Evolutionsgleichungen verstehen wollen.
Anschließend leiten wir eine Raum-Zeit-Variationsformulierung her und geben einen Satz an, der unter geeigneten Voraussetzungen Existenz und Eindeutigkeit einer Lösung dieser Variationsformulierung garantiert.

Wir befinden uns nun im Folgenden Setting:
wie in \cref{def:gl:br:gelfand_tripel} seien $V$ und $H$ zwei separable Hilberträume und $(V, H, V')$ das daraus resultierende Gelfand-Tripel.
Wie zuvor verwenden wir $\skprod{\blank}{\blank}$ mit entsprechendem Index sowohl für die Skalarprodukte auf $V$ und $H$, als auch für die duale Paarung auf $V' \times V$.

Es sei $0 < T < \infty$ und damit $[0, T]$ ein endliches Zeitintervall.
Weiterhin sei für fast alle $t \in [0, T]$ eine Familie $a(\blank, \blank; t)$ von Bilinearformen
\begin{equation}
    \label{eq:gl:le:bilinearformen_familie}
    a(\blank, \blank; t) \colon V \times V \to \mathbb{R}, \quad (\eta, \zeta) \mapsto a(\eta, \zeta; t)
\end{equation}
gegeben.
Für alle $\eta, \zeta \in V$ sei die Abbildung $t \mapsto a(\eta, \zeta; t)$ messbar auf $[0, T]$ und erfülle die folgenden Bedingungen:

\begin{Annahme}
\label{ann:gl:le:bilinearform_eigenschaften}
    \leavevmode
    \begin{thmenumerate}
        \item \emph{Stetigkeit.}
        Es existiert eine Konstante $0 < M_{a} < \infty$ mit
        \begin{equation}
            \label{eq:gl:le:bilinearform_stetig}
            \abs{a(\eta, \zeta; t)} \leq M_{a} \norm{\eta}_{V} \norm{\zeta}_{V} \quad \fa \eta, \zeta \in V
        \end{equation}
        für fast alle $t \in [0, T]$.
        \item \emph{G\r{a}rding-Ungleichung}.
        Es existieren Konstanten $\alpha > 0$ und $\lambda \geq 0$ mit
        \begin{equation}
            \label{eq:gl:le:bilinearform_garding}
            a(\eta, \eta; t) + \lambda \norm{\eta}_{H}^{2} \geq \alpha \norm{\eta}_{V}^{2} \quad \fa \eta \in V
        \end{equation}
        für fast alle $t \in [0, T]$.
    \end{thmenumerate}
\end{Annahme}

Nach dem Rieszschen Darstellungssatz, vergleiche \cite[Theorem \S{}22.1]{Halmos:1957vd}, wird durch die Familie $a(\blank, \blank; t)$ für fast alle $t \in [0, T]$ durch
\begin{equation}
    \skprod{A(t)\eta}{\zeta}_{H} = a(\eta, \zeta; t)
\end{equation}
eine Familie stetiger linearer Operatoren $A(t) \in \mathcal L(V, V')$ induziert.

Mit dieser Vorarbeit können wir nun definieren, was wir unter einer linearen Evolutionsgleichung verstehen wollen.

\mfix{Quelle angeben!}
\begin{Definition}
\label{def:gl:le:lineare_evolutionsgleichung}
    Seien $a(\blank, \blank; t)$ und $A(t)$ wie oben gegeben.
    Weiterhin sei ein \emph{Quellterm} $g \in L_{2}(0, T; V')$ und ein \emph{Anfangswert} $u_{0} \in H$ gegeben.
    Als \emph{lineare Evolutionsgleichung} bezeichnen wir die parabolische partielle Differentialgleichung
    \begin{equation}
        \label{eq:gl:le:lineare_evolutionsgleichung}
        \begin{cases}
            u_{t}(t) + A(t) u(t) = g(t)     &\text{in}~V', \quad \text{für fast alle}~t \in I, \\
            u(0) = u_{0}                    &\text{in}~H.
        \end{cases}
    \end{equation}
\end{Definition}

\begin{Bemerkung}
\leavevmode
\begin{thmenumerate}
    \item Die Anfangswertbedingung $u(0) = u_{0}$ in $H$ ist wegen \cref{kor:gl:br:bochner_spur_wohldefiniert} wohldefiniert.
    \item Ist die Bilinearform $a(\blank, \blank; t)$ respektive der zugehörige stetige lineare Operator $A(t)$ unabhängig von $t \in [0, T]$, dann sprechen wir von einer \emph{autonomen} linearen Evolutionsgleichung.
\end{thmenumerate}
\end{Bemerkung}

Als nächstes leiten wir eine Raum-Zeit-Variationsformulierung für~\cref{eq:gl:le:lineare_evolutionsgleichung} her.
Dazu werden geeignete Ansatz- und Testfunktionenräume benötigt.
Hier kommen nun die in \cref{sec:gl:br:bochner_raeume} definierten Bochner-Räume zum Einsatz.

\begin{Definition}
\label{def:gl:le:ansatz_und_testraum}
    Als Ansatzfunktionenraum $\mathcal X$ bezeichnen wir den Raum $W(0, T; V, V')$ aus \cref{def:gl:br:bochner_raum_W}.
    Es ist also
    \begin{equation}
        \label{eq:gl:le:ansatzraum_X}
        \begin{aligned}
            \mathcal X &= L_{2}(0, T; V) \cap H^{1}(0, T; V')
            \\&= \Set*{ u \in L_{2}(0, T; V) \given u_{t} \in L_{2}(0, T; V') }
        \end{aligned}
    \end{equation}
    ausgestattet mit der Norm
    \begin{equation}
        \label{eq:gl:le:ansatzraum_X_norm}
        \norm{u}_{\mathcal X} = \left( \norm{u}_{L_{2}{(0, T; V)}}^{2} + \norm{u_{t}}_{L_{2}{(0, T; V')}}^{2} \right)^{1 / 2}, \quad u \in \mathcal X.
    \end{equation}
    Der Testfunktionenraum $\mathcal Y$ sei gegeben durch
    \begin{equation}
        \label{eq:gl:le:testraum_Y}
        \mathcal Y = L_{2}(0, T; V) \oplus H
    \end{equation}
    mit der Norm
    \begin{equation}
        \label{eq:gl:le:testraum_Y_norm}
        \norm{v}_{\mathcal Y} = \left( \norm{v_{1}}_{L_{2}(0, T; V)}^{2} + \norm{v_{2}}_{H}^{2} \right)^{1 / 2}, \quad v = (v_{1}, v_{2}) \in \mathcal Y.
    \end{equation}
\end{Definition}

Beide Räume sind Hilberträume, $\mathcal X$ nach \cref{lem:gl:br:bochner_W_ist_hilbertraum} und $\mathcal Y$ als direkte Summe zweier Hilberträume.

Um aus~\cref{eq:gl:le:lineare_evolutionsgleichung} eine Variationsformulierung zu erhalten, multiplizieren wir die lineare Evolutionsgleichung mit $v = (v_{1}, v_{2}) \in \mathcal Y$ und integrieren anschließend über das Zeitintervall $[0, T]$.
Dadurch ergibt sich folgende schwache Formulierung der linearen Evolutionsgleichung aus \cref{def:gl:le:lineare_evolutionsgleichung}:

\begin{Definition}
\label{def:gl:le:schwache_raum_zeit_formulierung}
    Seien $\mathcal X$ und $\mathcal Y$ wie in~\cref{eq:gl:le:ansatzraum_X} respektive~\cref{eq:gl:le:testraum_Y}.
    Als \emph{Raum-Zeit-Variations"-for"-mu"-lie"-rung} der linearen Evolutionsgleichung~\cref{eq:gl:le:lineare_evolutionsgleichung} bezeichnen wir das folgende Problem:

    Gegeben ein Quellterm $g \in L_{2}(0, T; V')$ und ein Anfangswert $u_{0} \in H$.
    Finde ein $u \in \mathcal X$ mit
    \begin{equation}
        \label{eq:gl:le:schwache_formulierung}
        b(u, v) = f(v) \quad \fa v \in \mathcal Y.
    \end{equation}
    Dabei ist $b \colon \mathcal X \times \mathcal Y \to \mathbb{R}$ die durch
    \begin{equation}
        \label{eq:gl:le:schwache_formulierung_lhs_b}
        b(u, v) = \int_{0}^{T} \skprod{u_{t}(t)}{v_{1}(t)}_{H} + a(u(t), v_{1}(t); t) \diff t + \skprod{u(0)}{v_{2}}_{H}
    \end{equation}
    definierte Bilinearform und $f \colon \mathcal Y \to \mathbb{R}$ das durch
    \begin{equation}
        \label{eq:gl:le:schwache_formulierung_rhs_f}
        f(v) = \int_{0}^{T} \skprod{g(t)}{v_{1}(t)}_{H} \diff t + \skprod{u_{0}}{v_{2}}_{H}
    \end{equation}
    gegebene Funktional.
\end{Definition}

Es bleibt nun zu zeigen, welche Bedingungen hinreichend sind, damit obige Raum-Zeit-Variationsformulierung \emph{sachgemäß gestellt} ist.
Dazu definieren wir zunächst, was wir unter dem Begriff \enquote{sachgemäß gestellt} verstehen wollen und greifen dazu auf die Definition nach \textcite{hadamard1902problemes} zurück.

\begin{Definition}
\label{def:gl:le:hadamard_sachgemaess_gestellt}
    Seien $U$ und $V$ normierte Vektorräume.
    Seien weiter eine stetige Bilinearform $a \in \mathcal L(U \times V, \mathbb{R})$ und ein stetiges Funktional $f \in \mathcal L(V)$ gegeben.
    Betrachte das folgende Problem: finde ein $u \in U$, so dass
    \begin{equation}
    \label{eq:gl:le:hadamard_variationsproblem}
        a(u, v) = f(v) \quad \fa v \in V
    \end{equation}
    gilt.
    Wir nennen \cref{eq:gl:le:hadamard_variationsproblem} \emph{sachgemäß gestellt}, wenn eine eindeutige Lösung $u \in U$ existiert und diese eine A-priori-Abschätzung der Form
    \begin{equation}
    \label{eq:gl:le:hadamard_abschaetzung}
        \norm{u}_{U} \leq c \norm{f}_{V'} \quad \fa f \in V'
    \end{equation}
    und eine von $f$ unabhängige Konstante $c > 0$ erfüllt.
\end{Definition}

Um dies für die Raum-Zeit-Variationsformulierung aus \cref{def:gl:le:schwache_raum_zeit_formulierung} nachzuweisen, werden wir den folgenden bekannten und wichtigen Satz, welcher in dieser oder ähnlicher Form bei \textcites[Theorem 2.1]{Babuska:1971fx}[Theorem 5.2.1]{Aziz:2014wf}[Theorem \S{}3.3.6]{Braess:2007wm} zu finden ist, verwenden.

\begin{Satz}[Banach-Ne{\v c}as-Babu{\v s}ka-Theorem]
\label{satz:gl:le:bnb_theorem}
    Seien $U$ und $V$ Hilberträume.
    Eine lineare Abbildung $A \colon U \to V'$ ist genau dann ein Isomorphismus, das heißt stetig invertierbar, wenn die zugehörige Bilinearform $a \colon U \times V \to \mathbb{R}$ die folgenden Bedingungen erfüllt:
    \begin{thmenumerate}
        \item \label{satz:gl:le:bnb_theorem_bedingung_stetig}
        \emph{Stetigkeit}.
        Es existiert ein $0 < C < \infty$ mit
        \begin{equation}
            \abs{a(u, v)} \leq C \norm{u}_{U} \norm{v}_{V} \quad \fa u \in U,~v\in V.
        \end{equation}
        \item \label{satz:gl:le:bnb_theorem_bedingung_inf_sup}
        \emph{Inf-sup-Bedingung}.
        Es existiert ein $\alpha > 0$ mit
        \begin{equation}
            \inf_{u \in U} \sup_{v \in V} \frac{a(u, v)}{\norm{u}_{U} \norm{v}_{V}} \geq \alpha.
        \end{equation}
        \item \label{satz:gl:le:bnb_theorem_bedingung_surjektiv}
        Zu jedem $v \in V$, $v \neq 0$, existiert ein $u \in U$ mit
        \begin{equation}
            a(u, v) \neq 0.
        \end{equation}
    \end{thmenumerate}
    Ist dies der Fall und ist weiter ein Funktional $f \in V'$ gegeben, dann existiert eine eindeutige Lösung $\hat u \in U$ mit
    \begin{equation}
        a(\hat u, v) = f(v) \quad \fa v \in V
    \end{equation}
    und es gilt
    \begin{equation}
        \norm{\hat u}_{U} \leq \frac{1}{\alpha} \norm{f}_{V'}.
    \end{equation}
\end{Satz}

\begin{Bemerkung}
\label{bem:gl:le:bnb_staerkere_voraussetzungen}
    Die letzte Bedingung im vorherigen Satz kann durch eine stärkere inf-sup-Bedingung ersetzt werden, denn existiert ein $\beta > 0$ mit
    \begin{equation}
        \inf_{v \in V} \sup_{u \in U} \frac{a(u, v)}{\norm{u}_{U} \norm{v}_{V}} \geq \beta,
    \end{equation}
    dann gilt insbesondere auch \ref{satz:gl:le:bnb_theorem_bedingung_surjektiv}.
\end{Bemerkung}

Aus dem Banach-Ne{\v c}as-Babu{\v s}ka-Theorem lässt sich auch das bekannte Lax-Milgram-Lemma ableiten, weswegen wir es hier kurz wiederholen wollen.

\begin{Lemma}[Lax-Milgram]
\label{lem:gl:le:lax_milgram}
    Sei $V$ ein Hilbertraum, seien weiter $a \in \mathcal L(V \times V, \mathbb{R})$ eine stetige Bilinearform und $f \in \mathcal L (V, \mathbb{R})$ ein stetiges Funktional und betrachte das Problem:
    finde ein $u \in V$, so dass
    \begin{equation}
        a(u, v) = f(v) \quad \fa v \in V
    \end{equation}
    gilt.
    Ist die Bilinearform $a$ elliptisch, das heißt, es existiert ein $\alpha > 0$ mit
    \begin{equation}
    \label{eq:gl:le:lax_milgram_elliptisch}
        a(u, u) \geq \alpha \norm{u}_{V}^{2} \quad \fa u \in V,
    \end{equation}
    dann ist das Problem sachgemäß gestellt und es gilt
    \begin{equation}
        \norm{u}_{V} \leq \frac{1}{\alpha} \norm{f}_{V'}.
    \end{equation}

    \begin{Beweis}
        Aus der Elliptizität können wir eine inf-sup-Bedingung folgern, denn es gilt
        \begin{equation}
            \sup_{v \in V} \frac{a(u, v)}{\norm{v}_{V}} \geq \frac{a(u, u)}{\norm{u}_{V}} \geq \alpha \norm{u}_{V}.
        \end{equation}
        Damit sind neben der Stetigkeit von $a$ auch die die Bedingungen \ref{satz:gl:le:bnb_theorem_bedingung_inf_sup} und \ref{satz:gl:le:bnb_theorem_bedingung_surjektiv} aus dem Banach-Ne{\v c}as-Babu{\v s}ka-Theorem, \cref{satz:gl:le:bnb_theorem}, erfüllt und es folgt die Aussage.
    \end{Beweis}
\end{Lemma}

Für das Raum-Zeit-Variationsproblem aus \cref{def:gl:le:schwache_raum_zeit_formulierung} lässt sich die Tatsache, dass es sachgemäß gestellt ist, zu folgendem Satz zusammenfassen.

\begin{Satz}[{{\cite[Theorem 5.1]{Schwab:2009ec}}}]
\label{satz:gl:le:ss09_theorem51}
    Seien $\mathcal X$ und $\mathcal Y$ wie in~\cref{eq:gl:le:ansatzraum_X} respektive~\cref{eq:gl:le:testraum_Y}.
    Sei weiter $B \colon \mathcal X \to \mathcal Y'$ definiert durch
    \begin{equation}
        \label{eq:gl:le:ss09_variationsproblem_als_operatorgleichung}
        \skprod{B u}{v}_{\mathcal Y' \times \mathcal Y} = b(u, v), \quad u \in \mathcal X,~v \in \mathcal Y,
    \end{equation}
    wobei $b$ die Bilinearform aus~\cref{eq:gl:le:schwache_formulierung_lhs_b} sei.
    Dann ist $B$ stetig invertierbar.

    \begin{Beweis}
        Ein ausführlicher Beweis, in dem die Bedingungen des Banach-Ne\v{c}as-Babu\v{s}ka-Theorems nachgewiesen werden, ist bei \textcite[Appendix A]{Schwab:2009ec} zu finden.
        % \begin{Beweis}
        %     Wir weisen die Bedingungen von \cref{satz:babuska-aziz} nach.

        %     Zunächst sei anzumerken, dass wir in~\cref{eq:garding-inequality} ohne Einschränkung $\lambda = 0$ wählen können.
        %     Wähle
        %     \begin{equation}
        %         u(t) = \hat u(t) e^{\lambda t}, \quad v_{1}(t) = \hat v_{1}(t) e^{- \lambda t}, \quad g(t) = \hat g(t) e^{\lambda t},
        %     \end{equation}
        %     dann sieht man, dass $u$ die Gleichung~\cref{eq:bilinearform} genau dann löst, wenn $\hat u$ die Gleichung
        %     \begin{equation}
        %         \label{eq:bilinearform_tmp}
        %         \begin{gathered}
        %             \int_{I} \skprod{\hat{u}_{t}(t)}{\hat{v}_{1}(t)}_{H} + \lambda \skprod{\hat{u}(t)}{\hat{v}_{1}(t)}_{H} + a(t; \hat{u}(t), \hat{v}_{1}(t)) \diff t + \skprod{\hat{u}(0)}{v_{2}}_{H}
        %                 \\= \int_{I} \skprod{\hat{g}(t)}{\hat{v}_{1}(t)}_{H} \diff t + \skprod{u_{0}}{v_{2}}_{H}
        %         \end{gathered}
        %     \end{equation}
        %     für alle $\hat{v} = (\hat{v}_{1}, v) \in \mathcal Y$ löst.

        %     \paragraph{Stetigkeit} % (fold)
        %     \label{par:stetigkeit}
        %     Betrachte für $u \in \mathcal X$ und $v = (v_{1}, v_{2}) \in \mathcal Y$ die Bilinearform $b(u, v)$.
        %     Nach Anwenden der Dreiecksungleichung erhalten wir
        %     \begin{equation}
        %         \label{eq:stetigkeit_zweiter_term}
        %         \abs{b(u, v)} = \int_{I} \abs{\skprod{u_{t}(t)}{v_{1}(t)}_{H}} + \abs{a(u(t), v_{1}(t))} \diff t + \abs{\skprod{u(0)}{v_{2}}_{H}}.
        %     \end{equation}
        %     Betrachten wir zunächst den hinteren Term, dann erhalten wir unter Verwendung der Cauchy-Schwarz-Ungleichung und der Einbettungs-Konstante $M_{e}$ die Abschätzung
        %     \begin{equation}
        %         \abs{\skprod{u(0)}{v_{2}}_{H}} \leq \norm{u(0)}_{H} \norm{v_{2}}_{H} \leq M_{e} \norm{u}_{X} \norm{v_{2}}_{H}.
        %     \end{equation}
        %     Widmen wir uns nun dem ersten Term und wenden ebenfalls die Cauchy-Schwarz-Ungleichung sowie die Stetigkeit von $a$ an, dann erhalten wir
        %     \begin{align}
        %         &\int_{I} \abs{\skprod{u_{t}(t)}{v_{1}(t)}_{H}} + \abs{a(u(t), v_{1}(t))} \diff t
        %         \\&\qquad
        %         \leq \int_{I} \norm{u_{t}(t)}_{H} \norm{v_{1}(t)}_{H} + M_{a} \norm{u(t)}_{H} \norm{v_{1}(t)}_{H} \diff t
        %         \\&\qquad
        %         \leq \int_{I} \max\{1, M_{a}\} \norm{v_{1}(t)}_{H} \left(  \norm{u_{t}(t)}_{H} + \norm{u(t)}_{H} \right) \diff t
        %         \intertext{mittels Hölder-Ungleichung lässt sich dies weiter abschätzen zu}
        %         &\qquad
        %         \leq \left( \int_{I} \max\{1, M_{a}\}^{2} \norm{v_{1}(t)}_{H}^{2} \diff t \right)^{\frac 12} \left( \int_{I} \left( \norm{u_{t}(t)}_{H} + \norm{u(t)}_{H} \right)^{2} \diff t \right)^{\frac 12},
        %         \intertext{und unter Verwendung der Youngschen-Ungleichung zu}
        %         &\qquad
        %         \leq \left( \int_{I} \max\{1, M_{a}\}^{2} \norm{v_{1}(t)}_{H}^{2} \diff t \right)^{\frac 12} \left( \int_{I} 2 \left( \norm{u_{t}(t)}_{H}^{2} + \norm{u(t)}_{H}^{2} \right) \diff t \right)^{\frac 12}
        %         \intertext{was nach Definition der verwendeten Normen auch geschrieben werden kann als}
        %         &\qquad
        %         = \sqrt{2 \max\{1, M_{a}^{2}\}} \norm{u}_{\mathcal X} \norm{v_{1}}_{L_{2}(I; V)}
        %     \end{align}
        %     Zusammen mit~\cref{eq:stetigkeit_zweiter_term} liefert dies nach einer erneuten Anwendung der Cauchy-Schwarz-Ungleichung
        %     \begin{align}
        %     \abs{b(u, v)}
        %     &\leq \sqrt{2 \max\{1, M_{a}\}^{2}} \norm{u}_{\mathcal X} \norm{v_{1}}_{L_{2}(I; V)} + M_{e} \norm{u}_{X} \norm{v_{2}}_{H}
        %     \\
        %     &\leq \norm{u}_{\mathcal X} \left( \norm{v_{1}}_{L_{2}(I; V)}^{2} + \norm{v_{2}}_{H}^{2} \right)^{\frac 12} \left( 2 \max\{1, M_{a}\}^{2} + M_{e}^{2} \right)^{\frac 12}
        %     \\
        %     &= \sqrt{2 \max\{1, M_{a}^{2}\} + M_{e}^{2}} \norm{u}_{\mathcal X} \norm{v}_{\mathcal Y}.
        %     \end{align}
        %     Damit folgt die Stetigkeit.
        %     % paragraph stetigkeit (end)

        %     \paragraph{Inf-Sup-Bedingung} % (fold)
        %     \label{par:inf_sup_bedingung}

        %     % paragraph inf_sup_bedingung (end)
    \end{Beweis}
\end{Satz}

Aus dem Beweis des vorherigen Satzes bei \textcite{Schwab:2009ec} ergeben sich zugleich auch Abschätzungen für die Operatornormen von $B$ und $B^{-1}$.

\mdo{Ungleichung der stetigen Abhängigkeit einbauen. Wat? Was wollte ich mir damit sagen? ...}

\begin{Korollar}
\label{kor:gl:le:ss09_theorem51_ungleichungen}
    Unter der Gegebenheiten aus \cref{satz:gl:le:ss09_theorem51} und der Bedingung, dass die Bilinearform $a$ die G\aa{}rding-Ungleichung mit $\lambda = 0$ erfüllt, gilt
    \begin{equation}
        \norm{B}_{\mathcal L(\mathcal X, \mathcal Y')} \leq \sqrt{2 \max\Set{1, M_{a}^{2}} + M_{e}^{2}}
    \end{equation}
    und
    \begin{equation}
        \norm{B^{-1}}_{\mathcal L(\mathcal Y', \mathcal X)} \leq \frac{\sqrt{2 \max \Set{\alpha^{-2}, 1} + M_{e}^{2}}}{\min\Set{\alpha M_{a}^{-2}, \alpha}}.
    \end{equation}

    Ist $\lambda \neq 0$, dann gilt ferner
    \begin{equation}
        \label{kor:gl:le:norm_B_abschaetzung}
        \norm{B}_{\mathcal L(\mathcal X, \mathcal Y')} \leq \frac{\sqrt{2\max\Set{1, M_{a}^{2}} + M_{e}^{2}}}{\max\Set{\sqrt{1 + 2 \lambda^{2} \rho^{4}}, \sqrt{2}}}
    \end{equation}
    und
    \begin{equation}
        \label{kor:gl:le:norm_B_inv_abschaetzung}
        \norm{B^{-1}}_{\mathcal L(\mathcal Y', \mathcal X)} \leq e^{2 \lambda T} \max\Set{\sqrt{1 + 2 \lambda^{2} \rho^{4}}, \sqrt{2}}  \frac{\sqrt{2 \max\Set{ \alpha^{-2}, 1} + M_{e}^{2}}}{\min\Set{\alpha M_{a}^{-2}, \alpha}}.
    \end{equation}

    Die Größen $M_{a}$, $\alpha$ und $\lambda$ stammen aus \cref{ann:gl:le:bilinearform_eigenschaften},
    während die Konstanten $M_{e}$ und $\rho$ als die Einbettungskonstanten

    \begin{equation}
        \label{kor:gl:le:einbettungskonstante_M_e}
        M_{e} = \sup_{0 \neq u \in \mathcal X} \frac{\norm{u(0)}_{H}}{\norm{u}_{\mathcal X}},
    \end{equation}
    vergleiche \cref{kor:gl:br:einbettungskonstante_M_e}, beziehungsweise
    \begin{equation}
        \label{kor:gl:le:einbettungskonstante_rho}
        \rho = \sup_{0 \neq \eta \in V} \frac{\norm{\eta}_{H}}{\norm{\eta}_{V}}
    \end{equation}

    \begin{Beweis}
        $\,$\newline
        \mwarn{Will ich das beweisen?}
    \end{Beweis}
\end{Korollar}

% section lineare_evolutionsgleichungen (end)

% \section{Homogenisierung} % (fold)
% \label{sec:homogenisierung}

% \todo[inline]{Ist das nützlich?}
% \todo[inline]{Quelle!}
% Die im vorherigen Abschnitt hergeleitete inhomogene Raum-Zeit-Variationsformulierung lässt sich zu einer äquivalenten homogenen Raum-Zeit-Variationsformulierung umformen.

% Dazu definieren wir erneut zunächst die benötigten Ansatz- und Testfunktionenräume und anschließend das darauf aufbauende homogene Raum-Zeit-Variationsproblem.

% \begin{Definition}
% \label{def:gl:le:ansatz_und_testraum_homogen}
%     Der Ansatzfunktionenraum $\hat{\mathcal X}$ sei als der folgende Unterraum von $W(0, T; V, V')$
%     \begin{equation}
%         \label{eq:gl:le:ansatzraum_X_homogen}
%         \begin{aligned}
%             \hat{\mathcal X} &= L_{2}(0, T; V) \cap H^{1}_{\{0\}}(0, T; V')
%             \\&= \Set*{ u \in L_{2}(0, T; V) \given u_{t} \in L_{2}(0, T; V'),~u(0) = 0 }
%         \end{aligned}
%     \end{equation}
%     definiert, ausgestattet mit der bekannten Norm
%     \begin{equation}
%         \label{eq:gl:le:ansatzraum_X_norm_homogen}
%         \norm{u}_{\hat{\mathcal X}} = \left( \norm{u}_{L_{2}{(0, T; V)}}^{2} + \norm{u_{t}}_{L_{2}{(0, T; V')}}^{2} \right)^{1 / 2}, \quad u \in \hat{\mathcal X}.
%     \end{equation}
%     Der zugehörige Testfunktionenraum $\hat{\mathcal Y}$ sei gegeben durch
%     \begin{equation}
%         \label{eq:gl:le:testraum_Y_homogen}
%         \hat{\mathcal Y} = L_{2}(0, T; V)
%     \end{equation}
%     mit der üblichen Norm
%     \begin{equation}
%         \label{eq:gl:le:testraum_Y_norm_homogen}
%         \norm{v}_{\hat{\mathcal Y}} = \norm{v_{1}}_{L_{2}(0, T; V)}, \quad v \in \hat{\mathcal Y}.
%     \end{equation}
% \end{Definition}

% \begin{Definition}
% \label{def:gl:le:schwache_raum_zeit_formulierung_homogen}
%     Seien $\hat{\mathcal X}$ und $\hat{\mathcal Y}$ wie in~\cref{eq:gl:le:ansatzraum_X_homogen} respektive~\cref{eq:gl:le:testraum_Y_homogen}.
%     Als \emph{homogenisierte Raum-Zeit-Variations"-for"-mu"-lie"-rung} der linearen Evolutionsgleichung~\cref{eq:gl:le:lineare_evolutionsgleichung} bezeichnen wir das folgende Problem:

%     Gegeben ein Quellterm $g \in L_{2}(0, T; V')$ und ein Anfangswert $u_{0} \in H$, sodass ein $\hat u_{0} \in \mathcal X$ existiert, welches $\hat{u}_{0}(0) = u_{0}$ erfüllt.
%     Finde ein $u \in \hat{\mathcal X}$ mit
%     \begin{equation}
%         \label{eq:gl:le:schwache_formulierung_homogen}
%         \hat{b}(u, v) = \hat{f}(v) \quad \fa v \in \hat{\mathcal Y}.
%     \end{equation}
%     Dabei ist $\hat{b} \colon \hat{\mathcal X} \times \hat{\mathcal Y} \to \mathbb{R}$ die durch
%     \begin{equation}
%         \label{eq:gl:le:schwache_formulierung_lhs_b_homogen}
%         \hat{b}(u, v) = \int_{0}^{T} \skp{u_{t}(t)}{v(t)}{V' \times V} + a(u(t), v(t); t) \diff t
%     \end{equation}
%     definierte Bilinearform und $\hat{f} \colon \hat{\mathcal Y} \to \mathbb{R}$ das durch
%     \begin{equation}
%         \label{eq:gl:le:schwache_formulierung_rhs_f_homogen}
%         \hat{f}(v) = \int_{0}^{T} \skp{\hat g(t)}{v(t)}{V' \times V} \diff t
%     \end{equation}
%     gegebene Funktional,
%     wobei $\hat{g} \in $ als
%     \begin{equation}
%         \hat{g} = g(t) - \left[ (\hat u_{0})_{t}(t) + A(t) \hat u_{0}(t) \right]
%     \end{equation}
%     definiert sei.
% \end{Definition}

% Das es sich hierbei tatsächlich um eine Homogenisierung handelt, liefert der folgende Satz.

% \begin{Satz}
%     Sei $H = L_{2}(\Omega)$ für ein beschränktes Gebiet $\Omega \in \mathbb{R}^{n}$.
%     Dann gilt: $\hat u \in \tilde{\mathcal X}$ ist genau dann eine Lösung des homogenisierten RZVP aus \cref{def:gl:le:schwache_raum_zeit_formulierung_homogen}, wenn $u := \hat u + \hat u_{0} \in \mathcal X$ eine Lösung  des RZVP aus \cref{def:gl:le:schwache_raum_zeit_formulierung} ist.

%     \todo[inline]{Beweis}
%     % \begin{Beweis}
%     %     \emph{Hinrichtung.} Sei $\hat u \in \tilde{\mathcal X}$ eine Lösung, dann gilt
%     %     \begin{equation}
%     %         u(0) = \hat u(0) + \hat u_{0}(0) = 0 + u_{0} = u_{0}.
%     %     \end{equation}
%     %     Seien nun $v = (v_{1}, v_{2}) \in \mathcal Y$ beliebig.
%     %     Dann gilt
%     %     \begin{align}
%     %         b(u, v)
%     %           &= b(u, (v_{1}, v_{2}))
%     %         \\&= \hat b(u, v_{1}) + \skp{u(0)}{v_{2}}{H}
%     %         \\&= \hat b(\hat u, v_{1}) + \hat b(\hat u_{0}, v_{1}) + \skp{u(0)}{v_{2}}{H}
%     %         \\&= \hat f(v_{1}) + \int_{0}^{T} \skp{\hat u_{0})_{t}(t) + A(t) \hat u_{0}(t)}{V' \times V} + \skp{u(0)}{v_{2}}{H}
%     %         \\&= \int_{0}^{T} \skp{\hat g(t) + \hat u_{0})_{t}(t) + A(t) \hat u_{0}(t)}{v_{1}(t)}{V' \times V} + \skp{u(0)}{v_{2}}{H}
%     %         \\&= \int_{0}^{T} \skp{g(t)}{v_{1}(t)}{V' \times V} + \skp{u(0)}{v_{2}}{H}
%     %         \\&= f(v).
%     %     \end{align}

%     %     \emph{Rückrichtung.} Sei $u \in \mathcal X$ eine Lösung.
%     %     Dann gilt
%     %     \begin{equation}
%     %             \hat u(0) = u(0) - \hat u_{0}(0) = u_{0} - u_{0} = 0.
%     %     \end{equation}
%     %     \todo[inline]{Bla?}
%     %     Damit ist insbesondere $\hat u \in \tilde{\mathcal X}$.
%     %     Sei nun $v_{1} \in \tilde{\mathcal Y}$ beliebig und $v_{2} \in H$, dann gilt
%     %     \begin{align}
%     %         \hat b(u, v_{1})
%     %           &= \hat b(u - \hat u_{0}, v_{1})
%     %         \\&= (b(u, (v_{1}, v_{2})) - \skp{u(0)}{v_{2}}{H}) - (b(\hat u_{0}, (v_{1}, v_{2})) - \skp{\hat u_{0}(0)}{v_{2}}{H})
%     %         \\&= f(v_{1}, v_{2}) - b(\hat u_{0}, (v_{1}, v_{2}) - \skp{u(0) - \hat u_{0}(0)}{v_{2}}{H}
%     %         \\&= f(v_{1}, v_{2}) - b(\hat u_{0}, (v_{1}, v_{2}) - \skp{u(0)
%     %         \\&= \int_{0}^{T} \skp{g(t)}{v_{1}(t)}{V' \times V} \diff t + \skp{u_{0}}{v_{2}}{H}
%     %                 - \int_{0}^{T} \skp{(\hat u_{0})_{t}(t)}{v_{1}(t)}{V' \times V} + a(\hat u_{0}(t), v_{1}(t); t) \diff t - \skp{u_{0}}{v_{2}}_{H}
%     %         \\&= \hat f(v_{1}).
%     %     \end{align}
%     %     Damit ist die Äquivalenz gezeigt.
%     % \end{Beweis}
% \end{Satz}

% \begin{Bemerkung}
%     Ist $u_{0} \in V$ gegeben, dann erhalten wir $\hat{u}_{0}$ via $\hat{u}_{0} = 1 \otimes u_{0}$, denn dann gilt
%     $\hat{u}_{0}(0) = u_{0}$ und $\hat{u}_{0} \in \mathcal X$.
% \end{Bemerkung}

% section homogenisierung (end)

\section{Galerkin-Verfahren} % (fold)
\label{sec:gl:gv:galerkin_verfahren}

\mdo{Abschnitt sauber aufziehen, d.h. Formulierungen verbessern, eventuell ausführlicher.}

Wir wiederholen nun in Grundzügen das bekannte Galerkin-Verfahren, konzentrieren uns dabei aber auf das sogenannte \emph{Petrov-Galerkin-Verfahren}, welches eine Verallgemeinerung auf nicht-symmetrische Bilinearformen innerhalb der Variationsformulierung darstellt.

Seien $U$ und $V$ zwei Hilberträume, $a \colon U \times V \to \mathbb{R}$ eine Bilinearform und $f \colon V \to \mathbb{R}$ ein lineares Funktional.
Wir betrachten das folgende abstrakte Variationsproblem:
Finde ein $u \in U$, so dass
\begin{equation}
    \label{eq:gl:gv:variationsproblem}
    a(u, v) = f(v) \quad \fa v \in V
\end{equation}
gilt.

Unter gewissen Voraussetzungen, zum Beispiel, wenn die Bilinearform $a$ die Bedingungen des Banach-Ne{\v c}as-Babu{\v s}ka-Theorems (\cref{satz:gl:le:bnb_theorem}) erfüllt, existiert eine eindeutige Lösung $u \in U$ des obigen Variationsproblems.

Bei den Räumen $U$ und $V$ handelt es sich im Allgemeinen um unendlichdimensionale Hilberträume, wodurch eine numerische Bestimmung der Lösung $u \in U$ nicht ohne Weiteres möglich ist.
Um eine approximative Lösung von \cref{eq:gl:gv:variationsproblem} zu bestimmen, muss zunächst eine Diskretisierung durchgeführt werden.
Diese wird bei Petrov-Galerkin-Verfahren dadurch erreicht, dass für Ansatz- und Testfunktionenraum endlichdimensionale Unterräume $U_{n} \subset U$ mit $\dim U_{n} = n$ respektive $V_{m} \subset V$ mit $\dim V_{m} = m$ konstruiert werden.

Statt dem obigen Variationsproblem \cref{eq:gl:gv:variationsproblem} betrachtet man dann:
Finde ein $u_{n} \in U_{n}$, so dass
\begin{equation}
    \label{eq:gl:gv:variationsproblem_diskret}
    a_{n,m}(u_{n}, v_{m}) = f_{m}(v_{m}) \quad \fa v_{m} \in V_{m}
\end{equation}
gilt, wobei $a_{n, m} \colon U_{n} \times V_{m} \to \mathbb{R}$ die Einschränkung $a_{n,m} = \restr{a}{U_{n} \times V_{m}}$ beziehungsweise $f_{m} \colon V_{m} \to \mathbb{R}$ die Einschränkung $f_{m} = \restr{f}{V_{m}}$ bezeichne.

Zu Beachten ist, dass selbst wenn das eigentliche Variationsproblem \cref{eq:gl:gv:variationsproblem} eindeutig lösbar ist, dies noch nicht für das diskretisierte Variationsproblem \cref{eq:gl:gv:variationsproblem_diskret} gelten muss.
Auch muss im Falle einer existierenden Lösung $u_{n} \in U_{n}$ diese im Allgemeinen für $n \to \infty$ nicht gegen die Lösung $u \in U$ von \cref{eq:gl:gv:variationsproblem} konvergieren.

\mdo{Quellen prüfen}
Der folgende Satz, welcher in verschiedenen Varianten bei \cites[Theorem 6.2.1]{Aziz:2014wf}[Theorem 5.3.1]{Quarteroni:2009hp}[Lemma III.3.7]{Braess:2007wm} zu finden ist, liefert hinreichende Bedingungen für die eindeutige Lösbarkeit von \cref{eq:gl:gv:variationsproblem_diskret} und zudem eine A priori-Fehlerabschätzung.

\begin{Satz}
    Seien $U$ und $V$ zwei Hilberträume und $U_{n} \subset U$, respektive $V_{m} \subset V$ Unterräume.
    Weiter seien $a \colon U \times V \to \mathbb{R}$ eine Bilinearform und $f \colon V \to \mathbb{R}$ ein lineares Funktional, wobei $a$ die Bedingungen aus \cref{satz:gl:le:bnb_theorem} erfülle.
    Bezeichne ferner mit $a_{n, m} = \restr{a}{U_{n} \times V_{m}}$ respektive $f_{m} = \restr{f}{V_{m}}$ die Einschränkungen von $a$ beziehungsweise $f$ auf die Unterräume $U_{n}$ und $V_{m}$.

    Erfüllt $a_{n, m}$ die Bedingungen \ref{satz:gl:le:bnb_theorem_bedingung_inf_sup} und \ref{satz:gl:le:bnb_theorem_bedingung_surjektiv} aus \cref{satz:gl:le:bnb_theorem} mit der inf-sup-Konstante $\alpha_{n, m} > 0$ statt $\alpha$, dann existiert eine eindeutige Lösung $u_{n} \in U_{n}$ des Variationsproblems \cref{eq:gl:gv:variationsproblem_diskret} und diese erfüllt
    \begin{equation}
        \norm{u_{n}}_{U} \leq \frac{1}{\alpha_{n, m}} \norm{f_{m}}_{V_m'}.
    \end{equation}

    Ist weiter $u \in U$ eine Lösung des Variationsproblems \cref{eq:gl:gv:variationsproblem}, dann gilt
    \begin{equation}
        \norm{u - u_{n}} \leq \left( 1 + \frac{C}{\alpha_{n, m}} \right)  \inf_{w_{n} \in U_{n}} \norm{u - w_{n}}.
    \end{equation}

    \begin{Beweis}$\;$
        \mfix{Nötig?}
    \end{Beweis}
\end{Satz}

Um nun tatsächlich eine Lösung $u_{n} \in U_{n}$ zu bestimmen, verfährt man wie folgt:
Seien $\Set{ \varphi_{i} \in U \given i = 1 \dots n }$ und $\Set{ \psi_{j} \in V \given j = 1 \dots m}$ jeweils eine Basis von $U_{n}$ respektive $V_{m}$.

Schreibt man nun die Ansatz- beziehungsweise Testfunktionen als Linearkombination
\begin{equation}
    u_{n} = \sum_{i = 1}^{n} \vec{u}_{i} \varphi_{i}, \quad v_{m} = \sum_{j = 1}^{m} \vec{v}_{j} \psi_{j}, \qquad \vec u \in \mathbb{R}^{n},~\vec v \in \mathbb{R}^{m},
\end{equation}
dieser Basisfunktionen und setzt diese in das Variationsproblem \cref{eq:gl:gv:variationsproblem_diskret} ein, dann führt dies unmittelbar zu dem linearen Gleichungssystem $\mat{A} \vec{u} = \vec{f}$, wobei wir
\begin{equation}
    \mat A_{ji} = a_{n, m}(\varphi_{j}, \psi_{i}), \quad i = 1 \dots n,~j = 1 \dots m,
\end{equation}
als \emph{Steifigkeitsmatrix} $\mat A \in \mathbb{R}^{m \times n}$, und
\begin{equation}
    \vec f_{j} = f_{m}(\psi_{j}), \quad j = 1 \dots m,
\end{equation}
als \emph{Lastvektor} $\vec f \in \mathbb{R}^{m}$ bezeichnen.
Besitzt das lineare Gleichungssystems eine Lösung $\vec u \in \mathbb{R}^{n}$, dann erhält man die zugehörige Lösung $u_{n} \in U_{n}$ des obigen Variationsproblems \cref{eq:gl:gv:variationsproblem_diskret} durch die Linearkombination
\begin{equation}
    u_{n} = \sum_{i = 1}^{n} \vec{u}_{i} \varphi_{i}.
\end{equation}

% chapter grundlagen (end)



    %!TEX root = ../main.tex

\chapter{Parametrische Problem - Neuer Versuch} % (fold)
\label{cha:parametrische_problem_neuer_versuch}

In diesem Kapitel führen wir nun die konkrete PPDE, welche ihren Ursprung in der Polymerchemie hat, ein, wandeln sie anschließend in eine parametrische PPDE um und weisen dann Regularität bezüglich der Parameter nach.

\section{Motivation} % (fold)
\label{sec:einf_hrung_der_ppde}

Dieser Abschnitt führt die PPDE ein und leitet eine parametrische Variante dieser her.
Eine umfangreiche Beschreibung der physikalischen und chemischen Hintergründe, sowie eine ausführliche Herleitung der darauf aufbauenden mathematischen Modellierung findet sich bei \textcite{Fredrickson:2006th}.

Wir beschränken uns auf die daraus resultierende parabolische partielle Differentialgleichung, da diese den Mittelpunkt dieser Arbeit bildet.

Seien dazu $0 < T < \infty$ und $I = [0, T]$ ein endliches Zeitintervall und weiter $\Omega \subset \mathbb{R}^{n}$ eine offene, beschränkte Teilmenge mit Lipschitz-Rand.
In den tatsächlich auftretenden Fällen wird meist $T = 1$ und $\Omega = [0, L]^n$ für ein $0 < L < \infty$ und $n \in \Set{1, 2, 3}$ gelten, wir wollen dies aber zunächst ignorieren, da die folgenden Aussagen auch für den allgemeinen Fall gelten.

Gegeben seien weiter $\omega_{1}, \omega_{2} \colon \Omega \to \mathbb{R}$ zwei $L_{\infty}(\Omega)$-Abbildungen und ein $f \in (0, T)$.
Wir definieren damit
\begin{equation}
    \omega \colon I \times \Omega \to \mathbb{R}, \quad (t, x) \mapsto
    \begin{cases}
        \omega_{1}(x), & t \leq f \\
        \omega_{2}(x), & t > f.
    \end{cases}
\end{equation}

Wir betrachten nun die folgende parabolische partielle Differentialgleichung.
\begin{equation}
    u_{t}(t, x) = c \Delta_{x} u(t, x) - w(t, x) u(t, x) \qquad \text{auf}~I \times \Omega,
\end{equation}
wobei $c \in \mathbb{R}$ eine Konstante ist, und es seien weiter Anfangs- und Randwertbedingungen gegeben.
Diese sind im Falle der bei \textcite{Stasiak:2011ba} betrachteten Variante zum Beispiel periodische Randbedingungen in $\partial \Omega$ mit der Anfangsbedingung $u(0, \blank) = 1$.
Wir beschränken uns bei dieser Arbeit auf den Fall homogener Randbedingungen und
dazu kompatiblen Anfangsbedingungen.

\todo[inline]{Die Einleitung ist mies...}

% section einf_hrung_der_ppde (end)

\section{Der betrachtete Fall} % (fold)
\label{sec:der_betrachtete_fall}

Wir betrachten in diesem Abschnitt zunächst eine vereinfachte Variante der vorgestellten Differentialgleichung.
Zunächst ignorieren wir den Wechsel des Feldes $\omega$ ab einem bestimmten Zeitpunkt und erhalten dadurch einen autonomen linearen Differentialoperator $A$.
Weiter schränken wir uns auf homogene Dirichlet- statt periodischen Randbedingungen ein.

Unter diesen Gegebenheiten bietet es sich an, die Hilberträume als $V = H^{1}_{0}(\Omega)$ und $H = L_{2}(\Omega)$ zu wählen.
Bekanntlich sind diese separabel und es existiert eine dichte stetige Einbettung von $H^{1}_{0}(\Omega)$ in $L_{2}(\Omega)$.
Wegen $(H^{1}_{0}(\Omega))' = H^{-1}(\Omega)$ ergibt dies das Gelfand-Tripel
\begin{equation}
    H^{1}_{0}(\Omega) \denseinclusion L_{2}(\Omega) \denseinclusion H^{-1}(\Omega).
\end{equation}
Wie zuvor verwenden wir $\skprod{\blank}{\blank}$ mit entsprechendem Index sowohl für die Skalarprodukte als auch für die duale Paarung auf $H^{-1}(\Omega) \times H^{1}_{0}(\Omega)$.

Um obige partielle Differentialgleichung in das Setting aus \autoref{sec:lineare_evolutionsgleichungen} zu übertragen, definieren wir einen linearen Operator $A$ als
\begin{equation}
    \label{eq:def_op_A}
    A \colon H^{1}_{0}(\Omega) \to H^{-1}(\Omega), \quad \eta \mapsto A \eta = - c \Delta \eta + \omega \eta
\end{equation}
und weiter die zugehörige Bilinearform
\begin{equation}
    a \colon H^{1}_{0}(\Omega) \times H^{1}_{0}(\Omega) \to \mathbb{R}, \quad a(\eta, \zeta) = \skprod{A \eta}{\zeta}_{L_{2}(\Omega)}.
\end{equation}
Diese lässt sich unter Verwendung der Greenschen Formeln (TODO!) auch schreiben als
\begin{equation}
    \begin{aligned}
        a(\eta, \zeta)
        &= \skprod{- c \Delta \eta + \omega \eta}{\zeta}_{L_{2}(\Omega)}
        = - c \skprod{\Delta \eta}{\zeta}_{L_{2}(\Omega)} + \skprod{\omega \eta}{\zeta}_{L_{2}(\Omega)}
        \\&= c \skprod{\grad \eta}{\grad \zeta}_{L_{2}(\Omega)} + \skprod{\omega \eta}{\zeta}_{L_{2}(\Omega)}.
    \end{aligned}
\end{equation}

Diese Bilinearform ist stetig und erfüllt eine G\aa{}rding-Ungleichung, wie das folgende Lemma zeigt.

\begin{Lemma}
\label{lemma:a_bf_bounded_garding}
    Seien $c \in \mathbb{R}_{+}$, $\omega \in L_{\infty}(\Omega)$ und
    \begin{equation}
    \label{eq:bf_a}
        a \colon H^{1}_{0}(\Omega) \times H^{1}_{0}(\Omega) \to \mathbb{R}, \quad a(\eta, \zeta) = c \skprod{\grad \eta}{\grad \zeta}_{L_{2}(\Omega)} + \skprod{\omega \eta}{\zeta}_{L_{2}(\Omega)}.
    \end{equation}
    Dann erfüllt $a$ die Eigenschaften aus \thref{annahme:eigenschaften_bf_a}:
    \begin{thmenumerate}
        \item\label{lemma:a_bf_bounded_garding:1}
        \emph{Stetigkeit:} es gilt
        \begin{equation}
            \abs{a(\eta, \zeta)} \leq M_{a} \norm{\eta}_{H^{1}(\Omega)} \norm{\zeta}_{H^{1}(\Omega)} \quad \text{für alle}~\eta, \zeta \in H^{1}_{0}(\Omega)
        \end{equation}
        mit $M_{a} = \max\Set{c, \norm{\omega}_{L_{\infty}(\Omega)} } \geq 0$.
        \item\label{lemma:a_bf_bounded_garding:2}
        \emph{G\aa{}rding-Ungleichung:} es gilt
        \begin{equation}
                a(\eta, \eta) + \lambda \norm{\eta}_{L_{2}(\Omega)}^{2} \geq \alpha \norm{\eta}_{H^{1}(\Omega)}^{2} \quad \text{für alle}~\eta \in H^{1}_{0}(\Omega)
        \end{equation}
        mit $\alpha = c \gamma_{\Omega}^{2} > 0$ und $\lambda = \norm{\omega}_{L_{\infty}(\Omega)} \geq 0$, wobei $\gamma_{\Omega}$ die Poincaré-Friedrichs-Konstante ist.
    \end{thmenumerate}

    \begin{Beweis}
    Wir zeigen zunächst die Stetigkeit.
    Seien dazu $\eta, \zeta \in H^{1}_{0}(\Omega)$ beliebig.
    Unter Verwendung der Dreiecks- und der Cauchy-Schwarz-Ungleichung erhalten wir
    \begin{align}
        \abs{a(\eta, \zeta)}
        &= \abs{c \skprod{\grad \eta}{\grad \zeta}_{L_{2}(\Omega)} + \skprod{\omega \eta}{\zeta}_{L_{2}(\Omega)}}
        \\&\leq c \abs{\skprod{\grad \eta}{\grad \zeta}_{L_{2}(\Omega)}} + \abs{\skprod{\omega \eta}{\zeta}_{L_{2}(\Omega)}}
        \\&\leq c \norm{\grad \eta}_{L_{2}(\Omega)} \norm{\grad \zeta}_{L_{2}(\Omega)} + \norm{\omega}_{L_{\infty}(\Omega)} \norm{\eta}_{L_{2}(\Omega)} \norm{\zeta}_{L_{2}(\Omega)}
        \\&\leq \max \Set{ c, \norm{\omega}_{L_{\infty}(\Omega)} } \norm{\eta}_{H^{1}(\Omega)} \norm{\zeta}_{H^{1}(\Omega)}.
    \end{align}

    Für die G\aa{}rding-Ungleichung seien nun $\eta \in H^{1}_{0}(\Omega)$ und $\lambda \in \mathbb{R}$.
    Wir betrachten
    \begin{align}
        a(\eta, \eta) + \lambda \norm{\eta}^{2}_{L_{2}(\Omega)}
        &= c \norm{\grad \eta}^{2}_{L_{2}(\Omega)} + \skprod{\omega \eta}{\eta}_{L_{2}(\Omega)} + \lambda \skprod{\eta}{\eta}_{L_{2}(\Omega)}
        \\&= c \norm{\grad \eta}^{2}_{L_{2}(\Omega)} + \skprod{(\omega + \lambda) \eta}{\eta}_{L_{2}(\Omega)}.
    \end{align}
    Wählen wir nun $\lambda = \norm{\omega}_{L_{\infty}(\Omega)} \geq 0$, dann gilt $\omega + \lambda \geq 0$ fast überall in $\Omega$ und wir erhalten die Abschätzung
    \begin{align}
        a(\eta, \eta) + \lambda \norm{\eta}^{2}_{L_{2}(\Omega)}
        &\geq c \norm{\grad \eta}^{2}_{L_{2}(\Omega)},
        \intertext{woraus wir durch Anwenden der Poincaré-Friedrichs-Ungleichung \ref{satz:grundlagen:poincare_friedrichs_ungleichung}}
        a(\eta, \eta) + \lambda \norm{\eta}^{2}_{L_{2}(\Omega)}
        &\geq c \gamma_{\Omega}^{2} \norm{\eta}^{2}_{H^{1}(\Omega)}
    \end{align}
    folgern.
    \end{Beweis}
\end{Lemma}

Unter diesen Gegebenheiten erhalten wir nach \autoref{sec:lineare_evolutionsgleichungen} eine sachgemäß gestellte Raum-Zeit-Variationsformulierung.
Ansatz- und Testfunktionenraum ergeben sich mit den konkret gewählten Hilberträumen zu
\begin{equation}
    \label{eq:var_ansatzraum_testraum}
    \mathcal X = L_{2}(I; H^{1}_{0}(\Omega)) \cap H^{1}(I; H^{-1}(\Omega))
    \quad \text{und} \quad
    \mathcal Y = L_{2}(I; H^{1}_{0}(\Omega)) \times L_{2}(\Omega).
\end{equation}
Das Variationsproblem lautet damit:
    Gegeben ein $g \in L_{2}(I; H^{-1}(\Omega))$ und ein $u_{0} \in L_{2}(\Omega)$. Finde ein $u \in \mathcal X$ mit
    \begin{equation}
        \label{eq:varprob}
        b(u, v) = f(v) \quad \text{für alle}~v \in \mathcal Y,
    \end{equation}
    wobei $b(\blank, \blank) \colon \mathcal X \times \mathcal Y \to \mathbb{R}$ die durch
    \begin{equation}
        \label{eq:buv}
        b(u, v)
            = \int_{I} \skprod{u_{t}(t)}{v_{1}(t)}_{L_{2}(\Omega)} + a(u(t), v_{1}(t)) \diff t + \skprod{u(0)}{v_{2}}_{L_{2}(\Omega)}
    \end{equation}
    gegebene Bilinearform und $f(\blank) \colon \mathcal Y \to \mathbb{R}$ definiert ist durch
    \begin{equation}
        \label{eq:var_all_f_wiederholung}
        f(v) = \int_{I} \skprod{g(t)}{v_{1}(t)}_{L_{2}(\Omega)} \diff t + \skprod{u_{0}}{v_{2}}_{L_{2}(\Omega)}.
    \end{equation}

Aus \thref{thm:schwab09:theorem51} und \thref{thm:schwab09:theorem51:ungleichungen} erhalten wir nun die Wohldefiniertheit des obigen Variationsproblems und zugleich Schranken für die Operatoren.

\begin{Korollar}
\label{korollar:2.2}
    Seien $\mathcal X$ und $\mathcal Y$ gegeben wie in \eqref{eq:var_ansatzraum_testraum} und sei $B \colon \mathcal X \to \mathcal Y'$ definiert durch
    \begin{equation}
        \skprod{Bu}{v}_{\mathcal Y' \times \mathcal Y}  = b(u, v), \quad u \in \mathcal X,~ v \in \mathcal Y,
    \end{equation}
    mit $b(\blank, \blank)$ wie in \eqref{eq:buv}.
    Dann ist $B$ stetig invertierbar und es gilt
    \begin{equation}
        \norm{B}_{\mathcal L(\mathcal X, \mathcal Y')}
        \leq
        \frac{\sqrt{2 \max\Set{1, c^{2}, \norm{\omega}_{L_{\infty}(\Omega)}^{2}} + M_{e}^{2}}}{\max\Set{\sqrt{1 + 2 \norm{\omega}_{L_{\infty}(\Omega)}^{2} \rho^{4}}, \sqrt{2} }}
    \end{equation}
    und
    \begin{equation}
        \norm{B^{-1}}_{\mathcal L( \mathcal Y', \mathcal X)}
        \leq \frac{e^{2 T \norm{\omega}_{L_{\infty}(\Omega)}} \max\Set{\sqrt{1 + 2 \norm{\omega}_{L_{\infty}(\Omega)}^{2} \rho^{4}}, \sqrt{2}} \sqrt{2 \max\Set{c^{-2} \gamma_{\Omega}^{-4}, 1} + M_{e}^{2}}}{\min\Set{c^{-1} \gamma_{\Omega}^{2}, c \gamma_{\Omega}^{2} \norm{\omega}_{L_{\infty}(\Omega)}^{-2}, c \gamma_{\Omega}^{2} }}.
        % \leq
        % \frac{\max\{\sqrt{ 1 + 2 \norm{\omega}_{L_{\infty}(\Omega)} \rho^{4}}, \sqrt{2} \}}{e^{-2 \norm{\omega}_{L_{\infty}(\Omega)} T}}
        % \frac{\sqrt{2 \max\{ 1, \sigma^{-2} \gamma_{\Omega}^{-4} \} + M_{e}^{2}}}{\min\{ \sigma \gamma_{\Omega}^{2} \norm{\omega}_{L_{\infty}(\Omega)}^{-2}, \sigma \gamma_{\Omega}^{2} \}}
    \end{equation}
    mit $M_{e}$ und $\rho$ wie in \eqref{eq:var_all_M_e} respektive \eqref{eq:var_all_rho}.
\end{Korollar}

% section der_betrachtete_fall (end)

\section{Parametrische Formulierung} % (fold)
\label{sec:parametrische_formulierung}

\todo[inline]{Ordentlich aufschreiben}

Dieser Abschnitt dient der Einführung einer parametrischen Variante der zuvor vorgestellten parabolischen partiellen Differentialgleichung.

Für den weiteren Verlauf der Arbeit wählen wir $V = H^{1}_{0}(\Omega)$, $H = L_{2}(\Omega)$ und $V' = H^{-1}(\Omega)$.
% Diese separablen Hilberträume bilden ein Gelfand-Tripel $V \denseinclusion H \denseinclusion V'$.

Wir betrachten nun zunächst die folgende parametrische Operatorgleichung: Sei ein $g \in V'$ gegeben, finde für alle $\omega$ ein $u(\omega) \in V$, so dass
\begin{equation}
    A(\omega) u(\omega) = g \in V'
\end{equation}
gilt.
Durch die Wahl $V = H^{1}_{0}(\Omega)$ sind die Randbedingungen $\restr{u(\omega)}{\partial \Omega} = 0$ bereits implizit gegeben.
Dabei sei der Operator $A(\omega)$ gegeben durch
\begin{equation}
    \label{eq:pp:op_a}
    A(\omega) \colon V \to V', \quad A(\omega) u = - c \Delta u + \omega u + \mu u.
\end{equation}
Die zugehörige Bilinearform $a(\blank, \blank; \omega)$ ergibt sich damit zu
\begin{equation}
    \label{eq:pp:bf_a}
    a(\blank, \blank; \omega) \colon V \times V \to \mathbb{R},
    \quad (u, v) \mapsto c\skp{\grad u}{\grad v}{H} + \skp{\omega u}{v}{H} + \mu \skp{u}{v}{H}.
\end{equation}

Unter diesen Bedingungen erhalten wir für die Bilinearform aus \eqref{eq:pp:bf_a} respektive \eqref{eq:pp:bf_a_sigma} die folgenden Eigenschaften.
%
\begin{Satz}
\label{satz:pp:a_bf_bounded_garding}
    Seien $c \in \mathbb{R}_{+}$, $\mu \in \mathbb{R}$, $\omega \in L_{\infty}(\Omega)$ und
    \begin{equation}
    \label{eq:bf_a}
        \begin{aligned}
            &a(\blank, \blank) \colon H^{1}_{0}(\Omega) \times H^{1}_{0}(\Omega) \to \mathbb{R}, \\
            &(u, v) \mapsto c\skp{\grad u}{\grad v}{L_{2}(\Omega)} + \skp{\omega u}{v}{L_{2}(\Omega)} + \mu \skp{u}{v}{L_{2}(\Omega)}.
        \end{aligned}
    \end{equation}
    Dann erfüllt $a$ die Eigenschaften aus \thref{annahme:eigenschaften_bf_a}:
    \begin{thmenumerate}
        \item\label{satz:pp:a_bf_bounded_garding:1}
        \emph{Stetigkeit:} es gilt
        \begin{equation}
            \abs{a(\eta, \zeta)} \leq M_{a} \norm{\eta}_{H^{1}(\Omega)} \norm{\zeta}_{H^{1}(\Omega)} \quad \text{für alle}~\eta, \zeta \in H^{1}_{0}(\Omega)
        \end{equation}
        mit $M_{a} = \max\Set{c, \norm{\omega}_{L_{\infty}(\Omega)} + \abs{\mu}} \geq 0$.
        \item\label{satz:pp:a_bf_bounded_garding:2}
        \emph{G\aa{}rding-Ungleichung:} es gilt
        \begin{equation}
                a(\eta, \eta) + \lambda \norm{\eta}_{L_{2}(\Omega)}^{2} \geq \alpha \norm{\eta}_{H^{1}(\Omega)}^{2} \quad \text{für alle}~\eta \in H^{1}_{0}(\Omega)
        \end{equation}
        mit $\alpha = c \gamma_{\Omega}^{2} > 0$ und $\lambda = \min\Set{\norm{\omega}_{L_{\infty}(\Omega)} - \mu, 0} \geq 0$, wobei $\gamma_{\Omega}$ die Poincaré-Friedrichs-Konstante ist.
    \end{thmenumerate}

    \begin{Beweis}
    Wir zeigen zunächst die Stetigkeit.
    Seien dazu $\eta, \zeta \in H^{1}_{0}(\Omega)$ beliebig.
    Unter Verwendung der Dreiecks- und der Cauchy-Schwarz-Ungleichung erhalten wir
    \begin{align}
        \abs{a(\eta, \zeta)}
        &= \abs{c \skprod{\grad \eta}{\grad \zeta}_{L_{2}(\Omega)} + \skprod{\omega \eta}{\zeta}_{L_{2}(\Omega)} + \mu \skp{\eta}{\zeta}{L_{2}(\Omega)} }
        \\&\leq c \abs{\skprod{\grad \eta}{\grad \zeta}_{L_{2}(\Omega)}} + \abs{\skprod{\omega \eta}{\zeta}_{L_{2}(\Omega)}} + \abs{\mu} \abs{\skp{\eta}{\zeta}{L_{2}(\Omega)}}
        \\&\leq c \norm{\grad \eta}_{L_{2}(\Omega)} \norm{\grad \zeta}_{L_{2}(\Omega)} + (\norm{\omega}_{L_{\infty}(\Omega)} + \abs{\mu}) \norm{\eta}_{L_{2}(\Omega)} \norm{\zeta}_{L_{2}(\Omega)}
        \\&\leq \max \Set{ c, \norm{\omega}_{L_{\infty}(\Omega)} + \abs{\mu}} \norm{\eta}_{H^{1}(\Omega)} \norm{\zeta}_{H^{1}(\Omega)}.
    \end{align}

    Für die G\aa{}rding-Ungleichung seien nun $\eta \in H^{1}_{0}(\Omega)$ und $\lambda \in \mathbb{R}$.
    Wir betrachten
    \begin{align}
        a(\eta, \eta) + \lambda \norm{\eta}^{2}_{L_{2}(\Omega)}
        &= c \norm{\grad \eta}^{2}_{L_{2}(\Omega)} + \skprod{\omega \eta}{\eta}_{L_{2}(\Omega)} + \mu \skprod{\eta}{\eta}_{L_{2}(\Omega)} + \lambda \skprod{\eta}{\eta}_{L_{2}(\Omega)}
        \\&= c \norm{\grad \eta}^{2}_{L_{2}(\Omega)} + \skprod{(\omega + \mu + \lambda) \eta}{\eta}_{L_{2}(\Omega)}.
    \end{align}
    Wählen wir nun $\lambda = \min\Set{\norm{\omega}_{L_{\infty}(\Omega)} - \mu, 0} \geq 0$, dann gilt $\omega + \mu + \lambda \geq 0$ fast überall in $\Omega$ und wir erhalten die Abschätzung
    \begin{align}
        a(\eta, \eta) + \lambda \norm{\eta}^{2}_{L_{2}(\Omega)}
        &\geq c \norm{\grad \eta}^{2}_{L_{2}(\Omega)},
        \intertext{woraus wir durch Anwenden der Poincaré-Friedrichs-Ungleichung \ref{satz:grundlagen:poincare_friedrichs_ungleichung}}
        a(\eta, \eta) + \lambda \norm{\eta}^{2}_{L_{2}(\Omega)}
        &\geq c \gamma_{\Omega}^{2} \norm{\eta}^{2}_{H^{1}(\Omega)}
    \end{align}
    folgern.
    \end{Beweis}
\end{Satz}

\begin{Korollar}
    Ist $\mu \geq \norm{\omega}_{L_{\infty}(\Omega)}$, dann ist die Bilinearform koerziv.
\end{Korollar}

\begin{Satz}
\label{satz:pp:lax_auf_elliptisch}
    Seien $\omega \in L_{\infty}(\Omega)$, $\mu \geq \norm{\omega}_{L_{\infty}(\Omega)}$ und weiter $g \in H^{-1}(\Omega)$ und $A(\omega)$ wie in \eqref{eq:pp:op_a}, dann besitzt die Operatorgleichung
    \begin{equation}
        A(\omega) u(\omega) = g
    \end{equation}
    eine eindeutige Lösung $u(\omega) \in H^{1}_{0}(\Omega)$ und diese erfüllt
    \begin{equation}
        \norm{u(\omega)}_{H^{1}(\Omega)} \leq \frac{\norm{g}_{H^{-1}(\Omega)}}{\alpha}
    \end{equation}
    mit $\alpha$ aus \thref{satz:pp:a_bf_bounded_garding}.

    \begin{Beweis}
        Folgt aus dem Banach-Ne\v{c}as-Babu\v{s}ka-Theorem, \thref{satz:gl:bnb_theorem}.
    \end{Beweis}
\end{Satz}

\todo[inline]{Ab hier parametrisch}

Wir wollen nun die Abhängigkeit des Operators $A(\omega)$ von dem Parameter $\omega$ konkretisieren.

\begin{Definition}
\label{definition:pp:omega_affin}
    Die Funktion $\omega$ ist affin darstellbar.
    Genauer sei $\mathcal S \subset \mathbb{R}^{\mathbb{N}}$ ein Parameterraum und $\Set{ \varphi_{j} }_{j \in \mathbb{N}} \in L_{\infty}(\Omega)$ eine Folge von Funktion, so dass $\omega$ sich für $\sigma \in \mathcal S$ schreiben lässt als
    \begin{equation}
        w(\blank; \sigma) \colon \Omega \to \mathbb{R}, \quad w(x; \sigma) = \sum_{j = 1}^{\infty} \sigma_{j} \varphi_{j}(x).
    \end{equation}
\end{Definition}

\begin{Bemerkung}
    Wir wählen für den Rest der Arbeit $\mathcal S = [-1, 1]^{\mathbb{N}}$.
    Dies stellt keine Einschränkung dar, da die Funktionen $\Set{ \varphi_{j} }_{j \in \mathbb{N}}$ beliebig umskaliert werden können.
\end{Bemerkung}

Setzen wir diese affine Darstellung nun zunächst in den Operator $A(\omega)$ ein, dann erhalten wir die Darstellung
\begin{equation}
    A(\omega(\sigma)) \colon V \to V', \quad A(\omega(\sigma)) u = -c \Delta u + \sum_{j = 1}^{\infty} \sigma_{j} \varphi_{j} u + \mu u,
\end{equation}
und als zugehörige Bilinearform $a(\blank, \blank; \omega(\sigma))$ ergibt sich
\begin{equation}
\label{eq:pp:bf_a_sigma}
    a(\blank, \blank; \omega(\sigma)) \colon V \times V \to \mathbb{R}, \quad a(u, v) \mapsto c\skp{\grad u}{\grad v}{H} + \sum_{j = 1}^{\infty} \sigma_{j} \skp{\varphi_{j} u}{v}{H} + \mu \skp{u}{v}{H}.
\end{equation}

\begin{Bemerkung}
    Um die Schreibweisen zu verkürzen, verwenden wir meist $\omega(\sigma)$ statt $w(\blank; \sigma)$, sowie $A(\sigma)$ und $a(\blank, \blank; \sigma)$ statt $A(\omega(\sigma))$ respektive $a(\blank, \blank; \omega(\sigma))$.
\end{Bemerkung}

Damit der Operator $A(\sigma)$ sowie die Bilinearform $a(\blank, \blank; \sigma)$ wohldefiniert sind, müssen wir Wohldefiniertheit, dass heißt gleichmäßige Konvergenz, der obigen affinen Zerlegung von $\omega$ aus \thref{definition:pp:omega_affin} fordern.
Dies wird durch folgende Bedingung sichergestellt.
%%
\begin{Annahme}
    Das Funktionensystem $\Set{ \varphi_{j} }_{j \in \mathbb{N}} \in L_{\infty}(\Omega)$ sei einfach summierbar in der $L_{\infty}$-Norm, das heißt es gelte
    \begin{equation}
        \Set{ \norm{\varphi_{j}}_{L_{\infty}(\Omega) } }_{j \in \mathbb{N}} \in \ell_{1}(\mathbb{N}).
    \end{equation}
\end{Annahme}
%%
Hieraus folgt wegen $\mathcal S = [-1, 1]^{\mathbb{N}}$ insbesondere
\begin{equation}
    \sup_{\sigma \in \mathcal S} \norm{\omega(\sigma)}_{L_{\infty}(\Omega)} \leq \sum_{j = 1}^{\infty} \norm{\varphi_{j}}_{L_{\infty}(\Omega)} < \infty.
\end{equation}

% section parametrische_formulierung (end)

\section{Regularität bezüglich des Parameters} % (fold)
\label{sec:regularit_t_bez_glich_des_parameters}

In diesem Abschnitt wollen wir nun unter geeigneten, noch näher zu bestimmenden Bedingungen, die analytische Abhängigkeit der Lösung $u(\sigma)$ der zuvor eingeführten PPDE vom Parameter $\sigma \in \mathcal S$ nachweisen.
Dabei orientieren wir uns vor allem an den Arbeiten von \textcite{Cohen:2010kz,Kunoth:2013ef}.

\begin{Lemma}
    Seien $\omega_{1}, \omega_{2} \in L_{\infty}(\Omega)$ und $u_{1}, u_{2}$ die zugehörigen Lösungen, dann gilt
    \begin{equation}
        \norm{u_{1} - u_{2}}_{V} \leq \frac{\norm{f}_{V'}}{\gamma_{0}^{2}} \norm{\omega_{1} - \omega_{2}}_{L_{\infty}}.
    \end{equation}

    \begin{Beweis}
        Durch Subtraktion der beiden Variationsformulierungen erhalten wir für $v \in V$ die Gleichung
        \begin{align}
            0 &= a(u_{1}, v; \omega_{1}) - a(u_{2}, v; \omega_{2})
            \\&= c \skp{\grad u_{1} - \grad u_{2}}{\grad v}{H} + \skp{\omega_{1}u_{1} - \omega_{2} u_{2}}{v}{H} + \mu \skp{u_{1} - u_{2}}{v}{H},
            \intertext{durch setzen von $z = u_{1} - u_{2}$ erhalten wir weiter}
            0 &= c \skp{\grad z}{\grad v}{H} + \skp{\omega_{1} z}{v}{H} + \mu \skp{z}{v}{H} + \skp{(\omega_{1} - \omega_{2}) u_{2}}{v}{H}
            \\&= a(z, v; \omega_{1}) + \skp{(\omega_{1} - \omega_{2}) u_{2}}{v}{H}.
        \end{align}
        Dies lässt sich nun wieder in Form des Variationsproblems schreiben, konkret
        \begin{equation}
            a(z, v; \omega_{1}) = g(v) \quad \fa v \in V,
        \end{equation}
        mit
        \begin{equation}
            g(v) = - \skp{(\omega_{1} - \omega_{2}) u_{2}}{v}{H}.
        \end{equation}

        Nach \thref{satz:pp:lax_auf_elliptisch} ist die Lösung $z = u_{1} - u_{2} \in V$ eindeutig und erfüllt
        \begin{equation}
            \norm{z}_{V} \leq \frac{\norm{g}_{V'}}{\gamma_{0}}.
        \end{equation}

        Die Operatornorm von $g$ lässt sich mittels der Cauchy-Schwarz-Ungleichung bestimmen zu
        \begin{equation}
            \begin{aligned}
                \norm{g}_{V'}
                  &=    \sup_{\norm{v}_{V} = 1} \abs{g(v)}
                   =    \sup_{\norm{v}_{V} = 1} \abs{\skp{(\omega_{1} - \omega_{2}) u_{2}}{v}{H}}
                \\&\leq \sup_{\norm{v}_{V} = 1} \norm{\omega_{1} - \omega_{2}}_{L_{\infty}(\Omega)} \norm{u_{1}}_{H} \norm{v}_{H}
                   \leq \sup_{\norm{v}_{V} = 1} \norm{\omega_{1} - \omega_{2}}_{L_{\infty}(\Omega)} \norm{u_{1}}_{V} \norm{v}_{V}
                \\&=    \norm{\omega_{1} - \omega_{2}}_{L_{\infty}(\Omega)} \norm{u_{1}}_{V}
                   \leq \norm{\omega_{1} - \omega_{2}}_{L_{\infty}(\Omega)} \frac{\norm{f}_{V'}}{\gamma_{0}}.
            \end{aligned}
        \end{equation}
        Zusammen liefert dies die Ungleichung
        \begin{equation}
            \norm{u_{1} - u_{2}}_{V}
            = \norm{z}_{V} \leq \norm{\omega_{1} - \omega_{2}}_{L_{\infty}(\Omega)} \frac{\norm{f}_{V'}}{\gamma_{0}^{2}}
        \end{equation}
        und damit die Behauptung.
    \end{Beweis}
\end{Lemma}

\begin{Satz}
    Die Abbildung $\mathcal S \ni \sigma \mapsto u(\sigma) \in V$ besitzt für alle $\nu \in \mathfrak F$ die partielle Ableitung $\partial^{\nu}_{\sigma} u(\sigma)$.

    \begin{Beweis}
        Sei $\sigma \in \mathcal S$ fest.
        Wir zeigen die Behauptung für die partiellen Ableitungen erster Ordnung, das heißt, es sei $\nu = e_{j}$ und $j \in \mathbb{N}$.
        Sei weiter $h \in \mathbb{R} \setminus \Set{ 0 }$ gegeben, dann definieren wir $\omega_{h} := \omega(\sigma + h e_{j})$ und
        \begin{equation}
            u_{h}(\omega_{0}) := \frac{u(\omega_{h}) - u(\omega_{0})}{h}.
        \end{equation}
        Dieser Ausdruck ist wohldefiniert?!?! TODO?!?!?!

        Wir subtrahieren erneut die beiden Variationsformulierungen für $\omega_{h}$ und $\omega_{0}$ von einander und erhalten damit
        \todo[inline]{Zwischenschritte beschreiben}
        \begin{align}
            0
                &=  a(u(\omega_{h}), v; \omega_{h})
                    - a(u(\omega_{0}), v; \omega_{0})
            \\  &=
                    c \skp{\grad u(\omega_{h})}{\grad v}{H}
                    + \skp{\omega_{h} u(\omega_{h})}{v}{H}
                    + \mu \skp{u(\omega_{h})}{v}{H}
            \\&\qquad
                    - c \skp{\grad u(\omega_{0})}{\grad v}{H}
                    - \skp{\omega_{0} u(\omega_{0})}{v}{H}
                    - \mu \skp{u(\omega_{0})}{v}{H}
            \\  &=
                    c \skp{\grad u(\omega_{h}) - \grad u(\omega_{0})}{\grad v}{H}
                    + \skp{\omega_{h} u(\omega_{h}) - \omega_{0} u(\omega_{0})}{v}{H}
            \\&\qquad
                    + \mu \skp{u(\omega_{h}) - u(\omega_{0})}{v}{H}
            \\  &=
                    h c \skp{\grad u_{h}(\omega_{0})}{v}{H}
                    + \skp{\omega_{0} ( u(\omega_{h}) - u(\omega) ) }{v}{H}
            \\&\qquad
                    + \skp{(\omega_{h} - \omega) u(\omega_{h})}{v}{H}
                    + h \mu \skp{u_{h}(\omega_{0})}{v}{H}
            \\  &=
                    h c \skp{\grad u_{h}(\omega_{0})}{v}{H}
                    + h \skp{\omega_{h} u_{h}(\omega_{0}) }{v}{H}
                    + h \mu \skp{u_{h}(\omega_{0})}{v}{H}
            \\&\qquad
                    + \skp{(\omega_{h} - \omega) u(\omega_{h})}{v}{H}
            \\  &=
                    h a(u_{h}(\omega_{0}), v; \omega)
                    + \skp{(\omega_{h} - \omega) u(\omega_{h})}{v}{H}
        \end{align}
    \end{Beweis}
\end{Satz}

Seien $b := (b_{j})_{j} \in \mathbb{R}$ und $b_{j} := \frac{\norm{\varphi_{j}}_{L_{\infty}}}{\gamma_{0}}$.

\begin{Satz}
    Es gilt
    \begin{equation}
        \sup_{\sigma \in \mathcal S} \norm{\partial^{\nu}_{\sigma} u(\sigma)} \leq B \abs{\nu}! b^{\nu}.
    \end{equation}

    \begin{Beweis}
        folgt.
    \end{Beweis}
\end{Satz}

\todo[inline]{Daraus folgern, dass es für den parabolischen auch gilt.}

% section regularit_t_bez_glich_des_parameters (end)

% chapter parametrische_problem_neuer_versuch (end)


\chapter{Ab hier geht der Müll los} % (fold)
\label{cha:ab_hier_geht_der_m_ll_los}

% chapter ab_hier_geht_der_m_ll_los (end)

\todo[inline]{Alten Kram entfernen}

\todo[inline]{Anpassen an Zeitabhängige lineare Operatoren bzw. Bilinearformen!}

In diesem Kapitel liegt das Augenmerk erneut auf der linearen Evolutionsgleichung \eqref{eq:allgemeine_parabolische_pde}, diesmal aber mit der Erweiterung, dass der lineare Operator $A(t)$ zusätzlich von einem Parameter $\sigma$ abhängt.

Zunächst konkretisieren wir diese Parameterabhängigkeit für einen linearen Operator $A$, betrachten dann eine parametrische lineare Operatorgleichung, leiten Regularitätsergebnisse für diese her und übertragen diese anschließend auf die Raum-Zeit-Variationsformulierung einer parametrischen linearen Evolutionsgleichung.
Dabei orientieren wir uns hauptsächlich an den Arbeiten von \textcite{Kunoth:2013ef,Cohen:2010kz}.

\section{Parametrische Operatorgleichung} % (fold)
\label{sec:parametrische_operatorgleichung}

Seien $X$ und $Y$ zwei reflexive Banachräume.
Weiter sei $\mathcal S \subset \mathbb{R}^{\mathbb{N}}$ der sogenannte Parameterraum.
Der Einfachheit halber wählen wir $\mathcal S = [-1, 1]^{\mathbb{N}}$.
\todo[inline]{Warum reicht das?}

Wir betrachten parametrische Familien stetiger linearer Operatoren $A(\sigma) \in \mathcal L(X, Y')$ mit $\sigma \in \mathcal S$.
Folgende lineare Operatorgleichung ist für uns von Interesse:
Sei ein $g \in Y'$ gegeben.
Finde für alle $\sigma \in \mathcal S$ eine Lösung $u(\sigma) \in X$ von
\begin{equation}
    \label{eq:allgemeine_parametrische_elliptische_pde}
    A(\sigma) u(\sigma) = g \quad \text{in}~Y'.
\end{equation}
Wie zuvor sei $a(\blank, \blank; \sigma) \colon X \times Y \to \mathbb{R}$ die zugehörige Bilinearform.

Zunächst einige notationelle Vorbemerkungen.
\begin{Bemerkung}
    Wir bezeichnen mit $\mathfrak F = \Set{ \nu \in \mathbb{N}^{\mathbb{N}}_{0} \given \abs{\nu} < \infty }$ die Menge aller Folgen nichtnegativer ganzer Zahlen mit endlichem Träger, das heißt nur endlich vielen Einträgen ungleich Null.
    % NOTE: Eventuell mehr definieren, siehe $\mathfrak n$ und $\mathfrak m$

    Sei $\nu \in \mathfrak F$ und $b \in \ell_{p}(\mathbb{N})$, $p > 0$, dann schreiben wir
    \begin{equation}
        b^{\nu} = \prod_{j = 1}^{\infty} b_{j}^{\nu_{j}}
    \end{equation}
    mit der Konvention $0^{0} = 1$.
    Wegen $\abs{\nu} < \infty$ ist dieses Produkt stets endlich.
\end{Bemerkung}

Für die nachfolgenden Regularitätsaussagen über die Lösung $u(\sigma)$ von \eqref{eq:allgemeine_parametrische_elliptische_pde} benötigen wir Regularität der Operatorfamilie $A(\sigma)$ bezüglich $\sigma \in \mathcal S$.
Konkret fordern wir:
\begin{Annahme}[{{\cite[Assumption 1]{Kunoth:2013ef}}}]
\label{thm:kunoth:assumption1}
    Die parametrische Familie von Operatoren
    $\Set{ A(\sigma) \in \mathcal L(X, Y') \given \sigma \in \mathcal S }$ sei eine $\mathfrak p$-reguläre Operatorfamilie für ein $0 < \mathfrak p \leq 1$, das heißt,
    \begin{thmenumerate}
        \item $A(\sigma) \in \mathcal L(X, Y')$ sei stetig invertierbar für alle $\sigma \in \mathcal S$ mit gleichmäßig beschränktem Inversen $A{(\sigma)}^{-1} \in \mathcal L(Y', X)$, das heißt, es existiert ein $C_{0} > 0$ mit
        \begin{equation}
            \sup_{\sigma \in \mathcal S} \norm{A{(\sigma)}^{-1}}_{\mathcal L(Y', X)} \leq C_{0},
        \end{equation}
        \item für jedes feste $\sigma \in \mathcal S$ seien die Operatoren $A(\sigma)$ analytisch bezüglich $\sigma$.
        Konkret existiert eine nichtnegative Folge $b = (b_{j})_{j \in \mathbb{N}} \in \ell_{\mathfrak p}(\mathbb{N})$, so dass
        \begin{equation}
            \sup_{\sigma \in \mathcal S} \norm{(A{(0)})^{-1}(\partial^{\nu}_{\sigma} A(\sigma))}_{\mathcal L(X, X)} \leq C_{0} b^{\nu}
        \end{equation}
        für alle $\nu \in \mathfrak F \setminus \{ 0 \}$ gilt.
        Dabei sei $\partial^{\nu}_{\sigma} A(\sigma) \deq \frac{\partial^{\nu_{1}}}{\partial \sigma_{1}} \frac{\partial^{\nu_{2}}}{\partial \sigma_{2}} \cdots A(\sigma)$.
    \end{thmenumerate}
\end{Annahme}

Die bisherigen Anforderungen an $A(\sigma)$ decken einen noch sehr weiten Bereich ab.
Wir beschränken uns in dieser Arbeit aus praktischen Gründen ausschließlich auf den folgenden Fall, der affin parametrischen Operatoren.

\begin{Definition}
    Sei $\Set{ A(\sigma) \in \mathcal L(X, Y') \given \sigma \in \cal S }$ eine parametrische Operatorfamilie.
    Wir nennen $A(\sigma)$ einen \emph{affin parametrischen Operator}, falls eine Familie von Operatoren $\Set{ \hat A, A_{j} \given j \in \mathbb{N} } \subset \cal L(X, Y')$ existiert, so dass
    \begin{equation}
        \label{eq:all_affiner_operator}
        A(\sigma) = \hat A + \sum_{j = 1}^{\infty} \sigma_{j} A_{j} \qquad\fa \sigma \in \mathcal S
    \end{equation}
    gilt.
\end{Definition}

Seien $\hat a, a_{j} \colon X \times Y \to \mathbb{R}$ die durch den Rieszschen Darstellungssatz von $\hat A$ respektive $A_{j}$ induzierten Bilinearformen, das heißt also,
\begin{equation}
    \label{eq:allg_affine_bf}
    \begin{aligned}
    \hat a(\eta, \zeta) &= \skprod{\hat A \eta}{\zeta}_{Y' \times Y}
    \\
    a_{j}(\eta, \zeta) &= \skprod{A_{j} \eta}{\zeta}_{Y' \times Y}, \quad j \in \mathbb{N},
    \end{aligned}
\end{equation}
für $\eta \in X$, $\zeta \in Y$.

Um die Wohldefiniertheit von $A(\sigma)$, das heißt Konvergenz von \eqref{eq:all_affiner_operator}, sicherzustellen, stellen wir folgende Bedingungen:
\begin{Annahme}[{{\cite[Assumption 2]{Kunoth:2013ef}}}]
\label{thm:kunoth:assumption2}
    Die Operatorfamilie $\Set{\hat A, A_{j} \given j \in \mathbb{N}}$ erfülle folgende Eigenschaften:
    \begin{thmenumerate}
        \item Der \emph{Mean Field}-Operator $\hat A \in \mathcal L(X, Y')$ sei stetig invertierbar, das heißt, es existiert ein $\gamma_{0} > 0$ mit
        \begin{subequations}\label{eq:kunoth:ass2_gamma_0}
            \begin{align}
                \label{eq:kunoth:ass2_gamma_0_a}
                \inf_{0 \neq u \in X} \sup_{0 \neq v \in Y} \frac{\hat a(u, v)}{\norm{u}_{X} \norm{v}_{Y}} \geq \gamma_{0}
                \intertext{und}
                \label{eq:kunoth:ass2_gamma_0_b}
                \inf_{0 \neq v \in Y} \sup_{0 \neq u \in X} \frac{\hat a(u, v)}{\norm{u}_{X} \norm{v}_{Y}} \geq \gamma_{0}.
            \end{align}
        \end{subequations}
        \item Die \emph{Fluctuation}-Operatoren $\Set{ A_{j} }_{j \geq 1}$ seien \emph{klein} relativ zu $\hat A$ im folgenden Sinne: es existiert eine Konstante $0 < \kappa < 1$ so dass
        \begin{equation}
            \label{eq:kunoth:ass2_abs_reihe}
            \sum_{j = 1}^{\infty} \norm{A_{j}}_{\mathcal L(X, Y')} \leq \kappa \gamma_{0}
        \end{equation}
        gilt.
    \end{thmenumerate}
\end{Annahme}

Unter diesen Bedingungen liefert das Banach-Ne{\v c}as-Babu{\v s}ka-Theorem, \thref{satz:gl:bnb_theorem}, die stetige Invertierbarkeit von $A(\sigma)$ aus \eqref{eq:all_affiner_operator} gleichmäßig in $\sigma$.

\begin{Satz}[{{\cite[Theorem 2]{Kunoth:2013ef}}}]
    Der affin parametrische Operator $A(\sigma)$ erfülle \thref{thm:kunoth:assumption2}.
    Dann ist $A(\sigma)$ für alle $\sigma \in \mathcal S$ stetig invertierbar.

    Konkret gilt
    \begin{equation}
        \inf_{u \in H_{1}} \sup_{v \in H_{2}} \frac{a(u, v)}{\norm{u}_{H_{1}} \norm{v}_{H_{2}}} \geq (1 - \kappa) \gamma_{0} > 0 \quad \fa \sigma \in \mathcal S
    \end{equation}
    und
    \begin{equation}
        \inf_{v \in H_{2}} \sup_{u \in H_{1}} \frac{a(u, v)}{\norm{u}_{H_{1}} \norm{v}_{H_{2}}} \geq (1 - \kappa) \gamma_{0} > 0 \quad \fa \sigma \in \mathcal S.
    \end{equation}

    Ist ferner ein $g \in Y'$ gegeben, dann existiert für jedes $\sigma \in \mathcal S$ ein $\hat u(\sigma) \in X$ mit
    \begin{equation}
        a(\hat u(\sigma), v; \sigma) = \skprod{g}{v}_{Y' \times Y} \quad \fa v \in Y
    \end{equation}
    und es gilt die A-Priori-Abschätzung
    \begin{equation}
        \sup_{\sigma \in \mathcal S} \norm{\hat u(\sigma)}_{X} \leq \frac{\norm{g}_{Y'}}{(1 - \kappa) \gamma_{0}}.
    \end{equation}

    \begin{Beweis}
        Nachrechnen der beiden inf-sup-Bedingungen unter Verwendung der affinen Zerlegung von $A(\sigma)$ und anschließendes Anwenden des Banach-Ne{\v c}as-Babu{\v s}ka-Theorems liefert die gewünschten Aussagen.
    \end{Beweis}
\end{Satz}

\begin{Korollar}[{{\cite[Corollary 3]{Kunoth:2013ef}}}]
\label{thm:kunoth:corollary3}
    Die affin parametrische Operatorfamilie $\Set{\hat A, A_{j} \given j \in \mathbb{N}}$ erfülle \thref{thm:kunoth:assumption2}, dann wird auch \thref{thm:kunoth:assumption1} mit $\mathfrak p = 1$ und
    \begin{equation}
        C_{0} = \frac{1}{(1 - \kappa) \gamma_{0}}, \qquad b_{j} = \frac{\norm{A_{j}}_{\mathcal L(X, Y')}}{(1 - \kappa) \gamma_{0}} \quad \fa j \in \mathbb{N},
    \end{equation}
    erfüllt.
\end{Korollar}

Weiter erhält man unter den Bedingungen aus \thref{thm:kunoth:assumption1} folgendes Regularitätsergebnis bezüglich des Parameters $\sigma$.

\begin{Satz}[{{\cite[Theorem 4]{Kunoth:2013ef}}}]
\label{thm:kunoth:theorem4}
    Die parametrische Familie $\Set{ A(\sigma) \in \mathcal L(X, Y') \given \sigma \in \mathcal S }$ erfülle \thref{thm:kunoth:assumption1} für ein $0 < \mathfrak p \leq 1$.
    Dann existiert für jedes $g \in Y'$ und jedes $\sigma \in \mathcal S$ eine eindeutige Lösung $u(\sigma) \in X$ der parametrischen Operatorgleichung
    \begin{equation}
        A(\sigma) u(\sigma) = g \quad \text{in}~Y'.
    \end{equation}

    Die parametrische Familie von Lösungen $u(\sigma)$ hängt analytisch vom Parameter $\sigma$ ab und die partiellen Ableitungen von $u(\sigma)$ erfüllen
    \begin{equation}
        \label{eq:kunoth:schranke_part_abl}
        \sup_{\sigma \in \mathcal S} \norm{(\partial^{\nu}_{\sigma} u)(\sigma)}_{X} \leq C_{0} \norm{g}_{Y'} \abs{\nu}! \tilde{b}^{\nu}
    \end{equation}
    für alle $\nu \in \mathfrak F$, wobei die Folge $\tilde{b} = (\tilde{b}_{j})_{j \geq 1} \in \ell_{\mathfrak p}(\mathbb{N})$ definiert ist durch
    \begin{equation}
        \tilde{b}_{j} = \frac{b_{j}}{\ln 2} \qquad \text{für alle j} \in \mathbb{N}.
    \end{equation}

    \begin{Beweis}
        \todo[inline]{Beweis?}
    \end{Beweis}
\end{Satz}

% section parametrische_operatorgleichung (end)

\section{Parametrische lineare Evolutionsgleichung} % (fold)
\label{sec:parametrische_lineare_evolutionsgleichung}

Dieser Abschnitt soll nun dazu dienen, aufbauend auf \autoref{sec:lineare_evolutionsgleichungen} eine parametrische lineare Evolutionsgleichung zu definieren und anschließend die Regularitätsergebnisse aus dem vorherigen Abschnitt auf diese zu übertragen.

Wir wiederholen kurz das Setting aus \autoref{sec:lineare_evolutionsgleichungen}, in dem wir hier erneut arbeiten.
Seien $V$ und $H$ separable Hilberträume mit einer dichten stetigen Einbettung von $V$ in $H$ und $(V, H, V')$ sei das zugehörige Gelfand-Tripel.
Weiter seien ein $0 < T < \infty$ und ein endliches Zeitintervall $[0, T]$ gegeben.

Wir bezeichnen $\mathcal S = [-1, 1]^{\mathbb{N}}$ weiterhin als Parameterraum.
Es sei für fast alle $t \in [0, T]$ und für alle $\sigma \in \mathcal S$ eine Familie von Bilinearformen
\begin{equation}
    a(\blank, \blank; \sigma, t) \colon V \times V \to \mathbb{R}, \quad (\eta, \zeta) \mapsto a(\eta, \zeta; \sigma, t)
\end{equation}
gegeben, so dass $t \mapsto a(\eta, \zeta; \sigma, t)$ für alle $\sigma \in \mathcal S$ messbar auf $[0, T]$ ist.
Analog zu \thref{annahme:eigenschaften_bf_a} fordern wir diesmal für den Rest dieses Abschnitts:
\begin{Annahme}
\label{annahme:pp:eigenschaften_bf_a}
    \leavevmode
    \begin{thmenumerate}
        \item \emph{Stetigkeit.}
        Es existiert eine Konstante $0 < M_{a} < \infty$, so dass
        \begin{equation}
            \label{eq:allgemeine_parabolische_pde:bf_stetig}
            \abs{a(\eta, \zeta; \sigma, t)} \leq M_{a} \norm{\eta}_{V} \norm{\zeta}_{V} \quad \fa \eta, \zeta \in V
        \end{equation}
        für fast alle $t \in [0, T]$ und alle $\sigma \in \mathcal S$ gilt.
        \item \emph{G\r{a}rding-Ungleichung}.
        Es existieren Konstanten $\alpha > 0$ und $\lambda \geq 0$ mit
        \begin{equation}
            \label{eq:allgemeine_parabolische_pde:bf_garding}
            a(\eta, \eta; \sigma, t) + \lambda \norm{\eta}_{H}^{2} \geq \alpha \norm{\eta}_{V}^{2} \quad \fa \eta \in V
        \end{equation}
        für fast alle $t \in [0, T]$ und alle $\sigma \in \mathcal S$.
    \end{thmenumerate}
\end{Annahme}

Unter diesen Voraussetzungen existiert nach dem Rieszschen Darstellungssatz für jedes $\sigma \in \mathcal S$ und fast alle $t \in [0, T]$ ein stetiger linearer Operator $A(\sigma, t) \in \mathcal L(V, V')$ und es gilt für alle $\sigma \in \mathcal S$ die Gleichheit
\begin{equation}
    \skprod{A(\sigma, t) \eta}{\zeta} = a(\eta, \zeta; \sigma, t) \quad \eta, \zeta \in V.
\end{equation}

Vollkommen analog zur Herleitung der Raum-Zeit-Variationsformulierung in \autoref{sec:raum_zeit_variationsformulierung} erhalten wir damit das folgende parametrische Raum-Zeit-Variationsproblem:

\begin{Definition}
\label{definition:pp:variationsformulierung}
    Seien $\mathcal X$ und $\mathcal Y$ wie in \thref{definition:gl:ansatz_und_testraum}.
    Als \emph{parametrische Raum-Zeit-Variationsfor"-mu"-lie"-rung}
    %der linearen Evolutionsgleichung~\eqref{eq:allgemeine_parabolische_pde}
    bezeichnen wir das folgende Problem:

    Seien ein Quellterm $g \in L_{2}(0, T; V')$ und ein Anfangswert $u_{0} \in H$ gegeben.
    Finde für alle $\sigma \in \mathcal S$ ein $u(\sigma) \in \mathcal X$ mit
    \begin{equation}
        \label{eq:pp:var_all_problem}
        b(u(\sigma), v; \sigma) = f(v) \quad \fa v \in \mathcal Y.
    \end{equation}
    Dabei ist $b \colon \mathcal X \times \mathcal Y \to \mathbb{R}$ die durch
    \begin{equation}
        \label{eq:pp:var_all_bf_b}
        b(u, v; \sigma) = \int_{0}^{T} \skprod{u_{t}(t)}{v_{1}(t)}_{H} + a(u(t), v_{1}(t); \sigma, t) \diff t + \skprod{u(0)}{v_{2}}_{H}
    \end{equation}
    definierte Bilinearform und $f \colon \mathcal Y \to \mathbb{R}$ das durch
    \begin{equation}
        \label{eq:pp:var_all_f}
        f(v) = \int_{0}^{T} \skprod{g(t)}{v_{1}(t)}_{H} \diff t + \skprod{u_{0}}{v_{2}}_{H}
    \end{equation}
    gegebene Funktional.
\end{Definition}

Als nächstes wollen wir nachweisen, dass obiges Raum-Zeit-Variationsproblem sachgemäß gestellt ist und zudem die Lösungen $u(\sigma)$ analytisch vom Parameter $\sigma \in \mathcal S$ abhängen.
Ersteres erhalten wir analog zu \thref{thm:schwab09:theorem51} für den nichtparametrischen Fall.
Bezüglich der Regularität stellt sich heraus, dass wir lediglich Bedingungen an die Familie von stetigen linearen Operatoren $\Set{ A(\sigma, t) \in \mathcal L(V, V') \given \sigma \in \mathcal S, t \in [0, T] }$ stellen müssen, wie folgender Satz zeigt:

\begin{Satz}[{{\cite[Theorem 21]{Kunoth:2013ef}}}]
\label{thm:kunoth:theorem21}
    Seien $\mathcal X$ und $\mathcal Y$ gegeben wie in~\eqref{eq:var_all_ansatzraum_x} respektive~\eqref{eq:var_all_testraum_y}.
    Weiter erfülle die Familie von Operatoren $\Set{ A(\sigma, t) \in \mathcal L(V, V') \given \sigma \in \mathcal S, t \in [0, T] }$ \thref{thm:kunoth:assumption1} für ein $0 < \mathfrak p \leq 1$.
    Für jedes $\sigma \in \mathcal S$ sei $B(\sigma) \in \mathcal L(\mathcal X, \mathcal Y')$ definiert durch
    \begin{equation}
        \label{eq:var_all_gross_b_parametrisch}
        \skprod{B(\sigma) u}{v}_{\mathcal Y' \times \mathcal Y} = b(u, v; \sigma), \quad u \in \mathcal X,~y \in \mathcal Y,
    \end{equation}
    mit $b(\blank, \blank; \sigma)$ wie in~\eqref{eq:pp:var_all_bf_b}.
    Dann ist $B(\sigma)$ für jedes $\sigma \in \mathcal S$ stetig invertierbar und es existieren Konstanten $0 < \beta_{1} \leq \beta_{2} < \infty$ mit
    \begin{equation}
        \label{eq:var_all_norm_B_und_B_inv_parametrisch}
        \sup_{\sigma \in \mathcal S} \norm{B(\sigma)}_{\mathcal L(\mathcal X, \mathcal Y')} \leq \beta_{2} \quad \text{und} \quad  \sup_{\sigma \in \mathcal S} \norm{B(\sigma)^{-1}}_{\mathcal L(\mathcal Y', \mathcal X)} \leq \frac{1}{\beta_{1}}.
    \end{equation}

    Zudem erfüllt die parametrische Familie von Operatoren $\Set{ B(\sigma) \in \mathcal L(\mathcal X, \mathcal Y') \given \sigma \in \mathcal S }$ \thref{thm:kunoth:assumption1} mit dem gleichen Regularitätsparameter $\mathfrak p$, die parametrische Familie von Lösungen $u(\sigma)$ des parametrischen Raum-Zeit-Variationsproblems \eqref{eq:pp:var_all_problem} hängt analytisch von $\sigma$ ab und erfüllt die A-Priori-Abschätzung
    \begin{equation}
        \label{eq:var_all_a_priori_schranke}
        \sup_{\sigma \in \mathcal S} \norm{(\partial^{\nu}_{\sigma} u)(\sigma)}_{\mathcal X} \leq C_{0} \norm{f}_{\mathcal Y'} \abs{\nu}! \tilde{b}^{\nu}
    \end{equation}
    für alle $\nu \in \mathfrak F$, wobei $f$ wie in~\eqref{eq:pp:var_all_f} gegeben ist.
\end{Satz}

\begin{Lemma}
\label{lemma:norm_B_beschraenkt_durch_norm_A}
    Sei $\sigma \in \mathcal S$ und $\nu \in \mathfrak F \setminus \Set{ 0 }$, dann gilt
    \begin{equation}
        \norm{\partial^{\nu}_{\sigma} B(\sigma)}_{\mathcal L(\mathcal X, \mathcal Y')}
        \leq
        \norm{\partial^{\nu}_{\sigma} A(\sigma)}_{\mathcal L(V, V')}
    \end{equation}

    \begin{Beweis}
        \todo[inline]{Beweis}
    \end{Beweis}
\end{Lemma}

\begin{Beweis}[\thref{thm:kunoth:theorem21}]
\todo[inline]{Soll der ausgeführt werden?}
Bedingungen von \thref{thm:kunoth:assumption1} nachrechnen.
Zu (i): Folgt aus \thref{thm:schwab09:theorem51}, da $M_{a}, \alpha, \lambda$ unabhänging von $\sigma$.
Zu (ii): Folgt aus nachfolgendem \thref{lemma:norm_B_beschraenkt_durch_norm_A}.
\end{Beweis}

% section parametrische_lineare_evolutionsgleichung (end)

% chapter parametrisches_problem (end)

\newpage

\todo[inline]{Ebenfalls verarbeiten}

\todo[inline]{Kapitel komplett überarbeiten und am besten nochmal nachrechnen.}

In diesem Kapitel konzentrieren wir uns nun auf die in der Polymerchemie motivierte parabolische partielle Differentialgleichung.
Eine ausführliche Herleitung findet sich bei \textcite{Fredrickson:2006th}.

\section{Motivation} % (fold)
\label{sec:motivation}

\todo[inline]{schreiben!}

% section motivation (end)

\section{Vereinfachte Variante} % (fold)
\label{sec:vereinfachte_variante}

Wir betrachten in diesem Abschnitt zunächst eine vereinfachte Variante der vorgestellten Differentialgleichung.
Zunächst ignorieren wir den Wechsel des Feldes $\omega$ ab einem bestimmten Zeitpunkt und erhalten dadurch einen autonomen linearen Differentialoperator $A$.
Weiter schränken wir uns auf homogene Dirichlet- statt periodischen Randbedingungen ein.

Unter diesen Gegebenheiten bietet es sich an, die Hilberträume als $V = H^{1}_{0}(\Omega)$ und $H = L_{2}(\Omega)$ zu wählen.
Bekanntlich sind diese separabel und es existiert eine dichte stetige Einbettung von $H^{1}_{0}(\Omega)$ in $L_{2}(\Omega)$.
Wegen $(H^{1}_{0}(\Omega))' = H^{-1}(\Omega)$ ergibt dies das Gelfand-Tripel
\begin{equation}
    H^{1}_{0}(\Omega) \denseinclusion L_{2}(\Omega) \denseinclusion H^{-1}(\Omega).
\end{equation}
Wie zuvor verwenden wir $\skprod{\blank}{\blank}$ mit entsprechendem Index sowohl für die Skalarprodukte als auch für die duale Paarung auf $H^{-1}(\Omega) \times H^{1}_{0}(\Omega)$.

Um obige partielle Differentialgleichung in das Setting aus \autoref{sec:lineare_evolutionsgleichungen} zu übertragen, definieren wir einen linearen Operator $A$ als
\begin{equation}
    \label{eq:def_op_A}
    A \colon H^{1}_{0}(\Omega) \to H^{-1}(\Omega), \quad \eta \mapsto A \eta = - c \Delta \eta + \omega \eta
\end{equation}
und weiter die zugehörige Bilinearform
\begin{equation}
    a \colon H^{1}_{0}(\Omega) \times H^{1}_{0}(\Omega) \to \mathbb{R}, \quad a(\eta, \zeta) = \skprod{A \eta}{\zeta}_{L_{2}(\Omega)}.
\end{equation}
Diese lässt sich unter Verwendung der Greenschen Formeln (TODO!) auch schreiben als
\begin{equation}
    \begin{aligned}
        a(\eta, \zeta)
        &= \skprod{- c \Delta \eta + \omega \eta}{\zeta}_{L_{2}(\Omega)}
        = - c \skprod{\Delta \eta}{\zeta}_{L_{2}(\Omega)} + \skprod{\omega \eta}{\zeta}_{L_{2}(\Omega)}
        \\&= c \skprod{\grad \eta}{\grad \zeta}_{L_{2}(\Omega)} + \skprod{\omega \eta}{\zeta}_{L_{2}(\Omega)}.
    \end{aligned}
\end{equation}

Diese Bilinearform ist stetig und erfüllt eine G\aa{}rding-Ungleichung, wie das folgende Lemma zeigt.

\begin{Lemma}
\label{lemma:a_bf_bounded_garding}
    Seien $c \in \mathbb{R}_{+}$, $\omega \in L_{\infty}(\Omega)$ und
    \begin{equation}
    \label{eq:bf_a}
        a \colon H^{1}_{0}(\Omega) \times H^{1}_{0}(\Omega) \to \mathbb{R}, \quad a(\eta, \zeta) = c \skprod{\grad \eta}{\grad \zeta}_{L_{2}(\Omega)} + \skprod{\omega \eta}{\zeta}_{L_{2}(\Omega)}.
    \end{equation}
    Dann erfüllt $a$ die Eigenschaften aus \thref{annahme:eigenschaften_bf_a}:
    \begin{thmenumerate}
        \item\label{lemma:a_bf_bounded_garding:1}
        \emph{Stetigkeit:} es gilt
        \begin{equation}
            \abs{a(\eta, \zeta)} \leq M_{a} \norm{\eta}_{H^{1}(\Omega)} \norm{\zeta}_{H^{1}(\Omega)} \quad \text{für alle}~\eta, \zeta \in H^{1}_{0}(\Omega)
        \end{equation}
        mit $M_{a} = \max\Set{c, \norm{\omega}_{L_{\infty}(\Omega)} } \geq 0$.
        \item\label{lemma:a_bf_bounded_garding:2}
        \emph{G\aa{}rding-Ungleichung:} es gilt
        \begin{equation}
                a(\eta, \eta) + \lambda \norm{\eta}_{L_{2}(\Omega)}^{2} \geq \alpha \norm{\eta}_{H^{1}(\Omega)}^{2} \quad \text{für alle}~\eta \in H^{1}_{0}(\Omega)
        \end{equation}
        mit $\alpha = c \gamma_{\Omega}^{2} > 0$ und $\lambda = \norm{\omega}_{L_{\infty}(\Omega)} \geq 0$, wobei $\gamma_{\Omega}$ die Poincaré-Friedrichs-Konstante ist.
    \end{thmenumerate}

    \begin{Beweis}
    Wir zeigen zunächst die Stetigkeit.
    Seien dazu $\eta, \zeta \in H^{1}_{0}(\Omega)$ beliebig.
    Unter Verwendung der Dreiecks- und der Cauchy-Schwarz-Ungleichung erhalten wir
    \begin{align}
        \abs{a(\eta, \zeta)}
        &= \abs{c \skprod{\grad \eta}{\grad \zeta}_{L_{2}(\Omega)} + \skprod{\omega \eta}{\zeta}_{L_{2}(\Omega)}}
        \\&\leq c \abs{\skprod{\grad \eta}{\grad \zeta}_{L_{2}(\Omega)}} + \abs{\skprod{\omega \eta}{\zeta}_{L_{2}(\Omega)}}
        \\&\leq c \norm{\grad \eta}_{L_{2}(\Omega)} \norm{\grad \zeta}_{L_{2}(\Omega)} + \norm{\omega}_{L_{\infty}(\Omega)} \norm{\eta}_{L_{2}(\Omega)} \norm{\zeta}_{L_{2}(\Omega)}
        \\&\leq \max \Set{ c, \norm{\omega}_{L_{\infty}(\Omega)} } \norm{\eta}_{H^{1}(\Omega)} \norm{\zeta}_{H^{1}(\Omega)}.
    \end{align}

    Für die G\aa{}rding-Ungleichung seien nun $\eta \in H^{1}_{0}(\Omega)$ und $\lambda \in \mathbb{R}$.
    Wir betrachten
    \begin{align}
        a(\eta, \eta) + \lambda \norm{\eta}^{2}_{L_{2}(\Omega)}
        &= c \norm{\grad \eta}^{2}_{L_{2}(\Omega)} + \skprod{\omega \eta}{\eta}_{L_{2}(\Omega)} + \lambda \skprod{\eta}{\eta}_{L_{2}(\Omega)}
        \\&= c \norm{\grad \eta}^{2}_{L_{2}(\Omega)} + \skprod{(\omega + \lambda) \eta}{\eta}_{L_{2}(\Omega)}.
    \end{align}
    Wählen wir nun $\lambda = \norm{\omega}_{L_{\infty}(\Omega)} \geq 0$, dann gilt $\omega + \lambda \geq 0$ fast überall in $\Omega$ und wir erhalten die Abschätzung
    \begin{align}
        a(\eta, \eta) + \lambda \norm{\eta}^{2}_{L_{2}(\Omega)}
        &\geq c \norm{\grad \eta}^{2}_{L_{2}(\Omega)},
        \intertext{woraus wir durch Anwenden der Poincaré-Friedrichs-Ungleichung \ref{satz:grundlagen:poincare_friedrichs_ungleichung}}
        a(\eta, \eta) + \lambda \norm{\eta}^{2}_{L_{2}(\Omega)}
        &\geq c \gamma_{\Omega}^{2} \norm{\eta}^{2}_{H^{1}(\Omega)}
    \end{align}
    folgern.
    \end{Beweis}
\end{Lemma}

Unter diesen Gegebenheiten erhalten wir nach \autoref{sec:lineare_evolutionsgleichungen} eine sachgemäß gestellte Raum-Zeit-Variationsformulierung.
Ansatz- und Testfunktionenraum ergeben sich mit den konkret gewählten Hilberträumen zu
\begin{equation}
    \label{eq:var_ansatzraum_testraum}
    \mathcal X = L_{2}(I; H^{1}_{0}(\Omega)) \cap H^{1}(I; H^{-1}(\Omega))
    \quad \text{und} \quad
    \mathcal Y = L_{2}(I; H^{1}_{0}(\Omega)) \times L_{2}(\Omega).
\end{equation}
Das Variationsproblem lautet damit:
    Gegeben ein $g \in L_{2}(I; H^{-1}(\Omega))$ und ein $u_{0} \in L_{2}(\Omega)$. Finde ein $u \in \mathcal X$ mit
    \begin{equation}
        \label{eq:varprob}
        b(u, v) = f(v) \quad \text{für alle}~v \in \mathcal Y,
    \end{equation}
    wobei $b(\blank, \blank) \colon \mathcal X \times \mathcal Y \to \mathbb{R}$ die durch
    \begin{equation}
        \label{eq:buv}
        b(u, v)
            = \int_{I} \skprod{u_{t}(t)}{v_{1}(t)}_{L_{2}(\Omega)} + a(u(t), v_{1}(t)) \diff t + \skprod{u(0)}{v_{2}}_{L_{2}(\Omega)}
    \end{equation}
    gegebene Bilinearform und $f(\blank) \colon \mathcal Y \to \mathbb{R}$ definiert ist durch
    \begin{equation}
        \label{eq:var_all_f_wiederholung}
        f(v) = \int_{I} \skprod{g(t)}{v_{1}(t)}_{L_{2}(\Omega)} \diff t + \skprod{u_{0}}{v_{2}}_{L_{2}(\Omega)}.
    \end{equation}

Aus \thref{thm:schwab09:theorem51} und \thref{thm:schwab09:theorem51:ungleichungen} erhalten wir nun die Wohldefiniertheit des obigen Variationsproblems und zugleich Schranken für die Operatoren.

\begin{Korollar}
\label{korollar:2.2}
    Seien $\mathcal X$ und $\mathcal Y$ gegeben wie in \eqref{eq:var_ansatzraum_testraum} und sei $B \colon \mathcal X \to \mathcal Y'$ definiert durch
    \begin{equation}
        \skprod{Bu}{v}_{\mathcal Y' \times \mathcal Y}  = b(u, v), \quad u \in \mathcal X,~ v \in \mathcal Y,
    \end{equation}
    mit $b(\blank, \blank)$ wie in \eqref{eq:buv}.
    Dann ist $B$ stetig invertierbar und es gilt
    \begin{equation}
        \norm{B}_{\mathcal L(\mathcal X, \mathcal Y')}
        \leq
        \frac{\sqrt{2 \max\Set{1, c^{2}, \norm{\omega}_{L_{\infty}(\Omega)}^{2}} + M_{e}^{2}}}{\max\Set{\sqrt{1 + 2 \norm{\omega}_{L_{\infty}(\Omega)}^{2} \rho^{4}}, \sqrt{2} }}
    \end{equation}
    und
    \begin{equation}
        \norm{B^{-1}}_{\mathcal L( \mathcal Y', \mathcal X)}
        \leq \frac{e^{2 T \norm{\omega}_{L_{\infty}(\Omega)}} \max\Set{\sqrt{1 + 2 \norm{\omega}_{L_{\infty}(\Omega)}^{2} \rho^{4}}, \sqrt{2}} \sqrt{2 \max\Set{c^{-2} \gamma_{\Omega}^{-4}, 1} + M_{e}^{2}}}{\min\Set{c^{-1} \gamma_{\Omega}^{2}, c \gamma_{\Omega}^{2} \norm{\omega}_{L_{\infty}(\Omega)}^{-2}, c \gamma_{\Omega}^{2} }}.
        % \leq
        % \frac{\max\{\sqrt{ 1 + 2 \norm{\omega}_{L_{\infty}(\Omega)} \rho^{4}}, \sqrt{2} \}}{e^{-2 \norm{\omega}_{L_{\infty}(\Omega)} T}}
        % \frac{\sqrt{2 \max\{ 1, \sigma^{-2} \gamma_{\Omega}^{-4} \} + M_{e}^{2}}}{\min\{ \sigma \gamma_{\Omega}^{2} \norm{\omega}_{L_{\infty}(\Omega)}^{-2}, \sigma \gamma_{\Omega}^{2} \}}
    \end{equation}
    mit $M_{e}$ und $\rho$ wie in \eqref{eq:var_all_M_e} respektive \eqref{eq:var_all_rho}.
\end{Korollar}

% section vereinfachte_variante (end)

\section{Parametrische Variante} % (fold)
\label{sec:parametrische_variante}

Wir wollen nun aus dem gerade beschriebenen Variationsproblem eine parametrische Variante gewinnen und aufbauend auf \autoref{sec:parametrisches_problem} Regularität bezüglich des Parameters folgern.
Dazu müssen wir den Operator $A \in \mathcal L(V, V')$ aus \eqref{eq:def_op_A} zunächst zu einem parametrischen Operator $A(\sigma)$ mit $\sigma \in \mathcal S$, wobei $\mathcal S \subset \mathbb{R}^{\mathbb{N}}$ ein geeigneter Parameterraum ist, umschreiben.
Dabei beschränken wir uns auf den Fall affiner parametrischer Abhängigkeit \eqref{eq:all_affiner_operator}.
Der Einfachheit halber wählen wir $\mathcal S = [-1, 1]^{\mathbb{N}}$, das heißt $\mathcal S$ sei die Einheitskugel aus $\ell_{\infty}(\mathbb{N})$.

Sei $\Set{ \varphi_{j} }_{j \in \mathbb{N}} \subset L_{\infty}(\Omega)$ ein noch näher zu bestimmendes, passend gewähltes Funktionensystem und $\sigma \in \mathcal S$.
Wir entwickeln nun $\omega$ formal in eine Reihe der Form
\begin{equation}
    \label{eq:reihenentwicklung_omega}
    \omega(\blank; \sigma) = \sum_{j = 1}^{\infty} \sigma_{j} \varphi_{j}.
\end{equation}
Offenbar ist für die Konvergenz der Reihe \eqref{eq:reihenentwicklung_omega} hinreichend, dass $\Set{ \norm{\varphi_{j}}_{L_{\infty}(\Omega)} }_{j \in \mathbb{N}} \in \ell_{1}(\mathbb{N})$ gilt, insbesondere folgt daraus
\begin{equation}
    \norm{\omega(\blank; \sigma)}_{L_{\infty}(\Omega)} \leq \sum_{j = 1}^{\infty} \norm{\varphi_{j}}_{L_{\infty}(\Omega)} < \infty \quad \fa \sigma \in \mathcal S.
\end{equation}
% Diese Eigenschaft wird auch benötigt, denn dadurch erhalten wir aus \thref{lemma:2.2} die für \thref{thm:kunoth:theorem21} notwendigen, von $\sigma$ unabhängigen, Schranken $\beta_{1}$ und $\beta_{2}$.
Damit ist die Wahl des Funktionensystems $\Set{ \varphi_{j} }_{j \in \mathbb{N}} \subset L_{\infty}(\Omega)$ ist entscheidend für die Konvergenz von \eqref{eq:reihenentwicklung_omega}, aber auch für die Erfüllbarkeit von \thref{thm:kunoth:assumption1} respektive \thref{thm:kunoth:assumption2},
und wird in den nächsten Abschnitten genauer behandelt.

% \subsection{Affiner Operator} % (fold)
% \label{ssub:entwicklung_von_}

Wir wollen den Operator $A$ aus \eqref{eq:def_op_A} als affin parametrischen Operator der Form
\begin{equation}
    \label{eq:aff_zerlegung_A}
    A(\sigma) = \hat A + \sum_{j \geq 1} \sigma_{j} A_{j}
\end{equation}
auffassen, beziehungsweise als Bilinearformen
\begin{equation}
     \label{eq:aff_zerelgung_A_bf}
     a(\eta, \zeta; \sigma) = \hat a(\eta, \zeta) + \sum_{j \geq 1} \sigma_{j} a_{j}(\eta, \zeta), \quad \eta, \zeta \in V.
 \end{equation}
Dazu entwickeln wir $\omega$ in eine Reihe der Form \eqref{eq:reihenentwicklung_omega}, das heißt wir erhalten
\begin{equation}
    \label{eq:omega_reihenentwicklung}
    \omega(\blank; \sigma) \colon \Omega \to \mathbb{R}, \quad x \mapsto \omega(x; \sigma) = \sum_{j \geq 1} \sigma_{j} \varphi_{j}(x)
\end{equation}
mit $\sigma \in \mathcal S$.
Eine naheliegende affine Aufteilung des Operators $A$ erhalten wir damit durch die Wahl
\begin{equation}
    \label{eq:affine_zerlegung_A_def}
    \hat A = - c \Delta, \qquad
    A_{j} = \varphi_{j}, \quad j \geq 1.
\end{equation}
Die zugehörigen Bilinearformen lassen sich ebenfalls direkt angeben, denn es gilt
\begin{equation}
    \hat a(\eta, \zeta) = \skprod{\grad \eta}{\grad \zeta}_{L_{2}(\Omega)}, \qquad a_{j}(\eta, \zeta) = \skprod{\varphi_{j} \eta}{\zeta}_{L_{2}(\Omega)}, \quad j \geq 1.
\end{equation}

Die daraus resultierende Raum-Zeit-Variationsformulierung lautet nun:
\begin{Problem}
    Gegeben ein $g \in L_{2}(I; H^{-1}(\Omega))$ und ein $u_{0} \in L_{2}(\Omega)$.
    Finde für alle $\sigma \in \mathcal S$ ein $u(\sigma) \in \mathcal X$ mit
    \begin{equation}
        \label{eq:varprob_2}
        b(u, v; \sigma) = f(v) \quad \text{für alle}~v \in \mathcal Y,
    \end{equation}
    wobei $b(\blank, \blank; \sigma) \colon \mathcal X \times \mathcal Y \times \mathcal S \to \mathbb{R}$ die durch
    \begin{equation}
        \label{eq:buv_2}
        b(u, v; \sigma)
            = \int_{I} \skprod{u_{t}(t)}{v_{1}(t)}_{L_{2}(\Omega)} + a(u(t), v_{1}(t); \sigma) \diff t + \skprod{u(0)}{v_{2}}_{L_{2}(\Omega)}
    \end{equation}
    gegebene Bilinearform und $f(\blank) \colon \mathcal Y \to \mathbb{R}$ definiert ist durch
    \begin{equation}
        \label{eq:var_all_f_wiederholung_2}
        f(v) = \int_{I} \skprod{g(t)}{v_{1}(t)}_{L_{2}(\Omega)} \diff t + \skprod{u_{0}}{v_{2}}_{L_{2}(\Omega)}.
    \end{equation}
\end{Problem}


% \section{Periodische Randbedingungen} % (fold)
% \label{sec:periodische_randbedingungen}

% section periodische_randbedingungen (end)

    %!TEX root = ../main.tex

\setchapterpreamble[ul][0.6\textwidth]{%
    \dictum[Robert Heinlein, \textit{Time Enough For Love}]{\enquote{Progress isn't made by early risers. It's made by lazy men trying to find easier ways to do something.}}
    \vspace*{2\baselineskip}
}
\chapter{Der eindimensionale Fall}
\label{sec:der_eindimensionale_fall}
\label{cha:der_eindimensionale_fall}

\todo[inline]{Kapitel ordentlich überarbeiten. Mittlerweile besser, aber noch deutlich verbesserungswürdig!}
\todo[inline]{Sind ein paar notationelle Kleinigkeiten zu korrigieren, also mal ordentlich durchlesen!}
\todo[inline]{Die Abschätzung in Satz 4.4. ist schon heftig. Da bleibt ja so gut wie kein Spielraum für den Faktor K...}

In diesem Kapitel beschränken wir uns zunächst auf den vereinfachten Fall einer Raumdimension.
Es sei also $\Omega \subset \mathbb{R}$ ein Intervall.
Ohne Beschränkung der Allgemeinheit wählen wir $\Omega = [0, L]$ für ein $0 < L < \infty$.

\todo[inline]{Eventuell wäre, insbesondere mit der gewünschten Anwendung bei SCFT-Verfahren, eine Entwicklung in Cosinus-Funktionen (oder direkt Fourier-Reihen) sinnvoller.
Insbesondere wären Cosinus-Funktionen achsensymmetrisch und automatisch nicht-homogen; beides Eigenschaften, welche die Felder $\omega$ bei der SCFT oftmals aufweisen.}


Als Ansatzfunktionen für eine geeignete Entwicklung des Operators $A$ in eine affin parametrische Darstellung wählen wir Sinusfunktionen, welche zusätzlich gewichtet werden, so dass wir die gewünschten Konvergenzeigenschaften erhalten.
Zusätzlich zu den Sinusfunktionen fügen wir eine konstante Funktion $\varphi_{0}$ hinzu, um so inhomogene Randbedingungen für $\omega$ zuzulassen.

Da die Parameter $\sigma$ weiterhin aus der Menge $\mathcal S = [-1, 1]^{\mathbb{N}}$ kommen, skalieren wir die Ansatzfunktionen mit einem Faktor $K \in \mathbb{R}_{+}$.
Dieser Faktor wird später durch die Anforderungen, die wir durch die gewünschte analytische Abhängigkeit vom Parameter erhalten, genauer bestimmt.

Konkret kommt nun
\begin{equation}
    \label{eq:sinusfunktionen_ansatz}
    \varphi_{0} = K, \qquad
    \varphi_{j} = \frac{K}{(\pi j)^{1 + \epsilon}} \sin(\tfrac{\pi j}{L} \blank), \quad j \geq 1,
\end{equation}
als Funktionensystem $\Set{ \varphi_{j} }_{j \geq 0}$ zum Einsatz.
Damit schreiben wir den parametrischen Faktor $\omega$ der linearen Evolutionsgleichung aus dem vorherigen Kapitel als
\begin{equation}
    w(\blank; \sigma) \colon \Omega \to \mathbb{R}, \quad w(x; \sigma) = \sum_{j = 0}^{\infty} \sigma_{j} \varphi_{j}(x)
\end{equation}
mit $\sigma \in \mathcal S$.
Damit schreibt sich die affine Zerlegung des Differentialoperators $A = - c \Delta + \omega$ als
\begin{equation}
    A = \hat A + \sum_{j = 0}^{\infty} \sigma_{j} A_{j}
\end{equation}
mit
\begin{equation}
    \label{eq:1d:affine_zerlegung}
    \hat A = - c \Delta, \qquad A_{j} = \varphi_{j}
\end{equation}
und den zugehörigen Bilinearformen
\begin{equation}
    \hat a(\eta, \zeta) = c \skprod{\grad \eta}{\grad \zeta}_{L_{2}(\Omega)}, \qquad a_{j}(\eta, \zeta) = \skprod{\varphi_{j} \eta}{\zeta}_{L_{2}(\Omega)}.
\end{equation}


Zunächst rechnen wir nun verschiedene, später benötigte, Normen nach.
\begin{Lemma}
    Es gilt
    \begin{alignat}{2}
        \norm{\varphi_{0}}_{L_{\infty}(\Omega)} &= K,
        \qquad&
        \norm{\varphi_{j}}_{L_{\infty}(\Omega)} &= \frac{K}{(\pi j)^{1 + \epsilon}} , \quad j \geq 1,
    \intertext{sowie}
        \norm{\varphi_{0}}_{H^{1}(\Omega)}  &= K \sqrt{L},
        \qquad&
        \norm{\varphi_{j}}_{H^{1}(\Omega)}  &= \frac{K \sqrt{L^{2} + (\pi j)^{2}}}{\sqrt{2L} (\pi j)^{1 + \epsilon}}
        , \quad j \geq 1.
    \end{alignat}
\end{Lemma}

\begin{Lemma}
    Die Funktionen $\Set{ \varphi_{j} }_{j \geq 1}$ bilden ein Orthogonalsystem in $H^{1}(\Omega)$, denn es gilt
    \begin{equation}
        \skprod{\varphi_{j}}{\varphi_{k}}_{H^{1}(\Omega)} = \begin{cases}
            \frac{K^{2}}{2L} \frac{L^{2} + (\pi j)^{2}}{(\pi j)^{2(1 + \epsilon)}}
            ,   &j = k \\
            0,          &j \neq k.
        \end{cases}
    \end{equation}
\end{Lemma}

\begin{Lemma}
    Sei $\sigma \in \mathcal S$ und $\epsilon > 0$, dann konvergiert obiges $\omega(\blank; \sigma)$ in $L_{\infty}(\Omega)$.
    Ist $\epsilon > 1$, dann gilt auch Konvergenz in $H^{1}_{0}(\Omega)$.

    \begin{Beweis}
        Sei zunächst $\epsilon > 0$.
        Da $\mathcal S = [0, 1]^{\mathbb{N}}$ ist, erhalten wir die Konvergenz in $L_{\infty}(\Omega)$ nach dem Weierstraßschen Majorantenkriterium via
        \begin{align}
            \sum_{j = 0}^{\infty} \norm{\sigma_{j} \varphi_{j}}_{L_{\infty}(\Omega)}
            &= \sum_{j = 0}^{\infty} \abs{\sigma_{j}} \norm{\varphi_{j}}_{L_{\infty}(\Omega)}
             \leq \norm{\varphi_{0}}_{L_{\infty}(\Omega)} + \sum_{j = 1}^{\infty}  \norm{\varphi_{j}}_{L_{\infty}(\Omega)}
            \\&= K + \sum_{j = 1}^{\infty} \frac{K}{(\pi j)^{1 + \epsilon}}
            = K + \frac{K}{\pi^{1 + \epsilon}} \sum_{j = 1}^{\infty} \frac{1}{j^{1+\epsilon}}
        \end{align}
        Diese Reihe konvergiert bekanntlich für alle $\epsilon > 0$, womit wir bereits die Konvergenz von $\omega$ in $L_{\infty}(\Omega)$ erhalten.

        Sei nun $\epsilon > 1$.
        Betrachte
        \begin{align}
            \sum_{j = 0}^{\infty} \norm{\sigma_{j} \varphi_{j}}_{H^{1}(\Omega)}
            &= \sum_{j = 0}^{\infty} \abs{\sigma_{j}} \norm{\varphi_{j}}_{H^{1}(\Omega)}
            \leq  \norm{\varphi_{0}}_{H^{1}(\Omega)} + \sum_{j = 1}^{\infty} \norm{\varphi_{j}}_{H^{1}(\Omega)}
            \\&= K \sqrt{L} + \sum_{j = 1}^{\infty} \frac{K \sqrt{L^{2} + (\pi j)^{2}}}{\sqrt{2L} (\pi j)^{1 + \epsilon}}
            \\&= K \sqrt{L} + \frac{K}{\sqrt{2L}} \sum_{j = 1}^{\infty} \frac{\sqrt{L^{2} + (\pi j)^{2}}}{(\pi j)^{1 + \epsilon}}
            \\&\leq K \sqrt{L} + \frac{K}{\sqrt{2L}} \sum_{j = 1}^{\infty} \frac{L + \pi j}{(\pi j)^{1 + \epsilon}}
            \\&= K \sqrt{L} + \frac{K}{\sqrt{2L}} \sum_{j = 1}^{\infty} \frac{L}{(\pi j)^{1 + \epsilon}} + \frac{K}{\sqrt{2L}} \sum_{j = 1}^{\infty} \frac{1}{(\pi j)^{\epsilon}}
        \end{align}
        Wegen $\epsilon > 1$ konvergiert sowohl die erste als auch die zweite Reihe.
        Zusammen liefert dies die Konvergenz in $H^{1}_{0}(\Omega)$.
    \end{Beweis}
\end{Lemma}

Wir wollen nun die Regularität von $\omega(\blank; \sigma)$ in Abhängigkeit vom Parameter $\sigma$ nachweisen.

\todo[inline]{Anpassen! Lässt sich die Schranke verbessern?}
\begin{Satz}
\label{satz:regularitaet_nachrechnen}
    Seien $\epsilon > 0$ und $0 < \kappa < 1$ so gewählt, dass
    \begin{equation}
        \sum_{j = 1}^{\infty} \frac{1}{j^{2 + \epsilon}} \leq \frac{(\kappa c (\tfrac{\pi}{L})^{2} - K) \pi^{2 + \epsilon}}{4 C_{\infty} K L^{3/2}}
    \end{equation}
    gilt,
    wobei $c$ und $K$ die Konstanten aus \eqref{eq:def_op_A} respektive \eqref{eq:sinusfunktionen_ansatz} und $C_{\infty}$ die Einbettungskonstante von $H^{1}_{0}(\Omega) \hookrightarrow L_{\infty}(\Omega)$ sind.
    Dann erfüllt die affine Zerlegung $\Set{\hat A, A_{j} \given j \in \mathbb{N}_{0}}$ \thref{thm:kunoth:assumption2}.

    \begin{Beweis}
        Wir weisen zunächst die inf-sup-Bedingungen \eqref{eq:kunoth:ass2_gamma_0} für $\hat a(\blank, \blank)$ nach und bestimmen die Konstante $\gamma_{0}$.
        Da $\hat a(\blank, \blank)$ symmetrisch ist, genügt es, die inf-sup-Bedingung \eqref{eq:kunoth:ass2_gamma_0_a} nachzuweisen. Die zweite inf-sup-Bedingung \eqref{eq:kunoth:ass2_gamma_0_b} folgt dann analog mit dem selben $\gamma_{0}$.

        Nach \thref{lemma:sauter:2.1.48} reicht es, für alle $\eta \in H^{1}_{0}(\Omega)$ ein $\zeta = \zeta(\eta) \in H^{1}_{0}(\Omega)$ und von $\eta$ und $\zeta$ unabhängige Konstanten $C_{1}, C_{2} > 0$ mit
        \begin{equation}
            \hat a(\eta, \zeta) \geq C_{1} \norm{\eta}_{H^{1}(\Omega)}^{2} \quad \text{und} \quad \norm{\zeta}_{H^{1}(\Omega)} \leq C_{2} \norm{\eta}_{H^{1}(\Omega)}
        \end{equation}
        zu finden.
        Dann ist die inf-sup-Bedingung \eqref{eq:kunoth:ass2_gamma_0_a} mit $\gamma_{0} = \frac{C_{1}}{C_{2}}$ erfüllt.

        Sei nun also $\eta \in H^{1}_{0}(\Omega)$ beliebig.
        Wir wählen $\zeta = \eta \in H^{1}_{0}(\Omega)$, das heißt, es gilt $C_{2} = 1$.
        Es ergibt sich
        \begin{align}
            \hat a(\eta, \zeta) = \hat a(\eta, \eta) = c \skprod{\grad \eta}{\grad \eta}_{L_{2}(\Omega)} = c \norm{\grad \eta}_{L_{2}(\Omega)}^{2} \geq c \gamma_{\Omega}^{2} \norm{\eta}_{H^{1}(\Omega)}^{2},
        \end{align}
        wobei die letzte Abschätzung aus der Poincaré-Friedrichs-Ungleichung \eqref{eq:grundlagen:poincare_friedrichs_ungleichung} folgt.
        Zusammen liefert dies $\gamma_{0} = c \gamma_{\Omega}^{2}$ als inf-sup-Konstante.

        Für den vorliegenden Fall können wir $\gamma_{\Omega}^{2}$ exakt bestimmen.
        Nach \cite[Chapter 11]{Strauss:2007vz} entspricht das Quadrat der optimalen Poincaré-Friedrichs-Konstante $\gamma_{\Omega}^{2}$ gerade dem kleinsten Eigenwert des Laplace-Operators auf $\Omega$ mit Dirichlet-Randbedingung.
        Dieser hat für $\Omega = [0, L]$ den Wert $\frac{\pi^{2}}{L^{2}}$.
        Wir erhalten damit also $\gamma_{0} = c \frac{\pi^{2}}{L^{2}}$.

        Seien nun $\eta, \zeta \in H^{1}_{0}(\Omega)$.
        Für $j = 0$ gilt die simple Abschätzung
        \begin{equation}
            \begin{aligned}
                a_{0}(\eta, \zeta)
                &= \skprod{\varphi_{0} \eta}{\zeta}_{L_{2}(\Omega)}
                = K \skprod{\eta}{\zeta}_{L_{2}(\Omega)}
                \\&\leq K \norm{\eta}_{L_{2}(\Omega)} \norm{\zeta}_{L_{2}(\Omega)}
                \leq K \norm{\eta}_{H^{1}(\Omega)} \norm{\zeta}_{H^{1}(\Omega)}
            \end{aligned}
        \end{equation}
        Betrachte für $j \geq 1$
        \begin{align}
            a_{j}(\eta, \zeta)
            &= \skprod{\varphi_{j} \eta}{\zeta}_{L_{2}(\Omega)}
            = \int_{0}^{L} \varphi_{j} \eta \zeta \diff x
            \intertext{da $\varphi_{j}$ integrierbar ist und $\varphi_{j}(0) = 0$, können wir dies umschreiben zu}
            a_{j}(\eta, \zeta)
            &= \int_{0}^{L} \frac{\diff}{\diff x} \left( \int_{0}^{x} \varphi_{j}(y) \diff y \right) \eta \zeta \diff x
            \intertext{woraus wir mittels partieller Integration und $\eta, \zeta \in H^{1}_{0}(\Omega)$ folgenden Ausdruck erhalten}
            a_{j}(\eta, \zeta)
            &= - \int_{0}^{L} \left( \int_{0}^{x} \varphi_{j}(y) \diff y \right) (\eta \zeta)' \diff x
            \leq \norm*{\left( \int_{0}^{x} \varphi_{j}(y) \diff y \right) (\eta \zeta)'}_{L_{1}(\Omega)}
            \\&\leq \norm*{\int_{0}^{x} \varphi_{j}(y) \diff y }_{L_{\infty}(\Omega)} \norm{(\eta \zeta)'}_{L_{1}(\Omega)}.
        \end{align}
        Die erste Norm können wir weiter abschätzen mit
        \begin{equation}
            \begin{aligned}
                \norm*{\int_{0}^{x} \varphi_{j}(y) \diff y }_{L_{\infty}(\Omega)}
                &= \norm*{\int_{0}^{x} \frac{K}{(\pi j)^{1 + \epsilon}} \sin(\tfrac{\pi j}{L} y) \diff y}_{L_{\infty}(\Omega)}
                \\&= \norm*{\frac{K L}{(\pi j)^{2 + \epsilon}} \left( 1 - \cos(\tfrac{\pi j}{L} x) \right) }_{L_{\infty}(\Omega)}
                \leq \frac{2 K L}{(\pi j)^{2 + \epsilon}}
            \end{aligned}
        \end{equation}
        Aus der zweiten Norm erhalten wir mittels Minkowski- und Hölderungleichung sowie der Einbettung $H^{1}_{0}(\Omega) \hookrightarrow L_{\infty}(\Omega)$ die Abschätzung
        \begin{equation}
            \begin{aligned}
                \norm{(\eta \zeta)'}_{L_{1}(\Omega)}
                &= \norm{\eta' \zeta + \eta \zeta'}_{L_{1}(\Omega)}
                \leq \norm{\eta' \zeta}_{L_{1}(\Omega)} + \norm{\eta \zeta'}_{L_{1}(\Omega)}
                \\&\leq \norm{\eta'}_{L_{1}(\Omega)} \norm{\zeta}_{L_{\infty}(\Omega)} + \norm{\eta}_{L_{\infty}(\Omega)} \norm{\zeta'}_{L_{1}(\Omega)}
                \\&\leq \norm{1}_{L_{2}(\Omega)} \norm{\eta'}_{L_{2}(\Omega)} \norm{\zeta}_{L_{\infty}(\Omega)} + \norm{\eta}_{L_{\infty}(\Omega)} \norm{1}_{L_{2}(\Omega)} \norm{\zeta'}_{L_{2}(\Omega)}
                \\&\leq \sqrt{L} C_{\infty} \norm{\eta'}_{L_{2}(\Omega)} \norm{\zeta}_{H^{1}(\Omega)} + \sqrt{L} C_{\infty} \norm{\eta}_{H^{1}(\Omega)} \norm{\zeta'}_{L_{2}(\Omega)}
                % \\&\leq \norm{\eta'}_{L_{2}(\Omega)} \norm{\zeta}_{L_{2}(\Omega)} + \norm{\eta}_{L_{2}(\Omega)} \norm{\zeta'}_{L_{2}(\Omega)}
                \\&\leq 2 \sqrt{L} C_{\infty} \norm{\eta}_{H^{1}(\Omega)} \norm{\zeta}_{H^{1}(\Omega)}
            \end{aligned}
        \end{equation}
        Zusammen also
        \begin{align}
            a_{j}(u, v)
            &\leq \norm*{\int_{0}^{x} \varphi_{j}(y) \diff y }_{L_{\infty}(\Omega)} \norm{(\eta \zeta)'}_{L_{1}(\Omega)}
            \\&\leq \frac{4 K L^{3 / 2} C_{\infty}}{(\pi j)^{2 + \epsilon}} \norm{\eta}_{H^{1}(\Omega)} \norm{\zeta}_{H^{1}(\Omega)}.
        \end{align}
        Betrachte nun
        \begin{align}
                    \sum_{j = 0}^{\infty} \norm{A_{j}}_{\mathcal L(V, V')}
            &= \norm{A_{0}}_{\mathcal L(V, V')} + \sum_{j = 1}^{\infty} \norm{A_{j}}_{\mathcal L(V, V')}
            \\&\leq K + \sum_{j = 1}^{\infty} \frac{4 K L^{3 / 2} C_{\infty}}{(\pi j)^{2 + \epsilon}}
            \\&\leq K + \frac{4 K L^{3 / 2} C_{\infty}}{\pi^{2+ \epsilon}} \sum_{j = 1}^{\infty} \frac{1}{j^{2 + \epsilon}}
        \end{align}
        Fordern wir nun die Gültigkeit von \eqref{eq:kunoth:ass2_abs_reihe}, also
        \begin{equation}
            \sum_{j \geq 0} \norm{A_{j}}_{\mathcal L(V, V')} \leq \kappa \gamma_{0}
        \end{equation}
        für ein $0 < \kappa < 1$, dann ist damit also
        \begin{equation}
            \sum_{j = 1}^{\infty} \frac{1}{j^{2 + \epsilon}} \leq \frac{(\kappa (\tfrac{\pi}{L})^{2} - K) \pi^{2+ \epsilon}}{4 K L^{3/2} C_{\infty}}
        \end{equation}
        mit $\epsilon > 0$ hinreichend.
    \end{Beweis}
\end{Satz}

Zusammenfassend erhalten wir damit die folgende Aussage.

\todo[inline]{Besser ausformulieren}
\begin{Satz}
    Seien $\mathcal X$ und $\mathcal Y$ gegeben wie in~\eqref{eq:var_ansatzraum_testraum}.
    Weiter sei $\Set{\hat A, A_{j} \given j \in \mathbb{N}_{0}}$ die affine Zerlegung von $A = -c \Delta + \omega$ wie in \eqref{eq:1d:affine_zerlegung} und es gelte \thref{satz:regularitaet_nachrechnen}.
    Für jedes $\sigma \in \mathcal S$ sei $B(\sigma) \in \mathcal L(\mathcal X, \mathcal Y')$ definiert durch
    \begin{equation}
        \label{eq:var_all_gross_b_parametrisch}
        \skprod{B(\sigma) u}{v}_{\mathcal Y' \times \mathcal Y} = b(u, v; \sigma), \quad u \in \mathcal X,~y \in \mathcal Y,
    \end{equation}
    mit $b(\blank, \blank; \sigma)$ wie in~\eqref{eq:buv_2}.
    Dann ist $B(\sigma)$ für jedes $\sigma \in \mathcal S$ stetig invertierbar und die parametrische Familie von Lösungen $u(\sigma)$ des parametrischen Raum-Zeit-Variationsproblems \eqref{eq:varprob_2} hängt analytisch von $\sigma$ ab.

    \begin{Beweis}
        Direkte Folgerung aus \thref{satz:regularitaet_nachrechnen} und \thref{thm:kunoth:theorem21}.
    \end{Beweis}
\end{Satz}

% subsection nachrechnen_von_thref_thm_kunoth_assumption2 (end)

% section der_eindimensionale_fall (end)

\clearpage
\section{Zu klärende Fragen} % (fold)
\label{sub:zu_kl_rende_fragen}

\begin{enumerate}
    \item Wohldefiniertheit der PDE \eqref{eq:parabolische_pde}, das heißt die Voraussetzungen von \thref{thm:schwab09:theorem51} nachweisen. Weiterhin lassen sich damit die inf-sup-Bedingung von \eqref{eq:varprob} nachrechnen und damit die Schranken für $B$ und $B^{-1}$ bestimmen.
    \item Parametrische Variante des Variationsproblems herleiten.
    Dazu Ansetzen mit Entwicklung des Parameters $\omega$ in eine Reihe
    \begin{equation}
        \omega = \sum_{j = 0}^{\infty} \sigma_{j} \varphi_{j}.
    \end{equation}
    Dabei ergeben sich folgende Fragen:
    \begin{enumerate}
        \item Konvergenz der Reihe? Notwendig ist Konvergenz in $L_{\infty}(\Omega)$, da die Norm $\norm{\omega}_{L_{\infty}(\Omega)}$ mehrfach in Abschätzungen verwendet wird.
        \item Weiterhin ist eventuell Konvergenz in einem Unterraum $Z \hookrightarrow L_{\infty}(\Omega)$ wünschenswert.
        Zum Beispiel in $H^{1}(\Omega)$?
        \item Welche Bedingungen ergeben sich an $\sigma_{j}$ und $\varphi_{j}$?
        \item Welches Funktionensystem $\Set{ \varphi_{j} }_{j}$ ist überhaupt sinnvoll?
        Die Wahl der $\varphi_{j}$ entscheidet maßgeblich über Konvergenz der Reihenentwicklung.
        Welche Randvorgaben sind angestrebt?
        Dies wird ebenfalls durch die $\varphi_{j}$ geregelt.
    \end{enumerate}
    \item Welche affine Zerlegung $A(\sigma) = A_{0} + \sum_{j} \sigma_{j} A_{j}$ ist brauchbar?
    Wie genau sehen die $A_{j}$ aus?
    \item Nachweisen, dass $A(\sigma)$ \thref{thm:kunoth:assumption1} oder \thref{thm:kunoth:assumption2} erfüllt und mittels \thref{thm:kunoth:theorem21} die gewünschte Regularität von $B(\sigma)$ bezüglich $\sigma$ gewinnen.
    \item Die Abschätzungen in \thref{satz:regularitaet_nachrechnen} lassen sich noch deutlich verbessern.
    Das gilt wahrscheinlich auch für andere Abschätzungen!
\end{enumerate}


    %!TEX root = ../main.tex

\setchapterpreamble[ul][0.6\textwidth]{%
    \dictum[Alfréd Rényi]{\enquote{A mathematician is a device for turning coffee into theorems.}}
    \vspace*{2\baselineskip}
}
\chapter{Reduzierte-Basis-Methode} % (fold)
\label{cha:reduzierte_basis_methode}

\todo[inline]{Erklärung schreiben}

\blindtext

% chapter reduzierte_basis_methode (end)

    %!TEX root = ../main.tex

\chapter{Ausblick} % (fold)
\label{cha:ausblick}

% chapter ausblick (end)


    %%% Anhang
    \appendix{}

    % %!TEX root = ../main.tex

\pagestyle{plain}

\section*{Eindimensionaler Fall mit $\omega \in \mathbb{R}$ und ohne Quellterm}

Sei $I := [0, \hat t]$ für ein $0 < \hat t < \infty$ und $\Omega := [0, 1]$.
Betrachte folgende parametrisierte PDE
\begin{align}
    \begin{cases}
    u_{t}(t, x) = \sigma u_{xx}(t, x) - \omega u(t, x), & (t, x) \in I \times \Omega\\
    u(0, x) = g(x), & x \in \Omega \\
    u(t, 0) = u(t, 1) = 0, & t \in I
    \end{cases}
\end{align}
mit Konstanten $\sigma, \omega \in \mathbb{R}$.

Ein Separation der Variablen Ansatz $u(t, x) = X(x) T(t)$ liefert
\begin{equation}
    X(x)T'(t) = \sigma X''(x) T(t) - \omega X(x) T(t)
\end{equation}
oder auch
\begin{equation}
    \frac{T'(t)}{T(t)} = \sigma \frac{X''(x)}{X(x)} - \omega = \lambda
\end{equation}
mit $\lambda \in \mathbb{R}$.

Ohne Einschränkung sei $\lambda \neq 0$, dann erhalten wir zum einen die Dgl.
    $T'(t) = \lambda T(t)$,
deren Lösung
\begin{equation}
    T(t) = d_{3} e^{\lambda t}
\end{equation}
ist, und zum anderen die Dgl.
    $X''(x) =  \frac{\lambda + \omega}{\sigma} X(x)$
mit der Lösung
\begin{equation}
    X(x) = d_{1} e^{\sqrt{\frac{\lambda + \omega}{\sigma}} x} + d_{2} e^{-\sqrt{\frac{\lambda + \omega}{\sigma}}x},
\end{equation}
wobei $d_{1}, d_{2}, d_{3} \in \mathbb{R}$.

Als nächstes Verwenden wir die Anfangs- und Randbedingungen um die Konstanten $d_{i}$ zu bestimmen.
Sei
\begin{equation}
    u(t, x) = \left( d_{1} e^{\sqrt{\frac{\lambda + \omega}{\sigma}} x} + d_{2} e^{-\sqrt{\frac{\lambda + \omega}{\sigma}}x} \right) \left( d_{3} e^{\lambda t} \right),
\end{equation}
Betrachten wir zunächst die Randbedingung $u(t, 0) = u(t, 1) = 0$, dann erhalten wir aus
\begin{equation}
    0 = u(t, 0) = \left( d_{1} + d_{2} \right) \left( d_{3} e^{\lambda t} \right),
\end{equation}
oder äquivalent $d_{1} = - d_{2}$, und aus 
\begin{equation}
    0 = u(t, 1) = \left( d_{1} e^{\sqrt{\frac{\lambda + \omega}{\sigma}}} + d_{2} e^{-\sqrt{\frac{\lambda + \omega}{\sigma}}} \right) \left( d_{3} e^{\lambda t} \right) = 
    d_{1} \left( e^{\sqrt{\frac{\lambda + \omega}{\sigma}}} - e^{-\sqrt{\frac{\lambda + \omega}{\sigma}}} \right) \left( d_{3} e^{\lambda t} \right),
\end{equation}
ohne Einschränkung $d_{1} \neq 0$, die Gleichung
\begin{equation}
    0 = e^{\sqrt{\frac{\lambda + \omega}{\sigma}}} - e^{-\sqrt{\frac{\lambda + \omega}{\sigma}}}.
\end{equation}
Aus dieser erhalten wir durch Äquivalenzumformungen
\begin{align}
    e^{\sqrt{\frac{\lambda + \omega}{\sigma}}} - e^{-\sqrt{\frac{\lambda + \omega}{\sigma}}} = 0
    &\quad \iff \quad
    e^{2\sqrt{\frac{\lambda + \omega}{\sigma}}} = 1
    \quad \iff \quad
    \sqrt{\tfrac{\lambda + \omega}{\sigma}} = k \pi i
    \\&\quad \iff \quad
    \tfrac{\lambda + \omega}{\sigma} = -k^2 \pi^2
    \quad \iff \quad
    \lambda = -k^2 \pi^2 \sigma - \omega,
\end{align}
mit $k \in \mathbb{Z}$ beliebig.
Einsetzen liefert nun
\begin{align}
    u_{k}(t, x) &= d_{1} \left( e^{k \pi i x} - e^{-k \pi i x} \right) \left( d_{3} e^{- (k^2 \pi^2 \sigma + \omega) t} \right)
    \\&= 2 d_{1} d_{3} i \sin(k \pi x) e^{-(k^2 \pi^2 \sigma + \omega)t},
\end{align}
wobei wir $\beta_{k} := 2 d_{1} d_{3} i$ setzen.

Da jedes $u_{k}$, $k \in \mathbb{Z}$, eine Lösung ist, erhalten wir durch
\begin{equation}
    u(t, x) = \sum_{k = 1}^{\infty} u_{k}(t, x) = \sum_{k = 1}^{\infty} \beta_{k} \sin(k \pi x) e^{-(k^2 \pi^2 \sigma + \omega)t}}
\end{equation}
ebenfalls eine Lösung. 
Damit die Anfangsbedingung erfüllt wird, muss
\begin{equation}
    g(x) = u(x, 0) = \sum_{k = 1}^{\infty} \beta_{k} \sin(k \pi x)
\end{equation}
gelten, was genau dann der Fall ist, wenn
\begin{equation}
    \beta_{k} = 2 \int_{0}^{1} g(x) \sin(k \pi x) \diff x.
\end{equation}

Da $u_{k}$ analytisch in $\omega$ für alle $k \in \mathbb{Z}$, ist auch $u$ analytisch in $\omega$ und es gilt
\begin{equation}
    \frac{\partial^{j} u(t, x; \omega)}{\partial \omega^{j}} = \sum_{k = 1}^{\infty} (-t)^{j} \beta_{k} \sin(k \pi x) e^{-(k^{2} \pi^{2} \sigma + \omega)t}.
\end{equation}


    % Alles, was noch unbedingt rein muss
    %!TEX root = ../main.tex

\setchapterpreamble[ul][0.6\textwidth]{%
    \dictum[Terry Pratchett]{\enquote{Coffee is a way of stealing time that should by rights belong to your older self.}}
    \vspace*{2\baselineskip}
}
\chapter{Funktionalanalytische Grundlagen} % (fold)
\label{cha:funktionalanalytische_grundlagen}

\todo[inline]{Ständig: ordnen, sortieren, aufräumen, erweitern.}

\section{Orthogonale Funktionen und Polynome}
\label{sec:orthogonale_funktionen_und_polynome}

\begin{Satz}[Orthogonalität trigonometrischer Funktionen]
\label{satz:trigonometrische_funktionen_orthogonal}
    Seien $k, l \in \mathbb{N}$.
    Dann gilt
    \begin{align}
        \skprod{\sin(\pi k x)}{\sin(\pi l x)}_{L_{2}([0, 1])} &= \frac{1}{2} \delta_{kl},
        % \quad\text{und}\quad
        \\\skprod{\cos(\pi k x)}{\cos(\pi l x)}_{L_{2}([0, 1])} &= \frac{1}{2} \delta_{kl},
        \\\skprod{\sin(\pi k x)}{\cos(\pi l x)}_{L_{2}([0, 1])} &= 0.
    \end{align}
\end{Satz}

\begin{Definition}[Legendre-Polynome]
\label{definition:legendre_polynome}
    Sei $I = [-1, 1]$.
    Die Legendre-Polynome $L_{n} \in \Pi_{n}$ sind definiert durch
    \begin{equation}
        L_{n}(x) = \frac{1}{2^{n}n!}\frac{\diff^{n}}{\diff x^{n}} (x^{2} - 1)^{n}.
    \end{equation}
    Durch die Transformation $x \mapsto 2x - 1$ erhält man die auf das Interval $[0, 1]$ geshifteten Legendre-Polynome $\tilde L_{n}$.
\end{Definition}

\begin{Satz}[Orthogonalität der Legendre-Polynome]
\label{satz:legendre_polynome_orthogonal}
    Die Legendre-Polynome $L_{n}$ sind orthogonal bezüglich der $L_{2}([-1, 1])$-Norm, denn es gilt
    \begin{equation}
        \skprod{L_{n}}{L_{m}}_{L_{2}([-1, 1])} = \frac{2}{2n + 1} \delta_{n m}.
    \end{equation}
    Auch für die geshifteten Legendre-Polynome $\tilde L_{n}$ gilt Orthogonalität, denn es ist
    \begin{equation}
        \skprod{\tilde L_{n}}{\tilde L_{m}}_{L_{2}([0, 1])} = \frac{1}{2n + 1} \delta_{n m}.
    \end{equation}
\end{Satz}

\begin{Bemerkung}
\label{satz:legendre_polynome_rekursion}
    Die Legendre-Polynome $L_{n}$ erfüllen die Rekursionsformel
    \begin{equation}
        n L_{n}(x) = (2n - 1) x L_{n-1}(x) - (n - 1) L_{n-2}(x), \quad L_{0}(x) = 1, L_{1}(x) = x.
    \end{equation}
    Analog gilt für die erste Ableitung $L_{n}'$ die Rekursionsformel
    \begin{equation}
        (n - 1) L_{n}'(x) = (2n -1) x L_{n-1}'(x) - n L_{n-2}'(x), \quad L_{0}'(x) = 0, L_{1}'(x) = 1.
    \end{equation}
\end{Bemerkung}

\section{Sonstiges} % (fold)
\label{sec:sonstiges}

% \begin{Lemma}
%     $\mathcal C^{0}([a, b]; X)$ liegt dicht in $L_{p}(a, b; X)$ für $1 \leq p < \infty$.
% \end{Lemma}

% TODO: zitieren
\begin{Satz}[Poincaré-Friedrichs-Ungleichung, vgl. {{\cite[Lemma 89.4]{HankeBourgeois:2009fk}}}]
\label{satz:grundlagen:poincare_friedrichs_ungleichung}
    Sei $\Omega \subset \mathbb{R}^{n}$ offen, beschränkt und mit Lipschitz-Rand.
    Dann existiert eine Konstante $\gamma_{\Omega} > 0$ mit
    \begin{equation}
        \label{eq:grundlagen:poincare_friedrichs_ungleichung}
        \norm{\grad u}_{L_{2}(\Omega)} \geq \gamma_{\Omega} \norm{u}_{H^{1}(\Omega)} \quad \fa u \in H^{1}_{0}(\Omega).
    \end{equation}
\end{Satz}

\begin{Satz}[Poincaré-Friedrichs-Ungleichung, vgl. {{\cite[Theorem II.1.7]{Braess:2007wm}}}]
    Es sei $\Omega \subset \mathbb{R}^{n}$ beschränkt und in einem $n$-dimensionalen Würfel mit Seitenlänge $s$ enthalten.
    Dann gilt
    \begin{equation}
        (1 + s)^{m} \abs{u}_{H^{m}} \geq \norm{u}_{H^{m}} \geq \abs{u}_{H^{m}} \quad \text{für alle}~u \in H^{m}_{0}(\Omega).
    \end{equation}
\end{Satz}

\begin{Lemma}[{{\cite[Remark 2.1.48]{Sauter:9_WoPZ0Y}}}]
\label{lemma:sauter:2.1.48}
    Seien $X$ und $Y$ zwei reflexive Banachräume und $a \colon X \times Y \to \mathbb{R}$ eine Bilinearform.
    Finden wir für jedes $x \in X$ ein $y_{x} \in Y$, so dass
    \begin{equation}
        \label{eq:lemma:sauter:2.1.48:eq1}
        \abs{a(x, y_{x})} \geq C_{1} \norm{x}_{X}^{2} \quad \text{und} \quad \norm{y_{x}}_{Y} \leq C_{2} \norm{x}_{X}
    \end{equation}
    mit von $x$ und $y_{x}$ unabhängigen Konstanten $C_{1}, C_{2} > 0$ gilt, dann folgt daraus die inf-sup-Bedingung
    \begin{equation}
    \label{eq:lemma:sauter:2.1.48:inf_sup}
        \inf_{0 \neq x \in X} \sup_{0 \neq y \in Y} \frac{a(x, y)}{\norm{x}_{X}\norm{y}_{Y}} \geq \gamma > 0
    \end{equation}
    mit $\gamma = \frac{C_{1}}{C_{2}}$.

    \begin{Beweis}
        Seien $x \in X$ und $y_{x} \in Y$ so, dass \cref{eq:lemma:sauter:2.1.48:eq1} erfüllt ist.
        Dann gilt
        \begin{align}
            \inf_{0 \neq x \in X} \sup_{0 \neq y \in Y} \frac{\abs{a(x, y)}}{\norm{x}_{X} \norm{y}_{Y}}
            &\geq
            \inf_{0 \neq x \in X} \frac{\abs{a(x, y_{x})}}{\norm{x}_{X} \norm{y_{x}}_{Y}}
            \\&\geq
            \inf_{0 \neq x \in X} \frac{C_{1} \norm{x}^{2}_{X}}{\norm{x}_{X} C_{2} \norm{x}_{X}}
            =
            \frac{C_{1}}{C_{2}}
            > 0.
        \end{align}
    \end{Beweis}
\end{Lemma}

% section sonstiges (end)


    % Weitere Anhänge
    % ...

    %% Los geht's mit den Verzeichnissen

    % Theoremverzeichnis
    \chapter*{Theoremverzeichnis}
    \addcontentsline{toc}{chapter}{Theoremverzeichnis}
    \theoremlisttype{allname}
    \listtheorems{Satz,Lemma,Definition,Annahme,Bemerkung}

    % Abbildungsverzeichnis
    % \listoffigures

    % Tabellenverzeichnis
    % \listoftables

    % Symbolverzeichnis
    % \printnomenclature{}

    % Akronymverzeichnis
    \printacronyms[heading=chapter*]

    % Literaturverzeichnis
    \printbibliography

    % Eidesstattliche Erklärung
    % %!TEX root = ../main.tex

\chapter*{Eidesstattliche Erklärung}

Ich versichere hiermit, dass ich die vorliegende Masterarbeit selbständig
verfasst und keine anderen als die angegebenen Quellen und Hilfsmittel benutzt
habe, wobei ich alle wörtlichen und sinngemäßen Zitate als solche gekennzeichnet
habe. Die Arbeit wurde bisher keiner anderen Prüfungsbehörde vorgelegt und auch
nicht veröffentlicht.\\[6ex]

\begin{flushright}
\ort, den \today

\color{jgu_hellgrau}\hdashrule[-0.5cm]{5cm}{0.5pt}{1pt}
\end{flushright}

\end{document}
