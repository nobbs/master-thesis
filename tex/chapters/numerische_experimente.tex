%!TEX root = ../main.tex

\chapter{Numerische Experimente} % (fold)
\label{cha:numerische_experimente}

\section{Grundlagen zum Galerkin-Verfahren} % (fold)
\label{sec:grundlagen_zum_galerkin_verfahren}

Wir wollen zunächst in Grundzügen das bekannte Galerkin-Verfahren wiederholen.
Seien dazu $U$ und $V$ zwei Hilberträume.
Betrachte das folgende abstrakte Variationsproblem:
\begin{Problem}
    Finde ein $u \in U$, so dass
    \begin{equation}
        a(u, v) = f(v) \quad \fa v \in V
    \end{equation}
    gilt.
\end{Problem}
Dabei sei $a(\blank, \blank) \colon U \times V \to \mathbb{R}$ eine Bilinearform und $f(\blank) \colon V \to \mathbb{R}$ ein lineares Funktional.
Unter gewissen Voraussetzungen an $a(\blank, \blank)$, auf die wir in dieser Wiederholung nicht genauer eingehen, existiert eine eindeutige Lösung $u \in U$ zu diesem Variationsproblem.

Da es sich bei $U$ und $V$ im Allgemeinen um Funktionenräume, damit insbesondere also um unendlichdimensionale Hilberträume, handelt, ist die Idee des Galerkin-Verfahrens nun, die beiden Räume jeweils durch endlichdimensionale Unterräume $U_{n} \subset U$ und $V_{m} \subset V$ zu approximieren.
Statt dem eigentlichen Variationsproblem betrachtet man die sogenannte Galerkin-Projektion:
\begin{Problem}
    Finde ein $u_{n} \in U_{n}$, so dass
    \begin{equation}
        a(u_{n}, v_{m}) = f(v_{m}) \quad \fa v_{m} \in V_{m}
    \end{equation}
    gilt.
\end{Problem}

Die endlichdimensionalen Unterräume werden als Spann von endlich vielen Basisfunktionen
\begin{equation}
    U_{n} = \spn\Set{\varphi_{i} \given i = 1 \dots n},
    \qquad
    V_{m} = \spn\Set{ \psi_{j} \given j = 1 \dots m}
\end{equation}
konstruiert.
Diese Eigenschaft lässt sich ausnutzen, denn dadurch wird es uns ermöglicht eine Lösung der Galerkin-Projektion durch das Aufstellen und Lösen eines linearen Gleichungssystems zu bestimmen.
Dazu benötigt man die sogenannte Steifigkeitsmatrix $\mat A \in \mathbb{R}^{m \times n}$, welche durch
\begin{equation}
    \mat A_{ji} = a(\varphi_{i}, \psi_{j}), \quad i = 1 \dots n, j = 1 \dots m,
\end{equation}
definiert ist, und den Lastvektor $\vec f \in \mathbb{R}^{m}$, gegeben durch
\begin{equation}
    \vec f_{j} = f(\psi_{j}), \quad j = 1 \dots m.
\end{equation}
Löst man nun das lineare Gleichungssystem $\mat A \vec u = \vec f$, dann erhält man aus dem Koeffizientenvektor $\vec u$ die Lösung $u_{n} \in U_{n}$ durch
\begin{equation}
    u_{n} = \sum_{i = 1}^{n} \vec u_{i} \varphi_{i}.
\end{equation}

% section grundlagen_zum_galerkin_verfahren (end)

\section{Erste Versuche mit Galerkin-Projektion} % (fold)
\label{sec:erste_versuche_mit_galerkin_projektion}

Wir beschränken uns erneut auf den eindimensionalen Fall und wählen $T = 1$, also $I = [0, 1]$ und $\Omega = [0 ,1]$.
Um eine Approximation für das Variationsproblem \eqref{eq:varprob} mit fest gewähltem $\omega \in L_{\infty}(\Omega)$ zu bestimmen, verwenden wir die Galerkin-Projektion.

Dazu definieren wir endlichdimensionale Unterräume $\mathcal X_{\mathcal N} \subset \mathcal X$ und $\mathcal Y_{\mathcal M} \subset \mathcal Y$, wobei $\dim(\mathcal X_{\mathcal N}) = \mathcal N$ respektive $\dim(\mathcal Y_{\mathcal M}) = \mathcal M$ sei.
Der Einfachheit halber fordern wir zunächst $\mathcal N = \mathcal M$.
Das nun zu lösende Problem lautet damit

\begin{Problem}
    Gegeben ein $g \in L_{2}(I; V')$ und ein $u_{0} \in H$. Finde ein $u \in \mathcal X_{\mathcal N}$ mit
    \begin{equation}
        \label{eq:varprob_3}
        b(u, v) = f(v) \quad \text{für alle}~v \in \mathcal Y_{\mathcal M}.
    \end{equation}
\end{Problem}

Wie wählt man nun die endlichdimensionalen Unterräume $\mathcal X_{\mathcal N}$ und $\mathcal Y_{\mathcal M}$?

\paragraph{Erster Versuch} % (fold)
\label{par:erster_versuch}

Da wir es mit homogenen Randbedingungen zu tun haben, ist es naheliegend, Sinusfunktionen $\sin(\pi j x)$, $j \geq 1$, als Basisfunktionen für die Raum-Koordinate zu verwenden.
Für die Zeitkoordinate wählen wir orthogonale Polynome, konkret auf den Intervall $I = [0, 1]$ verschobene Legendre-Polynome $L_{k}$, $k \geq 0$.

Die Idee hinter der Wahl der orthogonalen Polynome, und auch der Sinusfunktionen, ist, dass diese bezüglich der $L_{2}$-Norm jeweils ein orthogonales System bilden, wodurch für die Galerkin-Projektion aufzustellende Steifigkeitsmatrix $\mat B$ dünn besetzt ist.

Da wir $\mathcal N = \mathcal M$ erreichen wollen, wählen wir $M, Q \in \mathbb{N}$ und setzen
\begin{equation}
    \mathcal X_{\mathcal N} = \spn\Set{ \sin(\pi j x) L_{k}(t) \given j = 1 \dots M, k = 0 \dots Q }
\end{equation}
und
\begin{equation}
    \mathcal Y_{\mathcal M} = \spn\Set{ (\sin(\pi l x) L_{m}(t), \sin(\pi n x)) \given l, n = 1 \dots M, m = 0 \dots Q - 1 }.
\end{equation}
Damit ergibt sich $\mathcal N = \dim(\mathcal X_{\mathcal N}) = M (Q + 1)$ und $\mathcal M = \dim(\mathcal Y_{\mathcal M}) = M Q + M = M ( Q + 1 )$, also $\mathcal N = \mathcal M$.

Für die Galerkin-Projektion berechnen wir nun
\begin{equation}
    \mat B_{ij} = b(u_{j}, v_{i}) \quad\text{und}\quad \vec f_{i} = f(v_{i}),
\end{equation}
lösen anschließend das lineare Gleichungssystem
\begin{equation}
    \mat B \vec u = \vec f
\end{equation}
und bestimmen die Lösung $u \in \mathcal X_{\mathcal N}$ als Linearkombination der Ansatzfunktionen von $\mathcal X_{\mathcal N}$ mit den zugehörigen Koeffizienten aus $\vec u$.

Um die Funktionsweise an einem einfachen Beispiel zu überprüfen, wählen wir $w = 1$, $u_{0}(x) = x \sin(\pi x)$ und $g = 0$.
Damit wird die Bilinearform $b(\blank, \blank)$ für $u(t, x) = \sin(\pi j x) L_{k}(t)$ und $v(t,x) = (v_{1}(t,x), v_{2}(t,x)) = (\sin(\pi l x) L_{m}(t), \sin(\pi n x))$ zu
\begin{subequations}
    \begin{align}
    b(u, v)
        &= \int_{I} \skprod{u_{t}(t)}{v_{1}(t)}_{V' \times V} + a(u(t), v_{1}(t)) \diff t + \skprod{u(0)}{v_{2}}_{H}
        \\&= \int_{I} \int_{\Omega} \sin(\pi j x) L_{k}'(t) \sin(\pi l x) L_{m}(t) \diff x \diff t \label{eq:int1}
        \\&\quad  + \int_{I} \int_{\Omega} j l \pi^2 \cos(\pi j x) L_{k}(t) \cos(\pi l x) L_{m}(t) \diff x \diff t  \label{eq:int2}
        \\&\quad + \int_{I} \int_{\Omega} \omega(x) \sin(\pi j x) L_{k}(t) \sin(\pi l x) L_{m}(t) \diff x \diff t  \label{eq:int3}
        \\&\quad + \int_{\Omega} \sin(\pi j x) L_{k}(0) \sin(\pi n x) \diff x  \label{eq:int4}
        % \\&=
    \end{align}
\end{subequations}

Unter Verwendung der Orthogonalität der Legendre-Polynome und der Sinusfunktionen, vgl. \thref{satz:legendre_polynome_orthogonal} respektive \thref{satz:trigonometrische_funktionen_orthogonal}, vereinfachen wir die obigen Integral soweit möglich.
Für das zweite und das vierte Integral erhalten wir so die expliziten Ausdrücke
\begin{equation}
    \int_{I} \int_{\Omega} j l \pi^2 \cos(\pi j x) L_{k}(t) \cos(\pi l x) L_{m}(t) \diff x \diff t
    = \frac{(j \pi)^{2}}{2(2k + 1)} \delta_{j l}  \delta_{k m},
\end{equation}
respektive
\begin{equation}
    \int_{\Omega} \sin(\pi j x) L_{k}(0) \sin(\pi n x) \diff x = (-1)^{k} \frac{1}{2} \delta_{jn}.
\end{equation}
Das erste und das dritte Integral werden zu
\begin{equation}
    \int_{I} \int_{\Omega} \sin(\pi j x) L_{k}'(t) \sin(\pi l x) L_{m}(t) \diff x \diff t
    = \frac{1}{2} \delta_{j l} \int_{I} L_{k}'(t) L_{m}(t) \diff t
\end{equation}
und
\begin{equation}
    \int_{I} \int_{\Omega} \omega(x) \sin(\pi j x) L_{k}(t) \sin(\pi l x) L_{m}(t) \diff x \diff t
    = \frac{1}{2k +1} \delta_{k m} \int_{\Omega} \omega(x) \sin(\pi j x)\sin(\pi l x) \diff x.
\end{equation}

% TODO: Alten Stuff raus!
% \begin{align}
%     &\int_{I} \int_{\Omega} \sin(\pi j x) L_{k}'(t) \sin(\pi l x) L_{m}(t) \diff x \diff t
%     \\&\qquad = \int_{I} L_{k}'(t) L_{m}(t) \diff t \int_{\Omega} \sin(\pi j x) \sin(\pi l x) \diff x
%     \\&\qquad= \frac{1}{2} \delta_{j l} \int_{I} L_{k}'(t) L_{m}(t) \diff t
% \end{align}
% \begin{align}
%     &\int_{I} \int_{\Omega} j l \pi^2 \cos(\pi j x) L_{k}(t) \cos(\pi l x) L_{m}(t) \diff x \diff t
%     \\&\qquad = j l \pi^{2} \int_{I} L_{k}(t) L_{m}(t) \diff t \int_{\Omega} \cos(\pi j x) \cos(\pi l x) \diff x
%     \\&\qquad = \frac{(j \pi)^{2}}{2(2k + 1)} \delta_{j l}  \delta_{k m}
% \end{align}
% \begin{align}
%     &\int_{I} \int_{\Omega} \omega(x) \sin(\pi j x) L_{k}(t) \sin(\pi l x) L_{m}(t) \diff x \diff t
%     \\&\qquad = \int_{I} L_{k}(t) L_{m}(t) \diff t \int_{\Omega} \omega(x) \sin(\pi j x)\sin(\pi l x) \diff x
%     \\&\qquad = \frac{1}{2k +1} \delta_{k m} \int_{\Omega} \omega(x) \sin(\pi j x)\sin(\pi l x) \diff x
% \end{align}

Neben der Steifigkeitsmatrix $\mat B$ benötigen wir auch den Lastvektor $\vec f$, wobei $\vec f_{i} = f(v_{i})$, $i = 1 \dots \mathcal N$.
Konkret wird dies, mit dem gleichen $v \in \mathcal Y_{\mathcal N}$ wie oben, zu
\begin{align}
    f(v)
    &= \int_{I} \skprod{g(t)}{v_{1}(t)}_{V' \times V} \diff t + \skprod{u_{0}}{v_{2}}_{H}
    \\&= \int_{I} \int_{\Omega} g(t, x) \sin(\pi l x) L_{m}(t) \diff x \diff t + \int_{\Omega} u_{0}(x) \sin(\pi n x) \diff x.
\end{align}
Diese beiden Integrale können im Allgemeinen nicht wie die vorherigen direkt ausgewertet werden und müssen mittels numerischer Quadratur approximiert werden.

Sind Steifigkeitsmatrix und Lastvektor bestimmt, dann liefert die Lösung $\vec u$ des linearen Gleichungssystems $\mat B \vec u = \vec f$ einen Koeffizientenvektor, aus dem wir mittels
\begin{equation}
    u_{\mathcal N}(t, x) = \sum_{j = 1}^{M} \sum_{k = 0}^{Q} \vec u_{jk} \sin(\pi j x) L_{k}(t)
\end{equation}
die Näherungslösung $u_{\mathcal N} \in \mathcal X_{\mathcal N}$ des Problems \eqref{eq:varprob_3} erhalten.

% paragraph erster_versuch (end)

% section erste_versuche_mit_galerkin_projektion (end)

\clearpage
\section{Zu klärende Fragen} % (fold)
\label{sec:zu_kl_rende_fragen}

Was?

Konditionszahlen?

% section zu_kl_rende_fragen (end)

% chapter numerische_experimente (end)
