%!TEX root = ../main.tex

\iftoggle{dictum}{
    \setchapterpreamble[ul][0.6\textwidth]{%
        \dictum[Andy Weir, \textit{The Martian}]{\enquote{I guess you could call it a \enquote{failure}, but I prefer the term \enquote{learning
        experience}.}}
        \vspace*{2\baselineskip}
    }
}{}
\chapter{Propagator-Differentialgleichung} % (fold)
\label{cha:ps:problemstellung}

Wir greifen nun die aus \cref{cha:el:einleitung} bekannten parabolischen Differentialgleichungen \cref{eq:el:forward_propagator,eq:el:backward_propagator}, welche von den beiden Propagatoren $q$ und $q^{\dagger}$ erfüllt werden, auf.
Der Einfachheit halber verwenden wir im Nachfolgenden den Begriff \emph{Propagator-Differentialglei"-chung"-en}, wenn wir uns auf die genannten partiellen Differentialgleichungen beziehen.

In diesem Kapitel konkretisieren wir diese Propagator-Differentialgleichungen, schaffen geeignete Rahmenbedingungen und leiten eine schwache Formulierung her.
Anschließend werden die in den Propagator-Differentialglei"-chung"-en auftretenden Felder $\omega_{\bullet}$ parametrisiert und als Grundlage für eine parametrische schwache Raum-Zeit-Formulierung verwendet.
Für diese weisen wir abschließend nach, dass sie sachgemäß gestellt ist und eine gewisse Regularität bezüglich der Parameter aufweist.

\section{Raum-Zeit-Variationsformulierung} % (fold)
\label{sub:ps:rzvp:raum_zeit_variationsformulierung}

% \section{Die parabolische partielle Propagator-Differentialgleichung} % (fold)
% \label{sec:ps:pde:die_parabolische_partielle_differentialgleichung}

Wir wollen nun zunächst das Setting festlegen und die aus der Einführung bekannten Propagator-Gleichungen in einem allgemeineren Rahmen auffassen.
Dabei halten wir an dem Fall zweier Felder fest, wobei diese Einschränkung nicht notwendig ist, da die nachfolgenden Ergebnisse in gleicher Weise auch für jede andere endliche Felderanzahl nachgewiesen werden können.
Weiter sei an dieser Stelle angemerkt, dass es ausreicht sich auf die Betrachtung des Vorwärts-Propagators \cref{eq:el:forward_propagator} zu beschränken, da der Rückwärts-Propagator \cref{eq:el:backward_propagator} durch die einfache Transformation $s \mapsto 1 - s$ auf die selbe Form, lediglich mit vertauschen Rollen bei den Feldern, gebracht werden kann.

Seien nun $0 < T_{f} < T < \infty$ reelle Konstanten und $I = [0, T]$ ein reelles Intervall, welches wir in die beiden Teilintervalle $I_{1} = [0, T_{f})$ und $I_{2} = [T_{f}, T]$ zerlegen.
Weiter sei $\Omega \subset \mathbb{R}^{n}$, $n \in \mathbb{N}$, ein beschränktes Gebiet, das heißt offen, nichtleer, zusammenhängend und beschränkt, welches einen Lipschitz-Rand besitzt.

Da wir den Feldern an dieser Stelle noch keine allzu einschränkenden Bedingungen auferlegen wollen, schreiben wir diese in Form der Abbildung
\begin{equation}
\label{eq:ps:pde:omega_definition}
    \omega \colon I \times \Omega \to \mathbb{R}, \quad (t, \vec{x}) \mapsto
    w_{1}(\vec{x}) \chi_{I_{1}}(t) + w_{2}(\vec{x}) \chi_{I_{2}}(t)
    =
    \begin{cases}
        w_{1}(\vec{x}), & t < T_{f}, \\
        w_{2}(\vec{x}), & t \geq T_{f},
    \end{cases}
\end{equation}
mit den $L_{\infty}(\Omega)$-Abbildungen $w_{1}, w_{2}$ und den charakteristischen Funktionen $\chi_{I_{i}}$ für die entsprechenden Intervalle.

Unter dem Begriff \emph{Propagator-Differentialgleichung} wollen wir im Weiteren nun die folgende parabolische Differentialgleichung verstehen.

\begin{Definition}
\label{def:ps:pde:propagator_dgl}
    Es seien Felder $\omega$ wie in \cref{eq:ps:pde:omega_definition}, eine Anfangsbedingung $u_{0} \colon \Omega \to \mathbb{R}$, ein Quellterm $g \colon I \times \Omega \to \mathbb{R}$, eine Randbedingung, sowie Konstanten $c \in \mathbb{R}_{+}$ und $\mu \in \mathbb{R}$ gegeben.
    Als \emph{Propagator-Differentialgleichung} bezeichnen wir die parabolische partielle Differentialgleichung
    \begin{equation}
    \label{eq:ps:pde:propagator_dgl}
        \left\{
        \begin{aligned}
            u_{t}(t, x) - c \Delta u(t, x) + \omega(t, x) u(t, x) + \mu u(t, x) &= g(t, x) \quad &&\text{auf}~I \times \Omega,\\
            u(0, x) &= u_{0}(x) \quad &&\text{auf}~\Omega, \\
            u(t, x) \text{ erfüllt Randbedingung} &\quad &&\text{auf}~I \times \partial \Omega.
        \end{aligned}
        \right.
    \end{equation}
\end{Definition}

Da, wie in \cref{cha:el:einleitung} erwähnt, der Mittelwert der Felder keinen Einfluss auf das Ergebnis des dort beschriebenen Iterationsverfahrens hat, führen wir den zusätzlichen Term $\mu u(t, x)$ ein.
Dieser wird sich bei der späteren numerischen Umsetzung als nützlich erweisen.

% \section{Raum-Zeit-Variationsformulierung} % (fold)
% \label{sub:ps:rzvp:raum_zeit_variationsformulierung}

Unser Ziel ist es nun, eine Raum-Zeit-Variationsformulierung der Propagator-Differen"-ti"-al"-gleichung
aus \cref{def:ps:pde:propagator_dgl} herzuleiten.
Diese wird uns als Ausgangspunkt für die angedachten numerischen Verfahren dienen, weswegen wir uns auch bei der theoretischen Arbeit auf diese konzentrieren werden.

Zunächst führen wir aber eine Einschränkung der möglichen Randbedingungen durch.
Von größtem Interesse sind für uns, bedingt durch die Motivation der parabolischen Differentialgleichung in \cref{cha:el:einleitung}, der Fall homogener Dirichlet-Randbedingungen, also $u(t, x) = 0$ auf $I \times \partial \Omega$, und der Fall periodischer Randbedingungen.
Letztere werden am Ende dieses Kapitels nochmals aufgegriffen, während wir uns im Rest der Ausführungen auf den Fall homogener Dirichlet-Randbedingungen beschränken.

Bei der Herleitung der Raum-Zeit-Variationsformulierung der Propagator-Differential"-glei"-chung werden wir Schrittweise vorgehen und zunächst den stationären Fall betrachten, bevor wir darauf aufbauend die schwache Raum-Zeit-Variationsformulierung erhalten.
Als Grundlage für die schwache Formulierung im stationären Fall verwenden wir die wohlbekannten Sobolev-Räume.
Die folgende Bemerkung führt die notwendigen Notationen in diesem Zusammenhang ein.

\begin{Bemerkung}
\label{bemerkung:prop:gelfand}
    Wir schreiben kurz $V = H^{1}_{0}(\Omega)$ und $H = L_{2}(\Omega)$ für den bekannten Sobolev- respektive Lebesgue-Raum auf $\Omega$.
    Da es sich hierbei jeweils um einen Hilbertraum handelt, kennzeichnen wir die entsprechenden Skalarprodukte $\skp{\blank}{\blank}{}$ und Normen $\norm{\blank}$ durch den jeweiligen Index.
    Da weiter $V$ ein dichter Unterraum von $H$ ist, können wir nach \cref{def:gl:br:gelfand_tripel} ein Gelfand-Tripel der Form
    \begin{equation}
        V \denseinclusion H \denseinclusion V' = (H^{1}_{0}(\Omega))' = H^{-1}(\Omega)
    \end{equation}
    definieren.
    Motiviert durch \cref{lemma:stetige_fortstetung_der_dualen_paarung} verwenden wir die Schreibweise $\skp{\blank}{\blank}{V' \times V}$ auch für die duale Paarung auf $V' \times V$.
\end{Bemerkung}

Damit können wir nun den folgenden Operator und eine zugehörige Bilinearform definieren.

\begin{Definition}
\label{definition:prop:operator_und_bilinearform}
    Es seien Felder $\omega$ wie in \cref{eq:ps:pde:omega_definition} und Konstanten $c \in \mathbb{R}_{+}$ sowie $\mu \in \mathbb{R}$ gegeben.
    Wir definieren für $t \in I$ eine Familie von Operatoren
    \begin{equation}
        A(t) \colon V \to V', \quad \eta \mapsto A(t) \eta = - c \Delta \eta + \omega(t, \blank) \eta + \mu \eta
    \end{equation}
    und eine Familie von Bilinearformen
    \begin{equation}
        a(\blank, \blank; t) \colon V \times V \to \mathbb{R}, \quad (\eta, \zeta) \mapsto a(\eta, \zeta; t) = \skp{A(t)\eta}{\zeta}{V' \times V}.
    \end{equation}
\end{Definition}

\begin{Bemerkung}
\label{bemerkung:prop:biliniearform_riesz}
    Die Existenz der Bilinearform $a(\blank, \blank; t)$ zu dem Operator $A(t)$ lässt sich durch den Rieszschen Darstellungssatz begründen, siehe beispielsweise \cite[Theorem \S{}22.1]{Halmos:1957vd}.
\end{Bemerkung}

Aufgrund der gewählten Rahmenbedingungen können wir die Bilinearform $a(\blank, \blank; t)$ aus der vorhergehenden Definition explizit angeben.
Dazu greifen wir unter anderem auf die wohlbekannte schwache Formulierung des Laplace-Operators $- \Delta \colon V \to V'$ zurück.
Dieser lässt sich aufgrund der Wahl von $V$ auch als eine Bilinearform auf $V \times V$ der Form
\begin{equation}
    (\eta, \zeta) \mapsto \skp{- \Delta \eta}{\zeta}{V' \times V} = \skp{\grad \eta}{\grad \zeta}{H}
\end{equation}
schreiben.
Damit ergibt sich zusammen mit der Linearität der dualen Paarung
\begin{equation}
\label{eq:prop:bilinearform_darstellung}
    \begin{aligned}
        a(\eta, \zeta; t)
            &= \skp{A(t) \eta}{\zeta}{V' \times V}
          \\&= \skp{-c\Delta \eta + \omega(t, \blank) \eta + \mu \eta}{\zeta}{V' \times V}
          \\&= - c\skp{\Delta \eta}{\zeta}{V' \times V} + \skp{\omega(t, \blank) \eta}{\zeta}{V' \times V} + \mu \skp{\eta}{\zeta}{V' \times V}
          \\&= c\skp{\grad \eta}{\grad \zeta}{H} + \skp{\omega(t, \blank) \eta}{\zeta}{H} + \mu \skp{\eta}{\zeta}{H},
    \end{aligned}
\end{equation}
wobei bei der Wechsel von dualer Paarung auf $V' \times V$ zum $H$-Skalarprodukt beim zweiten und dritten Summanden durch \cref{lemma:stetige_fortstetung_der_dualen_paarung} und die Tatsache, dass mit $\eta \in V \subset H$ und der Definition von $\omega$ auch $\omega(t, \blank) \eta \in H$ gilt, gerechtfertigt ist.

Wir weisen nun zwei Eigenschaften für diesen Operator respektive die zugehörige Bilinearform nach, welche eine wichtige Rolle für die spätere Raum-Zeit-Variationsformu"-lie"-rung spielen werden.

\begin{Satz}
\label{satz:ps:rzvp:bilinearform_a_eigenschaften}
    Sei $a(\blank, \blank; t)$, $t \in I$, die Familie von Bilinearformen aus \cref{definition:prop:operator_und_bilinearform} respektive \cref{eq:prop:bilinearform_darstellung}.
    Für festes $t \in I$ erfüllt $a(\blank, \blank; t)$ die folgenden Eigenschaften:
    \begin{thmenumerate}
        \item\label{satz:ps:rzvp:bilinearform_a_eigenschaften_stetig}
        \emph{Stetigkeit:} es gilt
        \begin{equation}
            \label{eq:ps:rzvp:bilinearform_a_eigenschaften_stetig}
            \abs{a(\eta, \zeta; t)} \leq \gamma_{a} \norm{\eta}_{V} \norm{\zeta}_{V} \quad \text{für alle}~\eta, \zeta \in V
        \end{equation}
        mit Stetigkeitskonstante $\gamma_{a} = \max\Set{c, \norm{\omega(t, \blank)}_{L_{\infty}(\Omega)} + \abs{\mu}} < \infty$.
        \item\label{satz:ps:rzvp:bilinearform_a_eigenschaften_garding}
        \emph{G\aa{}rding-Ungleichung:} es gilt
        \begin{equation}
            \label{eq:ps:rzvp:bilinearform_a_eigenschaften_garding}
            a(\eta, \eta; t) + \lambda \norm{\eta}_{H}^{2} \geq \alpha \norm{\eta}_{V}^{2} \quad \text{für alle}~\eta \in V
        \end{equation}
        mit $\alpha = c \gamma_{\Omega}^{2} > 0$ und $\lambda = \max\Set{\norm{\omega(t, \blank)}_{L_{\infty}(\Omega)} - \mu, 0} \geq 0$, wobei $\gamma_{\Omega}$ die Poincaré-Friedrichs-Konstante ist.
    \end{thmenumerate}

    \begin{Beweis}
        Für den Nachweis der beiden Eigenschaften nehmen wir ohne Einschränkung $t \in I_{1}$ an, das heißt, es gilt $\omega(t, \blank) = w_{1}$.

        Zunächst zeigen wir die Stetigkeit.
        Seien dazu $\eta, \zeta \in V$ beliebig, dann erhalten wir unter Verwendung der Dreiecks- und der Cauchy-Schwarz-Ungleichung
        \begin{align}
            \abs{a(\eta, \zeta; t)}
            &= \abs{c \skp{\grad \eta}{\grad \zeta}{H} + \skp{w_{1} \eta}{\zeta}{H} + \mu \skp{\eta}{\zeta}{H} }
            \\&\leq c \abs{\skp{\grad \eta}{\grad \zeta}{H}} + \abs{\skp{w_{1} \eta}{\zeta}{H}} + \abs{\mu} \abs{\skp{\eta}{\zeta}{H}}
            \\&\leq c \norm{\grad \eta}_{H} \norm{\grad \zeta}_{H} + \left( \norm{w_{1}}_{L_{\infty}(\Omega)} + \abs{\mu} \right) \norm{\eta}_{H} \norm{\zeta}_{H}
            \\&\leq \max \Set{ c, \norm{w_{1}}_{L_{\infty}(\Omega)} + \abs{\mu}} \norm{\eta}_{V} \norm{\zeta}_{V}.
            \\&= \max \Set{ c, \norm{\omega(t, \blank)}_{L_{\infty}(\Omega)} + \abs{\mu}} \norm{\eta}_{V} \norm{\zeta}_{V}.
        \end{align}
        Für die G\aa{}rding-Ungleichung seien nun $\eta \in V$ und $\lambda \in \mathbb{R}$.
        Es gilt
        \begin{align}
            a(\eta, \eta; t) + \lambda \norm{\eta}^{2}_{H}
            &= c \norm{\grad \eta}^{2}_{H} + \skprod{w_{1} \eta}{\eta}_{H} + \mu \skprod{\eta}{\eta}_{H} + \lambda \skprod{\eta}{\eta}_{H}
            \\&= c \norm{\grad \eta}^{2}_{H} + \skprod{(w_{1} + \mu + \lambda) \eta}{\eta}_{H}.
        \end{align}
        Wählen wir nun $\lambda = \max\Set{\norm{w_{1}}_{L_{\infty}(\Omega)} - \mu, 0} = \max\Set{\norm{\omega(t, \blank)}_{L_{\infty}(\Omega)} - \mu, 0} \geq 0$, dann gilt $w_{1} + \mu + \lambda \geq 0$ fast überall in $\Omega$ und wir erhalten die Abschätzung
        \begin{align}
            a(\eta, \eta; t) + \lambda \norm{\eta}^{2}_{H}
            &\geq c \norm{\grad \eta}^{2}_{H},
            \intertext{woraus wir durch Anwenden der Poincaré-Friedrichs-Ungleichung (\cref{satz:gl:poincare_friedrichs_ungleichung})}
            a(\eta, \eta;t ) + \lambda \norm{\eta}^{2}_{H}
            &\geq c \gamma_{\Omega}^{2} \norm{\eta}^{2}_{V}
        \end{align}
        folgern können.
    \end{Beweis}
\end{Satz}

\begin{Bemerkung}
\label{bemerkung:rzvp:bilinearform_zeitunabhaengig}
    Nach Definition von $\omega$ in \cref{eq:ps:pde:omega_definition} gilt offenbar
    \begin{equation}
        \norm{\omega(t, \blank)}_{L_{\infty}(\Omega)} = \norm{w_{1}}_{L_{\infty}(\Omega)} \chi_{I_{1}}(t) + \norm{w_{2}}_{L_{\infty}(\Omega)} \chi_{I_{2}}(t),
    \end{equation}
    damit gilt wegen der Zerlegung $I = I_{1} \cupdot I_{2}$ insbesondere die Ungleichung
    \begin{equation}
        \min\Set{ \norm{w_{1}}_{L_{\infty}(\Omega)}, \norm{w_{2}}_{L_{\infty}(\Omega)} } \leq \norm{\omega}_{L_{\infty}(I; L_{\infty}(\Omega))} \leq \max\Set{ \norm{w_{1}}_{L_{\infty}(\Omega)}, \norm{w_{2}}_{L_{\infty}(\Omega)} }.
    \end{equation}
    Dies erlaubt uns, die Konstanten $\gamma_{a}$ und $\lambda$ im vorherigen \cref{satz:ps:rzvp:bilinearform_a_eigenschaften} ebenfalls unabhängig von $t \in I$ zu wählen, beispielsweise als
    \begin{equation}
        \gamma_{a} = \max\Set{c, \norm{\omega}_{L_{\infty}(I; L_{\infty}(\Omega))} + \abs{\mu}}, \quad
        \lambda = \max\Set{\norm{\omega}_{L_{\infty}(I; L_{\infty}(\Omega))} - \mu, 0}.
    \end{equation}
\end{Bemerkung}

\begin{Korollar}
\label{kor:ps:rzvp:bilinearform_elliptisch}
    Ist $\mu \geq \norm{\omega(t, \blank)}_{L_{\infty}(\Omega)}$, dann ist die Bilinearform $a(\blank, \blank; t)$ aus \cref{satz:ps:rzvp:bilinearform_a_eigenschaften} elliptisch.
\end{Korollar}


Mit dieser Vorarbeit können wir uns nun der schwachen Formulierung der Propagator-Differentialgleichung widmen.
Diese können wir informell durch Multiplikation der parabolischen Differentialgleichung aus \cref{def:ps:pde:propagator_dgl} mit einer Raum-Zeit-Testfunktion $v_{1}$ und Integration über $\Omega$ und $I$ sowie Addition der Anfangsbedingung, welche mit einer Raum-Testfunktion $v_{2}$ multipliziert und anschließend über $\Omega$ integriert wird, herleiten.

Um das Ganze auf eine rigorose Ebene zu bringen, benötigen wir zunächst zwei Hilberträume, welche als sogenannter Ansatz- respektive Testraum dienen werden.
Dabei handelt es sich um die Räume
\begin{equation}
    \label{eq:ps:rzvp:ansatzraum_testraum}
    \mathcal X = L_{2}(I; V) \cap H^{1}(I; V')
    \quad \text{und} \quad
    \mathcal Y = L_{2}(I; V) \times H,
\end{equation}
welche uns bereits aus \cref{def:gl:le:ansatz_und_testraum} bekannt sind.
Damit können wir diesen Abschnitt mit der folgenden Definition der schwachen Formulierung abschließen:

\begin{Definition}[Schwache Formulierung]
\label{definition:schwache_formulierung}
    Es sei ein Quellterm $g \in L_{2}(I; V')$ und eine Anfangsbedingung $u_{0} \in H$ gegeben.
    Als \emph{schwache Formulierung} oder \emph{Raum-Zeit-Variationsformulierung} der Propagator-Differentialgleichung bezeichnen wir das folgende Variationsproblem:
    \begin{equation}
    \label{eq:ps:rzvp:schwache_formulierung}
        \text{Finde}~u \in \mathcal X \colon \quad  b(u, v) = f(v) \quad \fa v = (v_{1}, v_{2}) \in \mathcal Y.
    \end{equation}
    Dabei sei die Bilinearform $b(\blank, \blank) \colon \mathcal X \times \mathcal Y \to \mathbb{R}$ durch
    \begin{equation}
        \label{eq:ps:rzvp:schwache_formulierung_lhs_b}
        b(u, v)
            = \int_{I} \skprod{u_{t}(t)}{v_{1}(t)}_{V' \times V} + a(u(t), v_{1}(t); t) \diff t + \skprod{u(0)}{v_{2}}_{H}
    \end{equation}
    und das stetige lineare Funktional $f \colon \mathcal Y \to \mathbb{R}$ auf der rechten Seite durch
    \begin{equation}
        \label{eq:ps:rzvp:schwache_formulierung_rhs_f}
        f(v) = \int_{I} \skprod{g(t)}{v_{1}(t)}_{V' \times V} \diff t + \skprod{u_{0}}{v_{2}}_{H}
    \end{equation}
    gegeben.
\end{Definition}

\begin{Bemerkung}
    Die Linearität von $f$ ist direkt ersichtlich; die Stetigkeit aber wollen wir an dieser Stelle nachweisen.
    Durch Anwendung der Cauchy-Schwarz- und der Hölder-Ungleichung erhalten wir
    \begin{equation}
        \begin{aligned}
            f(v)
            % &= \int_{I} \skprod{g(t)}{v_{1}(t)}_{V' \times V} \diff t + \skprod{u_{0}}{v_{2}}_{H}
            &\leq \int_{I} \norm{g(t)}_{V'} \norm{v_{1}(t)}_{V} \diff t + \norm{u_{0}}_{H} \norm{v_{2}}_{H}
            \\&\leq \left( \int_{I} \norm{g(t)}^{2}_{V'} \diff t \right)^{1/2} \left( \int_{I} \norm{v_{1}(t)}^{2}_{V} \diff t \right)^{1/2} + \norm{u_{0}}_{H} \norm{v_{2}}_{H}
            \\&= \norm{g}_{L_{2}(I; V')} \norm{v_{1}}_{L_{2}(I; V)} + \norm{u_{0}}_{H} \norm{v_{2}}_{H}
            \\&\leq \max\Set*{\norm{g}_{L_{2}(I; V')}, \norm{u_{0}}_{H}} \left( \norm{v_{1}}_{L_{2}(I; V)} + \norm{v_{2}}_{H} \right)
            \\&\leq \sqrt{2} \max\Set*{\norm{g}_{L_{2}(I; V')}, \norm{u_{0}}_{H}} \left( \norm{v_{1}}_{L_{2}(I; V)}^{2} + \norm{v_{2}}_{H}^{2} \right)^{1/2}
            \\&= \sqrt{2} \max\Set*{\norm{g}_{L_{2}(I; V')}, \norm{u_{0}}_{H}} \norm{v}_{\mathcal Y}
        \end{aligned}
    \end{equation}
    und damit die Stetigkeit, wobei die Ungleichung $x + y \leq \sqrt{2} \sqrt{x^2 + y^2}$, welche für alle $x, y \in \mathbb{R}$ gilt, für die letzte Abschätzung verwendet wurde.
\end{Bemerkung}


\section{Existenz und Eindeutigkeit von Lösungen} % (fold)
\label{sub:ps:eel:existenz_und_eindeutigkeit_von_loesungen}

An dieser Stelle wollen wir nun nachweisen, dass die schwache Formulierung aus \cref{definition:schwache_formulierung} im Sinne von \cref{def:gl:le:hadamard_sachgemaess_gestellt} sachgemäß gestellt ist.
Hier können wir mit \cref{satz:gl:le:ss09_theorem51} ansetzen, welcher unter den gegebenen Rahmenbedingungen die zu prüfenden Bedingungen auf die in \cref{satz:ps:rzvp:bilinearform_a_eigenschaften,bemerkung:rzvp:bilinearform_zeitunabhaengig} bereits nachgewiesenen reduziert.
Wir fassen dies zu folgendem Satz zusammen:

\begin{Korollar}
\label{satz:ps:eel:schwache_formulierung_sachgemaess_gestellt}
    Seien Ansatz- und Testraum $\mathcal X$ und $\mathcal Y$ wie in \cref{eq:ps:rzvp:ansatzraum_testraum} gegeben.
    Sei weiter ein Operator $B \colon \mathcal X \to \mathcal Y'$ definiert durch
    \begin{equation}
        \skprod{Bu}{v}_{\mathcal Y' \times \mathcal Y}  = b(u, v), \quad u \in \mathcal X,~ v \in \mathcal Y,
    \end{equation}
    wobei $b(\blank, \blank)$ die Bilinearform aus \cref{eq:ps:rzvp:schwache_formulierung_lhs_b} sei,
    dann ist $B$ stetig invertierbar.
    Insbesondere ist damit die Raum-Zeit-Variationsformulierung sachgemäß gestellt.
\end{Korollar}

Weiter erhalten wir als Nebenprodukt aus \cref{kor:gl:le:ss09_theorem51_ungleichungen} auch Schranken für die Stetigkeitskonstante $\gamma_{b}$ sowie die inf-sup-Konstante $\beta$ der Bilinearform $b(\blank, \blank)$.

\begin{Korollar}
\label{kor:ps:eel:schwache_formulierung_operator_schranken}
    Unter den selben Gegebenheiten wie in \cref{satz:ps:eel:schwache_formulierung_sachgemaess_gestellt} gelten, falls die Bilinearform $a(\blank, \blank;t)$ die G\aa{}rding-Ungleichung \cref{eq:ps:rzvp:bilinearform_a_eigenschaften_garding} mit $\lambda = 0$ erfüllt, die Abschätzungen
    \begin{align}
        \gamma_{b} = \norm{B}_{\mathcal L(\mathcal X, \mathcal Y')} &\leq \sqrt{2 \max\Set{ 1, c, \norm{\omega}_{L_{\infty}(I; L_{\infty}(\Omega))} + \abs{\mu} }^{2} + M_{e}^{2} }, \\
        \beta^{-1} = \norm{B^{-1}}_{\mathcal L(\mathcal Y', \mathcal X)} &\leq \frac{\sqrt{2 \max\Set{c^{-2}\gamma_{\Omega}^{-4}, 1} + M_{e}^{2}}}{\gamma_{\Omega}^{2} \min\Set{c, c^{-1}, c (\norm{\omega}_{L_{\infty}(I; L_{\infty}(\Omega))} + \abs{\mu})^{-2}}}.
    \end{align}

    Ist dagegen $\lambda \neq 0$, dann erhält man zusätzliche Faktoren, wodurch die Abschätzungen zu
    \begin{align}
        \gamma'_{b} = \norm{B}_{\mathcal L(\mathcal X, \mathcal Y')} &\leq \frac{\gamma_{b}}{\max\Set{\sqrt{1 + 2 (\norm{\omega}_{L_{\infty}(I; L_{\infty}(\Omega))} - \mu)^{2} \rho^{4} }, \sqrt{2}}}, \\
        \beta'^{-1} = \norm{B^{-1}}_{\mathcal L(\mathcal Y', \mathcal X)} &\leq \frac{\max\Set{\sqrt{1 + 2 (\norm{\omega}_{L_{\infty}(I; L_{\infty}(\Omega))} - \mu)^{2} \rho^{4} }, \sqrt{2}}}{e^{-2(\norm{\omega}_{L_{\infty}(I; L_{\infty}(\Omega))} - \mu) T}} \beta^{-1}
    \end{align}
    werden.

    Die Größen $M_{e}$ und $\rho$ entsprechen dabei denen aus \cref{kor:gl:le:ss09_theorem51_ungleichungen}.
\end{Korollar}

% subsection existenz_und_eindeutigkeit_von_l_sungen (end)

% section die_parabolische_partielle_differentialgleichung (end)

\section{Parametrische Formulierung} % (fold)
\label{sec:ps:pf:parametrische_formulierung}

Nachdem nun eine erste schwache Formulierung der Propagator-Differentialgleichung eingeführt wurde, welche in dieser Form bereits als Grundlage für eine numerische Umsetzung verwendet werden kann, wollen wir nun als nächsten Schritt eine Parametrisierung dieser vornehmen.
Motiviert durch das Iterationsverfahren aus \cref{cha:el:einleitung}, in dem die Propagator-Differentialgleichung immer wieder für leicht variierte Felder $\omega$ berechnet wird, werden wir diese als Ausgangspunkt der Parametrisierung verwenden.

Dazu kehren wir nun zunächst zum Operator $A(t)$ aus \cref{definition:prop:operator_und_bilinearform} zurück und betrachten diesen zunächst unabhängig von der Zeit $t \in I$, aber in Abhängigkeit von einem Feld $w \in L_{\infty}(\Omega)$.
Wir definieren für $w \in L_{\infty}(\Omega)$ eine Familie von Operatoren $A(w)$ als
\begin{equation}
    \label{eq:ps:pf:operator_A}
    A(w) \colon V \to V', \quad A(w) \eta = - c \Delta \eta + w \eta + \mu \eta
\end{equation}
und wie zuvor auch eine Familie von zugehörigen Bilinearformen $a(\blank, \blank; w)$ durch
\begin{equation}
    \label{eq:ps:pf:bilinearform_a}
    \begin{aligned}
        a(\blank, \blank; w) \colon V \times V \to \mathbb{R}, \quad
        (\eta, \zeta) \mapsto c\skp{\grad \eta}{\grad \zeta}{H} + \skp{w \eta}{\zeta}{H} + \mu \skp{\eta}{\zeta}{H}.
    \end{aligned}
\end{equation}

Um die obige Abhängigkeit des Operators $A$ respektive der Bilinearformen vom Feld $w \in L_{\infty}(\Omega)$ auch für die nachfolgende numerische Umsetzung verwendbar zu machen, müssen wir die Abhängigkeit von einer Abbildung $w \in L_{\infty}(\Omega)$ durch eine Abhängigkeit von einer diskreten Größe, beispielsweise einer Koeffizientenfolge aus dem $\ell_{1}(\mathbb{N})$ oder ähnlichen Folgenräumen, ersetzen.
Dies erreichen wir durch folgende Einschränkung der verwendeten Felder $w \in L_{\infty}(\Omega)$ auf Abbildungen, die wir als Reihenentwicklung von, hier noch nicht näher spezifizierten,  Abbildungen $\varphi_{j}$ darstellen können.

\begin{Definition}
\label{def:ps:pf:omega_affin}
    Sei $\Set{\varphi_{j}}_{j \in \mathbb{N}} \subset L_{\infty}(\Omega)$ eine Folge von Funktionen und sei weiter ein Parameterraum $\mathcal P \subset \mathbb{R}^{\mathbb{N}}$ gegeben.
    Wir nennen $w \in L_{\infty}(\Omega)$ \emph{affin} durch $\Set{\varphi_{j}}_{j \in \mathbb{N}}$ darstellbar, wenn ein $\sigma \in \mathcal P$ existiert, so dass $w$ mit
    \begin{equation}
        w(\blank; \sigma) = \sum_{j = 1}^{\infty} \sigma_{j} \varphi_{j}
    \end{equation}
    übereinstimmt.
\end{Definition}

\begin{Bemerkung}
    Für den Rest dieses Kapitels beschränken wir uns bei der Wahl des Parameterraums auf $\mathcal P = [-1, 1]^{\mathbb{N}}$.
    Dies stellt keine Einschränkung dar, da die Funktionen $\Set{ \varphi_{j} }_{j \in \mathbb{N}}$ entsprechend umskaliert werden können.
\end{Bemerkung}

Um sicherzustellen, dass derartige Felder $w(\sigma)$ wohldefinierte Operatoren $A(w(\sigma))$ liefern, fordern wir die folgende Eigenschaften von der Funktionenfolge $\Set{\varphi_{j}}_{j \in \mathbb{N}}$.

\begin{Annahme}
    \label{annahme:l1_summierbar}
    Das Funktionensystem $\Set{ \varphi_{j} }_{j \in \mathbb{N}} \subset L_{\infty}(\Omega)$ sei einfach summierbar in der $L_{\infty}$-Norm, das heißt es gelte
    \begin{equation}
        \Set{ \norm{\varphi_{j}}_{L_{\infty}(\Omega) } }_{j \in \mathbb{N}} \in \ell_{1}(\mathbb{N}).
    \end{equation}
\end{Annahme}

Diese Annahme stellt insbesondere die gleichmäßige Konvergenz von $w(\sigma)$ für alle $\sigma \in \mathcal P$ sicher, denn es gilt
\begin{equation}
\label{eq:ps:pf:omega_norm_abschaetzung}
    \sup_{\sigma \in \mathcal P} \norm{w(\sigma)}_{L_{\infty}(\Omega)} \leq \sum_{j = 1}^{\infty} \norm{\varphi_{j}}_{L_{\infty}(\Omega)} < \infty.
\end{equation}

Legen wir uns auf ein konkretes Funktionensystem $\Set{\varphi_{j}}_{j \in \mathbb{N}}$, welches die \cref{annahme:l1_summierbar} erfüllt, fest, dann können wir die Operatoren $A(\omega)$ nun auch als Familie von Operatoren $A(\sigma)$ betrachten, denn durch Einsetzen von $w(\sigma)$ in \cref{eq:ps:pf:operator_A} erhalten wir
\begin{equation}
\label{eq:ps:pf:operator_A_sigma}
    A(\sigma) = A(w(\sigma)) \colon V \to V', \quad A(\sigma) \eta = -c \Delta \eta + \sum_{j = 1}^{\infty} \sigma_{j} \varphi_{j} \eta + \mu \eta
\end{equation}
Weiter können wir auch die zugehörige Bilinearform $a(\blank, \blank; \sigma)$ angeben als
\begin{equation}
\label{eq:ps:pf:bilinearform_a_sigma}
    \begin{aligned}
    &a(\blank, \blank; \sigma) = a(\blank, \blank; w(\sigma)) \colon V \times V \to \mathbb{R}, \\
    &(\eta, \zeta) \mapsto c\skp{\grad \eta}{\grad \zeta}{H} + \sum_{j = 1}^{\infty} \sigma_{j} \skp{\varphi_{j} \eta}{\zeta}{H} + \mu \skp{\eta}{\zeta}{H}.
    \end{aligned}
\end{equation}

Das es sich hierbei tatsächlich um wohldefinierte Operatoren respektive Bilinearformen handelt, welche die für uns wichtigen Eigenschaften wie Stetigkeit respektive die Gültigkeit einer G\aa{}rding-Ungleichung besitzt, werden wir im nächsten Abschnitt nachweisen, zunächst wollen wir an dieser Stelle noch ein parametrisches Äquivalent der Raum-Zeit-Variationsformulierung aus \cref{definition:schwache_formulierung} formulieren.

Dies Bedarf, wie zuvor, einen zeitlichen Wechsel zwischen mehreren Feldern $w_{i}$.
Wir beschränken uns auf den bereits bekannten Fall zweier Felder und erweitern die obige Operator-Definition um die zeitliche Abhängigkeit.
Zunächst definieren wir analog zu \cref{eq:ps:pde:omega_definition} ein zeitabhängiges Feld.
Dies geschieht auf Basis der affinen Darstellung von $w_{i}$ aus \cref{def:ps:pf:omega_affin}.
Sei dazu $\sigma = (\sigma^{1}, \sigma^{2}) \in \mathcal P^{2} =  \mathcal P \times \mathcal P$, dann definieren wir die parametrische Feld-Abbildung als
\begin{equation}
\label{eq:ps:pde:omega_definition_affin}
    \omega(\blank, \blank; \sigma) \colon I \times \Omega \to \mathbb{R}, \quad (t, \vec{x}) \mapsto
    w(\vec{x}; \sigma^{1}) \chi_{I_{1}}(t) + w(\vec{x}; \sigma^{2}) \chi_{I_{2}}(t).
\end{equation}
Offenbar lässt sich durch Einsetzen der entsprechenden affinen Darstellung von $w$ direkt die Darstellung
\begin{equation}
    \omega(t, \vec{x}; \sigma) = \sum_{j = 1}^{\infty} (\sigma^{1}_{j} \chi_{I_{1}}(t) + \sigma^{2}_{j} \chi_{I_{2}}(t)) \phi_{j}(\vec x)
\end{equation}
ableiten.
Anhand dieser und der Zerlegung $I = I_{1} \cupdot I_{2}$ ist direkt ersichtlich, dass analog zu \cref{eq:ps:pf:omega_norm_abschaetzung} die Abschätzung
\begin{equation}
    \sup_{\sigma \in \mathcal P^{2}} \norm{\omega(\sigma)}_{L_{\infty}(I; L_{\infty}(\Omega))} \leq \sup_{\sigma \in \mathcal P} \norm{w(\sigma)} \leq \sum_{j = 1}^{\infty} \norm{\varphi_{j}}_{L_{\infty}(\Omega)}
\end{equation}
gilt.

Erweitern wir nun die Operator-Definition \cref{eq:ps:pf:operator_A_sigma} um die Zeitabhängigkeit; setzen wir also für $\sigma \in \mathcal P^{2}$
\begin{equation}
    \label{eq:ps:pf:operator_A_sigma_zeit}
    A(t; \sigma) \colon V \to V', \quad A(t, \sigma) \eta = -c \Delta \eta + \omega(t, \blank; \sigma) \eta + \mu \eta,
\end{equation}
dann hat die zugehörige Familie von Bilinearformen die Form
\begin{equation}
    \label{eq:ps:pf:bilineaform_a_sigma_zeit}
    \begin{aligned}
        &a(\blank, \blank; t; \sigma) \colon V \times V \to \mathbb{R}, \\
        &(\eta, \zeta) \mapsto c\skp{\grad \eta}{\grad \zeta}{H} + \sum_{j = 1}^{\infty} \sigma^{1}_{j} \chi_{I_{1}}(t) \skp{\varphi_{j} \eta}{\zeta}{H} + \sum_{j = 1}^{\infty} \sigma^{2}_{j} \chi_{I_{2}}(t) \skp{\varphi_{j} \eta}{\zeta}{H} + \mu \skp{\eta}{\zeta}{H}.
    \end{aligned}
\end{equation}

Mit dieser Vorarbeit können wir die schwache Formulierung aus \cref{def:ps:rzvp:schwache_formulierung} nun zu der folgenden parametrischen schwachen Formulierung ausweiten.

\begin{Definition}[Parametrische schwache Formulierung]
\label{def:ps:pf:schwache_formulierung}
    Seien ein Quellterm $g \in L_{2}(I; V')$ und eine Anfangsbedingung $u_{0} \in H$ gegeben.
    Als \emph{parametrische schwache Formulierung} von \cref{definition:schwache_formulierung} bezeichnen wir das folgende Variationsproblem:
    \begin{equation}
        \text{Sei}~\sigma \in \mathcal P^{2},~\text{finde ein}~u(\sigma) \in \mathcal X : b(u(\sigma), v; \sigma) = f(v) \quad \fa v \in \mathcal Y.
    \end{equation}
    Dabei sei die Bilinearform $b(\blank, \blank; \sigma) \colon \mathcal X \times \mathcal Y \to \mathbb{R}$ gegeben durch
     \begin{equation}
         \label{eq:ps:rzvp:schwache_formulierung_lhs_b_sigma}
         b(u, v; \sigma)
             = \int_{I} \skprod{u_{t}(t)}{v_{1}(t)}_{V' \times V} + a(u(t), v_{1}(t); t; \sigma) \diff t + \skprod{u(0)}{v_{2}}_{H},
     \end{equation}
     wobei $a(\blank, \blank; t; \sigma)$ wie in \cref{eq:ps:pf:bilineaform_a_sigma_zeit} definiert sei.
     Das stetige lineare Funktional $f \colon \mathcal Y \to \mathbb{R}$ sei wie zuvor durch
     \begin{equation}
         \label{eq:ps:rzvp:schwache_formulierung_rhs_f_sigma}
         f(v) = \int_{I} \skprod{g(t)}{v_{1}(t)}_{V' \times V} \diff t + \skprod{u_{0}}{v_{2}}_{H}
     \end{equation}
     gegeben.
\end{Definition}

% section parametrische_formulierung (end)


\section{Regularität bezüglich der Parameter} % (fold)
\label{sec:ps:rg:regularitaet_bezueglich_der_parameter}


In diesem Abschnitt wollen wir nun nachweisen, dass die im vorherigen Abschnitt hergeleitete parametrische schwache Formulierung eine Regularität bezüglich des Parameters aufweist, konkret werden wir nachweisen, dass die Lösung $u(\sigma)$ analytisch im Parameter $\sigma$ ist.
Dabei orientieren wir uns an den Arbeiten von \textcite{Cohen:2010kz,Kunoth:2013ef} und weisen diese Eigenschaften zunächst für den stationären Fall nach, das heißt, wir verlieren die Zeitabhängigkeit und betrachten stattdessen die parametrische Operatorgleichung
\begin{equation}
\label{eq:ps:rg:operatorgleichung_parametrisch}
    A(\omega) \eta(\omega) = g \quad \text{in}~H^{-1}(\Omega)
\end{equation}
mit dem parametrischen Operator $A(\omega)$ aus Gleichung \cref{eq:ps:pf:operator_A}, beziehungsweise die zugehörige Variationsformulierung
\begin{equation}
\label{eq:ps:rg:variationsformulierung_parametrisch}
    a(\eta, \zeta; \omega) = \skp{g}{\zeta}{H^{-1} \times H^{1}_{0}}
\end{equation}
mit der parametrischen Bilinearform $a(\blank, \blank; \omega)$ aus Gleichung \cref{eq:ps:pf:bilinearform_a}.

Wir beginnen damit, die Existenz der partiellen Ableitung $\partial^{\nu}_{\sigma} \eta$ in jedem Punkt $\sigma \in \mathcal S$ nachzuweisen.

Zunächst einige notationelle Vorbemerkungen.
\begin{Bemerkung}
    Bezeichne mit $\mathfrak F = \Set{ \nu \in \mathbb{N}^{\mathbb{N}}_{0} \given \norm{\nu}_{\ell_{1}(\mathbb{N})} < \infty }$ die Menge aller Folgen nichtnegativer ganzer Zahlen, wobei
    \begin{equation}
        \norm{\nu}_{\ell_{1}(\mathbb{N})} = \sum_{k = 1}^{\infty} \abs{\nu_{k}}
    \end{equation}
    die $\ell_{1}(\mathbb{N})$-Norm sei.
    Damit ergibt sich, dass $\mathfrak F$ gerade diejenigen Folgen enthält, die nur endlich viele Einträgen ungleich Null enthalten.

    Sei $\nu \in \mathfrak F$ und $b \in \ell_{p}(\mathbb{N})$, $p > 0$, dann schreiben wir
    \begin{equation}
        b^{\nu} = \prod_{j = 1}^{\infty} b_{j}^{\nu_{j}}
    \end{equation}
    mit der Konvention $0^{0} = 1$.
    Wegen $\norm{\nu}_{\ell_{1}(\mathbb{N})} < \infty$ ist dieses Produkt stets endlich.
\end{Bemerkung}

Bevor wir uns an die partiellen Ableitungen wagen, beweisen wir eine Stabilitätsaussage für die obige parametrische Operatorgleichung \cref{eq:ps:rg:operatorgleichung_parametrisch}
Um diese nachzuweisen, benötigen wir zunächst, dass die obige Operatorgleichung sachgemäß gestellt ist.

\begin{Satz}
\label{satz:ps:rg:lax_milgram_anwendung}
    Seien $\omega \in L_{\infty}(\Omega)$, $A(\omega)$ wie in \cref{eq:ps:pf:operator_A} mit $\mu \geq \norm{\omega}_{L_{\infty}(\Omega)}$ und weiter $g \in H^{-1}(\Omega)$, dann besitzt die Operatorgleichung
    \begin{equation}
        \tag*{\cref{eq:ps:rg:operatorgleichung_parametrisch}}
        A(\omega) \eta(\omega) = g \quad \text{in}~H^{-1}(\Omega)
    \end{equation}
    eine eindeutige Lösung $\eta(\omega) \in H^{1}_{0}(\Omega)$ und diese erfüllt
    \begin{equation}
        \norm{\eta(\omega)}_{H^{1}(\Omega)} \leq \frac{\norm{g}_{H^{-1}(\Omega)}}{\alpha}
    \end{equation}
    mit $\alpha$ aus \cref{satz:ps:rzvp:bilinearform_a_eigenschaften}.

    \begin{Beweis}
        Folgt direkt aus dem Lemma vom Lax-Milgram, \cref{lem:gl:le:lax_milgram}.
    \end{Beweis}
\end{Satz}

Für den Rest des Abschnitts sei nun stets $\mu \geq \sum_{j = 1}^{\infty} \norm{\varphi_{j}}_{L_{\infty}(\Omega)}$ gegeben.
Da nach Ungleichung \cref{eq:ps:pf:omega_norm_abschaetzung} damit insbesondere $\norm{\omega(\sigma)}_{L_{\infty}(\Omega)}$ für alle $\sigma \in \mathcal S$ gilt, sind im Folgenden stets die Voraussetzungen für \cref{satz:ps:rg:lax_milgram_anwendung} erfüllt.

\mfix{Umformulieren für $\omega(\sigma)$ statt $\omega$!}
\begin{Lemma}
\label{lem:ps:rg:norm_abschaetzung}
    Seien $\omega_{1}, \omega_{2} \in L_{\infty}(\Omega)$ und $\eta_{1}, \eta_{2}$ die zugehörigen Lösungen von \cref{eq:ps:rg:operatorgleichung_parametrisch}, dann gilt
    \begin{equation}
        \norm{\eta_{1} - \eta_{2}}_{H^{1}(\Omega)} \leq \frac{\norm{g}_{H^{-1}(\Omega)}}{\alpha^{2}} \norm{\omega_{1} - \omega_{2}}_{L_{\infty}(\Omega)}.
    \end{equation}

    \begin{Beweis}
        Durch Subtraktion der Variationsformulierungen \cref{eq:ps:rg:variationsformulierung_parametrisch} für $\eta_{1}$ und $\eta_{2}$ erhalten wir für alle $\zeta \in H^{1}_{0}(\Omega)$ die Gleichung
        \begin{align}
            0 &= a(\eta_{1}, \zeta; \omega_{1}) - a(\eta_{2}, \zeta; \omega_{2})
            \\&= c \skp{\grad \eta_{1} - \grad \eta_{2}}{\grad \zeta}{L_{2}(\Omega)} + \skp{\omega_{1}\eta_{1} - \omega_{2} \eta_{2}}{\zeta}{L_{2}(\Omega)} + \mu \skp{\eta_{1} - \eta_{2}}{\zeta}{L_{2}(\Omega)},
            \intertext{durch setzen von $\theta = \eta_{1} - \eta_{2}$ erhalten wir weiter}
            0 &= c \skp{\grad \theta}{\grad \zeta}{L_{2}(\Omega)} + \skp{\omega_{1} \theta}{\zeta}{L_{2}(\Omega)} + \mu \skp{\theta}{\zeta}{L_{2}(\Omega)} + \skp{(\omega_{1} - \omega_{2}) \eta_{2}}{\zeta}{L_{2}(\Omega)}
            \\&= a(\theta, \zeta; \omega_{1}) + \skp{(\omega_{1} - \omega_{2}) \eta_{2}}{\zeta}{L_{2}(\Omega)}.
        \end{align}
        Dies lässt sich nun in der Form
        \begin{equation}
            a(\theta, \zeta; \omega_{1}) = - \skp{(\omega_{1} - \omega_{2}) \eta_{2}}{\zeta}{L_{2}(\Omega)} =: h(\zeta)
        \end{equation}
        wieder als Variationsproblems \cref{eq:ps:rg:variationsformulierung_parametrisch} interpretieren.
        Da es sich bei $h$ um ein stetiges lineares Funktional auf $H^{1}_{0}(\Omega)$ handelt, liefert nun \cref{satz:ps:rg:lax_milgram_anwendung}, dass $\theta = \eta_{1} - \eta_{2}$ die eindeutige Lösung dieses Variationsproblems ist und weiterhin die Ungleichung
        \begin{equation}
            \norm{\theta}_{H^{1}(\Omega)} \leq \frac{\norm{h}_{H^{-1}(\Omega)}}{\alpha}
        \end{equation}
        erfüllt.
        Die Operatornorm von $h$ lässt sich mittels der Cauchy-Schwarz-Ungleichung und \cref{satz:ps:rg:lax_milgram_anwendung} bestimmen zu
        \begin{equation}
            \begin{aligned}
                \norm{h}_{H^{-1}(\Omega)}
                  &=    \sup_{\norm{\zeta}_{H^{1}(\Omega)} = 1} \abs{h(\zeta)}
                \\&=    \sup_{\norm{\zeta}_{H^{1}(\Omega)} = 1} \abs{\skp{(\omega_{1} - \omega_{2}) \eta_{2}}{\zeta}{L_{2}(\Omega)}}
                \\&\leq \sup_{\norm{\zeta}_{H^{1}(\Omega)} = 1} \norm{\omega_{1} - \omega_{2}}_{L_{\infty}(\Omega)} \norm{\eta_{1}}_{L_{2}(\Omega)} \norm{\zeta}_{L_{2}(\Omega)}
                \\&\leq \sup_{\norm{\zeta}_{H^{1}(\Omega)} = 1} \norm{\omega_{1} - \omega_{2}}_{L_{\infty}(\Omega)} \norm{\eta_{1}}_{H^{1}(\Omega)} \norm{\zeta}_{H^{1}(\Omega)}
                \\&=    \norm{\omega_{1} - \omega_{2}}_{L_{\infty}(\Omega)} \norm{\eta_{1}}_{H^{1}(\Omega)}
                \\&\leq \norm{\omega_{1} - \omega_{2}}_{L_{\infty}(\Omega)} \frac{\norm{g}_{H^{-1}(\Omega)}}{\alpha}.
            \end{aligned}
        \end{equation}
        Zusammen liefert dies die Ungleichung
        \begin{equation}
            \norm{\eta_{1} - \eta_{2}}_{H^{1}(\Omega)}
            = \norm{\theta}_{H^{1}(\Omega)} \leq \frac{\norm{g}_{H^{-1}(\Omega)}}{\alpha^{2}} \norm{\omega_{1} - \omega_{2}}_{L_{\infty}(\Omega)}
        \end{equation}
        und damit die Behauptung.
    \end{Beweis}
\end{Lemma}

\begin{Satz}
\label{satz:ps:rg:existenz_partieller_ableitungen}
    Die Abbildung $\mathcal S \ni \sigma \mapsto \eta(\sigma) \in H^{1}_{0}(\Omega)$, welche einem Parameter $\sigma$ die zugehörige Lösung $\eta(\sigma)$ des Variationsproblems \cref{eq:ps:rg:variationsformulierung_parametrisch} zuordnet, besitzt für alle $\nu \in \mathfrak F$ eine partielle Ableitung $\partial^{\nu}_{\sigma} \eta(\sigma)$.

    \begin{Beweis}
        Wir beschränken uns darauf die Behauptung exemplarisch für die partiellen Ableitungen erster Ordnung für ein festes $\sigma \in \mathcal S$ nachzuweisen.
        Seien dazu $\nu = e_{j}$ für ein $j \in \mathbb{N}$ und sei weiter $h \in \mathbb{R} \setminus \Set{ 0 }$.
        Definiere $\sigma_{h} = \sigma + h e_{j}$ und folglich $\sigma_{0} = \sigma$.
        Weiter setzen wir
        \begin{equation}
            \theta_{h} = \frac{\eta(\sigma_{h}) - \eta(\sigma)}{h},
        \end{equation}
        wobei $\eta(\blank)$ die Lösungen des Variationsproblems zu den entsprechenden Parametern seien.
        Ist $\abs{h}$ klein genug, so dass $\sigma_{h} = \sigma + h e_{j} \in \mathcal S$ gilt, dann existieren die eindeutigen $\eta(\blank)$ nach \cref{satz:ps:rg:lax_milgram_anwendung}, das heißt, der obige Ausdruck für $\theta_{h}$ ist für diese $h$ wohldefiniert.

        Betrachte unter diesen Gegebenheiten nun die Differenz der zu $\eta(\sigma_{h})$ und $\eta(\sigma)$ zugehörigen Variationsprobleme \cref{eq:ps:rg:variationsformulierung_parametrisch}, dann gilt für alle $\zeta \in H^{1}_{0}(\Omega)$ die Gleichung
        \begin{align}
            0 &= a(\eta(\sigma_{h}), \zeta; \sigma_{h}) - a(\eta(\sigma), \zeta; \sigma)
            %
            \\ &= c \skp{\grad \eta(\sigma_{h}) - \grad \eta(\sigma)}{\grad \zeta}{L_{2}(\Omega)}
                    + \skp{\omega(\sigma_{h})\eta(\sigma_{h}) - \omega(\sigma) \eta(\sigma)}{\zeta}{L_{2}(\Omega)}
                    \\&\qquad+ \mu \skp{\eta(\sigma_{h}) - \eta(\sigma)}{\zeta}{L_{2}(\Omega)}
            \\ &= h a(\theta_{h}, \zeta; \sigma) + \skp{(\omega(\sigma_{h}) - \omega(\sigma))\eta(\sigma_{h})}{\zeta}{L_{2}(\Omega)}
        \end{align}
        Erneut können wir diese Gleichung in die Form des Variationsproblems \cref{eq:ps:rg:variationsformulierung_parametrisch} bringen, konkret also
        \begin{equation}
            a(\theta_{h}, \zeta; \sigma) = F_{h}(\zeta) \quad \fa \zeta \in H^{1}_{0}(\Omega).
        \end{equation}
        Dabei lässt sich das stetige lineare Funktional $F_{h} \in H^{-1}(\Omega)$ wegen der affinen Darstellung \cref{eq:ps:pf:omega_affine_zerlegung} von $\omega(\blank)$ schreiben als
        \begin{equation}
            F_{h} \colon H^{1}_{0}(\Omega) \to \mathbb{R},
            \quad \zeta \mapsto - h^{-1} \skp{(\omega(\sigma_{h}) - \omega(\sigma))\eta(\sigma_{h})}{\zeta}{L_{2}(\Omega)}
            = - \skp{\varphi_{j} \eta(\sigma_{h})}{\zeta}{L_{2}(\Omega)}.
        \end{equation}
        Weiter ist $F_{h}(\blank)$ stetig in $h = 0$, denn für festes $\zeta \in H^{1}_{0}(\Omega)$ gilt unter Verwendung der Cauchy-Schwarz-Ungleichung die Abschätzung
        \begin{align}
            \abs{F_{h}(\zeta) - F_{0}(\zeta)}
            &= \abs{\skp{\varphi_{j} (\eta(\sigma_{h}) - \eta(\sigma))}{\zeta}{L_{2}(\Omega)}}
            \\&\leq \norm{\varphi_{j}}_{L_{\infty}(\Omega)} \abs{\skp{\eta(\sigma_{h}) - \eta(\sigma)}{\zeta}{L_{2}(\Omega)}}
            \\&\leq \norm{\varphi_{j}}_{L_{\infty}(\Omega)} \norm{\eta(\sigma_{h}) - \eta(\sigma)}_{L_{2}(\Omega)} \norm{\zeta}_{L_{2}(\Omega)}
            \\&\leq \norm{\varphi_{j}}_{L_{\infty}(\Omega)} \norm{\eta(\sigma_{h}) - \eta(\sigma)}_{H^{1}(\Omega)} \norm{\zeta}_{H^{1}(\Omega)}.
        \end{align}
        Weiter können wir die Stabilitätsaussage aus \cref{lem:ps:rg:norm_abschaetzung} für die hier auftretenden Parameter vereinfachen zu
        \begin{equation}
            \norm{\eta(\sigma_{h}) - \eta(\sigma)}_{H^{1}(\Omega)}
            \leq \frac{\norm{g}_{H^{-1}(\Omega)}}{\alpha^{2}} \norm{\omega(\sigma_{h}) - \omega(\sigma)}_{L_{\infty}(\Omega)}
            = \abs{h} \norm{\varphi_{j}}_{L_{\infty}(\Omega)} \frac{\norm{g}_{H^{-1}(\Omega)}}{\alpha^{2}}.
        \end{equation}
        Zusammen liefern die beiden obigen Abschätzungen
        \begin{equation}
            \abs{F_{h}(\zeta) - F_{0}(\zeta)} \leq \abs{h} \norm{\varphi_{j}}^{2}_{L_{\infty}(\Omega)} \norm{\zeta}_{H^{1}(\Omega)} \frac{\norm{g}_{H^{-1}(\Omega)}}{\gamma_{0}^{2}} \to 0 \quad \text{für}~h \to 0,
        \end{equation}
        das heißt, es gilt $F_{h} \to F_{0}$ in $H^{-1}(\Omega)$ für $h \to 0$.
        \mfix{Genauer ausführen, warum. (vergleiche $\eta(\sigma_{h}) = A(\sigma_{h})^{-1} F_{h}$)}
        Dies impliziert insbesondere $\theta_{h} \to \theta_{0}$ in $H^{1}_{0}(\Omega)$ für $h \to 0$.
        Weiter erfüllt $\theta_{0}$ die Gleichung
        \begin{equation}
            a(\theta_{0}, \zeta; \sigma) = F_{0}(\zeta) \quad \fa \zeta \in H^{1}_{0}(\Omega).
        \end{equation}
        Damit existiert $\partial_{\sigma_{j}} \eta(\sigma) = \theta_{0}$ in $H^{1}_{0}(\Omega)$ und ist die eindeutige Lösung des Variationsproblems
        \begin{equation}
        \label{eq:ps:rg:variationsproblem_partielle_ableitung}
            a(\partial_{\sigma_{j}} \eta(\sigma), \zeta; \sigma) = - \skp{\varphi_{j} \eta(\sigma)}{\zeta}{L_{2}(\Omega)} \quad \fa \zeta \in H^{1}_{0}(\Omega).
        \end{equation}

        Analog kann man auch die Existenz der partiellen Ableitungen höherer Ordnung nachweisen, indem man die in diesem Beweis verwendeten Schritte nun auf das neue Variationsproblem \cref{eq:ps:rg:variationsproblem_partielle_ableitung} anwendet.
    \end{Beweis}
\end{Satz}

\begin{Bemerkung}
\label{bem:ps:rg:partielle_ableitungen_alternativ_ueber_ableitung_des_operators}
    Alternativ erhält man das Variationsproblem \cref{eq:ps:rg:variationsproblem_partielle_ableitung} auch durch formales Differenzieren der Variationsformulierung \cref{eq:ps:rg:variationsformulierung_parametrisch} nach $\sigma_{j}$.
    Denn es gilt
    \begin{align}
        \partial_{\sigma_{j}} a(\eta(\sigma), \zeta; \sigma)
        &= \partial_{\sigma_{j}} \left( c \skp{\grad \eta(\sigma)}{\grad \zeta}{L_{2}(\Omega)} + \skp{\omega(\sigma)\eta(\sigma)}{\zeta}{L_{2}(\Omega)} + \mu \skp{\eta(\sigma)}{\zeta}{L_{2}(\Omega)} \right)
        \\&= c \skp{\grad \partial_{\sigma_{j}} \eta(\sigma)}{\grad \zeta}{L_{2}(\Omega)}
                + \skp{\partial_{\sigma_{j}} \omega(\sigma) \eta(\sigma) + \omega(\sigma) \partial_{\sigma_{j}} \eta(\sigma)}{\zeta}{L_{2}(\Omega)}
            \\&\qquad + \mu \skp{\partial_{\sigma_{j}} \eta(\sigma)}{\zeta}{L_{2}(\Omega)}
        \\&= a(\partial_{\sigma_{j}} \eta(\sigma), \zeta; \sigma) + \skp{\partial_{\sigma_{j}} \omega(\sigma) \eta(\sigma)}{\zeta}{L_{2}(\Omega)}
        \\&= a(\partial_{\sigma_{j}} \eta(\sigma), \zeta; \sigma) + \skp{\varphi_{j} \eta(\sigma)}{\zeta}{L_{2}(\Omega)}
    \end{align}
    und
    \begin{equation}
        \partial_{\sigma_{j}} \skp{g}{\zeta}{H^{-1}(\Omega) \times H^{1}_{0}(\Omega)} = 0,
    \end{equation}
    woraus man insgesamt erneut das Variationsproblem \cref{eq:ps:rg:variationsproblem_partielle_ableitung},
    \begin{equation}
        a(\partial_{\sigma_{j}} \eta(\sigma), \zeta; \sigma) = - \skp{\varphi_{j} \eta(\sigma)}{\zeta}{L_{2}(\Omega)} \quad \fa \zeta \in H^{1}_{0}(\Omega),
    \end{equation}
    erhält.
\end{Bemerkung}

Nach der Existenz der partiellen Ableitungen beliebiger Ordnung weisen wir nun weiter nach, dass diese jeweils gleichmäßig in $\sigma \in \mathcal S$ beschränkt sind.

\begin{Satz}
\label{satz:ps:rg:partielle_ableitungen_schranke}
    Sei $b := (b_{j})_{j \in \mathbb{N}} \in \mathbb{R}^{\mathbb{N}}$ mit $b_{j} := \alpha^{-1} \norm{\varphi_{j}}_{L_{\infty}(\Omega)}$, dann gilt
    \begin{equation}
    \label{eq:ps:rg:partielle_ableitungen_schranke}
        \sup_{\sigma \in \mathcal S} \norm{\partial^{\nu}_{\sigma} \eta(\sigma)} \leq \frac{\norm{g}_{H^{-1}(\Omega)}}{\alpha} \norm{\nu}_{\ell_{1}(\mathbb{N})}! b^{\nu}.
    \end{equation}

    \begin{Beweis}
        Betrachte die Variationsprobleme, welche von den partiellen Ableitungen $\partial^{\nu}_{\sigma} \eta(\sigma)$ erfüllt werden.
        Wir zeigen zunächst, dass diese durch
        \begin{equation}
        \label{eq:ps:rg:partielle_ableitungen_rekursive_ableitungen}
            a(\partial^{\nu}_{\sigma} \eta(\sigma), \zeta; \sigma)
            = - \sum_{\Set{j \given \nu_{j} \neq 0}} \nu_{j} \skp{\varphi_{j} \partial^{\nu - e_{j}}_{\sigma} \eta(\sigma)}{\zeta}{L_{2}(\Omega)}.
        \end{equation}
        rekursiv dargestellt werden können.
        Die Gültigkeit dieser Darstellung zeigen wir induktiv.

        Den Fall $\norm{\nu}_{\ell_{1}(\mathbb{N})} = 1$ haben wir in \cref{bem:ps:rg:partielle_ableitungen_alternativ_ueber_ableitung_des_operators} bereits gezeigt.
        Betrachte also $\norm{\nu}_{\ell_{1}(\mathbb{N})} > 1$.
        Sei $k \in \mathbb{N}$ ein Index mit $\nu_{k} > 0$, dann definieren wir $\tilde{\nu} := \nu - e_{k}$ und es gilt offenbar $\norm{\tilde\nu}_{\ell_{1}(\mathbb{N})} = \norm{\nu}_{\ell_{1}(\mathbb{N})} - 1$.
        Nach Induktionsvoraussetzung gilt damit
        \begin{equation}
            a(\partial^{\tilde{\nu}}_{\sigma} \eta(\sigma), \zeta; \sigma) + \sum_{\Set{j \given \tilde{\nu}_{j} \neq 0}} \tilde{\nu}_{j} \skp{\varphi_{j} \partial^{\tilde{\nu} - e_{j}}_{\sigma} \eta(\sigma)}{\zeta}{L_{2}(\Omega)} = 0,
        \end{equation}
        wobei nach Definition $\nu_{j} = \tilde{\nu}_{j}$ für $j \neq k$ und $\tilde{\nu}_{k} = \nu_{k} - 1$ ist.
        Partielles Differenzieren dieser Gleichung nach $\sigma_{k}$ analog zu \cref{bem:ps:rg:partielle_ableitungen_alternativ_ueber_ableitung_des_operators} liefert dann die Gleichung
        \begin{align}
            0 &=
                a(\partial^{\nu}_{\sigma} \eta(\sigma), \zeta; \sigma)
                + \skp{\varphi_{k} \partial^{\nu - e_{k}}_{\sigma} \eta(\sigma)}{\zeta}{L_{2}(\Omega)}
                + (\nu_{k} - 1) \skp{\varphi_{k} \partial^{\nu - e_{k}}_{\sigma} \eta(\sigma) }{\zeta}{L_{2}(\Omega)}
           \\&\qquad     + \sum_{\Set{j \neq k \given \nu_{j} \neq 0}} \nu_{j} \skp{\varphi_{j} \partial^{\nu - e_{j}}_{\sigma} \eta(\sigma)}{\zeta}{L_{2}(\Omega)},
        \end{align}
        welche nach Zusammenfassen der Gleichung \cref{eq:ps:rg:partielle_ableitungen_rekursive_ableitungen} entspricht.

        Wählt man nun $v = \partial^{\nu}_{\sigma} \eta(\sigma)$, dann gilt einerseits aufgrund der Koerzivität von $a(\blank, \blank; \sigma)$ die Ungleichung
        \begin{equation}
            a(\partial^{\nu}_{\sigma} \eta(\sigma), \partial^{\nu}_{\sigma} \eta(\sigma); \sigma) \geq \alpha \norm{\partial^{\nu}_{\sigma} \eta(\sigma)}_{H^{1}(\Omega)}^{2}.
        \end{equation}
        Andererseits erhalten wir aus der rekursiven Darstellung \cref{eq:ps:rg:partielle_ableitungen_rekursive_ableitungen} mit Hilfe er Cauchy-Schwarz-Ungleichung die Abschätzung
        \begin{align}
            a(\partial^{\nu}_{\sigma} \eta(\sigma), \partial^{\nu}_{\sigma} \eta(\sigma); \sigma)
            &= - \sum_{\Set{j \given \nu_{j} \neq 0}} \nu_{j} \skp{\varphi_{j} \partial^{\nu - e_{j}}_{\sigma} \eta(\sigma) }{\partial^{\nu}_{\sigma} \eta(\sigma)}{L_{2}(\Omega)}
            \\&\leq \sum_{\Set{j \given \nu_{j} \neq 0}} \nu_{j} \norm{\varphi_{j}}_{L_{\infty}(\Omega)} \norm{\partial^{\nu - e_{j}}_{\sigma} \eta(\sigma)}_{H^{1}(\Omega)} \norm{\partial^{\nu}_{\sigma} \eta(\sigma)}_{H^{1}(\Omega)}.
        \end{align}
        Beide Ungleichungen zusammen ergeben
        \begin{equation}
        \label{eq:ps:rg:partielle_ableitungen_schranke_rekursiv}
            \norm{\partial^{\nu}_{\sigma} \eta(\sigma)}_{H^{1}(\Omega)} \leq \sum_{\Set{j \given \nu_{j} \neq 0}} \nu_{j} \frac{\norm{\varphi_{j}}_{L_{\infty}(\Omega)}}{\alpha} \norm{\partial^{\nu - e_{j}}_{\sigma} \eta(\sigma)}_{H^{1}(\Omega)}.
        \end{equation}

        Um nun die eigentliche Behauptung zu beweisen, verfolgen wir erneut einen Induktionsansatz.
        Sei zunächst $\norm{\nu}_{\ell_{1}(\mathbb{N})} = 0$, dann entspricht
        \begin{equation}
            \norm{\eta(\sigma)}_{H^{1}(\Omega)} \leq \frac{\norm{g}_{H^{-1}(\Omega)}}{\alpha},
        \end{equation}
        der Ungleichung \cref{eq:ps:rg:partielle_ableitungen_schranke} und ist nach \cref{satz:ps:rg:lax_milgram_anwendung} erfüllt.
        Sei also weiter $\norm{\nu}_{\ell_{1}(\mathbb{N})} > 0$, dann gilt für die rekursive Darstellung \cref{eq:ps:rg:partielle_ableitungen_schranke_rekursiv} unter Verwendung der Induktionsvoraussetzung für $\norm{\partial^{\nu - e_{j}}_{\sigma} \eta(\sigma)}_{H^{1}(\Omega)}$ die Abschätzung
        \begin{align}
            \norm{\partial^{\nu}_{\sigma} \eta(\sigma)}_{H^{1}(\Omega)}
            &\leq
            \sum_{\Set{j \given \nu_{j} \neq 0}} \nu_{j} \frac{\norm{\varphi_{j}}_{L_{\infty}(\Omega)}}{\alpha} \norm{\partial^{\nu - e_{j}}_{\sigma} \eta(\sigma)}_{H^{1}(\Omega)}
            \\&\leq
            \sum_{\Set{j \given \nu_{j} \neq 0}} \nu_{j} \frac{\norm{\varphi_{j}}_{L_{\infty}(\Omega)}}{\alpha} \frac{\norm{g}_{H^{-1}(\Omega)}}{\alpha} \norm{\nu - e_{j}}_{\ell_{1}(\mathbb{N})}! b^{\nu - e_{j}}
            \\&=
            \Bigg( \sum_{\Set{j \given \nu_{j} \neq 0}} \nu_{j} \Bigg) \Bigg( \frac{\norm{g}_{H^{-1}(\Omega)}}{\alpha} (\norm{\nu}_{\ell_{1}(\mathbb{N})} - 1)! b^{\nu} \Bigg)
            \\&=
            \frac{\norm{g}_{H^{-1}(\Omega)}}{\alpha} \norm{\nu}_{\ell_{1}(\mathbb{N})}! b^{\nu}
         \end{align}
         und damit die Behauptung.
    \end{Beweis}
\end{Satz}

Zusammenfassend erhalten wir damit die folgende Aussage.
\mfix{Sauber formulieren.}

\begin{Satz}
\label{satz:ps:rg:operatorgleichung_zusammenfassung}
    Sei $\mu \geq \sum_{j = 1}^{\infty} \norm{\varphi_{j}}_{L_{\infty}(\Omega)}$ und $A(\sigma)$ wie in \cref{eq:ps:pf:operator_A_sigma}.
    Dann existiert für jedes $g \in H^{-1}(\Omega)$ und jedes $\sigma \in \mathcal S$ eine eindeutige Lösung $\eta(\sigma)$ der parametrischen Operatorgleichung
    \begin{equation}
        A(\sigma) \eta(\sigma) = g \quad \text{in}~H^{-1}(\Omega).
    \end{equation}
    Weiter hängt diese Lösung $\eta(\sigma)$ analytisch vom Parameter $\sigma$ ab und es gilt die Abschätzung
    \begin{equation}
        \sup_{\sigma \in \mathcal S} \norm{\partial^{\nu}_{\sigma} \eta(\sigma)}_{H^{1}_{0}(\Omega)} \leq \frac{\norm{g}_{H^{-1}(\Omega)}}{\alpha} \norm{\nu}_{\ell_{1}(\mathbb{N})}! b^{\nu},
    \end{equation}
    mit $b = (b_{j})_{j \in \mathbb{N}} \in \ell_{1}(\mathbb{N})$ und $b_{j} = \alpha^{-1} \norm{\varphi_{j}}_{L_{\infty}(\Omega)}$.

    \begin{Beweis}
        Existenz und Eindeutigkeit folgen analog zu \cref{satz:ps:rg:lax_milgram_anwendung} aus dem Lemma von Lax-Milgram, \cref{lem:gl:le:lax_milgram}.
        Die Abschätzung wurde bereits in \cref{satz:ps:rg:partielle_ableitungen_schranke} gezeigt.
        \mdo{Daraus folgern, dass die Lösung analytisch im Parameter ist.}
    \end{Beweis}
\end{Satz}

\mdo{Daraus folgern, dass es für den parabolischen Fall auch gilt.}

\Cref{satz:ps:rg:operatorgleichung_zusammenfassung} wurde bei \textcite[Theorem 4]{Kunoth:2013ef}, unter Verweisung auf die Beweise von \cref{satz:ps:rg:existenz_partieller_ableitungen} und \cref{satz:ps:rg:partielle_ableitungen_schranke} bei \textcite[Theorem 4.2, 4.3]{Cohen:2010kz}, für allgemeinere parametrische Operatoren respektive Bilinearformen nachgewiesen.
Dafür wurde im Wesentlichen gefordert, dass die partiellen Ableitungen der Operatoren beziehungsweise Bilinearformen wohldefiniert sind.
Dies ist bei uns beispielsweise der Fall, wie in den Beweisen von \cref{satz:ps:rg:existenz_partieller_ableitungen} und \cref{satz:ps:rg:partielle_ableitungen_schranke} und an \cref{bem:ps:rg:partielle_ableitungen_alternativ_ueber_ableitung_des_operators} zu sehen ist.

Aufbauend auf das Ergebnis in \cref{satz:ps:rg:operatorgleichung_zusammenfassung} können wir nun, analog zu \textcite[Section 4]{Kunoth:2013ef}, ein ähnliches Ergebnis für das parabolische Variationsproblem nachweisen.
Dies wollen wir anhand der folgenden, aus \cite{Kunoth:2013ef} entnommenen, Punkte motivieren.

\begin{Annahme}[{{\cite[Assumption 1]{Kunoth:2013ef}}}]
\label{ann:ps:rg:kunoth13_assumption1}
    Seien $X$ und $Y$ zwei reflexive Banachräume.
    Die parametrische Familie von Operatoren
    $\Set{ A(\sigma) \in \mathcal L(X, Y') \given \sigma \in \mathcal S }$ sei eine $\mathfrak p$-reguläre Operatorfamilie für ein $0 < \mathfrak p \leq 1$, das heißt,
    \begin{thmenumerate}
        \item $A(\sigma) \in \mathcal L(X, Y')$ sei stetig invertierbar für alle $\sigma \in \mathcal S$ mit gleichmäßig beschränktem Inversen $A{(\sigma)}^{-1} \in \mathcal L(Y', X)$, das heißt, es existiert ein $C_{0} > 0$ mit
        \begin{equation}
            \sup_{\sigma \in \mathcal S} \norm{A{(\sigma)}^{-1}}_{\mathcal L(Y', X)} \leq C_{0},
        \end{equation}
        \item für jedes feste $\sigma \in \mathcal S$ seien die Operatoren $A(\sigma)$ analytisch bezüglich $\sigma$.
        Konkret existiert eine nichtnegative Folge $b = (b_{j})_{j \in \mathbb{N}} \in \ell_{\mathfrak p}(\mathbb{N})$, so dass
        \begin{equation}
            \sup_{\sigma \in \mathcal S} \norm{(A{(0)})^{-1}(\partial^{\nu}_{\sigma} A(\sigma))}_{\mathcal L(X, X)} \leq C_{0} b^{\nu}
        \end{equation}
        für alle $\nu \in \mathfrak F \setminus \{ 0 \}$ gilt.
    \end{thmenumerate}
\end{Annahme}
%
Diese Annahme alleine reicht bereits aus, um ein Analogon zu \cref{satz:ps:rg:operatorgleichung_zusammenfassung} zu erhalten, vergleiche \cite[Theorem 4]{Kunoth:2013ef}.

Weiter wird die obige Annahme von unserem Variationsproblem für $\mathfrak p = 1$ erfüllt, denn in unserem Falle ist $C_{0} = \alpha^{-1}$ und $b_{j} = \alpha^{-1} \norm{\varphi_{j}}_{L_{\infty}(\Omega)}$ und nach Konstruktion insbesondere $b \in \ell_{1}(\mathbb{N})$.\mfix{referenzieren.}

% \begin{Satz}[{{\cite[Theorem 4]{Kunoth:2013ef}}}]
% \label{satz:ps:rg:kunoth13_theorem4}
%     Die parametrische Familie von Operatoren $\Set{ A(\sigma) \in \mathcal L(X, Y') \given \sigma \in \mathcal S }$ erfülle \cref{ann:ps:rg:kunoth13_assumption1} für ein $0 \leq \mathfrak p \leq 1$.
%     Dann existiert für jedes $g \in Y'$ und jedes $\sigma \in \mathcal S$ eine eindeutige Lösung $\eta(\sigma)$ der parametrischen Operatorgleichung
%     \begin{equation}
%         A(\sigma) \eta(\sigma) = g \quad \text{in}~Y'.
%     \end{equation}
%     Weiter hängt die Lösung $\eta(\sigma)$ analytisch vom Parameter $\sigma$ ab und es gilt
%     \begin{equation}
%     \label{eq:ps:rg:kunoth13_theorem4_abschaetzung}
%         \sup_{\sigma \in \mathcal S} \norm{\partial^{\nu}_{\sigma} \eta(\sigma)}_{X} \leq C_{0} \norm{g}_{Y'} \norm{\nu}_{\ell_{1}(\mathbb{N})} ! \tilde{b}^{\nu},
%     \end{equation}
%     wobei $\tilde{b} \in \ell_{\mathfrak p}(\mathbb{N})}$ durch
%     \begin{equation}
%         \tilde{b}_{j} = \frac{b_{j}}{\ln 2} \quad \fa j \in \mathbb{N}
%     \end{equation}
%     definiert ist.
% \end{Satz}

% Bis auf einen konstanten Faktor bei der Abschätzung \cref{eq:ps:rg:kunoth13_theorem4_abschaetzung} entspricht dies den von uns nachgewiesenen Aussagen.

% Der Grund für die Einführung der obigen Ergebnisse von \textcite{Kunoth:2013ef} ist der folgende:
% Aus den Eigenschaften der Familie von parametrischen Operatoren $\Set{ A(\sigma, t) \in \mathcal L(V, V') \given \sigma \in \mathcal S, t \in [0, T] }$ können wir Rückschlüsse auf die parametrische Regularität der Lösungen des parabolischen Variationsproblems ziehen.

Der Grund für die Einführung von \cref{ann:ps:rg:kunoth13_assumption1} ist der folgende:
Erfüllt die Familie von parametrischen Operatoren $\Set{ A(\sigma, t) \in \mathcal L(V, V') \given \sigma \in \mathcal S, t \in [0, T] }$ \cref{ann:ps:rg:kunoth13_assumption1}, dann lässt sich zeigen, dass dies auch für die parametrische Raum-Zeit-Variationsformulierung aus \cref{def:ps:pf:schwache_formulierung} gilt.
Zusammenfassend ergibt sich damit der folgende Satz, welcher eine auf unser Problem abgewandelte Version von \cite[Theorem 21]{Kunoth:2013ef} darstellt.

\begin{Satz}
\label{satz:ps:rg:kunoth13_theorem21}
    Seien $\mathcal X$ und $\mathcal Y$ gegeben wie in \cref{eq:ps:rzvp:ansatzraum_testraum}.
    Sei weiter für jedes $\sigma \in \mathcal S$ der Operator $B(\sigma) \in \mathcal L(\mathcal X, \mathcal Y')$ durch
    \begin{equation}
        \skp{B(\sigma) u}{v}{\mathcal Y' \times \mathcal Y} = b(u, v; \sigma), \quad u \in \mathcal X,~y \in \mathcal Y
    \end{equation}
    für die Bilinearform $b(\blank, \blank; \sigma)$ aus \cref{eq:ps:pf:schwache_formulierung_lhs_b} gegeben.
    Dann ist $B(\sigma)$ für jedes $\sigma \in \mathcal S$ stetig invertierbar und es existieren Konstanten $0 < \beta_{1} \leq \beta_{2} < \infty$ mit
    \begin{equation}
        \label{eq:ps:rg:theorem21_abschaetzungen_normen_B_und_B_inv_parametrisch}
        \sup_{\sigma \in \mathcal S} \norm{B(\sigma)}_{\mathcal L(\mathcal X, \mathcal Y')} \leq \beta_{2} \quad \text{und} \quad  \sup_{\sigma \in \mathcal S} \norm{B(\sigma)^{-1}}_{\mathcal L(\mathcal Y', \mathcal X)} \leq \frac{1}{\beta_{1}}.
    \end{equation}

    Ferner hängen die Lösungen $u(\sigma)$ des parametrischen Raum-Zeit-Variationsproblems \cref{eq:ps:pf:schwache_formulierung} analytisch vom Parameter $\sigma$ ab und es gilt die Abschätzung
    \begin{equation}
        \label{eq:ps:rg:theorem21_apriori_schranke}
        \sup_{\sigma \in \mathcal S} \norm{(\partial^{\nu}_{\sigma} u)(\sigma)}_{\mathcal X} \leq \frac{\norm{f}_{\mathcal Y'}}{\beta_{1}} \norm{\nu}_{\ell_{1}(\mathbb{N})}! b^{\nu}
    \end{equation}
    für alle $\nu \in \mathfrak F$, wobei $f$ wie in~\cref{eq:ps:pf:schwache_formulierung_rhs_f} und $b = (b_{j})_{j \in \mathbb{N}} \in \ell_{1}(\mathbb{N})$ durch $b_{j} = \beta_{1}^{-1} \norm{\varphi_{j}}_{L_{\infty}(\Omega)}$ gegeben sind.

    \begin{Beweis}
        Wir beginnen mit der stetigen Invertierbarkeit von $B(\sigma)$.
        Es bietet sich an, dies durch Anwendung von \cref{satz:gl:le:ss09_theorem51} nachzuweisen, das bedeutet, wir müssen dementsprechend die Voraussetzungen, welche in \cref{ann:gl:le:bilinearform_eigenschaften} zu finden sind, überprüfen.

        Dies wurde in \cref{satz:ps:rzvp:bilinearform_a_eigenschaften} und \cref{kor:ps:rzvp:bilinearform_elliptisch} bereits getan.
        Da wir weiter vom Fall $\mu \geq \sum_{j = 1}^{\infty} \norm{\varphi_{j}}_{L_{\infty}(\Omega)}$ ausgehen, liefert dies von $\omega$ respektive $\sigma$ unabhängige Konstanten $M_{a} = \max\Set{c, 2 \mu}$ und $\alpha = c \gamma_{\Omega}^{2}$, sowie $\lambda = 0$.
        \Cref{satz:gl:le:ss09_theorem51} liefert damit die stetige Invertierbarkeit von $B(\sigma)$ für jedes $\sigma \in \mathcal S$.
        Weiter erhalten wir damit durch \cref{kor:gl:le:ss09_theorem51_ungleichungen} auch die Konstanten $\beta_{1}$ und $\beta_{2}$, welche durch die obigen $M_{a}$ und $\alpha$ wiederum unabhängig von $\omega$ beziehungsweise $\sigma$ sind.

        Die analytische Abhängigkeit und die zugehörige Abschätzung \cref{eq:ps:rg:theorem21_apriori_schranke} lassen sich nun vollkommen analog zu \cref{lem:ps:rg:norm_abschaetzung}, \cref{satz:ps:rg:existenz_partieller_ableitungen} und \cref{satz:ps:rg:partielle_ableitungen_schranke} nachweisen, weswegen wir dies an dieser Stelle auch nicht weiter ausführen wollen.
    \end{Beweis}
\end{Satz}

\section{Exkurs: Periodische Randbedingungen} % (fold)
\label{sec:ps:pr:periodische_randbedingungen}

Da wir bisher mit homogenen Randbedingungen gearbeitet haben und dies auch für den Rest der Arbeit fortführen wollen, gehen wir an dieser Stelle auf den Fall periodischer Randbedingungen ein.
Dabei wird sich herausstellen, das es nur sehr wenige Unterschiede zum bereits betrachteten homogenen Fall gibt.

Wir beschränken uns der Einfachheit halber auf den Fall, dass $\Omega = \bigtimes_{i = 1}^{n} (0, l_{i}) \subset \mathbb{R}^{n}$ ein beschränkter offener Quader ist, wobei $l_{i} \in \mathbb{R}_{+}$ für $i = 1 \dots n$ sei.

\begin{Definition}
    Sei $\mathcal C_{\text{per}}^{\infty}(\Omega) \subset \mathcal C^{\infty}(\mathbb{R}^{n})$ die Teilmenge der $\Omega$-periodischen Funktionen.
    Als Sobolev-Raum $H^{1}_{\text{per}}(\Omega)$ der auf $\Omega$ periodischen Funktionen bezeichnen wir den Abschluss von $\mathcal C_{\text{per}}^{\infty}(\Omega)$ bezüglich der $H^{1}$-Norm.

    Weiter bezeichnen wir mit $H^{1}_{\text{per}, 0}(\Omega)$ den Sobolev-Raum $\Omega$-periodischer Funktionen mit Mittelwert Null und definieren diesen als den Quotientenraum
    \begin{equation}
        H^{1}_{\text{per}, 0}(\Omega) = H^{1}_{\text{per}}(\Omega) / \mathbb{R}.
    \end{equation}
\end{Definition}
%
\mfix{Stimmt das folgende?}
%
Wir wählen nun $V = H^{1}_{\text{per}, 0}(\Omega)$ und $H = L_{2}(\Omega) / \mathbb{R}$; diese Räume liefern uns ein Gelfand-Tripel $H^{1}_{\text{per}, 0}(\Omega) \denseinclusion L_{2}(\Omega) / \mathbb{R} \denseinclusion (H^{1}_{\text{per}, 0}(\Omega))'$ und wir betrachten nun die Bilinearform $a$ aus \cref{satz:ps:rzvp:bilinearform_a_eigenschaften} auf diesen Räumen, also
\begin{equation}
    \begin{aligned}
        &a(\blank, \blank) \colon H^{1}_{\text{per}, 0}(\Omega) \times H^{1}_{\text{per}, 0}(\Omega) \to \mathbb{R}, \\
        &(\eta, \zeta) \mapsto c\skp{\grad \eta}{\grad \zeta}{L_{2}(\Omega)} + \skp{\omega \eta}{\zeta}{L_{2}(\Omega)} + \mu \skp{\eta}{\zeta}{L_{2}(\Omega)}.
    \end{aligned}
\end{equation}

Wir wollen \cref{satz:ps:rzvp:bilinearform_a_eigenschaften} respektive \cref{kor:ps:rzvp:bilinearform_elliptisch} analog nachweisen.
Schaut man sich dazu den Beweis von \cref{satz:ps:rzvp:bilinearform_a_eigenschaften} an, dann sieht man, dass nur die G\aa{}rding-Ungleichung ein Problem darstellt, da hierfür eine Poincaré-Friedrichs-Ungleichung verwendet wurde, welche auf $H^{1}_{\text{per}}(\Omega)$ in dieser Form nicht gilt, jedoch auf $H^{1}_{\text{per}, 0}(\Omega)$.

\begin{Lemma}[Poincaré-Ungleichung]
    Sei $\Omega \subset \mathbb{R}^{n}$ ein beschränkter offener Quader.
    Dann existiert eine Konstante $C = C(\Omega) \in \mathbb{R}_{+}$, so dass
    \begin{equation}
        \norm{u}_{L_{2}(\Omega)} \leq C \norm{\grad u}_{L_{2}(\Omega)} \quad \fa u \in H^{1}_{\text{per}, 0}(\Omega)
    \end{equation}
    gilt.
\end{Lemma}

% Insbesondere ist $\norm{\grad \blank}_{L_{2}(\Omega)}$ auf $H^{1}_{\text{per}, 0}(\Omega)$ eine zu $\norm{\blank}_{H^{1}(\Omega)}$ äquivalente Norm.

Damit können wir den Beweis vollkommen analog führen, lediglich die dabei auftretende Konstante $\alpha > 0$ wird eine andere sein; der exakte Wert dieser ist aber nicht weiter von Belang.
Insbesondere erhalten wir damit alle nachfolgenden Aussagen zur Variationsformulierung der parabolischen partiellen Differentialgleichung auch für den Fall periodischer Randbedingungen, da sonst keine weiteren Komplikationen auftreten.

Der Wechsel auf den Quotientenraum $H^{1}_{\text{per}, 0}(\Omega)$ stellt durch die Eigenschaft, dass die betrachteten Funktionen Mittelwert Null haben, sicher, dass wir eine eindeutige Lösung der schwachen Formulierung haben.
Da wir bei der numerischen Umsetzung im Allgemeinen aber nicht explizit an einer Lösung mit Mittelwert Null interessiert sind, muss darauf geachtet werden, den korrekten Repräsentanten aus der Äquivalenzklasse der Lösung zu wählen.

% section periodische_randbedingungen (end)
