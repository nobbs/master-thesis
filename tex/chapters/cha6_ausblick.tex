% -*- root: ../main.tex -*-

\documentclass[../main.tex]{subfiles}
\begin{document}

\chapter{Fazit \& Ausblick} % (fold)
\label{chapter:ausblick}

In diesem abschließenden Kapitel wollen wir die Ergebnisse dieser Arbeit noch einmal diskutieren.
Dabei betrachten wir diese hauptsächlich unter dem Aspekt der Anwendbarkeit auf die selbstkonsistente Feldtheorie aus der \nameref{chapter:einleitung}.

Analog zum Aufbau der Arbeit beginnen wir mit den funktionalanalytischen Ergebnissen aus \cref{chapter:propagator_differentialgleichung}, bevor wir weiter auf die zugrundeliegende Diskretisierung mittels \nameref{chapter:galerkin} und zuletzt auf die darauf aufbauende \nameref{chapter:rbm} eingehen.

\paragraph{Funktionalanalytische Ergebnisse.} % (fold)
\label{par:funktionalanalytische_ergebnisse}

\blindtext

\paragraph{Petrov-Galerkin-Diskretisierung.} % (fold)
\label{par:petrov_galerkin_diskretisierung}

\blindtext

\paragraph{Reduzierte-Basis-Methode.} % (fold)
\label{par:reduzierte_basis_methode}

\blindtext

\end{document}
