% -*- root: ../main.tex -*-

\documentclass[../main.tex]{subfiles}
\begin{document}

\chapter{Inhalt der Begleit-DVD} % (fold)
\label{cha:inhalt_der_begleit_dvd}

Dieser Arbeit liegt eine DVD bei, welche die Implementierungen der beschriebenen Verfahren, sowie die in \cref{cha:Beispiele} angeführten Beispiele enthält.
Weiterhin findet man den Inhalt der DVD identisch auch als Git-Repository unter \url{https://github.com/nobbs/thesis}.

Die Implementierungen sind vollständig in \textcite{Matlab} gehalten und daher plattformunabhängig.
Neben dem eigentlichen MATLAB-Softwarepaket wird nur die \emph{Optimization Toolbox} benötigt für die in \cref{scm} beschrieben Successive Constraint Method benötigt.

\begin{figure}[tb]
    \dirtree{%
    .1 /.
    .2 code.
    .3 examples.
    .4 datasets.
    .3 lib.
    .3 src.
    .3 test.
    .2 config.
    .2 doc.
    .3 index.html.
    .2 tex.
    .2 README.md.
    }
    \caption{Ausschnitt der Verzeichnisstruktur}
    \label{fig:dvd}
\end{figure}

Ein Ausschnitt der wichtigsten Elemente der Verzeichnisstruktur findet sich in \cref{fig:dvd}.

\section{Sonstiger Kram} % (fold)
\label{sec:sonstiger_kram}


\begin{Lemma}[Berechnung der Rieszchen Darstellung]
\label{lemma:berechnung_rieszsche_darstellung}
    Sei $X$ ein endlichdimensionaler Hilbertraum mit Basis ${\phi_i}_{i=1}^{N}$.
    Sei weiter $g \in X'$.
    Der Koeffizientenvektor $\vec{v} \in \mathbb{R}^{N}$ der Rieszschen Darstellung $v_g = \sum_{i=1}^{N} v_{i} \phi_{i} \in X$ von $g$, das heißt, es gilt $\skp{v_g}{w}{X} = \skp{g}{w}{X' \times X}$ für alle $w \in X$, lässt sich durch das Gleichungssystem $\mat{K}\vec{v} = \vec{g}$ berechnen, wobei $\mat{K} = [\skp{\phi_{i}}{\phi_{j}}{X}]_{i,j}$ die Massematrix zum inneren Produkt auf $X$ sei und weiter $\vec{g} = [\skp{g}{\phi_{i}}{X' \times X}]_{i}$ sei.
\end{Lemma}

% section sonstiger_kram (end)

\end{document}
