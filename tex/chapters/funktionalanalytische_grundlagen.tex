%!TEX root = ../main.tex

\chapter{Funktionalanalytische Grundlagen} % (fold)
\label{cha:funktionalanalytische_grundlagen}

\section{Orthogonale Funktionen und Polynome}
\label{sec:orthogonale_funktionen_und_polynome}

\begin{Satz}[Orthogonalität trigonometrischer Funktionen]
\label{satz:trigonometrische_funktionen_orthogonal}
    Seien $k, l \in \mathbb{N}$.
    Dann gilt
    \begin{align}
        \skprod{\sin(\pi k x)}{\sin(\pi l x)}_{L_{2}([0, 1])} = \frac{1}{2} \delta_{kl}
        \quad\text{und}\quad
        \skprod{\cos(\pi k x)}{\cos(\pi l x)}_{L_{2}([0, 1])} = \frac{1}{2} \delta_{kl}.
    \end{align}
\end{Satz}

\begin{Definition}[Legendre-Polynome]
\label{definition:legendre_polynome}
    Sei $I = [-1, 1]$.
    Die Legendre-Polynome $L_{n} \in \Pi_{n}$ sind definiert durch
    \begin{equation}
        L_{n}(x) = \frac{1}{2^{n}n!}\frac{\diff^{n}}{\diff x^{n}} (x^{2} - 1)^{n}.
    \end{equation}
    Durch die Transformation $x \mapsto 2x - 1$ erhält man die auf das Interval $[0, 1]$ geshifteten Legendre-Polynome $\tilde L_{n}$.
\end{Definition}

\begin{Satz}[Orthogonalität der Legendre-Polynome]
\label{satz:legendre_polynome_orthogonal}
    Die Legendre-Polynome $L_{n}$ sind orthogonal bezüglich der $L_{2}([-1, 1])$-Norm, denn es gilt
    \begin{equation}
        \skprod{L_{n}}{L_{m}}_{L_{2}([-1, 1])} = \frac{2}{2n + 1} \delta_{n m}.
    \end{equation}
    Auch für die geshifteten Legendre-Polynome $\tilde L_{n}$ gilt Orthogonalität, denn es ist
    \begin{equation}
        \skprod{\tilde L_{n}}{\tilde L_{m}}_{L_{2}([0, 1])} = \frac{1}{2n + 1} \delta_{n m}.
    \end{equation}
\end{Satz}

\begin{Bemerkung}
\label{satz:legendre_polynome_rekursion}
    Die Legendre-Polynome $L_{n}$ erfüllen die Rekursionsformel
    \begin{equation}
        n L_{n}(x) = (2n - 1) x L_{n-1}(x) - (n - 1) L_{n-2}(x), \quad L_{0}(x) = 1, L_{1}(x) = x.
    \end{equation}
    Analog gilt für die erste Ableitung $L_{n}'$ die Rekursionsformel
    \begin{equation}
        (n - 1) L_{n}'(x) = (2n -1) x L_{n-1}'(x) - n L_{n-2}'(x), \quad L_{0}'(x) = 0, L_{1}'(x) = 1.
    \end{equation}
\end{Bemerkung}

\section{Sonstiges} % (fold)
\label{sec:sonstiges}

% \begin{Lemma}
%     $\mathcal C^{0}([a, b]; X)$ liegt dicht in $L_{p}(a, b; X)$ für $1 \leq p < \infty$.
% \end{Lemma}

% TODO: zitieren
\begin{Satz}[Poincaré-Friedrichs-Ungleichung, vgl. {{\cite[Lemma 89.4]{HankeBourgeois:2009fk}}}]
\label{satz:grundlagen:poincare_friedrichs_ungleichung}
    Sei $\Omega \subset \mathbb{R}^{n}$ offen, beschränkt und mit Lipschitz-Rand.
    Dann existiert eine Konstante $\gamma_{\Omega} > 0$ mit
    \begin{equation}
        \label{eq:grundlagen:poincare_friedrichs_ungleichung}
        \norm{\grad u}_{L_{2}(\Omega)} \geq \gamma_{\Omega} \norm{u}_{H^{1}(\Omega)} \quad \fa u \in H^{1}_{0}(\Omega).
    \end{equation}
\end{Satz}

\begin{Satz}[Poincaré-Friedrichs-Ungleichung, vgl. {{\cite[Theorem II.1.7]{Braess:2007wm}}}]
    Es sei $\Omega \subset \mathbb{R}^{n}$ beschränkt und in einem $n$-dimensionalen Würfel mit Seitenlänge $s$ enthalten.
    Dann gilt
    \begin{equation}
        (1 + s)^{m} \abs{u}_{H^{m}} \geq \norm{u}_{H^{m}} \geq \abs{u}_{H^{m}} \quad \text{für alle}~u \in H^{m}_{0}(\Omega).
    \end{equation}
\end{Satz}

\begin{Lemma}[vgl {{\cite[Remark 2.1.48]{Sauter:9_WoPZ0Y}}}]
\label{lemma:sauter:2.1.48}
    Seien $X$ und $Y$ zwei reflexive Banachräume und $a \colon X \times Y \to \mathbb{R}$ eine Bilinearform.
    Finden wir für jedes $x \in X$ ein $y_{x} \in Y$, so dass
    \begin{equation}
        \label{eq:lemma:sauter:2.1.48:eq1}
        \abs{a(x, y_{x})} \geq C_{1} \norm{x}_{X}^{2} \quad \text{und} \quad \norm{y_{x}}_{Y} \leq C_{2} \norm{x}_{X}
    \end{equation}
    mit von $x$ und $y_{x}$ unabhängigen Konstanten $C_{1}, C_{2} > 0$ gilt, dann folgt daraus die inf-sup-Bedingung
    \begin{equation}
    \label{eq:lemma:sauter:2.1.48:inf_sup}
        \inf_{0 \neq x \in X} \sup_{0 \neq y \in Y} \frac{a(x, y)}{\norm{x}_{X}\norm{y}_{Y}} \geq \gamma > 0
    \end{equation}
    mit $\gamma = \frac{C_{1}}{C_{2}}$.

    \begin{Beweis}
        Seien $x \in X$ und $y_{x} \in Y$ so, dass \eqref{eq:lemma:sauter:2.1.48:eq1} erfüllt ist.
        Dann gilt
        \begin{align}
            \inf_{0 \neq x \in X} \sup_{0 \neq y \in Y} \frac{\abs{a(x, y)}}{\norm{x}_{X} \norm{y}_{Y}}
            &\geq
            \inf_{0 \neq x \in X} \frac{\abs{a(x, y_{x})}}{\norm{x}_{X} \norm{y_{x}}_{Y}}
            \\&\geq
            \inf_{0 \neq x \in X} \frac{C_{1} \norm{x}^{2}_{X}}{\norm{x}_{X} C_{2} \norm{x}_{X}}
            =
            \frac{C_{1}}{C_{2}}
            > 0.
        \end{align}
    \end{Beweis}
\end{Lemma}

% section sonstiges (end)
