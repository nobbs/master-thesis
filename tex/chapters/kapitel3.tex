%!TEX root = ../main.tex

\chapter{Überschrift in Arbeit}

In diesem Kapitel konzentrieren wir uns nun auf die in der Polymerchemie motivierte parabolische partielle Differentialgleichung.
Eine ausführliche Herleitung findet sich bei \textcite{Fredrickson:2006th}.

\section{Motivation} % (fold)
\label{sec:motivation}

% section motivation (end)

\section{Vereinfachte Variante} % (fold)
\label{sec:vereinfachte_variante}

Wir betrachten in diesem Abschnitt zunächst eine vereinfachte Variante der vorgestellten Differentialgleichung.
Zunächst ignorieren wir den Wechsel des Feldes $\omega$ ab einem bestimmten Zeitpunkt und erhalten dadurch einen autonomen linearen Differentialoperator $A$.
Weiter schränken wir uns auf homogene Dirichlet- statt periodischen Randbedingungen ein.

Unter diesen Gegebenheiten bietet es sich an, die Hilberträume als $V = H^{1}_{0}(\Omega)$ und $H = L_{2}(\Omega)$ zu wählen.
Bekanntlich sind diese separabel und es existiert eine dichte stetige Einbettung von $H^{1}_{0}(\Omega)$ in $L_{2}(\Omega)$.
Wegen $(H^{1}_{0}(\Omega))' = H^{-1}(\Omega)$ ergibt dies das Gelfand-Tripel
\begin{equation}
    H^{1}_{0}(\Omega) \denseinclusion L_{2}(\Omega) \denseinclusion H^{-1}(\Omega).
\end{equation}
Wie zuvor verwenden wir $\skprod{\blank}{\blank}$ mit entsprechendem Index sowohl für die Skalarprodukte als auch für die duale Paarung auf $H^{-1}(\Omega) \times H^{1}_{0}(\Omega)$.

Um obige partielle Differentialgleichung in das Setting aus \autoref{sec:lineare_evolutionsgleichungen} zu übertragen, definieren wir einen linearen Operator $A$ als
\begin{equation}
    \label{eq:def_op_A}
    A \colon H^{1}_{0}(\Omega) \to H^{-1}(\Omega), \quad \eta \mapsto A \eta = - c \Delta \eta + \omega \eta
\end{equation}
und weiter die zugehörige Bilinearform
\begin{equation}
    a \colon H^{1}_{0}(\Omega) \times H^{1}_{0}(\Omega) \to \mathbb{R}, \quad a(\eta, \zeta) = \skprod{A \eta}{\zeta}_{L_{2}(\Omega)}.
\end{equation}
Diese lässt sich unter Verwendung der Greenschen Formeln (TODO!) auch schreiben als
\begin{equation}
    \begin{aligned}
        a(\eta, \zeta)
        &= \skprod{- c \Delta \eta + \omega \eta}{\zeta}_{L_{2}(\Omega)}
        = - c \skprod{\Delta \eta}{\zeta}_{L_{2}(\Omega)} + \skprod{\omega \eta}{\zeta}_{L_{2}(\Omega)}
        \\&= c \skprod{\grad \eta}{\grad \zeta}_{L_{2}(\Omega)} + \skprod{\omega \eta}{\zeta}_{L_{2}(\Omega)}.
    \end{aligned}
\end{equation}

Diese Bilinearform ist stetig und erfüllt eine G\aa{}rding-Ungleichung, wie das folgende Lemma zeigt.

\begin{Lemma}
\label{lemma:a_bf_bounded_garding}
    Seien $c \in \mathbb{R}_{+}$, $\omega \in L_{\infty}(\Omega)$ und
    \begin{equation}
    \label{eq:bf_a}
        a \colon H^{1}_{0}(\Omega) \times H^{1}_{0}(\Omega) \to \mathbb{R}, \quad a(\eta, \zeta) = c \skprod{\grad \eta}{\grad \zeta}_{L_{2}(\Omega)} + \skprod{\omega \eta}{\zeta}_{L_{2}(\Omega)}.
    \end{equation}
    Dann erfüllt $a$ die Eigenschaften aus \thref{annahme:eigenschaften_bf_a}:
    \begin{thmenumerate}
        \item\label{lemma:a_bf_bounded_garding:1}
        \emph{Stetigkeit:} es gilt
        \begin{equation}
            \abs{a(\eta, \zeta)} \leq M_{a} \norm{\eta}_{H^{1}(\Omega)} \norm{\zeta}_{H^{1}(\Omega)} \quad \text{für alle}~\eta, \zeta \in H^{1}_{0}(\Omega)
        \end{equation}
        mit $M_{a} = \max\Set{c, \norm{\omega}_{L_{\infty}(\Omega)} } \geq 0$.
        \item\label{lemma:a_bf_bounded_garding:2}
        \emph{G\aa{}rding-Ungleichung:} es gilt
        \begin{equation}
                a(\eta, \eta) + \lambda \norm{\eta}_{L_{2}(\Omega)}^{2} \geq \alpha \norm{\eta}_{H^{1}(\Omega)}^{2} \quad \text{für alle}~\eta \in H^{1}_{0}(\Omega)
        \end{equation}
        mit $\alpha = c \gamma_{\Omega}^{2} > 0$ und $\lambda = \norm{\omega}_{L_{\infty}(\Omega)} \geq 0$, wobei $\gamma_{\Omega}$ die Poincaré-Friedrichs-Konstante ist.
    \end{thmenumerate}

    \begin{Beweis}
    Wir zeigen zunächst die Stetigkeit.
    Seien dazu $\eta, \zeta \in H^{1}_{0}(\Omega)$ beliebig.
    Unter Verwendung der Dreiecks- und der Cauchy-Schwarz-Ungleichung erhalten wir
    \begin{align}
        \abs{a(\eta, \zeta)}
        &= \abs{c \skprod{\grad \eta}{\grad \zeta}_{L_{2}(\Omega)} + \skprod{\omega \eta}{\zeta}_{L_{2}(\Omega)}}
        \\&\leq c \abs{\skprod{\grad \eta}{\grad \zeta}_{L_{2}(\Omega)}} + \abs{\skprod{\omega \eta}{\zeta}_{L_{2}(\Omega)}}
        \\&\leq c \norm{\grad \eta}_{L_{2}(\Omega)} \norm{\grad \zeta}_{L_{2}(\Omega)} + \norm{\omega}_{L_{\infty}(\Omega)} \norm{\eta}_{L_{2}(\Omega)} \norm{\zeta}_{L_{2}(\Omega)}
        \\&\leq \max \Set{ c, \norm{\omega}_{L_{\infty}(\Omega)} } \norm{\eta}_{H^{1}(\Omega)} \norm{\zeta}_{H^{1}(\Omega)}.
    \end{align}

    Für die G\aa{}rding-Ungleichung seien nun $\eta \in H^{1}_{0}(\Omega)$ und $\lambda \in \mathbb{R}$.
    Wir betrachten
    \begin{align}
        a(\eta, \eta) + \lambda \norm{\eta}^{2}_{L_{2}(\Omega)}
        &= c \norm{\grad \eta}^{2}_{L_{2}(\Omega)} + \skprod{\omega \eta}{\eta}_{L_{2}(\Omega)} + \lambda \skprod{\eta}{\eta}_{L_{2}(\Omega)}
        \\&= c \norm{\grad \eta}^{2}_{L_{2}(\Omega)} + \skprod{(\omega + \lambda) \eta}{\eta}_{L_{2}(\Omega)}.
    \end{align}
    Wählen wir nun $\lambda = \norm{\omega}_{L_{\infty}(\Omega)} \geq 0$, dann gilt $\omega + \lambda \geq 0$ fast überall in $\Omega$ und wir erhalten die Abschätzung
    \begin{align}
        a(\eta, \eta) + \lambda \norm{\eta}^{2}_{L_{2}(\Omega)}
        &\geq c \norm{\grad \eta}^{2}_{L_{2}(\Omega)},
        \intertext{woraus wir durch Anwenden der Poincaré-Friedrichs-Ungleichung \ref{satz:grundlagen:poincare_friedrichs_ungleichung}}
        a(\eta, \eta) + \lambda \norm{\eta}^{2}_{L_{2}(\Omega)}
        &\geq c \gamma_{\Omega}^{2} \norm{\eta}^{2}_{H^{1}(\Omega)}
    \end{align}
    folgern.
    \end{Beweis}
\end{Lemma}

Unter diesen Gegebenheiten erhalten wir nach \autoref{sec:lineare_evolutionsgleichungen} eine sachgemäß gestellte Raum-Zeit-Variationsformulierung.
Ansatz- und Testfunktionenraum ergeben sich mit den konkret gewählten Hilberträumen zu
\begin{equation}
    \label{eq:var_ansatzraum_testraum}
    \mathcal X = L_{2}(I; H^{1}_{0}(\Omega)) \cap H^{1}(I; H^{-1}(\Omega))
    \quad \text{und} \quad
    \mathcal Y = L_{2}(I; H^{1}_{0}(\Omega)) \times L_{2}(\Omega).
\end{equation}
Das Variationsproblem lautet damit:
    Gegeben ein $g \in L_{2}(I; H^{-1}(\Omega))$ und ein $u_{0} \in L_{2}(\Omega)$. Finde ein $u \in \mathcal X$ mit
    \begin{equation}
        \label{eq:varprob}
        b(u, v) = f(v) \quad \text{für alle}~v \in \mathcal Y,
    \end{equation}
    wobei $b(\blank, \blank) \colon \mathcal X \times \mathcal Y \to \mathbb{R}$ die durch
    \begin{equation}
        \label{eq:buv}
        b(u, v)
            = \int_{I} \skprod{u_{t}(t)}{v_{1}(t)}_{L_{2}(\Omega)} + a(u(t), v_{1}(t)) \diff t + \skprod{u(0)}{v_{2}}_{L_{2}(\Omega)}
    \end{equation}
    gegebene Bilinearform und $f(\blank) \colon \mathcal Y \to \mathbb{R}$ definiert ist durch
    \begin{equation}
        \label{eq:var_all_f_wiederholung}
        f(v) = \int_{I} \skprod{g(t)}{v_{1}(t)}_{L_{2}(\Omega)} \diff t + \skprod{u_{0}}{v_{2}}_{L_{2}(\Omega)}.
    \end{equation}

Aus \thref{thm:schwab09:theorem51} und \thref{thm:schwab09:theorem51:ungleichungen} erhalten wir nun die Wohldefiniertheit des obigen Variationsproblems und zugleich Schranken für die Operatoren.

\begin{Korollar}
\label{korollar:2.2}
    Seien $\mathcal X$ und $\mathcal Y$ gegeben wie in \eqref{eq:var_ansatzraum_testraum} und sei $B \colon \mathcal X \to \mathcal Y'$ definiert durch
    \begin{equation}
        \skprod{Bu}{v}_{\mathcal Y' \times \mathcal Y}  = b(u, v), \quad u \in \mathcal X,~ v \in \mathcal Y,
    \end{equation}
    mit $b(\blank, \blank)$ wie in \eqref{eq:buv}.
    Dann ist $B$ stetig invertierbar und es gilt
    \begin{equation}
        \norm{B}_{\mathcal L(\mathcal X, \mathcal Y')}
        \leq
        \frac{\sqrt{2 \max\Set{1, c^{2}, \norm{\omega}_{L_{\infty}(\Omega)}^{2}} + M_{e}^{2}}}{\max\Set{\sqrt{1 + 2 \norm{\omega}_{L_{\infty}(\Omega)}^{2} \rho^{4}}, \sqrt{2} }}
    \end{equation}
    und
    \begin{equation}
        \norm{B^{-1}}_{\mathcal L( \mathcal Y', \mathcal X)}
        \leq \frac{e^{2 T \norm{\omega}_{L_{\infty}(\Omega)}} \max\Set{\sqrt{1 + 2 \norm{\omega}_{L_{\infty}(\Omega)}^{2} \rho^{4}}, \sqrt{2}} \sqrt{2 \max\Set{c^{-2} \gamma_{\Omega}^{-4}, 1} + M_{e}^{2}}}{\min\Set{c^{-1} \gamma_{\Omega}^{2}, c \gamma_{\Omega}^{2} \norm{\omega}_{L_{\infty}(\Omega)}^{-2}, c \gamma_{\Omega}^{2} }}.
        % \leq
        % \frac{\max\{\sqrt{ 1 + 2 \norm{\omega}_{L_{\infty}(\Omega)} \rho^{4}}, \sqrt{2} \}}{e^{-2 \norm{\omega}_{L_{\infty}(\Omega)} T}}
        % \frac{\sqrt{2 \max\{ 1, \sigma^{-2} \gamma_{\Omega}^{-4} \} + M_{e}^{2}}}{\min\{ \sigma \gamma_{\Omega}^{2} \norm{\omega}_{L_{\infty}(\Omega)}^{-2}, \sigma \gamma_{\Omega}^{2} \}}
    \end{equation}
    mit $M_{e}$ und $\rho$ wie in \eqref{eq:var_all_M_e} respektive \eqref{eq:var_all_rho}.
\end{Korollar}

% section vereinfachte_variante (end)

\section{Parametrische Variante} % (fold)
\label{sec:parametrische_variante}

Wir wollen nun aus dem gerade beschriebenen Variationsproblem eine parametrische Variante gewinnen und aufbauend auf \autoref{sec:parametrisches_problem} Regularität bezüglich des Parameters folgern.
Dazu müssen wir den Operator $A \in \mathcal L(V, V')$ aus \eqref{eq:def_op_A} zunächst zu einem parametrischen Operator $A(\sigma)$ mit $\sigma \in \mathcal S$, wobei $\mathcal S \subset \mathbb{R}^{\mathbb{N}}$ ein geeigneter Parameterraum ist, umschreiben.
Dabei beschränken wir uns auf den Fall affiner parametrischer Abhängigkeit \eqref{eq:all_affiner_operator}.
Der Einfachheit halber wählen wir $\mathcal S = [-1, 1]^{\mathbb{N}}$, das heißt $\mathcal S$ sei die Einheitskugel aus $\ell_{\infty}(\mathbb{N})$.

Sei $\Set{ \varphi_{j} }_{j \in \mathbb{N}} \subset L_{\infty}(\Omega)$ ein noch näher zu bestimmendes, passend gewähltes Funktionensystem und $\sigma \in \mathcal S$.
Wir entwickeln nun $\omega$ formal in eine Reihe der Form
\begin{equation}
    \label{eq:reihenentwicklung_omega}
    \omega(\blank; \sigma) = \sum_{j = 1}^{\infty} \sigma_{j} \varphi_{j}.
\end{equation}
Offenbar ist für die Konvergenz der Reihe \eqref{eq:reihenentwicklung_omega} hinreichend, dass $\Set{ \norm{\varphi_{j}}_{L_{\infty}(\Omega)} }_{j \in \mathbb{N}} \in \ell_{1}(\mathbb{N})$ gilt, insbesondere folgt daraus
\begin{equation}
    \norm{\omega(\blank; \sigma)}_{L_{\infty}(\Omega)} \leq \sum_{j = 1}^{\infty} \norm{\varphi_{j}}_{L_{\infty}(\Omega)} < \infty \quad \fa \sigma \in \mathcal S.
\end{equation}
% Diese Eigenschaft wird auch benötigt, denn dadurch erhalten wir aus \thref{lemma:2.2} die für \thref{thm:kunoth:theorem21} notwendigen, von $\sigma$ unabhängigen, Schranken $\beta_{1}$ und $\beta_{2}$.
Damit ist die Wahl des Funktionensystems $\Set{ \varphi_{j} }_{j \in \mathbb{N}} \subset L_{\infty}(\Omega)$ ist entscheidend für die Konvergenz von \eqref{eq:reihenentwicklung_omega}, aber auch für die Erfüllbarkeit von \thref{thm:kunoth:assumption1} respektive \thref{thm:kunoth:assumption2},
und wird in den nächsten Abschnitten genauer behandelt.

% \subsection{Affiner Operator} % (fold)
% \label{ssub:entwicklung_von_}

Wir wollen den Operator $A$ aus \eqref{eq:def_op_A} als affin parametrischen Operator der Form
\begin{equation}
    \label{eq:aff_zerlegung_A}
    A(\sigma) = \hat A + \sum_{j \geq 1} \sigma_{j} A_{j}
\end{equation}
auffassen, beziehungsweise als Bilinearformen
\begin{equation}
     \label{eq:aff_zerelgung_A_bf}
     a(\eta, \zeta; \sigma) = \hat a(\eta, \zeta) + \sum_{j \geq 1} \sigma_{j} a_{j}(\eta, \zeta), \quad \eta, \zeta \in V.
 \end{equation}
Dazu entwickeln wir $\omega$ in eine Reihe der Form \eqref{eq:reihenentwicklung_omega}, das heißt wir erhalten
\begin{equation}
    \label{eq:omega_reihenentwicklung}
    \omega(\blank; \sigma) \colon \Omega \to \mathbb{R}, \quad x \mapsto \omega(x; \sigma) = \sum_{j \geq 1} \sigma_{j} \varphi_{j}(x)
\end{equation}
mit $\sigma \in \mathcal S$.
Eine naheliegende affine Aufteilung des Operators $A$ erhalten wir damit durch die Wahl
\begin{equation}
    \label{eq:affine_zerlegung_A_def}
    \hat A = - c \Delta, \qquad
    A_{j} = \varphi_{j}, \quad j \geq 1.
\end{equation}
Die zugehörigen Bilinearformen lassen sich ebenfalls direkt angeben, denn es gilt
\begin{equation}
    \hat a(\eta, \zeta) = \skprod{\grad \eta}{\grad \zeta}_{L_{2}(\Omega)}, \qquad a_{j}(\eta, \zeta) = \skprod{\varphi_{j} \eta}{\zeta}_{L_{2}(\Omega)}, \quad j \geq 1.
\end{equation}

Die daraus resultierende Raum-Zeit-Variationsformulierung lautet nun:
\begin{Problem}
    Gegeben ein $g \in L_{2}(I; H^{-1}(\Omega))$ und ein $u_{0} \in L_{2}(\Omega)$.
    Finde für alle $\sigma \in \mathcal S$ ein $u(\sigma) \in \mathcal X$ mit
    \begin{equation}
        \label{eq:varprob_2}
        b(u, v; \sigma) = f(v) \quad \text{für alle}~v \in \mathcal Y,
    \end{equation}
    wobei $b(\blank, \blank; \sigma) \colon \mathcal X \times \mathcal Y \times \mathcal S \to \mathbb{R}$ die durch
    \begin{equation}
        \label{eq:buv_2}
        b(u, v; \sigma)
            = \int_{I} \skprod{u_{t}(t)}{v_{1}(t)}_{L_{2}(\Omega)} + a(u(t), v_{1}(t); \sigma) \diff t + \skprod{u(0)}{v_{2}}_{L_{2}(\Omega)}
    \end{equation}
    gegebene Bilinearform und $f(\blank) \colon \mathcal Y \to \mathbb{R}$ definiert ist durch
    \begin{equation}
        \label{eq:var_all_f_wiederholung_2}
        f(v) = \int_{I} \skprod{g(t)}{v_{1}(t)}_{L_{2}(\Omega)} \diff t + \skprod{u_{0}}{v_{2}}_{L_{2}(\Omega)}.
    \end{equation}
\end{Problem}


% \section{Periodische Randbedingungen} % (fold)
% \label{sec:periodische_randbedingungen}

% section periodische_randbedingungen (end)
