%!TEX root = ../main.tex

\chapter{Grundlagen} % (fold)
\label{cha:grundlagen}

Verschiedenste Grundlagen, die hier abgedeckt werden sollten.
Dazu gehören folgende Abschnitte:

\section{Funktionalanalytische Grundlagen} % (fold)
\label{sec:funktionalanalytische_grundlagen}

\subsection{Bochner-Räume} % (fold)
\label{sub:bochner_r_ume}

\begin{Definition}[Bochner-Räume]
    Sei $X$ ein Banachraum, $(a, b) \subset \mathbb{R}$, $- \infty \leq a < b \leq \infty$, ein offenes Intervall und $1 \leq p < \infty$.
    Wir nennen $L_{p}(a, b; X)$ einen Bochner-Raum und meinen damit die Menge der Äquivalenzklassen $L_{p}$-integrierbarer Funktionen $f \colon [a, b] \to X$, das heißt alle Lebesgue-messbaren Funktionen auf $[a, b]$ mit
    \begin{equation}
        \norm{f}_{L_{p}(a, b; X)} \coloneqq \left( \int_{a}^{b} \norm{f(t)}_{X}^{p} \diff t \right)^{\frac 1 p} < \infty.
    \end{equation}
    Analog bezeichnen wir mit $L_{\infty}(a, b; X)$ die Menge der Äquivalenzklassen die fast überall auf $[a, b]$ beschränkt sind, das heißt
    \begin{equation}
        \norm{f}_{L_{\infty}(a, b; X)} \coloneqq \esssup_{t \in [a, b]} \norm{f(t)}_{X} < \infty.
    \end{equation}
\end{Definition}

\begin{Lemma}
    Sei $1 \leq p \leq \infty$. Dann ist $L_{p}(a, b; X)$ ein Banachraum.
    Ist $H$ ein Hilbertraum, dann ist insbesondere auch $L_{2}(a, b; H)$ ein Hilbertraum.
\end{Lemma}

\begin{Lemma}[Eigenschaften]
    \begin{enumerate}
        \item Ist $[a, b] \subset \mathbb{R}$ ein endliches Intervall, dann gilt die stetige Einbettung
        \begin{equation}
            L_{q}(a, b; X) \hookrightarrow L_{p}(a, b; X), \qquad q \geq p \geq 1.
        \end{equation}
        \item Sind $X$ und $Y$ Banachräume mit $X \hookrightarrow Y$, dann gilt die stetige Einbettung
        \begin{equation}
            L_{p}(a, b; X) \hookrightarrow L_{p}(a, b; Y), \qquad 1 \leq p \leq q.
        \end{equation}
    \end{enumerate}
\end{Lemma}


\section{Space-Time-Variational-Formulation} % (fold)
\label{sec:space_time_variational_formulation}

Seien $V$ und $H$ Hilberträume mti $V \hookrightarrow H$.
Dann bildet $V \hookrightarrow H \cong H' \hookrightarrow V'$ ein sogenanntes
Gelfand-Triple.
Wir schreiben $\skprod{\cdot}{\cdot}_{V}$, $\skprod{\cdot}{\cdot}_{H}$ für das
Skalarprodukt auf $V$ und $\skprod{\cdot}{\cdot}_{V' \times V}$ für das
\emph{Duality Pairing} auf $V' \times V$.:w

Sei $0 < T < \infty$ und $I \coloneqq [0, T] \subset \mathbb{R}$.
Für fast alle $t \in I$ sei $A(t) \in \mathcal L(V, V')$ und
\begin{equation}
    a(t; \cdot, \cdot) \colon V \times V \to \mathbb{R}
\end{equation}
die zugehörige Bilinearform, das heißt es gilt
\begin{equation}
    \skprod{A(t)u}{v}_{H} = a(t; u, v)
\end{equation}

Wir fordern von der Bilinearform $a(\cdot; \cdot, \cdot)$, dass Konstanten $M_a,
\alpha > 0$ und $\lambda \in \mathbb{R}$ existieren, so dass für fast alle $t
\in I$ gilt
\begin{itemize}
    \item \emph{Stetigkeit:} für alle $u, v \in V$ gilt
        \begin{equation}
            \label{eq:stetigkeit}
            \abs{a(t; u, v)} \leq M_a \norm{u}_V \norm{v}_V,
        \end{equation}
    \item \emph{Garding-Ungleichung:} für alle $u \in V$ gilt
        \begin{equation}
            \label{eq:garding-inequality}
            a(t; u, u) + \lambda \norm{u}^2_H \geq \alpha \norm{u}^2_V.
        \end{equation}
\end{itemize}

Wir wollen nun die parabolische partielle Differentialgleichung
\begin{equation}
    \label{eq:}
    \begin{aligned}
        u_t(t) + A(t) u(t) &= g(t) &\qquad \text{in}~V'\\
        u(0) &= u_0 \qquad \text{in}~H.
    \end{aligned}
\end{equation}

Nach Multiplikation mit einer Testfunktion $v \coloneqq (v_1, v_2) \in \mathcal{Y}$ und Integration nach Ort und Zeit erhalten wir
\begin{equation}
    \label{eq:bilinearform}
    b(u, v) = F(v),
\end{equation}
wobei
\begin{equation}
    b(u,v) = \int_{I} \skprod{u_{t}(t)}{v_{1}(t)}_{H} + a(t; u(t), v_{1}(t)) \diff t + \skprod{u(0)}{v_{2}}_{H}
\end{equation}
und
\begin{equation}
    F(v) = \int_{I} \skprod{g(t)}{v_{1}(t)}_{H} \diff t + \skprod{u_{0}}{v_{2}}_{H}
\end{equation}
