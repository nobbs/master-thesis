% -*- root: main.tex -*-

\documentclass[
  draft,
  % final,
  a4paper,
  % BCOR=1cm,
  % twoside,
  twoside=semi,
  % oneside,
  toc=bibliography,
  toc=listof,
  chapterprefix=true,
  version=last,
  cleardoublepage=empty,
]{scrbook}

%!TEX root = main.tex

%%% Grundlegendes

% Verwende etex als zugrundeliegende Implementierung (-> mehr Speicher für Tex!)
\usepackage{etex}
% Encoding
\usepackage[utf8]{inputenc}


%%% Schriften, Lokalisierung und Typographie

% T1 Schriftsystem verwenden
\usepackage[T1]{fontenc}
% LModern als Standard-Schrift
\usepackage{lmodern}
% Typographische Kleinigkeiten
\usepackage{microtype}
% Fraktur-Schriftsatz
\usepackage{eufrak}

%% Lokalisierung

% Deutsche Silbentrennung
\usepackage[english,ngerman]{babel}
% Deutsche Anführungszeichen
\usepackage[babel,german=quotes]{csquotes}

%% Weitere Schriften vordefinieren

\newcommand{\altfont}{\fontfamily{qpl}}
% \newcommand{\altsffont}{\fontfamily{phv}}
\newcommand{\altsffont}{\fontfamily{pfr}\fontseries{l}}

% Zeilenabstand anpassen (MUST NOT USE!)
% \usepackage{setspace}
% \setstretch{1.1}

% Textsatzbereich nach unten vergrößern
% TODO: richtig konfigurieren
% \usepackage[a4paper,twoside,bottom=4cm]{geometry}

%%% Graphiken und Farben

% Graphiken und Farben
\usepackage{xcolor}
\usepackage{graphicx}

% Vordefinierte Farben laden
\definecolor{jgu_rot}{RGB}{193,0,42}
\definecolor{jgu_hellgrau}{RGB}{172,172,172}
\definecolor{jgu_dunkelgrau}{RGB}{99,99,99}
\definecolor{matlab_blau}{rgb}{0.00000,0.44700,0.74100}
\definecolor{matlab_orange}{rgb}{0.85000,0.32500,0.09800}


% Caption anpassen und Subfigures erlauben
\usepackage[style=base,font+=small,labelfont+=bf,margin=1em]{caption}
\usepackage{subcaption}

% Tikz ist kein Zeichenprogramm!
\usepackage{tikz}
\usepackage{pgfplots}
\pgfplotsset{compat=1.12}
\usetikzlibrary{patterns}

% Standalone-Paket, damit die Tikz Bilder extern auch funktionieren
\usepackage{standalone}


%%% Bibliographie

\usepackage[%
    backend=biber,
    style=alphabetic,
    backref=true,
    firstinits=true
]{biblatex}

% Fix für doi2bib generierte Bibliography-Einträge
\newcommand\mathplus{+}


%%% Grundlegende Mathematik-Pakete

% Standardpackages
\usepackage{amsmath}
\usepackage{amssymb}
\usepackage{stmaryrd}
\usepackage{mathtools}

% "Bessere" Theorem-Umgebungen
\usepackage[
    amsmath,
    thmmarks,
    hyperref,
]{ntheorem}


%%% Hyperref

\usepackage[
	colorlinks=true,
	linkcolor=jgu_rot,          % color of internal links (change box color with linkbordercolor)
	citecolor=jgu_rot,        % color of links to bibliography
	filecolor=jgu_rot,      % color of file links
	urlcolor=jgu_rot,
    hypertexnames=false
]{hyperref}


%%% Clevere Referenzen

\usepackage[german,noabbrev,nameinlink,capitalise]{cleveref}
\crefname{equation}{}{}

% Nummern nur für referenzierte Gleichungen
\usepackage{autonum}


%%% Layout und Stil

% Fancyhdr-Ersatz für scrbook
\usepackage[
    automark,
    % headsepline,
    draft=false
]{scrlayer-scrpage}

% Inhaltsverzeichnis anpassen (Schriften angleichen!)
\usepackage{tocstyle}
\usetocstyle{KOMAlike}

%% Chapter-Style und mehr

% Alle Überschriften in TeX Gyre Pagella
\addtokomafont{disposition}{\normalfont\altfont\selectfont}
% Alle Überschriften in Dunkelgrau
% \addtokomafont{disposition}{\color{jgu_dunkelgrau}}

% Einzelne Optionen bei Bedarf
% \addtokomafont{chapter}{\normalfont\altfont\selectfont}
% \addtokomafont{chapter}{\color{black}}
% \addtokomafont{section}{\normalfont\altfont\selectfont}
% \addtokomafont{subsection}{\normalfont\altfont\selectfont}
% \addtokomafont{subsubsection}{\normalfont\altfont\selectfont}
% \addtokomafont{minisec}{\normalfont\altfont\selectfont}
% \addtokomafont{paragraph}{\normalfont\altfont\selectfont}
\addtokomafont{paragraph}{\bfseries}

% Kopfzeile anpassen
\addtokomafont{pageheadfoot}{\color{jgu_dunkelgrau}\normalfont\altfont\selectfont}
% Seitenzahl anpassen
\addtokomafont{pagenumber}{\color{jgu_dunkelgrau}\normalfont\altfont\selectfont}

% Zitatblock für Kapitelanfang
\renewcommand*\dictumwidth{.95\linewidth}
\renewcommand*\dictumrule{}
\renewcommand*\dictumauthorformat[1]{--- #1}
\addtokomafont{dictumtext}{\color{jgu_dunkelgrau}\normalfont\altsffont\fontsize{9}{12}\selectfont\itshape}
\addtokomafont{dictumauthor}{\color{jgu_dunkelgrau}\normalfont\altsffont\fontsize{8}{12}\selectfont}

% Kapitel-Stil
\renewcommand*{\chapterheadstartvskip}{\vspace*{3\baselineskip}}
\renewcommand*{\chapterheadendvskip}{\vspace*{2\baselineskip}}
\renewcommand*{\chapterformat}{%
				\raggedright
				\color{jgu_rot}
                \altfont\fontsize{60}{30}\selectfont\thechapter
                \fontsize{20}{30}\scshape\selectfont\enskip\chapappifchapterprefix
                }
\renewcommand*{\raggedchapter}{\raggedleft}

% Inhaltsverzeichnis
% \addtokomafont{sectionentrypagenumber}{\color{black}}
\addtokomafont{chapterentrypagenumber}{\color{black}\rmfamily}
\addtokomafont{chapterentry}{\rmfamily\bfseries}


%%% Verzeichnisse

% Akronymverzeichnis
\usepackage{acro}
\acsetup{
    page-ref=paren,
    pages=first,
    page-name={siehe S.\@\,},
}

% Symbolverzeichnis
\usepackage[german,intoc]{nomencl}
\setlength{\nomitemsep}{-\parsep}
\makenomenclature


% ntheorem-Umgebungen laden
%%
%% This is file `ntheorem.std',
%% generated with the docstrip utility.
%%
%% The original source files were:
%%
%% ntheorem.dtx  (with options: `standard')
%%
%% IMPORTANT NOTICE:
%%
%% For the copyright see the source file.
%%
%% Any modified versions of this file must be renamed
%% with new filenames distinct from ntheorem.std.
%%
%% For distribution of the original source see the terms
%% for copying and modification in the file ntheorem.dtx.
%%
%% This generated file may be distributed as long as the
%% original source files, as listed above, are part of the
%% same distribution. (The sources need not necessarily be
%% in the same archive or directory.)
\def\filedate{2011/08/15}
\def\docdate{2011/08/15}
\def\fileversion{1.33}
\def\basename{ntheorem}
%% This file may be distributed and/or modified under the
%% conditions of the LaTeX Project Public License, either version 1.2
%% of this license or (at your option) any later version.
%% The latest version of this license is in
%%    http://www.latex-project.org/lppl.txt
%% and version 1.2 or later is part of all distributions of LaTeX
%% version 1999/12/01 or later.
%% \CharacterTable
%%  {Upper-case    \A\B\C\D\E\F\G\H\I\J\K\L\M\N\O\P\Q\R\S\T\U\V\W\X\Y\Z
%%   Lower-case    \a\b\c\d\e\f\g\h\i\j\k\l\m\n\o\p\q\r\s\t\u\v\w\x\y\z
%%   Digits        \0\1\2\3\4\5\6\7\8\9
%%   Exclamation   \!     Double quote  \"     Hash (number) \#
%%   Dollar        \$     Percent       \%     Ampersand     \&
%%   Acute accent  \'     Left paren    \(     Right paren   \)
%%   Asterisk      \*     Plus          \+     Comma         \,
%%   Minus         \-     Point         \.     Solidus       \/
%%   Colon         \:     Semicolon     \;     Less than     \<
%%   Equals        \=     Greater than  \>     Question mark \?
%%   Commercial at \@     Left bracket  \[     Backslash     \\
%%   Right bracket \]     Circumflex    \^     Underscore    \_
%%   Grave accent  \`     Left brace    \{     Vertical bar  \|
%%   Right brace   \}     Tilde         \~}

\theoremnumbering{arabic}
\theoremstyle{plain}
\RequirePackage{latexsym}
% \theoremsymbol{\ensuremath{_\Box}}
\theorembodyfont{\itshape}
\theoremheaderfont{\normalfont\bfseries}
\theoremseparator{}
\newtheorem{Theorem}{Theorem}[chapter]
\newtheorem{theorem}[Theorem]{Theorem}
\newtheorem{Satz}[Theorem]{Satz}
\newtheorem{satz}[Theorem]{Satz}
\newtheorem{Proposition}[Theorem]{Proposition}
\newtheorem{proposition}[Theorem]{Proposition}
\newtheorem{Lemma}[Theorem]{Lemma}
\newtheorem{lemma}[Theorem]{Lemma}
\newtheorem{Korollar}[Theorem]{Korollar}
\newtheorem{korollar}[Theorem]{Korollar}
\newtheorem{Corollary}[Theorem]{Corollary}
\newtheorem{corollary}[Theorem]{Corollary}

\theorembodyfont{\upshape}
\newtheorem{Example}[Theorem]{Example}
\newtheorem{example}[Theorem]{Example}
\newtheorem{Beispiel}[Theorem]{Beispiel}
\newtheorem{beispiel}[Theorem]{Beispiel}
\newtheorem{Bemerkung}[Theorem]{Bemerkung}
\newtheorem{bemerkung}[Theorem]{Bemerkung}
\newtheorem{Anmerkung}[Theorem]{Anmerkung}
\newtheorem{anmerkung}[Theorem]{Anmerkung}
\newtheorem{Remark}[Theorem]{Remark}
\newtheorem{remark}[Theorem]{Remark}
\newtheorem{Definition}[Theorem]{Definition}
\newtheorem{definition}[Theorem]{Definition}
\newtheorem{Annahme}[Theorem]{Annahme}
\newtheorem{annahme}[Theorem]{Annahme}

\theoremstyle{nonumberplain}
\theoremheaderfont{\scshape}
\theorembodyfont{\normalfont}
\theoremseparator{.}
\theoremsymbol{\ensuremath{_\square}}
\RequirePackage{amssymb}
\newtheorem{Proof}{Proof}
\newtheorem{proof}{Proof}
\newtheorem{Beweis}{Beweis}
\newtheorem{beweis}{Beweis}
\qedsymbol{\ensuremath{_\square}}

\theoremstyle{nonumberplain}
\theoremheaderfont{\normalfont\bfseries}
\theoremseparator{}
\theorembodyfont{\normalfont}
\theoremsymbol{}
\RequirePackage{amssymb}
\newtheorem{Problem}{Problem}
\newtheorem{problem}{Problem}
\qedsymbol{}

\theoremclass{LaTeX}
\endinput
%%
%% End of file `ntheorem.std'.



%%% Alles andere

% Querformatseiten
\usepackage{pdflscape}

% Enumerates anpassen
\usepackage{enumitem}
% Numerierte Umgebung für Theoreme definieren
\newlist{thmenumerate}{enumerate}{1}
\setlist[thmenumerate]{label={\upshape(\roman*)}, align=left, widest=iii, leftmargin=*}
\crefname{thmenumeratei}{}{}

% Konturen um Text zeichnen
\usepackage[outline]{contour}
\contourlength{0.09em}

% Gepunktete Linien
\usepackage{dashrule}


%%% "Debugging"

% Labels am Seitenrand anzeigen
\usepackage[
    inline,
    final
]{showlabels}

% todos
\usepackage[%
    german,
    colorinlistoftodos,
    % disable
]{todonotes}

% Todos vordefinieren
\newcommand{\mdo}[1]{\todo[color=yellow!50]{#1}}
\newcommand{\mfix}[1]{\todo[color=red!50]{#1}}
\newcommand{\mwarn}[1]{\todo[color=blue!50]{#1}}

% Blindtext
\usepackage{blindtext}
\blindmathtrue{}

% etoolbox?
\usepackage{etoolbox}

% \usepackage{showframe}



%!TEX root = main.tex

% griechisches Alphabet anpassen
\renewcommand{\epsilon}{\varepsilon}
\renewcommand{\theta}{\vartheta}


%%% Operatoren und anderes mathematisches Geplänkel

% Beträge und Normen
\DeclarePairedDelimiter{\abs}{\lvert}{\rvert}
\DeclarePairedDelimiter{\norm}{\lVert}{\rVert}

% Gauß-Klammern
\DeclarePairedDelimiter{\ceil}{\lceil}{\rceil}
\DeclarePairedDelimiter{\floor}{\lfloor}{\rfloor}

% Skalarprodukte und Duale Paarungen
\newcommand{\skprod}[2]{\left\langle#1,#2\right\rangle}
\newcommand{\Skp}[3]{\left\langle#1,#2\right\rangle_{#3}}
\newcommand{\skp}[3]{\langle#1,#2\rangle_{#3}}
\newcommand{\Dual}[2]{\left(#1,#2\right)}
\newcommand{\dual}[2]{(#1,#2)}

% Inneres und Äußeres
\newcommand{\Int}[1]{#1^\circ}
\newcommand{\Ext}[1]{\overline{#1}}

% Diverse Operatoren
\DeclareMathOperator{\spn}{span}
\DeclareMathOperator{\ee}{e}
\DeclareMathOperator{\ii}{i}
\newcommand{\Transp}{^{\mathrm{T}}}
\newcommand{\Stern}{^{*}}
\newcommand{\grad}{\nabla}
\DeclareMathOperator{\divergenz}{div}
\newcommand{\hesse}{\nabla^2}

% Einschränkung einer Funktion
\newcommand\restr[2]{\ensuremath{\left.#1\right|_{#2}}}

% Ordentlicher Befehl um Mengen zu setzen
\providecommand\given{} % so it exists
\newcommand\SetSymbol[1][]{
   \nonscript\,#1\vert\nonscript\,\mathopen{}\allowbreak}
\DeclarePairedDelimiterX\Set[1]{\lbrace}{\rbrace}{ \renewcommand\given{\SetSymbol[\delimsize]} #1 }

% Vektoren und Matrizen anders setzen
\renewcommand{\vec}[1]{\mathbf{#1}}
\newcommand{\mat}[1]{\mathbf{#1}}

% Differential-d für Integral ordentlich setzen
\newcommand*\diff{\mathop{}\!\mathrm{d}}
\newcommand*\Diff[1]{\mathop{}\!\mathrm{d^#1}}

% wesentliches Supremum
\DeclareMathOperator*{\esssup}{ess\,sup}
\newcommand\blank{{\mkern2mu\cdot\mkern2mu}}

% "Definition-Gleich"
% FIXME: deprecated
\newcommand\deq{\coloneqq}

% dichte Einbettung
% FIXME: zu hoch für Fließtext!
\newcommand\denseinclusion{\stackrel{d}{\hookrightarrow}}

% Schrift einfach anpassen
\newcommand{\changefont}[3]{
\fontfamily{#1} \fontseries{#2} \fontshape{#3} \selectfont}


%%% Textbausteine

\newcommand{\fa}{\text{für alle}~}

% um fremde Begriffe einheitlich mit Übersetzung zu setzen
\newcommand\foreign[2]{#1 \textit{#2}}

%!TEX root = main.tex

\newcommand{\titel}{- in Arbeit - Parabolische partielle Differentialgleichungen und Reduzierte-Basis-Methoden}
\newcommand{\art}{Masterarbeit}
\newcommand{\autor}{Alexej Disterhoft}
\newcommand{\fach}{Mathematik}
\newcommand{\matrikelnr}{2669611}
\newcommand{\erstgutachter}{JProf. Dr. Thorsten Raasch}
\newcommand{\zweitgutachter}{???}
\newcommand{\monat}{Irgendwann}
\newcommand{\jahr}{2015}
\newcommand{\ort}{Mainz}
\newcommand{\logo}{figures/misc/logo_gross.eps}

\hypersetup{pdfinfo={
  Title=\titel,
  Author=\autor}
}

%!TEX root = main.tex

\DeclareAcronym{scft}{
	short            = SCFT,
	long             = selbstkonsistente Feldtheorie,
	long-plural-form = selbstkonsistenten Feldtheorie,
	short-plural     = {},
	foreign          = \foreign{engl.}{self-consistent field theory}
}

\DeclareAcronym{rbm}{
	short = RBM,
	long  = Reduzierte-Basis-Methode
}

\DeclareAcronym{fem}{
	short = FEM,
	long = Finite-Elemente-Methode
}

\DeclareAcronym{bnb}{
	short = BNB,
	long = Banach-Ne{\v c}as-Babu{\v s}ka-Theorem
}

\DeclareAcronym{SCM}{
	short = SCM,
	long = Successive Constraint Method
}

% -*- root: main.tex -*-

\newglossaryentry{symb:stetige_inklusion} {
    name={\ensuremath{\hookrightarrow}},
    description={Stetige Einbettung}
}
\newglossaryentry{symb:bochner_skp} {
    name={\ensuremath{[f, g]}},
    description={Kurzschreibweise für \ensuremath{\int_{T} \skprod{f(t)}{g(t)} \diff t}}
}


\addbibresource{literature.bib}

\begin{document}
    %%% Frontmatter-Teil
    \frontmatter{}

    % Deckblack / Titelseite
    %!TEX root = ../main.tex

\thispagestyle{plain}
\begin{titlepage}
\begin{center}
% $\;$\\[2em]
\includegraphics[scale=1.5,clip,trim=0em 2em 0em 2.8em]{\logo}\\[3em]
%
% {\fontfamily{qpl}\selectfont{\LARGE{\textbf{\titel}}\par}}
{\fontfamily{qpl}\selectfont{\LARGE{\textbf{Numerische Behandlung von \\Self-Consistent Field Theory-Modellen\\ mittels Reduzierter-Basis-Methoden}}\par}}
\normalsize$\;$\\[1em]
{\large{\textbf{\art}}}\\[1em]
{\normalsize
am Institut für Mathematik,\\
Fachbereich Physik, Mathematik und Informatik\\
der Johannes Gutenberg-Universität\\
in Mainz
}\\[6em]
%
{\large{\textbf{\autor}}}\\[0.4em]
{\normalsize{geboren in Leonidowka}}\\[4em]
%
\begin{tabular}{p{3cm}p{7cm}}\\
Erstgutachter:  & \quad \erstgutachter\\[1.2ex]
Zweitgutachter: & \quad \zweitgutachter\\[3ex]
\end{tabular}

{\ort,~\today}
% {\ort,~\monat~\jahr}
%
\end{center}
\end{titlepage}


    % Danksagung
    % %!TEX root = ../main.tex

\thispagestyle{plain}
\begin{titlepage}
\dictum[Andy Weir, \textit{The Martian}]{\enquote{I guess you could call it a \enquote{failure}, but I prefer the term \enquote{learning
experience}.}}
\vfill{}
\begin{flushright}
% \emph{Für meine Eltern.}
\end{flushright}
\end{titlepage}


    % Abstrakt
    % %!TEX root = ../main.tex

\chapter{Zusammenfassung} % (fold)
\label{cha:Zusammenfassung}

\blindtext

% chapter Zusammenfassung (end)


    % Inhaltsverzeichnis
    \tableofcontents

    % TODO-Liste
    % \listoftodos

    %%% Hauptteil
    \mainmatter{}

    % Kapitel einbinden
    \subfile{chapters/cha1_einleitung}
    \subfile{chapters/cha2_grundlagen}
    \subfile{chapters/cha3_propagator_dgl}
    \subfile{chapters/cha4_galerkin}
    \subfile{chapters/cha5_rbm}
    \subfile{chapters/cha6_ausblick}

    % % % -*- root: ../main.tex -*-

\iftoggle{dictum}{
    \setchapterpreamble[ul][0.6\textwidth]{%
        \dictum[Robert Heinlein, \textit{Time Enough For Love}]{\enquote{Progress isn't made by early risers. It's made by lazy men trying to find easier ways to do something.}}
        \vspace*{2\baselineskip}
    }
}{}
\chapter{Der eindimensionale Fall}
\label{sec:der_eindimensionale_fall}
\label{cha:der_eindimensionale_fall}

\todo[inline]{Kapitel ordentlich überarbeiten. Mittlerweile besser, aber noch deutlich verbesserungswürdig!}
\todo[inline]{Sind ein paar notationelle Kleinigkeiten zu korrigieren, also mal ordentlich durchlesen!}
\todo[inline]{Die Abschätzung in Satz 4.4. ist schon heftig. Da bleibt ja so gut wie kein Spielraum für den Faktor K...}

In diesem Kapitel beschränken wir uns zunächst auf den vereinfachten Fall einer Raumdimension.
Es sei also $\Omega \subset \mathbb{R}$ ein Intervall.
Ohne Beschränkung der Allgemeinheit wählen wir $\Omega = [0, L]$ für ein $0 < L < \infty$.

\todo[inline]{Eventuell wäre, insbesondere mit der gewünschten Anwendung bei SCFT-Verfahren, eine Entwicklung in Cosinus-Funktionen (oder direkt Fourier-Reihen) sinnvoller.
Insbesondere wären Cosinus-Funktionen achsensymmetrisch und automatisch nicht-homogen; beides Eigenschaften, welche die Felder $\omega$ bei der SCFT oftmals aufweisen.}


Als Ansatzfunktionen für eine geeignete Entwicklung des Operators $A$ in eine affin parametrische Darstellung wählen wir Sinusfunktionen, welche zusätzlich gewichtet werden, so dass wir die gewünschten Konvergenzeigenschaften erhalten.
Zusätzlich zu den Sinusfunktionen fügen wir eine konstante Funktion $\varphi_{0}$ hinzu, um so inhomogene Randbedingungen für $\omega$ zuzulassen.

Da die Parameter $\sigma$ weiterhin aus der Menge $\mathcal S = [-1, 1]^{\mathbb{N}}$ kommen, skalieren wir die Ansatzfunktionen mit einem Faktor $K \in \mathbb{R}_{+}$.
Dieser Faktor wird später durch die Anforderungen, die wir durch die gewünschte analytische Abhängigkeit vom Parameter erhalten, genauer bestimmt.

Konkret kommt nun
\begin{equation}
    \label{eq:sinusfunktionen_ansatz}
    \varphi_{0} = K, \qquad
    \varphi_{j} = \frac{K}{(\pi j)^{1 + \epsilon}} \sin(\tfrac{\pi j}{L} \blank), \quad j \geq 1,
\end{equation}
als Funktionensystem $\Set{ \varphi_{j} }_{j \geq 0}$ zum Einsatz.
Damit schreiben wir den parametrischen Faktor $\omega$ der linearen Evolutionsgleichung aus dem vorherigen Kapitel als
\begin{equation}
    w(\blank; \sigma) \colon \Omega \to \mathbb{R}, \quad w(x; \sigma) = \sum_{j = 0}^{\infty} \sigma_{j} \varphi_{j}(x)
\end{equation}
mit $\sigma \in \mathcal S$.
Damit schreibt sich die affine Zerlegung des Differentialoperators $A = - c \Delta + \omega$ als
\begin{equation}
    A = \hat A + \sum_{j = 0}^{\infty} \sigma_{j} A_{j}
\end{equation}
mit
\begin{equation}
    \label{eq:1d:affine_zerlegung}
    \hat A = - c \Delta, \qquad A_{j} = \varphi_{j}
\end{equation}
und den zugehörigen Bilinearformen
\begin{equation}
    \hat a(\eta, \zeta) = c \skprod{\grad \eta}{\grad \zeta}_{L_{2}(\Omega)}, \qquad a_{j}(\eta, \zeta) = \skprod{\varphi_{j} \eta}{\zeta}_{L_{2}(\Omega)}.
\end{equation}


Zunächst rechnen wir nun verschiedene, später benötigte, Normen nach.
\begin{Lemma}
    Es gilt
    \begin{alignat}{2}
        \norm{\varphi_{0}}_{L_{\infty}(\Omega)} &= K,
        \qquad&
        \norm{\varphi_{j}}_{L_{\infty}(\Omega)} &= \frac{K}{(\pi j)^{1 + \epsilon}} , \quad j \geq 1,
    \intertext{sowie}
        \norm{\varphi_{0}}_{H^{1}(\Omega)}  &= K \sqrt{L},
        \qquad&
        \norm{\varphi_{j}}_{H^{1}(\Omega)}  &= \frac{K \sqrt{L^{2} + (\pi j)^{2}}}{\sqrt{2L} (\pi j)^{1 + \epsilon}}
        , \quad j \geq 1.
    \end{alignat}
\end{Lemma}

\begin{Lemma}
    Die Funktionen $\Set{ \varphi_{j} }_{j \geq 1}$ bilden ein Orthogonalsystem in $H^{1}(\Omega)$, denn es gilt
    \begin{equation}
        \skprod{\varphi_{j}}{\varphi_{k}}_{H^{1}(\Omega)} = \begin{cases}
            \frac{K^{2}}{2L} \frac{L^{2} + (\pi j)^{2}}{(\pi j)^{2(1 + \epsilon)}}
            ,   &j = k \\
            0,          &j \neq k.
        \end{cases}
    \end{equation}
\end{Lemma}

\begin{Lemma}
    Sei $\sigma \in \mathcal S$ und $\epsilon > 0$, dann konvergiert obiges $\omega(\blank; \sigma)$ in $L_{\infty}(\Omega)$.
    Ist $\epsilon > 1$, dann gilt auch Konvergenz in $H^{1}_{0}(\Omega)$.

    \begin{Beweis}
        Sei zunächst $\epsilon > 0$.
        Da $\mathcal S = [0, 1]^{\mathbb{N}}$ ist, erhalten wir die Konvergenz in $L_{\infty}(\Omega)$ nach dem Weierstraßschen Majorantenkriterium via
        \begin{align}
            \sum_{j = 0}^{\infty} \norm{\sigma_{j} \varphi_{j}}_{L_{\infty}(\Omega)}
            &= \sum_{j = 0}^{\infty} \abs{\sigma_{j}} \norm{\varphi_{j}}_{L_{\infty}(\Omega)}
             \leq \norm{\varphi_{0}}_{L_{\infty}(\Omega)} + \sum_{j = 1}^{\infty}  \norm{\varphi_{j}}_{L_{\infty}(\Omega)}
            \\&= K + \sum_{j = 1}^{\infty} \frac{K}{(\pi j)^{1 + \epsilon}}
            = K + \frac{K}{\pi^{1 + \epsilon}} \sum_{j = 1}^{\infty} \frac{1}{j^{1+\epsilon}}
        \end{align}
        Diese Reihe konvergiert bekanntlich für alle $\epsilon > 0$, womit wir bereits die Konvergenz von $\omega$ in $L_{\infty}(\Omega)$ erhalten.

        Sei nun $\epsilon > 1$.
        Betrachte
        \begin{align}
            \sum_{j = 0}^{\infty} \norm{\sigma_{j} \varphi_{j}}_{H^{1}(\Omega)}
            &= \sum_{j = 0}^{\infty} \abs{\sigma_{j}} \norm{\varphi_{j}}_{H^{1}(\Omega)}
            \leq  \norm{\varphi_{0}}_{H^{1}(\Omega)} + \sum_{j = 1}^{\infty} \norm{\varphi_{j}}_{H^{1}(\Omega)}
            \\&= K \sqrt{L} + \sum_{j = 1}^{\infty} \frac{K \sqrt{L^{2} + (\pi j)^{2}}}{\sqrt{2L} (\pi j)^{1 + \epsilon}}
            \\&= K \sqrt{L} + \frac{K}{\sqrt{2L}} \sum_{j = 1}^{\infty} \frac{\sqrt{L^{2} + (\pi j)^{2}}}{(\pi j)^{1 + \epsilon}}
            \\&\leq K \sqrt{L} + \frac{K}{\sqrt{2L}} \sum_{j = 1}^{\infty} \frac{L + \pi j}{(\pi j)^{1 + \epsilon}}
            \\&= K \sqrt{L} + \frac{K}{\sqrt{2L}} \sum_{j = 1}^{\infty} \frac{L}{(\pi j)^{1 + \epsilon}} + \frac{K}{\sqrt{2L}} \sum_{j = 1}^{\infty} \frac{1}{(\pi j)^{\epsilon}}
        \end{align}
        Wegen $\epsilon > 1$ konvergiert sowohl die erste als auch die zweite Reihe.
        Zusammen liefert dies die Konvergenz in $H^{1}_{0}(\Omega)$.
    \end{Beweis}
\end{Lemma}

Wir wollen nun die Regularität von $\omega(\blank; \sigma)$ in Abhängigkeit vom Parameter $\sigma$ nachweisen.

\todo[inline]{Anpassen! Lässt sich die Schranke verbessern?}
\begin{Satz}
\label{satz:regularitaet_nachrechnen}
    Seien $\epsilon > 0$ und $0 < \kappa < 1$ so gewählt, dass
    \begin{equation}
        \sum_{j = 1}^{\infty} \frac{1}{j^{2 + \epsilon}} \leq \frac{(\kappa c (\tfrac{\pi}{L})^{2} - K) \pi^{2 + \epsilon}}{4 C_{\infty} K L^{3/2}}
    \end{equation}
    gilt,
    wobei $c$ und $K$ die Konstanten aus \eqref{eq:def_op_A} respektive \eqref{eq:sinusfunktionen_ansatz} und $C_{\infty}$ die Einbettungskonstante von $H^{1}_{0}(\Omega) \hookrightarrow L_{\infty}(\Omega)$ sind.
    Dann erfüllt die affine Zerlegung $\Set{\hat A, A_{j} \given j \in \mathbb{N}_{0}}$ \thref{thm:kunoth:assumption2}.

    \begin{Beweis}
        Wir weisen zunächst die inf-sup-Bedingungen \eqref{eq:kunoth:ass2_gamma_0} für $\hat a(\blank, \blank)$ nach und bestimmen die Konstante $\gamma_{0}$.
        Da $\hat a(\blank, \blank)$ symmetrisch ist, genügt es, die inf-sup-Bedingung \eqref{eq:kunoth:ass2_gamma_0_a} nachzuweisen. Die zweite inf-sup-Bedingung \eqref{eq:kunoth:ass2_gamma_0_b} folgt dann analog mit dem selben $\gamma_{0}$.

        Nach \thref{lem:sauter:2.1.48} reicht es, für alle $\eta \in H^{1}_{0}(\Omega)$ ein $\zeta = \zeta(\eta) \in H^{1}_{0}(\Omega)$ und von $\eta$ und $\zeta$ unabhängige Konstanten $C_{1}, C_{2} > 0$ mit
        \begin{equation}
            \hat a(\eta, \zeta) \geq C_{1} \norm{\eta}_{H^{1}(\Omega)}^{2} \quad \text{und} \quad \norm{\zeta}_{H^{1}(\Omega)} \leq C_{2} \norm{\eta}_{H^{1}(\Omega)}
        \end{equation}
        zu finden.
        Dann ist die inf-sup-Bedingung \eqref{eq:kunoth:ass2_gamma_0_a} mit $\gamma_{0} = \frac{C_{1}}{C_{2}}$ erfüllt.

        Sei nun also $\eta \in H^{1}_{0}(\Omega)$ beliebig.
        Wir wählen $\zeta = \eta \in H^{1}_{0}(\Omega)$, das heißt, es gilt $C_{2} = 1$.
        Es ergibt sich
        \begin{align}
            \hat a(\eta, \zeta) = \hat a(\eta, \eta) = c \skprod{\grad \eta}{\grad \eta}_{L_{2}(\Omega)} = c \norm{\grad \eta}_{L_{2}(\Omega)}^{2} \geq c \gamma_{\Omega}^{2} \norm{\eta}_{H^{1}(\Omega)}^{2},
        \end{align}
        wobei die letzte Abschätzung aus der Poincaré-Friedrichs-Ungleichung \eqref{eq:gl:poincare_friedrichs_ungleichung} folgt.
        Zusammen liefert dies $\gamma_{0} = c \gamma_{\Omega}^{2}$ als inf-sup-Konstante.

        Für den vorliegenden Fall können wir $\gamma_{\Omega}^{2}$ exakt bestimmen.
        Nach \cite[Chapter 11]{Strauss:2007vz} entspricht das Quadrat der optimalen Poincaré-Friedrichs-Konstante $\gamma_{\Omega}^{2}$ gerade dem kleinsten Eigenwert des Laplace-Operators auf $\Omega$ mit Dirichlet-Randbedingung.
        Dieser hat für $\Omega = [0, L]$ den Wert $\frac{\pi^{2}}{L^{2}}$.
        Wir erhalten damit also $\gamma_{0} = c \frac{\pi^{2}}{L^{2}}$.

        Seien nun $\eta, \zeta \in H^{1}_{0}(\Omega)$.
        Für $j = 0$ gilt die simple Abschätzung
        \begin{equation}
            \begin{aligned}
                a_{0}(\eta, \zeta)
                &= \skprod{\varphi_{0} \eta}{\zeta}_{L_{2}(\Omega)}
                = K \skprod{\eta}{\zeta}_{L_{2}(\Omega)}
                \\&\leq K \norm{\eta}_{L_{2}(\Omega)} \norm{\zeta}_{L_{2}(\Omega)}
                \leq K \norm{\eta}_{H^{1}(\Omega)} \norm{\zeta}_{H^{1}(\Omega)}
            \end{aligned}
        \end{equation}
        Betrachte für $j \geq 1$
        \begin{align}
            a_{j}(\eta, \zeta)
            &= \skprod{\varphi_{j} \eta}{\zeta}_{L_{2}(\Omega)}
            = \int_{0}^{L} \varphi_{j} \eta \zeta \diff x
            \intertext{da $\varphi_{j}$ integrierbar ist und $\varphi_{j}(0) = 0$, können wir dies umschreiben zu}
            a_{j}(\eta, \zeta)
            &= \int_{0}^{L} \frac{\diff}{\diff x} \left( \int_{0}^{x} \varphi_{j}(y) \diff y \right) \eta \zeta \diff x
            \intertext{woraus wir mittels partieller Integration und $\eta, \zeta \in H^{1}_{0}(\Omega)$ folgenden Ausdruck erhalten}
            a_{j}(\eta, \zeta)
            &= - \int_{0}^{L} \left( \int_{0}^{x} \varphi_{j}(y) \diff y \right) (\eta \zeta)' \diff x
            \leq \norm*{\left( \int_{0}^{x} \varphi_{j}(y) \diff y \right) (\eta \zeta)'}_{L_{1}(\Omega)}
            \\&\leq \norm*{\int_{0}^{x} \varphi_{j}(y) \diff y }_{L_{\infty}(\Omega)} \norm{(\eta \zeta)'}_{L_{1}(\Omega)}.
        \end{align}
        Die erste Norm können wir weiter abschätzen mit
        \begin{equation}
            \begin{aligned}
                \norm*{\int_{0}^{x} \varphi_{j}(y) \diff y }_{L_{\infty}(\Omega)}
                &= \norm*{\int_{0}^{x} \frac{K}{(\pi j)^{1 + \epsilon}} \sin(\tfrac{\pi j}{L} y) \diff y}_{L_{\infty}(\Omega)}
                \\&= \norm*{\frac{K L}{(\pi j)^{2 + \epsilon}} \left( 1 - \cos(\tfrac{\pi j}{L} x) \right) }_{L_{\infty}(\Omega)}
                \leq \frac{2 K L}{(\pi j)^{2 + \epsilon}}
            \end{aligned}
        \end{equation}
        Aus der zweiten Norm erhalten wir mittels Minkowski- und Hölderungleichung sowie der Einbettung $H^{1}_{0}(\Omega) \hookrightarrow L_{\infty}(\Omega)$ die Abschätzung
        \begin{equation}
            \begin{aligned}
                \norm{(\eta \zeta)'}_{L_{1}(\Omega)}
                &= \norm{\eta' \zeta + \eta \zeta'}_{L_{1}(\Omega)}
                \leq \norm{\eta' \zeta}_{L_{1}(\Omega)} + \norm{\eta \zeta'}_{L_{1}(\Omega)}
                \\&\leq \norm{\eta'}_{L_{1}(\Omega)} \norm{\zeta}_{L_{\infty}(\Omega)} + \norm{\eta}_{L_{\infty}(\Omega)} \norm{\zeta'}_{L_{1}(\Omega)}
                \\&\leq \norm{1}_{L_{2}(\Omega)} \norm{\eta'}_{L_{2}(\Omega)} \norm{\zeta}_{L_{\infty}(\Omega)} + \norm{\eta}_{L_{\infty}(\Omega)} \norm{1}_{L_{2}(\Omega)} \norm{\zeta'}_{L_{2}(\Omega)}
                \\&\leq \sqrt{L} C_{\infty} \norm{\eta'}_{L_{2}(\Omega)} \norm{\zeta}_{H^{1}(\Omega)} + \sqrt{L} C_{\infty} \norm{\eta}_{H^{1}(\Omega)} \norm{\zeta'}_{L_{2}(\Omega)}
                % \\&\leq \norm{\eta'}_{L_{2}(\Omega)} \norm{\zeta}_{L_{2}(\Omega)} + \norm{\eta}_{L_{2}(\Omega)} \norm{\zeta'}_{L_{2}(\Omega)}
                \\&\leq 2 \sqrt{L} C_{\infty} \norm{\eta}_{H^{1}(\Omega)} \norm{\zeta}_{H^{1}(\Omega)}
            \end{aligned}
        \end{equation}
        Zusammen also
        \begin{align}
            a_{j}(u, v)
            &\leq \norm*{\int_{0}^{x} \varphi_{j}(y) \diff y }_{L_{\infty}(\Omega)} \norm{(\eta \zeta)'}_{L_{1}(\Omega)}
            \\&\leq \frac{4 K L^{3 / 2} C_{\infty}}{(\pi j)^{2 + \epsilon}} \norm{\eta}_{H^{1}(\Omega)} \norm{\zeta}_{H^{1}(\Omega)}.
        \end{align}
        Betrachte nun
        \begin{align}
                    \sum_{j = 0}^{\infty} \norm{A_{j}}_{\mathcal L(V, V')}
            &= \norm{A_{0}}_{\mathcal L(V, V')} + \sum_{j = 1}^{\infty} \norm{A_{j}}_{\mathcal L(V, V')}
            \\&\leq K + \sum_{j = 1}^{\infty} \frac{4 K L^{3 / 2} C_{\infty}}{(\pi j)^{2 + \epsilon}}
            \\&\leq K + \frac{4 K L^{3 / 2} C_{\infty}}{\pi^{2+ \epsilon}} \sum_{j = 1}^{\infty} \frac{1}{j^{2 + \epsilon}}
        \end{align}
        Fordern wir nun die Gültigkeit von \eqref{eq:kunoth:ass2_abs_reihe}, also
        \begin{equation}
            \sum_{j \geq 0} \norm{A_{j}}_{\mathcal L(V, V')} \leq \kappa \gamma_{0}
        \end{equation}
        für ein $0 < \kappa < 1$, dann ist damit also
        \begin{equation}
            \sum_{j = 1}^{\infty} \frac{1}{j^{2 + \epsilon}} \leq \frac{(\kappa (\tfrac{\pi}{L})^{2} - K) \pi^{2+ \epsilon}}{4 K L^{3/2} C_{\infty}}
        \end{equation}
        mit $\epsilon > 0$ hinreichend.
    \end{Beweis}
\end{Satz}

Zusammenfassend erhalten wir damit die folgende Aussage.

\todo[inline]{Besser ausformulieren}
\begin{Satz}
    Seien $\mathcal X$ und $\mathcal Y$ gegeben wie in~\eqref{eq:ps:rzvp:ansatzraum_testraum}.
    Weiter sei $\Set{\hat A, A_{j} \given j \in \mathbb{N}_{0}}$ die affine Zerlegung von $A = -c \Delta + \omega$ wie in \eqref{eq:1d:affine_zerlegung} und es gelte \thref{satz:regularitaet_nachrechnen}.
    Für jedes $\sigma \in \mathcal S$ sei $B(\sigma) \in \mathcal L(\mathcal X, \mathcal Y')$ definiert durch
    \begin{equation}
        \label{eq:ps:rg:theorem21_variationsproblem_als_operatorgleichung_parametrisch}
        \skprod{B(\sigma) u}{v}_{\mathcal Y' \times \mathcal Y} = b(u, v; \sigma), \quad u \in \mathcal X,~y \in \mathcal Y,
    \end{equation}
    mit $b(\blank, \blank; \sigma)$ wie in~\eqref{eq:ps:rzvp:schwache_formulierung_lhs_b_2}.
    Dann ist $B(\sigma)$ für jedes $\sigma \in \mathcal S$ stetig invertierbar und die parametrische Familie von Lösungen $u(\sigma)$ des parametrischen Raum-Zeit-Variationsproblems \eqref{eq:ps:rzvp:schwache_formulierung_2} hängt analytisch von $\sigma$ ab.

    \begin{Beweis}
        Direkte Folgerung aus \thref{satz:regularitaet_nachrechnen} und \thref{satz:ps:rg:kunoth13_theorem21}.
    \end{Beweis}
\end{Satz}

% subsection nachrechnen_von_thref_thm_kunoth_assumption2 (end)

% section der_eindimensionale_fall (end)

\clearpage
\section{Zu klärende Fragen} % (fold)
\label{sub:zu_kl_rende_fragen}

\begin{enumerate}
    \item Wohldefiniertheit der PDE \eqref{eq:parabolische_pde}, das heißt die Voraussetzungen von \thref{satz:gl:le:ss09_theorem51} nachweisen. Weiterhin lassen sich damit die inf-sup-Bedingung von \eqref{eq:ps:rzvp:schwache_formulierung} nachrechnen und damit die Schranken für $B$ und $B^{-1}$ bestimmen.
    \item Parametrische Variante des Variationsproblems herleiten.
    Dazu Ansetzen mit Entwicklung des Parameters $\omega$ in eine Reihe
    \begin{equation}
        \omega = \sum_{j = 0}^{\infty} \sigma_{j} \varphi_{j}.
    \end{equation}
    Dabei ergeben sich folgende Fragen:
    \begin{enumerate}
        \item Konvergenz der Reihe? Notwendig ist Konvergenz in $L_{\infty}(\Omega)$, da die Norm $\norm{\omega}_{L_{\infty}(\Omega)}$ mehrfach in Abschätzungen verwendet wird.
        \item Weiterhin ist eventuell Konvergenz in einem Unterraum $Z \hookrightarrow L_{\infty}(\Omega)$ wünschenswert.
        Zum Beispiel in $H^{1}(\Omega)$?
        \item Welche Bedingungen ergeben sich an $\sigma_{j}$ und $\varphi_{j}$?
        \item Welches Funktionensystem $\Set{ \varphi_{j} }_{j}$ ist überhaupt sinnvoll?
        Die Wahl der $\varphi_{j}$ entscheidet maßgeblich über Konvergenz der Reihenentwicklung.
        Welche Randvorgaben sind angestrebt?
        Dies wird ebenfalls durch die $\varphi_{j}$ geregelt.
    \end{enumerate}
    \item Welche affine Zerlegung $A(\sigma) = A_{0} + \sum_{j} \sigma_{j} A_{j}$ ist brauchbar?
    Wie genau sehen die $A_{j}$ aus?
    \item Nachweisen, dass $A(\sigma)$ \thref{ann:ps:rg:kunoth13_assumption1} oder \thref{thm:kunoth:assumption2} erfüllt und mittels \thref{satz:ps:rg:kunoth13_theorem21} die gewünschte Regularität von $B(\sigma)$ bezüglich $\sigma$ gewinnen.
    \item Die Abschätzungen in \thref{satz:regularitaet_nachrechnen} lassen sich noch deutlich verbessern.
    Das gilt wahrscheinlich auch für andere Abschätzungen!
\end{enumerate}


    % Alles, was noch unbedingt rein muss
    % % -*- root: ../main.tex -*-

\iftoggle{dictum}{
    \setchapterpreamble[ul][0.6\textwidth]{%
        \dictum[Terry Pratchett]{\enquote{Coffee is a way of stealing time that should by rights belong to your older self.}}
        \vspace*{2\baselineskip}
    }
}{}
\chapter{Funktionalanalytische Grundlagen} % (fold)
\label{cha:funktionalanalytische_grundlagen}

\todo[inline]{Ständig: ordnen, sortieren, aufräumen, erweitern.}

\section{Orthogonale Funktionen und Polynome}
\label{sec:orthogonale_funktionen_und_polynome}

\begin{Satz}[Orthogonalität trigonometrischer Funktionen]
\label{satz:trigonometrische_funktionen_orthogonal}
    Seien $k, l \in \mathbb{N}$.
    Dann gilt
    \begin{align}
        \skprod{\sin(\pi k x)}{\sin(\pi l x)}_{L_{2}([0, 1])} &= \frac{1}{2} \delta_{kl},
        % \quad\text{und}\quad
        \\\skprod{\cos(\pi k x)}{\cos(\pi l x)}_{L_{2}([0, 1])} &= \frac{1}{2} \delta_{kl},
        \\\skprod{\sin(\pi k x)}{\cos(\pi l x)}_{L_{2}([0, 1])} &= 0.
    \end{align}
\end{Satz}

\begin{Definition}[Legendre-Polynome]
\label{def:legendre_polynome}
    Sei $I = [-1, 1]$.
    Die Legendre-Polynome $L_{n} \in \Pi_{n}$ sind definiert durch
    \begin{equation}
        L_{n}(x) = \frac{1}{2^{n}n!}\frac{\diff^{n}}{\diff x^{n}} (x^{2} - 1)^{n}.
    \end{equation}
    Durch die Transformation $x \mapsto 2x - 1$ erhält man die auf das Interval $[0, 1]$ geshifteten Legendre-Polynome $\tilde L_{n}$.
\end{Definition}

\begin{Satz}[Orthogonalität der Legendre-Polynome]
\label{satz:legendre_polynome_orthogonal}
    Die Legendre-Polynome $L_{n}$ sind orthogonal bezüglich der $L_{2}([-1, 1])$-Norm, denn es gilt
    \begin{equation}
        \skprod{L_{n}}{L_{m}}_{L_{2}([-1, 1])} = \frac{2}{2n + 1} \delta_{n m}.
    \end{equation}
    Auch für die geshifteten Legendre-Polynome $\tilde L_{n}$ gilt Orthogonalität, denn es ist
    \begin{equation}
        \skprod{\tilde L_{n}}{\tilde L_{m}}_{L_{2}([0, 1])} = \frac{1}{2n + 1} \delta_{n m}.
    \end{equation}
\end{Satz}

\begin{Bemerkung}
\label{satz:legendre_polynome_rekursion}
    Die Legendre-Polynome $L_{n}$ erfüllen die Rekursionsformel
    \begin{equation}
        n L_{n}(x) = (2n - 1) x L_{n-1}(x) - (n - 1) L_{n-2}(x), \quad L_{0}(x) = 1, L_{1}(x) = x.
    \end{equation}
    Analog gilt für die erste Ableitung $L_{n}'$ die Rekursionsformel
    \begin{equation}
        (n - 1) L_{n}'(x) = (2n -1) x L_{n-1}'(x) - n L_{n-2}'(x), \quad L_{0}'(x) = 0, L_{1}'(x) = 1.
    \end{equation}
\end{Bemerkung}

\section{Sonstiges} % (fold)
\label{sec:sonstiges}

% \begin{Lemma}
%     $\mathcal C^{0}([a, b]; X)$ liegt dicht in $L_{p}(a, b; X)$ für $1 \leq p < \infty$.
% \end{Lemma}

% TODO: zitieren
\begin{Satz}[Poincaré-Friedrichs-Ungleichung, vgl. {{\cite[Lemma 89.4]{HankeBourgeois:2009fk}}}]
\label{satz:poincare_ungleichung}
    Sei $\Omega \subset \mathbb{R}^{n}$ offen, beschränkt und mit Lipschitz-Rand.
    Dann existiert eine Konstante $\gamma_{\Omega} > 0$ mit
    \begin{equation}
        \label{eq:gl:poincare_friedrichs_ungleichung}
        \norm{\grad u}_{L_{2}(\Omega)} \geq \gamma_{\Omega} \norm{u}_{H^{1}(\Omega)} \quad \fa u \in H^{1}_{0}(\Omega).
    \end{equation}
\end{Satz}

\begin{Satz}[Poincaré-Friedrichs-Ungleichung, vgl. {{\cite[Theorem II.1.7]{Braess:2007wm}}}]
    Es sei $\Omega \subset \mathbb{R}^{n}$ beschränkt und in einem $n$-dimensionalen Würfel mit Seitenlänge $s$ enthalten.
    Dann gilt
    \begin{equation}
        (1 + s)^{m} \abs{u}_{H^{m}} \geq \norm{u}_{H^{m}} \geq \abs{u}_{H^{m}} \quad \text{für alle}~u \in H^{m}_{0}(\Omega).
    \end{equation}
\end{Satz}

\begin{Lemma}[{{\cite[Remark 2.1.48]{Sauter:9_WoPZ0Y}}}]
\label{lem:sauter:2.1.48}
    Seien $X$ und $Y$ zwei reflexive Banachräume und $a \colon X \times Y \to \mathbb{R}$ eine Bilinearform.
    Finden wir für jedes $x \in X$ ein $y_{x} \in Y$, so dass
    \begin{equation}
        \label{eq:lem:sauter:2.1.48:eq1}
        \abs{a(x, y_{x})} \geq C_{1} \norm{x}_{X}^{2} \quad \text{und} \quad \norm{y_{x}}_{Y} \leq C_{2} \norm{x}_{X}
    \end{equation}
    mit von $x$ und $y_{x}$ unabhängigen Konstanten $C_{1}, C_{2} > 0$ gilt, dann folgt daraus die inf-sup-Bedingung
    \begin{equation}
    \label{eq:lem:sauter:2.1.48:inf_sup}
        \inf_{0 \neq x \in X} \sup_{0 \neq y \in Y} \frac{a(x, y)}{\norm{x}_{X}\norm{y}_{Y}} \geq \gamma > 0
    \end{equation}
    mit $\gamma = \frac{C_{1}}{C_{2}}$.

    \begin{Beweis}
        Seien $x \in X$ und $y_{x} \in Y$ so, dass \cref{eq:lem:sauter:2.1.48:eq1} erfüllt ist.
        Dann gilt
        \begin{align}
            \inf_{0 \neq x \in X} \sup_{0 \neq y \in Y} \frac{\abs{a(x, y)}}{\norm{x}_{X} \norm{y}_{Y}}
            &\geq
            \inf_{0 \neq x \in X} \frac{\abs{a(x, y_{x})}}{\norm{x}_{X} \norm{y_{x}}_{Y}}
            \\&\geq
            \inf_{0 \neq x \in X} \frac{C_{1} \norm{x}^{2}_{X}}{\norm{x}_{X} C_{2} \norm{x}_{X}}
            =
            \frac{C_{1}}{C_{2}}
            > 0.
        \end{align}
    \end{Beweis}
\end{Lemma}

% section sonstiges (end)

    % %!TEX root = ../main.tex

\chapter{Notizen} % (fold)
\label{cha:notizen}

% section zum_petrov_galerkin_verfahren (end)

\section{Reduzierte-Basis-Methode} % (fold)
\label{sec:reduzierte_basis_methode}

Bei der Reduzierte-Basis-Methode für die Raum-Zeit-Variationsformulierung parabolischer partieller Differentialgleichungen haben sich folgende Punkte ergeben, welche eine Anmerkung verdienen.

Seien dazu $\mathcal X^{\mathcal N} = \spn\Set{\phi_{n}}_{n=1}^{\mathcal N}$ und $\mathcal Y^{\mathcal M} = \spn\Set{\psi_{m}}_{m = 1}^{\mathcal M}$ endlichdimensionale Hilberträume, beispielsweise aus dem Petrov-Galerkin-Verfahren.
Im Allgeimeinen muss hier nicht $\mathcal N = \mathcal M$ gelten.
Weiter bezeichnen wir mit $\mathcal X_{N} \subset \mathcal X^{\mathcal N}$ und $\mathcal Y_{\mathcal M} \subset \mathcal Y_{M}$ die Reduzierte-Basis-Räume.

Wir betrachten das abstrakte Variationsproblem
\begin{equation}
    b(u, v; \mu) = f(v; \mu) \qquad \text{mit}~u \in \mathcal X,~v \in \mathcal Y,
\end{equation}
wobei $b \colon \mathcal X \times \mathcal Y \times \mathcal P \to \mathbb{R}$ eine affin parametrische Bilinearform und $f \colon \mathcal Y \times \mathcal P \to \mathbb{R}$ ein affin parametrisches lineares stetiges Funktional sei,
das heißt, es gelte
\begin{equation}
    b(u, v; \mu) = \sum_{q = 1}^{Q_b} \theta^{b}_{q}(\mu) b_{q}(u, v)
    \qquad \text{und} \qquad
    f(v; \mu) = \sum_{q = 1}^{Q_f} \theta^{f}_{q}(\mu) f_{q}(v).
\end{equation}

\paragraph{A posteriori Fehlerschätzer} % (fold)
\label{par:a_posteriori_fehlersch_tzer}

Der A-posteriori-Fehlerschätzer ergibt sich wie bei Reduzierte-Basis-Methoden üblich folgendermaßen:

Sei $\mu \in \mathcal P$ ein Parameter, $u^{\mathcal N}(\mu) \in \mathcal X^{\mathcal N}$ die Truth-Lösung des Variationsproblem für $\mu$ und $u_{N}(\mu) \in \mathcal X_{N}$ die entsprechende Reduzierte-Basis-Lösung.
Definiere den Fehler
\begin{equation}
    e_{N}(\mu) := u^{\mathcal N}(\mu) - u_{N}(\mu) \in \mathcal X^{\mathcal N}.
\end{equation}
Weiter wird das Residuum für alle $v \in \mathcal Y^{\mathcal M}$ definiert als
\begin{equation}
    r_{N}(v; \mu) = b(e_{N}(\mu), v; \mu) = f(v; \mu) - b(u_{N}(\mu), v; \mu).
\end{equation}
Fasst man das Residuum als rechte Seite des obigen Variationsproblems auf, dann ist $e_{N}(\mu)$ die zugehörige Lösung.
Weiter können wir die übliche Abschätzung (Lemma von Cea bzw. ähnliche Aussage) verwenden und erhalten die Ungleichung
\begin{equation}
    \norm{e_{N}(\mu)}_{\mathcal X} \leq \frac{1}{\beta^{\mathcal N}(\mu)} \norm{r_{N}(\blank; \mu)}_{\mathcal Y^{\mathcal M}'}.
\end{equation}
Dabei ist $\beta^{\mathcal N}(\mu)$ die inf-sup-Konstante des Truth-Probelms für den Parameter $\mu$.

\paragraph{Berechnung der inf-sup-Konstante} % (fold)
\label{par:berechnung_der_inf_sup_konstante}

Die obige inf-sup-Konstante $\beta^{\mathcal N}(\mu)$ wird in der Offline-Phase der Reduzierte-Basis-Methode für jeden Parameter $\mu$ des verwendeten Trainingsraums benötigt, muss also effizient auswertbar sein.

Zunächst eine Erklärung, wie man $\beta^{\mathcal N}(\mu)$ grundsätzlich berechnen kann.
Dazu greift man auf den sogenannten \emph{Supremizing Operator} zurück.
Dieser ist eine Abbildung $T_{\mu} \colon \mathcal X^{\mathcal N} \to \mathcal Y^{\mathcal M}$ definiert durch
\begin{equation}
    \skp{T_{\mu} u}{v}{\mathcal Y^{\mathcal M}} = b(u, v; \mu) \quad \fa v \in \mathcal Y^{\mathcal M}.
\end{equation}
Weiter gilt
\begin{equation}
    T_{\mu}u = \arg \sup_{v \in \mathcal{Y}^{\mathcal M}}  \frac{b(u, v; \mu)}{\norm{v}_{\mathcal Y^{\mathcal M}}}
\end{equation}
und damit
\begin{equation}
    \beta^{\mathcal N}(\mu) = \inf_{u \in \mathcal X^{\mathcal N}} \frac{\norm{T_{\mu}u}_{\mathcal Y^{\mathcal M}}}{\norm{u}_{\mathcal X^{\mathcal N}}}.
\end{equation}
Mittels des Rieszschen Darstellungssatzes lässt sich der Operator $T_{\mu}$ berechnen.

Sei dazu $\mat{Y} = [\skp{\psi_{m}}{\psi_{m'}}{\mathcal Y^{\mathcal M}}]_{m, m'}$, $\mat{X} = [\skp{\phi_{n}}{\phi_{n'}}{\mathcal X^{\mathcal N}}]_{n, n'}$ und $\mat{B}_{\mu} = [b(\phi_{n}, \psi_{m}; \mu)]_{m, n}$.
Dann gilt
\begin{equation}
    \mat{Y} \vec{T}_{\mu} = \mat{B}_{\mu}.
\end{equation}
Eingesetzt ergibt sich dann das Quadrat der inf-sup-Konstante dann als
\begin{equation}
    (\beta^{\mathcal N}(\mu))^{2} = \inf_{\vec{u} \in \mathbb{R}^{\mathcal N}} \frac{\vec{u}\Transp \mat{B}_{\mu}\tranps \mat{Y}^{-1} \mat{B}_{\mu} \vec{u}}{\vec{u}\Transp \mat{X} \vec{u}}
\end{equation}
und lässt sich als kleinster Eigenwert $\lambda$ des verallgemeinerten Eigenwertproblems
\begin{equation}
    \mat{B}_{\mu}\tranps \mat{Y}^{-1} \mat{B}_{\mu} \vec{x} = \lambda \mat{X} \vec{x}
\end{equation}
bestimmen.

Für die Successive Constraint Method, siehe \textcite{Huynh2007}.


% paragraph berechnung_der_inf_sup_konstante (end)

\paragraph{Stabiler Reduzierte-Basis-Testraum} % (fold)
\label{par:stabiler_reduzierte_basis_testraum}

Bei der Reduzierte-Basis-Methode für parabolische Probleme ergibt sich das Problem, dass die Lösungen nur den Reduzierte-Basis-Ansatzraum aufspannen. Der Testraum muss dagegen anderweitig konstruiert werden.
Hier scheint es verschiedene, hauptsächlich heuristisch motivierte Ansätze zu geben, die wiederum den obigen Supremizing Operator verwenden.
Beispiele sind \textcite[Abschnitt 4.2]{Mayerhofer:2014vx} beziehungsweise \textcite{Dahmen:2014cl}.

Momentan implementiert ist \textcite[Abschnitt 4.2]{Mayerhofer:2014vx}.
% paragraph stabiler_reduzierte_basis_testraum (end)


\section{Petrov-Galerkin} % (fold)
\label{sec:petrov_galerkin}

Die Zeitdiskretisierung wurde ausgetauscht. Statt Legendre-Polynomen werden nun, da es weit verbreitet zu sein scheint und laut \textcite{Andreev:2012uh,Andreev:2012ep,Andreev:2013gk} mit guten Stabilitätsergebnissen, nodale Hutfunktionen für den Ansatzraum und Indikatorfunktionen für den Testraum verwendet. Dabei wird im Allgeimeinen die Zeitdiskretisierung des Testraumes um den Faktor 2 verfeinert, wodurch sich die inf-sup-Stabilität ohne Beachtung einer CFL-Bedingung ergibt. (Die räumliche Diskretisierung muss ein, zwei Bedingungen erfüllen, mal checken).

Durch den größeren Testraum ergibt sich ein überbestimmtes System, welches im Sinne einer Residuum-Minimierung gelöst werden muss. Es lässt sich zeigen, dass dieses \emph{Minimales Residuum Petrov-Galerkin-Verfahren} ähnliche Aussagen wie das übliche Petrov-Galerkin-Verfahren erfüllt.

% section petrov_galerkin (end)

\section{Weiteres} % (fold)
\label{sec:weiteres}

\begin{itemize}
    \item Im Moment ist die periodische Randbedingung am laufen. Wird durch Fourier-Disrektisierung mit Konstanter Basisfunktion geregelt. Die Theorie wird aber für $H^{1}_{0, per}(\Omega) := H^{1}_{per}(\Omega) / \mathbb{R}$ angeregt. Schlimm? Wie anders lösbar?
    \item Wie könnte man die Anzahl der verwendeten Feld-Entwicklungsfunktionen einschränken?
    \item
\end{itemize}

% section weiteres (end)

    % %!TEX root = ../main.tex

\section{Eindimensionaler Fall mit $\omega \in \mathbb{R}$ und ohne Quellterm}

Sei $I := [0, \hat t]$ für ein $0 < \hat t < \infty$ und $\Omega := [0, 1]$.
Betrachte folgende parametrisierte PDE
\begin{align}
    \begin{cases}
    u_{t}(t, x) = \sigma u_{xx}(t, x) - \omega u(t, x), & (t, x) \in I \times \Omega\\
    u(0, x) = g(x), & x \in \Omega \\
    u(t, 0) = u(t, 1) = 0, & t \in I
    \end{cases}
\end{align}
mit Konstanten $\sigma, \omega \in \mathbb{R}$.

Ein Separation der Variablen Ansatz $u(t, x) = X(x) T(t)$ liefert
\begin{equation}
    X(x)T'(t) = \sigma X''(x) T(t) - \omega X(x) T(t)
\end{equation}
oder auch
\begin{equation}
    \frac{T'(t)}{T(t)} = \sigma \frac{X''(x)}{X(x)} - \omega = \lambda
\end{equation}
mit $\lambda \in \mathbb{R}$.

Ohne Einschränkung sei $\lambda \neq 0$, dann erhalten wir zum einen die Dgl.
    $T'(t) = \lambda T(t)$,
deren Lösung
\begin{equation}
    T(t) = d_{3} e^{\lambda t}
\end{equation}
ist, und zum anderen die Dgl.
    $X''(x) =  \frac{\lambda + \omega}{\sigma} X(x)$
mit der Lösung
\begin{equation}
    X(x) = d_{1} e^{\sqrt{\frac{\lambda + \omega}{\sigma}} x} + d_{2} e^{-\sqrt{\frac{\lambda + \omega}{\sigma}}x},
\end{equation}
wobei $d_{1}, d_{2}, d_{3} \in \mathbb{R}$.

Als nächstes Verwenden wir die Anfangs- und Randbedingungen um die Konstanten $d_{i}$ zu bestimmen.
Sei
\begin{equation}
    u(t, x) = \left( d_{1} e^{\sqrt{\frac{\lambda + \omega}{\sigma}} x} + d_{2} e^{-\sqrt{\frac{\lambda + \omega}{\sigma}}x} \right) \left( d_{3} e^{\lambda t} \right),
\end{equation}
Betrachten wir zunächst die Randbedingung $u(t, 0) = u(t, 1) = 0$, dann erhalten wir aus
\begin{equation}
    0 = u(t, 0) = \left( d_{1} + d_{2} \right) \left( d_{3} e^{\lambda t} \right),
\end{equation}
oder äquivalent $d_{1} = - d_{2}$, und aus
\begin{equation}
    0 = u(t, 1) = \left( d_{1} e^{\sqrt{\frac{\lambda + \omega}{\sigma}}} + d_{2} e^{-\sqrt{\frac{\lambda + \omega}{\sigma}}} \right) \left( d_{3} e^{\lambda t} \right) =
    d_{1} \left( e^{\sqrt{\frac{\lambda + \omega}{\sigma}}} - e^{-\sqrt{\frac{\lambda + \omega}{\sigma}}} \right) \left( d_{3} e^{\lambda t} \right),
\end{equation}
ohne Einschränkung $d_{1} \neq 0$, die Gleichung
\begin{equation}
    0 = e^{\sqrt{\frac{\lambda + \omega}{\sigma}}} - e^{-\sqrt{\frac{\lambda + \omega}{\sigma}}}.
\end{equation}
Aus dieser erhalten wir durch Äquivalenzumformungen
\begin{align}
    e^{\sqrt{\frac{\lambda + \omega}{\sigma}}} - e^{-\sqrt{\frac{\lambda + \omega}{\sigma}}} = 0
    &\quad \iff \quad
    e^{2\sqrt{\frac{\lambda + \omega}{\sigma}}} = 1
    \quad \iff \quad
    \sqrt{\tfrac{\lambda + \omega}{\sigma}} = k \pi i
    \\&\quad \iff \quad
    \tfrac{\lambda + \omega}{\sigma} = -k^2 \pi^2
    \quad \iff \quad
    \lambda = -k^2 \pi^2 \sigma - \omega,
\end{align}
mit $k \in \mathbb{Z}$ beliebig.
Einsetzen liefert nun
\begin{align}
    u_{k}(t, x) &= d_{1} \left( e^{k \pi i x} - e^{-k \pi i x} \right) \left( d_{3} e^{- (k^2 \pi^2 \sigma + \omega) t} \right)
    \\&= 2 d_{1} d_{3} i \sin(k \pi x) e^{-(k^2 \pi^2 \sigma + \omega)t},
\end{align}
wobei wir $\beta_{k} := 2 d_{1} d_{3} i$ setzen.

Da jedes $u_{k}$, $k \in \mathbb{Z}$, eine Lösung ist, erhalten wir durch
\begin{equation}
    u(t, x) = \sum_{k = 1}^{\infty} u_{k}(t, x) = \sum_{k = 1}^{\infty} \beta_{k} \sin(k \pi x) e^{-(k^2 \pi^2 \sigma + \omega)t}
\end{equation}
ebenfalls eine Lösung.
Damit die Anfangsbedingung erfüllt wird, muss
\begin{equation}
    g(x) = u(x, 0) = \sum_{k = 1}^{\infty} \beta_{k} \sin(k \pi x)
\end{equation}
gelten, was genau dann der Fall ist, wenn
\begin{equation}
    \beta_{k} = 2 \int_{0}^{1} g(x) \sin(k \pi x) \diff x.
\end{equation}

Da $u_{k}$ analytisch in $\omega$ für alle $k \in \mathbb{Z}$, ist auch $u$ analytisch in $\omega$ und es gilt
\begin{equation}
    \frac{\partial^{j} u(t, x; \omega)}{\partial \omega^{j}} = \sum_{k = 1}^{\infty} (-t)^{j} \beta_{k} \sin(k \pi x) e^{-(k^{2} \pi^{2} \sigma + \omega)t}.
\end{equation}


    %%% Anhang
    \appendix{}

    % DVD Inhalt
    % \subfile{chapters/app_code}

    %% Los geht's mit den Verzeichnissen

    % Abbildungsverzeichnis
    % \listoffigures

    % Tabellenverzeichnis
    % \listoftables

    % Symbolverzeichnis
    % \glsaddall[]
    % \printglossary[type=symbolslist, nonumberlist, style=long]

    % Literaturverzeichnis
    \printbibliography

    % Eidesstattliche Erklärung
    % %!TEX root = ../main.tex

\chapter{Eidesstattliche Erklärung} % (fold)
\label{cha:eidesstattliche_erklaerung}

Ich versichere hiermit, dass ich die vorliegende Masterarbeit selbständig
verfasst und keine anderen als die angegebenen Quellen und Hilfsmittel benutzt
habe, wobei ich alle wörtlichen und sinngemäßen Zitate als solche gekennzeichnet
habe. Die Arbeit wurde bisher keiner anderen Prüfungsbehörde vorgelegt und auch
nicht veröffentlicht.\\[6ex]

\ort, den \today

\hdashrule[-0.5cm]{5cm}{0.5pt}{1pt}

\textsc{\autor}

% chapter eidesstattliche_erklaerung (end)

\end{document}
