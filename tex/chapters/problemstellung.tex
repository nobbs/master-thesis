%!TEX root = ../main.tex

\chapter{Überschrift in Arbeit}

In diesem Kapitel führen wir die in \cite{Stasiak:2011ba} motivierte parabolische partielle Differentialgleichung ein und betrachten danach eine parametrisierte Variante dieser.
Anschließend nutzen wir die Aussagen aus \autoref{cha:einfuehrung} um die Eigenschaften dieses Problems zu diskutieren.

\section{Problemstellung} % (fold)
\label{sub:problemstellung}

Es sei $0 < T < \infty$ und $I = [0, T]$.
Weiter sei $\Omega \subset \mathbb{R}^{n}$ eine beschränkte Teilmenge des $\mathbb{R}^{n}$ mit Lipschitz-Rand.
Wir wählen als konkrete Hilberträume $V = H^{1}_{0}(\Omega)$ und $H = L_{2}(\Omega)$.
Bekanntlich sind diese separabel und es existiert eine dichte stetige Einbettung von $H^{1}_{0}(\Omega)$ in $L_{2}(\Omega)$.
Dies liefert uns wegen $(H^{1}_{0}(\Omega))' = H^{-1}(\Omega)$ das Gelfand-Tripel
\begin{equation}
    H^{1}_{0}(\Omega) \denseinclusion L_{2}(\Omega) \denseinclusion H^{-1}(\Omega).
\end{equation}
Wie zuvor verwenden wir $\skprod{\blank}{\blank}$ mit entsprechendem Index sowohl für die Skalarprodukte als auch für die duale Paarung auf $H^{-1}(\Omega) \times H^{1}_{0}(\Omega)$.

Seien $c \in \mathbb{R}_{+}$ und $\omega \in L_{\infty}(\Omega)$ gegeben, dann definieren wir den linearen Operator $A$ als
\begin{equation}
    \label{eq:def_op_A}
    A \colon H^{1}_{0}(\Omega) \to H^{-1}(\Omega), \quad \eta \mapsto A \eta = - c \Delta \eta + \omega \eta.
\end{equation}

Wir interessieren uns nun für Lösungen der parabolischen partiellen Differentialgleichung \eqref{eq:allgemeine_parabolische_pde} mit $A$ aus \eqref{eq:def_op_A}.
\begin{Problem}
Seien $g \in L_{2}(I; H^{-1}(\Omega))$ und $u_{0} \in L_{2}(\Omega)$ gegeben.
Gesucht ist eine Lösung $u$ von
\begin{equation}
    \label{eq:parabolische_pde}
    u_{t}(t) - c \Delta u(t) + \omega u(t) = g(t) \quad \text{in}~H^{-1}(\Omega),
    \qquad
    u(0) = u_{0} \quad \text{in}~L_{2}(\Omega).
\end{equation}
\end{Problem}

Bevor wir analog zu \autoref{sec:raum_zeit_variationsformulierung} eine Raum-Zeit-Variationsformulierung für \eqref{eq:parabolische_pde} angeben, weisen wir zunächst einige Eigenschaften von $A$ nach.

Bezeichne mit $a$ die zu $A$ zugehörige Bilinearform, es gilt also
\begin{equation}
    a \colon H^{1}_{0}(\Omega) \times H^{1}_{0}(\Omega) \to \mathbb{R}, \quad a(\eta, \zeta) = \skprod{A \eta}{\zeta}_{L_{2}(\Omega)}.
\end{equation}
Unter Verwendung der Greenschen Formeln erhalten wir
\begin{equation}
    \begin{aligned}
        a(\eta, \zeta)
        &= \skprod{A \eta}{\zeta}_{L_{2}(\Omega)}
        \\&= \skprod{- c \Delta \eta + \omega \eta}{\zeta}_{L_{2}(\Omega)}
        \\&= - c \skprod{\Delta \eta}{\zeta}_{L_{2}(\Omega)} + \skprod{\omega \eta}{\zeta}_{L_{2}(\Omega)}
        \\&= c \skprod{\grad \eta}{\grad \zeta}_{L_{2}(\Omega)} + \skprod{\omega \eta}{\zeta}_{L_{2}(\Omega)}.
    \end{aligned}
\end{equation}


\begin{Lemma}
\label{lemma:a_bf_bounded_garding}
    Seien $c \in \mathbb{R}_{+}$ und $\omega \in L_{\infty}(\Omega)$ und
    \begin{equation}
    \label{eq:bf_a}
        a \colon H^{1}_{0}(\Omega) \times H^{1}_{0}(\Omega) \to \mathbb{R}, \quad a(\eta, \zeta) = c \skprod{\grad \eta}{\grad \zeta}_{L_{2}(\Omega)} + \skprod{\omega \eta}{\zeta}_{L_{2}(\Omega)}.
    \end{equation}
    Dann erfüllt $a(\blank, \blank)$ die Eigenschaften aus \ref{annahme:eigenschaften_bf_a}:
    \begin{thmenumerate}
        \item\label{lemma:a_bf_bounded_garding:1}
        \emph{Stetigkeit:} es gilt
        \begin{equation}
            \abs{a(\eta, \zeta)} \leq M_{a} \norm{\eta}_{H^{1}(\Omega)} \norm{\zeta}_{H^{1}(\Omega)} \quad \text{für alle}~\eta, \zeta \in H^{1}_{0}(\Omega)
        \end{equation}
        mit $M_{a} = \max\Set{c, \norm{\omega}_{L_{\infty}(\Omega)} } \geq 0$.
        \item\label{lemma:a_bf_bounded_garding:2}
        \emph{G\aa{}rding-Ungleichung:} es gilt
        \begin{equation}
                a(\eta, \eta) + \lambda \norm{\eta}_{L_{2}(\Omega)}^{2} \geq \alpha \norm{\eta}_{H^{1}(\Omega)}^{2} \quad \text{für alle}~\eta \in H^{1}_{0}(\Omega)
        \end{equation}
        mit $\alpha = c \gamma_{\Omega}^{2} > 0$ und $\lambda = \norm{\omega}_{L_{\infty}(\Omega)} \geq 0$, wobei $\gamma_{\Omega}$ die Poincaré-Friedrichs-Konstante ist.
    \end{thmenumerate}

    \begin{Beweis}
    Wir zeigen zunächst die Stetigkeit.
    Seien dazu $\eta, \zeta \in H^{1}_{0}(\Omega)$ beliebig.
    Unter Verwendung der Dreiecks- und der Cauchy-Schwarz-Ungleichung erhalten wir
    \begin{align}
        \abs{a(\eta, \zeta)}
        &= \abs{c \skprod{\grad \eta}{\grad \zeta}_{L_{2}(\Omega)} + \skprod{\omega \eta}{\zeta}_{L_{2}(\Omega)}}
        \\&\leq c \abs{\skprod{\grad \eta}{\grad \zeta}_{L_{2}(\Omega)}} + \abs{\skprod{\omega \eta}{\zeta}_{L_{2}(\Omega)}}
        \\&\leq c \norm{\grad \eta}_{L_{2}(\Omega)} \norm{\grad \zeta}_{L_{2}(\Omega)} + \norm{\omega}_{L_{\infty}(\Omega)} \norm{\eta}_{L_{2}(\Omega)} \norm{\zeta}_{L_{2}(\Omega)}
        \\&\leq \max \Set{ c, \norm{\omega}_{L_{\infty}(\Omega)} } \norm{\eta}_{H^{1}(\Omega)} \norm{\zeta}_{H^{1}(\Omega)}.
    \end{align}

    Für die G\aa{}rding-Ungleichung seien nun $\eta \in H^{1}_{0}(\Omega)$ und $\lambda \in \mathbb{R}$.
    Wir betrachten
    \begin{align}
        a(\eta, \eta) + \lambda \norm{\eta}^{2}_{L_{2}(\Omega)}
        &= c \norm{\grad \eta}^{2}_{L_{2}(\Omega)} + \skprod{\omega \eta}{\eta}_{L_{2}(\Omega)} + \lambda \skprod{\eta}{\eta}_{L_{2}(\Omega)}
        \\&= c \norm{\grad \eta}^{2}_{L_{2}(\Omega)} + \skprod{(\omega + \lambda) \eta}{\eta}_{L_{2}(\Omega)}.
    \end{align}
    Wählen wir nun $\lambda = \norm{\omega}_{L_{\infty}(\Omega)} \geq 0$, dann gilt $\omega + \lambda \geq 0$ fast überall in $\Omega$ und wir erhalten die Abschätzung
    \begin{align}
        a(\eta, \eta) + \lambda \norm{\eta}^{2}_{L_{2}(\Omega)}
        &\geq c \norm{\grad \eta}^{2}_{L_{2}(\Omega)},
        \intertext{woraus wir durch Anwenden der Poincaré-Friedrichs-Ungleichung \ref{satz:grundlagen:poincare_friedrichs_ungleichung}}
        a(\eta, \eta) + \lambda \norm{\eta}^{2}_{L_{2}(\Omega)}
        &\geq c \gamma_{\Omega}^{2} \norm{\eta}^{2}_{H^{1}(\Omega)}
    \end{align}
    folgern.
    \end{Beweis}
\end{Lemma}

Für die Raum-Zeit-Variationsformulierung verwenden wir die aus \eqref{eq:var_all_ansatzraum_x} und \eqref{eq:var_all_testraum_y} bekannten Ansatz- und Testfunktionenräume.
Mit unserer konkreten Wahl $V = H^{1}_{0}(\Omega)$ und $H = L_{2}(\Omega)$ lauten diese also
\begin{equation}
    \label{eq:var_ansatzraum_testraum}
    \mathcal X = L_{2}(I; H^{1}_{0}(\Omega)) \cap H^{1}(I; H^{-1}(\Omega))
    \quad \text{und} \quad
    \mathcal Y = L_{2}(I; H^{1}_{0}(\Omega)) \times L_{2}(\Omega).
\end{equation}

Damit ergibt sich analog zu \autoref{sec:raum_zeit_variationsformulierung} das folgende Variationsproblem:

\begin{Problem}
    Gegeben ein $g \in L_{2}(I; H^{-1}(\Omega))$ und ein $u_{0} \in L_{2}(\Omega)$. Finde ein $u \in \mathcal X$ mit
    \begin{equation}
        \label{eq:varprob}
        b(u, v) = f(v) \quad \text{für alle}~v \in \mathcal Y,
    \end{equation}
    wobei $b(\blank, \blank) \colon \mathcal X \times \mathcal Y \to \mathbb{R}$ die durch
    \begin{equation}
        \label{eq:buv}
        b(u, v)
            = \int_{I} \skprod{u_{t}(t)}{v_{1}(t)}_{L_{2}(\Omega)} + a(u(t), v_{1}(t)) \diff t + \skprod{u(0)}{v_{2}}_{L_{2}(\Omega)}
    \end{equation}
    gegebene Bilinearform und $f(\blank) \colon \mathcal Y \to \mathbb{R}$ definiert ist durch
    \begin{equation}
        \label{eq:var_all_f_wiederholung}
        f(v) = \int_{I} \skprod{g(t)}{v_{1}(t)}_{L_{2}(\Omega)} \diff t + \skprod{u_{0}}{v_{2}}_{L_{2}(\Omega)}.
    \end{equation}
\end{Problem}

Aus \thref{thm:schwab09:theorem51} erhalten wir nun die Wohldefiniertheit des obigen Variationsproblems und zugleich Schranken für die Operatoren.

\begin{Korollar}
\label{korollar:2.2}
    Seien $\mathcal X$ und $\mathcal Y$ gegeben wie in \eqref{eq:var_ansatzraum_testraum} und sei $B \colon \mathcal X \to \mathcal Y'$ definiert durch
    \begin{equation}
        \skprod{Bu}{v}_{\mathcal Y' \times \mathcal Y}  = b(u, v), \quad u \in \mathcal X,~ v \in \mathcal Y,
    \end{equation}
    mit $b(\blank, \blank)$ wie in \eqref{eq:buv}.
    Dann ist $B$ stetig invertierbar und es gilt
    \begin{equation}
        \norm{B}_{\mathcal L(\mathcal X, \mathcal Y')}
        \leq
        \frac{\sqrt{2 \max\Set{1, c^{2}, \norm{\omega}_{L_{\infty}(\Omega)}^{2}} + M_{e}^{2}}}{\max\Set{\sqrt{1 + 2 \norm{\omega}_{L_{\infty}(\Omega)}^{2} \rho^{4}}, \sqrt{2} }}
    \end{equation}
    und
    \begin{equation}
        \norm{B^{-1}}_{\mathcal L( \mathcal Y', \mathcal X)}
        \leq \frac{e^{2 T \norm{\omega}_{L_{\infty}(\Omega)}} \max\Set{\sqrt{1 + 2 \norm{\omega}_{L_{\infty}(\Omega)}^{2} \rho^{4}}, \sqrt{2}} \sqrt{2 \max\Set{c^{-2} \gamma_{\Omega}^{-4}, 1} + M_{e}^{2}}}{\min\Set{c^{-1} \gamma_{\Omega}^{2}, c \gamma_{\Omega}^{2} \norm{\omega}_{L_{\infty}(\Omega)}^{-2}, c \gamma_{\Omega}^{2} }}.
        % \leq
        % \frac{\max\{\sqrt{ 1 + 2 \norm{\omega}_{L_{\infty}(\Omega)} \rho^{4}}, \sqrt{2} \}}{e^{-2 \norm{\omega}_{L_{\infty}(\Omega)} T}}
        % \frac{\sqrt{2 \max\{ 1, \sigma^{-2} \gamma_{\Omega}^{-4} \} + M_{e}^{2}}}{\min\{ \sigma \gamma_{\Omega}^{2} \norm{\omega}_{L_{\infty}(\Omega)}^{-2}, \sigma \gamma_{\Omega}^{2} \}}
    \end{equation}
    mit $M_{e}$ und $\rho$ wie in \eqref{eq:var_all_M_e} respektive \eqref{eq:var_all_rho}.
\end{Korollar}

\section{Parametrische Variante} % (fold)
\label{sec:parametrische_variante}

Wir wollen nun aus dem gerade beschriebenen Variationsproblem eine parametrische Variante gewinnen und aufbauend auf \autoref{sec:parametrisches_problem} Regularität bezüglich des Parameters folgern.
Dazu müssen wir den Operator $A \in \mathcal L(V, V')$ aus \eqref{eq:def_op_A} zunächst zu einem parametrischen Operator $A(\sigma)$ mit $\sigma \in \mathcal S$, wobei $\mathcal S \subset \mathbb{R}^{\mathbb{N}}$ ein geeigneter Parameterraum ist, umschreiben.
Dabei beschränken wir uns auf den Fall affiner parametrischer Abhängigkeit \eqref{eq:all_affiner_operator}.
Der Einfachheit halber wählen wir $\mathcal S = [-1, 1]^{\mathbb{N}}$, das heißt $\mathcal S$ sei die Einheitskugel aus $\ell_{\infty}(\mathbb{N})$.

Sei $\Set{ \varphi_{j} }_{j \in \mathbb{N}} \subset L_{\infty}(\Omega)$ ein noch näher zu bestimmendes, passend gewähltes Funktionensystem und $\sigma \in \mathcal S$.
Wir entwickeln nun $\omega$ formal in eine Reihe der Form
\begin{equation}
    \label{eq:reihenentwicklung_omega}
    \omega(\blank; \sigma) = \sum_{j = 1}^{\infty} \sigma_{j} \varphi_{j}.
\end{equation}
Offenbar ist für die Konvergenz der Reihe \eqref{eq:reihenentwicklung_omega} hinreichend, dass $\Set{ \norm{\varphi_{j}}_{L_{\infty}(\Omega)} }_{j \in \mathbb{N}} \in \ell_{1}(\mathbb{N})$ gilt, insbesondere folgt daraus
\begin{equation}
    \norm{\omega(\blank; \sigma)}_{L_{\infty}(\Omega)} \leq \sum_{j = 1}^{\infty} \norm{\varphi_{j}}_{L_{\infty}(\Omega)} < \infty \quad \fa \sigma \in \mathcal S.
\end{equation}
% Diese Eigenschaft wird auch benötigt, denn dadurch erhalten wir aus \thref{lemma:2.2} die für \thref{thm:kunoth:theorem21} notwendigen, von $\sigma$ unabhängigen, Schranken $\beta_{1}$ und $\beta_{2}$.
Damit ist die Wahl des Funktionensystems $\Set{ \varphi_{j} }_{j \in \mathbb{N}} \subset L_{\infty}(\Omega)$ ist entscheidend für die Konvergenz von \eqref{eq:reihenentwicklung_omega}, aber auch für die Erfüllbarkeit von \thref{thm:kunoth:assumption1} respektive \thref{thm:kunoth:assumption2},
und wird in den nächsten Abschnitten genauer behandelt.

% \subsection{Affiner Operator} % (fold)
% \label{ssub:entwicklung_von_}

Wir wollen den Operator $A$ aus \eqref{eq:def_op_A} als affin parametrischen Operator der Form
\begin{equation}
    \label{eq:aff_zerlegung_A}
    A(\sigma) = \hat A + \sum_{j \geq 1} \sigma_{j} A_{j}
\end{equation}
auffassen, beziehungsweise als Bilinearformen
\begin{equation}
     \label{eq:aff_zerelgung_A_bf}
     a(\eta, \zeta; \sigma) = \hat a(\eta, \zeta) + \sum_{j \geq 1} \sigma_{j} a_{j}(\eta, \zeta), \quad \eta, \zeta \in V.
 \end{equation}
Dazu entwickeln wir $\omega$ in eine Reihe der Form \eqref{eq:reihenentwicklung_omega}, das heißt wir erhalten
\begin{equation}
    \label{eq:omega_reihenentwicklung}
    \omega(\blank; \sigma) \colon \Omega \to \mathbb{R}, \quad x \mapsto \omega(x; \sigma) = \sum_{j \geq 1} \sigma_{j} \varphi_{j}(x)
\end{equation}
mit $\sigma \in \mathcal S$.
Eine naheliegende affine Aufteilung des Operators $A$ erhalten wir damit durch die Wahl
\begin{equation}
    \label{eq:affine_zerlegung_A_def}
    \hat A = - c \Delta, \qquad
    A_{j} = \varphi_{j}, \quad j \geq 1.
\end{equation}
Die zugehörigen Bilinearformen lassen sich ebenfalls direkt angeben, denn es gilt
\begin{equation}
    \hat a(\eta, \zeta) = \skprod{\grad \eta}{\grad \zeta}_{L_{2}(\Omega)}, \qquad a_{j}(\eta, \zeta) = \skprod{\varphi_{j} \eta}{\zeta}_{L_{2}(\Omega)}, \quad j \geq 1.
\end{equation}

Die daraus resultierende Raum-Zeit-Variationsformulierung lautet nun:
\begin{Problem}
    Gegeben ein $g \in L_{2}(I; H^{-1}(\Omega))$ und ein $u_{0} \in L_{2}(\Omega)$.
    Finde für alle $\sigma \in \mathcal S$ ein $u(\sigma) \in \mathcal X$ mit
    \begin{equation}
        \label{eq:varprob_2}
        b(u, v; \sigma) = f(v) \quad \text{für alle}~v \in \mathcal Y,
    \end{equation}
    wobei $b(\blank, \blank; \sigma) \colon \mathcal X \times \mathcal Y \times \mathcal S \to \mathbb{R}$ die durch
    \begin{equation}
        \label{eq:buv_2}
        b(u, v; \sigma)
            = \int_{I} \skprod{u_{t}(t)}{v_{1}(t)}_{L_{2}(\Omega)} + a(u(t), v_{1}(t); \sigma) \diff t + \skprod{u(0)}{v_{2}}_{L_{2}(\Omega)}
    \end{equation}
    gegebene Bilinearform und $f(\blank) \colon \mathcal Y \to \mathbb{R}$ definiert ist durch
    \begin{equation}
        \label{eq:var_all_f_wiederholung_2}
        f(v) = \int_{I} \skprod{g(t)}{v_{1}(t)}_{L_{2}(\Omega)} \diff t + \skprod{u_{0}}{v_{2}}_{L_{2}(\Omega)}.
    \end{equation}
\end{Problem}

% subsection entwicklung_von_ (end)

% section parametrische_variante (end)

\section{Der eindimensionale Fall} % (fold)
\label{sec:der_eindimensionale_fall}

Wir beschränken uns nun zunächst auf den Fall einer Raumdimension.
Es sei also $\Omega \subset \mathbb{R}$ ein Intervall, ohne Beschränkung der Allgemeinheit wählen wir $\Omega = [0, 1]$.

Die Wahl des Funktionensystems $\Set{ \varphi_{j} }_{j}$ ist von wesentlicher Bedeutung, denn sie legt fest, welche Funktionen $\omega$ wir durch die Reihenentwicklung \eqref{eq:omega_reihenentwicklung} darstellen können.
Zudem bestimmen wir damit das Konvergenzverhalten in verschiedenen Räumen und die möglichen Randbedingungen.

% TODO: besser formulieren
Eine durch die beabsichtigte Anwendung in \cite{Stasiak:2011ba} motivierte Wahl ist die Entwicklung in Sinusfunktionen.
Da Konvergenz in $L_{\infty}(\Omega)$ wegen $\omega \in L_{\infty}(\Omega)$ notwending ist, erzwingen wir diese durch Gewichtung der Sinusfunktionen.

Zusätzlich zu den Sinusfunktionen fügen wir eine konstante Funktion hinzu, um so inhomogene Randbedingungen für $\omega$ zuzulassen, das heißt wir haben das Funktionensystem $\Set{ \varphi_{j} }_{j \in \mathbb{N}_{0}}$ mit
\begin{equation}
    \label{eq:sinusfunktionen_ansatz}
    \varphi_{0} = K, \qquad
    \varphi_{j} = \frac{K}{(\pi j)^{1 + \epsilon}} \sin(\pi j \blank), \quad j \geq 1,
\end{equation}
wobei $K \in \mathbb{R}_{+}$ als Skalierungsfaktor dient.

An dieser Stelle weisen wir zunächst einige Eigenschaften für \eqref{eq:sinusfunktionen_ansatz} nach.
\begin{Lemma}
    Es gilt
    \begin{alignat}{2}
        \norm{\varphi_{0}}_{L_{\infty}(\Omega)} &= K,
        \qquad&
        \norm{\varphi_{j}}_{L_{\infty}(\Omega)} &= \frac{K}{(\pi j)^{1 + \epsilon}}, \quad j \geq 1,
    % \end{equation}
    % sowie
    % \begin{equation}
    \intertext{sowie}
        \norm{\varphi_{0}}_{H^{1}(\Omega)}  &= K,
        \qquad&
        \norm{\varphi_{j}}_{H^{1}(\Omega)}  &= \frac{K}{\sqrt 2 } \frac{\sqrt{1 + (\pi j)^{2}}}{(\pi j)^{1 + \epsilon}}, \quad j \geq 1.
    \end{alignat}
\end{Lemma}

\begin{Lemma}
    Die Funktionen $\Set{ \varphi_{j} }_{j \in \mathbb{N}}$ bilden ein Orthogonalsystem in $H^{1}(\Omega)$, denn es gilt
    \begin{equation}
        \skprod{\varphi_{j}}{\varphi_{k}}_{H^{1}(\Omega)} = \begin{cases}
            \frac{K}{\sqrt 2 } \frac{\sqrt{1 + (\pi j)^{2}}}{(\pi j)^{1 + \epsilon}},   &j = k \\
            0,          &j \neq k.
        \end{cases}
    \end{equation}
\end{Lemma}

\begin{Lemma}
    Sei $\sigma \in \mathcal S$ und $\epsilon > 0$, dann konvergiert $\omega(\blank; \sigma)$ in $L_{\infty}(\Omega)$.
    Ist $\epsilon > 1$, dann gilt auch Konvergenz in $H^{1}_{0}(\Omega)$.

    \begin{Beweis}
        Sei zunächst $\epsilon > 0$.
        Da $\mathcal S = [0, 1]^{\mathbb{N}}$ ist, erhalten wir die Konvergenz in $L_{\infty}(\Omega)$ nach dem Weierstraßschen Majorantenkriterium via
        \begin{align}
            \sum_{j = 0}^{\infty} \norm{\sigma_{j} \varphi_{j}}_{L_{\infty}(\Omega)}
            &= \sum_{j = 0}^{\infty} \abs{\sigma_{j}} \norm{\varphi_{j}}_{L_{\infty}(\Omega)}
             \leq \norm{\varphi_{0}}_{L_{\infty}(\Omega)} + \sum_{j = 1}^{\infty}  \norm{\varphi_{j}}_{L_{\infty}(\Omega)}
            \\&= K + \frac{K}{\pi^{1+\epsilon}} \sum_{j = 1}^{\infty} \frac{1}{j^{1+\epsilon}}.
        \end{align}
        Diese Reihe konvergiert bekanntlich für alle $\epsilon > 0$, womit wir Konvergenz von $\omega$ in $L_{\infty}(\Omega)$ erhalten.

        Sei nun $\epsilon > 1$.
        Betrachte
        \begin{align}
            \sum_{j = 0}^{\infty} \norm{\sigma_{j} \varphi_{j}}_{H^{1}(\Omega)}
            &= \sum_{j = 0}^{\infty} \abs{\sigma_{j}} \norm{\varphi_{j}}_{H^{1}(\Omega)}
            \leq  \norm{\varphi_{0}}_{H^{1}(\Omega)} + \sum_{j \geq 1} \norm{\varphi_{j}}_{H^{1}(\Omega)}
            \\&= K + \frac{K}{\sqrt{2}} \sum_{j = 1}^{\infty} \frac{1 + \pi j}{(\pi j)^{1+\epsilon}}
            \\&\leq K + \frac{K}{\sqrt{2} \pi^{1 + \epsilon}} \sum_{j = 1}^{\infty} \frac{1}{j^{1+\epsilon}} + \frac{K}{\sqrt{2} \pi^{\epsilon}} \sum_{j = 1}^{\infty} \frac{1}{j^{\epsilon}}.
        \end{align}
        Wegen $\epsilon > 1$ konvergiert sowohl die erste als auch die zweite Reihe.
        Zusammen liefert dies die Konvergenz in $H^{1}_{0}(\Omega)$.
    \end{Beweis}
\end{Lemma}

Wir wollen nun die Regularität von $\omega(\blank; \sigma)$ in Abhängigkeit vom Parameter $\sigma$ nachweisen.

\begin{Satz}
\label{satz:regularitaet_nachrechnen}
    Seien $\epsilon > 0$ und $0 < \kappa < 1$ so gewählt, dass
    \begin{equation}
        \zeta(2 + \epsilon) \leq (\kappa c \pi^{2} - K) \frac{\pi^{2+\epsilon}}{4K C_{\infty}}
    \end{equation}
    gilt,
    wobei $\zeta$ die Riemannsche-$\zeta$-Funktion, $c$ und $K$ die Konstanten aus \eqref{eq:def_op_A} respektive \eqref{eq:sinusfunktionen_ansatz} und $C_{\infty}$ die Einbettungskonstante von $H^{1}_{0}(\Omega) \hookrightarrow L_{\infty}(\Omega)$ sind.
    Dann erfüllt $\Set{\hat A} \cup \Set{ A_{j} }_{j \geq 0}$ \thref{thm:kunoth:assumption2}.

    \begin{Beweis}
        Wir weisen zunächst die inf-sup-Bedingungen \eqref{eq:kunoth:ass2_gamma_0} für $\hat a(\blank, \blank)$ nach und bestimmen die Konstante $\gamma_{0}$.
        Da $\hat a(\blank, \blank)$ symmetrisch ist, genügt es die inf-sup-Bedingung \eqref{eq:kunoth:ass2_gamma_0_a} nachzuweisen. Die zweite inf-sup-Bedingung \eqref{eq:kunoth:ass2_gamma_0_b} folgt dann analog mit dem selben $\gamma_{0}$.

        Nach \thref{lemma:sauter:2.1.48} reicht es, für alle $u \in H^{1}_{0}(\Omega)$ ein $v_{u} \in H^{1}_{0}(\Omega)$ und von $u$ und $v_{u}$ unabhängige Konstanten $C_{1}, C_{2} > 0$ mit
        \begin{equation}
            \hat a(u, v_{u}) \geq C_{1} \norm{u}_{H^{1}(\Omega)}^{2} \quad \text{und} \quad \norm{v_{u}}_{H^{1}(\Omega)} \leq C_{2} \norm{u}_{H^{1}(\Omega)}
        \end{equation}
        zu finden.
        Dann ist die inf-sup-Bedingung \eqref{eq:kunoth:ass2_gamma_0_a} mit $\gamma_{0} = \frac{C_{1}}{C_{2}}$ erfüllt.

        Sei nun also $u \in H^{1}_{0}(\Omega)$ beliebig.
        Wir wählen $v_{u} = u \in V$, es gilt also $C_{2} = 1$.
        Es ergibt sich
        \begin{align}
            \hat a(u, v_{u}) = \hat a(u, u) = c \skprod{\grad u}{\grad u}_{L_{2}(\Omega)} = c \norm{\grad u}_{L_{2}(\Omega)}^{2} \geq c \gamma_{\Omega}^{2} \norm{u}_{H^{1}(\Omega)}^{2},
        \end{align}
        wobei die letzte Abschätzung aus der Poincaré-Friedrichs-Ungleichung \eqref{eq:grundlagen:poincare_friedrichs_ungleichung} folgt.
        Zusammen liefert dies $\gamma_{0} = c \gamma_{\Omega}^{2}$ als inf-sup-Konstante.

        Für den vorliegenden Fall können wir $\gamma_{\Omega}^{2}$ exakt bestimmen.
        Nach \cite[Chapter 11]{Strauss:2007vz} entspricht das Quadrat der optimalen Poincaré-Friedrichs-Konstante $\gamma_{\Omega}^{2}$ gerade dem kleinsten Eigenwert des Laplace-Operators auf $\Omega$ mit Dirichlet-Randbedingung.
        Dieser hat für $\Omega = [0, 1]$ den Wert $\pi^{2}$.
        Wir erhalten damit also $\gamma_{0} = c \pi^{2}$.

        Seien $u, v \in H^{1}_{0}(\Omega)$.
        Für $j = 0$ gilt die simple Abschätzung
        \begin{equation}
            \begin{aligned}
                a_{0}(u, v)
                &= \skprod{\varphi_{0} u}{v}_{L_{2}(\Omega)}
                = K \skprod{u}{v}_{L_{2}(\Omega)}
                \\&\leq K \norm{u}_{L_{2}(\Omega)} \norm{v}_{L_{2}(\Omega)}
                \leq K \norm{u}_{H^{1}(\Omega)} \norm{v}_{H^{1}(\Omega)}
            \end{aligned}
        \end{equation}
        Betrachte für $j \geq 1$
        \begin{align}
            a_{j}(u, v)
            &= \skprod{\varphi_{j} u}{v}_{L_{2}(\Omega)}
            = \int_{I} \varphi_{j} u v \diff x
            \intertext{da $\varphi_{j}$ integrierbar ist und $\varphi_{j}(0) = 0$, können wir dies umschreiben zu}
            a_{j}(u, v)
            &= \int_{I} \frac{\diff}{\diff x} \left( \int_{0}^{x} \varphi_{j}(y) \diff y \right) u v \diff x
            \intertext{woraus wir mittels partieller Integration folgenden Ausdruck erhalten}
            a_{j}(u, v)
            &= - \int_{I} \left( \int_{0}^{x} \varphi_{j}(y) \diff y \right) (u v)' \diff x
            \leq \norm*{\left( \int_{0}^{x} \varphi_{j}(y) \diff y \right) (u v)'}_{L_{1}(\Omega)}.
            \\&\leq \norm*{\int_{0}^{x} \varphi_{j}(y) \diff y }_{L_{\infty}(\Omega)} \norm{(uv)'}_{L_{1}(\Omega)}
        \end{align}
        Die erste Norm können wir weiter abschätzen mit
        \begin{equation}
            \begin{aligned}
                \norm*{\int_{0}^{x} \varphi_{j}(y) \diff y }_{L_{\infty}(\Omega)}
                &= \frac{K}{(\pi j)^{1+\epsilon}} \norm*{\int_{0}^{x} \sin(\pi j y) \diff y }_{L_{\infty}(\Omega)}
                \\&= \frac{K}{(\pi j)^{2+\epsilon}} \norm{ \cos(\pi j x) - 1 }_{L_{\infty}(\Omega)}
                \leq \frac{2 K}{(\pi j)^{2+\epsilon}}.
            \end{aligned}
        \end{equation}
        Aus der zweiten Norm erhalten wir mittels Minkowski- und Hölderungleichung sowie der Einbettung $H^{1}_{0}(\Omega) \hookrightarrow L_{\infty}(\Omega)$ die Abschätzung
        \begin{equation}
            \begin{aligned}
                \norm{(uv)'}_{L_{1}(\Omega)}
                &= \norm{u'v + uv'}_{L_{1}(\Omega)}
                \leq \norm{u'v}_{L_{1}(\Omega)} + \norm{uv'}_{L_{1}(\Omega)}
                \\&\leq \norm{u'}_{L_{1}(\Omega)} \norm{v}_{L_{\infty}(\Omega)} + \norm{u}_{L_{\infty}(\Omega)} \norm{v'}_{L_{1}(\Omega)}
                \\&\leq \norm{1}_{L_{2}(\Omega)} \norm{u'}_{L_{2}(\Omega)} \norm{v}_{L_{\infty}(\Omega)} + \norm{u}_{L_{\infty}(\Omega)} \norm{1}_{L_{2}(\Omega)} \norm{v'}_{L_{2}(\Omega)}
                \\&\leq C_{\infty} \norm{u'}_{L_{2}(\Omega)} \norm{v}_{H^{1}(\Omega)} + C_{\infty} \norm{u}_{H^{1}(\Omega)} \norm{v'}_{L_{2}(\Omega)}
                % \\&\leq \norm{u'}_{L_{2}(\Omega)} \norm{v}_{L_{2}(\Omega)} + \norm{u}_{L_{2}(\Omega)} \norm{v'}_{L_{2}(\Omega)}
                \\&\leq 2 C_{\infty} \norm{u}_{H^{1}(\Omega)} \norm{v}_{H^{1}(\Omega)}
            \end{aligned}
        \end{equation}
        Zusammen also
        \begin{equation}
            a_{j}(u, v)
            \leq \norm*{\int_{0}^{x} \varphi_{j}(y) \diff y }_{L_{\infty}(\Omega)} \norm{(uv)'}_{L_{1}(\Omega)}
            \leq \frac{4 K C_{\infty}}{(\pi j)^{2 + \epsilon}} \norm{u}_{H^{1}(\Omega)} \norm{v}_{H^{1}(\Omega)}.
        \end{equation}
        Betrachte nun
        \begin{align}
                    \sum_{j = 0}^{\infty} \norm{A_{j}}_{\mathcal L(V, V')}
            &= \norm{A_{0}}_{\mathcal L(V, V')} + \sum_{j = 1}^{\infty} \norm{A_{j}}_{\mathcal L(V, V')}
            \leq K + \sum_{j = 1}^{\infty} \frac{4 K  C_{\infty}}{(\pi j)^{2 + \epsilon}}
            \\&\leq K + \frac{4 K  C_{\infty}}{\pi^{2 + \epsilon}} \sum_{j = 1}^{\infty} \frac{1}{j^{2 + \epsilon}}
            = K + \frac{4 K  C_{\infty}}{\pi^{2 + \epsilon}} \zeta(2 + \epsilon)
            % \leq    \sum_{j = 0}^{\infty} \norm{\varphi_{j}}_{L_{\infty}(\Omega)}
            % =       a_{0} + \frac{1}{\pi^{1+\epsilon}} \sum_{j = 1}^{\infty} \frac{1}{j^{1+\epsilon}}
            % = a_{0} + \frac{\zeta(1+\epsilon)}{\pi^{1+\epsilon}}.
        \end{align}
        Für die Gültigkeit von
        \begin{equation}
            \sum_{j \geq 0} \norm{A_{j}}_{\mathcal L(V, V')} \leq \kappa \gamma_{0}
        \end{equation}
        für ein $0 < \kappa < 1$ ist damit also
        \begin{equation}
            \zeta(2 + \epsilon) \leq (\kappa c \pi^{2} - K) \frac{\pi^{2+\epsilon}}{4K  C_{\infty}}
        \end{equation}
        mit $\epsilon > 0$ hinreichend.
    \end{Beweis}
\end{Satz}

Zusammenfassend erhalten wir für diese Wahl des Funktionensystems $\Set{ \varphi_{j} }_{j}$ die Aussage

\begin{Korollar}
    Unter den Voraussetzungen von \thref{satz:regularitaet_nachrechnen} gilt \thref{thm:kunoth:theorem21}.
\end{Korollar}

% subsection nachrechnen_von_thref_thm_kunoth_assumption2 (end)

% section der_eindimensionale_fall (end)

\clearpage
\section{Zu klärende Fragen} % (fold)
\label{sub:zu_kl_rende_fragen}

\begin{enumerate}
    \item Wohldefiniertheit der PDE \eqref{eq:parabolische_pde}, das heißt die Voraussetzungen von \thref{thm:schwab09:theorem51} nachweisen. Weiterhin lassen sich damit die inf-sup-Bedingung von \eqref{eq:varprob} nachrechnen und damit die Schranken für $B$ und $B^{-1}$ bestimmen.
    \item Parametrische Variante des Variationsproblems herleiten.
    Dazu Ansetzen mit Entwicklung des Parameters $\omega$ in eine Reihe
    \begin{equation}
        \omega = \sum_{j = 0}^{\infty} \sigma_{j} \varphi_{j}.
    \end{equation}
    Dabei ergeben sich folgende Fragen:
    \begin{enumerate}
        \item Konvergenz der Reihe? Notwendig ist Konvergenz in $L_{\infty}(\Omega)$, da die Norm $\norm{\omega}_{L_{\infty}(\Omega)}$ mehrfach in Abschätzungen verwendet wird.
        \item Weiterhin ist eventuell Konvergenz in einem Unterraum $Z \hookrightarrow L_{\infty}(\Omega)$ wünschenswert.
        Zum Beispiel in $H^{1}(\Omega)$?
        \item Welche Bedingungen ergeben sich an $\sigma_{j}$ und $\varphi_{j}$?
        \item Welches Funktionensystem $\Set{ \varphi_{j} }_{j}$ ist überhaupt sinnvoll?
        Die Wahl der $\varphi_{j}$ entscheidet maßgeblich über Konvergenz der Reihenentwicklung.
        Welche Randvorgaben sind angestrebt?
        Dies wird ebenfalls durch die $\varphi_{j}$ geregelt.
    \end{enumerate}
    \item Welche affine Zerlegung $A(\sigma) = A_{0} + \sum_{j} \sigma_{j} A_{j}$ ist brauchbar?
    Wie genau sehen die $A_{j}$ aus?
    \item Nachweisen, dass $A(\sigma)$ \thref{thm:kunoth:assumption1} oder \thref{thm:kunoth:assumption2} erfüllt und mittels \thref{thm:kunoth:theorem21} die gewünschte Regularität von $B(\sigma)$ bezüglich $\sigma$ gewinnen.
    \item Die Abschätzungen in \thref{satz:regularitaet_nachrechnen} lassen sich noch deutlich verbessern.
    Das gilt wahrscheinlich auch für andere Abschätzungen!
\end{enumerate}

