%!TEX root = ../main.tex

\chapter{Funktionalanalytische Grundlagen} % (fold)
\label{cha:funktionalanalytische_grundlagen}

\section{Bochner-Integrale und Bochner-Räume} % (fold)
\label{sec:bochner_r_ume}

\begin{Definition}[Bochner-Räume]
    Sei $X$ ein Banachraum, $(a, b) \subset \mathbb{R}$, $- \infty \leq a < b \leq \infty$, ein offenes Intervall und $1 \leq p < \infty$.
    Wir nennen $L_{p}(a, b; X)$ einen Bochner-Raum und meinen damit die Menge der Äquivalenzklassen $L_{p}$-integrierbarer Funktionen $f \colon [a, b] \to X$, das heißt alle Lebesgue-messbaren Funktionen auf $[a, b]$ mit
    \begin{equation}
        \norm{f}_{L_{p}(a, b; X)} \coloneqq \left( \int_{a}^{b} \norm{f(t)}_{X}^{p} \diff t \right)^{\frac 1 p} < \infty.
    \end{equation}
    Analog bezeichnen wir mit $L_{\infty}(a, b; X)$ die Menge der Äquivalenzklassen die fast überall auf $[a, b]$ beschränkt sind, das heißt
    \begin{equation}
        \norm{f}_{L_{\infty}(a, b; X)} \coloneqq \esssup_{t \in [a, b]} \norm{f(t)}_{X} < \infty.
    \end{equation}
\end{Definition}

\begin{Lemma}
    Sei $1 \leq p \leq \infty$. Dann ist $L_{p}(a, b; X)$ ein Banachraum.
    Ist $H$ ein Hilbertraum, dann ist insbesondere auch $L_{2}(a, b; H)$ ein Hilbertraum.
\end{Lemma}

\begin{Lemma}[Eigenschaften]
    \begin{enumerate}
        \item Ist $[a, b] \subset \mathbb{R}$ ein endliches Intervall, dann gilt die stetige Einbettung
        \begin{equation}
            L_{q}(a, b; X) \hookrightarrow L_{p}(a, b; X), \qquad q \geq p \geq 1.
        \end{equation}
        \item Sind $X$ und $Y$ Banachräume mit $X \hookrightarrow Y$, dann gilt die stetige Einbettung
        \begin{equation}
            L_{p}(a, b; X) \hookrightarrow L_{p}(a, b; Y), \qquad 1 \leq p \leq q.
        \end{equation}
    \end{enumerate}
\end{Lemma}
