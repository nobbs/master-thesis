%!TEX root = main.tex

%%%%%%%%%%%%%%%%%%%%%%%%%%%%%%%%%%%%%%%%%%%%%%%%%%%%%%%%%%%%%%%%%%%%%%%%%%%%%%%
%%% Allgemeines

% Mehr Speicher für latex
\usepackage{etex}

% Encoding und Schriftsystem
\usepackage[utf8]{inputenc}
\usepackage[T1]{fontenc}

% Deutsche Silbentrennung
\usepackage[ngerman]{babel}

% Deutsche Anführungszeichen
\usepackage[babel,german=quotes]{csquotes}

% Bessere Standard-Schriftart
\usepackage{lmodern}

% Fraktur-Schriftsatz
\usepackage{eufrak}

% Typographische Kleinigkeiten
\usepackage{microtype}

% Graphiken und Farben
\usepackage{graphicx, color}

% PDF-Verlinkungen und Metadaten
\usepackage[
    colorlinks=true,
]{hyperref}

% \usepackage{nameref}

% Bibliographie-Stil
\usepackage[%
    backend=biber,
    style=alphabetic,
    backref=true,
]{biblatex}
\addbibresource{literature.bib}

% Fancyhdr-Ersatz für scrbook
\usepackage[
    automark,
    headsepline
]{scrlayer-scrpage}

% enums
\usepackage{enumitem}
\newlist{thmenumerate}{enumerate}{1}
\setlist[thmenumerate]{label={\upshape(\roman*)}, align=left, widest=iii, leftmargin=*}

% Dashed Linien...
\usepackage{dashrule}

% Symbolverzeichnis
\usepackage[german,intoc]{nomencl}
\setlength{\nomitemsep}{-\parsep}
\makenomenclature

%%%%%%%%%%%%%%%%%%%%%%%%%%%%%%%%%%%%%%%%%%%%%%%%%%%%%%%%%%%%%%%%%%%%%%%%%%%%%%%
%%% Mathematik

% Standardpackages
\usepackage{amsmath, amssymb, stmaryrd, mathtools}

% "Bessere" Theorem-Umgebungen
\usepackage[amsmath, thmmarks, hyperref, thref]{ntheorem}

% Setze Label-Nummern nur, wenn diese auch referenziert werden
\mathtoolsset{showonlyrefs, showmanualtags}

\providecommand\given{} % so it exists
\newcommand\SetSymbol[1][]{
   \nonscript\,#1\vert\nonscript\,\mathopen{}\allowbreak}
\DeclarePairedDelimiterX\Set[1]{\lbrace}{\rbrace}{ \renewcommand\given{\SetSymbol[\delimsize]} #1 }

%%%%%%%%%%%%%%%%%%%%%%%%%%%%%%%%%%%%%%%%%%%%%%%%%%%%%%%%%%%%%%%%%%%%%%%%%%%%%%%
%%% Nebensächliches

% Querformatseiten
\usepackage{pdflscape}

% Labels am Seitenrand anzeigen
\usepackage{showlabels}

% todos
\usepackage[german]{todonotes}

% Blindtext
\usepackage{blindtext}
% \blindmathtrue{}
