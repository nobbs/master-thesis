%!TEX root = ../main.tex

\chapter{Inhalt der Begleit-DVD} % (fold)
\label{cha:inhalt_der_begleit_dvd}

Dieser Arbeit liegt eine DVD bei, welche die Implementierungen der beschriebenen Verfahren, sowie die in \autoref{cha:Beispiele} angeführten Beispiele enthält.
Weiterhin findet man den Inhalt der DVD identisch auch als Git-Repository unter \url{https://github.com/nobbs/thesis}.

Die Implementierungen sind vollständig in \textcite{Matlab} gehalten und daher plattformunabhängig.
Neben dem eigentlichen MATLAB-Softwarepaket wird nur die \emph{Optimization Toolbox} benötigt für die in \autoref{scm} beschrieben Successive Constraint Method benötigt.

\begin{figure}[tb]
    \dirtree{%
    .1 /.
    .2 code.
    .3 examples.
    .4 datasets.
    .3 lib.
    .3 src.
    .3 test.
    .2 config.
    .2 doc.
    .3 index.html.
    .2 tex.
    .2 README.md.
    }
    \caption{Ausschnitt der Verzeichnisstruktur}
    \label{fig:dvd}
\end{figure}

Ein Ausschnitt der wichtigsten Elemente der Verzeichnisstruktur findet sich in \autoref{fig:dvd}.



% chapter inhalt_der_begleit_dvd (end)
