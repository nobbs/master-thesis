%!TEX root = ../main.tex

\chapter{Petrov"=Galerkin"=Verfahren} % (fold)
\label{chapter:galerkin}

In diesem Kapitel führen wir das sogenannte Petrov"=Galerkin"=Verfahren ein, welches als Grundlage für die Reduzierte-Basis-Methoden des nächsten Kapitels dienen wird, und betrachten erste numerische Experimente.
Dies werden wir, wie auch bereits bei den funktionalanalytischen Grundlagen in \cref{chapter:grundlagen}, zunächst möglichst allgemein halten.
Als Quelle dienen vor allem die Arbeiten von \textcite{Braess:2007wm,Patera:2007un,Quarteroni:2011jm}, weitere werden an den entsprechenden Stellen angegeben.

Die Rahmenbedingungen für die nachfolgenden Ausführungen seien wie folgt:
Seien $\mathcal X$ und $\mathcal Y$ zwei Hilberträume und sei weiter $b \colon \mathcal X \times \mathcal Y \to \mathbb{R}$ eine stetige Bilinearform und $f \colon \mathcal Y \to \mathbb{R}$ ein stetiges lineares Funktional.
Wir betrachten das abstrakte Variationsproblem
\begin{equation}
    \label{eq:abstraktes_variationsproblem_galerkin}
    \text{Finde}~u \in \mathcal X \colon \quad  b(u, v) = f(v) \quad \fa v \in \mathcal Y.
\end{equation}
Unter welchen Bedingungen dieses Variationsproblem lösbar ist, haben wir bereits in \cref{chapter:grundlagen} diskutiert.
Für den Rest dieses Kapitels nehmen wir an, dass \cref{eq:abstraktes_variationsproblem_galerkin} korrekt gestellt ist und insbesondere die inf-sup-Konstante $\beta$ die Bedingung
\begin{equation}
    \label{eq:abstraktes_variationsproblem_galerkin_infsup}
    \beta := \inf_{u \in \mathcal X} \sup_{v \in \mathcal Y} \frac{b(u, v)}{\norm{u}_{\mathcal X} \norm{v}_{\mathcal Y}} > 0
\end{equation}
erfüllt.

\section{Grundlagen des Petrov"=Galerkin"=Verfahrens} % (fold)
\label{section:petrov_galerkin_grundlagen}

Die \emph{Petrov-Galerkin-Verfahren} stelle eine Unterklasse der Galerkin-Verfahren dar und werden als Bezeichnung für Verfahren, die auf Variationsprobleme, bei denen Ansatz- und Testfunktionenraum nicht identisch sind, zugeschnitten sind, verwendet.

Die grundlegende Idee hinter Galerkin-Verfahren ist die Diskretisierung des Variationsproblems \cref{eq:abstraktes_variationsproblem_galerkin} durch Approximation der im Allgemeinen unendlichdimensionalen Hilberträume $\mathcal X$ und $\mathcal Y$ mittels endlichdimensionaler Unterräume $\mathcal X^{\mathcal N} \subset \mathcal X$ und $\mathcal Y^{\mathcal M} \subset \mathcal Y$.
Wir beschränken uns hier auf den Fall $\mathcal N = \mathcal M$.
Zwar ist auch für den Fall $\mathcal M > \mathcal N$ eine sinnvolle Anwendung des Galerkin-Verfahrens möglich, allerdings erfolgt diese dann im Sinne einer Residuums-Minimierung, vergleiche \cite{Andreev:2012ep}, und verträgt sich deswegen nicht mit der Anwendung der Reduzierte-Basis-Methoden in \cref{chapter:rbm}.

Wir halten uns hauptsächlich an die Ausführungen von \textcite[Section 3.1]{Nochetto:2009il} und beginnen, indem wir den Begriff der Petrov"=Galerkin"=Lösung definieren.

\begin{Definition}
    \label{definition:disrekte_loesung}
    Seien durch $\mathcal X^{\mathcal N} \subset \mathcal X$ und $\mathcal Y^{\mathcal N} \subset \mathcal Y$ Unterräume der Dimension $\mathcal N \in \mathbb{N}$ gegeben.
    Als \emph{Petrov"=Galerkin"=Lösung} von \cref{eq:abstraktes_variationsproblem_galerkin} bezeichnen wir eine Lösung $u^{\mathcal N} \in \mathcal X^{\mathcal N}$ des Variationsproblems:
    \begin{equation}
        \label{eq:diskretes_variationsproblem}
        \text{Finde}~u^{\mathcal N} \in \mathcal X^{\mathcal N} \colon \quad  b(u^{\mathcal N}, v) = f(v) \quad \fa v \in \mathcal Y^{\mathcal N}.
    \end{equation}
\end{Definition}

Zur einfacheren Unterscheidung bezeichnen wir \cref{eq:abstraktes_variationsproblem_galerkin} im Weiteren als stetiges und \cref{eq:diskretes_variationsproblem} als diskretes Variationsproblem.

\begin{Bemerkung}
    \label{bemerkung:zur_wohldefiniertheit}
    Anders als bei den bekannten Galerkin"=Verfahren für elliptische Probleme führt eine solche Diskretisierung eines korrekt gestellten Problems nicht automatisch zu einem korrekt gestellten diskreten Problem.
    Dies wird deutlich durch
    \begin{equation}
        \sup_{v \in \mathcal Y} \frac{b(u, v)}{\norm{v}_{\mathcal Y}} \geq \sup_{v \in \mathcal Y^{\mathcal N}} \frac{b(u, v)}{\norm{v}_{\mathcal Y}} \quad \fa u \in \mathcal X.
    \end{equation}
    Dadurch ist, selbst wenn die stetige inf-sup-Bedingung \cref{eq:abstraktes_variationsproblem_galerkin_infsup} gilt, nicht sichergestellt, dass die diskrete inf-sup-Konstante $\beta^{\mathcal N}$ von \cref{eq:diskretes_variationsproblem} die entsprechende Bedingung
    \begin{equation}
        \label{eq:diskrete_inf_sup_kosntante}
        \beta^{\mathcal N} := \inf_{u \in \mathcal X^{\mathcal N}} \sup_{v \in \mathcal Y^{\mathcal N}} \frac{b(u, v)}{\norm{u}_{\mathcal X} \norm{v}_{\mathcal Y}} = \inf_{v \in \mathcal Y^{\mathcal N}} \sup_{u \in \mathcal X^{\mathcal N}} \frac{b(u, v)}{\norm{u}_{\mathcal X} \norm{v}_{\mathcal Y}} > 0
    \end{equation}
    erfüllt.
   Anders als im stetigen Fall, gilt aufgrund der endlichen Dimension von $\mathcal X^{\mathcal N}$ und $\mathcal Y^{\mathcal N}$ stets die Gleichheit der beiden inf"=sup"=Konstanten in \cref{eq:diskrete_inf_sup_kosntante}.

   Lediglich die Stetigkeit muss nicht explizit nachgewiesen werden, da die diskrete Stetigkeitskonstante $\gamma^{\mathcal N}$ stets durch die des stetigen Variationsproblems von oben beschränkt wird.
\end{Bemerkung}

Der folgende Satz fasst einige äquivalente Bedingungen zusammen, welche sicherstellen, dass das resultierende diskrete Variationsproblem \cref{eq:diskretes_variationsproblem} korrekt gestellt ist.

\begin{Satz}
\label{satz:galerkin_wohldefiniertheit}
    Seien $\mathcal X^{\mathcal N}$ und $\mathcal Y^{\mathcal N}$ Unterräume von $\mathcal X$ respektive $\mathcal Y$ mit Dimension $\mathcal N$ und sei weiter $f \in (\mathcal Y^{\mathcal N})'$.
    Dann besitzt das diskrete Variationsproblem \cref{eq:diskretes_variationsproblem} genau dann eine eindeutige Lösung $u^{\mathcal N} \in \mathcal X^{\mathcal N}$, wenn eine der folgenden äquivalenten Bedingungen erfüllt ist:
    \begin{thmenumerate}
        \item \label{punkt:gal:tfae_diskrete_inf_sup} Es gilt die diskrete inf-sup-Bedingung \cref{eq:diskrete_inf_sup_kosntante}.
        \item \label{punkt:gal:tfae_infsup_x} Es gilt
            \begin{equation}
                \inf_{u \in \mathcal X^{\mathcal N}} \sup_{v \in \mathcal Y^{\mathcal N}} \frac{b(u, v)}{\norm{u}_{\mathcal X} \norm{v}_{\mathcal Y}} > 0.
            \end{equation}
        \item \label{punkt:gal:tfae_infsup_y} Es gilt
            \begin{equation}
                \inf_{v \in \mathcal Y^{\mathcal N}} \sup_{u \in \mathcal X^{\mathcal N}} \frac{b(u, v)}{\norm{u}_{\mathcal X} \norm{v}_{\mathcal Y}} > 0.
            \end{equation}
        \item \label{punkt:gal:tfae_injektiv1} Für jedes $0 \neq u \in \mathcal X^{\mathcal N}$ existiert ein $v \in \mathcal Y^{\mathcal N}$ mit $b(u, v) \neq 0$.
        \item \label{punkt:gal:tfae_injektiv2} Für jedes $0 \neq v \in \mathcal Y^{\mathcal N}$ existiert ein $u \in \mathcal X^{\mathcal N}$ mit $b(u, v) \neq 0$.
    \end{thmenumerate}

    \begin{Beweis}
        Siehe \cite[Theorem 3.1, Proposition 3.1]{Nochetto:2009il}.
    \end{Beweis}
\end{Satz}

Obwohl die Bedingungen \cref{punkt:gal:tfae_infsup_x,punkt:gal:tfae_infsup_y,punkt:gal:tfae_injektiv1,punkt:gal:tfae_injektiv2} augenscheinlich angenehmer zu Handhaben sind, hat Bedingung \cref{punkt:gal:tfae_diskrete_inf_sup} eine besondere Bedeutung, da die diskrete inf"=sup"=Konstante wichtiger Bestandteil der nachfolgenden Stabilitätsaussage ist.

\begin{Satz}
    \label{satz:galerkin_stabilitaet}
    Gilt die diskrete inf-sup-Bedingung \cref{eq:diskrete_inf_sup_kosntante}, dann erfüllt die Petrov"=Galerkin"=Lösung $u^{\mathcal N} \in \mathcal X^{\mathcal N}$ die Abschätzung
    \begin{equation}
        \label{eq:galerkin_statibilitaet}
        \norm{u^{\mathcal N}}_{\mathcal X} \leq \frac{1}{\beta^{\mathcal N}} \norm{f}_{\mathcal Y'}.
    \end{equation}

    \begin{Beweis}
        Direkte Folgerung aus dem \acl{bnb}, \cref{satz:bnb_theorem}.
    \end{Beweis}
\end{Satz}

Eine weitere, äußerst nützliche Eigenschaft der Galerkin"=Verfahren ist die sogenannte Galerkin"=Orthogonalität, welche zusammen mit obiger Stabilitätsaussage zu einer Fehlerabschätzung verarbeitet werden kann.

\begin{Lemma}[Galerkin-Orthogonalität]
    \label{lemma:galerkin_orthogonalitaet}
    Sei $u \in \mathcal X$ die Lösung des stetigen Variationsproblems \cref{eq:abstraktes_variationsproblem_galerkin} und sei $u^{\mathcal N} \in \mathcal X^{\mathcal N}$ die Lösung eines zugehörigen diskreten Variationsproblems \cref{eq:diskretes_variationsproblem}.
    Dann gilt
    \begin{equation}
        \label{eq:galerkin_orthogonalitaet}
        b(u - u^{\mathcal N}, v) = 0 \quad \fa v \in \mathcal Y^{\mathcal N},
    \end{equation}
    das heißt, der Fehler $u - u^{\mathcal N}$ ist orthogonal zu $\mathcal Y^{\mathcal N}$.

    \begin{Beweis}
        Ausnutzen der Bilinearität von $b$ liefert für beliebiges $v \in \mathcal Y^{\mathcal N}$ die Gleichung
        \begin{equation}
            b(u - u^{\mathcal N}, v) = b(u, v) - b(u^{\mathcal N}, v) = f(v) - f(v) = 0,
        \end{equation}
        wobei die zweite Gleichheit durch $\mathcal Y^{\mathcal N} \subset \mathcal Y$ gerechtfertigt ist.
    \end{Beweis}
\end{Lemma}

\begin{Satz}[Lemma von Céa]
    \label{satz:lemma_von_cea}
    Sei $u \in \mathcal X$ die Lösung des stetigen Variationsproblems \cref{eq:abstraktes_variationsproblem_galerkin} und $u^{\mathcal N} \in \mathcal X^{\mathcal N}$ die diskrete Lösung von \cref{eq:diskretes_variationsproblem}.
    Sei weiter $\gamma$ die Stetigkeitskonstante des stetigen Variationsproblems.
    Der Fehler $u - u^{\mathcal N}$ erfüllt die Ungleichung
    \begin{equation}
        \label{eq:lemma_von_cea}
        \norm{u - u^{\mathcal N}}_{\mathcal X} \leq \frac{\gamma}{\beta^{\mathcal N}} \inf_{w \in \mathcal X^{\mathcal N}} \norm{u - w}_{\mathcal X}.
    \end{equation}

    \begin{Beweis}
        Siehe \cite[Theorem 3.2]{Nochetto:2009il}.
    \end{Beweis}
\end{Satz}

Anhand von \cref{satz:galerkin_stabilitaet} und \cref{satz:lemma_von_cea} wird die Wichtigkeit der diskreten inf-sup-Konstante $\beta^{\mathcal N}$ deutlich.
Insbesondere motiviert dies die abschließende Forderung der folgenden Stabilitätseigenschaft für die diskreten Variationsprobleme.

\begin{Definition}
\label{definition:stabile_diskretisierung}
    Sei $\Set{(\mathcal X^{\mathcal N}, \mathcal Y^{\mathcal N})}_{\mathcal N \geq 1}$ eine Folge von endlichdimensionalen Unterräumen mit zugehörigen diskreten inf-sup-Konstanten $\Set{\beta^{\mathcal N}}_{\mathcal N \geq 1}$.
    Wir nennen diese Diskretisierungen \emph{stabil}, wenn ein $\beta > 0$ mit
    \begin{equation}
        \inf_{\mathcal N \geq 1} \beta^{\mathcal N} \geq \beta > 0
    \end{equation}
    existiert.
\end{Definition}


\section{Raum-Zeit-Diskretisierung} % (fold)
\label{section:raum_zeit_diskretisierung}

Wir kehren zu der Raum"=Zeit"=Variationsformulierung \cref{eq:raum_zeit_variationsformulierung} respektive \cref{eq:parametrisches_rz_variationsproblem} der Propagator"=Differentialgleichung aus \cref{chapter:propagator_differentialgleichung} zurück und wollen an dieser Stelle eine Diskretisierung mit Hilfe eines Petrov"=Galerkin"=Verfahrens durchführen.
Dazu konstruieren wir nun endlichdimensionale Unterräume $\mathcal X^{\mathcal N}$ und $\mathcal Y^{\mathcal N}$, für welche wir anschließend Bedingungen angeben, die zu einer stabilen Diskretisierung im Sinne von \cref{definition:stabile_diskretisierung} führen.

Um dies zu bewerkstelligen, greifen wir auf die Charakterisierung der Bochner"=Sobolev"=Räume als Hilbertraum-Tensorprodukte aus \cref{satz:bochner_sobolev_raum_als_tensorprodukt} zurück und können so die nachfolgende Konstruktion in einen rein-räumlichen und einen rein-zeitlichen Anteil zerlegen.

\begin{Korollar}
    Der verwendete Ansatz- und Testraum \cref{eq:ansatzraum_X,eq:testraum_Y} lassen sich auch schreiben als
    \begin{equation}
        \label{eq:ansatzraum_testraum_tensor}
        \begin{aligned}
        \mathcal X &= L_{2}(I; V) \cap H^{1}(I; V')
            = (L_2(I) \otimes V) \cap (H^{1}(I) \otimes V'),\\
        \mathcal Y &= L_{2}(I; V) \times H = (L_{2}(I) \otimes V) \times H.
        \end{aligned}
    \end{equation}
\end{Korollar}

Die nachfolgende Konstruktion orientiert sich an \cite{Andreev:2012uh}.
Für die zeitliche Komponente verwenden wir die zwei endlichdimensionalen Räume $E^{\mathcal K} \subset H^{1}(I)$ und $F^{\mathcal K} \subset L_{2}(I)$ konstruieren, wobei die Dimensionen so gewählt werden, dass
\begin{equation}
    \label{eq:dimensionen_zeitliche_raeume}
    \dim E^{\mathcal K} = \mathcal K + 1, \qquad \dim F^{\mathcal K} = \mathcal K
\end{equation}
gilt.
Die Räume $V, H$ und $V'$ der räumlichen Komponenten können wir aufgrund der Gelfand-Tripel-Struktur alle durch den selben endlichdimensionalen Raum $V^{\mathcal J}$ diskretisieren, welcher die Dimension
\begin{equation}
    \label{eq:dimension_raeumliche_raeume}
    \dim V^{\mathcal J} = \mathcal J
\end{equation}
hat.
Diese Teilräume liefern nun zusammen mit der Tensorprodukt-Darstellung \cref{eq:ansatzraum_testraum_tensor} von Ansatz- und Testraum die Diskretisierungen
\begin{equation}
\label{eq:diskrete_tensor_raueme}
    \mathcal X^{\mathcal N} := E^{\mathcal K} \otimes V^{\mathcal J}, \qquad \mathcal Y^{\mathcal N} := (F^{\mathcal K} \otimes V^{\mathcal J}) \times V^{\mathcal J}
\end{equation}
mit den Dimensionen
\begin{equation}
    \mathcal N := \dim \mathcal X^{\mathcal N} = (\mathcal K + 1) \mathcal J = \mathcal K \mathcal J + \mathcal J = \dim \mathcal Y^{\mathcal N}.
\end{equation}

Nachdem nun geklärt ist, wie die Raum-Zeit-Räume zusammengesetzt werden können, konstruieren wir nun die einzelnen Bauteile.

\subsection*{Zeitliche Komponente}

Für die zeitliche Komponente benötigen wir zunächst eine Diskretisierung des Zeitintervalls $I = [0, T]$ in Form eines nicht notwendigerweise äquidistanten Gitters
\begin{equation}
\label{eq:zeitgitter}
    \mathcal T^{\mathcal K} := \Set{0 = t_0 < t_1 < \dots < t_{\mathcal K - 1} < t_{\mathcal K} = T} \subset I.
\end{equation}

Für die Diskretisierung $E^{\mathcal K}$ des Ansatzraumes verwenden wir stetige, stückweise affine Funktionen, genauer die klassischen Hutfunktionen $\theta_{k}$ auf den Gitterpunkten $t_{k} \in \mathcal T^{\mathcal K}$ für $k = 0, \dots, \mathcal K$.
Diese lassen sich alternativ mittels $\theta_{k}(t_{\tilde{k}}) = \delta_{k \tilde k}$ auch durch das bekannte Kronecker-Delta $\delta_{k \tilde k}$ charakterisieren.
Wir fassen diese Hutfunktionen zu einer Basis $B(E, \mathcal K)$ zusammen und definieren damit
\begin{equation}
    \label{eq:zeitanteil_ansatzraum}
    E^{\mathcal K } := \spn B(E, \mathcal K) = \spn \Set{ \theta_{k} \given k = 0, \dots, \mathcal K }.
\end{equation}

Für den Testraum-Anteil $F^{\mathcal K}$ verwenden wir stattdessen durch charakteristische Funktionen gegebene stückweise konstante Funktionen $\xi_{k} = \chi_{(t_{k-1}, t_{k})}$ auf den Teilintervallen $(t_{k - 1}, t_{k}) \subset I$ mit den Gitterpunkten $t_{k} \in \mathcal T^{\mathcal K}$.
Erneut fassen wir diese zu einer Basismenge $B(F, \mathcal K)$ zusammen und definieren
\begin{equation}
    \label{eq:zeitanteil_testraum}
    F^{\mathcal K} := \spn B(F, \mathcal K) = \spn \Set{ \xi_{k} \given k = 1, \dots, \mathcal K}.
\end{equation}

Nach \cite{Andreev:2012ep} führt diese Wahl von Basisfunktionen für Ansatz- und Testraum zu einem Crank-Nicolson-ähnlichen Verfahren, welches auch als Time-Stepping-Verfahren interpretiert werden kann.
Darauf wollen wir an dieser Stelle nicht weiter eingehen, da wir die resultieren diskreten Raum-Zeit-Variationsprobleme direkt lösen werden.


\subsection*{Räumliche Komponente} % (fold)

Auf die räumliche Diskretisierung im Allgemeinen Fall wollen wir hier nur kurz eingehen.
Hierzu können die meisten von Galerkin-Verfahren bekannten Ansätze verwendet werden, beispielsweise die Bestimmung einer Triangulation von $\Omega$ und anschließende Verwendung von Finiten-Elementen, aber auch globale Basisfunktionen sind denkbar.
Wir werden bei der Stabilitätsuntersuchung im nachfolgenden Abschnitt feststellen, dass das resultierende Raum-Zeit-Verfahren in gewisser Weise modular bezüglich der räumlichen Komponente ist, weswegen wir an dieser Stelle nur eine notationelle Definition tätigen wollen.

Für den räumlichen Anteil definieren wir ebenfalls eine Basis, $B(V, \mathcal J)$, und den endlichdimensionalen Raum durch
\begin{equation}
    \label{eq:raeumlicher_anteil}
    V^{\mathcal J} := \spn B(V, \mathcal J) = \spn \Set{ \eta_{j} \given j = 1, \dots, \mathcal J}.
\end{equation}


\subsection*{Raum-Zeit-Diskretisierung} % (fold)

Unter Verwendung der Tensorprodukt-Darstellung \cref{eq:ansatzraum_testraum_tensor} können wir nun die beiden einzeln betrachteten Komponenten zu den endlichdimensionalen Raum"=Zeit"=Unterräumen zusammensetzen.
Dazu definieren wir mittels Tensorprodukt zunächst die Basen
\begin{align}
    \Phi &:= \Set{\theta \otimes \eta \given \theta \in B(E, \mathcal K), \eta \in B(V, \mathcal J)}, \\
    \Psi_{1} &:= \Set{\xi \otimes \eta \given \xi \in B(F, \mathcal K), \eta \in B(V, \mathcal J)}.
    \intertext{und weiter}
    \Psi &:= (\Psi_{1} \times \Set{ 0 }) \cup (\Set{0} \times B(V, \mathcal J)).
\end{align}
Nach Konstruktion und den Definitionen \cref{eq:diskrete_tensor_raueme} gilt nun
\begin{equation}
    \label{eq:diskreter_ansatz_und_testraum_als_span}
    \mathcal X^{\mathcal N} = E^{\mathcal K} \otimes V^{\mathcal J} = \spn \Phi,
    \quad
    \mathcal Y^{\mathcal N} = (F^{\mathcal K} \otimes V^{\mathcal J}) \times V^{\mathcal J} = \spn \Psi.
\end{equation}

\subsection*{Stabilität der Diskretisierung} % (fold)

Wie bereits erwähnt, wollen wir an dieser Stelle Bedingungen angeben, unter denen die diskreten Raum-Zeit-Räume $\mathcal X^{\mathcal N}$ und $\mathcal Y^{\mathcal N}$ aus \cref{eq:diskreter_ansatz_und_testraum_als_span} zu stabilen Diskretisierungen im Sinne von \cref{definition:stabile_diskretisierung} führen.
Hierzu verweisen wir erneut auf die Arbeit \cite[Section 5.2]{Andreev:2012ep}, in welcher \citeauthor{Andreev:2012ep} für drei verschiedene Ansätze für die zeitliche Komponente die Stabilität unter gewissen Bedingungen nachweist.
Da für einen korrekten und verständlichen Nachweis dieser Aussagen eine extensive Vorarbeit notwendig ist, verzichten wir an dieser Stelle darauf und verweisen auf obige Arbeit.
Stattdessen beschränken wir uns auf die Einführung des Nötigsten um die Bedingungen anzugeben und kehren bei den numerischen Experimenten noch einmal zur Stabilität zurück.

Für den hier konstruierten Fall führen die Ausführungen in \cite{Andreev:2012ep} zu einer Courant-Friedrichs-Levi-Bedingung, kurz CFL-Bedingung, wie man sie üblicherweise bei der Diskretisierung von hyperbolischen partiellen Differentialgleichung antrifft.
Dazu definieren wir zunächst die maximale Schrittweite des Zeitgitters $\mathcal T^{\mathcal K}$ als
\begin{equation}
    \label{eq:maximale_zeitschrittweite}
    \max \Delta \mathcal T^{\mathcal K} := \max_{k = 1, \dots, \mathcal K} \abs{t_{k} - t_{k - 1}}
\end{equation}
und weiter die sogenannte CFL-Zahl nach \cite[62]{Andreev:2012ep}.

\begin{Definition}
    Seien $\mathcal T^{\mathcal K}$ ein Gitter des Zeitintervalls $I = [0, T]$ und $V^{\mathcal J} \subset V$ ein endlichdimensionaler Unterraum.
    Als \emph{Courant-Friedrichs-Levi-Zahl}, kurz \emph{CFL-Zahl}, der Diskretisierung $(\mathcal X^{\mathcal N}, \mathcal Y^{\mathcal N})$ aus \cref{eq:diskreter_ansatz_und_testraum_als_span} bezeichnen wir
    \begin{equation}
        \label{eq:cfl_zahl}
        \mathrm{CFL}_{\mathcal N} := \max \Delta \mathcal T^{\mathcal K} \sup_{\eta \in V^{\mathcal J} \setminus \Set{0} } \frac{\norm{\eta}_{V}}{\norm{\eta}_{V'}}.
    \end{equation}
\end{Definition}

\begin{Bemerkung}
    Ist $V^{\mathcal J}$ endlichdimensional, dann gilt insbesondere $\mathrm{CFL}_{\mathcal N} < \infty$.
\end{Bemerkung}

Weiter werden wir noch eine gewisse inf"=sup"=Konstante benötigen, welche in \cite[57]{Andreev:2012ep} aus der Konstruktion einer äquivalenten Norm auf $\mathcal X$ resultiert, die dann für den Stabilitätsnachweis \cite[Theorem 5.2.6]{Andreev:2012ep} verwendet wird.

An dieser Stelle benötigen wir die Operatorfamilie $\Set{A(t)}_{t \in I}$ aus \cref{eq:operator_zeit}.
Ohne Einschränkung nehmen wir nach \cref{lemma:transformation_zu_elliptischem_operator} an, dass diese eine G\aa{}rding-Ungleichung mit $\lambda = 0$ erfüllen.
Dadurch sind diese Operatoren nach \cref{korollar:bilinearform_elliptisch} elliptisch und nach dem Satz von Lax-Milgram \cite[Section 6.2.1]{evans2010partial} stetig invertierbar.
Weiter wird die Selbstadjungiertheit aus \cref{lemma:operator_selbstadjungiert} benötigt.
Dies erlaubt es uns aufbauend auf diese Operatoren zwei Skalarprodukte respektive Normen durch
\begin{equation}
    \begin{aligned}
        \skp{v_{1}}{\tilde{v}_{1}}{+} &:= \int_{I} \skp{A(t)v_{1}(t)}{\tilde{v}_{1}(t)}{V' \times V} \diff t, \\
         \norm{v_{1}}^{2}_{+} &:= \skp{v_{1}}{v_{1}}{+},
    \end{aligned}
    \qquad \text{für } v_{1}, \tilde{v}_{1} \in \mathcal Y_{1} := L_{2}(I; V),
\end{equation}
und
\begin{equation}
    \begin{aligned}
        \skp{z}{\tilde{z}}{-} &:= \int_{I} \skp{A(t)^{-1} z}{\tilde z}{V \times V'} \diff t, \\
        \norm{z}_{-}^{2} &:= \skp{z}{z}{-},
    \end{aligned}
    \qquad \text{für } z, \tilde z \in \mathcal Y_{1}' = L_{2}(I; V'),
\end{equation}
zu definieren.
Mit Hilfe dieser Normen definieren wir nun die angesprochene inf"=sup"=Konstante nach \cite[57]{Andreev:2012ep} als
\begin{equation}
    \beta_{1}(\mathcal X^{\mathcal N}, \mathcal Y^{\mathcal N}) := \inf_{u \in \mathcal X^{\mathcal N}} \sup_{v \in \mathcal Y^{\mathcal N}} \frac{\int_{I} \skp{u_t(t)}{v_{1}(t)}{V' \times V} \diff t}{\norm{u_t}_{-} \norm{v_{1}}_{+}},
\end{equation}
wobei Infimum und Supremum bezüglich aller Elemente gebildet werden, für die der Nenner nicht Null wird.

Unter Verwendung dieser Definitionen weist \citeauthor{Andreev:2012ep} letztendlich die nachfolgende Stabilitätsaussage nach.

\begin{Satz}
    Seien $\Set{(\mathcal X^{\mathcal N}, \mathcal Y^{\mathcal N})}_{\mathcal N \geq 1}$ Diskretisierungen der Form \cref{eq:diskreter_ansatz_und_testraum_als_span}.
    Gelten die Bedingungen
    \begin{equation}
        \sup_{\mathcal N \geq 1} \mathrm{CFL}_{\mathcal N} < \infty
        \quad \text{und} \quad
        \inf_{\mathcal N \geq 1} \beta_{1}(\mathcal X^{\mathcal N}, \mathcal Y^{\mathcal N}) > 0,
    \end{equation}
    dann existiert eine Konstante $c_{0} > 0$ und es gilt für die diskrete inf"=sup"=Konstante $\beta^{\mathcal N}$ aus \cref{eq:diskrete_inf_sup_kosntante} die Abschätzung
    \begin{equation}
        \beta^{\mathcal N} \geq c_{0} \min\Set{1, \beta_{1}(\mathcal X^{\mathcal N}, \mathcal Y^{\mathcal N})} \min\Set{1, \mathrm{CFL}_{\mathcal N}^{-1}} \quad \fa \mathcal N \geq 1.
    \end{equation}

    \begin{Beweis}
        Siehe \cite[Subsection 5.2.2]{Andreev:2012ep}.
    \end{Beweis}
\end{Satz}


\section{Numerische Umsetzung und Experimente} % (fold)
\label{section:galerkin_numerische_umsetzung_und_experimente}

% \subsection{Sammlung zur numerischen Umsetzung} % (fold)
% \label{sub:sammlung_zur_numerischen_umsetzung}

% \begin{Lemma}[Berechnung der Rieszchen Darstellung]
% \label{lemma:berechnung_rieszsche_darstellung}
%     Sei $X$ ein endlichdimensionaler Hilbertraum mit Basis ${\phi_i}_{i=1}^{N}$.
%     Sei weiter $g \in X'$.
%     Der Koeffizientenvektor $\vec{v} \in \mathbb{R}^{N}$ der Rieszschen Darstellung $v_g = \sum_{i=1}^{N} v_{i} \phi_{i} \in X$ von $g$, das heißt, es gilt $\skp{v_g}{w}{X} = \skp{g}{w}{X' \times X}$ für alle $w \in X$, lässt sich durch das Gleichungssystem $\mat{K}\vec{v} = \vec{g}$ berechnen, wobei $\mat{K} = [\skp{\phi_{i}}{\phi_{j}}{X}]_{i,j}$ die Massematrix zum inneren Produkt auf $X$ sei und weiter $\vec{g} = [\skp{g}{\phi_{i}}{X' \times X}]_{i}$ sei.
% \end{Lemma}

% \begin{Lemma}[Diskrete $\mathcal{X}$-Norm]
% % todo: fixme
%     Sei $\mathcal{X}^{\mathcal{N}} = \spn\Set{ \phi_{i} \given i = 1 \dots \mathcal N } \subset \mathcal X$ ein endlichdimensionaler Unterraum.
%     Die $\mathcal X$-Norm aus \eqref{eq:gl:le:ansatzraum_X_norm}
%     wird für den Unterraum $\mathcal{X}^{\mathcal N}$ durch den folgenden Operator
%     \begin{equation}
%         \mat{M} = \mat{M}_{t} \otimes (\mat{M}_{x} + \mat{A}_{x}) + \mat{A}_{t} \otimes (\mat{M}_{x} (\mat{M}_{x} + \mat{A}_{x})^{-1} \mat{M}_{x})
%     \end{equation}
%     induziert.
% \end{Lemma}

% \begin{Lemma}[Diskrete $\mathcal{Y}$-Norm]
% % todo: fixme
%     Sei bla.
%     Die $\mathcal{Y}^{\mathcal N}$-Norm aus \eqref{eq:testraum_Y_skalarprodukt} wird durch den folgenden Operator
%     \begin{equation}
%         \mat{N} = \begin{pmatrix}
%             \mat{M}_{t} \otimes (\mat{M}_{x} + \mat{A}_{x}) & \mat{0} \\
%             \mat{0} & \mat{M}_{x}.
%         \end{pmatrix}
%     \end{equation}
%     induziert.
% \end{Lemma}

