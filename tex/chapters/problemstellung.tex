%!TEX root = ../main.tex

\chapter{Überschrift in Arbeit}

\todo[inline]{Kapitel komplett überarbeiten}

\section{Der eindimensionale Fall} % (fold)
\label{sec:der_eindimensionale_fall}

Wir beschränken uns nun zunächst auf den Fall einer Raumdimension.
Es sei also $\Omega \subset \mathbb{R}$ ein Intervall, ohne Beschränkung der Allgemeinheit wählen wir $\Omega = [0, 1]$.

Die Wahl des Funktionensystems $\Set{ \varphi_{j} }_{j}$ ist von wesentlicher Bedeutung, denn sie legt fest, welche Funktionen $\omega$ wir durch die Reihenentwicklung \eqref{eq:omega_reihenentwicklung} darstellen können.
Zudem bestimmen wir damit das Konvergenzverhalten in verschiedenen Räumen und die möglichen Randbedingungen.

% TODO: besser formulieren
Eine durch die beabsichtigte Anwendung in \cite{Stasiak:2011ba} motivierte Wahl ist die Entwicklung in Sinusfunktionen.
Da Konvergenz in $L_{\infty}(\Omega)$ wegen $\omega \in L_{\infty}(\Omega)$ notwending ist, erzwingen wir diese durch Gewichtung der Sinusfunktionen.

Zusätzlich zu den Sinusfunktionen fügen wir eine konstante Funktion hinzu, um so inhomogene Randbedingungen für $\omega$ zuzulassen, das heißt wir haben das Funktionensystem $\Set{ \varphi_{j} }_{j \in \mathbb{N}_{0}}$ mit
\begin{equation}
    \label{eq:sinusfunktionen_ansatz}
    \varphi_{0} = K, \qquad
    \varphi_{j} = \frac{K}{(\pi j)^{1 + \epsilon}} \sin(\pi j \blank), \quad j \geq 1,
\end{equation}
wobei $K \in \mathbb{R}_{+}$ als Skalierungsfaktor dient.

An dieser Stelle weisen wir zunächst einige Eigenschaften für \eqref{eq:sinusfunktionen_ansatz} nach.
\begin{Lemma}
    Es gilt
    \begin{alignat}{2}
        \norm{\varphi_{0}}_{L_{\infty}(\Omega)} &= K,
        \qquad&
        \norm{\varphi_{j}}_{L_{\infty}(\Omega)} &= \frac{K}{(\pi j)^{1 + \epsilon}}, \quad j \geq 1,
    % \end{equation}
    % sowie
    % \begin{equation}
    \intertext{sowie}
        \norm{\varphi_{0}}_{H^{1}(\Omega)}  &= K,
        \qquad&
        \norm{\varphi_{j}}_{H^{1}(\Omega)}  &= \frac{K}{\sqrt 2 } \frac{\sqrt{1 + (\pi j)^{2}}}{(\pi j)^{1 + \epsilon}}, \quad j \geq 1.
    \end{alignat}
\end{Lemma}

\begin{Lemma}
    Die Funktionen $\Set{ \varphi_{j} }_{j \in \mathbb{N}}$ bilden ein Orthogonalsystem in $H^{1}(\Omega)$, denn es gilt
    \begin{equation}
        \skprod{\varphi_{j}}{\varphi_{k}}_{H^{1}(\Omega)} = \begin{cases}
            \frac{K}{\sqrt 2 } \frac{\sqrt{1 + (\pi j)^{2}}}{(\pi j)^{1 + \epsilon}},   &j = k \\
            0,          &j \neq k.
        \end{cases}
    \end{equation}
\end{Lemma}

\begin{Lemma}
    Sei $\sigma \in \mathcal S$ und $\epsilon > 0$, dann konvergiert $\omega(\blank; \sigma)$ in $L_{\infty}(\Omega)$.
    Ist $\epsilon > 1$, dann gilt auch Konvergenz in $H^{1}_{0}(\Omega)$.

    \begin{Beweis}
        Sei zunächst $\epsilon > 0$.
        Da $\mathcal S = [0, 1]^{\mathbb{N}}$ ist, erhalten wir die Konvergenz in $L_{\infty}(\Omega)$ nach dem Weierstraßschen Majorantenkriterium via
        \begin{align}
            \sum_{j = 0}^{\infty} \norm{\sigma_{j} \varphi_{j}}_{L_{\infty}(\Omega)}
            &= \sum_{j = 0}^{\infty} \abs{\sigma_{j}} \norm{\varphi_{j}}_{L_{\infty}(\Omega)}
             \leq \norm{\varphi_{0}}_{L_{\infty}(\Omega)} + \sum_{j = 1}^{\infty}  \norm{\varphi_{j}}_{L_{\infty}(\Omega)}
            \\&= K + \frac{K}{\pi^{1+\epsilon}} \sum_{j = 1}^{\infty} \frac{1}{j^{1+\epsilon}}.
        \end{align}
        Diese Reihe konvergiert bekanntlich für alle $\epsilon > 0$, womit wir Konvergenz von $\omega$ in $L_{\infty}(\Omega)$ erhalten.

        Sei nun $\epsilon > 1$.
        Betrachte
        \begin{align}
            \sum_{j = 0}^{\infty} \norm{\sigma_{j} \varphi_{j}}_{H^{1}(\Omega)}
            &= \sum_{j = 0}^{\infty} \abs{\sigma_{j}} \norm{\varphi_{j}}_{H^{1}(\Omega)}
            \leq  \norm{\varphi_{0}}_{H^{1}(\Omega)} + \sum_{j \geq 1} \norm{\varphi_{j}}_{H^{1}(\Omega)}
            \\&= K + \frac{K}{\sqrt{2}} \sum_{j = 1}^{\infty} \frac{1 + \pi j}{(\pi j)^{1+\epsilon}}
            \\&\leq K + \frac{K}{\sqrt{2} \pi^{1 + \epsilon}} \sum_{j = 1}^{\infty} \frac{1}{j^{1+\epsilon}} + \frac{K}{\sqrt{2} \pi^{\epsilon}} \sum_{j = 1}^{\infty} \frac{1}{j^{\epsilon}}.
        \end{align}
        Wegen $\epsilon > 1$ konvergiert sowohl die erste als auch die zweite Reihe.
        Zusammen liefert dies die Konvergenz in $H^{1}_{0}(\Omega)$.
    \end{Beweis}
\end{Lemma}

Wir wollen nun die Regularität von $\omega(\blank; \sigma)$ in Abhängigkeit vom Parameter $\sigma$ nachweisen.

\begin{Satz}
\label{satz:regularitaet_nachrechnen}
    Seien $\epsilon > 0$ und $0 < \kappa < 1$ so gewählt, dass
    \begin{equation}
        \zeta(2 + \epsilon) \leq (\kappa c \pi^{2} - K) \frac{\pi^{2+\epsilon}}{4K C_{\infty}}
    \end{equation}
    gilt,
    wobei $\zeta$ die Riemannsche-$\zeta$-Funktion, $c$ und $K$ die Konstanten aus \eqref{eq:def_op_A} respektive \eqref{eq:sinusfunktionen_ansatz} und $C_{\infty}$ die Einbettungskonstante von $H^{1}_{0}(\Omega) \hookrightarrow L_{\infty}(\Omega)$ sind.
    Dann erfüllt $\Set{\hat A} \cup \Set{ A_{j} }_{j \geq 0}$ \thref{thm:kunoth:assumption2}.

    \begin{Beweis}
        Wir weisen zunächst die inf-sup-Bedingungen \eqref{eq:kunoth:ass2_gamma_0} für $\hat a(\blank, \blank)$ nach und bestimmen die Konstante $\gamma_{0}$.
        Da $\hat a(\blank, \blank)$ symmetrisch ist, genügt es die inf-sup-Bedingung \eqref{eq:kunoth:ass2_gamma_0_a} nachzuweisen. Die zweite inf-sup-Bedingung \eqref{eq:kunoth:ass2_gamma_0_b} folgt dann analog mit dem selben $\gamma_{0}$.

        Nach \thref{lemma:sauter:2.1.48} reicht es, für alle $u \in H^{1}_{0}(\Omega)$ ein $v_{u} \in H^{1}_{0}(\Omega)$ und von $u$ und $v_{u}$ unabhängige Konstanten $C_{1}, C_{2} > 0$ mit
        \begin{equation}
            \hat a(u, v_{u}) \geq C_{1} \norm{u}_{H^{1}(\Omega)}^{2} \quad \text{und} \quad \norm{v_{u}}_{H^{1}(\Omega)} \leq C_{2} \norm{u}_{H^{1}(\Omega)}
        \end{equation}
        zu finden.
        Dann ist die inf-sup-Bedingung \eqref{eq:kunoth:ass2_gamma_0_a} mit $\gamma_{0} = \frac{C_{1}}{C_{2}}$ erfüllt.

        Sei nun also $u \in H^{1}_{0}(\Omega)$ beliebig.
        Wir wählen $v_{u} = u \in V$, es gilt also $C_{2} = 1$.
        Es ergibt sich
        \begin{align}
            \hat a(u, v_{u}) = \hat a(u, u) = c \skprod{\grad u}{\grad u}_{L_{2}(\Omega)} = c \norm{\grad u}_{L_{2}(\Omega)}^{2} \geq c \gamma_{\Omega}^{2} \norm{u}_{H^{1}(\Omega)}^{2},
        \end{align}
        wobei die letzte Abschätzung aus der Poincaré-Friedrichs-Ungleichung \eqref{eq:grundlagen:poincare_friedrichs_ungleichung} folgt.
        Zusammen liefert dies $\gamma_{0} = c \gamma_{\Omega}^{2}$ als inf-sup-Konstante.

        Für den vorliegenden Fall können wir $\gamma_{\Omega}^{2}$ exakt bestimmen.
        Nach \cite[Chapter 11]{Strauss:2007vz} entspricht das Quadrat der optimalen Poincaré-Friedrichs-Konstante $\gamma_{\Omega}^{2}$ gerade dem kleinsten Eigenwert des Laplace-Operators auf $\Omega$ mit Dirichlet-Randbedingung.
        Dieser hat für $\Omega = [0, 1]$ den Wert $\pi^{2}$.
        Wir erhalten damit also $\gamma_{0} = c \pi^{2}$.

        Seien $u, v \in H^{1}_{0}(\Omega)$.
        Für $j = 0$ gilt die simple Abschätzung
        \begin{equation}
            \begin{aligned}
                a_{0}(u, v)
                &= \skprod{\varphi_{0} u}{v}_{L_{2}(\Omega)}
                = K \skprod{u}{v}_{L_{2}(\Omega)}
                \\&\leq K \norm{u}_{L_{2}(\Omega)} \norm{v}_{L_{2}(\Omega)}
                \leq K \norm{u}_{H^{1}(\Omega)} \norm{v}_{H^{1}(\Omega)}
            \end{aligned}
        \end{equation}
        Betrachte für $j \geq 1$
        \begin{align}
            a_{j}(u, v)
            &= \skprod{\varphi_{j} u}{v}_{L_{2}(\Omega)}
            = \int_{I} \varphi_{j} u v \diff x
            \intertext{da $\varphi_{j}$ integrierbar ist und $\varphi_{j}(0) = 0$, können wir dies umschreiben zu}
            a_{j}(u, v)
            &= \int_{I} \frac{\diff}{\diff x} \left( \int_{0}^{x} \varphi_{j}(y) \diff y \right) u v \diff x
            \intertext{woraus wir mittels partieller Integration folgenden Ausdruck erhalten}
            a_{j}(u, v)
            &= - \int_{I} \left( \int_{0}^{x} \varphi_{j}(y) \diff y \right) (u v)' \diff x
            \leq \norm*{\left( \int_{0}^{x} \varphi_{j}(y) \diff y \right) (u v)'}_{L_{1}(\Omega)}.
            \\&\leq \norm*{\int_{0}^{x} \varphi_{j}(y) \diff y }_{L_{\infty}(\Omega)} \norm{(uv)'}_{L_{1}(\Omega)}
        \end{align}
        Die erste Norm können wir weiter abschätzen mit
        \begin{equation}
            \begin{aligned}
                \norm*{\int_{0}^{x} \varphi_{j}(y) \diff y }_{L_{\infty}(\Omega)}
                &= \frac{K}{(\pi j)^{1+\epsilon}} \norm*{\int_{0}^{x} \sin(\pi j y) \diff y }_{L_{\infty}(\Omega)}
                \\&= \frac{K}{(\pi j)^{2+\epsilon}} \norm{ \cos(\pi j x) - 1 }_{L_{\infty}(\Omega)}
                \leq \frac{2 K}{(\pi j)^{2+\epsilon}}.
            \end{aligned}
        \end{equation}
        Aus der zweiten Norm erhalten wir mittels Minkowski- und Hölderungleichung sowie der Einbettung $H^{1}_{0}(\Omega) \hookrightarrow L_{\infty}(\Omega)$ die Abschätzung
        \begin{equation}
            \begin{aligned}
                \norm{(uv)'}_{L_{1}(\Omega)}
                &= \norm{u'v + uv'}_{L_{1}(\Omega)}
                \leq \norm{u'v}_{L_{1}(\Omega)} + \norm{uv'}_{L_{1}(\Omega)}
                \\&\leq \norm{u'}_{L_{1}(\Omega)} \norm{v}_{L_{\infty}(\Omega)} + \norm{u}_{L_{\infty}(\Omega)} \norm{v'}_{L_{1}(\Omega)}
                \\&\leq \norm{1}_{L_{2}(\Omega)} \norm{u'}_{L_{2}(\Omega)} \norm{v}_{L_{\infty}(\Omega)} + \norm{u}_{L_{\infty}(\Omega)} \norm{1}_{L_{2}(\Omega)} \norm{v'}_{L_{2}(\Omega)}
                \\&\leq C_{\infty} \norm{u'}_{L_{2}(\Omega)} \norm{v}_{H^{1}(\Omega)} + C_{\infty} \norm{u}_{H^{1}(\Omega)} \norm{v'}_{L_{2}(\Omega)}
                % \\&\leq \norm{u'}_{L_{2}(\Omega)} \norm{v}_{L_{2}(\Omega)} + \norm{u}_{L_{2}(\Omega)} \norm{v'}_{L_{2}(\Omega)}
                \\&\leq 2 C_{\infty} \norm{u}_{H^{1}(\Omega)} \norm{v}_{H^{1}(\Omega)}
            \end{aligned}
        \end{equation}
        Zusammen also
        \begin{equation}
            a_{j}(u, v)
            \leq \norm*{\int_{0}^{x} \varphi_{j}(y) \diff y }_{L_{\infty}(\Omega)} \norm{(uv)'}_{L_{1}(\Omega)}
            \leq \frac{4 K C_{\infty}}{(\pi j)^{2 + \epsilon}} \norm{u}_{H^{1}(\Omega)} \norm{v}_{H^{1}(\Omega)}.
        \end{equation}
        Betrachte nun
        \begin{align}
                    \sum_{j = 0}^{\infty} \norm{A_{j}}_{\mathcal L(V, V')}
            &= \norm{A_{0}}_{\mathcal L(V, V')} + \sum_{j = 1}^{\infty} \norm{A_{j}}_{\mathcal L(V, V')}
            \leq K + \sum_{j = 1}^{\infty} \frac{4 K  C_{\infty}}{(\pi j)^{2 + \epsilon}}
            \\&\leq K + \frac{4 K  C_{\infty}}{\pi^{2 + \epsilon}} \sum_{j = 1}^{\infty} \frac{1}{j^{2 + \epsilon}}
            = K + \frac{4 K  C_{\infty}}{\pi^{2 + \epsilon}} \zeta(2 + \epsilon)
            % \leq    \sum_{j = 0}^{\infty} \norm{\varphi_{j}}_{L_{\infty}(\Omega)}
            % =       a_{0} + \frac{1}{\pi^{1+\epsilon}} \sum_{j = 1}^{\infty} \frac{1}{j^{1+\epsilon}}
            % = a_{0} + \frac{\zeta(1+\epsilon)}{\pi^{1+\epsilon}}.
        \end{align}
        Für die Gültigkeit von
        \begin{equation}
            \sum_{j \geq 0} \norm{A_{j}}_{\mathcal L(V, V')} \leq \kappa \gamma_{0}
        \end{equation}
        für ein $0 < \kappa < 1$ ist damit also
        \begin{equation}
            \zeta(2 + \epsilon) \leq (\kappa c \pi^{2} - K) \frac{\pi^{2+\epsilon}}{4K  C_{\infty}}
        \end{equation}
        mit $\epsilon > 0$ hinreichend.
    \end{Beweis}
\end{Satz}

Zusammenfassend erhalten wir für diese Wahl des Funktionensystems $\Set{ \varphi_{j} }_{j}$ die Aussage

\begin{Korollar}
    Unter den Voraussetzungen von \thref{satz:regularitaet_nachrechnen} gilt \thref{thm:kunoth:theorem21}.
\end{Korollar}

% subsection nachrechnen_von_thref_thm_kunoth_assumption2 (end)

% section der_eindimensionale_fall (end)

\clearpage
\section{Zu klärende Fragen} % (fold)
\label{sub:zu_kl_rende_fragen}

\begin{enumerate}
    \item Wohldefiniertheit der PDE \eqref{eq:parabolische_pde}, das heißt die Voraussetzungen von \thref{thm:schwab09:theorem51} nachweisen. Weiterhin lassen sich damit die inf-sup-Bedingung von \eqref{eq:varprob} nachrechnen und damit die Schranken für $B$ und $B^{-1}$ bestimmen.
    \item Parametrische Variante des Variationsproblems herleiten.
    Dazu Ansetzen mit Entwicklung des Parameters $\omega$ in eine Reihe
    \begin{equation}
        \omega = \sum_{j = 0}^{\infty} \sigma_{j} \varphi_{j}.
    \end{equation}
    Dabei ergeben sich folgende Fragen:
    \begin{enumerate}
        \item Konvergenz der Reihe? Notwendig ist Konvergenz in $L_{\infty}(\Omega)$, da die Norm $\norm{\omega}_{L_{\infty}(\Omega)}$ mehrfach in Abschätzungen verwendet wird.
        \item Weiterhin ist eventuell Konvergenz in einem Unterraum $Z \hookrightarrow L_{\infty}(\Omega)$ wünschenswert.
        Zum Beispiel in $H^{1}(\Omega)$?
        \item Welche Bedingungen ergeben sich an $\sigma_{j}$ und $\varphi_{j}$?
        \item Welches Funktionensystem $\Set{ \varphi_{j} }_{j}$ ist überhaupt sinnvoll?
        Die Wahl der $\varphi_{j}$ entscheidet maßgeblich über Konvergenz der Reihenentwicklung.
        Welche Randvorgaben sind angestrebt?
        Dies wird ebenfalls durch die $\varphi_{j}$ geregelt.
    \end{enumerate}
    \item Welche affine Zerlegung $A(\sigma) = A_{0} + \sum_{j} \sigma_{j} A_{j}$ ist brauchbar?
    Wie genau sehen die $A_{j}$ aus?
    \item Nachweisen, dass $A(\sigma)$ \thref{thm:kunoth:assumption1} oder \thref{thm:kunoth:assumption2} erfüllt und mittels \thref{thm:kunoth:theorem21} die gewünschte Regularität von $B(\sigma)$ bezüglich $\sigma$ gewinnen.
    \item Die Abschätzungen in \thref{satz:regularitaet_nachrechnen} lassen sich noch deutlich verbessern.
    Das gilt wahrscheinlich auch für andere Abschätzungen!
\end{enumerate}

