%!TEX root = ../main.tex

\chapter*{Einleitung} % (fold)
\addcontentsline{toc}{chapter}{Einleitung}
\label{cha:einleitung}

Lineare Evolutiongsgleichungen treten in vielen Bereichen der Mathematik und der Naturwissenschaften auf.
Zu den prominentesten Vertreten gehören sicherlich die Wärmeleitungsgleichung, die Wellengleichung und viele lineare Reaktionsdiffusionsgleichungen.
Auch in der Polymerchemie lassen sie sich antreffen, so zum Beispiel bei der Modellierung von Copolymer-Melts [zitieren, Helfand oder Stasiak], wo auch die Motivation für diese Arbeit herrührt.

Konkret stammt das vorliegende Problem aus dem Bereich der \emph{Self Consistent Field Theory}.
Diese wird zum Beispiel verwendet um in obengenannten Copolymer-Melts Gleichgewichte, oder auch Konfigurationen, durch Minimierung eines Funktionals zu finden.
Dazu werden Iterationsverfahren verwendet, die darauf zurückgreifen in jedem Schritt eine Reaktionsdiffusionsgleichung zu lösen, wobei der Reaktionsterm durch das Iterationsverfahren in jedem Schritt variiert wird.
Dies motiviert dazu, die Reaktionsdiffusionsgleichung als parametrische partielle Differentialgleichung aufzufassen und bereits bekannte Methoden zur Dimensionsreduktion zu verwenden.

Eines dieser Verfahren ist die Reduzierte-Basis-Methode.
Ziel dieser ist es, die Eigenschaft, dass die Abhängigkeit der Lösung der parametrischen partiellen Differentialgleichung eine gewisse Regularität bezüglich des Parameters aufweist, auszunutzen.
Die Reduzierte-Basis-Methode beruht, wie zum Beispiel auch die Finite-Elemente-Methode, auf der Galerkin-Projektion auf einen endlichdimensionalen Unterraum.
Anders als bei der Finite-Elemente-Methode werden die Unterräume aber von Lösungen der parametrischen partiellen Differentialgleichung zu gewissen, clever gewählten, Parametern aufgespannt.
Zur Wahl dieser Parameter gibt es verschiedene Ansätze, zum Beispiel \emph{Greedy-Verfahren}, oder auch die \emph{Proper Orthogonal Decomposition}.
Für die meisten dieser Ansätze werden \emph{a priori}-Fehlerschätzer benötigt.

\section{Problemherkunft} % (fold)
\label{sec:problemherkunft}

Dieser Abschnitt dient als Einführung in das zugrundeliegende Problem und die physikalischen und chemischen Motivationen und Modelle dahinter.
Als Quellen dienen vor allem die ausführlichen Arbeiten von \textcite{Fredrickson:2006th} und \textcite{Matsen:2002do,Matsen:2006ud}, wobei diese eine leicht andere Herangehensweise verfolgt.
Es folgt eine einfach gehaltene und dadurch formal nicht all zu exakte Einführung in die physikalischen Hintergründe.

\subsection{Begriffsklärung} % (fold)
\label{sub:begriffskl_rung}

% subsection begriffskl_rung (end)
% A polymer is any molecule constructed by linking together chemical units or monomers to form a long-chain molecule.

Als Polymer, oder auch Makromolekül, bezeichnet man ein Molekül, welches sich aus vielen kleineren Molekülen, sogenannten Monomeren, zusammensetzt.
Typischerweise sind Polymere lange lineare Ketten, die durch aneinanderreihen von Monomoren entstehen.
Besteht das Polymer aus einer einzigen Monomer-Gattung, dann sprechen wir von einem Homopolymer, sonst von einem Copolymer.
Bilden die verschiedenen Monomer-Gattung im Falle eines Copolymers größere, homogene zusammenhängende Gruppen, welche wiederum durch Aneinanderreihen das Polymer bilden, dann sprechen wir von einem Blockcopolymer.


Es existieren unüberschaubar viele Konfigurationen solcher Blockcopolymere, weswegen man das Studium dieser anhand vergleichsweise simpler Anordnungen, das heißt, von AB-Diblockcopolymeren, beginnt.
Diese werden, wie der Name schon vermuten lässt, durch längere homogene Bläcke zweier verschiedener Monomere, kurz A und B genannt, gebildet.

\subsection{Modell} % (fold)
\label{sub:modell}

% subsection modell (end)

\subsection{Self-Consistent-Field-Theory} % (fold)
\label{sub:self_consistent_field_theory}

% subsection self_consistent_field_theory (end)

\subsection{Mean-Field-Theory} % (fold)
\label{sub:mean_field_theory}

% subsection mean_field_theory (end)

% section problemherkunft (end)

\section{Motivation} % (fold)
\label{sec:einf_hrung_der_ppde}

Dieser Abschnitt führt die PPDE ein und leitet eine parametrische Variante dieser her.
Eine umfangreiche Beschreibung der physikalischen und chemischen Hintergründe, sowie eine ausführliche Herleitung der darauf aufbauenden mathematischen Modellierung findet sich bei \textcite{Fredrickson:2006th}.

Wir beschränken uns auf die daraus resultierende parabolische partielle Differentialgleichung, da diese den Mittelpunkt dieser Arbeit bildet.

Seien dazu $0 < T < \infty$ und $I = [0, T]$ ein endliches Zeitintervall und weiter $\Omega \subset \mathbb{R}^{n}$ eine offene, beschränkte Teilmenge mit Lipschitz-Rand.
In den tatsächlich auftretenden Fällen wird meist $T = 1$ und $\Omega = [0, L]^n$ für ein $0 < L < \infty$ und $n \in \Set{1, 2, 3}$ gelten, wir wollen dies aber zunächst ignorieren, da die folgenden Aussagen auch für den allgemeinen Fall gelten.

Gegeben seien weiter $\omega_{1}, \omega_{2} \colon \Omega \to \mathbb{R}$ zwei $L_{\infty}(\Omega)$-Abbildungen und ein $f \in (0, T)$.
Wir definieren damit
\begin{equation}
    \omega \colon I \times \Omega \to \mathbb{R}, \quad (t, x) \mapsto
    \begin{cases}
        \omega_{1}(x), & t \leq f \\
        \omega_{2}(x), & t > f.
    \end{cases}
\end{equation}

Wir betrachten nun die folgende parabolische partielle Differentialgleichung.
\begin{equation}
    u_{t}(t, x) = c \Delta_{x} u(t, x) - w(t, x) u(t, x) \qquad \text{auf}~I \times \Omega,
\end{equation}
wobei $c \in \mathbb{R}$ eine Konstante ist, und es seien weiter Anfangs- und Randwertbedingungen gegeben.
Diese sind im Falle der bei \textcite{Stasiak:2011ba} betrachteten Variante zum Beispiel periodische Randbedingungen in $\partial \Omega$ mit der Anfangsbedingung $u(0, \blank) = 1$.
Wir beschränken uns bei dieser Arbeit auf den Fall homogener Randbedingungen und
dazu kompatiblen Anfangsbedingungen.

\todo[inline]{Die Einleitung ist mies...}

% section einf_hrung_der_ppde (end)


% chapter einleitung (end)
