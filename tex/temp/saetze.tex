%!TEX root = ../main.tex

\section{Sätze} % (fold)
\label{sec:saetze}

TODO: Deutlich besser formulieren, fehlendes ergänzen!

\begin{Satz}[Babu{\v{s}}ka-Aziz, \cite{Aziz:2014wf}]
    \label{satz:babuska-aziz}
    Seien $H_{1}$ und $H_{2}$ zwei reelle Hilberträume mit Skalarprodukten $\skprod{\cdot}{\cdot}_{H_{1}}$ respektive $\skprod{\cdot}{\cdot}_{H_{2}}$.
    Sei $B \colon H_{1} \to H_{2}'$ ein linearer Operator und
    sei $b \colon H_{1} \times H_{2} \to \mathbb{R}$, $(u, v) \mapsto B(u, v)$ die zugehörige Bilinearform, das heißt $(Bu)(v) = b(u, v)$.
    Erfüllt $b$ die sogenannten Babu{\v{s}}ka-Aziz-Bedingungen
    \begin{thmenumerate}
        \item \emph{Stetigkeit:} Es existiert ein $C < \infty$ mit
        \begin{equation}
            \abs{B(u, v)} \leq C \norm{u}_{H_{1}} \norm{v}_{H_{2}}, \qquad \text{für alle}~u \in H_{1},~v \in H_{2}
        \end{equation}
        \item \emph{$\inf$-$\sup$-Bedingung:} Es existiert ein $\beta > 0$ mit
        \begin{equation}
            \inf_{u \in H_{1}} \sup_{v \in H_{2}} \frac{\abs{B(u, v)}}{\norm{u}_{H_{1}} \norm{v}_{H_{2}}} \geq \beta.
        \end{equation}
        \item \emph{Surjektivität:} Es gilt
        \begin{equation}
            \sup_{u \in H_{1}} \abs{B(u, v)} > 0, \qquad \text{für alle}~v \neq 0.
        \end{equation}
    \end{thmenumerate}
    Dann ist $B$ ein Isomorphismus.
\end{Satz}

\begin{Satz}[vgl. {{\cite[Theorem 5.1]{Schwab:2009ec}}}]
    Der Operator $B \in \mathcal L(\mathcal X, \mathcal Y')$ sei gegeben durch $(Bw)(v) = b(w, v)$ mit $b(\cdot, \cdot)$, $\mathcal X$ und $\mathcal Y$ wie in (...).
    Dann ist $B$ ein Isomorphismus.

    % \begin{Beweis}
    %     Wir weisen die Bedingungen von \thref{satz:babuska-aziz} nach.

    %     Zunächst sei anzumerken, dass wir in \eqref{eq:garding-inequality} ohne Einschränkung $\lambda = 0$ wählen können.
    %     Wähle
    %     \begin{equation}
    %         u(t) = \hat u(t) e^{\lambda t}, \quad v_{1}(t) = \hat v_{1}(t) e^{- \lambda t}, \quad g(t) = \hat g(t) e^{\lambda t},
    %     \end{equation}
    %     dann sieht man, dass $u$ die Gleichung \eqref{eq:bilinearform} genau dann löst, wenn $\hat u$ die Gleichung
    %     \begin{equation}
    %         \label{eq:bilinearform_tmp}
    %         \begin{gathered}
    %             \int_{I} \skprod{\hat{u}_{t}(t)}{\hat{v}_{1}(t)}_{H} + \lambda \skprod{\hat{u}(t)}{\hat{v}_{1}(t)}_{H} + a(t; \hat{u}(t), \hat{v}_{1}(t)) \diff t + \skprod{\hat{u}(0)}{v_{2}}_{H}
    %                 \\= \int_{I} \skprod{\hat{g}(t)}{\hat{v}_{1}(t)}_{H} \diff t + \skprod{u_{0}}{v_{2}}_{H}
    %         \end{gathered}
    %     \end{equation}
    %     für alle $\hat{v} = (\hat{v}_{1}, v) \in \mathcal Y$ löst.

    %     \paragraph{Stetigkeit} % (fold)
    %     \label{par:stetigkeit}
    %     Betrachte für $u \in \mathcal X$ und $v = (v_{1}, v_{2}) \in \mathcal Y$ die Bilinearform $b(u, v)$.
    %     Nach Anwenden der Dreiecksungleichung erhalten wir
    %     \begin{equation}
    %         \label{eq:stetigkeit_zweiter_term}
    %         \abs{b(u, v)} = \int_{I} \abs{\skprod{u_{t}(t)}{v_{1}(t)}_{H}} + \abs{a(u(t), v_{1}(t))} \diff t + \abs{\skprod{u(0)}{v_{2}}_{H}}.
    %     \end{equation}
    %     Betrachten wir zunächst den hinteren Term, dann erhalten wir unter Verwendung der Cauchy-Schwarz-Ungleichung und der Einbettungs-Konstante $M_{e}$ die Abschätzung
    %     \begin{equation}
    %         \abs{\skprod{u(0)}{v_{2}}_{H}} \leq \norm{u(0)}_{H} \norm{v_{2}}_{H} \leq M_{e} \norm{u}_{X} \norm{v_{2}}_{H}.
    %     \end{equation}
    %     Widmen wir uns nun dem ersten Term und wenden ebenfalls die Cauchy-Schwarz-Ungleichung sowie die Stetigkeit von $a$ an, dann erhalten wir
    %     \begin{align}
    %         &\int_{I} \abs{\skprod{u_{t}(t)}{v_{1}(t)}_{H}} + \abs{a(u(t), v_{1}(t))} \diff t
    %         \\&\qquad
    %         \leq \int_{I} \norm{u_{t}(t)}_{H} \norm{v_{1}(t)}_{H} + M_{a} \norm{u(t)}_{H} \norm{v_{1}(t)}_{H} \diff t
    %         \\&\qquad
    %         \leq \int_{I} \max\{1, M_{a}\} \norm{v_{1}(t)}_{H} \left(  \norm{u_{t}(t)}_{H} + \norm{u(t)}_{H} \right) \diff t
    %         \intertext{mittels Hölder-Ungleichung lässt sich dies weiter abschätzen zu}
    %         &\qquad
    %         \leq \left( \int_{I} \max\{1, M_{a}\}^{2} \norm{v_{1}(t)}_{H}^{2} \diff t \right)^{\frac 12} \left( \int_{I} \left( \norm{u_{t}(t)}_{H} + \norm{u(t)}_{H} \right)^{2} \diff t \right)^{\frac 12},
    %         \intertext{und unter Verwendung der Youngschen-Ungleichung zu}
    %         &\qquad
    %         \leq \left( \int_{I} \max\{1, M_{a}\}^{2} \norm{v_{1}(t)}_{H}^{2} \diff t \right)^{\frac 12} \left( \int_{I} 2 \left( \norm{u_{t}(t)}_{H}^{2} + \norm{u(t)}_{H}^{2} \right) \diff t \right)^{\frac 12}
    %         \intertext{was nach Definition der verwendeten Normen auch geschrieben werden kann als}
    %         &\qquad
    %         = \sqrt{2 \max\{1, M_{a}^{2}\}} \norm{u}_{\mathcal X} \norm{v_{1}}_{L_{2}(I; V)}
    %     \end{align}
    %     Zusammen mit \eqref{eq:stetigkeit_zweiter_term} liefert dies nach einer erneuten Anwendung der Cauchy-Schwarz-Ungleichung
    %     \begin{align}
    %     \abs{b(u, v)}
    %     &\leq \sqrt{2 \max\{1, M_{a}\}^{2}} \norm{u}_{\mathcal X} \norm{v_{1}}_{L_{2}(I; V)} + M_{e} \norm{u}_{X} \norm{v_{2}}_{H}
    %     \\
    %     &\leq \norm{u}_{\mathcal X} \left( \norm{v_{1}}_{L_{2}(I; V)}^{2} + \norm{v_{2}}_{H}^{2} \right)^{\frac 12} \left( 2 \max\{1, M_{a}\}^{2} + M_{e}^{2} \right)^{\frac 12}
    %     \\
    %     &= \sqrt{2 \max\{1, M_{a}^{2}\} + M_{e}^{2}} \norm{u}_{\mathcal X} \norm{v}_{\mathcal Y}.
    %     \end{align}
    %     Damit folgt die Stetigkeit.
    %     % paragraph stetigkeit (end)

    %     \paragraph{Inf-Sup-Bedingung} % (fold)
    %     \label{par:inf_sup_bedingung}

    %     % paragraph inf_sup_bedingung (end)
    % \end{Beweis}
\end{Satz}

\subsubsection{Regularität bezüglich Parameter} % (fold)
\label{ssub:aus_cite_kunoth_2013ef}

Benötigte Aussagen aus \cite{Kunoth:2013ef} um analytische Abhängigkeit der Lösung von den Parametern zu erhalten.

\begin{Annahme}[{{\cite[Assumption 1]{Kunoth:2013ef}}}]
    \label{thm:kunoth:assumption1}
    Die parametrische Operatorfamilie
    $\{ A(\sigma) \in \mathcal L(X, Y') : \sigma \in \mathcal S\}$ sei eine reguläre $\mathfrak p$-analytische Operatorfamilie für ein $0 < \mathfrak p \leq 1$, das heißt
    \begin{thmenumerate}
        \item $A(\sigma) \in \mathcal L(X, Y')$ ist stetig invertierbar für alle $\sigma \in \mathcal S$ mit gleichmäßig beschränkten Inversen $A(\sigma)^{-1} \in \mathcal L(Y', X)$, das heißt es existiert ein $C_{0} > 0$, so dass
        \begin{equation}
            \sup_{\sigma \in \mathcal S} \norm{A(\sigma)^{-1}}_{\mathcal L(Y', X)} \leq C_{0},
        \end{equation}
        \item für jedes feste $\sigma \in \mathcal S$ sind die Operatoren $A(\sigma)$ analytisch bezüglich $\sigma$. Konkret existiert eine nichtnegative Folge $b = (b_{j})_{j \geq 1} \in \ell^{\mathfrak p}(\mathbb{N})$ so dass
        \begin{equation}
            \sup_{\sigma \in \mathcal S} \norm{(A(0)^{-1})(\partial^{\nu}_{\sigma} A(\sigma))}_{\mathcal L(X, X)} \leq C_{0} b^{\nu}
        \end{equation}
        für alle $\nu \in \mathfrak F \setminus \{ 0 \}$ gilt.
        Dabei sei $\partial^{\nu}_{\sigma} A(\sigma) = \frac{\partial^{\nu_{1}}}{\partial \sigma_{1}} \frac{\partial^{\nu_{2}}}{\partial \sigma_{2}} \cdots A(\sigma)$, wobei $b^{\nu}$ für das (endliche, da $\nu \in \mathfrak F$) Produkt $b_{1}^{\nu_{1}} b_{2}^{\nu_{2}} \cdots$ steht; mit der Konvention $0^{0} = 1$.
    \end{thmenumerate}
\end{Annahme}

\begin{Annahme}[{{\cite[Assumption 2]{Kunoth:2013ef}}}]
    \label{thm:kunoth:assumption2}
    Der parametrische Operator sei affin abhänging von $\sigma$, das heißt es existiert eine Familie von Operatoren $\{A_{j}\}_{j \geq 0} \subset \mathcal L(X, Y')$ so dass
    \begin{equation}
        A(\sigma) = A_{0} + \sum_{j \geq 1} \sigma_{j} A_{j} \qquad\text{für alle}~\sigma \in \mathcal S
    \end{equation}
    gilt.
    Die Operatorfamilie $\{ A_{j} \}_{j \geq 0}$ erfülle folgende Eigenschaften:
    \begin{thmenumerate}
        \item Der \emph{mean field}-Operator $A_{0} \in \mathcal L(X, Y')$ sei stetig invertierbar, das heißt es existiert ein $\gamma_{0} > 0$ mit
        \begin{equation}
            \inf_{0 \neq u \in X} \sup_{0 \neq v \in Y} \frac{a_{0}(u, v)}{\norm{u}_{X} \norm{v}_{Y}} \geq \gamma_{0}
        \end{equation}
        und
        \begin{equation}
            \inf_{0 \neq v \in Y} \sup_{0 \neq u \in X} \frac{a_{0}(u, v)}{\norm{u}_{X} \norm{v}_{Y}} \geq \gamma_{0}.
        \end{equation}
        \item Die \emph{fluctuation}-Operatoren $\{ A_{j} \}_{j \geq 1}$ seien \emph{klein} verglichen mit $A_{0}$ im folgenden Sinne: es existiert eine Konstante $0 < \kappa < 1$ so dass
        \begin{equation}
            \sum_{j \geq 1} \norm{A_{j}} \leq \kappa \gamma^{0}
        \end{equation}
        gilt.
    \end{thmenumerate}
\end{Annahme}

\begin{Korollar}[{{\cite[Corollary 3]{Kunoth:2013ef}}}]
    \label{thm:kunoth:corollary3}
    Die affin parametrische Operatorfamilie $\{ A_{j} \}_{j \geq 0}$ erfülle \thref{thm:kunoth:assumption2}, dann wird auch \thref{thm:kunoth:assumption1} mit $\mathfrak p = 1$ und
    \begin{equation}
        C_{0} = \frac{1}{(1 - \kappa) \gamma_{0}}, \qquad b_{j} = \frac{\norm{G_{j}}}{(1 - \kappa) \gamma_{0}}, \quad \text{für alle}~j \geq 1,
    \end{equation}
    erfüllt.
\end{Korollar}

\begin{Satz}[{{\cite[Theorem 4]{Kunoth:2013ef}}}]
    \label{thm:kunoth:theorem4}
    Die parametrische Operatorfamilie $\{ A(\sigma) \in \mathcal L(X, Y') : \sigma \in \mathcal S \}$ erfülle \thref{thm:kunoth:assumption1} für ein $0 < \mathfrak p \leq 1$.
    Dann existiert für jedes $f \in Y'$ und jedes $\sigma \in \mathcal S$ eine eindeutige Lösung $u(\sigma) \in X$ der parametrischen Operatorgleichung
    \begin{equation}
        A(\sigma) u(\sigma) = f \quad \text{in}~Y'.
    \end{equation}
    Die parametrische Familie von Lösungen $u(\sigma)$ hängt analytisch von den Parametern $\sigma$ ab und die partiellen Ableitungen von $u(\sigma)$ erfüllen
    \begin{equation}
        \sup_{\sigma \in \mathcal S} \norm{(\partial^{\nu}_{\sigma} u)(\sigma)}_{X} \leq C_{0} \norm{f}_{Y'} \abs{\nu}! \tilde{b}^{\nu}
    \end{equation}
    für alle $\nu \in \mathfrak F$, wobei $0! = 1$ und die Folge $\tilde{b} = (\tilde{b}_{j})_{j \geq 1} \in \ell^{\mathfrak p}(\mathbb{N})$ definiert ist durch
    \begin{equation}
        \tilde{b}_{j} = \frac{b_{j}}{\ln 2} \qquad \text{für alle j} \in \mathbb{N}.
    \end{equation}
\end{Satz}

\begin{Satz}[{{\cite[Theorem 21]{Kunoth:2013ef}}}]
    \label{thm:kunoth:theorem21}
    Für jedes $\sigma \in \mathcal S$ ist die parametrische Evolutionsgleichung $B(\sigma) \in \mathcal L(\mathcal X, \mathcal Y')$, definiert durch $\skprod{B(\sigma) u}{v} := b(u, v; \sigma)$ mit $u \in \mathcal X$ und $v \in \mathcal Y$, stetig invertierbar.
    Konkret existieren Konstanten $0 < \beta_{1} \leq \beta_{2} < \infty$ mit
    \begin{equation}
        \sup_{\sigma \in \mathcal S} \norm{B(\sigma)} \leq \beta_{2} \quad \text{und} \quad  \sup_{\sigma \in \mathcal S} \norm{B(\sigma)^{-1}} \leq \frac{1}{\beta_{1}}.
    \end{equation}
    Außerdem erfüllt die parametrische Operatorfamilie $\{ B(\sigma) : \sigma \in \mathcal S \}$ \thref{thm:kunoth:assumption1} mit dem selben Regularitätsparameter $\mathfrak p$.
    Genauer erfüllt die parametrische Familie von Lösungen $u(\sigma)$ die \emph{a priori}-Abschätzung
    \begin{equation}
        \sup_{\sigma \in \mathcal S} \norm{(\partial^{\nu}_{\sigma} u)(\sigma)}_{\mathcal X} \leq C_{0} \norm{f}_{\mathcal Y'} \abs{\nu}! \tilde{b}^{\nu}.
    \end{equation}
\end{Satz}

% subsubsection aus_cite_kunoth_2013ef (end)

% section saetze (end)
