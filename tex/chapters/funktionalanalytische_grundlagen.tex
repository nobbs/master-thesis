%!TEX root = ../main.tex

\chapter{Funktionalanalytische Grundlagen} % (fold)
\label{cha:funktionalanalytische_grundlagen}

\section{Bochner-Integrale und Bochner-Räume} % (fold)
\label{sec:bochner_r_ume}

\begin{Definition}[Bochner-Räume]
    Sei $X$ ein Banachraum, $(a, b) \subset \mathbb{R}$, $- \infty \leq a < b \leq \infty$, ein offenes Intervall und $1 \leq p < \infty$.
    Wir nennen $L_{p}(a, b; X)$ einen Bochner-Raum und meinen damit die Menge der Äquivalenzklassen $L_{p}$-integrierbarer Funktionen $f \colon [a, b] \to X$, das heißt alle Lebesgue-messbaren Funktionen auf $[a, b]$ mit
    \begin{equation}
        \norm{f}_{L_{p}(a, b; X)} \coloneqq \left( \int_{a}^{b} \norm{f(t)}_{X}^{p} \diff t \right)^{\frac 1 p} < \infty.
    \end{equation}
    Analog bezeichnen wir mit $L_{\infty}(a, b; X)$ die Menge der Äquivalenzklassen die fast überall auf $[a, b]$ beschränkt sind, das heißt
    \begin{equation}
        \norm{f}_{L_{\infty}(a, b; X)} \coloneqq \esssup_{t \in [a, b]} \norm{f(t)}_{X} < \infty.
    \end{equation}
\end{Definition}

\begin{Lemma}
    Sei $1 \leq p \leq \infty$. Dann ist $L_{p}(a, b; X)$ ein Banachraum.
    Ist $H$ ein Hilbertraum, dann ist insbesondere auch $L_{2}(a, b; H)$ ein Hilbertraum.
\end{Lemma}

% \begin{Lemma}[Eigenschaften]
%     \begin{enumerate}
%         \item Ist $[a, b] \subset \mathbb{R}$ ein endliches Intervall, dann gilt die stetige Einbettung
%         \begin{equation}
%             L_{q}(a, b; X) \hookrightarrow L_{p}(a, b; X), \qquad q \geq p \geq 1.
%         \end{equation}
%         \item Sind $X$ und $Y$ Banachräume mit $X \hookrightarrow Y$, dann gilt die stetige Einbettung
%         \begin{equation}
%             L_{p}(a, b; X) \hookrightarrow L_{p}(a, b; Y), \qquad 1 \leq p \leq q.
%         \end{equation}
%     \end{enumerate}
% \end{Lemma}

\section{Sonstiges} % (fold)
\label{sec:sonstiges}

% \begin{Lemma}
%     $\mathcal C^{0}([a, b]; X)$ liegt dicht in $L_{p}(a, b; X)$ für $1 \leq p < \infty$.
% \end{Lemma}

\begin{Satz}[Poincaré-Friedrichs-Ungleichung, vgl. {{\cite[Theorem II.1.7]{Braess:2007wm}}}]
\label{satz:grundlagen:poincare_friedrichs_ungleichung}
    Es sei $\Omega \subset \mathbb{R}^{n}$ beschränkt und in einem $n$-dimensionalen Würfel mit Seitenlänge $s$ enthalten.
    Dann gilt
    \begin{equation}
        \label{eq:grundlagen:poincare_friedrichs_ungleichung}
        \abs{u}_{H^{m}} \leq \norm{u}_{H^{m}} \leq (1 + s)^{m} \abs{u}_{H^{m}} \quad \text{für alle}~u \in H^{m}_{0}(\Omega).
    \end{equation}
\end{Satz}

\begin{Lemma}[vgl {{\cite[Remark 2.1.48]{Sauter:9_WoPZ0Y}}}]
\label{lemma:sauter:2.1.48}
    Seien $X$ und $Y$ zwei reflexive Banachräume und $a \colon X \times Y \to \mathbb{R}$ eine Bilinearform.
    Finden wir für jedes $x \in X$ ein $y_{x} \in Y$, so dass
    \begin{equation}
        \label{eq:lemma:sauter:2.1.48:eq1}
        \abs{a(x, y_{x})} \geq C_{1} \norm{x}_{X}^{2} \quad \text{und} \quad \norm{y_{x}}_{Y} \leq C_{2} \norm{x}_{X}
    \end{equation}
    mit von $x$ und $y_{x}$ unabhängigen Konstanten $C_{1}, C_{2} > 0$ gilt, dann folgt daraus die inf-sup-Bedingung
    \begin{equation}
    \label{eq:lemma:sauter:2.1.48:inf_sup}
        \inf_{0 \neq x \in X} \sup_{0 \neq y \in Y} \frac{a(x, y)}{\norm{x}_{X}\norm{y}_{Y}} \geq \gamma > 0
    \end{equation}
    mit $\gamma = \frac{C_{1}}{C_{2}}$.

    \begin{Beweis}
        Seien $x \in X$ und $y_{x} \in Y$ so, dass \eqref{eq:lemma:sauter:2.1.48:eq1} erfüllt ist.
        Dann gilt
        \begin{align}
            \inf_{0 \neq x \in X} \sup_{0 \neq y \in Y} \frac{\abs{a(x, y)}}{\norm{x}_{X} \norm{y}_{Y}}
            &\geq
            \inf_{0 \neq x \in X} \frac{\abs{a(x, y_{x})}}{\norm{x}_{X} \norm{y_{x}}_{Y}}
            \\&\geq
            \inf_{0 \neq x \in X} \frac{C_{1} \norm{x}^{2}_{X}}{\norm{x}_{X} C_{2} \norm{x}_{X}}
            =
            \frac{C_{1}}{C_{2}}
            > 0.
        \end{align}
    \end{Beweis}
\end{Lemma}

% section sonstiges (end)
