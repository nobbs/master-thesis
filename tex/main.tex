%!TEX root = main.tex

\documentclass[
  % draft,
  % final,
  a4paper,
  % 11pt,
  % twoside,
  % BCOR=1cm,
  twoside=semi,
  % oneside,
  % parskip=half,
  toc=bibliography,
  toc=listof,
  chapterprefix=true,
  version=last,
  cleardoublepage=empty,
]{scrbook}

%!TEX root = main.tex

%%%%%%%%%%%%%%%%%%%%%%%%%%%%%%%%%%%%%%%%%%%%%%%%%%%%%%%%%%%%%%%%%%%%%%%%%%%%%%%
%%% Allgemeines

% Mehr Speicher für latex
\usepackage{etex}

% Encoding und Schriftsystem
\usepackage[utf8]{inputenc}
\usepackage[T1]{fontenc}

% Deutsche Silbentrennung
\usepackage[ngerman]{babel}

% Bessere Standard-Schriftart
\usepackage{lmodern}

% Typographische Kleinigkeiten
\usepackage{microtype}

% Graphiken und Farben
\usepackage{graphicx, color}

% PDF-Verlinkungen und Metadaten
\usepackage[colorlinks=false]{hyperref}

% Bibliographie-Stil
\bibliographystyle{amsplain}

% Blindtext
\usepackage{blindtext}
\blindmathtrue

% Dashed Linien...
\usepackage{dashrule}

%%%%%%%%%%%%%%%%%%%%%%%%%%%%%%%%%%%%%%%%%%%%%%%%%%%%%%%%%%%%%%%%%%%%%%%%%%%%%%%
%%% Mathematik

% Standardpackages
\usepackage{amsmath, amssymb, stmaryrd, mathtools}

% "Bessere" Theorem-Umgebungen
\usepackage[standard,amsmath,thmmarks,hyperref]{ntheorem}

% Setze Label-Nummern nur, wenn diese auch referenziert werden
\mathtoolsset{showonlyrefs=true}

%!TEX root = main.tex

% griechisches Alphabet
\renewcommand{\epsilon}{\varepsilon}
\renewcommand{\theta}{\vartheta}

%%% Operatoren und Funktionen
\DeclarePairedDelimiter{\abs}{\lvert}{\rvert}
\DeclarePairedDelimiter{\norm}{\lVert}{\rVert}

\DeclarePairedDelimiter{\ceil}{\lceil}{\rceil}
\DeclarePairedDelimiter{\floor}{\lfloor}{\rfloor}

\newcommand{\skprod}[2]{\left\langle#1,#2\right\rangle}
\newcommand{\fracpart}[2]{\frac{\partial#1}{\partial#2}}

\newcommand{\Transp}{^{\mathrm{T}}}
\newcommand{\Stern}{^{*}}
\newcommand{\Int}[1]{#1^\circ}
\newcommand{\Ext}[1]{\overline{#1}}

\DeclareMathOperator{\spn}{span}
\DeclareMathOperator{\ee}{e}
\DeclareMathOperator{\ii}{i}

\newcommand{\grad}{\nabla}
\DeclareMathOperator{\divergenz}{div}
\newcommand{\hesse}{\nabla^2}

\newcommand\restr[2]{\ensuremath{\left.#1\right|_{#2}}}

\newcommand{\fa}{\text{für alle}~}

% fetter Vektor
\renewcommand{\vec}[1]{\mathbf{#1}}
\newcommand{\mat}[1]{\mathbf{#1}}

% Differential-d
\newcommand*\diff{\mathop{}\!\mathrm{d}}
\newcommand*\Diff[1]{\mathop{}\!\mathrm{d^#1}}

\DeclareMathOperator*{\esssup}{ess\,sup}
\newcommand\blank{{\mkern2mu\cdot\mkern2mu}}

% -*- root: main.tex -*-

\newcommand{\titel}{Numerische Behandlung von Self-Consistent Field Theory-Modellen mittels Reduzierter-Basis-Methoden}
\newcommand{\art}{Masterarbeit}
\newcommand{\autor}{Alexej Disterhoft}
\newcommand{\fach}{Mathematik}
\newcommand{\matrikelnr}{2669611}
\newcommand{\erstgutachter}{Prof. Dr. Thorsten Raasch}
\newcommand{\zweitgutachter}{Prof. Dr. Martin Hanke-Bourgeois}
% \newcommand{\monat}{Irgendwann}
% \newcommand{\jahr}{2015}
\newcommand{\ort}{Mainz}
\newcommand{\logo}{figures/title/logo_gross.eps}

\hypersetup{pdfinfo={
  Title=\titel,
  Author=\autor}
}

%!TEX root = main.tex

\DeclareAcronym{scft}{
	short            = SCFT,
	long             = selbstkonsistente Feldtheorie,
	long-plural-form = selbstkonsistenten Feldtheorie,
	short-plural     = {},
	foreign          = \foreign{engl.}{self-consistent field theory}
}

\DeclareAcronym{rbm}{
	short = RBM,
	long  = Reduzierte-Basis-Methode
}

\DeclareAcronym{fem}{
	short = FEM,
	long = Finite-Elemente-Methode
}

\DeclareAcronym{bnb}{
	short = BNB,
	long = Banach-Ne{\v c}as-Babu{\v s}ka-Theorem
}

% -*- root: main.tex -*-

\newglossaryentry{symb:stetige_inklusion} {
    name={\ensuremath{\hookrightarrow}},
    description={Stetige Einbettung}
}
\newglossaryentry{symb:bochner_skp} {
    name={\ensuremath{[f, g]}},
    description={Kurzschreibweise für \ensuremath{\int_{T} \skprod{f(t)}{g(t)} \diff t}}
}


\addbibresource{literature.bib}

\begin{document}
    %%% Frontmatter-Teil
    \frontmatter{}

    % Deckblack / Titelseite
    %!TEX root = ../main.tex

\thispagestyle{plain}
\begin{titlepage}

\begin{center}

\huge{\textbf{\titel}}\\[1.5ex]
\LARGE{\textbf{\untertitel}}\\[6ex]
\LARGE{\textbf{\art}}\\[1.5ex]
\Large{im Fach \fach}\\[18ex]

\includegraphics[scale=0.5]{\logo}\\[6ex]

\normalsize
\begin{tabular}{p{5.4cm}p{6cm}}\\
vorgelegt von:  & \quad \autor\\[1.2ex]
Matrikelnummer: & \quad \matrikelnr\\[1.2ex]
Erstgutachter:  & \quad \erstgutachter\\[1.2ex]
Zweitgutachter: & \quad \zweitgutachter\\[3ex]
\end{tabular}

\end{center}
\end{titlepage}


    % Danksagung
    % % -*- root: ../main.tex -*-

\thispagestyle{empty}
\vspace*{0.2\textheight}
\noindent\enquote{I guess you could call it a \enquote{failure}, but I prefer the term \enquote{learning
experience}.}\bigbreak

\hfill Andy Weir, \textit{The Martian}

\vfill{}
\begin{flushright}
\emph{Für meine Eltern.}
\end{flushright}
\cleardoublepage


    % Abstrakt
    % %!TEX root = ../main.tex

\chapter{Zusammenfassung} % (fold)
\label{cha:Zusammenfassung}

\blindtext

% chapter Zusammenfassung (end)


    % Inhaltsverzeichnis
    \tableofcontents

    % TODO-Liste
    % \listoftodos

    %%% Hauptteil
    \mainmatter{}

    % Kapitel einbinden
    %!TEX root = ../main.tex

\setchapterpreamble[ul][0.6\textwidth]{%
    \dictum[Douglas Adams, \textit{The Hitchhiker’s Guide to the Galaxy}]{%
        \enquote{\foreignlanguage{english}{He attacked everything in life with a mixture of extraordinary genius and naive incompetence and it was often difficult to tell which was which.}}
    }
    \vspace*{2\baselineskip}
}
\chapter{Einleitung} % (fold)
\label{cha:el:einleitung}

Ein Polymer, oder auch Makromolekül, ist ein Molekül, welches sich aus vielen kleineren, sich wiederholenden Molekülen, sogenannten Monomeren, zusammensetzt.
Besteht ein Polymer aus nur einer Monomer-Art, dann spricht man von einem Homopolymer, sonst von einem Heteropolymer oder auch Copolymer.
Typischerweise besteht ein Polymer aus einer langen Kette von aneinanderhängenden Monomeren, es existieren aber auch weitere Konfigurationen, zum Beispiel stern- oder ringförmige Anordnungen.

Copolymere lassen sich anhand der Anordnung der Monomere weiter klassifizieren.
Bilden die verschiedenen Monomer-Gattungen homogene, zusammenhängende Gruppen, welche wiederum durch aneinanderreihen das Copolymer bilden, dann nennen wir dies ein Blockcopolymer, vergleiche \cref{fig:el:polymerketten}.

Es existieren unüberschaubar viele solcher Konfigurationen von Polymeren, und insbesondere Blockcopolymeren, weswegen man häufig auf das Studium vergleichsweise simpler Anordnungen zurückgreift.
Als besonders beliebt hat sich der Fall des kettenförmigen Blockcopolymers mit zwei Monomer-Typen, der Einfachheit halber A und B genannt, herausgestellt.
Diese Konfiguration wird auch als AB-Diblockcopolymer bezeichnet.

\begin{figure}[tb]
    \centering
    \begin{subfigure}[b]{\textwidth}
        \centering
        \includestandalone[width=0.8\textwidth]{tikz/einleitung/fig1}
    \end{subfigure}
    \\[1em]
    \begin{subfigure}[b]{\textwidth}
        \centering
        \includestandalone[width=0.8\textwidth]{tikz/einleitung/fig2}
    \end{subfigure}
    \\[1em]
    \begin{subfigure}[b]{\textwidth}
        \centering
        \includestandalone[width=0.8\textwidth]{tikz/einleitung/fig3}
    \end{subfigure}
    \caption[Skizzenhafte Darstellung verschiedener Polymerarten]{%
        Skizzenhafte Darstellung verschiedener Polymerarten.
        Von oben nach unten: Homopolymer, ein AB-Diblockcopolymer und ein sogenanntes statistisches AB-Copolymer, bei dem die beiden Monomer-Arten zufällig verteilt sind.
    }
    \label{fig:el:polymerketten}
\end{figure}

Von großem Interesse ist das Verhalten von Polymerschmelzen (\foreign{engl.}{polymer melt}), das heißt, des flüssigen Aggregatzustands eines Polymers, sowie das Verhalten von Gemischen verschiedener polymerer Stoffe.
So neigen die Gemische vieler Paare von Homopolymeren zu makroskopischer Phasenseparation, wie man es zum Beispiel von Öl und Essig kennt.
Eine ähnliche Tendenz findet man auch bei den Polymerschmelzen von Blockcopolymeren, hierbei kann aufgrund der Verbindung der verschiedenen Monomer-Blöcke aber keine makroskopische Phasenseparation auftreten, stattdessen kommt es zu einer periodischen, mikroskopischen Separation.
\cref{fig:el:phasen} zeigt einige mögliche Anordnungen, die bei Diblockcopolymeren tatsächlich experimentell beobachtet wurden.

\begin{figure}[tb]
    \centering
    \begin{subfigure}[b]{0.18\textwidth}
        \includegraphics[width=\textwidth]{figures/einleitung/fig1}
    \end{subfigure}
    \begin{subfigure}[b]{0.18\textwidth}
        \includegraphics[width=\textwidth]{figures/einleitung/fig2}
    \end{subfigure}
    \begin{subfigure}[b]{0.18\textwidth}
        \includegraphics[width=\textwidth]{figures/einleitung/fig3}
    \end{subfigure}
    \begin{subfigure}[b]{0.18\textwidth}
        \includegraphics[width=\textwidth]{figures/einleitung/fig4}
    \end{subfigure}
    \begin{subfigure}[b]{0.18\textwidth}
        \includegraphics[width=\textwidth]{figures/einleitung/fig5}
    \end{subfigure}
    \caption[Verschiedene Phasen bei Diblockcopolymeren]{%
        Verschiedene Phasen bei Diblockcopolymeren, welche experimentell beobachtet wurden, wobei hierbei nur eine der beiden Monomer-Gattungen dargestellt wird.
        Diese heißen von links nach rechts: Lamellar, Perforiert-Lamellar, Sphärisch, Zylindrisch, Gyroid.
        Diese Abbildung wurde \cite[Figure 1.18]{Matsen:2006ud} entnommen.
    }
    \label{fig:el:phasen}
\end{figure}

Da die experimentelle Bestimmung ohne Vorwissen über die möglichen, stabilen Anordnungen nur wenig erfolgversprechend ist, wird eine fundierte Theorie benötigt, auf Basis derer theoretische Vorhersagen getroffen werden können, die vorzugsweise wiederum experimentell belegbar sein sollten.
Da Diblockcopolymere einen relativ simplen Fall eines Copolymers darstellen, wurde sowohl in die experimentelle als auch theoretische Untersuchung dieser bereits vergleichsweise viel Arbeit investiert.

Als besonders nützliche und dennoch relativ einfache Theorie hat sich das auf der sogenannten \acp{scft} basierende Modell herausgestellt.

\subsection*{Mathematische Modellierung} % (fold)

Als Grundlage für die \acl{scft} dient eine Modellierung der Polymere als frei bewegliche Ketten (\foreign{engl.}{ideal chain}).
Dabei gibt es einige verschiedene Modelle, die hierfür verwendet werden.
Als relevante Modelle wollen wir hier ein diskretes, \enquote{grobkörniges} (\foreign{engl.}{coarse-grained}) Modell und das stetige Gaußsche Kettenmodell erwähnen, eine ausführliche Ausarbeitung findet man bei \textcites[Chapter 2]{Fredrickson:2006th}{rubinstein2003polymer}.
Im Folgenden sei $\vec{r}$ ein Vektor, der eine Position in einem Volumen angibt.

\begin{figure}[tb]
    \centering
        \includestandalone[width=0.6\textwidth]{tikz/einleitung/chains}
    \caption[%
        Polymerkette in diskretem und Gaußschen Kettenmodell
    ]{%
        Schematische Darstellung einer Polymerkette im diskreten Kettenmodell (links) und im stetigen Gaußschen Kettenmodell (rechts).
        Abbildung reproduziert nach \cite[Figure 2.1 und 2.5]{Fredrickson:2006th}.
    }
    \label{fig:el:kettenmodelle}
\end{figure}

Das \enquote{grobkörnige} Modell stellt die Polymerkette als eine diskrete Kette von Partikeln so dar, dass aneinanderhängende Monomere ähnlich einem Scharnier frei beweglich sind.
Dabei werden Wechselwirkung zwischen benachbarten Monomeren berücksichtigt, zwischen auf der Kette weit auseinanderliegenden Partikeln aber ignoriert.
Diese Wechselwirkungen können am Beispiel von \cref{fig:el:kettenmodelle} beispielsweise als Einschränkung des Winkels $\vartheta_9$ durch die gegenseitige Beeinflussung der Partikel $8, 9$ und $10$ auftreten.
Das diskrete Modell hängt stark mit den aus der Stochastik bekannten Random Walks zusammen und lässt sich deswegen auch ausführlich mit stochastischen und statistischen Methoden untersuchen.

Das stetige Gaußsche Kettenmodell, welches man unter anderem auch als stetigen Grenzfall des beschriebenen diskreten Modells erhält, hat sich als besonders nützlich erwiesen, sowohl bei analytischen als auch numerischen Betrachtungen.
Dabei wird die Polymerkette als stetige, linear elastische Faser aufgefasst und durch eine Kurve $\vec{r}(s)$ parametrisiert, wobei $s \in [0, 1]$ eine entlang der Kontur der Kette laufende Variable ist.
Ähnlich wie beim diskreten Modell findet man auch hier viele Zusammenhänge zu stochastischen Prozessen, hier vor allem zu Brownschen Bewegungen,
wodurch auch bei diesem Modell ein umfangreicher \enquote{Werkzeugkasten} zur Untersuchung zur Verfügung steht.

Da die benötigten stochastischen Ausführungen und Herleitungen für diese Arbeit nebensächlich sind, belassen wir es bei diesen informalen Beschreibungen und widmen uns nun der darauf aufbauenden \ac{scft}.

% subsection mathematische_modellierung (end)

\subsection*{Selbstkonsistente Feldtheorie} % (fold)

Die \acl{scft} ist ein weit verbreitetes theoretisches Modell der Physik um das Verhalten von Teilchen unter Einwirkung von Kräften, die durch Wechselwirkungen mit weiteren Teilchen auftreten, zu studieren und wird nicht nur im Zusammenhang mit Polymeren sondern zum Beispiel auch in der Thermodynamik oder Informatik verwendet.

Als Grundidee dient dabei, dass in einem System mit sehr vielen Objekten, welche miteinander wechselwirken, eine hinreichend gute Beschreibung der auf eines dieser Objekte wirkenden Kräfte durch Mitteln der Wechselwirkungen vorliegt.
Diese gemittelten Einwirkungen werden als externes Feld aufgefasst und ignorieren dabei Fluktuationen, das heißt, Veränderungen der wirkenden Kräfte durch das lokale Verhalten der einzelnen Teilchen.
Damit erreicht man effektiv die Reduktion eines Mehrkörperproblems auf ein Einkörperproblem und kann so das Verhalten eines solchen Systems untersuchen.

Dieses Prinzip lässt sich auch zur Untersuchung von Polymeren verwenden,
da ein einzelnes Polymer oftmals aus einer hohen vierstelligen Zahl von Atomen besteht und dadurch die Wechselwirkungen auf atomarem Level vernachlässigbar sind, weswegen auch die im vorherigen Abschnitt beschriebene Modellierung als Kette sinnvoll erscheint.
Aufbauend auf den beschriebenen Modellen, in diesem Fall dem Gaußschen Kettenmodell, kann man so die statistische räumliche Verteilung beziehungsweise Ausrichtung eines Polymers bestimmen.

Wir beschränken uns im Folgenden auf die Beschreibung der \ac{scft} für die inkompressible Schmelze eines AB-Diblockcopolymers aufbauend auf dem stetigen Gaußschen Kettenmodell und folgen dabei größtenteils den Ausführungen von \textcite{Matsen:1994bz,Stasiak:2011ba}.

Betrachte eine einzelne Volumenzelle, beispielsweise einen Würfel, welche selbst Teil eines größeren Systems sein kann.
Diese Zelle enthalte $n$ AB-Diblockcopolymere, welche jeweils aus einem A-Block und einem B-Block bestehen, wobei diese wiederum aus $N_{\mathrm{A}}$ Monomeren vom Typ A respektive aus $N_{\mathrm{B}}$ Monomeren vom Typ B bestehen.
Der Polymerisationsgrad, das heißt, die Gesamtanzahl an Monomeren in einem Polymer, ergibt sich damit zu $N = N_{\mathrm{A}} + N_{\mathrm{B}}$.
Weiter bezeichne $f = N_{\mathrm{A}} / N$ den Anteil an A-Monomeren im gesamten Polymer.

Als vereinfachende Annahmen sei die statistische Länge $a$ eines Monomers, auch Kuhn-Länge genannt, der beiden Monomer-Gattungen gleich und ein Monomer beider Gattungen nehme das selbe Volumen $1 / \rho_{0}$ ein.
Das Gesamtvolumen der Schmelze in dieser Zelle ist damit gegeben durch $V = n N / \rho_{0}$.

Wie bei der Beschreibung des Gaußschen Modells sei $s \in [0, 1]$ eine Distanz entlang der Kontur eines Polymers, wobei $s = 0$ und $s = 1$ den beiden Enden entspreche.

Die wichtigsten Größen bei der \ac{scft} sind nun die Konzentrationen $\phi_{\mathrm{A}}(\vec{r})$ und $\phi_{\mathrm{B}}(\vec{r})$ der A- und B-Monomere an einer Position $\vec{r}$ in der betrachteten Zelle und die externen Felder $\omega_{\mathrm{A}}(\vec{r})$ und $\omega_{\mathrm{B}}(\vec{r})$, welche auf die jeweiligen Monomer-Gattungen wirken.

Als Ausgangspunkt für die Bestimmungen möglicher stabiler Anordnungen der Polymere in der Schmelze betrachtet man das folgende Funktional, welches eine Approximation für die sogenannte freie Energie des Systems liefert und dessen physikalische Motivation den Rahmen dieser Einführung sprengen würde, siehe zum Beispiel \cite{Matsen:2006ud,Fredrickson:2006th}.
Die freie Energie $F$ eines einzelnen Polymers lässt sich bestimmen als
\begin{equation}
\label{eq:el:freie_energie_funktional}
    \frac{F}{nk_{\mathrm{B}}T} = - \ln \frac{Q}{V} + \frac{1}{V} \int \chi N \phi_{\mathrm{A}}(\vec{r}) \phi_{\mathrm{B}}(\vec{r}) - \omega_{\mathrm{A}}(\vec{r}) \phi_{\mathrm{A}}(\vec{r}) - \omega_{\mathrm{B}}(\vec{r}) \phi_{\mathrm{B}}(\vec{r}) \diff \vec{r},
\end{equation}
wobei $\chi$ der sogenannte Flory-Huggins-Wechselwirkungsparameter für die Wechselwirkungen zwischen den Monomeren vom Typ A und B und $k_{\mathrm{B}} T$ die thermische Energie ist.

Stabile Anordnungen entsprechen dabei Sattelpunkten von $F$ bezüglich der Konzentrationen $\phi_{\mathrm{A}}$, $\phi_{\mathrm{B}}$ und der Felder $\omega_{\mathrm{A}}$ und $\omega_{\mathrm{B}}$.
Betrachtet man die Funktionalableitungen von $F$ bezüglich dieser Größen, dann erhält man eine Menge von Gleichungen, welche oft als \ac{scft}-Gleichungen bezeichnet werden, da Anhand dieser die gesuchten Sattelpunkte bestimmt werden können.

Diese Gleichungen bestehen aus der Inkompressibilität der Schmelze, welche durch die Bedingung
\begin{equation}
\label{eq:el:inkompressibilitaet}
    \phi_{\mathrm{A}}(\vec{r}) + \phi_{\mathrm{B}}(\vec{r}) = 1
\end{equation}%
berücksichtigt wird, sowie der Kopplung der Felder und der Konzentrationen durch
\begin{equation}
\label{eq:el:felder}
    \begin{aligned}
        \omega_{\mathrm{A}}(\vec{r}) = \chi N \phi_{\mathrm{B}}(\vec{r}) + \xi(\vec{r}),\\
        \omega_{\mathrm{B}}(\vec{r}) = \chi N \phi_{\mathrm{A}}(\vec{r}) + \xi(\vec{r}),
    \end{aligned}
\end{equation}%
wobei mit dem Lagrange-Multiplikator $\xi(\vec{r})$ die Inkompressibilität \cref{eq:el:inkompressibilitaet} erzwungen wird.
Weiter erhält man eine Darstellung der Konzentrationen in Form von
\begin{equation}
\label{eq:el:konzentrationen}
    \begin{aligned}
        \phi_{\mathrm{A}}(\vec{r}) = \frac{V}{Q} \int_{0}^{f} q(\vec{r}, s) q^{\dagger}(\vec{r}, s) \diff s,\\
        \phi_{\mathrm{B}}(\vec{r}) = \frac{V}{Q} \int_{f}^{1} q(\vec{r}, s) q^{\dagger}(\vec{r}, s) \diff s,
    \end{aligned}
\end{equation}%
wobei $Q = Q(\omega_{\mathrm{A}}, \omega_{\mathrm{B}})$ die Partitionsfunktion (\foreign{engl.}{partition function}) eines einzelnen Polymers ist und durch
\begin{equation}
\label{eq:el:partitionsfunktion}
    Q = \int q(\vec{r}, 1) \diff \vec{r}
\end{equation}%
bestimmt wird.

Die in den Gleichungen \cref{eq:el:konzentrationen,eq:el:partitionsfunktion} auftretende Funktion $q(\vec{r}, s)$ wird als Vorwärts-Propagator bezeichnet und erfüllt die parabolische partielle Differentialgleichung
\begin{equation}
\label{eq:el:forward_propagator}
    \left\{
    \begin{aligned}
        \frac{\partial q}{\partial s}(\vec{r}, s) &= \frac{a^{2}N}{6} \Delta q(\vec{r}, s) - \omega_{\alpha(s)}(\vec{r}) q(\vec{r}, s),\\
        q(\vec{r}, 0) &= 1.
    \end{aligned}
    \right.
\end{equation}%
Analog bezeichnet man $q^{\dagger}(\vec{r}, s)$ als Rückwärts-Propagator, welcher eine ähnliche Differentialgleichung, nämlich
\begin{equation}
\label{eq:el:backward_propagator}
    \left\{
    \begin{aligned}
        -\frac{\partial q^{\dagger}}{\partial s}(\vec{r}, s) &= \frac{a^{2}N}{6} \Delta q^{\dagger}(\vec{r}, s) - \omega_{\alpha(s)}(\vec{r}) q^{\dagger}(\vec{r}, s),\\
        q^{\dagger}(\vec{r}, 1) &= 1,
    \end{aligned}
    \right.
\end{equation}%
erfüllt.
Dabei ist die Funktion $\omega_{\alpha(s)}$ gegeben durch
\begin{equation}
    \omega_{\alpha(s)}(\vec{r}) = \begin{cases}
        \omega_{\mathrm{A}}(\vec{r}), & 0 \leq s \leq f\\
        \omega_{\mathrm{B}}(\vec{r}), & f < s \leq 1.
    \end{cases}
\end{equation}

Der Propagator $q(\vec{r}, s)$ repräsentiert das statistische Gewicht, im Wesentlichen also eine nichtnormalisierte Wahrscheinlichkeit, dass man ein Polymer findet, welches irgendwo innerhalb des Volumens beginnt, und dessen Teilstück, das zur Konturvariable $s$ gehört, sich an der Position $\vec{r}$ befindet.
Die Anfangsbedingung $q(\vec{r}, 0) = 1$ lässt sich damit so interpretieren, dass ein Polymer der Länge Null von den externen Feldern nicht beeinflusst wird.
Der Rückwärts-Propagator hat die selbe Bedeutung, allerdings wird hierbei das andere Ende des Polymers festgehalten.

Die Tatsache, dass sich die Konzentrationen $\phi_{\mathrm{A}}$ und $\phi_{\mathrm{B}}$ aus Lösungen der obigen Differentialgleichungen bestimmen lassen, entstammt aus dem Gaußschen Kettenmodell und dessen stochastischen Hintergründen.

Je nachdem, welches Szenario man betrachtet, das heißt, eine Volumenzelle innerhalb eines größeren Systems, oder eine Zelle, die durch feste Wände begrenzt wird, erhält man verschiedene Randbedingungen an die beiden Differentialgleichungen, so entspricht ersteres beispielsweise periodischen Randbedingungen.

Ein weiterer Punkt, der hier noch erwähnt werden soll, da er sich im Rahmen dieser Arbeit als äußerst nützlich erweisen wird, ist, dass das Funktional $F$ der freien Energie \cref{eq:el:freie_energie_funktional} invariant ist bezüglich konstanter Verschiebungen der Felder $\omega_{\mathrm{A}}$ und $\omega_{\mathrm{B}}$, wie beispielsweise \cite{Ceniceros:2006is} entnommen werden kann.

Diese Gleichungen, genauer \cref{eq:el:inkompressibilitaet,eq:el:felder,eq:el:konzentrationen,eq:el:partitionsfunktion,eq:el:forward_propagator,eq:el:backward_propagator}, lassen sich nun numerisch lösen und so stabile Anordnungen über die Sattelpunkte von \cref{eq:el:freie_energie_funktional} bestimmen.

% subsection selbstkonsistente_feldtheorie (end)

\subsection*{Einsatz numerischer Methoden} % (fold)

Es gibt verschiedene Ansätze, die \ac{scft}-Gleichungen zu einem numerischen Verfahren zu verarbeiten, die meisten führen auf das folgende iterative, Newton-artige Schema:
\begin{enumerate}[label={\itshape\roman*.},ref={\itshape\roman*}]
    \item Zunächst werden zwei externe Felder $\omega^{(0)}_{\mathrm{A}}$ und $\omega^{(0)}_{\mathrm{B}}$ generiert, typischerweise zufällig, um von vornherein auftretende Verzerrungen zu verhindern.
    \item\label{eq:el:iterationsverfahren_punkt_2} Die beiden partiellen Differentialgleichungen \cref{eq:el:forward_propagator,eq:el:backward_propagator} werden für die Felder $\omega^{(k)}_{\mathrm{A}}$ und $\omega^{(k)}_{\mathrm{B}}$ gelöst.
    \item Die Konzentrationen $\phi^{(k)}_{\mathrm{A}}$ und $\phi^{(k)}_{\mathrm{B}}$ werden durch Auswerten der Gleichungen \cref{eq:el:partitionsfunktion,eq:el:konzentrationen} bestimmt.
    \item Diese werden nun benutzt um aus den Gleichungen \cref{eq:el:felder} die zu diesen Konzentrationen zugehörigen Felder zu bestimmen.
    Aus diesen Feldern werden nun mit einem sogenannten Mixing-Verfahren die neuen Felder $\omega^{(k+1)}_{\mathrm{A}}$ und $\omega^{(k+1)}_{\mathrm{B}}$ für die nächste Iteration erzeugt.
    Das Mixing dient dazu, die Konvergenz des Verfahrens sicherzustellen beziehungsweise zu verbessern, typischerweise gehen hier die Inkompressibilität \cref{eq:el:inkompressibilitaet} und zurückliegende Iterationen ein.
    \item Sind die neuen Konzentrationen und Felder noch kein Sattelpunkt von \cref{eq:el:freie_energie_funktional}, dann gehe zurück zu \cref{eq:el:iterationsverfahren_punkt_2}.
\end{enumerate}

Ein Beispiel für eine mögliche Anordnung, die durch dieses Iterationsverfahren für den Fall einer Raumdimension bestimmt wurde, ist in \cref{fig:el:felder_nach_iterationsverfahren} zu sehen.

\begin{figure}[tb]
    \centering
    \begin{subfigure}[b]{0.45\textwidth}
        \centering
        \includestandalone[width=1\textwidth]{tikz/einleitung/iter1}
        \caption{Felder}
    \end{subfigure}
    ~
    \begin{subfigure}[b]{0.45\textwidth}
        \centering
        \includestandalone[width=1\textwidth]{tikz/einleitung/iter2}
        \caption{Konzentrationen}
    \end{subfigure}
    \caption[%
    Eindimensionales Beispiel einer stabilen Anordnung eines Diblockcopolymers
    ]{%
        Eindimensionales Beispiel einer stabilen Anordnung eines Diblockcopolymers, welche mittels \ac{scft} bestimmt wurde.
        Simuliert wurde auf einem Intervall der Länge $L = 10$ mit den relevanten Größen $f = 1/2$, $\chi N = 25$ und $a^{2} N / 6 = 10 / 3$.
        Monomer-Typ A entspricht den blauen und B dementsprechend den orangen Graphen.
    }
    \label{fig:el:felder_nach_iterationsverfahren}
\end{figure}

Bei dem beschriebenen Verfahren stellen sich die folgenden beiden Probleme als besonders wichtig heraus, da sie massiv die Laufzeit des Iterationsverfahren beeinflussen:
\begin{enumerate}[label={\itshape\roman*.}]
    \item das Lösungsverfahren für die parabolischen partiellen Differentialgleichungen \cref{eq:el:forward_propagator} und \cref{eq:el:backward_propagator},
    \item das Mixing-Verfahren, mit dem iterativ neue Felder $\omega_{\mathrm{A}}$ und $\omega_{\mathrm{B}}$ bestimmt werden.
\end{enumerate}

Auf den zweiten Punkt, das Mixing-Verfahren, werden wir im Verlauf dieser Arbeit nicht weiter eingehen.
Trotz dessen, und insbesondere, da es viele, sehr verschiedene Verfahren gibt, die hierfür zum Einsatz kommen, wollen wir hier einige Ansätze nennen.
Da es sich beim Mixing-Schritt im Wesentlichen um die Suche nach einem Sattelpunkt eines nichtlinearen Funktionals handelt, lassen sich hierfür viele bekannte Verfahren der nichtlinearen Optimierung, aber auch aus anderen Bereichen, anwenden.

Dies reicht von einem Quasi-Newton-Verfahren bei \textcite{Matsen:1994bz}, welches direkt an den darin verwendeten Löser für die partielle Differentialgleichung gekoppelt ist, bis zu Integrationsverfahren, wie zum Beispiel Runge-Kutta-Verfahren oder Mehrschrittverfahren.
Ein auf einem solchen Integrationsverfahren basierendes Mixing, neben einem weiteren, welches auf einem Conjugated-Gradients-Verfahren aufbaut, findet sich bei \textcite{Ceniceros:2006is}.
Als besonders nützlicher, da effektiver, Ansatz hat sich das sogenannte Anderson-Mixing erwiesen, siehe die Arbeiten von \textcite{Thompson:2004um,Stasiak:2011ba}.
Dabei werden neue Felder durch Kombination der Felder vieler zurückliegender Iterationen gewonnen.
Ein weiteres, einer Picard-Iteration ähnliches, Verfahren findet man bei \textcite{Drolet:1999bs}.

Unser Hauptaugenmerk in dieser Arbeit liegt auf dem ersten Problem, dem wiederholten Lösen der parabolischen partiellen Differentialgleichung \cref{eq:el:forward_propagator}.
Da es, abhängig vom gewählten Mixing-Verfahren, oftmals eine Iterationsanzahl im dreistelligen Bereich oder höher benötigt, bis eine zufriedenstellende Genauigkeit bei der Sattelpunktgleichung erreicht ist, und damit insbesondere auch die partielle Differentialgleichung so oft gelöst werden muss, ist es wichtig, dass das Lösungsverfahren möglichst effizient ist.
Weiter darf zu Gunsten der Laufzeit aber auch nicht die Genauigkeit des Lösers vernachlässigt werden, da sich dies im Iterationsverfahren durch Instabilität und zusätzliche Iterationen niederschlagen kann.

Ähnlich wie beim Mixing-Schritt wurden bereits viele verschiedene Ansätze mit mehr oder weniger zufriedenstellenden Ergebnissen verfolgt.
Da es sich bei \cref{eq:el:forward_propagator} im Grunde um eine Diffusionsgleichung handelt, lassen sich gut bekannte Verfahren, zum Beispiel ein Finite-Differenzen-Verfahren, anwenden.
So wird in der Arbeit von \textcite{Drolet:1999bs} ein Crank-Nicolson-Verfahren eingesetzt, wobei hierbei explizit der Laufzeit Vorrang gegenüber der Genauigkeit gegeben wurde.

Als guter Kompromiss zwischen Laufzeit und Genauigkeit haben sich Spektral- und Pseudospektralverfahren herausgestellt.
Erstere wurden von \textcite{Matsen:1994bz} erfolgreich eingesetzt, wobei hier erst das explizite Berücksichtigen der Symmetrien der zu erwartenden resultierenden Anordnung bei der Konstruktion des Spektralverfahrens zu annehmbaren Laufzeiten führt.
Die verwendeten Pseudospektralverfahren kommen zwar nicht an die Genauigkeit der Spektralverfahren heran, können aber unter Ausnutzung der Struktur der partiellen Differentialgleichung massiv Laufzeit einsparen.
Dazu wird bei \textcite{Rasmussen:2002kt} der Differentialoperator mittels Operator-Splitting so zerlegt, dass man das Lösen der Differentialgleichung mittels schneller Fourier-Transformation im Wesentlichen auf komponentenweise Vektor-Multiplikationen zurückführt.
Das daraus resultierende Verfahren zweiter Ordnung wurde von \textcite{GarciaCervera:2006uu,Ranjan:2007kl} auf unterschiedliche Weisen zu Verfahren vierter Ordnung erweitert, ohne signifikant Laufzeit einzubüßen.
Eine gute Übersicht über die meisten der genannten Methoden findet man bei \textcites[Section 3.6]{Fredrickson:2006th}{Audus:2013ep}.

Obwohl man in der Literatur zur \ac{scft} für Polymere verschiedenste Verfahren für das Lösen der partiellen Differentialgleichung findet, ist ein Finite-Elemente-Ansatz basierend auf einer Raum-Zeit-Variationsformulierung der partiellen Differentialgleichung unseres Wissens nach bisher nicht verfolgt worden.
Hier wollen wir anknüpfen und zugleich die Tatsache, dass die Differentialgleichung wiederholt für verschiedene Felder gelöst werden muss, in Form eines Reduzierte-Basis-Ansatzes ausnutzen.
Dies ist in aller Kürze das angestrebte Ziel dieser Arbeit.

% subsection einsatz_numerischer_methoden (end)

\subsection*{Aufbau der restlichen Arbeit}

In \cref{cha:gl:grundlagen} werden zunächst die benötigten funktionalanalytischen und numerischen Grundlagen eingeführt, beziehungsweise wiederholt.
Einige weitere Aussagen, die sich thematisch nur bedingt in diesem Kapitel unterbringen lassen, werden in \cref{cha:funktionalanalytische_grundlagen} gesammelt.

Es folgt \cref{cha:ps:problemstellung}, in dem die parabolische partielle Differentialgleichung, welche im Rahmen dieser Einleitung bereits vorgestellt wurde, konkretisiert und als parametrische Differentialgleichung formalisiert wird.
Anschließend werden wünschenswerte Aussagen hergeleitet, die eine sinnvolle Bearbeitung durch numerische Methoden erst ermöglichen.

\cref{cha:der_eindimensionale_fall} dient dazu, die bis dahin ausgearbeitete Theorie auf den einfachen, eindimensionalen Fall anzuwenden und numerische Methoden für diesen zu entwickeln.

Im nachfolgenden \cref{cha:reduzierte_basis_methode} werden diese Methoden als Grundlage für eine Reduzierte-Basis-Methode verwendet und dadurch ein effizientes Verfahren für die Approximation der parametrischen Differentialgleichung entwickelt.

% subsection aufbau_der_restlichen_arbeit (end)

% chapter einleitung (end)

    %!TEX root = ../main.tex

\setchapterpreamble[ul][0.6\textwidth]{%
    \dictum[Iain M. Banks, \textit{The Algebraist}]{\enquote{Elegance is an algorithm.}}
    \vspace*{2\baselineskip}
}
\chapter{Grundlagen} % (fold)
\label{cha:gl:grundlagen}

\mdo{Bei Bedarf erweitern!}

In diesem ersten Kapitel werden die für diese Arbeit benötigten Grundlagen aus der Funktionalanalysis und der Numerik zusammengefasst und wiederholt.
Wir orientieren uns in den ersten beiden, funktionalanalytischen Abschnitten maßgeblich an \textcite{Dautray:1992by,Lions:1972tg}, während für die nachfolgenden Abschnitte [Quelle].

\mdo{Quellen ergänzen!}

\section{Bochner-Räume} % (fold)
\label{sec:gl:br:bochner_raeume}

Wir beginnen mit der Einführung sogenannter Bochner-Räume.
Dabei handelt es sich um Verallgemeinerungen der Lebesgue-Räume $L_{p}$ auf Banachraum-wertige Funktionen.
Bochner-Räume treten bei der Betrachtung parabolischer partieller Differentialgleichungen in natürlicher Weise auf.

\begin{Definition}[{{{\cite[Definition XVIII.1.1]{Dautray:1992by}}}}]
\label{def:gl:br:bochner_raum}
    Sei $X$ ein Banachraum.
    Weiter seien $- \infty \leq a < b \leq \infty$ und $1 \leq p < \infty$.
    Als \emph{Bochner-Raum} $L_{p}(a, b; X)$ bezeichnen wir die Menge (der Äquivalenzklassen) $L_{p}$-integrierbarer Funktionen $f \colon [a, b] \to X$, das heißt, aller Lebesgue-messbarer Funktionen auf $[a, b]$ mit
    \begin{equation}
        \norm{f}_{L_{p}(a, b; X)} \deq \left( \int_{a}^{b} \norm{f(t)}_{X}^{p} \diff t \right)^{1 / p} < \infty.
    \end{equation}
    Weiterhin ist der \emph{Bochner-Raum} $L_{\infty}(a, b; X)$ definiert als die Menge (der Äquivalenzklassen) der für fast alle $t \in [a, b]$ wesentlich beschränkten Funktionen mit
    \begin{equation}
        \norm{f}_{L_{\infty}(a, b; X)} \deq \esssup_{t \in [a, b]} \norm{f(t)}_{X} < \infty.
    \end{equation}
\end{Definition}
\nomenclature{$X, Y$}{Banachräume}

\begin{Lemma}[{{\cite[Proposition XVIII.1.1]{Dautray:1992by}}}, {{\cite[Abschnitt 1.1.3]{Lions:1972tg}}}]
\label{lem:gl:br:bochner_ist_banachraum}
    Für alle $1 \leq p \leq \infty$ ist $L_{p}(a, b; X)$ ein Banachraum.
    Falls $H$ ein Hilbertraum ist, so auch $L_{2}(a, b; H)$.
\end{Lemma}

\begin{Definition}[Schwache Zeitableitung, {{{\cite{Dautray:1992by}}}}]
\label{def:gl:br:schwache_zeitableitung}
    Seien $X$ und $Y$ Banachräume mit $X \hookrightarrow Y$ und $u \in L_{2}(a, b; X)$.
    Die distributionelle Ableitung $\frac{\partial}{\partial t} u \in L_{2}(a, b; Y)$ sei definiert als das $v \in L_{2}(a, b; Y)$, welches
    \begin{equation}
        \int_{a}^{b} v(t) \varphi(t) \diff t = - \int_{a}^{b} u(t) \frac{\partial}{\partial t} \varphi(t) \diff t \quad \fa \phi \in C^{\infty}_{0}((a, b), \mathbb{R})
    \end{equation}
    erfüllt, falls ein solches $v$ existiert.
\end{Definition}

\begin{Bemerkung}
    Je nach Situation werden wir der Einfachheit halber eine der Schreibweisen $\frac{\partial}{\partial t} u = u' = u_{t}$ verwenden.
\end{Bemerkung}

Im Zuge dieser Arbeit werden wir es stets mit Hilberträumen zu tun haben.
Dabei werden wir oft auf folgendes Konstrukt zurückgreifen.

\begin{Definition}
\label{def:gl:br:gelfand_tripel}
    Seien $V$ und $H$ separable Hilberträume mit den Dualräumen $V'$ und $H'$.
    Weiter sei $V$ ein dichter Unterraum von $H$.
    Durch Identifikation von $H$ mit seinem Dualraum $H'$ erhalten wir das sogenannte \emph{Gelfand-Tripel} $(V, H, V')$, oder auch
    \begin{equation}
        V \denseinclusion H \simeq H' \denseinclusion V',
    \end{equation}
    wobei die Inklusionen jeweils dichte stetige Einbettungen sind.
\end{Definition}
\nomenclature{$V, H$}{Hilberträume}

\begin{Bemerkung}
    Mit $\skprod{\blank}{\blank}_{V}$ und $\skprod{\blank}{\blank}_{H}$ bezeichnen wir das Skalarprodukt auf $V$ respektive $H$.
    Weiterhin $\skprod{\blank}{\blank}_{V' \times V}$ wird auch für die duale Paarung auf $V' \times V$, die als die eindeutige stetige Fortsetzung von $\skprod{\blank}{\blank}_{H}$ definiert ist, verwendet.
    Insbesondere gilt für $u \in H \subset V'$ und $v \in V$ die Gleichheit
    \begin{equation}
        \skprod{u}{v}_{V' \times V} = \skprod{u}{v}_{H}.
    \end{equation}
\end{Bemerkung}
\nomenclature{$\skprod{\blank}{\blank}$}{Je nach Index Skalarprodukt oder duale Paarung}

\begin{Definition}[{{{\cite[Definition XVIII.2.4]{Dautray:1992by}}}}]
\label{def:gl:br:bochner_raum_W}
    Sei $V \denseinclusion H \denseinclusion V'$ ein Gelfand-Tripel.
    Definiere den Raum $W(a, b; V, V')$ als
    \begin{equation}
        W(a, b; V, V') \deq \Set{u \in L_{2}(a, b; V) \given u' \in L_{2}(a, b; V')}
    \end{equation}
    wobei $u'$ im Sinne von \cref{def:gl:br:schwache_zeitableitung} zu verstehen ist.
\end{Definition}
\nomenclature{$\denseinclusion$}{Dichte stetige Einbettung}

\begin{Bemerkung}
\label{bem:gl:br:bochner_alternative_darstellung}
    Eine alternative Darstellung von $W(a, b; V, V')$ ist
    \begin{equation}
        W(a, b; V, V') = L_{2}(a, b; V) \cap H^{1}(a, b; V').
    \end{equation}
\end{Bemerkung}

Bochner-Räume lassen sich anhand folgender Charakterisierung auch als Tensor-Produkte auffassen.
\begin{Satz}[{{\cite[Theorem 12.6.1, Theorem 12.7.1]{Aubin:2000un}}}]
\label{satz:gl:br:bochner_als_tensorprodukt}
    Seien $H$ ein Hilbertraum und $I \subset \mathbb{R}$ eine offene Menge.
    Dann ist der Bochner-Sobolev-Raum $H^{m}(I; H)$, $m \geq 0$, isometrisch zum Hilbert-Tensor-Produkt $H^{m}(I) \otimes H$.
\end{Satz}

\begin{Lemma}[{{{\cite[Proposition XVIII.2.6]{Dautray:1992by}}}}]
\label{lem:gl:br:bochner_W_ist_hilbertraum}
    Mit dem Skalarprodukt
    \begin{equation}
        \skprod{u}{v}_{W(a, b; V, V')} \deq \skprod{u'}{v'}_{L_{2}(a, b; V')} +  \skprod{u}{v}_{L_{2}(a, b; V)}
    \end{equation}
    und der induzierten Norm
    \begin{equation}
        \begin{aligned}
            \norm{u}_{W(a, b; V, V')}
            \deq& \left( \int_{a}^{b} \norm{u'(t)}_{V'}^{2} \diff t + \int_{a}^{b} \norm{u(t)}_{V}^{2} \diff t \right)^{1/2}
            \\=& \left( \norm{u'}_{L_{2}(a, b; V')}^{2} + \norm{u}_{L_{2}(a, b; V)}^{2} \right)^{1/2}
        \end{aligned}
    \end{equation}
    ist $W(a, b; V, V')$ ein Hilbertraum.
\end{Lemma}

\begin{Definition}
\label{def:gl:br:stetige_funktionen}
    Mit $\mathcal C([a, b]; X)$ bezeichnen wir die Menge aller bezüglich der Norm $\norm{f} = \sup_{t \in [a, b]} \norm{f(t)}_{X}$ stetigen Funktionen $f \colon [a, b] \to X$.
\end{Definition}

\begin{Satz}[{{{\cite[Theorem XVIII.2.1]{Dautray:1992by}}}}]
\label{satz:gl:br:bochner_eingebettet_in_stetigen_funktionen}
    Seien $a, b \in \mathbb{R}$, dann stimmt jedes $u \in W(a, b)$ fast überall mit einer stetigen Funktion von $[a, b]$ nach $H$ überein.
    Genauer gilt
    \begin{equation}
        W(a, b; V, V') \hookrightarrow \mathcal C([a, b]; H).
    \end{equation}
\end{Satz}
\nomenclature{$\hookrightarrow$}{Stetige Einbettung}

\begin{Korollar}[{{{\cite[Remark XVIII.2.4]{Dautray:1992by}}}}]
\label{kor:gl:br:bochner_spur_wohldefiniert}
    Sei $a, b \in \mathbb{R}$ und $u \in W(a, b; V, V')$.
    Dann sind $u(a), u(b) \in H$ wohldefiniert.
\end{Korollar}

\mfix{Prüfen, ob das so auch stimmt.}
\begin{Korollar}
\label{kor:gl:br:einbettungskonstante_M_e}
    Seien $a, b \in \mathbb{R}$.
    Die Einbettungskonstante
    \begin{equation}
        \label{eq:gl:br:einbettungskonst_M_e}
        M_{e} \deq \sup_{\substack{u\in W(a, b; V, V')\\u \neq 0}} \frac{\norm{u(0)}_{H}}{\norm{u}_{W(a, b; V, V')}}
    \end{equation}
    ist gleichmäßig beschränkt in der Wahl $V \hookrightarrow H$ und hängt nur im Fall $T \to 0$ von $T$ ab.

    \begin{Beweis}
        Siehe \textcite[Beweis zu Theorem XVIII.2.1]{Dautray:1992by}.
    \end{Beweis}
\end{Korollar}

% section bochner_r_ume (end)

\section{Lineare Evolutionsgleichungen} % (fold)
\label{sec:gl:le:lineare_evolutionsgleichungen}

In diesem Abschnitt werden lineare Evolutionsgleichungen, eine bestimmte Unterart parabolischer partieller Differentialgleichungen, eingeführt.
Diese Einführung orientiert sich an \textcite{Lions:1971wp,Schwab:2009ec,Urban:2014kg}.

Zunächst definieren wir, was wir unter dem Begriff einer linearen Evolutionsgleichungen verstehen wollen.
Anschließend leiten wir eine Raum-Zeit-Variationsformulierung her und geben einen Satz an, der unter geeigneten Voraussetzungen Existenz und Eindeutigkeit einer Lösung dieser Variationsformulierung garantiert.

Wir befinden uns nun im Folgenden Setting:
wie in \cref{def:gl:br:gelfand_tripel} seien $V$ und $H$ zwei separable Hilberträume und $(V, H, V')$ das daraus resultierende Gelfand-Tripel.
Wie zuvor verwenden wir $\skprod{\blank}{\blank}$ mit entsprechendem Index sowohl für die Skalarprodukte auf $V$ und $H$, als auch für die duale Paarung auf $V' \times V$.

Es sei $0 < T < \infty$ und damit $[0, T]$ ein endliches Zeitintervall.
Weiterhin sei für fast alle $t \in [0, T]$ eine Familie $a(\blank, \blank; t)$ von Bilinearformen
\begin{equation}
    \label{eq:gl:le:bilinearformen_familie}
    a(\blank, \blank; t) \colon V \times V \to \mathbb{R}, \quad (\eta, \zeta) \mapsto a(\eta, \zeta; t)
\end{equation}
gegeben.
Für alle $\eta, \zeta \in V$ sei die Abbildung $t \mapsto a(\eta, \zeta; t)$ messbar auf $[0, T]$ und erfülle die folgenden Bedingungen:

\begin{Annahme}
\label{ann:gl:le:bilinearform_eigenschaften}
    \leavevmode
    \begin{thmenumerate}
        \item \emph{Stetigkeit.}
        Es existiert eine Konstante $0 < M_{a} < \infty$ mit
        \begin{equation}
            \label{eq:gl:le:bilinearform_stetig}
            \abs{a(\eta, \zeta; t)} \leq M_{a} \norm{\eta}_{V} \norm{\zeta}_{V} \quad \fa \eta, \zeta \in V
        \end{equation}
        für fast alle $t \in [0, T]$.
        \item \emph{G\r{a}rding-Ungleichung}.
        Es existieren Konstanten $\alpha > 0$ und $\lambda \geq 0$ mit
        \begin{equation}
            \label{eq:gl:le:bilinearform_garding}
            a(\eta, \eta; t) + \lambda \norm{\eta}_{H}^{2} \geq \alpha \norm{\eta}_{V}^{2} \quad \fa \eta \in V
        \end{equation}
        für fast alle $t \in [0, T]$.
    \end{thmenumerate}
\end{Annahme}

Nach dem Rieszschen Darstellungssatz, vergleiche \cite[Theorem \S{}22.1]{Halmos:1957vd}, wird durch die Familie $a(\blank, \blank; t)$ für fast alle $t \in [0, T]$ durch
\begin{equation}
    \skprod{A(t)\eta}{\zeta}_{H} = a(\eta, \zeta; t)
\end{equation}
eine Familie stetiger linearer Operatoren $A(t) \in \mathcal L(V, V')$ induziert.

Mit dieser Vorarbeit können wir nun definieren, was wir unter einer linearen Evolutionsgleichung verstehen wollen.

\mfix{Quelle angeben!}
\begin{Definition}
\label{def:gl:le:lineare_evolutionsgleichung}
    Seien $a(\blank, \blank; t)$ und $A(t)$ wie oben gegeben.
    Weiterhin sei ein \emph{Quellterm} $g \in L_{2}(0, T; V')$ und ein \emph{Anfangswert} $u_{0} \in H$ gegeben.
    Als \emph{lineare Evolutionsgleichung} bezeichnen wir die parabolische partielle Differentialgleichung
    \begin{equation}
        \label{eq:gl:le:lineare_evolutionsgleichung}
        \begin{cases}
            u_{t}(t) + A(t) u(t) = g(t)     &\text{in}~V', \quad \text{für fast alle}~t \in I, \\
            u(0) = u_{0}                    &\text{in}~H.
        \end{cases}
    \end{equation}
\end{Definition}

\begin{Bemerkung}
\leavevmode
\begin{thmenumerate}
    \item Die Anfangswertbedingung $u(0) = u_{0}$ in $H$ ist wegen \cref{kor:gl:br:bochner_spur_wohldefiniert} wohldefiniert.
    \item Ist die Bilinearform $a(\blank, \blank; t)$ respektive der zugehörige stetige lineare Operator $A(t)$ unabhängig von $t \in [0, T]$, dann sprechen wir von einer \emph{autonomen} linearen Evolutionsgleichung.
\end{thmenumerate}
\end{Bemerkung}

Als nächstes leiten wir eine Raum-Zeit-Variationsformulierung für~\cref{eq:gl:le:lineare_evolutionsgleichung} her.
Dazu werden geeignete Ansatz- und Testfunktionenräume benötigt.
Hier kommen nun die in \cref{sec:gl:br:bochner_raeume} definierten Bochner-Räume zum Einsatz.

\begin{Definition}
\label{def:gl:le:ansatz_und_testraum}
    Als Ansatzfunktionenraum $\mathcal X$ bezeichnen wir den Raum $W(0, T; V, V')$ aus \cref{def:gl:br:bochner_raum_W}.
    Es ist also
    \begin{equation}
        \label{eq:gl:le:ansatzraum_X}
        \begin{aligned}
            \mathcal X &= L_{2}(0, T; V) \cap H^{1}(0, T; V')
            \\&= \Set*{ u \in L_{2}(0, T; V) \given u_{t} \in L_{2}(0, T; V') }
        \end{aligned}
    \end{equation}
    ausgestattet mit der Norm
    \begin{equation}
        \label{eq:gl:le:ansatzraum_X_norm}
        \norm{u}_{\mathcal X} = \left( \norm{u}_{L_{2}{(0, T; V)}}^{2} + \norm{u_{t}}_{L_{2}{(0, T; V')}}^{2} \right)^{1 / 2}, \quad u \in \mathcal X.
    \end{equation}
    Der Testfunktionenraum $\mathcal Y$ sei gegeben durch
    \begin{equation}
        \label{eq:gl:le:testraum_Y}
        \mathcal Y = L_{2}(0, T; V) \oplus H
    \end{equation}
    mit der Norm
    \begin{equation}
        \label{eq:gl:le:testraum_Y_norm}
        \norm{v}_{\mathcal Y} = \left( \norm{v_{1}}_{L_{2}(0, T; V)}^{2} + \norm{v_{2}}_{H}^{2} \right)^{1 / 2}, \quad v = (v_{1}, v_{2}) \in \mathcal Y.
    \end{equation}
\end{Definition}

Beide Räume sind Hilberträume, $\mathcal X$ nach \cref{lem:gl:br:bochner_W_ist_hilbertraum} und $\mathcal Y$ als direkte Summe zweier Hilberträume.

Um aus~\cref{eq:gl:le:lineare_evolutionsgleichung} eine Variationsformulierung zu erhalten, multiplizieren wir die lineare Evolutionsgleichung mit $v = (v_{1}, v_{2}) \in \mathcal Y$ und integrieren anschließend über das Zeitintervall $[0, T]$.
Dadurch ergibt sich folgende schwache Formulierung der linearen Evolutionsgleichung aus \cref{def:gl:le:lineare_evolutionsgleichung}:

\begin{Definition}
\label{def:gl:le:schwache_raum_zeit_formulierung}
    Seien $\mathcal X$ und $\mathcal Y$ wie in~\cref{eq:gl:le:ansatzraum_X} respektive~\cref{eq:gl:le:testraum_Y}.
    Als \emph{Raum-Zeit-Variations"-for"-mu"-lie"-rung} der linearen Evolutionsgleichung~\cref{eq:gl:le:lineare_evolutionsgleichung} bezeichnen wir das folgende Problem:

    Gegeben ein Quellterm $g \in L_{2}(0, T; V')$ und ein Anfangswert $u_{0} \in H$.
    Finde ein $u \in \mathcal X$ mit
    \begin{equation}
        \label{eq:gl:le:schwache_formulierung}
        b(u, v) = f(v) \quad \fa v \in \mathcal Y.
    \end{equation}
    Dabei ist $b \colon \mathcal X \times \mathcal Y \to \mathbb{R}$ die durch
    \begin{equation}
        \label{eq:gl:le:schwache_formulierung_lhs_b}
        b(u, v) = \int_{0}^{T} \skprod{u_{t}(t)}{v_{1}(t)}_{H} + a(u(t), v_{1}(t); t) \diff t + \skprod{u(0)}{v_{2}}_{H}
    \end{equation}
    definierte Bilinearform und $f \colon \mathcal Y \to \mathbb{R}$ das durch
    \begin{equation}
        \label{eq:gl:le:schwache_formulierung_rhs_f}
        f(v) = \int_{0}^{T} \skprod{g(t)}{v_{1}(t)}_{H} \diff t + \skprod{u_{0}}{v_{2}}_{H}
    \end{equation}
    gegebene Funktional.
\end{Definition}

Es bleibt nun zu zeigen, welche Bedingungen hinreichend sind, damit obige Raum-Zeit-Variationsformulierung \emph{sachgemäß gestellt} ist.
Dazu definieren wir zunächst, was wir unter dem Begriff \enquote{sachgemäß gestellt} verstehen wollen und greifen dazu auf die Definition nach \textcite{hadamard1902problemes} zurück.

\begin{Definition}
\label{def:gl:le:hadamard_sachgemaess_gestellt}
    Seien $U$ und $V$ normierte Vektorräume.
    Seien weiter eine stetige Bilinearform $a \in \mathcal L(U \times V, \mathbb{R})$ und ein stetiges Funktional $f \in \mathcal L(V)$ gegeben.
    Betrachte das folgende Problem: finde ein $u \in U$, so dass
    \begin{equation}
    \label{eq:gl:le:hadamard_variationsproblem}
        a(u, v) = f(v) \quad \fa v \in V
    \end{equation}
    gilt.
    Wir nennen \cref{eq:gl:le:hadamard_variationsproblem} \emph{sachgemäß gestellt}, wenn eine eindeutige Lösung $u \in U$ existiert und diese eine A-priori-Abschätzung der Form
    \begin{equation}
    \label{eq:gl:le:hadamard_abschaetzung}
        \norm{u}_{U} \leq c \norm{f}_{V'} \quad \fa f \in V'
    \end{equation}
    und eine von $f$ unabhängige Konstante $c > 0$ erfüllt.
\end{Definition}

Um dies für die Raum-Zeit-Variationsformulierung aus \cref{def:gl:le:schwache_raum_zeit_formulierung} nachzuweisen, werden wir den folgenden bekannten und wichtigen Satz, welcher in dieser oder ähnlicher Form bei \textcites[Theorem 2.1]{Babuska:1971fx}[Theorem 5.2.1]{Aziz:2014wf}[Theorem \S{}3.3.6]{Braess:2007wm} zu finden ist, verwenden.

\begin{Satz}[Banach-Ne{\v c}as-Babu{\v s}ka-Theorem]
\label{satz:gl:le:bnb_theorem}
    Seien $U$ und $V$ Hilberträume.
    Eine lineare Abbildung $A \colon U \to V'$ ist genau dann ein Isomorphismus, das heißt stetig invertierbar, wenn die zugehörige Bilinearform $a \colon U \times V \to \mathbb{R}$ die folgenden Bedingungen erfüllt:
    \begin{thmenumerate}
        \item \label{satz:gl:le:bnb_theorem_bedingung_stetig}
        \emph{Stetigkeit}.
        Es existiert ein $0 < C < \infty$ mit
        \begin{equation}
            \abs{a(u, v)} \leq C \norm{u}_{U} \norm{v}_{V} \quad \fa u \in U,~v\in V.
        \end{equation}
        \item \label{satz:gl:le:bnb_theorem_bedingung_inf_sup}
        \emph{Inf-sup-Bedingung}.
        Es existiert ein $\alpha > 0$ mit
        \begin{equation}
            \inf_{u \in U} \sup_{v \in V} \frac{a(u, v)}{\norm{u}_{U} \norm{v}_{V}} \geq \alpha.
        \end{equation}
        \item \label{satz:gl:le:bnb_theorem_bedingung_surjektiv}
        Zu jedem $v \in V$, $v \neq 0$, existiert ein $u \in U$ mit
        \begin{equation}
            a(u, v) \neq 0.
        \end{equation}
    \end{thmenumerate}
    Ist dies der Fall und ist weiter ein Funktional $f \in V'$ gegeben, dann existiert eine eindeutige Lösung $\hat u \in U$ mit
    \begin{equation}
        a(\hat u, v) = f(v) \quad \fa v \in V
    \end{equation}
    und es gilt
    \begin{equation}
        \norm{\hat u}_{U} \leq \frac{1}{\alpha} \norm{f}_{V'}.
    \end{equation}
\end{Satz}

\begin{Bemerkung}
\label{bem:gl:le:bnb_staerkere_voraussetzungen}
    Die letzte Bedingung im vorherigen Satz kann durch eine stärkere inf-sup-Bedingung ersetzt werden, denn existiert ein $\beta > 0$ mit
    \begin{equation}
        \inf_{v \in V} \sup_{u \in U} \frac{a(u, v)}{\norm{u}_{U} \norm{v}_{V}} \geq \beta,
    \end{equation}
    dann gilt insbesondere auch \ref{satz:gl:le:bnb_theorem_bedingung_surjektiv}.
\end{Bemerkung}

Aus dem Banach-Ne{\v c}as-Babu{\v s}ka-Theorem lässt sich auch das bekannte Lax-Milgram-Lemma ableiten, weswegen wir es hier kurz wiederholen wollen.

\begin{Lemma}[Lax-Milgram]
\label{lem:gl:le:lax_milgram}
    Sei $V$ ein Hilbertraum, seien weiter $a \in \mathcal L(V \times V, \mathbb{R})$ eine stetige Bilinearform und $f \in \mathcal L (V, \mathbb{R})$ ein stetiges Funktional und betrachte das Problem:
    finde ein $u \in V$, so dass
    \begin{equation}
        a(u, v) = f(v) \quad \fa v \in V
    \end{equation}
    gilt.
    Ist die Bilinearform $a$ elliptisch, das heißt, es existiert ein $\alpha > 0$ mit
    \begin{equation}
    \label{eq:gl:le:lax_milgram_elliptisch}
        a(u, u) \geq \alpha \norm{u}_{V}^{2} \quad \fa u \in V,
    \end{equation}
    dann ist das Problem sachgemäß gestellt und es gilt
    \begin{equation}
        \norm{u}_{V} \leq \frac{1}{\alpha} \norm{f}_{V'}.
    \end{equation}

    \begin{Beweis}
        Aus der Elliptizität können wir eine inf-sup-Bedingung folgern, denn es gilt
        \begin{equation}
            \sup_{v \in V} \frac{a(u, v)}{\norm{v}_{V}} \geq \frac{a(u, u)}{\norm{u}_{V}} \geq \alpha \norm{u}_{V}.
        \end{equation}
        Damit sind neben der Stetigkeit von $a$ auch die die Bedingungen \ref{satz:gl:le:bnb_theorem_bedingung_inf_sup} und \ref{satz:gl:le:bnb_theorem_bedingung_surjektiv} aus dem Banach-Ne{\v c}as-Babu{\v s}ka-Theorem, \cref{satz:gl:le:bnb_theorem}, erfüllt und es folgt die Aussage.
    \end{Beweis}
\end{Lemma}

Für das Raum-Zeit-Variationsproblem aus \cref{def:gl:le:schwache_raum_zeit_formulierung} lässt sich die Tatsache, dass es sachgemäß gestellt ist, zu folgendem Satz zusammenfassen.

\begin{Satz}[{{\cite[Theorem 5.1]{Schwab:2009ec}}}]
\label{satz:gl:le:ss09_theorem51}
    Seien $\mathcal X$ und $\mathcal Y$ wie in~\cref{eq:gl:le:ansatzraum_X} respektive~\cref{eq:gl:le:testraum_Y}.
    Sei weiter $B \colon \mathcal X \to \mathcal Y'$ definiert durch
    \begin{equation}
        \label{eq:gl:le:ss09_variationsproblem_als_operatorgleichung}
        \skprod{B u}{v}_{\mathcal Y' \times \mathcal Y} = b(u, v), \quad u \in \mathcal X,~v \in \mathcal Y,
    \end{equation}
    wobei $b$ die Bilinearform aus~\cref{eq:gl:le:schwache_formulierung_lhs_b} sei.
    Dann ist $B$ stetig invertierbar.

    \begin{Beweis}
        Ein ausführlicher Beweis, in dem die Bedingungen des Banach-Ne\v{c}as-Babu\v{s}ka-Theorems nachgewiesen werden, ist bei \textcite[Appendix A]{Schwab:2009ec} zu finden.
        % \begin{Beweis}
        %     Wir weisen die Bedingungen von \cref{satz:babuska-aziz} nach.

        %     Zunächst sei anzumerken, dass wir in~\cref{eq:garding-inequality} ohne Einschränkung $\lambda = 0$ wählen können.
        %     Wähle
        %     \begin{equation}
        %         u(t) = \hat u(t) e^{\lambda t}, \quad v_{1}(t) = \hat v_{1}(t) e^{- \lambda t}, \quad g(t) = \hat g(t) e^{\lambda t},
        %     \end{equation}
        %     dann sieht man, dass $u$ die Gleichung~\cref{eq:bilinearform} genau dann löst, wenn $\hat u$ die Gleichung
        %     \begin{equation}
        %         \label{eq:bilinearform_tmp}
        %         \begin{gathered}
        %             \int_{I} \skprod{\hat{u}_{t}(t)}{\hat{v}_{1}(t)}_{H} + \lambda \skprod{\hat{u}(t)}{\hat{v}_{1}(t)}_{H} + a(t; \hat{u}(t), \hat{v}_{1}(t)) \diff t + \skprod{\hat{u}(0)}{v_{2}}_{H}
        %                 \\= \int_{I} \skprod{\hat{g}(t)}{\hat{v}_{1}(t)}_{H} \diff t + \skprod{u_{0}}{v_{2}}_{H}
        %         \end{gathered}
        %     \end{equation}
        %     für alle $\hat{v} = (\hat{v}_{1}, v) \in \mathcal Y$ löst.

        %     \paragraph{Stetigkeit} % (fold)
        %     \label{par:stetigkeit}
        %     Betrachte für $u \in \mathcal X$ und $v = (v_{1}, v_{2}) \in \mathcal Y$ die Bilinearform $b(u, v)$.
        %     Nach Anwenden der Dreiecksungleichung erhalten wir
        %     \begin{equation}
        %         \label{eq:stetigkeit_zweiter_term}
        %         \abs{b(u, v)} = \int_{I} \abs{\skprod{u_{t}(t)}{v_{1}(t)}_{H}} + \abs{a(u(t), v_{1}(t))} \diff t + \abs{\skprod{u(0)}{v_{2}}_{H}}.
        %     \end{equation}
        %     Betrachten wir zunächst den hinteren Term, dann erhalten wir unter Verwendung der Cauchy-Schwarz-Ungleichung und der Einbettungs-Konstante $M_{e}$ die Abschätzung
        %     \begin{equation}
        %         \abs{\skprod{u(0)}{v_{2}}_{H}} \leq \norm{u(0)}_{H} \norm{v_{2}}_{H} \leq M_{e} \norm{u}_{X} \norm{v_{2}}_{H}.
        %     \end{equation}
        %     Widmen wir uns nun dem ersten Term und wenden ebenfalls die Cauchy-Schwarz-Ungleichung sowie die Stetigkeit von $a$ an, dann erhalten wir
        %     \begin{align}
        %         &\int_{I} \abs{\skprod{u_{t}(t)}{v_{1}(t)}_{H}} + \abs{a(u(t), v_{1}(t))} \diff t
        %         \\&\qquad
        %         \leq \int_{I} \norm{u_{t}(t)}_{H} \norm{v_{1}(t)}_{H} + M_{a} \norm{u(t)}_{H} \norm{v_{1}(t)}_{H} \diff t
        %         \\&\qquad
        %         \leq \int_{I} \max\{1, M_{a}\} \norm{v_{1}(t)}_{H} \left(  \norm{u_{t}(t)}_{H} + \norm{u(t)}_{H} \right) \diff t
        %         \intertext{mittels Hölder-Ungleichung lässt sich dies weiter abschätzen zu}
        %         &\qquad
        %         \leq \left( \int_{I} \max\{1, M_{a}\}^{2} \norm{v_{1}(t)}_{H}^{2} \diff t \right)^{\frac 12} \left( \int_{I} \left( \norm{u_{t}(t)}_{H} + \norm{u(t)}_{H} \right)^{2} \diff t \right)^{\frac 12},
        %         \intertext{und unter Verwendung der Youngschen-Ungleichung zu}
        %         &\qquad
        %         \leq \left( \int_{I} \max\{1, M_{a}\}^{2} \norm{v_{1}(t)}_{H}^{2} \diff t \right)^{\frac 12} \left( \int_{I} 2 \left( \norm{u_{t}(t)}_{H}^{2} + \norm{u(t)}_{H}^{2} \right) \diff t \right)^{\frac 12}
        %         \intertext{was nach Definition der verwendeten Normen auch geschrieben werden kann als}
        %         &\qquad
        %         = \sqrt{2 \max\{1, M_{a}^{2}\}} \norm{u}_{\mathcal X} \norm{v_{1}}_{L_{2}(I; V)}
        %     \end{align}
        %     Zusammen mit~\cref{eq:stetigkeit_zweiter_term} liefert dies nach einer erneuten Anwendung der Cauchy-Schwarz-Ungleichung
        %     \begin{align}
        %     \abs{b(u, v)}
        %     &\leq \sqrt{2 \max\{1, M_{a}\}^{2}} \norm{u}_{\mathcal X} \norm{v_{1}}_{L_{2}(I; V)} + M_{e} \norm{u}_{X} \norm{v_{2}}_{H}
        %     \\
        %     &\leq \norm{u}_{\mathcal X} \left( \norm{v_{1}}_{L_{2}(I; V)}^{2} + \norm{v_{2}}_{H}^{2} \right)^{\frac 12} \left( 2 \max\{1, M_{a}\}^{2} + M_{e}^{2} \right)^{\frac 12}
        %     \\
        %     &= \sqrt{2 \max\{1, M_{a}^{2}\} + M_{e}^{2}} \norm{u}_{\mathcal X} \norm{v}_{\mathcal Y}.
        %     \end{align}
        %     Damit folgt die Stetigkeit.
        %     % paragraph stetigkeit (end)

        %     \paragraph{Inf-Sup-Bedingung} % (fold)
        %     \label{par:inf_sup_bedingung}

        %     % paragraph inf_sup_bedingung (end)
    \end{Beweis}
\end{Satz}

Aus dem Beweis des vorherigen Satzes bei \textcite{Schwab:2009ec} ergeben sich zugleich auch Abschätzungen für die Operatornormen von $B$ und $B^{-1}$.

\mdo{Ungleichung der stetigen Abhängigkeit einbauen. Wat? Was wollte ich mir damit sagen? ...}

\begin{Korollar}
\label{kor:gl:le:ss09_theorem51_ungleichungen}
    Unter der Gegebenheiten aus \cref{satz:gl:le:ss09_theorem51} und der Bedingung, dass die Bilinearform $a$ die G\aa{}rding-Ungleichung mit $\lambda = 0$ erfüllt, gilt
    \begin{equation}
        \norm{B}_{\mathcal L(\mathcal X, \mathcal Y')} \leq \sqrt{2 \max\Set{1, M_{a}^{2}} + M_{e}^{2}}
    \end{equation}
    und
    \begin{equation}
        \norm{B^{-1}}_{\mathcal L(\mathcal Y', \mathcal X)} \leq \frac{\sqrt{2 \max \Set{\alpha^{-2}, 1} + M_{e}^{2}}}{\min\Set{\alpha M_{a}^{-2}, \alpha}}.
    \end{equation}

    Ist $\lambda \neq 0$, dann gilt ferner
    \begin{equation}
        \label{kor:gl:le:norm_B_abschaetzung}
        \norm{B}_{\mathcal L(\mathcal X, \mathcal Y')} \leq \frac{\sqrt{2\max\Set{1, M_{a}^{2}} + M_{e}^{2}}}{\max\Set{\sqrt{1 + 2 \lambda^{2} \rho^{4}}, \sqrt{2}}}
    \end{equation}
    und
    \begin{equation}
        \label{kor:gl:le:norm_B_inv_abschaetzung}
        \norm{B^{-1}}_{\mathcal L(\mathcal Y', \mathcal X)} \leq e^{2 \lambda T} \max\Set{\sqrt{1 + 2 \lambda^{2} \rho^{4}}, \sqrt{2}}  \frac{\sqrt{2 \max\Set{ \alpha^{-2}, 1} + M_{e}^{2}}}{\min\Set{\alpha M_{a}^{-2}, \alpha}}.
    \end{equation}

    Die Größen $M_{a}$, $\alpha$ und $\lambda$ stammen aus \cref{ann:gl:le:bilinearform_eigenschaften},
    während die Konstanten $M_{e}$ und $\rho$ als die Einbettungskonstanten

    \begin{equation}
        \label{kor:gl:le:einbettungskonstante_M_e}
        M_{e} = \sup_{0 \neq u \in \mathcal X} \frac{\norm{u(0)}_{H}}{\norm{u}_{\mathcal X}},
    \end{equation}
    vergleiche \cref{kor:gl:br:einbettungskonstante_M_e}, beziehungsweise
    \begin{equation}
        \label{kor:gl:le:einbettungskonstante_rho}
        \rho = \sup_{0 \neq \eta \in V} \frac{\norm{\eta}_{H}}{\norm{\eta}_{V}}
    \end{equation}

    \begin{Beweis}
        $\,$\newline
        \mwarn{Will ich das beweisen?}
    \end{Beweis}
\end{Korollar}

% section lineare_evolutionsgleichungen (end)

% \section{Homogenisierung} % (fold)
% \label{sec:homogenisierung}

% \todo[inline]{Ist das nützlich?}
% \todo[inline]{Quelle!}
% Die im vorherigen Abschnitt hergeleitete inhomogene Raum-Zeit-Variationsformulierung lässt sich zu einer äquivalenten homogenen Raum-Zeit-Variationsformulierung umformen.

% Dazu definieren wir erneut zunächst die benötigten Ansatz- und Testfunktionenräume und anschließend das darauf aufbauende homogene Raum-Zeit-Variationsproblem.

% \begin{Definition}
% \label{def:gl:le:ansatz_und_testraum_homogen}
%     Der Ansatzfunktionenraum $\hat{\mathcal X}$ sei als der folgende Unterraum von $W(0, T; V, V')$
%     \begin{equation}
%         \label{eq:gl:le:ansatzraum_X_homogen}
%         \begin{aligned}
%             \hat{\mathcal X} &= L_{2}(0, T; V) \cap H^{1}_{\{0\}}(0, T; V')
%             \\&= \Set*{ u \in L_{2}(0, T; V) \given u_{t} \in L_{2}(0, T; V'),~u(0) = 0 }
%         \end{aligned}
%     \end{equation}
%     definiert, ausgestattet mit der bekannten Norm
%     \begin{equation}
%         \label{eq:gl:le:ansatzraum_X_norm_homogen}
%         \norm{u}_{\hat{\mathcal X}} = \left( \norm{u}_{L_{2}{(0, T; V)}}^{2} + \norm{u_{t}}_{L_{2}{(0, T; V')}}^{2} \right)^{1 / 2}, \quad u \in \hat{\mathcal X}.
%     \end{equation}
%     Der zugehörige Testfunktionenraum $\hat{\mathcal Y}$ sei gegeben durch
%     \begin{equation}
%         \label{eq:gl:le:testraum_Y_homogen}
%         \hat{\mathcal Y} = L_{2}(0, T; V)
%     \end{equation}
%     mit der üblichen Norm
%     \begin{equation}
%         \label{eq:gl:le:testraum_Y_norm_homogen}
%         \norm{v}_{\hat{\mathcal Y}} = \norm{v_{1}}_{L_{2}(0, T; V)}, \quad v \in \hat{\mathcal Y}.
%     \end{equation}
% \end{Definition}

% \begin{Definition}
% \label{def:gl:le:schwache_raum_zeit_formulierung_homogen}
%     Seien $\hat{\mathcal X}$ und $\hat{\mathcal Y}$ wie in~\cref{eq:gl:le:ansatzraum_X_homogen} respektive~\cref{eq:gl:le:testraum_Y_homogen}.
%     Als \emph{homogenisierte Raum-Zeit-Variations"-for"-mu"-lie"-rung} der linearen Evolutionsgleichung~\cref{eq:gl:le:lineare_evolutionsgleichung} bezeichnen wir das folgende Problem:

%     Gegeben ein Quellterm $g \in L_{2}(0, T; V')$ und ein Anfangswert $u_{0} \in H$, sodass ein $\hat u_{0} \in \mathcal X$ existiert, welches $\hat{u}_{0}(0) = u_{0}$ erfüllt.
%     Finde ein $u \in \hat{\mathcal X}$ mit
%     \begin{equation}
%         \label{eq:gl:le:schwache_formulierung_homogen}
%         \hat{b}(u, v) = \hat{f}(v) \quad \fa v \in \hat{\mathcal Y}.
%     \end{equation}
%     Dabei ist $\hat{b} \colon \hat{\mathcal X} \times \hat{\mathcal Y} \to \mathbb{R}$ die durch
%     \begin{equation}
%         \label{eq:gl:le:schwache_formulierung_lhs_b_homogen}
%         \hat{b}(u, v) = \int_{0}^{T} \skp{u_{t}(t)}{v(t)}{V' \times V} + a(u(t), v(t); t) \diff t
%     \end{equation}
%     definierte Bilinearform und $\hat{f} \colon \hat{\mathcal Y} \to \mathbb{R}$ das durch
%     \begin{equation}
%         \label{eq:gl:le:schwache_formulierung_rhs_f_homogen}
%         \hat{f}(v) = \int_{0}^{T} \skp{\hat g(t)}{v(t)}{V' \times V} \diff t
%     \end{equation}
%     gegebene Funktional,
%     wobei $\hat{g} \in $ als
%     \begin{equation}
%         \hat{g} = g(t) - \left[ (\hat u_{0})_{t}(t) + A(t) \hat u_{0}(t) \right]
%     \end{equation}
%     definiert sei.
% \end{Definition}

% Das es sich hierbei tatsächlich um eine Homogenisierung handelt, liefert der folgende Satz.

% \begin{Satz}
%     Sei $H = L_{2}(\Omega)$ für ein beschränktes Gebiet $\Omega \in \mathbb{R}^{n}$.
%     Dann gilt: $\hat u \in \tilde{\mathcal X}$ ist genau dann eine Lösung des homogenisierten RZVP aus \cref{def:gl:le:schwache_raum_zeit_formulierung_homogen}, wenn $u := \hat u + \hat u_{0} \in \mathcal X$ eine Lösung  des RZVP aus \cref{def:gl:le:schwache_raum_zeit_formulierung} ist.

%     \todo[inline]{Beweis}
%     % \begin{Beweis}
%     %     \emph{Hinrichtung.} Sei $\hat u \in \tilde{\mathcal X}$ eine Lösung, dann gilt
%     %     \begin{equation}
%     %         u(0) = \hat u(0) + \hat u_{0}(0) = 0 + u_{0} = u_{0}.
%     %     \end{equation}
%     %     Seien nun $v = (v_{1}, v_{2}) \in \mathcal Y$ beliebig.
%     %     Dann gilt
%     %     \begin{align}
%     %         b(u, v)
%     %           &= b(u, (v_{1}, v_{2}))
%     %         \\&= \hat b(u, v_{1}) + \skp{u(0)}{v_{2}}{H}
%     %         \\&= \hat b(\hat u, v_{1}) + \hat b(\hat u_{0}, v_{1}) + \skp{u(0)}{v_{2}}{H}
%     %         \\&= \hat f(v_{1}) + \int_{0}^{T} \skp{\hat u_{0})_{t}(t) + A(t) \hat u_{0}(t)}{V' \times V} + \skp{u(0)}{v_{2}}{H}
%     %         \\&= \int_{0}^{T} \skp{\hat g(t) + \hat u_{0})_{t}(t) + A(t) \hat u_{0}(t)}{v_{1}(t)}{V' \times V} + \skp{u(0)}{v_{2}}{H}
%     %         \\&= \int_{0}^{T} \skp{g(t)}{v_{1}(t)}{V' \times V} + \skp{u(0)}{v_{2}}{H}
%     %         \\&= f(v).
%     %     \end{align}

%     %     \emph{Rückrichtung.} Sei $u \in \mathcal X$ eine Lösung.
%     %     Dann gilt
%     %     \begin{equation}
%     %             \hat u(0) = u(0) - \hat u_{0}(0) = u_{0} - u_{0} = 0.
%     %     \end{equation}
%     %     \todo[inline]{Bla?}
%     %     Damit ist insbesondere $\hat u \in \tilde{\mathcal X}$.
%     %     Sei nun $v_{1} \in \tilde{\mathcal Y}$ beliebig und $v_{2} \in H$, dann gilt
%     %     \begin{align}
%     %         \hat b(u, v_{1})
%     %           &= \hat b(u - \hat u_{0}, v_{1})
%     %         \\&= (b(u, (v_{1}, v_{2})) - \skp{u(0)}{v_{2}}{H}) - (b(\hat u_{0}, (v_{1}, v_{2})) - \skp{\hat u_{0}(0)}{v_{2}}{H})
%     %         \\&= f(v_{1}, v_{2}) - b(\hat u_{0}, (v_{1}, v_{2}) - \skp{u(0) - \hat u_{0}(0)}{v_{2}}{H}
%     %         \\&= f(v_{1}, v_{2}) - b(\hat u_{0}, (v_{1}, v_{2}) - \skp{u(0)
%     %         \\&= \int_{0}^{T} \skp{g(t)}{v_{1}(t)}{V' \times V} \diff t + \skp{u_{0}}{v_{2}}{H}
%     %                 - \int_{0}^{T} \skp{(\hat u_{0})_{t}(t)}{v_{1}(t)}{V' \times V} + a(\hat u_{0}(t), v_{1}(t); t) \diff t - \skp{u_{0}}{v_{2}}_{H}
%     %         \\&= \hat f(v_{1}).
%     %     \end{align}
%     %     Damit ist die Äquivalenz gezeigt.
%     % \end{Beweis}
% \end{Satz}

% \begin{Bemerkung}
%     Ist $u_{0} \in V$ gegeben, dann erhalten wir $\hat{u}_{0}$ via $\hat{u}_{0} = 1 \otimes u_{0}$, denn dann gilt
%     $\hat{u}_{0}(0) = u_{0}$ und $\hat{u}_{0} \in \mathcal X$.
% \end{Bemerkung}

% section homogenisierung (end)

\section{Galerkin-Verfahren} % (fold)
\label{sec:gl:gv:galerkin_verfahren}

\mdo{Abschnitt sauber aufziehen, d.h. Formulierungen verbessern, eventuell ausführlicher.}

Wir wiederholen nun in Grundzügen das bekannte Galerkin-Verfahren, konzentrieren uns dabei aber auf das sogenannte \emph{Petrov-Galerkin-Verfahren}, welches eine Verallgemeinerung auf nicht-symmetrische Bilinearformen innerhalb der Variationsformulierung darstellt.

Seien $U$ und $V$ zwei Hilberträume, $a \colon U \times V \to \mathbb{R}$ eine Bilinearform und $f \colon V \to \mathbb{R}$ ein lineares Funktional.
Wir betrachten das folgende abstrakte Variationsproblem:
Finde ein $u \in U$, so dass
\begin{equation}
    \label{eq:gl:gv:variationsproblem}
    a(u, v) = f(v) \quad \fa v \in V
\end{equation}
gilt.

Unter gewissen Voraussetzungen, zum Beispiel, wenn die Bilinearform $a$ die Bedingungen des Banach-Ne{\v c}as-Babu{\v s}ka-Theorems (\cref{satz:gl:le:bnb_theorem}) erfüllt, existiert eine eindeutige Lösung $u \in U$ des obigen Variationsproblems.

Bei den Räumen $U$ und $V$ handelt es sich im Allgemeinen um unendlichdimensionale Hilberträume, wodurch eine numerische Bestimmung der Lösung $u \in U$ nicht ohne Weiteres möglich ist.
Um eine approximative Lösung von \cref{eq:gl:gv:variationsproblem} zu bestimmen, muss zunächst eine Diskretisierung durchgeführt werden.
Diese wird bei Petrov-Galerkin-Verfahren dadurch erreicht, dass für Ansatz- und Testfunktionenraum endlichdimensionale Unterräume $U_{n} \subset U$ mit $\dim U_{n} = n$ respektive $V_{m} \subset V$ mit $\dim V_{m} = m$ konstruiert werden.

Statt dem obigen Variationsproblem \cref{eq:gl:gv:variationsproblem} betrachtet man dann:
Finde ein $u_{n} \in U_{n}$, so dass
\begin{equation}
    \label{eq:gl:gv:variationsproblem_diskret}
    a_{n,m}(u_{n}, v_{m}) = f_{m}(v_{m}) \quad \fa v_{m} \in V_{m}
\end{equation}
gilt, wobei $a_{n, m} \colon U_{n} \times V_{m} \to \mathbb{R}$ die Einschränkung $a_{n,m} = \restr{a}{U_{n} \times V_{m}}$ beziehungsweise $f_{m} \colon V_{m} \to \mathbb{R}$ die Einschränkung $f_{m} = \restr{f}{V_{m}}$ bezeichne.

Zu Beachten ist, dass selbst wenn das eigentliche Variationsproblem \cref{eq:gl:gv:variationsproblem} eindeutig lösbar ist, dies noch nicht für das diskretisierte Variationsproblem \cref{eq:gl:gv:variationsproblem_diskret} gelten muss.
Auch muss im Falle einer existierenden Lösung $u_{n} \in U_{n}$ diese im Allgemeinen für $n \to \infty$ nicht gegen die Lösung $u \in U$ von \cref{eq:gl:gv:variationsproblem} konvergieren.

\mdo{Quellen prüfen}
Der folgende Satz, welcher in verschiedenen Varianten bei \cites[Theorem 6.2.1]{Aziz:2014wf}[Theorem 5.3.1]{Quarteroni:2009hp}[Lemma III.3.7]{Braess:2007wm} zu finden ist, liefert hinreichende Bedingungen für die eindeutige Lösbarkeit von \cref{eq:gl:gv:variationsproblem_diskret} und zudem eine A priori-Fehlerabschätzung.

\begin{Satz}
    Seien $U$ und $V$ zwei Hilberträume und $U_{n} \subset U$, respektive $V_{m} \subset V$ Unterräume.
    Weiter seien $a \colon U \times V \to \mathbb{R}$ eine Bilinearform und $f \colon V \to \mathbb{R}$ ein lineares Funktional, wobei $a$ die Bedingungen aus \cref{satz:gl:le:bnb_theorem} erfülle.
    Bezeichne ferner mit $a_{n, m} = \restr{a}{U_{n} \times V_{m}}$ respektive $f_{m} = \restr{f}{V_{m}}$ die Einschränkungen von $a$ beziehungsweise $f$ auf die Unterräume $U_{n}$ und $V_{m}$.

    Erfüllt $a_{n, m}$ die Bedingungen \ref{satz:gl:le:bnb_theorem_bedingung_inf_sup} und \ref{satz:gl:le:bnb_theorem_bedingung_surjektiv} aus \cref{satz:gl:le:bnb_theorem} mit der inf-sup-Konstante $\alpha_{n, m} > 0$ statt $\alpha$, dann existiert eine eindeutige Lösung $u_{n} \in U_{n}$ des Variationsproblems \cref{eq:gl:gv:variationsproblem_diskret} und diese erfüllt
    \begin{equation}
        \norm{u_{n}}_{U} \leq \frac{1}{\alpha_{n, m}} \norm{f_{m}}_{V_m'}.
    \end{equation}

    Ist weiter $u \in U$ eine Lösung des Variationsproblems \cref{eq:gl:gv:variationsproblem}, dann gilt
    \begin{equation}
        \norm{u - u_{n}} \leq \left( 1 + \frac{C}{\alpha_{n, m}} \right)  \inf_{w_{n} \in U_{n}} \norm{u - w_{n}}.
    \end{equation}

    \begin{Beweis}$\;$
        \mfix{Nötig?}
    \end{Beweis}
\end{Satz}

Um nun tatsächlich eine Lösung $u_{n} \in U_{n}$ zu bestimmen, verfährt man wie folgt:
Seien $\Set{ \varphi_{i} \in U \given i = 1 \dots n }$ und $\Set{ \psi_{j} \in V \given j = 1 \dots m}$ jeweils eine Basis von $U_{n}$ respektive $V_{m}$.

Schreibt man nun die Ansatz- beziehungsweise Testfunktionen als Linearkombination
\begin{equation}
    u_{n} = \sum_{i = 1}^{n} \vec{u}_{i} \varphi_{i}, \quad v_{m} = \sum_{j = 1}^{m} \vec{v}_{j} \psi_{j}, \qquad \vec u \in \mathbb{R}^{n},~\vec v \in \mathbb{R}^{m},
\end{equation}
dieser Basisfunktionen und setzt diese in das Variationsproblem \cref{eq:gl:gv:variationsproblem_diskret} ein, dann führt dies unmittelbar zu dem linearen Gleichungssystem $\mat{A} \vec{u} = \vec{f}$, wobei wir
\begin{equation}
    \mat A_{ji} = a_{n, m}(\varphi_{j}, \psi_{i}), \quad i = 1 \dots n,~j = 1 \dots m,
\end{equation}
als \emph{Steifigkeitsmatrix} $\mat A \in \mathbb{R}^{m \times n}$, und
\begin{equation}
    \vec f_{j} = f_{m}(\psi_{j}), \quad j = 1 \dots m,
\end{equation}
als \emph{Lastvektor} $\vec f \in \mathbb{R}^{m}$ bezeichnen.
Besitzt das lineare Gleichungssystems eine Lösung $\vec u \in \mathbb{R}^{n}$, dann erhält man die zugehörige Lösung $u_{n} \in U_{n}$ des obigen Variationsproblems \cref{eq:gl:gv:variationsproblem_diskret} durch die Linearkombination
\begin{equation}
    u_{n} = \sum_{i = 1}^{n} \vec{u}_{i} \varphi_{i}.
\end{equation}

% chapter grundlagen (end)



    % -*- root: ../main.tex -*-

\documentclass[../main.tex]{subfiles}
\begin{document}

\chapter{Propagator-Differentialgleichung} % (fold)
\label{chapter:propagator_differentialgleichung}

Wir greifen nun die in \cref{chapter:einleitung} eingeführten parabolischen Differentialgleichungen \cref{eq:forward_propagator,eq:backward_propagator} auf, welche von den beiden Propagatoren $q$ und $q^{\dagger}$ erfüllt werden.
Im weiteren Verlauf der Arbeit verwenden wir den Begriff \emph{Propagator"=Differentialgleichung}, den wir noch genauer definieren werden, wenn wir uns
auf die genannten partiellen Differentialgleichungen beziehen.

In diesem Kapitel konkretisieren wir zunächst diese Propagator"=Differentialgleichungen, schaffen geeignete Rahmenbedingungen und leiten eine schwache Raum"=Zeit"=Formulierung her.
Anschließend werden die in den Propagator"=Differentialgleichungen auftretenden Felder $\omega_{i}$ parametrisiert und als Grundlage für eine parametrische schwache Formulierung verwendet.
Für letztere weisen wir abschließend nach, dass sie korrekt gestellt ist und unter zusätzlichen Bedingungen eine gewisse Regularität bezüglich der Parameter aufweist.


\section{Eine Raum-Zeit-Variationsformulierung}
\label{section:raum_zeit_variationsformulierung}

Wir wollen nun zunächst die aus der Einführung bekannten Propagator"=Differentialgleichungen in einem allgemeineren Rahmen formulieren.
Dabei halten wir an dem Fall zweier Felder fest, wobei diese Einschränkung nicht notwendig ist, da die nachfolgenden Ergebnisse in gleicher Weise auch für jede andere endliche Felderanzahl nachgewiesen werden können.
Weiter sei an dieser Stelle angemerkt, dass es ausreicht sich auf die Betrachtung des Vorwärts"=Propagators \cref{eq:forward_propagator} zu beschränken, da der Rückwärts"=Propagator \cref{eq:backward_propagator} durch die einfache Transformation $s \mapsto 1 - s$ auf dieselbe Form, lediglich mit vertauschen Rollen bei den Feldern, gebracht werden kann.

Seien nun $0 < T_{f} < T < \infty$ reelle Konstanten und $I := [0, T]$ ein reelles Intervall, welches wir in die beiden disjunkten Teilintervalle $I_{1} := [0, T_{f})$ und $I_{2} := [T_{f}, T]$ zerlegen.
Wir interpretieren die Größe $t \in I$ als Zeit, wobei dies einen rein notationellen und keinen physikalischen Hintergrund hat.
Weiter sei $\Omega \subset \mathbb{R}^{n}$, $n \in \mathbb{N}$, ein beschränktes Gebiet, das heißt offen, nichtleer, zusammenhängend und beschränkt, welches einen Lipschitz"=Rand besitzt.

Zunächst wollen wir den Begriff der Felder und der Propagator-Differentialgleichung konkretisieren.
Dies dient vor allem der einfacheren Notation und Benennung, weswegen wir den definierten Abbildungen an dieser Stelle möglichst wenige einschränkende Bedingungen auferlegen wollen.

\begin{Definition}
\label{definition:feld_raum_zeit_feld}
    Unter einem \emph{Feld} verstehen wir eine $L_{\infty}(\Omega)$-Abbildung $w$.
    Seien $w_{1}, w_{2}$ Felder und $\chi_{I_{1}}, \chi_{I_{2}}$ charakteristische Funktionen.
    Wir bezeichnen die Abbildung
    \begin{equation}
    \label{eq:raum_zeit_feld}
        \omega \colon I \times \Omega \to \mathbb{R}, \quad \omega(t, \vec{x}) :=
        w_{1}(\vec{x}) \chi_{I_{1}}(t) + w_{2}(\vec{x}) \chi_{I_{2}}(t)
        =
        \begin{cases}
            w_{1}(\vec{x}), & t < T_{f}, \\
            w_{2}(\vec{x}), & t \geq T_{f},
        \end{cases}
    \end{equation}
    als \emph{Raum"=Zeit"=Feld}.
\end{Definition}

\begin{Definition}
\label{definition:propagator_differentialgleichung}
    Seien $\omega$ ein Raum"=Zeit"=Feld wie in \cref{eq:raum_zeit_feld}, $u_{0} \colon \Omega \to \mathbb{R}$ eine Anfangsbedingung, $g \colon I \times \Omega \to \mathbb{R}$ ein Quellterm sowie $c \in \mathbb{R}_{+}$ und $\mu \in \mathbb{R}$ Konstanten.
    Als \emph{Propagator"=Differentialgleichung} bezeichnen wir die parabolische partielle Differentialgleichung
    \begin{equation}
    \label{eq:propagator_differentialgleichung}
        \left\{
        \begin{aligned}
            u_{t}(t, \vec{x}) - c \Delta u(t, \vec{x}) + \omega(t, \vec{x}) u(t, \vec{x}) + \mu u(t, \vec{x}) &= g(t, \vec{x}) \quad &&\text{auf}~I \times \Omega,\\
            u(0, \vec{x}) &= u_{0}(\vec{x}) \quad &&\text{auf}~\Omega, \\
            u(t, \vec{x})~\text{erfüllt eine vorgegebene Randbe}&\text{dingung} &&\text{auf}~I \times \partial \Omega.
        \end{aligned}
        \right.
    \end{equation}
\end{Definition}

Wie in der Einleitung erwähnt, hat der Mittelwert der Felder keinen Einfluss auf das Ergebnis des dort beschriebenen Iterationsverfahrens, weswegen wir den zusätzlichen Term $\mu u(t, \vec{x})$ einführen können.
Dieser wird sich bei den folgenden theoretischen Untersuchungen und der numerischen Umsetzung als nützlich erweisen.

Unser Ziel ist es nun, eine Raum"=Zeit"=Variationsformulierung der Propagator"=Differentialgleichung aus obiger Definition herzuleiten.
Diese wird uns als Ausgangspunkt für unsere numerischen Verfahren dienen.

Zunächst schränken wir aber die möglichen Randbedingungen ein.
Von größtem Interesse sind für uns, bedingt durch die Motivation der parabolischen Differentialgleichung in \cref{chapter:einleitung}, der Fall homogener Dirichlet"=Randbedingungen, also $u(t, \vec{x}) = 0$ auf $I \times \partial \Omega$, sowie der Fall periodischer Randbedingungen.
Letztere werden am Ende dieses Kapitels wieder aufgegriffen, während wir uns im Rest der Ausführungen auf den Fall homogener Dirichlet"=Randbedingungen beschränken.

Bei der Herleitung der Raum"=Zeit"=Variationsformulierung der Propagator"=Differentialgleichung werden wir schrittweise vorgehen und zunächst den stationären Fall betrachten, bevor wir darauf aufbauend die schwache Raum"=Zeit"=Variationsformulierung erhalten.
Als Grundlage für die schwache Formulierung im stationären Fall verwenden wir die wohlbekannten Sobolev"=Räume.
Die folgende Bemerkung führt die notwendigen Notationen in diesem Zusammenhang ein.

\begin{Bemerkung}
\label{bemerkung:raeume_und_gelfand_tripel}
    Wir schreiben kurz $V = H^{1}_{0}(\Omega)$ und $H = L_{2}(\Omega)$ für den bekannten Sobolev- respektive Lebesgue"=Raum auf $\Omega$.
    Dabei handelt es sich jeweils um einen Hilbertraum.
    Wir kennzeichnen durch den jeweiligen Index die entsprechenden Skalarprodukte und Normen.
    Als $V$-Norm wählen wir $\norm{\eta}_{V} = (\norm{\eta}^{2}_{H} + \norm{\grad \eta}^{2}_{H})^{1/2}$.
    Da $V$ dicht in $H$ eingebettet werden kann, definiert
    \begin{equation}
        V \denseinclusion H \cong H' \denseinclusion V' = (H^{1}_{0}(\Omega))' = H^{-1}(\Omega)
    \end{equation}
    ein Gelfand"=Tripel, vergleiche \cref{definition:gelfand_tripel}.
    Motiviert durch \cref{bemerkung:skalarprodukte_und_duality_pairing} verwenden wir die Schreibweise $\skp{\blank}{\blank}{V' \times V}$ auch für die duale Paarung auf $V' \times V$.
\end{Bemerkung}

Damit können wir nun die folgende Familie von Bilinearformen und die zugehörigen Operatoren definieren.

\begin{Definition}
\label{definition:operator_bilinearform_zeit}
    Seien $\omega$ ein Raum"=Zeit"=Feld sowie $c \in \mathbb{R}_{+}$ und $\mu \in \mathbb{R}$ Konstanten.
    Wir definieren für fast alle $t \in I$ eine Familie von Bilinearformen
    \begin{equation}
        \label{eq:bilinearform_zeit}
        a(\blank, \blank; t) \colon V \times V \to \mathbb{R}, \quad  a(\eta, \zeta; t) := c\skp{\grad \eta}{\grad \zeta}{H} + \skp{\omega(t, \blank) \eta}{\zeta}{H} + \mu \skp{\eta}{\zeta}{H}
    \end{equation}
    und eine zugehörige Familie von Operatoren $A(t) \colon V \to V'$ via
    \begin{equation}
        \label{eq:operator_zeit}
        \skp{A(t)\eta}{\zeta}{V' \times V} := a(\eta, \zeta; t), \quad \eta, \zeta \in V.
    \end{equation}
\end{Definition}

\begin{Bemerkung}
\label{bemerkung:operator_bilinearform_riesz}
    Die Existenz des Operators $A(t)$ zur jeweiligen Bilinearform $a(\blank, \blank; t)$ lässt sich durch den Rieszschen Darstellungssatz begründen, siehe beispielsweise \cite[Theorem \S{}22.1]{Halmos:1957vd}.
\end{Bemerkung}

Wir wollen nun motivieren, warum die Bilinearformen so gewählt werden.
Hierzu greifen wir auf die bekannte schwache Formulierung des Laplace-Operators $- \Delta \colon V \to V'$ zurück.
Diese lässt sich für die vorliegende Wahl von $V$ gerade durch die Bilinearform
\begin{equation}
    V \times V \to \mathbb{R}, \quad (\eta, \zeta) \mapsto \skp{\grad \eta}{\grad \zeta}{H}
\end{equation}
beschreiben.
Unter Verwendung dieser ergibt sich, dass die Bilinearform $a(\blank, \blank; t)$ für fast alle $t \in I$ gerade die schwache Formulierung des Operators
\begin{equation}
    A(t) \colon V \to V', \quad A(t)\eta = - \Delta \eta + \omega(t, \blank) \eta + \mu \eta
\end{equation}
ist, was gerade dem räumlichen Differentialoperator der Propagator"=Differentialgleichung \cref{eq:propagator_differentialgleichung} entspricht.

\begin{Bemerkung}
\label{bemerkung:raum_zeit_feld_norm_zeitunabhaengig}
    Nach Definition von $\omega$ in \cref{eq:raum_zeit_feld} gilt für fast alle $t \in I$ offenbar
    \begin{equation}
        \norm{\omega(t, \blank)}_{L_{\infty}(\Omega)} = \norm{w_{1}}_{L_{\infty}(\Omega)} \chi_{I_{1}}(t) + \norm{w_{2}}_{L_{\infty}(\Omega)} \chi_{I_{2}}(t)
    \end{equation}
    und somit wegen der disjunkten Zerlegung $I = I_{1} \cup I_{2}$ insbesondere
    \begin{equation}
        \norm{\omega}_{L_{\infty}(I; L_{\infty}(\Omega))} = \esssup_{t \in I} \norm{\omega(t, \blank)}_{L_{\infty}(\Omega)} = \max\Set{ \norm{w_{1}}_{L_{\infty}(\Omega)}, \norm{w_{2}}_{L_{\infty}(\Omega)} } < \infty.
    \end{equation}
\end{Bemerkung}

Wir weisen nun zwei Eigenschaften für diesen Operator beziehungsweise die zugehörige Bilinearform nach, welche eine wichtige Rolle für die spätere Raum"=Zeit"=Variationsformulierung spielen werden.

% TODO: für fast alle t besser einbauen
\begin{Satz}
\label{satz:bilinearform_messbar_stetig_garding}
    Sei $\Set{a(\blank, \blank; t)}_{t \in I}$ die Familie von Bilinearformen aus \cref{definition:operator_bilinearform_zeit}.
    Diese Bilinearformen erfüllen die folgenden Eigenschaften:
    \begin{thmenumerate}
        \item\label{satz:bilinearform_messbar_stetig_garding:messbar}
        \emph{Messbarkeit}: Die Abbildung $I \ni t \mapsto a(\eta, \zeta; t)$ ist messbar für alle $\eta, \zeta \in V$.
        \item\label{satz:bilinearform_messbar_stetig_garding:stetig}
        \emph{Stetigkeit:} Es gilt
        \begin{equation}
            \label{eq:bilinearform_messbar_stetig_garding:stetig}
            \abs{a(\eta, \zeta; t)} \leq \gamma_{a} \norm{\eta}_{V} \norm{\zeta}_{V} \quad \fa \eta, \zeta \in V \text{ und fast alle } t \in I
        \end{equation}
        mit Stetigkeitskonstante $\gamma_{a} = \max\Set{c, \norm{\omega}_{L_{\infty}(I; L_{\infty}(\Omega))} + \abs{\mu}} < \infty$.
        \item\label{satz:bilinearform_messbar_stetig_garding:garding}
        \emph{G\aa{}rding"=Ungleichung:} Es gilt
        \begin{equation}
            \label{eq:bilinearform_messbar_stetig_garding:garding}
            a(\eta, \eta; t) + \lambda \norm{\eta}_{H}^{2} \geq \alpha \norm{\eta}_{V}^{2} \quad \fa \eta \in V \text{ und fast alle } t \in I
        \end{equation}
        mit $\alpha = c \gamma_{\Omega}^{2} > 0$ und $\lambda = \max\Set{\norm{\omega}_{L_{\infty}(I; L_{\infty}(\Omega))} - \mu, 0} \geq 0$, wobei $\gamma_{\Omega}$ die Poincaré"=Friedrichs"=Konstante ist.
    \end{thmenumerate}

    \begin{Beweis}
        Für den Nachweis der Messbarkeit stützen wir uns auf die Aussagen in \cite[177]{fattorini2005infinite} und \cite[Theorem 2.7.9, Corollary 2.7.10, Lemma 8.1.1]{Andreev:2012ep}.
        Demnach ist für $\xi \in L_{\infty}(I \times \Omega)$ und $\psi \in L_{1}(\Omega)$ die Abbildung $I \ni t \mapsto \skp{\xi(t, \blank)}{\psi}{L_{2}(\Omega)}$ messbar.

        Wegen $\eta, \zeta \in V = H^{1}_{0}(\Omega)$ ist sowohl die Abbildung $\Omega \ni \vec x \mapsto \skp{\grad \eta(\vec{x})}{\grad \zeta (\vec x)}{H}$ als auch $\Omega \ni \vec x \mapsto \eta(\vec{x})\zeta (\vec x)$ in $L_{1}(\Omega)$.
        Es bleibt also lediglich $\omega \in L_{\infty}(I \times \Omega)$ zu zeigen.
        Dies ist aber aufgrund von $\chi_{I_{1}}, \chi_{I_{2}} \in L_{\infty}(I)$ und $w_{1}, w_{2} \in L_{\infty}(\Omega)$ ebenfalls gegeben.
        Damit erfüllt jeder der Summanden von $a(\blank, \blank; t)$ und folglich auch die Bilinearform selbst die geforderte Messbarkeit.

        Als nächstes zeigen wir die Stetigkeit.
        Seien dazu $\eta, \zeta \in V$ beliebig, dann erhalten wir unter Verwendung der Dreiecks- und der Cauchy"=Schwarz"=Ungleichung für beliebiges $t \in I$ die Abschätzung
        \begin{align}
            \abs{a(\eta, \zeta; t)}
            &= \abs{c \skp{\grad \eta}{\grad \zeta}{H} + \skp{\omega(t, \blank) \eta}{\zeta}{H} + \mu \skp{\eta}{\zeta}{H} }
            \\&\leq c \abs{\skp{\grad \eta}{\grad \zeta}{H}} + \abs{\skp{\omega(t, \blank) \eta}{\zeta}{H}} + \abs{\mu} \abs{\skp{\eta}{\zeta}{H}}
            \\&\leq c \norm{\grad \eta}_{H} \norm{\grad \zeta}_{H} + \left( \norm{\omega}_{L_{\infty}(I; L_{\infty}(\Omega))} + \abs{\mu} \right) \norm{\eta}_{H} \norm{\zeta}_{H}
            \\&\leq \max \Set{ c, \norm{\omega}_{L_{\infty}(I; L_{\infty}(\Omega))} + \abs{\mu}} \norm{\eta}_{V} \norm{\zeta}_{V}.
        \end{align}

        Für die G\aa{}rding"=Ungleichung seien nun $\eta \in V$ und $\lambda \in \mathbb{R}$.
        Es gilt
        \begin{align}
            a(\eta, \eta; t) + \lambda \norm{\eta}^{2}_{H}
            &= c \norm{\grad \eta}^{2}_{H} + \skprod{\omega(t, \blank) \eta}{\eta}_{H} + \mu \skprod{\eta}{\eta}_{H} + \lambda \skprod{\eta}{\eta}_{H}
            \\&= c \norm{\grad \eta}^{2}_{H} + \skprod{(\omega(t, \blank) + \mu + \lambda) \eta}{\eta}_{H}.
        \end{align}
        Wählen wir nun $\lambda := \max\Set{\norm{\omega}_{L_{\infty}(I; L_{\infty}(\Omega))} - \mu, 0} \geq 0$, dann gilt $\omega + \mu + \lambda \geq 0$ fast überall in $\Omega$ für fast alle $t \in I$ und wir erhalten die Abschätzung
        \begin{align}
            a(\eta, \eta; t) + \lambda \norm{\eta}^{2}_{H}
            &\geq c \norm{\grad \eta}^{2}_{H},
            \intertext{woraus wir durch Anwenden der Poincaré"=Friedrichs"=Ungleichung \cite[Lemma 89.4]{HankeBourgeois:2009fk}}
            a(\eta, \eta;t ) + \lambda \norm{\eta}^{2}_{H}
            &\geq c \gamma_{\Omega}^{2} \norm{\eta}^{2}_{V}
        \end{align}
        folgern können.
    \end{Beweis}
\end{Satz}

\begin{Korollar}
\label{korollar:bilinearform_elliptisch}
    Ist $\lambda = 0$, insbesondere wenn $\mu \geq \norm{\omega}_{L_{\infty}(I; L_{\infty}(\Omega))}$ gilt, dann sind die Bilinearformen $\Set{a(\blank, \blank; t)}_{t \in I}$ für fast alle $t \in I$ elliptisch.
\end{Korollar}

Ferner wollen wir an dieser Stelle eine Aussage nachweisen, welche wir erst in \cref{chapter:galerkin} benötigen werden.

\begin{Lemma}
\label{lemma:operator_selbstadjungiert}
    Die Operatoren $\Set{A(t)}_{t \in I}$ aus \cref{definition:operator_bilinearform_zeit} bilden eine Familie von selbstadjungierten Operatoren.

    \begin{Beweis}
        Diese Aussage kann mit Hilfe der zu den Operatoren $A(t)$ zugehörigen Bilinearformen $a(\blank, \blank; t)$ gefolgert werden, da diese analog zu \cref{eq:bilinearform_formel} umgeschrieben werden können zu
        \begin{equation}
        \begin{aligned}
            a(\eta, \zeta; t)
            &= c\skp{\grad \eta}{\grad \zeta}{H} + \skp{\omega(t, \blank) \eta}{\zeta}{H} + \mu \skp{\eta}{\zeta}{H},
            \\&= - c\skp{\eta}{\Delta \zeta}{V \times V'} + \skp{\eta}{\omega(t, \blank) \zeta}{V \times V'} + \mu \skp{\eta}{\zeta}{V \times V'}
            \\&= \skp{\eta}{A(t) \zeta}{V \times V'},
        \end{aligned}
        \end{equation}
        damit insbesondere also
        \begin{equation}
            \skp{A(t)\eta}{\zeta}{V' \times V} = a(\eta, \zeta; t) = \skp{\eta}{A(t) \zeta}{V \times V'} \quad \fa \eta, \zeta \in V.
        \end{equation}
        Dies zeigt gerade die Selbstadjungiertheit.
    \end{Beweis}
\end{Lemma}

Mit diesem Satz sind die notwendigen theoretischen Grundlagen abgeschlossen, sodass im Folgenden die Raum"=Zeit"=Variationsformulierung der Propagator-Differentialgleichung formuliert werden kann.
Diese können wir informell durch Multiplikation der parabolischen Differentialgleichung \cref{eq:propagator_differentialgleichung} mit einer Raum"=Zeit"=Testfunktion $v_{1}$ und Integration über $\Omega$ und $I$ sowie Addition der Anfangsbedingung, welche mit einer Raum"=Testfunktion $v_{2}$ multipliziert und anschließend über $\Omega$ integriert wird, herleiten.

Um eine exakte Definition zu ermöglichen, benötigen wir zunächst zwei Raum"=Zeit"=Hilberträume, welche als Ansatz- respektive Testraum dienen werden.
Es bietet sich an, die aus \cref{definition:ansatz_und_testraum} bereits bekannten Räume
\begin{equation}
    \label{eq:raum_zeit_ansatzraum_testraum}
    \mathcal X = L_{2}(I; V) \cap H^{1}(I; V')
    \quad \text{und} \quad
    \mathcal Y = L_{2}(I; V) \times H,
\end{equation}
zu verwenden.
% Damit können wir die schwache Formulierung nun wie folgt definieren.

\begin{Definition}[Schwache Formulierung]
\label{definition:raum_zeit_variationsformulierung}
    Seien $g \in L_{2}(I; V')$ ein Quellterm und $u_{0} \in H$ eine Anfangsbedingung.
    Als \emph{schwache Formulierung} oder \emph{Raum"=Zeit"=Variationsformulierung} der Propagator"=Differentialgleichung \cref{eq:propagator_differentialgleichung} bezeichnen wir das folgende Variationsproblem:
    \begin{equation}
    \label{eq:raum_zeit_variationsformulierung}
        \text{Finde}~u \in \mathcal X \text{ mit} \quad  b(u, v) = f(v) \quad \fa v = (v_{1}, v_{2}) \in \mathcal Y.
    \end{equation}
    Dabei sei die Bilinearform $b(\blank, \blank) \colon \mathcal X \times \mathcal Y \to \mathbb{R}$ durch
    \begin{equation}
        \label{eq:raum_zeit_variationsformulierung:lhs}
        b(u, v)
            := \int_{I} \left[ \skp{u_{t}(t)}{v_{1}(t)}{V' \times V} + a(u(t), v_{1}(t); t) \right]  \diff t + \skprod{u(0)}{v_{2}}_{H}
    \end{equation}
    und das stetige lineare Funktional $f \colon \mathcal Y \to \mathbb{R}$ auf der rechten Seite durch
    \begin{equation}
        \label{eq:raum_zeit_variationsformulierung:rhs}
        f(v) := \int_{I} \skp{g(t)}{v_{1}(t)}{V' \times V} \diff t + \skprod{u_{0}}{v_{2}}_{H}
    \end{equation}
    gegeben.
\end{Definition}

\begin{Bemerkung}
\label{bemerkung:raum_zeit_variationsformulierung_rhs_stetig}
    Die Linearität von $f$ ist direkt ersichtlich; die Stetigkeit aber wollen wir an dieser Stelle nachweisen.
    Durch Anwendung der Cauchy"=Schwarz- und der Hölder"=Ungleichung erhalten wir
    \begin{equation}
        \begin{aligned}
            f(v)
            % &= \int_{I} \skprod{g(t)}{v_{1}(t)}_{V' \times V} \diff t + \skprod{u_{0}}{v_{2}}_{H}
            &\leq \int_{I} \norm{g(t)}_{V'} \norm{v_{1}(t)}_{V} \diff t + \norm{u_{0}}_{H} \norm{v_{2}}_{H}
            \\&\leq \left( \int_{I} \norm{g(t)}^{2}_{V'} \diff t \right)^{1/2} \left( \int_{I} \norm{v_{1}(t)}^{2}_{V} \diff t \right)^{1/2} + \norm{u_{0}}_{H} \norm{v_{2}}_{H}
            \\&= \norm{g}_{L_{2}(I; V')} \norm{v_{1}}_{L_{2}(I; V)} + \norm{u_{0}}_{H} \norm{v_{2}}_{H}
            \\&\leq \max\Set*{\norm{g}_{L_{2}(I; V')}, \norm{u_{0}}_{H}} \left( \norm{v_{1}}_{L_{2}(I; V)} + \norm{v_{2}}_{H} \right)
            \\&\leq \sqrt{2} \max\Set*{\norm{g}_{L_{2}(I; V')}, \norm{u_{0}}_{H}} \left( \norm{v_{1}}_{L_{2}(I; V)}^{2} + \norm{v_{2}}_{H}^{2} \right)^{1/2}
            \\&= \sqrt{2} \max\Set*{\norm{g}_{L_{2}(I; V')}, \norm{u_{0}}_{H}} \norm{v}_{\mathcal Y}
        \end{aligned}
    \end{equation}
    und damit die Stetigkeit, wobei die Ungleichung $x + y \leq \sqrt{2} \sqrt{x^2 + y^2}$, welche für alle $x, y \in \mathbb{R}$ gilt, für die letzte Abschätzung verwendet wurde.
\end{Bemerkung}

Der nächste Schritt ist nun, nachzuweisen, dass obige Raum"=Zeit"=Variationsformulierung im Sinne von \cref{definition:sachgemaess_gestellt_nach_hadamard} korrekt gestellt ist, also eine eindeutige Lösung besitzt, welche stetig von dem Funktional $f \in \mathcal Y'$ abhängt.
Hierzu werden wir mit \cref{satz:ss09:theorem51} ansetzen, welcher unter den gegebenen Rahmenbedingungen die zu prüfenden Eigenschaften auf die in \cref{satz:bilinearform_messbar_stetig_garding} bereits nachgewiesenen reduziert.

\begin{Korollar}
\label{satz:raum_zeit_variationsformulierung_sachgemaess_gestellt}
    Seien Ansatz- und Testraum $\mathcal X$ und $\mathcal Y$ wie in \cref{eq:raum_zeit_ansatzraum_testraum}.
    Dann ist die Raum"=Zeit"=Variationsformulierung \cref{eq:raum_zeit_variationsformulierung} korrekt gestellt.

    \begin{Beweis}
        Dies ist nach Definition der Raum"=Zeit"=Variationsformulierung eine unmittelbare Folgerung aus \cref{satz:ss09:theorem51} und \cref{satz:bilinearform_messbar_stetig_garding}.
    \end{Beweis}
\end{Korollar}

Weiter erhalten wir als Nebenprodukt aus \cref{korrolar:ss09:theorem51_abschaetzungen} auch Schranken für die Stetigkeitskonstante $\gamma_{b}$ sowie die inf-sup"=Konstante $\beta$ der Bilinearform $b(\blank, \blank)$.

\begin{Korollar}
\label{korollar:rz_variationsformulierung_stetig_infsup_schranken}
    Seien die Voraussetzungen von \cref{satz:raum_zeit_variationsformulierung_sachgemaess_gestellt} gegeben.
    Dann erfüllt die Bilinearform $b(\blank, \blank)$ die folgenden Eigenschaften:
    \begin{thmenumerate}
        \item \emph{Stetigkeit:} Es gilt
            \begin{equation}
                \gamma_{b} := \supsup{u \in \mathcal X}{v \in \mathcal Y} \frac{b(u, v)}{\norm{u}_{\mathcal X} \norm{v}_{\mathcal Y}} < \infty.
            \end{equation}
        \item \emph{inf-sup-Bedingung:} Es gilt
            \begin{equation}
                \beta := \infsup{u \in \mathcal X}{v \in \mathcal Y} \frac{b(u, v)}{\norm{u}_{\mathcal X} \norm{v}_{\mathcal Y}} > 0.
            \end{equation}
    \end{thmenumerate}
    Erfüllt die Bilinearform $a(\blank, \blank; t)$ die G\aa{}rding"=Ungleichung \cref{eq:bilinearform_messbar_stetig_garding:garding} mit $\lambda = 0$ für fast alle $t \in I$, dann gilt
    \begin{align}
        \gamma_{b}  &\leq \sqrt{2 \max\Set{ 1, c, \norm{\omega}_{L_{\infty}(I; L_{\infty}(\Omega))} + \abs{\mu} }^{2} + \gamma_{e}^{2} }, \\
        \beta &\geq \frac{\gamma_{\Omega}^{2} \min\Set{c, c^{-1}, c (\norm{\omega}_{L_{\infty}(I; L_{\infty}(\Omega))} + \abs{\mu})^{-2}}}{\sqrt{2 \max\Set{c^{-2}\gamma_{\Omega}^{-4}, 1} + \gamma_{e}^{2}}}.
    \end{align}
    Ist dagegen $\lambda > 0$, dann erhalten wir stattdessen die erweiterten Abschätzungen
    \begin{align}
        \gamma_{b} &\leq \frac{\gamma'_{b}}{\max\Set{\sqrt{1 + 2 (\norm{\omega}_{L_{\infty}(I; L_{\infty}(\Omega))} - \mu)^{2} \rho^{4} }, \sqrt{2}}}, \\
        \beta &\geq \frac{e^{-2(\norm{\omega}_{L_{\infty}(I; L_{\infty}(\Omega))} - \mu) T}}{\max\Set{\sqrt{1 + 2 (\norm{\omega}_{L_{\infty}(I; L_{\infty}(\Omega))} - \mu)^{2} \rho^{4} }, \sqrt{2}}} \beta',
    \end{align}
    wobei $\gamma'_{b}$ und $\beta'$ den Größen $\gamma_{b}$ und $\beta$ des ersten Falles entsprechen.
    Die Größen $\gamma_{e}$ und $\rho$ entsprechen dabei denen aus \cref{korrolar:ss09:theorem51_abschaetzungen}.

    \begin{Beweis}
        Die Schranken ergeben sich durch Einsetzen der Größen $\gamma_{a}, \lambda$ und $\alpha$ aus \cref{satz:bilinearform_messbar_stetig_garding} in die Schranken von \cref{korrolar:ss09:theorem51_abschaetzungen}.
    \end{Beweis}
\end{Korollar}

% TODO: Beispiel


\section{Parametrische Formulierung} % (fold)
\label{section:parametrische_formulierung}

% TODO: Anmerkung besser umsetzen
Nachdem nun eine erste schwache Formulierung der Propagator"=Differentialgleichung eingeführt wurde, welche in dieser Form bereits als Grundlage für eine numerische Umsetzung verwendet werden kann, wollen wir nun als nächsten Schritt die darin auftretenden Felder möglichst niedrigdimensional parametrisieren.
Dies wird vor allem durch das Iterationsverfahren aus \cref{chapter:einleitung} motiviert, in dem die Propagator"=Differentialgleichung immer wieder für leicht variierte Raum-Zeit-Felder $\omega$ berechnet wird.

Dazu kehren wir nun zunächst zum Operator $A(t)$ aus \cref{definition:operator_bilinearform_zeit} zurück und betrachten diesen zunächst unabhängig von der Zeit $t \in I$, aber in Abhängigkeit von einem Feld $w \in L_{\infty}(\Omega)$.
Wir definieren für $w \in L_{\infty}(\Omega)$ eine Familie von Operatoren $A(w)$ als
\begin{equation}
    \label{eq:operator_feld}
    A(w) \colon V \to V', \quad A(w) \eta := - c \Delta \eta + w \eta + \mu \eta.
\end{equation}
Wie zuvor sei auch eine Familie von zugehörigen Bilinearformen $a(\blank, \blank; w)$ gegeben durch
\begin{equation}
    \label{eq:bilinearform_feld}
    \begin{aligned}
        a(\blank, \blank; w) \colon V \times V \to \mathbb{R}, \quad
        a(\eta, \zeta; w) := c\skp{\grad \eta}{\grad \zeta}{H} + \skp{w \eta}{\zeta}{H} + \mu \skp{\eta}{\zeta}{H}.
    \end{aligned}
\end{equation}

Um die Abhängigkeit der obigen Operatoren respektive Bilinearformen vom Feld auch für die nachfolgende numerische Umsetzung verwendbar zu machen, wollen wir diese Abhängigkeit von einer Abbildung $w \in L_{\infty}(\Omega)$ durch eine von einer diskreten Größe, beispielsweise einer Koeffizientenfolge aus $\ell_{1}(\mathbb{N})$ oder ähnlichen Folgenräumen, ersetzen.
Dies erreichen wir durch folgende Einschränkung der verwendeten Felder $w \in L_{\infty}(\Omega)$ auf Funktionen, die wir als Reihenentwicklung von hier noch nicht näher spezifizierten Funktionen $\varphi_{i}$ darstellen können.
Auf die Wahl dieser $\varphi_{i}$ werden wir bei der numerischen Untersuchung in \cref{chapter:galerkin} erneut eingehen.

\begin{Definition}
\label{definition:feld_entwickelbar}
    Sei $\Set{\varphi_{i}}_{i \in \mathbb{N}} \subset L_{\infty}(\Omega)$ ein System von Funktionen und sei weiter ein Parameterraum $\mathcal P \subset \ell_{\infty}(\mathbb{N})$ gegeben.
    Wir nennen ein Feld $w \in L_{\infty}(\Omega)$ \emph{darstellbar} durch $\Set{\varphi_{i}}_{i \in \mathbb{N}}$, wenn ein $\bm{\sigma} \in \mathcal P$ existiert, so dass $w$ mit
    \begin{equation}
        w(\bm\sigma) = \sum_{i = 1}^{\infty} \sigma_{i} \varphi_{i}
    \end{equation}
    im Sinne der gleichmäßigen Konvergenz übereinstimmt.
\end{Definition}

Wir können mit einem festen System $\Set{\varphi_{i}}_{i \in \mathbb{N}}$ im Allgemeinen nicht alle möglichen $L_{\infty}(\Omega)$-Funktionen darstellen.
Dies ist aber auch nicht nötig: Wie man in der Einführung bereits gesehen hat, weisen die während des Iterationsverfahrens tatsächlich auftretenden Felder gewisse Eigenschaften, beispielsweise Symmetrie oder Regularität, auf und wir können diese in die Wahl des Systems $\Set{\varphi_{i}}_{i \in \mathbb{N}}$ einfließen lassen.

\begin{Bemerkung}
    Für den Rest dieses Kapitels beschränken wir uns bei der Wahl des Parameterraums auf $\mathcal P = [-1, 1]^{\mathbb{N}}$.
    Dies dient hauptsächlich der Vereinfachung der Beweise und stellt keine Einschränkung dar, da die Funktionen $\Set{ \varphi_{i} }_{i \in \mathbb{N}}$ entsprechend umskaliert werden können.
\end{Bemerkung}

Um sicherzustellen, dass derartige Felder $w(\bm\sigma)$ wohldefinierte Operatoren $A(w(\bm\sigma))$ liefern, fordern wir die in der folgenden Annahme festgehaltene Eigenschaft von der Funktionenfolge $\Set{\varphi_{i}}_{i \in \mathbb{N}}$.

\begin{Annahme}
\label{annahme:system_l1_summierbar}
    Das Funktionensystem $\Set{ \varphi_{i} }_{i \in \mathbb{N}} \subset L_{\infty}(\Omega)$ sei einfach summierbar in der $L_{\infty}$-Norm, das heißt, es gelte $\Set{ \norm{\varphi_{i}}_{L_{\infty}(\Omega) } }_{i \in \mathbb{N}} \in \ell_{1}(\mathbb{N})$.
\end{Annahme}
%
Im Folgenden bezeichnen wir die obige $\ell_{1}(\mathbb{N})$-Norm der Kürze wegen als
\begin{equation}
\label{eq:system_l1_norm}
    c_{\varphi} := \sum_{i = 1}^{\infty} \norm{\varphi_{i}}_{L_{\infty}(\Omega)}.
\end{equation}
Diese Annahme stellt insbesondere die gleichmäßige Konvergenz von $w(\bm\sigma)$ für alle $\bm\sigma \in \mathcal P$ sicher, denn es gilt
\begin{equation}
\label{eq:rz_feld_norm_schranke}
    \sup_{\bm\sigma \in \mathcal P} \norm{w(\bm\sigma)}_{L_{\infty}(\Omega)} \leq \sum_{i = 1}^{\infty} \norm{\varphi_{i}}_{L_{\infty}(\Omega)} = c_{\varphi} < \infty.
\end{equation}

Legen wir uns auf ein konkretes Funktionensystem $\Set{\varphi_{i}}_{i \in \mathbb{N}}$, welches die \cref{annahme:system_l1_summierbar} erfüllt, fest, dann können wir die Operatoren $A(\omega)$ nun auch als Familie von Operatoren $A(\bm\sigma)$ betrachten, denn durch Einsetzen von $w(\bm\sigma)$ in \cref{eq:operator_feld} erhalten wir
\begin{equation}
\label{eq:operator_parameter}
    A(\bm\sigma) \colon V \to V', \quad A(\bm\sigma) \eta = -c \Delta \eta + \sum_{i = 1}^{\infty} \sigma_{i} \varphi_{i} \eta + \mu \eta.
\end{equation}
Weiter können wir auch die zugehörige Bilinearform $a(\blank, \blank; \bm\sigma)$ angeben als
\begin{equation}
\label{eq:bilinearform_parameter}
    \begin{aligned}
    a(\blank, \blank; \bm\sigma) \colon V \times V \to \mathbb{R},
    \quad a(\eta, \zeta; \bm\sigma) = c\skp{\grad \eta}{\grad \zeta}{H} + \sum_{i = 1}^{\infty} \sigma_{i} \skp{\varphi_{i} \eta}{\zeta}{H} + \mu \skp{\eta}{\zeta}{H}.
    \end{aligned}
\end{equation}

Die bei der Bilinearform vorgenommene Vertauschung von Summe und $H$"=Skalarprodukt ist durch \cref{annahme:system_l1_summierbar} respektive Ungleichung \cref{eq:rz_feld_norm_schranke} und den Satz von Lebesgue gerechtfertigt.
Wie auch für den nicht"=parametrischen Fall werden wir nachweisen, dass es sich hierbei um Operatoren beziehungsweise Bilinearformen handelt, welche die für uns wichtigen Eigenschaften der Stetigkeit und der Gültigkeit einer G\aa{}rding"=Ungleichung besitzen.
Zunächst wollen wir an dieser Stelle noch ein parametrisches Äquivalent der Raum"=Zeit"=Variationsformulierung aus \cref{definition:raum_zeit_variationsformulierung} formulieren.

Dies bedarf, wie zuvor, eines zeitlichen Wechsels zwischen mehreren Feldern $w_{j}$.
Wir beschränken uns auf den bereits bekannten Fall zweier Felder und erweitern die obige Operator"=Definition um die zeitliche Abhängigkeit.
Zunächst definieren wir analog zu \cref{eq:raum_zeit_feld} ein parametrisches Raum"=Zeit"=Feld.
Dies geschieht auf Basis der Darstellung von $w_{j}$ aus \cref{definition:feld_entwickelbar}.
Da diese den zeitlichen Wechsel noch nicht enthält, führen wir diesen durch die Aufteilung von $\bm \sigma$ in zwei Teilfolgen ein, um so die Notation möglichst einfach zu halten.
Sei dazu $\bm\sigma \in \mathcal P$, dann definieren wir die folgenden Teilfolgen der ungeraden respektive geraden Indizes als $\bm\sigma_{\odd} = (\sigma_{2i-1})_{i \in \mathbb{N}}$ und $\bm\sigma_{\even} = (\sigma_{2i})_{i \in \mathbb{N}}$.

\begin{Definition}
\label{definition:parametrisches_raum_zeit_feld}
    Sei $\bm\sigma \in \mathcal P$.
    Unter einem \emph{parametrischen Raum-Zeit-Feld} verstehen wir die Abbildung
    $\omega(\blank, \blank; \bm\sigma) \colon I \times \Omega \to \mathbb{R}$, die durch
    \begin{equation}
    \label{eq:parametrisches_raum_zeit_feld}
        \begin{aligned}
            \omega(t, \vec{x}; \bm\sigma)
            &:= w(\vec{x}; \bm\sigma_{\odd}) \chi_{I_{1}}(t) + w(\vec{x}; \bm\sigma_{\even}) \chi_{I_{2}}(t)\\
            &\phantom{:}= \sum_{i = 1}^{\infty} \left[\sigma_{2i-1} \chi_{I_{1}}(t) + \sigma_{2i} \chi_{I_{2}}(t)\right] \varphi_{i}(\vec x)
        \end{aligned}
    \end{equation}
    gegeben ist.
\end{Definition}

Wegen der disjunkten Zerlegung $I = I_{1} \cup I_{2}$ ist direkt ersichtlich, dass analog zu \cref{eq:rz_feld_norm_schranke} die Abschätzung
\begin{equation}
\label{eq:parametrisches_raum_zeit_feld_schranke}
    \sup_{\bm\sigma \in \mathcal P} \norm{\omega(\bm\sigma)}_{L_{\infty}(I; L_{\infty}(\Omega))}
    = \sup_{\bm\sigma \in \mathcal P} \norm{w(\bm\sigma)}_{L_{\infty}(\Omega)} \leq \sum_{i = 1}^{\infty} \norm{\varphi_{i}}_{L_{\infty}(\Omega)} = c_{\varphi}
\end{equation}
gilt.

Wir erweitern nun die Operator"=Definition \cref{eq:operator_parameter} um die Zeitabhängigkeit.
Definieren wir also für $t \in I$ und $\bm\sigma \in \mathcal P$ die Operatorfamilie
\begin{equation}
    \label{eq:operator_zeit_parameter}
    A(t, \bm\sigma) \colon V \to V', \quad A(t, \bm\sigma) \eta := -c \Delta \eta + \omega(t, \blank; \bm\sigma) \eta + \mu \eta,
\end{equation}
dann hat die zugehörige Familie von Bilinearformen $a(\blank, \blank; t, \bm\sigma) \colon V \times V \to \mathbb{R}$ nach \cref{eq:parametrisches_raum_zeit_feld} die Form
\begin{equation}
    \label{eq:bilinearform_zeit_parameter}
    \begin{aligned}
        a(\eta, \zeta; t, \bm\sigma) = c\skp{\grad \eta}{\grad \zeta}{H} + \sum_{i = 1}^{\infty} \left[ \sigma_{2i-1} \chi_{I_{1}}(t) + \sigma_{2i} \chi_{I_{2}}(t)  \right] \skp{\varphi_{i} \eta}{\zeta}{H} + \mu \skp{\eta}{\zeta}{H}.
    \end{aligned}
\end{equation}
Mit dieser Vorarbeit können wir analog zu \cref{definition:raum_zeit_variationsformulierung} nun die folgende parametrische schwache Formulierung definieren.

\begin{Definition}%[Parametrische schwache Formulierung]
\label{definition:parametrische_rz_variationsformulierung}
    Seien $g \in L_{2}(I; V')$ ein Quellterm und $u_{0} \in H$ eine Anfangsbedingung.
    Als \emph{parametrische schwache Formulierung} oder \emph{parametrische Raum"=Zeit"=Variationsformulierung} der Propagator"=Differentialgleichung \cref{eq:propagator_differentialgleichung} bezeichnen wir das folgende Variationsproblem:
    \begin{equation}
    \label{eq:parametrisches_rz_variationsproblem}
        \text{Sei}~\bm\sigma \in \mathcal P,~\text{finde}~u(\bm\sigma) \in \mathcal X \text{ mit} \quad b(u(\bm\sigma), v; \bm\sigma) = f(v) \quad \fa v \in \mathcal Y.
    \end{equation}
    Dabei sei die Familie von Bilinearformen $b(\blank, \blank; \bm\sigma) \colon \mathcal X \times \mathcal Y \to \mathbb{R}$ gegeben durch
     \begin{equation}
     \label{eq:parametrisches_rz_variationsproblem:lhs}
         b(u, v; \bm\sigma)
             := \int_{I} \left[ \skp{u_{t}(t)}{v_{1}(t)}{V' \times V} + a(u(t), v_{1}(t); t, \bm\sigma) \right] \diff t + \skp{u(0)}{v_{2}}{H},
     \end{equation}
     wobei $a(\blank, \blank; t, \bm\sigma)$ wie in \cref{eq:bilinearform_zeit_parameter} definiert sei.
     Das stetige lineare Funktional $f \colon \mathcal Y \to \mathbb{R}$ sei wie zuvor
     \begin{equation}
     \label{eq:parametrisches_rz_variationsproblem:rhs}
         f(v) := \int_{I} \skp{g(t)}{v_{1}(t)}{V' \times V} \diff t + \skp{u_{0}}{v_{2}}{H}.
     \end{equation}
\end{Definition}

Hierfür weisen wir nun erneut nach, dass die schwache Formulierung korrekt gestellt ist.
Wie zuvor wollen wir \cref{satz:ss09:theorem51} verwenden.

\begin{Satz}
\label{satz:parametrische_bilinearform_messbar_stetig_garding}
    Sei $a(\blank, \blank; t; \bm\sigma)$, $t \in I$ und $\bm\sigma \in \mathcal P$, die Familie von Bilinearformen aus \cref{eq:bilinearform_zeit_parameter}.
    Dann gelten für alle $t \in I$ und $\bm\sigma \in \mathcal P$ die folgenden Eigenschaften:
    \begin{thmenumerate}
        \item\label{satz:parametrische_bilinearform_messbar_stetig_garding:messbar}
        \emph{Messbarkeit}: Die Abbildung $I \ni t \mapsto a(\eta, \zeta; t, \bm\sigma)$ ist messbar für alle $\eta, \zeta \in V$.
        \item\label{satz:parametrische_bilinearform_messbar_stetig_garding:stetig}
        \emph{Stetigkeit:} Es gilt
        \begin{equation}
            \label{eq:parametrische_bilinearform_messbar_stetig_garding:stetig}
            \abs{a(\eta, \zeta; t; \bm\sigma)} \leq \gamma_{a} \norm{\eta}_{V} \norm{\zeta}_{V} \quad \fa \eta, \zeta \in V \text{ und fast alle } t \in I
        \end{equation}
        mit Stetigkeitskonstante $\gamma_{a} = \max\Set{c, c_{\varphi} + \abs{\mu}} < \infty$.
        \item\label{satz:parametrische_bilinearform_messbar_stetig_garding:garding}
        \emph{G\aa{}rding"=Ungleichung:} Es gilt
        \begin{equation}
            \label{eq:parametrische_bilinearform_messbar_stetig_garding:garding}
            a(\eta, \eta; t; \bm\sigma) + \lambda \norm{\eta}_{H}^{2} \geq \alpha \norm{\eta}_{V}^{2} \quad \fa \eta \in V \text{ und fast alle } t \in I
        \end{equation}
        mit $\alpha = c \gamma_{\Omega}^{2} > 0$ und $\lambda = \max\Set{c_{\varphi} - \mu, 0} \geq 0$.
    \end{thmenumerate}
    Dabei ist $c_{\varphi}$ die Konstante aus \cref{eq:system_l1_norm} und die Konstanten $\gamma_{a}$, $\lambda$ und $\alpha$ sind sowohl unabhängig von $t \in I$ als auch von $\bm\sigma \in \mathcal P$.

    \begin{Beweis}
        Der Nachweis erfolgt analog zum nicht"=parametrischen Fall in \cref{satz:bilinearform_messbar_stetig_garding}.
        Wir müssen zusätzlich lediglich die $L_{\infty}(\Omega)$-Norm von $\omega(t, \blank; \bm\sigma)$ mit Hilfe von \cref{eq:parametrisches_raum_zeit_feld_schranke} weiter durch $c_{\varphi}$ abschätzen.
    \end{Beweis}
\end{Satz}

\begin{Korollar}
\label{korollar:parametrisches_rz_variationsproblem_sachgemaess}
    Die parametrische schwache Formulierung \cref{eq:parametrisches_rz_variationsproblem} ist für alle $\bm\sigma \in \mathcal P$ korrekt gestellt.
    Ferner existieren analog zu \cref{korollar:rz_variationsformulierung_stetig_infsup_schranken} Schranken für die Stetigkeitskonstante $\gamma_{b}$ und die inf-sup"=Konstante $\beta$, welche unabhängig von $\bm \sigma \in \mathcal P$ sind.
\end{Korollar}


\section{Regularität bezüglich der Parameter} % (fold)
\label{section:regularitaet_bezueglich_der_parameter}

Wir interessieren uns nun für die Regularität der Abhängigkeit der Lösung $u(\bm\sigma)$ der parametrischen schwachen Formulierung vom Parameter $\bm \sigma \in \mathcal P$.
Konkret werden wir nachweisen, dass die Lösung unter gewissen Annahmen an das Funktionensystem $\Set{\varphi_{i}}_{i \in \mathbb{N}}$ analytisch vom Parameter $\bm \sigma$ abhängt.
Diese Eigenschaft ist wünschenswert, da hierdurch die Anwendung der Reduzierte-Basis-Methode motiviert werden kann.
Genauer gehen wir hierauf in der Einführung in diese Methode in \cref{chapter:rbm} ein.

In diesem Abschnitt orientieren wir uns an den Arbeiten von \textcite{Cohen:2010kz,Cohen:2011jp} sowie \textcite{Kunoth:2013ef}, weisen die Regularität aber direkt für die Raum"=Zeit"=Variationsformulierung nach, statt wie in den genannten Arbeiten den Umweg über den stationären Fall zu gehen.

Wir beginnen mit einigen notationellen Vorbemerkungen.
\begin{Bemerkung}
    Die Multiindexmenge $\mathcal F$ definieren wir als $\mathcal F := \Set{ \bm\nu \in \mathbb{N}^{\mathbb{N}}_{0} \given \abs{\bm\nu} < \infty }$, wobei
    \begin{equation}
        \abs{\bm\nu} := \sum_{i = 1}^{\infty} \nu_{i}
    \end{equation}
    die $\ell_{1}(\mathbb{N})$-Norm sei.
    Anders formuliert, besteht $\mathcal F$ gerade aus denjenigen Folgen in $\mathbb{N}_{0}$, welche nur endliche viele Einträge ungleich Null aufweisen.

    Seien $\bm\nu \in \mathcal F$ und $\vec{b} \in \ell_{p}(\mathbb{N})$, $p > 0$, dann schreiben wir
    \begin{equation}
        \vec{b}^{\bm\nu} := \prod_{i = 1}^{\infty} b_{i}^{\nu_{i}}
    \end{equation}
    mit der Konvention $0^{0} = 1$.
    Wegen $\abs{\bm\nu} < \infty$ ist dieses Produkt stets endlich.
\end{Bemerkung}

Um die Notation für die nachfolgenden Beweise zu vereinfachen, ordnen wir die Darstellung der parametrischen Raum"=Zeit"=Felder um.
\begin{Bemerkung}
    Definiere neue charakteristische Funktionen und Entwicklungsfunktionen für $i \in \mathbb{N}$ durch
    \begin{equation}
        \tilde{\chi}_{i} := \begin{cases}
            \chi_{I_{1}}, & i~\text{ungerade},\\
            \chi_{I_{2}}, & i~\text{gerade},
        \end{cases} \qquad
        \tilde{\varphi}_{i} := \begin{cases}
            \varphi_{(i+1)/{2}}, & i~\text{ungerade},\\
            \varphi_{i / 2}, & i~\text{gerade}.
        \end{cases}
    \end{equation}
    Damit können wir \cref{eq:parametrisches_raum_zeit_feld} auch schreiben als
    \begin{equation}
        w(t, \vec{x}; \bm \sigma) = \sum_{i = 1}^{\infty} \sigma_{i} \tilde{\chi}_{i}(t) \tilde{\varphi}_{i}(\vec{x}).
    \end{equation}
\end{Bemerkung}

Wir fixieren nun die rechte Seite $f \in \mathcal Y'$ der schwachen Formulierungen \cref{eq:raum_zeit_variationsformulierung,eq:parametrisches_rz_variationsproblem} und beginnen den Nachweis der Regularität mit einer Stabilitätsaussage, welche in den folgenden Beweisen nützlich sein wird.

\begin{Lemma}
\label{lemma:stabilitaetsaussage}
    Seien $\omega, \tilde{\omega}$ zwei Raum"=Zeit"=Felder wie in \cref{eq:raum_zeit_feld} und $u, \tilde{u}$ die zugehörigen Lösungen der schwachen Formulierung \cref{eq:raum_zeit_variationsformulierung}.
    Dann gilt
    \begin{equation}
        \norm{u - \tilde{u}}_{\mathcal X} \leq \frac{\norm{f}_{\mathcal Y'}}{\beta^{2}} \norm{\omega - \tilde{\omega}}_{L_{\infty}(I; L_{\infty}(\Omega))},
    \end{equation}
    wobei $\beta$ eine Feld-unabhängige inf-sup"=Konstante ist.

    \begin{Beweis}
        Wir vernachlässigen der Kürze wegen im Folgenden die explizite Angabe der Zeitabhängigkeit der jeweiligen Funktionen.
        Weiter setzen wir $\theta = u - \tilde{u}$.
        Subtraktion der Variationsformulierung für die beiden Lösungen $u$ und $\tilde{u}$ liefert für beliebige Testfunktionen $v = (v_{1}, v_{2}) \in \mathcal Y$ die Gleichung
        \begin{align}
            0
            &= f(v) - f(v) = b(u, v; \omega) - b(\tilde{u}, v; \tilde{\omega})
           \\&= \int_{I} \left[ \skp{u_{t} - \tilde{u}_{t}}{v_{1}}{V' \times V} + a(u, v; \omega) - a(\tilde{u}, v; \tilde{\omega}) \right] \diff t + \skp{u(0) - \tilde{u}(0)}{v_{2}}{H}
           \\&= \int_{I} \left[ \skp{\theta_{t}}{v_{1}}{V' \times V} + c \skp{\grad \theta}{\grad v_{1}}{H} + \mu \skp{\theta}{v_{1}}{H} + \skp{\omega u - \tilde{\omega}\tilde{u}}{v_{1}}{H} \right] \diff t + \skp{\theta(0)}{v_{2}}{H}
           \\&= \int_{I} \left[ \skp{\theta_{t}}{v_{1}}{V' \times V} + a(\theta, v; \omega) \right] \diff t + \skp{\theta(0)}{v_{2}}{H} + \int_{I} \skp{(w - \tilde{w})\tilde{u}}{v_{1}}{H} \diff t
           \\&= b(\theta, v; \omega) + \int_{I} \skp{(w - \tilde{w})\tilde{u}}{v_{1}}{H} \diff t,
        \end{align}
        welche wir auch als
        \begin{equation}
            \label{eq:stabilitaetsaussage:beweis:variationsproblem}
            b(\theta, v; \omega) = h(v)
        \end{equation}
        mit der Abbildung
        \begin{equation}
            h \colon \mathcal Y \to \mathbb{R}, \quad h(v) := - \int_{I} \skp{(w - \tilde{w})\tilde{u}}{v_{1}}{H} \diff t
        \end{equation}
        auffassen können.
        Die Linearität von $h$ ist klar.
        Wir weisen nun die Stetigkeit nach, betrachten also
        \begin{align}
            \norm{h}_{\mathcal Y'}
            &= \sup_{\norm{v}_{\mathcal Y} = 1} \abs{\int_{I} \skp{(w - \tilde{w})\tilde{u}}{v_{1}}{H} \diff t}
            \\&\leq \norm{w - \tilde{w}}_{L_{\infty}(I; L_{\infty}(\Omega))} \sup_{\norm{v}_{\mathcal Y} = 1} \abs{\skp{\tilde{u}}{v_{1}}{L_{2}(I; H)}}
            \\&\leq \norm{w - \tilde{w}}_{L_{\infty}(I; L_{\infty}(\Omega))} \sup_{\norm{v}_{\mathcal Y} = 1} \norm{\tilde{u}}_{L_{2}(I; H)} \norm{v_{1}}_{L_{2}(I; H)}
            \\&\leq \norm{w - \tilde{w}}_{L_{\infty}(I; L_{\infty}(\Omega))} \sup_{\norm{v}_{\mathcal Y} = 1} \norm{\tilde{u}}_{\mathcal X} \norm{v}_{\mathcal Y}
            \\&= \norm{w - \tilde{w}}_{L_{\infty}(I; L_{\infty}(\Omega))} \norm{\tilde{u}}_{\mathcal X}
            < \infty.
        \end{align}
        Damit ist $h \in \mathcal Y'$, wir können \cref{eq:stabilitaetsaussage:beweis:variationsproblem} also selbst als Raum"=Zeit"=Variationsproblem der Form \cref{eq:raum_zeit_variationsformulierung} auffassen und erhalten damit durch \cref{satz:raum_zeit_variationsformulierung_sachgemaess_gestellt} die Abschätzung
        \begin{equation}
            \norm{\theta}_{\mathcal X} \leq \frac{1}{\beta} \norm{h}_{\mathcal Y'}.
        \end{equation}
        Wenden wir \cref{satz:raum_zeit_variationsformulierung_sachgemaess_gestellt} weiter auf die schwache Formulierung zu $\tilde{\omega}$ an, dann gilt
        \begin{equation}
            \norm{\tilde{u}}_{\mathcal X} \leq \frac{1}{\beta} \norm{f}_{\mathcal Y'}.
        \end{equation}
        Die Unabhängigkeit von $\beta$ von den Feldern kann durch das Minimum der beiden Feld-abhängigen inf-sup-Konstanten sichergestellt werden.

        Durch Zusammenfassen dieser drei Abschätzungen erhalten wir die Behauptung
        \begin{equation}
            \norm{u - \tilde{u}}_{\mathcal X} = \norm{\theta}_{\mathcal X} \leq \frac{\norm{f}_{\mathcal Y'}}{\beta^{2}} \norm{\omega - \tilde{\omega}}_{L_{\infty}(I; L_{\infty}(\Omega))}.
        \end{equation}
    \end{Beweis}
\end{Lemma}

Als erster Schritt des Regularitätsnachweises wird zunächst die Existenz beliebiger partieller Ableitungen gezeigt, bevor dann nachfolgend schrittweise die Konvergenz der Taylorreihe der Lösung $u(\bm \sigma) \in \mathcal X$ nachgewiesen wird.

\begin{Satz}
\label{satz:existenz_partieller_ableitungen}
    Die Abbildung $\mathcal P \ni \bm\sigma \mapsto u(\bm\sigma) \in \mathcal X$, welche einem Parameter $\bm\sigma$ die eindeutige Lösung $u(\bm\sigma)$ der parametrischen schwachen Formulierung \cref{eq:parametrisches_rz_variationsproblem} zuordnet, besitzt für alle $\bm\nu \in \mathcal F$ eine partielle Ableitung $\partial^{\bm\nu}_{\bm\sigma} u(\bm\sigma)$.

    \begin{Beweis}
        Erneut verzichten wir auf die explizite Notation der Zeitabhängigkeit.
        Wir beschränken uns darauf, die Behauptung exemplarisch für die partiellen Ableitungen erster Ordnung für ein festes $\bm\sigma \in \mathcal P$ nachzuweisen.
        Ohne Einschränkung sei nun $\bm\nu = \vec{e}_{j} \in \mathcal F$ für ein $j \in \mathbb{N}$ und ferner sei $h \in \mathbb{R}$.
        Wir definieren $\bm\sigma_{h} := \bm\sigma + h \bm\nu = \bm\sigma + h \vec e_{j}$ und
        \begin{equation}
            \theta_{h} := \frac{u(\bm\sigma_{h}) - u(\bm\sigma)}{h},
        \end{equation}
        wobei $u(\blank)$ die Lösung der parametrischen schwachen Formulierung \cref{eq:parametrisches_rz_variationsproblem} für die entsprechenden Parameter ist.
        Ist $\abs{h}$ klein genug, dann existieren diese auch im Fall $\bm\sigma_{h} \not\in \mathcal P$, da analog zu \cref{eq:parametrisches_raum_zeit_feld_schranke} die schwache Formulierung nach wie vor korrekt gestellt ist.
        Dies ergibt sich durch die Abschätzung
        \begin{equation}
            \norm{\omega(\bm \sigma_{h})}_{L_{\infty}(I; L_{\infty}(\Omega))} \leq \sum_{i = 1}^{\infty} \norm{\varphi_{i}}_{L_{\infty}(\Omega)} + \abs{h} \norm{\varphi_{j}}_{L_{\infty}(\Omega)}\leq c_{\varphi} + \abs{h} \norm{\varphi_{j}}_{L_{\infty}(\Omega)} < \infty.
        \end{equation}

        Zunächst schreiben wir die Differenz der parametrischen Raum-Zeit-Felder um zu
        \begin{equation}
            \omega(t, \vec{x}; \bm\sigma_{h}) - \omega(t, \vec{x}; \bm\sigma)
            = \sum_{i = 1}^{\infty} (\sigma_{h,i} - \sigma_{i} ) \tilde{\chi}_{i}(t) \tilde{\varphi}_{i}(\vec{x})
            = h \tilde{\chi}_{j}(t) \tilde{\varphi}_{j}(\vec{x}).
        \end{equation}
        Unter diesen Gegebenheiten betrachten wir nun die Differenz der zu $u(\bm\sigma_{h})$ und $u(\bm\sigma)$ zugehörigen Variationsprobleme.
        Für $v = (v_{1}, v_{2}) \in \mathcal Y$ gilt dann:
        \begin{align}
            0
            &= b(u(\bm\sigma_{h}), v; \bm\sigma_{h}) - b(u(\bm\sigma), v; \bm\sigma)
            \\&= \int_{I} \left[ \skp{u_{t}(\bm\sigma_{h}) - u_{t}(\bm\sigma)}{v_{1}}{V' \times V} + a(u(\bm\sigma_{h}), v_{1}; \bm\sigma_{h}) - a(u(\bm\sigma), v_{1}; \bm\sigma) \right] \diff t
            \\&\qquad + \skp{u(0; \bm\sigma_{h}) - u(0; \bm\sigma)}{v_{2}}{H}
            \\&=  h \int_{I} \left[ \skp{(\theta_{h})_{t}}{v_{1}}{V' \times V} + c\skp{\grad \theta_{h}}{\grad v_{1}}{H}  +  \mu \skp{\theta_{h}}{v_{1}}{H} \right] \diff t
            \\&\qquad + \int_{I} \left[ \skp{\omega(\bm\sigma_{h}) u(\bm\sigma_{h})}{v_{1}}{H} - \skp{\omega(\bm\sigma) u(\bm\sigma)}{v_{1}}{H}  \right] \diff t + h \skp{\theta_{h}(0)}{v_{2}}{H}
            \\&= h \int_{I} \left[ \skp{(\theta_{h})_{t}}{v_{1}}{V' \times V} + a(\theta_{h}, v_{1}; \bm\sigma)  \right] \diff t + h \skp{\theta_{h}(0)}{v_{2}}{H}
            \\&\qquad +\int_{I} \skp{(\omega(\bm\sigma_{h}) - \omega(\bm\sigma))u(\bm\sigma_{h})}{v_{1}}{H} \diff t
            \\&= h \cdot b(\theta_{h}, v; \bm\sigma) + h \int_{I} \tilde{\chi}_{j} \skp{\tilde{\varphi}_{j} u(\bm\sigma_{h})}{v_{1}}{H} \diff t.
        \end{align}
        Dies schreiben wir erneut in Form der Gleichung
        \begin{equation}
            \label{eq:existenz_partieller_ableitungen:beweis:variationsproblem}
            b(\theta_{h}, v; \bm\sigma) = F_{h}(v)
        \end{equation}
        mit der Abbildung
        \begin{equation}
            F_{h} \colon \mathcal Y \to \mathbb{R}, \quad F_{h}(v) := - \int_{I} \tilde{\chi}_{j} \skp{\tilde{\varphi}_{j}  u(\bm\sigma_{h})}{v_{1}}{H} \diff t.
        \end{equation}
        Vollkommen analog zum Beweis des vorherigen Lemmas kann gezeigt werden, dass $F_{h}$ ein stetiges lineares Funktional auf $\mathcal Y$ definiert, das heißt, $\theta_{h}$ ist die eindeutige Lösung des Variationsproblems \cref{eq:existenz_partieller_ableitungen:beweis:variationsproblem}.
        Weiter ist $h \mapsto F_{h}(\blank)$ stetig in $h = 0$, denn für festes $v \in \mathcal Y$ gilt unter Verwendung der Cauchy-Schwarz-Ungleichung die Abschätzung
        \begin{align}
            \abs{F_{h}(v) - F_{0}(v)}
            &= \abs[\Big]{\int_{I} \tilde{\chi}_{j} \skp{\tilde{\varphi}_{j}  (u(\bm\sigma_{h}) - u(\bm\sigma))}{v_{1}}{H} \diff t }
            \\&\leq \norm{\tilde{\varphi}_{j}}_{L_{\infty}(\Omega)} \abs{\skp{u(\bm\sigma_{h}) - u(\bm\sigma)}{v_{1}}{L_{2}(I; H)} }
            \\&\leq \norm{\tilde{\varphi}_{j}}_{L_{\infty}(\Omega)} \norm{u(\bm\sigma_{h}) - u(\bm\sigma)}_{L_{2}(I; H)} \norm{v_{1}}_{L_{2}(I; H)}
            \\&\leq \norm{\tilde{\varphi}_{j}}_{L_{\infty}(\Omega)} \norm{u(\bm\sigma_{h}) - u(\bm\sigma)}_{\mathcal X} \norm{v}_{\mathcal Y}.
        \end{align}
        Hier setzen wir mit der Stabilitätsaussage aus \cref{lemma:stabilitaetsaussage} an, um $\norm{u(\bm\sigma_{h}) - u(\bm\sigma)}_{\mathcal X}$ weiter abzuschätzen und erhalten
        \begin{equation}
            \begin{aligned}
                \norm{u(\bm\sigma_{h}) - u(\bm\sigma)}_{\mathcal X}
                &\leq \frac{\norm{f}_{\mathcal Y'}}{\beta^{2}} \norm{\omega(\bm\sigma_{h}) - \omega(\bm\sigma)}_{L_{\infty}(I; L_{\infty}(\Omega))}
                = \frac{\norm{f}_{\mathcal Y'}}{\beta^{2}} \norm{h \tilde{\chi}_{j} \tilde{\varphi}_{j} }_{L_{\infty}(I; L_{\infty}(\Omega))}
                \\&\leq \frac{\norm{f}_{\mathcal Y'}}{\beta^{2}} \abs{h} \norm{\tilde{\varphi}_{j}}_{L_{\infty}(\Omega)}.
            \end{aligned}
        \end{equation}
        Zusammen mit obiger Ungleichung liefert dies
        \begin{equation}
            \abs{F_{h}(v) - F_{0}(v)}
            \leq \norm{\tilde{\varphi}_{j}}^{2}_{L_{\infty}(\Omega)} \norm{v}_{\mathcal Y} \frac{\norm{f}_{\mathcal Y'}}{\beta^{2}} \abs{h} \to 0 \quad \text{für}~h \to 0.
        \end{equation}
        Das bedeutet, dass $F_{h} \to F_{0}$ in $\mathcal Y'$ für $h \to 0$ gilt,
        was insbesondere $\theta_{h} \to \theta_{0}$ in $\mathcal X$ für $h \to 0$ impliziert, da $\theta_{h}$ als Lösung des Variationsproblems \cref{eq:existenz_partieller_ableitungen:beweis:variationsproblem} nach \cref{korollar:parametrisches_rz_variationsproblem_sachgemaess} stetig von $F_{h}$ abhängt.
        Ferner ist durch $\partial^{\bm\nu}_{\bm\sigma} u(\bm\sigma) = \theta_{0}$ als Lösung von
        \begin{equation}
            \label{eq:existenz_partieller_ableitungen:beweis:variationsproblem_ableitung}
            \text{Finde}~\theta_{0} \in \mathcal X \text{ mit} \quad b(\theta_{0}, v; \bm\sigma) = - \int_{I} \tilde{\chi}_{j} \skp{\tilde{\varphi}_{j}  u(\bm\sigma)}{v_{1}}{H} \diff t \quad \fa v \in \mathcal Y;
        \end{equation}
        die Existenz und Wohldefiniertheit der gesuchten partiellen Ableitung gegeben.

        Die Ableitungen höherer Ordnung lassen sich auf gleiche Weise durch Anwendung der beschriebenen Schritte auf die Variationsformulierung \cref{eq:existenz_partieller_ableitungen:beweis:variationsproblem_ableitung} et cetera konstruieren.
    \end{Beweis}
\end{Satz}

\begin{Bemerkung}
\label{bemerkung:alternative_herleitung_variationsproblem_ableitung}
    Sei erneut ohne Einschränkung $\bm\nu = \vec e_{j} \in \mathcal{F}$ für ein $j \in \mathbb{N}$.
    Alternativ erhält man das Variationsproblem \cref{eq:existenz_partieller_ableitungen:beweis:variationsproblem_ableitung} für die partielle Ableitung $\partial^{\bm\nu}_{\bm\sigma} u(\bm\sigma)$ auch durch formales Differenzieren der Variationsformulierung \cref{eq:parametrisches_rz_variationsproblem} nach $\sigma_{j}$, denn es gilt
    \begin{align}
        \partial^{\bm\nu}_{\bm\sigma} b(u(\bm\sigma), v; \bm\sigma)
        &= \partial^{\bm\nu}_{\bm\sigma} \left( \int_{I} \left[ \skp{u_{t}(\bm\sigma)}{v_{1}}{V' \times V} + c \skp{\grad u(\bm\sigma)}{\grad v_{1}}{H} + \mu \skp{u(\bm\sigma)}{v_{1}}{H} \right.\right.
        \\&\qquad\qquad \left.\vphantom{\int_{I}}\left. + \skp{\omega(\bm\sigma) u(\bm\sigma)}{v_{1}}{H} \right] \diff t + \skp{u(0; \bm\sigma)}{v_{2}}{H} \right)
        \\&= \int_{I} \left[ \skp{\partial^{\bm\nu}_{\bm\sigma} u_{t}(\bm\sigma)}{v_{1}}{V' \times V}
            + \skp{\grad \partial^{\bm\nu}_{\bm\sigma} u(\bm\sigma)}{\grad v_{1}}{H} + \mu \skp{\partial^{\bm\nu}_{\bm\sigma} u(\bm\sigma)}{v_{1}}{H} \right.
        \\&\qquad \quad \left.+ \skp{\partial^{\bm\nu}_{\bm\sigma} \omega(\bm\sigma) u(\bm\sigma) + \omega(\bm\sigma) \partial^{\bm\nu}_{\bm\sigma} u(\bm\sigma)}{v_{1}}{H} \right] \diff t + \skp{\partial^{\bm\nu}_{\bm\sigma} u(0; \bm\sigma)}{v_{2}}{H}
        \\&= b(\partial^{\bm\nu}_{\bm\sigma} u(\bm\sigma), v; \bm\sigma) + \int_{I} \skp{\partial^{\bm\nu}_{\bm\sigma} \omega(\bm\sigma) u(\bm\sigma)}{v_{1}}{H} \diff t
        \\&= b(\partial^{\bm\nu}_{\bm\sigma} u(\bm\sigma), v; \bm\sigma) + \int_{I} \tilde{\chi}_{j} \skp{\tilde{\varphi}_{j} u(\bm\sigma)}{v_{1}}{H} \diff t
    \end{align}
    und ferner $\partial^{\bm\nu}_{\bm\sigma} f(v) = 0$.
    Daraus erhält man insgesamt erneut das Variationsproblem \cref{eq:existenz_partieller_ableitungen:beweis:variationsproblem_ableitung}.
    % \begin{equation}
    %     \text{Finde}~\partial^{\bm\nu}_{\bm\sigma} u(\bm\sigma) \in \mathcal X \colon \quad b(\partial^{\bm\nu}_{\bm\sigma} u(\bm\sigma), v; \bm\sigma) = - \int_{I} \tilde{\chi}_{j} \skp{\tilde{\varphi}_{j}  u(\bm\sigma)}{v_{1}}{H} \diff t \quad \fa v \in \mathcal Y.
    % \end{equation}
\end{Bemerkung}

\begin{Satz}
\label{satz:abschaetzung_norm_partieller_ableitungen}
    Sei $\vec b = (b_i)_{i \in \mathbb{N}} \in \mathbb{R}^{\mathbb{N}}$ die durch $b_{i} = \beta^{-1} \norm{\tilde{\varphi}_{i}}_{L_{\infty}(\Omega)}$ gegebene Folge, wobei $\beta$ die nach \cref{satz:raum_zeit_variationsformulierung_sachgemaess_gestellt} existierende parameterunabhängige inf-sup"=Konstante ist.
    Dann gilt
    \begin{equation}
        \label{eq:abschaetzung_norm_partieller_ableitungen}
        \sup_{\bm\sigma \in \mathcal P} \norm{\partial^{\bm\nu}_{\bm\sigma} u(\bm\sigma)}_{\mathcal X} \leq \frac{\norm{f}_{\mathcal Y'}}{\beta} \abs{\bm\nu}! \vec b^{\bm\nu} \quad \fa \bm \nu \in \mathcal F.
    \end{equation}

    \begin{Beweis}
        Wir beginnen damit, eine Darstellung der Variationsprobleme, welche von den partiellen Ableitungen erfüllt werden, herzuleiten.
        Diese lassen sich durch
        \begin{equation}
        \label{eq:abschaetzung_norm_partieller_ableitungen:rekursiv}
            b(\partial^{\bm\nu}_{\bm\sigma} u(\bm\sigma), v; \bm\sigma)
            = - \sum_{\Set{j \given \nu_{j} \neq 0}} \nu_{j} \int_{I} \tilde{\chi}_{j} \skp{\tilde{\varphi}_{j} \partial^{\bm\nu - \vec e_{j}}_{\bm\sigma} u(\bm\sigma)}{v_{1}}{H} \diff t
        \end{equation}
        rekursiv darstellen, was wir im Folgenden induktiv zeigen werden.

        Den Fall $\abs{\bm\nu} = 1$ haben wir in \cref{bemerkung:alternative_herleitung_variationsproblem_ableitung} bereits gezeigt.
        Sei nun also $\abs{\bm\nu} > 1$.
        Sei weiter $k \in \mathbb{N}$ ein Index mit $\nu_{k} > 0$.
        Dann definieren wir $\tilde{\bm\nu} := \bm\nu - \vec e_{k}$ und es gilt offenbar $\abs{\tilde{\bm\nu}} = \abs{\bm\nu} - 1$.
        Nach Induktionsvoraussetzung gilt damit
        \begin{equation}
            b(\partial^{\tilde{\bm\nu}}_{\bm\sigma} u(\bm\sigma), v; \bm\sigma) + \sum_{\Set{j \given \tilde{\nu}_{j} \neq 0}} \tilde{\nu}_{j} \int_{I} \tilde{\chi}_{j} \skp{\tilde{\varphi}_{j} \partial^{\tilde{\bm\nu} - \vec e_{j}}_{\bm\sigma} u(\bm\sigma)}{v_{1}}{H} \diff t = 0,
        \end{equation}
        wobei nach Definition $\nu_{j} = \tilde{\nu}_{j}$ für $j \neq k$ und $\tilde{\nu}_{k} = \nu_{k} - 1$ ist.
        Partielles Differenzieren dieser Gleichung nach $\sigma_{k}$ analog zu \cref{bemerkung:alternative_herleitung_variationsproblem_ableitung} liefert dann die Gleichung
        \begin{align}
            0 &=
                b(\partial^{\bm\nu}_{\bm\sigma} u(\bm\sigma), v; \bm\sigma)
           \\&\qquad          + \int_{I} \tilde{\chi}_{k} \skp{\tilde{\varphi}_{k} \partial^{\bm\nu - \vec e_{k}}_{\bm\sigma} u(\bm\sigma)}{v_{1}}{H} \diff t
                + (\nu_{k} - 1) \int_{I} \tilde{\chi}_{k} \skp{\tilde{\varphi}_{k} \partial^{\bm\nu - \vec e_{k}}_{\bm\sigma} u(\bm\sigma) }{v_{1}}{H} \diff t
           \\&\qquad     + \sum_{\Set{j \neq k \given \nu_{j} \neq 0}} \nu_{j} \int_{I} \tilde{\chi}_{j} \skp{\tilde{\varphi}_{j} \partial^{\bm\nu - \vec e_{j}}_{\bm\sigma} u(\bm\sigma)}{v_{1}}{H} \diff t,
        \end{align}
        welche nach Zusammenfassen der Summanden Gleichung \cref{eq:abschaetzung_norm_partieller_ableitungen:rekursiv} entspricht.

        Für die rechte Seite von \cref{eq:abschaetzung_norm_partieller_ableitungen:rekursiv} können wir wie zuvor nachweisen, dass es sich um ein stetiges lineares Funktional auf $\mathcal Y$ handelt.
        Wir können also \cref{satz:raum_zeit_variationsformulierung_sachgemaess_gestellt} verwenden, um die Abschätzung
        \begin{equation}
            \norm{\partial^{\bm\nu}_{\bm\sigma} u(\bm\sigma)}_{\mathcal X} \leq \frac{1}{\beta} \norm[\Big]{\sum_{\Set{j \given \nu_{j} \neq 0}} \nu_{j} \int_{I} \tilde{\chi}_{j} \skp{\tilde{\varphi}_{j} \partial^{\bm\nu - \vec e_{j}}_{\bm\sigma} u(\bm\sigma)}{v_{1}}{H} \diff t}_{\mathcal Y'}
        \end{equation}
        zu erhalten.
        Wir wollen nun die $\mathcal Y'$-Norm der rechten Seite weiter abschätzen.
        Dazu verwenden wir erneut die Cauchy"=Schwarz"=Ungleichung und erhalten wegen
        \begin{align}
            &\abs[\Big]{\sum_{\Set{j \given \nu_{j} \neq 0}} \nu_{j} \int_{I} \tilde{\chi}_{j} \skp{\tilde{\varphi}_{j} \partial^{\bm\nu - \vec e_{j}}_{\bm\sigma} u(\bm\sigma)}{v_{1}}{H} \diff t}
            \\\leq~
            &\sum_{\Set{j \given \nu_{j} \neq 0}} \nu_{j} \abs[\Big]{\int_{I} \tilde{\chi}_{j} \skp{\tilde{\varphi}_{j} \partial^{\bm\nu - \vec e_{j}}_{\bm\sigma} u(\bm\sigma)}{v_{1}}{H} \diff t}
            \\\leq~
            &\sum_{\Set{j \given \nu_{j} \neq 0}} \nu_{j} \norm{\tilde{\varphi}_{j}}_{L_{\infty}(\Omega)} \norm{\partial^{\bm\nu - \vec e_{j}}_{\bm\sigma} u(\bm\sigma)}_{L_{2}(I; H)} \norm{v_{1}}_{L_{2}(I; H)}
            \\\leq~
            &\sum_{\Set{j \given \nu_{j} \neq 0}} \nu_{j} \norm{\tilde{\varphi}_{j}}_{L_{\infty}(\Omega)} \norm{\partial^{\bm\nu - \vec e_{j}}_{\bm\sigma} u(\bm\sigma)}_{\mathcal X} \norm{v}_{\mathcal Y},
        \end{align}
        weiter die Abschätzung
        \begin{equation}
            \label{eq:abschaetzung_norm_partieller_ableitungen:rekursive_schranke}
            \norm{\partial^{\bm\nu}_{\bm\sigma} u(\bm\sigma)}_{\mathcal X} \leq \sum_{\Set{j \given \nu_{j} \neq 0}} \nu_{j} \frac{\norm{\tilde{\varphi}_{j}}_{L_{\infty}(\Omega)}}{\beta} \norm{\partial^{\bm\nu - \vec e_{j}}_{\bm\sigma} u(\bm\sigma)}_{\mathcal X}.
        \end{equation}

        Um nun die eigentliche Behauptung zu beweisen, verfolgen wir erneut einen Induktionsansatz.
        Sei zunächst $\abs{\bm\nu} = 0$.
        Dann entspricht
        \begin{equation}
            \sup_{\bm \sigma \in \mathcal P} \norm{u(\bm\sigma)}_{\mathcal X} \leq \frac{\norm{f}_{\mathcal Y'}}{\beta}
        \end{equation}
        Ungleichung \cref{eq:abschaetzung_norm_partieller_ableitungen} und ist nach \cref{satz:raum_zeit_variationsformulierung_sachgemaess_gestellt} erfüllt.
        Sei also nun $\abs{\bm\nu} > 0$.
        Dann gilt für die rekursive Darstellung \cref{eq:abschaetzung_norm_partieller_ableitungen:rekursive_schranke} unter Verwendung der Induktionsvoraussetzung \cref{eq:abschaetzung_norm_partieller_ableitungen} für $\norm{\partial^{\bm\nu - \vec e_{j}}_{\bm\sigma} u(\bm\sigma)}_{\mathcal X}$ die Abschätzung
        \begin{align}
            \norm{\partial^{\bm\nu}_{\bm\sigma} u(\bm\sigma)}_{\mathcal X}
            &\leq
            \sum_{\Set{j \given \nu_{j} \neq 0}} \nu_{j} b_{j} \norm{\partial^{\bm\nu - \vec e_{j}}_{\bm\sigma} u(\bm\sigma)}_{\mathcal X}
            \\&\leq
            \sum_{\Set{j \given \nu_{j} \neq 0}} \nu_{j} b_{j} \frac{\norm{f}_{\mathcal Y'}}{\beta} \abs{\bm\nu - \vec e_{j}}! \vec b^{\bm\nu - \vec e_{j}}
            \\&=
            \bigg( \sum_{\Set{j \given \nu_{j} \neq 0}} \nu_{j} \bigg) \bigg( \frac{\norm{f}_{\mathcal Y'}}{\beta} (\abs{\bm\nu} - 1)! \vec b^{\bm\nu} \bigg)
            \\&=
            \frac{\norm{f}_{\mathcal Y'}}{\beta} \abs{\bm\nu}! \vec b^{\bm\nu}
         \end{align}
         und damit die Behauptung.
    \end{Beweis}
\end{Satz}

Bevor zusätzliche Annahmen notwendig werden, um die benötigten Aussagen zu beweisen, wird an dieser Stelle zunächst erläutert, wie die obigen Aussagen zum Nachweis der analytischen Abhängigkeit der Lösung vom Parameter beitragen.

\begin{Definition}
\label{definition:analytisch}
    Wir nennen die Abbildung $\mathcal P \ni \bm \sigma \mapsto u(\bm \sigma) \in \mathcal X$ \emph{analytisch}, wenn sie in jedem $\bm\sigma_{0} \in \mathcal P$ als lokal gleichmäßig konvergente Potenzreihe
    \begin{equation}
        \sum_{\bm \nu \in \mathcal F} t_{\bm \nu} (\bm \sigma - \bm \sigma_{0})^{\bm \nu}
    \end{equation}
    dargestellt werden kann.
\end{Definition}

Ist die Abbildung analytisch, dann entspricht die Potenzreihe gerade ihrer Taylorreihe.
Diese Eigenschaft wollen wir ausnutzen, denn in diesem Fall sind die Koeffizienten $t_{\bm \nu}$ gerade durch
\begin{equation}
    t_{\bm \nu} = \frac{1}{\bm\nu!} \partial^{\bm \nu}_{\bm \sigma} u(\bm \sigma_{0})
\end{equation}
gegeben.

Betrachten wir beispielsweise die Taylorreihe von $u(\blank)$ um den Nullpunkt $\vec 0 \in \mathcal P$, dann erhalten wir wegen $\mathcal P = [-1, 1]^{\mathbb{N}}$ die Abschätzung
\begin{equation}
    \sup_{\bm\sigma \in \mathcal P} \norm[\Big]{\sum_{\bm\nu \in \mathcal F} t_{\bm\nu} \bm\sigma^{\bm\nu}}_{\mathcal X}
    \leq \sup_{\bm\sigma \in \mathcal P} \sum_{\bm\nu \in \mathcal F} \norm{t_{\bm\nu} \bm\sigma^{\bm\nu}}_{\mathcal X}
    \leq \sum_{\bm\nu \in \mathcal F} \norm{t_{\bm\nu}}_{\mathcal X}
    \leq \sum_{\bm \nu \in \mathcal F} \frac{1}{\bm\nu!} \norm{ \partial^{\bm \nu}_{\bm \sigma} u(\vec 0)}_{\mathcal X},
\end{equation}
beziehungsweise nach \cref{satz:abschaetzung_norm_partieller_ableitungen} weiter
\begin{equation}
    \sup_{\bm\sigma \in \mathcal P} \norm{\sum_{\bm\nu \in \mathcal F} t_{\bm\nu} \bm\sigma^{\bm\nu}}_{\mathcal X}
    \leq \frac{\norm{f}_{\mathcal Y'}}{\beta} \sum_{\bm \nu \in \mathcal F} \frac{\abs{\bm\nu}!}{\bm\nu!} \vec b^{\bm\nu}.
\end{equation}
Die Frage, ob die Abbildung $\mathcal P \ni \bm \sigma \mapsto u(\bm \sigma) \in \mathcal X$ analytisch ist, hat sich somit auf jene, unter welchen Bedingungen $\sum_{\bm \nu \in \mathcal F} (\abs{\bm\nu}!)(\bm\nu!)^{-1} \vec b^{\bm\nu}$ konvergiert, reduziert.
Wann dies der Fall ist, wurde beispielsweise in \cite[Theorem 7.2]{Cohen:2010kz} untersucht.
Wir geben die Aussage hier ohne Beweis wieder.

\begin{Satz}
\label{satz:cohen2010kz:theorem72}
    Sei $0 < p \leq 1$.
    Die Folge $(\frac{\abs{\bm\nu}!}{\bm\nu!} \vec b^{\bm\nu})_{\bm\nu \in \mathcal F}$ liegt genau dann in $\ell_{p}(\mathcal F)$, wenn $\norm{\vec b}_{\ell_{1}(\mathbb{N})} < 1$ und $\vec b \in \ell_{p}(\mathbb{N})$ gilt.
\end{Satz}

\begin{Satz}
\label{satz:loesungen_analytisch}
    Das Funktionensystem $\Set{\varphi_{i}}_{i \in \mathbb{N}}$ sei so gewählt, dass $\vec b \in \ell_{1}(\mathbb{N})$, definiert als $b_{i} := \beta^{-1} \norm{\varphi_{i}}_{L_{\infty}(\Omega)}$, die Bedingung $\norm{\vec b}_{\ell_{1}(\mathbb{N})} < 1$ erfüllt.
    Dann hängt die Lösung $u(\bm \sigma)$ des parametrischen Raum"=Zeit"=Variationsproblems \cref{eq:parametrisches_rz_variationsproblem} analytisch vom Parameter $\bm \sigma \in \mathcal P$ ab.

    \begin{Beweis}
        Der vorherige Satz und vorangegangene Überlegungen liefern unter diesen Voraussetzungen die behauptete Aussage.
    \end{Beweis}
\end{Satz}

Die Bedingung $\norm{\vec b}_{\ell_{1}(\mathbb{N})} < 1$ kann dabei so interpretiert werden, dass der feldabhängige Teil der partiellen Differentialgleichung den feldunabhängigen Anteil nicht zu stark stören darf, um analytische Abhängigkeit garantieren zu können.
Dies verträgt sich allerdings nur bedingt mit der Motivation der Propagator"=Differentialgleichung, da bei dieser die Felder einen starken Einfluss haben.

In \cref{chapter:galerkin} werden wir auf diesen Punkt noch einmal eingehen und anhand der numerischen Ergebnisse deutlich machen, inwiefern die Bedingung eingehalten wird beziehungsweise eingehalten werden kann.

\section{Periodische Randbedingungen} % (fold)
\label{section:periodische_randbedingungen}

Da wir in diesem Kapitel bisher ausschließlich mit homogenen Dirichlet"=Randbedingungen gearbeitet haben, wollen wir an dieser Stelle auf den Fall periodischer Randbedingungen eingehen.
Dabei werden wir feststellen, dass auf Grund der Struktur der Propagator-Differentialgleichung nur sehr geringe Unterschiede zum betrachteten homogenen Fall bestehen.

Zunächst müssen wir die Rahmenbedingungen für die Betrachtung periodischer Randbedingungen festlegen.
Dazu beschränken wir uns auf den Fall, dass $\Omega = \bigtimes_{i = 1}^{n} (0, l_{i}) \subset \mathbb{R}^{n}$ ein beschränkter offener Quader ist, wobei $l_{i} \in \mathbb{R}_{+}$ für $i = 1 \dots n$ sei.
Weiter führen wir nun die Äquivalente des Lebesgue-Raums $L_{2}(\Omega)$ und des Sobolev-Raums $H^{1}(\Omega)$ für periodische Funktionen ein.
Da dies für die diese Arbeit von untergeordneter Bedeutung ist, wird an dieser Stelle nur ein Überblick über die benötigten Ergebnisse gegeben.
Genauere Ausführungen findet man beispielsweise bei \textcite{Han2009}.

\begin{Definition}
\label{definition:periodische_sobolev_raeume}
    Sei $\mathcal C_{\mathrm{per}}^{\infty}(\Omega) \subset \mathcal C^{\infty}(\mathbb{R}^{n})$ die Teilmenge der glatten $\Omega$-periodischen Funktionen.
    Als den Lebesgue-Raum $\Omega$-periodischer Funktionen $L_{2,\mathrm{per}}(\Omega)$  definieren wir den Abschluss von $C_{\mathrm{per}}^{\infty}(\Omega)$ bezüglich der $L_{2}$-Norm.
    Weiter definieren wir den Sobolev-Raum $\Omega$-periodischer Funktionen $H^{1}_{\mathrm{per}}(\Omega)$ als den Abschluss von $C_{\mathrm{per}}^{\infty}(\Omega)$ bezüglich der $H^{1}$-Norm.
\end{Definition}

Diese Räume verwenden wir nun, um die Räume $V$ und $H$ (vergleiche \cref{bemerkung:raeume_und_gelfand_tripel}) zu definieren.
Nach Konstruktion handelt es sich bei $H^{1}_{\mathrm{per}}(\Omega)$ und $L_{2,\mathrm{per}}(\Omega)$ um Hilberträume und desweiteren ist $H^{1}_{\mathrm{per}}(\Omega)$ ein dichter Unterraum von $L_{2,\mathrm{per}}(\Omega)$.
Wählen wir also $V := H^{1}_{\mathrm{per}}(\Omega)$ und $H := L_{2,\mathrm{per}}(\Omega)$, dann erhalten wir nach \cref{definition:gelfand_tripel} wie zuvor ein Gelfand-Tripel der Form
\begin{equation}
    V \denseinclusion H \cong H' \denseinclusion V' = (H^{1}_{\mathrm{per}}(\Omega))'.
\end{equation}

Untersucht man die Ausführungen dieses Kapitels für den Fall homogener Dirichlet-Randbedingungen, dann stellt man fest, dass lediglich die aus \cref{annahme:eigenschaften_der_bilinearform_a} stammenden Bedingungen an die Bilinearformen $a(\blank, \blank; t)$, deren Darstellung durch den Wechsel zu periodischen Randbedingungen unverändert bleibt, nachgewiesen werden müssen.

\begin{Lemma}
\label{lemma:bilinearform_periodisch_messbar_stetig_garding}
    Sei $\Set{a(\blank, \blank; t)}_{t \in I}$ die Familie von Bilinearformen aus \cref{definition:operator_bilinearform_zeit}, wobei die Hilberträume als $V = H^{1}_{\mathrm{per}}(\Omega)$ und $H = L_{2,\mathrm{per}}(\Omega)$ gegeben seien.
    Diese Bilinearformen erfüllen die folgenden Eigenschaften:
    \begin{thmenumerate}
        \item \label{lemma:bilinearform_periodisch_messbar_stetig_garding:messbar}
        \emph{Messbarkeit:} Die Abbildung $I \ni t \mapsto a(\eta, \zeta; t)$ ist messbar für alle $\eta, \zeta \in V$.
        \item\label{lemma:bilinearform_periodisch_messbar_stetig_garding:stetig}
        \emph{Stetigkeit:} Es gilt
        \begin{equation}
            \label{eq:bilinearform_periodisch_messbar_stetig_garding:stetig}
            \abs{a(\eta, \zeta; t)} \leq \gamma_{a} \norm{\eta}_{V} \norm{\zeta}_{V} \quad \fa \eta, \zeta \in V \text{ und fast alle } t \in I
        \end{equation}
        mit Stetigkeitskonstante $\gamma_{a} = \max\Set{c, \norm{\omega}_{L_{\infty}(I; L_{\infty}(\Omega))} + \abs{\mu}} < \infty$.
        \item\label{lemma:bilinearform_periodisch_messbar_stetig_garding:garding}
        \emph{G\aa{}rding-Ungleichung:} Es gilt
        \begin{equation}
            \label{eq:bilinearform_periodisch_messbar_stetig_garding:garding}
            a(\eta, \eta; t) + \lambda \norm{\eta}_{H}^{2} \geq \alpha \norm{\eta}_{V}^{2} \quad \fa \eta \in V \text{ und fast alle } t \in I
        \end{equation}
        mit $\alpha = c > 0$ und $\lambda = \max\Set{\norm{\omega}_{L_{\infty}(I;L_{\infty}(\Omega))} - \mu + c, c} \geq c > 0$.
    \end{thmenumerate}

    \begin{Beweis}
        Die Nachweise der Messbarkeit und Stetigkeit bleiben unverändert wie in \cref{satz:bilinearform_messbar_stetig_garding} und werden hier nicht wiederholt.

        Für die G\aa{}rding-Ungleichung sei zunächst $\eta \in V$ beliebig und $\lambda \in \mathbb{R}$.
        Dann gilt
        \begin{align}
            a(\eta, \eta; t) + \lambda \norm{\eta}^{2}_{H}
            &= c \norm{\grad \eta}^{2}_{H} + \skprod{\omega(t, \blank) \eta}{\eta}_{H} + \mu \skprod{\eta}{\eta}_{H} + \lambda \skprod{\eta}{\eta}_{H}
            \\&= c \norm{\grad \eta}^{2}_{H} + \skprod{\omega(t, \blank) + \mu + \lambda) \eta}{\eta}_{H}.
        \end{align}
        Wählen wir $\lambda := \max\Set{\norm{\omega}_{L_{\infty}(I; L_{\infty}(\Omega))} - \mu + c, c} \geq c > 0$, dann gilt $\omega(t, \blank) + \mu + \lambda - c \geq 0$ fast überall in $\Omega$ und wir erhalten die Abschätzung
        \begin{align}
            a(\eta, \eta; t) + \lambda \norm{\eta}^{2}_{H}
            &= c \norm{\grad \eta}^{2}_{H} + \skprod{(\omega(t, \blank) + \mu + \lambda - c) \eta}{\eta}_{H} + c \norm{\eta}^{2}_{H} \\
            &\geq c \norm{\grad \eta}^{2}_{H} + c \norm{\eta}^{2}_{H}
            \\&= c \norm{\eta}^{2}_{V}.
        \end{align}
    \end{Beweis}
\end{Lemma}

Auf diesem Lemma aufbauend können die restlichen Ergebnisse des Kapitels analog auch auf den periodischen Fall übertragen werden.

\end{document}

    % -*- root: ../main.tex -*-

\documentclass[../main.tex]{subfiles}
\begin{document}

\chapter{Petrov"=Galerkin"=Verfahren} % (fold)
\label{chapter:galerkin}

Mit diesem Kapitel beginnt der numerische Anteil dieser Arbeit.
Wir führen zunächst das sogenannte Petrov"=Galerkin"=Verfahren ein, welches als Grundlage für die Reduzierte"=Basis"=Methode des nächsten Kapitels dienen wird.
Wie bereits bei den funktionalanalytischen Grundlagen in \cref{chapter:grundlagen}, wird dies zu Beginn allgemein gehalten und erst später auf die Problemstellung aus \cref{chapter:propagator_differentialgleichung} zugeschnitten.
Als Quellen für dieses Kapitel wurden vor allem die Arbeiten von \textcite{Nochetto:2009il} sowie \textcite{Braess:2007wm} verwendet.

Wir beginnen nun, indem wir die Rahmenbedingungen in Form eines abstrakten Variationsproblems festlegen.
Seien $\mathcal X$ und $\mathcal Y$ zwei reelle Hilberträume und seien weiter $b \colon \mathcal X \times \mathcal Y \to \mathbb{R}$ eine stetige Bilinearform und $f \colon \mathcal Y \to \mathbb{R}$ ein stetiges lineares Funktional.
Das \emph{abstrakte Variationsproblem} sei gegeben durch:
\begin{equation}\label{eq:galerkin_abstraktes_variationsproblem}
    \text{Finde}~u \in \mathcal X \text{ mit} \quad  b(u, v) = f(v) \quad \fa v \in \mathcal Y.
\end{equation}
Hinreichende Bedingungen für die Existenz und Eindeutigkeit einer Lösung haben wir bereits in \cref{chapter:grundlagen} gesehen, weswegen wir für den Rest dieses Kapitels annehmen, dass es sich dabei um ein korrekt gestelltes Problem handelt und insbesondere die inf-sup-Konstante $\beta$ die Bedingung
\begin{equation}\label{eq:galerkin_abstraktes_variationsproblem_infsup_bedingung}
    \beta := \infsup{u \in \mathcal X}{v \in \mathcal Y} \frac{b(u, v)}{\norm{u}_{\mathcal X} \norm{v}_{\mathcal Y}} > 0
\end{equation}
erfüllt.


\section{Grundlagen} % (fold)
\label{section:petrov_galerkin_grundlagen}

Als \emph{Petrov"=Galerkin"=Verfahren} werden diejenigen Galerkin"=Verfahren bezeichnet, die auf Variationsprobleme mit nicht-übereinstimmendem Ansatz- und Testraum $\mathcal X$ und $\mathcal Y$ zugeschnitten sind.

Wie es bei Galerkin"=Verfahren üblich ist, wird eine Diskretisierung des Variationsproblems \cref{eq:galerkin_abstraktes_variationsproblem} durch Approximation der im Allgemeinen unendlichdimensionalen Hilberträume mittels $\mathcal N$- beziehungsweise $\mathcal M$-dimensionaler Unterräume $\mathcal X_{\mathcal N} \subset \mathcal X$ und $\mathcal Y_{\mathcal M} \subset \mathcal Y$ erreicht.
Zwar ist auch für den Fall $\mathcal M > \mathcal N$ eine sinnvolle Formulierung einer Diskretisierung möglich, diese erfolgt dann allerdings im Sinne einer Residuum-Minimierung, wie beispielsweise bei \cite{Andreev:2012ep}.
Da sich dies nicht ohne Weiteres mit der beabsichtigten Verwendung als Grundlage für eine Reduzierte"=Basis"=Methode verträgt, beschränken wir uns auf den Fall $\mathcal N = \mathcal M$.

Wir orientieren uns an \cite[Section 3.1]{Nochetto:2009il}, beschränken uns aber auf eine minimale Einführung.

\begin{Definition}\label{definition:disrekte_loesung}
    Seien $\mathcal X_{\mathcal N} \subset \mathcal X$ und $\mathcal Y_{\mathcal N} \subset \mathcal Y$ Unterräume der Dimension $\mathcal N \in \mathbb{N}$.
    Als \emph{Petrov"=Galerkin"=Lösung} von \cref{eq:galerkin_abstraktes_variationsproblem} bezeichnen wir eine Lösung $u_{\mathcal N} \in \mathcal X_{\mathcal N}$ des Variationsproblems:
    \begin{equation}\label{eq:diskretes_variationsproblem}
        \text{Finde}~u_{\mathcal N} \in \mathcal X_{\mathcal N} \text{ mit} \quad  b(u_{\mathcal N}, v) = f(v) \quad \fa v \in \mathcal Y_{\mathcal N}.
    \end{equation}
\end{Definition}

Zur einfacheren Unterscheidung bezeichnen wir \cref{eq:galerkin_abstraktes_variationsproblem} im Weiteren als stetiges und \cref{eq:diskretes_variationsproblem} als diskretes Variationsproblem.

\begin{Bemerkung}\label{bemerkung:zur_wohldefiniertheit}
    Anders als bei den Galerkin"=Verfahren für elliptische Probleme führt eine solche Diskretisierung eines korrekt gestellten Problems nicht automatisch zu einem korrekt gestellten diskreten Problem.
    Wegen der Ungleichung
    \begin{equation}
        \sup_{v \in \mathcal Y} \frac{b(u, v)}{\norm{v}_{\mathcal Y}} \geq \sup_{v \in \mathcal Y_{\mathcal N}} \frac{b(u, v)}{\norm{v}_{\mathcal Y}} \quad \fa u \in \mathcal X
    \end{equation}
    ist die stetige inf-sup-Bedingung \cref{eq:galerkin_abstraktes_variationsproblem_infsup_bedingung} kein hinreichendes Kriterium für die Gültigkeit der diskreten inf-sup-Bedingung
    \begin{equation}\label{eq:diskrete_inf_sup_kosntante}
        \beta_{\mathcal N} := \infsup{u \in \mathcal X_{\mathcal N}}{v \in \mathcal Y_{\mathcal N}} \frac{b(u, v)}{\norm{u}_{\mathcal X} \norm{v}_{\mathcal Y}} > 0.
    \end{equation}
    Allerdings gilt aufgrund der endlichen Dimension von $\mathcal X_{\mathcal N}$ und $\mathcal Y_{\mathcal N}$ stets die Gleichheit
    \begin{equation}
        \infsup{u \in \mathcal X_{\mathcal N}}{v \in \mathcal Y_{\mathcal N}} \frac{b(u, v)}{\norm{u}_{\mathcal X} \norm{v}_{\mathcal Y}} = \infsup{v \in \mathcal Y_{\mathcal N}}{u \in \mathcal X_{\mathcal N}} \frac{b(u, v)}{\norm{u}_{\mathcal X} \norm{v}_{\mathcal Y}}.
    \end{equation}
    Die Stetigkeit dagegen muss nicht explizit nachgewiesen werden, da die diskreten Stetigkeitskonstanten stets durch die des stetigen Variationsproblems von oben beschränkt werden.
\end{Bemerkung}

Dass die diskrete inf-sup-Konstante bei den Petrov"=Galerkin"=Verfahren eine wichtige Rolle spielt, wird unter anderem bei den folgenden beiden Aussagen deutlich.

\begin{Satz}\label{satz:galerkin_stabilitaet}
    Gilt die diskrete inf-sup-Bedingung \cref{eq:diskrete_inf_sup_kosntante}, dann erfüllt die Petrov"=Galerkin"=Lösung $u_{\mathcal N} \in \mathcal X_{\mathcal N}$ die Abschätzung
    \begin{equation}\label{eq:galerkin_statibilitaet}
        \norm{u_{\mathcal N}}_{\mathcal X} \leq \frac{1}{\beta_{\mathcal N}} \norm{f}_{\mathcal Y_{\mathcal N}'}.
    \end{equation}

    \begin{Beweis}
        Folgt direkt aus dem \acl{bnb}, \cref{satz:bnb_theorem}.
    \end{Beweis}
\end{Satz}

\begin{Satz}[Lemma von Céa]\label{satz:lemma_von_cea}
    Seien $u \in \mathcal X$ die Lösung des stetigen Variationsproblems \cref{eq:galerkin_abstraktes_variationsproblem} und $u_{\mathcal N} \in \mathcal X_{\mathcal N}$ die diskrete Lösung von \cref{eq:diskretes_variationsproblem}.
    Sei weiter $\gamma$ die Stetigkeitskonstante der Bilinearform $b$ aus \cref{eq:galerkin_abstraktes_variationsproblem}.
    Der Fehler $u - u_{\mathcal N} \in \mathcal X$ erfüllt die Ungleichung
    \begin{equation}\label{eq:lemma_von_cea}
        \norm{u - u_{\mathcal N}}_{\mathcal X} \leq \frac{\gamma}{\beta_{\mathcal N}} \inf_{w \in \mathcal X_{\mathcal N}} \norm{u - w}_{\mathcal X}.
    \end{equation}

    \begin{Beweis}
        Siehe \cite[Theorem 3.2]{Nochetto:2009il}.
    \end{Beweis}
\end{Satz}

Durch Verfeinerung der Diskretisierung will man üblicherweise erreichen, dass die Abweichung zwischen stetiger und diskreter Lösung kleiner wird.
Um dies aus dem Lemma von Céa zu erhalten, wird der folgende Stabilitätsbegriff notwendig.

\begin{Definition}\label{definition:stabile_diskretisierung}
    Sei $\Set{(\mathcal X_{\mathcal N}, \mathcal Y_{\mathcal N})}_{\mathcal N \geq 1}$ eine Folge von endlichdimensionalen Unterräumen mit zugehörigen diskreten inf-sup-Konstanten $\Set{\beta_{\mathcal N}}_{\mathcal N \geq 1}$.
    Wir nennen diese Diskretisierungen \emph{stabil}, wenn ein $\beta_{\mathrm{LB}} > 0$ mit
    \begin{equation}
        \inf_{\mathcal N \geq 1} \beta_{\mathcal N} \geq \beta_{\mathrm{LB}} > 0
    \end{equation}
    existiert.
\end{Definition}

Abschließend führen wir eine alternative Darstellung der inf-sup-Konstante \cref{eq:galerkin_abstraktes_variationsproblem_infsup_bedingung} ein, welche später insbesondere die numerische Berechnung von $\beta_{\mathcal N}$ ermöglichen wird.

\begin{Definition}\label{definition:supremizing_operator}
    Der \emph{Supremizing-Operator} $T \colon \mathcal X \to \mathcal Y$ der Bilinearform $b$ aus \cref{eq:galerkin_abstraktes_variationsproblem} sei definiert über die Gleichung
    \begin{equation}
        \skp{Tu}{v}{\mathcal Y} = b(u, v) \quad \fa u \in \mathcal X \text{ und } v \in \mathcal Y.
    \end{equation}
\end{Definition}

\begin{Lemma}\label{lemma:supremizing_operator}
    Der Supremizing-Operator $T$ ist wohldefiniert, linear und stetig.
    Ferner gelten die Gleichungen
    \begin{equation}
        \label{eq:supremizing_operator_ist_argsup}
        Tu = \argsup_{v \in \mathcal Y} \frac{b(u, v)}{\norm{v}_{\mathcal Y}}
        \qquad\text{und}\qquad
        \beta = \inf_{u \in \mathcal X} \frac{\norm{Tu}_{\mathcal Y}}{\norm{u}_{\mathcal X}}.
    \end{equation}

    \begin{Beweis}
        Wohldefiniertheit, Linearität und Stetigkeit ergeben sich aus der Definition und dem Rieszschen Darstellungssatz.
        Die erste Gleichung folgt mit der Anwendung der Cauchy-Schwarz-Ungleichung auf
        \begin{equation}
            \frac{b(u, v)}{\norm{v}_{\mathcal Y}}
            = \frac{\skp{Tu}{v}{\mathcal Y}}{\norm{v}_{\mathcal Y}}
            \leq \frac{\norm{Tu}_{\mathcal Y} \norm{v}_{\mathcal Y}}{\norm{v}_{\mathcal Y}}
            = \norm{Tu}_{\mathcal Y}
        \end{equation}
        und der Gleichung
        \begin{equation}
            \frac{b(u, Tu)}{\norm{Tu}_{\mathcal Y}}
            = \frac{\skp{Tu}{Tu}{\mathcal Y}}{\norm{Tu}_{\mathcal Y}}
            = \norm{Tu}_{\mathcal Y}.
        \end{equation}
        Dies impliziert, dass das Supremum von $b$ bezüglich des zweiten Arguments gerade von $v = Tu$ angenommen wird.
        Daraus folgt auch die zweite Gleichung, denn es gilt
        \begin{equation}
            \beta
            = \infsup{u \in \mathcal X}{v \in \mathcal Y} \frac{b(u, v)}{\norm{u}_{\mathcal X} \norm{v}_{\mathcal Y}}
            = \inf_{u \in \mathcal X} \frac{b(u, Tu)}{\norm{u}_{\mathcal X} \norm{Tu}_{\mathcal Y}}
            = \inf_{u \in \mathcal X} \frac{\skp{Tu}{Tu}{\mathcal Y}}{\norm{u}_{\mathcal X}\norm{Tu}_{\mathcal Y}}
            = \inf_{u \in \mathcal X} \frac{\norm{Tu}_{\mathcal Y}}{\norm{u}_{\mathcal X}}.
        \end{equation}
    \end{Beweis}
\end{Lemma}
Ferner erhalten wir damit auch die nützliche Darstellung
\begin{equation}\label{eq:quadrat_der_inf_sup_konstante_ueber_supremizer}
    \beta^{2} = \inf_{u \in \mathcal X} \frac{\skp{Tu}{Tu}{\mathcal Y}}{\norm{u}_{\mathcal X}^{2}}.
\end{equation}


\section{Raum-Zeit-Diskretisierung} % (fold)
\label{section:raum_zeit_diskretisierung}

Wir kehren nun zu der Raum"=Zeit"=Variationsformulierungen \cref{eq:raum_zeit_variationsformulierung} und \cref{eq:parametrisches_rz_variationsproblem} der Propagator"=Differentialgleichung aus \cref{chapter:propagator_differentialgleichung} zurück und wollen diese mit Hilfe eines Petrov"=Galerkin"=Verfahrens diskretisieren.
Dies erfordert die Konstruktion endlichdimensionaler Unterräume $\mathcal X_{\mathcal N}$ und $\mathcal Y_{\mathcal N}$.
Anschließend werden wir für diese eine hinreichende Bedingung für die Stabilität im Sinne von \cref{definition:stabile_diskretisierung} angeben.

Dieser Abschnitt orientiert sich an den Ausführungen von \textcite{Andreev:2012uh,Andreev:2012ep} und nutzt die Charakterisierung der Bochner"=Sobolev"=Räume als Hilbertraum"=Tensorprodukte nach \cref{satz:bochner_sobolev_raum_als_tensorprodukt}, um die Konstruktion so in einen rein zeitlichen und einen rein räumlichen Anteil zu zerlegen.

\begin{Korollar}\label{korollar:ansatz_und_testraum_als_tensorprodukt}
    Die Hilberträume $\mathcal X$ und $\mathcal Y$ aus \cref{eq:ansatzraum_X,eq:testraum_Y} lassen sich schreiben als
    \begin{equation}\label{eq:ansatzraum_testraum_tensor}
        \mathcal X \cong (L_2(I) \otimes V) \cap (H^{1}(I) \otimes V'),
        \qquad
        \mathcal Y \cong (L_{2}(I) \otimes V) \times H.
    \end{equation}
\end{Korollar}

Für den zeitlichen Anteil werden wir die beiden endlichdimensionalen Unterräume $E_{\mathcal K} \subset H^{1}(I) \subset L_{2}(I)$ und $F_{\mathcal K} \subset L_{2}(I)$ mit den Dimensionen $\dim E_{\mathcal K} = \mathcal K + 1$ und $\dim F_{\mathcal K} = \mathcal K$ verwenden.
Die Räume $V, H$ und $V'$ der räumlichen Komponente können aufgrund der Gelfand"=Tripel"=Struktur
durch den selben endlichdimensionalen Raum $V_{\mathcal J} \subset V$ diskretisiert werden.
Dieser wird die Dimension $\dim V_{\mathcal J} = \mathcal J$ haben.

Diese Teilräume liefern dann zusammen mit der Tensorprodukt-Darstellung \cref{eq:ansatzraum_testraum_tensor} die Diskretisierungen
\begin{equation}
\label{eq:diskrete_tensor_raueme}
    \mathcal X_{\mathcal N} := E_{\mathcal K} \otimes V_{\mathcal J}, \qquad \mathcal Y_{\mathcal N} := (F_{\mathcal K} \otimes V_{\mathcal J}) \times V_{\mathcal J}
\end{equation}
mit der Dimension
\begin{equation}
    \mathcal N := \dim \mathcal X_{\mathcal N} = (\mathcal K + 1) \mathcal J = \mathcal K \mathcal J + \mathcal J = \dim \mathcal Y_{\mathcal N}.
\end{equation}

Wir merken an dieser Stelle an, dass wir stets die Skalarprodukte und die jeweiligen induzierten Normen von $\mathcal X$ und $\mathcal Y$ auf die jeweiligen Unterräume übertragen und für diese weiter die bekannten Bezeichnungen verwenden.

\paragraph{Zeitliche Komponente.} % (fold)
\label{par:zeitliche_komponente}

Hierfür benötigen wir zunächst eine Diskretisierung des Zeitintervalls $I = [0, T]$ in Form eines nicht notwendigerweise äquidistanten Gitters
\begin{equation}\label{eq:zeitgitter}
    \mathcal T_{\mathcal K} := \Set{0 = t_0 < t_1 < \dots < t_{\mathcal K - 1} < t_{\mathcal K} = T} \subset I.
\end{equation}
Die Diskretisierung $E_{\mathcal K}$ des Ansatzraumes basiert auf stetigen, stückweise affinen Funktionen, genauer den klassischen Hutfunktionen $\theta_{k}$ auf den Gitterpunkten $t_{k} \in \mathcal T_{\mathcal K}$ für $k = 0, \dots, \mathcal K$.
Diese erfüllen $\theta_{k}(t_{\tilde{k}}) = \delta_{k \tilde k}$, wobei $\delta_{k \tilde k}$ das bekannte Kronecker-Delta sei.
Wir fassen diese Hutfunktionen zu einer Basis $\mathcal B^{E}_{\mathcal K} := \Set{ \theta_{k} \given k = 0, \dots, \mathcal K }$ zusammen und definieren $E_{\mathcal K} := \spn \mathcal B^{E}_{\mathcal K}$.

Für den Testraum-Anteil $F_{\mathcal K}$ verwenden wir stattdessen stückweise konstante Funktionen, die als charakteristische Funktionen $\xi_{k} := \chi_{(t_{k-1}, t_{k})}$ der Teilintervalle $(t_{k - 1}, t_{k}) \subset I$ mit $k = 1, \dots, \mathcal K$ gegeben sind.
Diese seien zu der Basismenge $\mathcal B^{F}_{\mathcal K} := \Set{ \xi_{k} \given k = 1, \dots, \mathcal K}$ zusammengefasst und weiter definieren wir $F_{\mathcal K} := \spn \mathcal B^{F}_{\mathcal K}$.

Diese Wahl von Ansatz- und Testraum-Anteil führt nach \cite{Andreev:2012ep} zu einem Crank"=Nicolson"=ähnlichen Verfahren, welches dementsprechend auch als Time"=Stepping"=Verfahren aufgefasst werden kann.


\paragraph{Räumliche Komponente.} % (fold)
\label{par:raeumliche_komponente}

Hier wollen wir nur die verwendete Notation festlegen, da für den räumlichen Anteil die meisten von Galerkin"=Verfahren bekannten Ansätze, beispielsweise Finite"=Elemente oder globale Basisfunktionen, verwendet werden können.

Wir definieren erneut eine endliche Basismenge $\mathcal B^{V}_{\mathcal J} := \Set{ \eta_{j} \given j = 1, \dots, \mathcal J}$ und darauf aufbauend den endlichdimensionalen Raum $V_{\mathcal J} := \spn \mathcal B^{V}_{\mathcal J}$.


\paragraph{Raum-Zeit-Diskretisierung.} % (fold)
\label{par:raum_zeit_diskretisierung}

Unter Verwendung der Tensorprodukt-Darstellung \cref{eq:ansatzraum_testraum_tensor} können wir nun die beiden einzeln betrachteten Komponenten zu den endlichdimensionalen Raum"=Zeit"=Unterräumen zusammensetzen.
Dazu definieren wir zunächst die Basen
\begin{equation}
\label{eq:diskreter_ansatzraum_basis_phi}
\label{eq:diskreter_testraum_basis_psi1}
    \mathcal B^{\mathcal X}_{\mathcal N} := \Set{\theta \otimes \eta \given \theta \in \mathcal B^{E}_{\mathcal K}, \eta \in \mathcal B^{V}_{\mathcal J}},
    \qquad
    \mathcal B^{\mathcal Y_{1}}_{\mathcal K \mathcal J} := \Set{\xi \otimes \eta \given \xi \in \mathcal B^{F}_{\mathcal K}, \eta \in \mathcal B^{V}_{\mathcal J}}.
\end{equation}
und weiter
\begin{equation}
\label{eq:diskreter_testraum_basis_psi}
    \mathcal B^{\mathcal Y}_{\mathcal N} := (\mathcal B^{\mathcal Y_{1}}_{\mathcal K \mathcal J} \times \Set{ 0 }) \cup (\Set{0} \times \mathcal B^{V}_{\mathcal J}).
\end{equation}
Nach Konstruktion und den Definitionen \cref{eq:diskrete_tensor_raueme} gilt nun
\begin{equation}
    \label{eq:diskreter_ansatz_und_testraum_als_span}
    \mathcal X_{\mathcal N} = E_{\mathcal K} \otimes V_{\mathcal J} = \spn \mathcal B^{\mathcal X}_{\mathcal N},
    \quad
    \mathcal Y_{\mathcal N} = (F_{\mathcal K} \otimes V_{\mathcal J}) \times V_{\mathcal J} = \spn \mathcal B^{\mathcal Y}_{\mathcal N}.
\end{equation}


\paragraph{Stabilität der Diskretisierung.} % (fold)
\label{par:stabilit_t_der_diskretisierung}

Wie bereits erwähnt, wollen wir eine hinreichende Bedingung für die Stabilität der beschriebenen Diskretisierung im Sinne von \cref{definition:stabile_diskretisierung} angeben.
Dies können wir an dieser Stelle nur für die nicht"=parametrische Variationsformulierung \cref{eq:raum_zeit_variationsformulierung} in Form einer knappen Übersicht ohne Beweise und tiefergehende Motivation abhandeln, weswegen auf die zugrundeliegende Arbeit \cite[Section 5.2]{Andreev:2012ep} verwiesen sei.

Die hier konstruierte Diskretisierung führt hauptsächlich zu einer Bedingung an die räumliche Komponente $V_{\mathcal J}$.
Bevor wir diese angeben können, benötigen wir die nachfolgende Definition nach \cite[62]{Andreev:2012ep}, die in etwas anderer Form oft bei der Behandlung von hyperbolischen partiellen Differentialgleichungen auftritt.

\begin{Definition}\label{definiton:cfl_zahl}
    Es seien $V_{\mathcal J} \subset V$ ein endlichdimensionaler Unterraum, $\mathcal T_{\mathcal K}$ ein Gitter des Zeitintervalls $I$ und weiter durch $\max \Delta \mathcal T_{\mathcal K} := \max_{k = 1, \dots, \mathcal K} \abs{t_{k} - t_{k - 1}}$ die maximale Schrittweite des Zeitgitters gegeben.
    Wir bezeichnen die Größe
    \begin{equation}
        \label{eq:cfl_zahl}
        \mathrm{CFL}_{\mathcal N} := \max \Delta \mathcal T_{\mathcal K} \sup_{\eta \in V_{\mathcal J} \setminus \Set{0} } \frac{\norm{\eta}_{V}}{\norm{\eta}_{V'}}
    \end{equation}
    als \emph{Courant-Friedrichs-Levi-Zahl}, kurz \emph{CFL-Zahl}, der Diskretisierung $(\mathcal X_{\mathcal N}, \mathcal Y_{\mathcal N})$ aus \cref{eq:diskreter_ansatz_und_testraum_als_span}.
\end{Definition}

\begin{Bemerkung}
    Ist $V_{\mathcal J}$ endlichdimensional, so gilt insbesondere $\mathrm{CFL}_{\mathcal N} < \infty$.
\end{Bemerkung}

Weiter führen wir eine gewisse inf"=sup"=Konstante ein, welche in \cite[57]{Andreev:2012ep} aus der Konstruktion einer äquivalenten Norm auf $\mathcal X$ resultiert, die dann für den Stabilitätsnachweis \cite[Theorem 5.2.6]{Andreev:2012ep} verwendet wird.
Dazu benötigen wir die Operatorfamilie $\Set{A(t)}_{t \in I}$ aus \cref{eq:operator_zeit}, von welcher wir weiter fordern, dass sie selbstadjungiert ist und die G\aa{}rding"=Ungleichung mit $\lambda = 0$ erfüllt.
Ersteres gilt nach \cref{lemma:operator_selbstadjungiert}, während letzteres durch \cref{lemma:transformation_zu_elliptischem_operator} erreicht werden kann.
Unter diesen Voraussetzungen garantiert der Satz von Lax-Milgram \cite[Section 6.2.1]{evans2010partial} die stetige Invertierbarkeit von $A(t)$.
Ferner erlaubt es uns die Definition zweier Skalarprodukte beziehungsweise Normen durch
\begin{equation}
    \begin{aligned}
        \skp{v}{\tilde{v}}{+} &:= \int_{I} \skp{A(t)v(t)}{\tilde{v}(t)}{V' \times V} \diff t,
        \quad&
        \norm{v}^{2}_{+} &:= \skp{v}{v}{+},
        \qquad
        v, \tilde{v} \in L_{2}(I; V), \\
        \skp{z}{\tilde{z}}{-} &:= \int_{I} \skp{A(t)^{-1} z(t)}{\tilde z(t)}{V \times V'} \diff t,
        \quad&
        \norm{z}_{-}^{2} &:= \skp{z}{z}{-},
        \qquad z, \tilde z \in L_{2}(I; V').
    \end{aligned}
\end{equation}

Mit dieser Vorarbeit können wir nun die angesprochene inf"=sup"=Konstante nach \cite[57]{Andreev:2012ep} als
\begin{equation}
    \beta_{\pm}(\mathcal X_{\mathcal N}, \mathcal Y_{\mathcal N}) := \infsup{u \in \mathcal X_{\mathcal N}}{v \in \mathcal Y_{\mathcal N}} \frac{\int_{I} \skp{u_t(t)}{v_{1}(t)}{V' \times V} \diff t}{\norm{u_t}_{-} \norm{v_{1}}_{+}}
\end{equation}
definieren, wobei Infimum und Supremum bezüglich aller Elemente gebildet werden, für die der Nenner nicht Null wird.

Unter Verwendung der CFL"=Zahl und dieser inf"=sup"=Konstante lässt sich letztendlich die folgende Stabilitätsaussage nachweisen.

\begin{Satz}\label{satz:untere_schranke_fuer_infsup_nach_andreev}
    Seien $\Set{(\mathcal X_{\mathcal N}, \mathcal Y_{\mathcal N})}_{\mathcal N \geq 1}$ Diskretisierungen der Form \cref{eq:diskreter_ansatz_und_testraum_als_span}.
    Gelten die Bedingungen
    \begin{equation}
        \sup_{\mathcal N \geq 1} \mathrm{CFL}_{\mathcal N} < \infty
        \quad \text{und} \quad
        \inf_{\mathcal N \geq 1} \beta_{\pm}(\mathcal X_{\mathcal N}, \mathcal Y_{\mathcal N}) > 0,
    \end{equation}
    dann gilt für die diskrete inf"=sup"=Konstante $\beta_{\mathcal N}$ aus \cref{eq:diskrete_inf_sup_kosntante} die Abschätzung
    \begin{equation}
        \beta_{\mathcal N} \geq c_{0} \min\Set{1, \beta_{\pm}(\mathcal X_{\mathcal N}, \mathcal Y_{\mathcal N})} \min\Set{1, \mathrm{CFL}_{\mathcal N}^{-1}} \quad \text{für } \mathcal N \geq 1
    \end{equation}
    mit einer von $\mathcal X_{\mathcal N}$ und $\mathcal Y_{\mathcal N}$ unabhängigen Konstanten $c_{0} > 0$.
    \begin{Beweis}
        Siehe \cite[Subsection 5.2.2]{Andreev:2012ep}.
    \end{Beweis}
\end{Satz}

Da eine ausführlichere Bearbeitung der Stabilität respektive obiger Stabilitätsaussage den Rahmen dieser Arbeit übersteigt, werden wir erst am Ende dieses Kapitels dazu zurückkehren und die Stabilität anhand von Beispielen numerisch überprüfen.


\section{Numerische Umsetzung} % (fold)
\label{section:galerkin_numerische_umsetzung}

Nachdem das theoretische Fundament für die Diskretisierung gelegt ist, widmen wir uns nun der tatsächlichen Anwendung auf die Variationsformulierung der Propagator"=Differentialgleichung.
Um die in \cref{section:parametrische_formulierung} durchgeführte Parametrisierung dieser numerisch umsetzen zu können, müssen wir zunächst den dort noch unendlichdimensionalen Parameterraum $\mathcal P$ auf einen endlichdimensionalen einschränken.
Da wir in Einklang mit den Bedingungen aus \cref{chapter:propagator_differentialgleichung} bleiben wollen, erreichen wir dies durch die Wahl einer Dimension $N_{\mathcal P} \in \mathbb{N}$ und der Entwicklungsfunktionen $\varphi_{i} = 0$ für alle $i > N_{\mathcal P}$.

Wir wiederholen an dieser Stelle das parametrische Raum"=Zeit"=Variationsproblem noch einmal in aller Kürze.
Seien $\mathcal X$ und $\mathcal Y$ die Ansatz- respektive Testräume aus \cref{eq:ansatzraum_X,eq:testraum_Y}.
Weiter seien die Anzahl der Parameter durch $N_{\mathcal P} \in \mathbb{N}$, der Parameterraum $\mathcal P = [-1, 1]^{N_\mathcal P}$ und die Feld"=Entwicklungsfunktionen durch die Menge
\begin{equation}
    \mathcal B^{\omega}_{N_{\mathcal P}} := \Set{ \varphi_{i} \in L_{\infty}(\Omega) \given i = 1, \dots, N_{\mathcal P}}
\end{equation}
gegeben.
Das parametrische Raum"=Zeit"=Variationsproblem lautet damit:
\begin{equation}\label{eq:wiederholung_variationsproblem}
    \text{Sei }\bm \sigma \in \mathcal P, \text{ finde } u(\bm \sigma) \in \mathcal X \text{ mit} \quad b(u(\bm \sigma), v; \bm \sigma) = f(v) \quad \fa v \in \mathcal Y,
\end{equation}
wobei $b$ und $f$ wie in \cref{definition:parametrische_rz_variationsformulierung} gegeben seien.

\begin{Lemma}
\label{lemma:bilinearform_affin_parametrisch}
    Die parametrische Bilinearform aus \cref{eq:parametrisches_rz_variationsproblem:lhs} ist unter diesen Gegebenheiten affin vom Parameter $\bm \sigma \in \mathcal P$ abhängig.
    Genauer gilt
    \begin{equation}\label{eq:bilinearform_affin_parametrisch}
        b(\blank, \blank; \bm \sigma) = b_{0}(\blank, \blank) + \sum_{i = 1}^{N_{\mathcal P}} \sigma_{i} b_{i}(\blank, \blank)
    \end{equation}
    mit parameterunabhängigen stetigen Bilinearformen $b_{i} \colon \mathcal X \times \mathcal Y \to \mathbb{R}$, $i = 0, \dots, N_{\mathcal P}$.

    \begin{Beweis}
        Ohne Einschränkung schreiben wir das parametrische Raum"=Zeit"=Feld $\omega$ in der Darstellung
        \begin{equation}
            \omega \colon I \times \Omega \times \mathcal P \to \mathbb{R}, \quad
            \omega(t, \vec{x}; \bm \sigma) = \sum_{i = 1}^{N_{\mathcal P}} \sigma_{i} \chi_{i}(t) \varphi_{i}(\vec x)
        \end{equation}
        mit charakteristischen Funktionen $\chi_{i} \in \Set{\chi_{I_{1}}, \chi_{I_{2}}}$ für $i = 1, \dots, N_{\mathcal P}$.
        Wir definieren die Bilinearformen $b_{i} \colon \mathcal X \times \mathcal Y \to \mathbb{R}$ durch
        \begin{equation}
            \begin{aligned}
                b_{0} &:= \int_{I} \left[ \skp{u_{t}}{v_{1}}{V' \times V} + c\skp{\grad u}{\grad v_{1}}{H} + \mu \skp{u}{v_{1}}{H} \right] \diff t + \skp{u(0)}{v_{2}}{H},\\
                b_{i} &:= \int_{I} \chi_{i} \skp{\varphi_{i}u}{v_{1}}{H} \diff t, \quad i = 1, \dots, N_{\mathcal P},
            \end{aligned}
        \end{equation}
        wobei die Stetigkeit durch einfache Abschätzungen folgt und hier nicht weiter ausgeführt wird.
        Unter Verwendung dieser gilt nun offenbar
        \begin{align}
                b(u, v; \bm \sigma)
                = \int_{I} \left[ \skp{u_{t}}{v_{1}}{V' \times V} + a(u, v_{1}; \bm \sigma) \right] \diff t + \skp{u(0)}{v_{2}}{H}
                = b_{0}(u, v) + \sum_{i = 1}^{N_{\mathcal P}} \sigma_{i} b_{i}(u, v).
        \end{align}
    \end{Beweis}
\end{Lemma}

Wir beginnen die Herleitung der diskreten Darstellung des Variationsproblems \cref{eq:wiederholung_variationsproblem} durch die Definition der Elemente des diskreten Ansatzraums $\mathcal X_{\mathcal N}$ und des Testraums $\mathcal Y_{\mathcal N}$.

\begin{Definition}
\label{definition:diskrete_ansatz_und_testfunktionen}
    Als \emph{diskrete Ansatzfunktion} $u_{\mathcal N} \in \mathcal X_{\mathcal N}$ bezeichnen wir eine Linearkombination der Form
    \begin{equation}
    \label{eq:darstellung_diskrete_ansatzfunktion}
        u_{\mathcal N} := \sum_{k = 0}^{\mathcal K} \sum_{j = 1}^{\mathcal J} u_{j}^{k} (\theta_{k} \otimes \eta_{j})
    \end{equation}
    mit dem durch
    \begin{equation}
        \vec{u}_{\mathcal N} := [\vec{u}^{k}_{\bullet}]_{k = 0, \dots, \mathcal K} := [ u^{k}_{j} ]_{j = 1, \dots, \mathcal J;k = 0, \dots, \mathcal K}
    \end{equation}
    definierten zugehörigen Koeffizientenvektor $\vec{u}_{\mathcal N} \in \mathbb{R}^{\mathcal N}$.

    Ebenso sei eine \emph{diskrete Testfunktion} $v_{\mathcal N} = (y_{\mathcal K \mathcal J}, z_{\mathcal J}) \in \mathcal Y_{\mathcal N}$ als das Paar der Linearkombinationen
    \begin{equation}
    \label{eq:darstellung_diskrete_testfunktion}
        y_{\mathcal K \mathcal J} := \sum_{k = 1}^{\mathcal K} \sum_{j = 1}^{\mathcal J} y_{j}^{k} (\xi_{k} \otimes \eta_{j}), \qquad
        z_{\mathcal J} := \sum_{l = 1}^{\mathcal J} z_{l} \eta_{l}
    \end{equation}
    definiert,
    wobei die Koeffizientenvektoren $\vec{y}_{\mathcal K \mathcal J} \in \mathbb{R}^{\mathcal K \mathcal J}$ und $\vec{z}_{\mathcal J} \in \mathbb{R}^{\mathcal J}$ durch
    \begin{equation}
        \vec{y}_{\mathcal K \mathcal J} := [\vec{y}^{k}_{\bullet}]_{k = 1, \dots, \mathcal K} := [y^{k}_{j}]_{j, = 1, \dots, \mathcal J; k = 1, \dots, \mathcal K}, \qquad
        \vec{z}_{\mathcal J} := [z_{l}]_{l = 1, \dots, \mathcal J}
    \end{equation}
    gegeben sind.
\end{Definition}

Ferner definieren wir an dieser Stelle vorbereitend einige Gramsche Matrizen, welche sich aus dem rein zeitlichen beziehungsweise räumlichen Anteil der Diskretisierung gewinnen lassen.
Diese werden uns später als Bausteine zur Berechnung der Raum"=Zeit"=Systeme dienen.

\begin{Definition}
\label{definition:zeitliche_bausteine}
    Die zeitlichen Gramschen Matrizen $\mat{M}_{t}^{FE}, \mat{C}_{t}^{FE}, \mat{M}_{t,i}^{FE} \in \mathbb{R}^{\mathcal K \times (\mathcal K + 1)}$ des $L_{2}$"=Skalarprodukts von Ansatz- und Testraumbasisfunktionen seien definiert als
    \begin{align}
        \mat{M}_{t}^{FE} &:= \big[ \skp{\theta_{k}}{\xi_{l}}{L_{2}(I)} \big]_{\subalign{l &= 1, \dots, \mathcal K \cr k &= 0, \dots, \mathcal K}},
        \qquad
        \mat{C}_{t}^{FE} := \big[ \skp{\theta_{k}'}{\xi_{l}}{L_{2}(I)} \big]_{\subalign{l &= 1, \dots, \mathcal K \cr k &= 0, \dots, \mathcal K}},
        \\
        \mat{M}_{t,i}^{FE} &:= \big[ \skp{\chi_{i}\theta_{k}}{\xi_{l}}{L_{2}(I)} \big]_{\subalign{l &= 1, \dots, \mathcal K \cr k &= 0, \dots, \mathcal K}} \quad \text{für } i = 1, \dots, N_{\mathcal P}.
    \end{align}
    Weiter seien durch $\mat{M}_{t}^{E}, \mat{A}_{t}^{E} \in \mathbb{R}^{(\mathcal K + 1)\times (\mathcal K + 1)}$ und $\mat{M}_{t}^{F} \in \mathbb{R}^{\mathcal K \times \mathcal K}$ die Gramschen Matrizen
    \begin{align}
        \mat{M}_{t}^{E} &:= \big[ \skp{\theta_{k}}{\theta_{l}}{L_{2}(I)}\big]_{k,l = 0, \dots, \mathcal K},
        \qquad
        \mat{A}_{t}^{E} := \big[ \skp{\theta_{k}'}{\theta_{l}'}{L_{2}(I)}\big]_{k,l = 0, \dots, \mathcal K},
        \\
        \mat{M}_{t}^{F} &:= \big[ \skp{\xi_{k}}{\xi_{l}}{L_{2}(I)}\big]_{k,l = 1, \dots, \mathcal K}
    \end{align}
    definiert sowie ein Zeilenvektor $\vec{e}^{E}_{t} \in \mathbb{R}^{1 \times (\mathcal K + 1)}$ durch
    $\vec{e}^{E}_{t} := \big[ \theta_{k}(0) \big]_{k = 0, \dots, \mathcal K}$.
\end{Definition}

\begin{Definition}
\label{definition:raeumliche_bausteine}
    Die räumlichen Gramschen Matrizen $\mat{H}_{x}, \mat{A}_{x}, \mat{V}_{x}, \mat{W}_{x,i} \in \mathbb{R}^{\mathcal J \times \mathcal J}$ seien als
    \begin{equation}
    \begin{aligned}
        \mat{H}_{x} &:= \left[ \skp{\eta_{k}}{\eta_{l}}{H} \right]_{k,l=1, \dots, \mathcal J},&
        \quad
        \mat{A}_{x} &:= \left[ \skp{\grad \eta_{k}}{\grad \eta_{l}}{H} \right]_{k,l=1, \dots, \mathcal J},
        \\
        \mat{V}_{x} &:= \left[ \skp{\eta_{k}}{\eta_{l}}{V} \right]_{k,l = 1, \dots, \mathcal J},&
        \quad
        \mat{W}_{x,i} &:= \left[ \skp{\varphi_{i} \eta_{k}}{\eta_{l}}{H} \right]_{k, l = 1, \dots, \mathcal J} \quad \text{für } i = 1, \dots, N_{\mathcal P},
    \end{aligned}
    \end{equation}
    definiert.
\end{Definition}

Da nach \cref{bemerkung:raeume_und_gelfand_tripel} das $V$-Skalarprodukt durch $\skp{\blank}{\blank}{V} = \skp{\blank}{\blank}{H} + \skp{\grad\blank}{\grad\blank}{H}$ gegeben ist, gilt insbesondere $\mat V_{x} = \mat H_{x} + \mat A_{x}$.

Als letzten Teil der Vorbereitung beweisen wir das folgende allgemein gehalten Lemma, welches eine Möglichkeit liefert, die Rieszsche Darstellung eines Funktionals zu bestimmen.

\begin{Lemma}\label{lemma:berechnung_der_rieszschen_darstellung}
    Sei $X$ ein endlichdimensionaler Hilbertraum mit Basis $\Set{ \phi_{i} }_{i = 1, \dots, N }$.
    Weiter seien $g \in X'$ ein stetiges lineares Funktional und $v \in X$ dessen Rieszsche Darstellung, es gilt also $\skp{g}{w}{X' \times X} = \skp{v}{w}{X}$ für alle $w \in X$.
    Dann ist der Koeffizientenvektor $\vec{v} \in \mathbb{R}^{N}$ von $v$ gegeben durch das lineare Gleichungssystem
    \begin{equation}
        \mat{X} \vec{v} = \vec{g},
    \end{equation}
    wobei $\mat{X} := [\skp{\phi_{i}}{\phi_{j}}{X}]_{i,j = 1, \dots, N} \in \mathbb{R}^{N \times N}$ und $\vec{g} := [\skp{g}{\phi_{i}}{X' \times X}]_{i = 1, \dots N}$ seien.

    \begin{Beweis}
        Einsetzen einer beliebigen Testfunktion $w = \sum_{i = 1}^{N} w_{i} \phi_{i}$ liefert
        \begin{equation}
            \skp{g}{w}{X' \times X}
            = \sum_{i = 1}^{N} w_i \skp{g}{\phi_{i}}{X' \times X}
            = \vec{w}\tran \vec{g}
            = \vec{w}\tran \mat{X} \vec{v}
            = \skp{\sum_{i = 1}^{N} w_i \phi_{i}}{\sum_{j = 1}^{N} v_j \phi_{j}}{X}
            = \skp{v}{w}{X}.
        \end{equation}
    \end{Beweis}
\end{Lemma}

Mit Hilfe dieser Vorarbeit können nun alle benötigten Raum"=Zeit"=Objekte für die Diskretisierung hergeleitet werden.
Wir beginnen mit dem Hauptanteil, der durch die Bilinearform erzeugten Systemmatrix, fahren danach mit dem Lastvektor fort und schließen mit der diskreten Darstellung der Normen und einer Möglichkeit zur Berechnung der diskreten inf"=sup"=Konstante ab.

Um die Notation möglichst knapp zu halten, verzichten wir sowohl auf die Argumente der Funktionen als auch auf das Symbol $\otimes$ für das Tensorprodukt.

\paragraph{Systemmatrix.} % (fold)
\label{par:systemmatrix}

Wir leiten die diskrete Darstellung für die einzelnen parameterunabhängigen Bausteine der affin-parametrischen Darstellung \cref{eq:bilinearform_affin_parametrisch} her.
Dazu werten wir die Bilinearformen für die Basisfunktionen $\theta_{k} \otimes \eta_{j} \in \mathcal X_{\mathcal N}$ sowie $(\xi_{m} \otimes \eta_{l}, 0), (0, \eta_{n}) \in \mathcal Y_{\mathcal N}$ aus und vereinfachen unter Verwendung der Tensorprodukt-Struktur soweit möglich.
Im Fall von $b_{0}$ führt dies zu
\begin{align}
    \phantom{=}&~ b_{0}(\theta_{k} \otimes \eta_{j}, (\xi_{m} \otimes \eta_{l}, 0))
    \\=&~ \int_{I} \left[ \skp{\theta_{k}' \eta_{j}}{\xi_{m} \eta_{l}}{V' \times V} + c \skp{\theta_{k} \grad \eta_{j}}{\xi_{m} \grad \eta_{l}}{H} + \mu \skp{\theta_{k} \eta_{j}}{\xi_{m} \eta_{l}}{H} \right] \diff t
    \\=&~ \skp{\theta_{k}'}{\xi_{m}}{L_{2}(I)} \skp{\eta_{j}}{\eta_{l}}{H} + c \skp{\theta_{k}}{\xi_{m}}{L_{2}(I)}\skp{\grad \eta_{j}}{\grad \eta_{l}}{H} + \mu \skp{\theta_{k}}{\xi_{m}}{L_{2}(I)} \skp{\eta_{j}}{\eta_{l}}{H}
\end{align}
und
\begin{equation}
    b_{0}(\theta_{k} \otimes \eta_{j}, (0, \eta_{n}))
    = \skp{\theta_{k}(0) \eta_{j}}{\eta_{n}}{H}
    = \delta_{k, 0} \skp{\eta_{j}}{\eta_{n}}{H}.
\end{equation}
Für die restlichen $b_{i}$ mit $i = 1, \dots, N_{\mathcal P}$ ergibt sich
\begin{equation}
    b_{i}(\theta_{k} \otimes \eta_{j}, (\xi_{m} \otimes \eta_{l}, 0))
    = \int_{I} \chi_{i} \skp{\varphi_{i} \theta_{k} \eta_{j}}{\xi_{m} \eta_{l}}{H} \diff t
    = \skp{\chi_{i} \theta_{k}}{\xi_{m}}{L_{2}(I)} \skp{\varphi_{i} \eta_{j}}{\eta_{l}}{H}
\end{equation}
sowie
\begin{equation}
    b_{i}(\theta_{k} \otimes \eta_{j}, (0, \eta_{n})) = 0.
\end{equation}
Diese Darstellungen lassen sich nun mit Hilfe der zuvor definierten Gramschen Matrizen und des üblichen Kronecker-Produkts $\otimes$ zu Matrizen $\mat{B}_{i} \in \mathbb{R}^{\mathcal N \times \mathcal N}$, $i = 0, \dots, N_{\mathcal P}$, der Form
\begin{equation}
    \mat{B}_{0} := \begin{bmatrix}
    \mat{C}_{t}^{FE} \otimes \mat{H}_{x} + c\,\mat{M}_{t}^{FE} \otimes \mat{A}_{x} + \mu \mat{M}_{t}^{FE} \otimes \mat{H}_{x} \\
    \vec{e}_{t}^{E} \otimes \mat{H}_{x}
    \end{bmatrix},
    \quad
    \mat{B}_{i} :=  \begin{bmatrix}
    \mat{M}_{t,i}^{FE} \otimes \mat{W}_{x,i} \\
    \vec{0}
    \end{bmatrix}
\end{equation}
zusammenfassen.
Dadurch gilt nun für die diskreten Ansatz- und Testfunktionen aus \cref{definition:diskrete_ansatz_und_testfunktionen} offenbar die Gleichung
\begin{equation}
    b_{i}(u_{\mathcal N}, (y_{\mathcal K \mathcal J}, z_{\mathcal J})) = (\vec{y}_{\mathcal K \mathcal J}; \vec{z}_{\mathcal J})\tran \mat{B}_{i} \vec{u}_{\mathcal N}, \quad \text{für } i = 0, \dots, N_{\mathcal P}.
\end{equation}
Weiter gilt die affin-parametrische Darstellung \cref{eq:bilinearform_affin_parametrisch} auch für diese Matrizen, sodass die Systemmatrix $\mat B(\bm \sigma) \in \mathbb{R}^{\mathcal N \times \mathcal N}$ für ein $\bm \sigma \in \mathcal P$ durch
\begin{equation}
    \mat{B}(\bm \sigma) := \mat{B}_{0} + \sum_{i = 1}^{N_{\mathcal P}} \sigma_{i} \mat{B}_{i}
\end{equation}
ausgewertet werden kann.


\paragraph{Lastvektor.} % (fold)
\label{par:lastvektor}

Als nächsten Schritt leiten wir eine diskrete Darstellung der rechten Seite des Variationsproblems \cref{eq:wiederholung_variationsproblem} her.
Diese hat nach Definition die Form
\begin{equation}
    f(v) = \int_{I} \skp{g}{v_{1}}{V' \times V} \diff t + \skp{u_{0}}{v_{2}}{H}.
\end{equation}
Da der Quellterm $g \in L_{2}(I; V')$ im Allgemeinen nicht in rein zeit- und raumabhängige Funktionen zerlegt werden kann, wenden wir aufgrund der verwendeten stückweise konstanten Basisfunktionen für den Testraum in der zeitlichen Komponente eine Trapezformel als numerische Quadratur an.
Dies führt für das Tupel $(\xi_{k} \otimes \eta_{j}, \eta_{l}) \in \mathcal Y_{\mathcal N}$ zu folgender Darstellung
\begin{align}
    f((\xi_{k} \otimes \eta_{j}, \eta_{l}))
    &= \int_{I} \skp{g}{\xi_{k} \eta_{j}}{V' \times V} \diff t + \skp{u_{0}}{\eta_{l}}{H}
    \\&= \tfrac{1}{2} \Delta t_{k} \skp{g(t_{k}) + g(t_{k - 1})}{\eta_{j}}{V' \times V}
         + \skp{u_{0}}{\eta_{l}}{H},
\end{align}
wobei $\Delta t_{k} := t_{k} - t_{k - 1}$ die Schrittweite des Zeitgitters sei.
Wir definieren die Vektoren $\vec{u}_{0}, \vec{g}^{k - 1/2} \in \mathbb{R}^{\mathcal J}$ für $k = 1, \dots, \mathcal K$ als
\begin{equation}
    \vec{u}_{0} := \left[ \skp{u_{0}}{\eta_{l}}{H} \right]_{l = 1, \dots, \mathcal J },
    \qquad
    \vec{g}^{k - 1/2} := \left[ \tfrac{1}{2} \skp{g(t_{k}) + g(t_{k - 1})}{\eta_{j}}{V' \times V}  \right]_{j = 1, \dots, \mathcal J}.
\end{equation}
Die hier vorkommenden Skalarprodukte lassen sich nun mittels geeigneter Quadraturformeln auswerten, so dass wir den diskreten Lastvektor $\vec{f} \in \mathbb{R}^{\mathcal N}$ als
\begin{equation}
    \vec{f} := \begin{bmatrix}
        \vec{g}^{1/2} \\
        \vdots\\
        \vec{g}^{\mathcal K - 1/2}\\
        \vec{u}_{0}
    \end{bmatrix}
\end{equation}
definieren können.

\paragraph{Bestimmung der diskreten Lösung.} % (fold)
\label{par:bestimmung_der_diskreten_l_sung_}

Die beschriebene Systemmatrix $\mat{B}(\bm \sigma)$ und der Lastvektor $\vec f$ reichen bereits aus, um die Petrov"=Galerkin"=Lösung $u_{\mathcal N}(\bm \sigma)$ des Variationsproblems \cref{eq:wiederholung_variationsproblem} zu bestimmen.
Dazu muss lediglich der Koeffizientenvektor $\vec{u}_{\mathcal N}(\bm \sigma)$ der Lösung durch das Gleichungssystem
\begin{equation}
    \mat{B}(\bm \sigma) \vec{u}_{\mathcal N}(\bm \sigma) = \vec f
\end{equation}
bestimmt werden.

\paragraph{Normen.} % (fold)
\label{par:normen}

Weiter benötigen wir eine Möglichkeit, die Norm einer diskreten Ansatzfunktion beziehungsweise Testfunktion mit Hilfe des entsprechenden Koeffizientenvektors zu bestimmen.
Dazu sei zunächst an das norminduzierende $\mathcal X$-Skalarprodukt
\begin{equation}
    \tag*{\cref{eq:ansatzraum_skalarprodukt}}
    \skp{u}{v}{\mathcal X} = \skp{u}{v}{L_{2}(I; V)} + \skp{u_t}{v_t}{L_2(I; V')} \quad \text{für } u, v \in \mathcal X
\end{equation}
erinnert.
Einsetzen der Basisfunktionen des Ansatzraumes liefert
\begin{align}
    \skp{\theta_{k} \otimes \eta_{j}}{\theta_{m} \otimes \eta_{l}}{\mathcal X}
    &= \int_{I} \skp{\theta_{k} \eta_{j}}{\theta_{m} \eta_{l}}{V} \diff t
        + \int_{I} \skp{\theta_{k}' \eta_{j}}{\theta_{m}' \eta_{l}}{V'} \diff t
    \\&= \skp{\theta_{k}}{\theta_{m}}{L_{2}(I)} \skp{\eta_{j}}{\eta_{l}}{V} + \skp{\theta_{k}'}{\theta_{m}'}{L_{2}(I)} \skp{\eta_{j}}{\eta_{l}}{V'}
\end{align}

Bevor wir diese Darstellung zu einer handlichen Matrix-Schreibweise umformulieren können, müssen wir zunächst klären, wie das diskrete $V'$-Skalarprodukt bestimmt werden kann.
Dazu verwenden wir \cref{lemma:berechnung_der_rieszschen_darstellung} und erhalten für jedes $n = 1, \dots, \mathcal J$ für das Funktional $\skp{\eta_{n}}{\blank}{V' \times V}$ den Koeffizientenvektor $\vec{h}_{n} = \mat{V}_{x}^{-1} \bm \eta_{n}$ mit $\bm \eta_{n} := [\skp{\eta_{n}}{\eta_{j}}{V' \times V}]_{j = 1, \dots, \mathcal J}$.
Nach dem Rieszschen Darstellungssatz gilt nun unter Verwendung der durch $\vec{h}_{n} \in \mathbb{R}^{\mathcal J}$, $n = 1, \dots, \mathcal J$, definierten $h_n \in V$ die Gleichung
\begin{equation}
    \skp{\eta_{j}}{\eta_{l}}{V'} = \skp{h_{j}}{h_{l}}{V} = (\mat V_{x}^{-1} \bm \eta_{j})\tran \mat V_{x} (\mat V_{x}^{-1} \bm \eta_{l}) = \bm{\eta}_{j}\tran \mat{V}_{x}^{-1} \bm{\eta}_{l}.
\end{equation}
Die Gramsche Matrix $\mat V_{\mathrm{dual}} \in \mathbb{R}^{\mathcal J \times \mathcal J}$ des $V'$-Skalarprodukts ist dementsprechend
\begin{equation}
    \mat V_{\mathrm{dual}} := [\bm{\eta}_{j}\tran \mat{V}_{x}^{-1} \bm{\eta}_{l}]_{j,l = 1, \dots, \mathcal J} = \mat H_{x}\tran \mat V_{x}^{-1} \mat H_{x}.
\end{equation}

Weiter können wir damit die Gramsche Matrix $\mat X \in \mathbb{R}^{\mathcal N \times \mathcal N}$ des $\mathcal X$-Skalarprodukts auf $\mathcal X_{\mathcal N}$ durch
\begin{equation}
    \mat X := \mat M^{E}_{t} \otimes \mat V_{x} + \mat A^{E}_{t} \otimes (\mat H_{x}\tran \mat V_{x}^{-1} \mat H_{x})
\end{equation}
definieren und können nun für alle $u \in \mathcal X_{\mathcal N}$ die $\mathcal X$-Norm über $\norm{u}_{\mathcal X}^{2} = \vec{u}\tran \mat X \vec{u}$ bestimmen.

Analog leiten wir auch eine diskrete Darstellung der Testraumnorm her.
Auch hier sei an das norminduzierende $\mathcal Y$-Skalarprodukt
\begin{equation}
    \tag*{\cref{eq:testraum_skalarprodukt}}
    \skp{u}{v}{\mathcal Y} = \skp{u_{1}}{v_{1}}{L_2(I; V)} + \skp{u_{2}}{v_{2}}{H} \quad \text{für } u, v \in \mathcal Y
\end{equation}
erinnert.
Die Auswertung dieses Skalarprodukts für die Funktionen aus $\mathcal Y_{\mathcal N}$ liefert analog zur vorherigen Betrachtung
\begin{align}
    \skp{(\xi_{k} \otimes \eta_{j}, \eta_{l})}{(\xi_{n} \otimes \eta_{m}, \eta_{o})}{\mathcal Y}
    = \skp{\xi_{k}}{\xi_{n}}{L_{2}(I)} \skp{\eta_{j}}{\eta_{m}}{V} + \skp{\eta_{l}}{\eta_{o}}{H}.
\end{align}
Zusammengefasst führt dies zu der Gramschen Matrix $\mat Y \in \mathbb{R}^{\mathcal N \times \mathcal N}$, gegeben durch
\begin{equation}
    \mat{Y} := \begin{bmatrix}
    \mat{M}_{t}^{F} \otimes \mat{V}_{x} & \mat 0\\
    \mat 0 & \mat H_{x}
    \end{bmatrix}.
\end{equation}


\paragraph{Berechnung der inf-sup-Konstanten.} % (fold)
\label{par:berechnung_der_inf_sup_konstante}

Um die inf-sup-Konstante $\beta_{\mathcal N}(\bm \sigma)$ zu berechnen, greifen wir auf den Supremizing-Operator aus \cref{definition:supremizing_operator} zurück und nutzen die in \cref{lemma:supremizing_operator} hergeleitete alternative Darstellung von $\beta_{\mathcal N}(\bm \sigma)$.

Seien $\bm \sigma \in \mathcal P$ und $T_{\bm \sigma}$ der Supremizing-Operator von $\restr{b(\blank, \blank; \bm \sigma)}{\mathcal X_{\mathcal N} \times \mathcal Y_{\mathcal N}}$.
Die Matrixdarstellung von $T_{\bm \sigma}$ ergibt sich nach dem Rieszschen Darstellungssatz durch
\begin{equation}
    \mat{T}_{\bm \sigma} = \mat{Y}^{-1} \mat{B}(\bm \sigma) \in \mathbb{R}^{\mathcal N \times \mathcal N}.
\end{equation}
Dies führt weiter zu
\begin{equation}
    \beta_{\mathcal N}^{2}(\bm \sigma)
    = \inf_{u \in \mathcal X_{\mathcal N}} \frac{\skp{T_{\bm \sigma} u}{T_{\bm \sigma} u}{\mathcal Y}}{\norm{u}_{\mathcal X}^{2}}
    = \inf_{u \in \mathbb{R}^{\mathcal N}} \frac{\vec{u}\tran \mat{T}_{\bm \sigma}\tran \mat{Y} \mat{T}_{\bm \sigma} \vec{u}}{\vec{u}\tran \mat{X} \vec{u}}
    = \inf_{u \in \mathbb{R}^{\mathcal N}} \frac{\vec{u}\tran \mat{B}(\bm \sigma)\tran \mat{Y}^{-1} \mat{B}(\bm \sigma) \vec{u}}{\vec{u}\tran \mat{X} \vec{u}}.
\end{equation}
Es ist bekannt (vergleiche \cite[Subsection 1.3.5]{Patera:2007un}) dass der Ausdruck auf der rechten Seite mit Hilfe des verallgemeinerten Eigenwertproblems
\begin{equation}
    \mat{B}(\bm \sigma)\tran \mat{Y}^{-1} \mat{B}(\bm \sigma) \vec{v} = \lambda \mat{X} \vec{v}
\end{equation}
berechnet werden kann, da das gesuchte Infimum gerade dem minimalen Eigenwert $\lambda_{\min}$ entspricht.
Insbesondere gilt damit $\beta_{\mathcal N}(\bm \sigma) = \sqrt{\lambda_{\min}}$.

% paragraph berechnung_der_inf_sup_konstante (end)

\section{Beispiele} % (fold)
\label{sec:cha4_galerkin:beispiele}

Wir widmen uns nun einigen Beispielen, wobei wir uns im Rahmen dieser Arbeit aus laufzeittechnischen Gründen auf den Fall einer Raumdimension beschränken wollen.

Zunächst erinnern wir an dieser Stelle an einige Modellgrößen.
Das Zeitintervall sei durch $I := [0, T]$ für ein $0 < T < \infty$ gegeben.
Ferner sei durch $T_{f} \in \mathbb{R}$ mit $0 < T_{f} < T$ der Zeitpunkt des Feldwechsels gegeben.
Für den räumlichen Anteil sei ohne Einschränkung $\Omega := [0, L] \subset \mathbb{R}$ mit einem $L > 0$ gegeben.

Uns interessiert an dieser Stelle hauptsächlich die Stabilität der in diesem Kapitel beschriebenen Diskretisierung, weswegen wir den parametrischen Aspekt der betrachteten Raum"=Zeit"=Variationsformulierung erst im nächsten Kapitel weiterverfolgen werden.

\paragraph{Homogene Randbedingungen.} % (fold)
\label{par:homogene_randbedingungen_}

In \cref{section:raum_zeit_diskretisierung} wurde bisher nur die zeitliche Diskretisierung festgelegt.
Es bleibt an dieser Stelle also zu klären, wie die räumliche Diskretisierung $V_{\mathcal J}$ gewählt wird.
Wie in dieser Arbeit bereits mehrfach angeführt, weisen die Felder und die Lösungen der Propagator"=Differentialgleichung aus \cref{chapter:einleitung} oftmals Symmetrien und hohe Regularität auf, weswegen wir hierfür keinen Finite"=Elemente\mbox{-,} sondern einen Fourier"=Ansatz mit global definierten Funktionen verfolgen.
Um die homogenen Randbedingungen direkt in die Basisfunktionen einzubauen, wählen wir
\begin{equation}
    V_{\mathcal J} := \spn \Set{ \sin(\pi j \blank / L) \given j = 1, \dots, \mathcal J}.
\end{equation}

\begin{figure}[tb]
    \centering
    \begin{subfigure}[b]{0.495\textwidth}
        \centering
        \includegraphics[width=1\textwidth]{figures/chapter4/stability_sine_dataset1_fig_1.pdf}
        \caption{Exakte inf-sup-Konstante $\beta_{\mathcal N}$.}
    \end{subfigure}
    \begin{subfigure}[b]{0.495\textwidth}
        \centering
        \includegraphics[width=1\textwidth]{figures/chapter4/stability_sine_dataset1_fig_2.pdf}
        \caption{Untere Schranke nach \cref{satz:untere_schranke_fuer_infsup_nach_andreev}.}
    \end{subfigure}
    \caption[Stabilität der Diskretisierung mit homogenen Randbedingungen, erstes Beispiel.]{%
        Zur Stabilität der Diskretisierung des ersten Beispiels.
        Die Farben repräsentieren dabei die verschiedenen Werte für die Dimension $\mathcal J$ der räumlichen Diskretisierung $V_{\mathcal J}$, während auf der horizontalen Achse die Anzahl $\mathcal K$ der Zeitgitterpunkte aufgetragen ist.
        }
    \label{figure:infsup_homogen_ein_feld}
\end{figure}

Als erstes einfaches Modell betrachten wir die Propagator"=Differentialgleichung ohne zeitlichen Wechsel der Felder.
Konkret wählen wir $I = [0, 1]$ sowie $\Omega = [0, 1]$ und ferner den Differentialoperator $A(t) \equiv A \colon V \to V'$ als
\begin{equation}
    A \eta = - \Delta \eta + \eta + \sin(\pi \blank) \eta.
\end{equation}
Dies entspricht einer Verschiebung um $\mu = 1$ und einem zeitlich konstanten Feld $\omega \colon \Omega \to \mathbb{R}$, $\omega = \sin(\pi \blank)$ mit $\norm{\omega}_{L_{\infty}(I; L_{\infty}(\Omega))} = 1 = \mu$.
Damit wird in \cref{satz:bilinearform_messbar_stetig_garding} die G\aa{}rding-Ungleichung mit $\lambda = 0$ erfüllt und wir können die in \cref{satz:untere_schranke_fuer_infsup_nach_andreev} beschriebene untere Schranke für die inf-sup-Konstante $\beta_{\mathcal N}$ auswerten.

\cref{figure:infsup_homogen_ein_feld} zeigt sowohl die exakt bestimmte diskrete inf-sup-Konstante $\beta_{\mathcal N}$ als auch die untere Schranke für verschiedene Werte der Dimension $\mathcal J$ der räumlichen Diskretisierung sowie der Anzahl $\mathcal K$ an Zeitgitterpunkten.
Zwar kann an dieser Stelle die Konstante $c_{0}$ aus \cref{satz:untere_schranke_fuer_infsup_nach_andreev} nicht näher bestimmt werden, aber es ist auf den ersten Blick ersichtlich, dass die Schranke das tatsächliche Verhalten der inf-sup-Konstante qualitativ gut widerspiegelt.
Ferner wird die CFL"=Bedingung bekräftigt, da sich die inf"=sup"=Konstante für festes $\mathcal J$ und wachsendes $\mathcal K$ von unten dem Wert $1$ annähert.

Das zweite, etwas komplexere Beispiel enthält nun einen zeitlichen Wechsel bei $T_{f} = 0.5$.
Der Differentialoperator $A(t) \colon V \to V'$ sei gegeben durch
\begin{equation}
    \begin{aligned}
        A(t)\eta =
        &- \Delta \eta + 3 \eta + \chi_{[0, 0.5)}(t) \left[ \sin(\pi \blank) - \sin(6 \pi \blank) \right] \eta
        \\&+ \chi_{[0.5, 1]}(t) \left[ \sin(3 \pi \blank ) + \sin(6 \pi \blank) \right] \eta.
    \end{aligned}
\end{equation}
Erneut ist mit $\mu = 3$ sichergestellt, dass die G\aa{}rding"=Ungleichung mit $\lambda = 0$ erfüllt wird.
\cref{figure:infsup_homogen_zwei_felder} zeigt die Ergebnisse für diese Modelldaten.
Auffallend ist, dass die resultierenden Werte nahezu identisch sind mit denen des einfachen Beispiels.
Dies ist dadurch bedingt, dass in beiden Beispielen die Verschiebung $\mu$ so gewählt wurde, dass der Differentialoperator $A(t)$ elliptisch wird.

\begin{figure}[p]
    \centering
    \centering
    \begin{subfigure}[b]{0.495\textwidth}
        \centering
        \includegraphics[width=1\textwidth]{figures/chapter4/stability_sine_dataset2_fig_1.pdf}
        \caption{Exakte inf-sup-Konstante $\beta_{\mathcal N}$.}
    \end{subfigure}
    \begin{subfigure}[b]{0.495\textwidth}
        \centering
        \includegraphics[width=1\textwidth]{figures/chapter4/stability_sine_dataset2_fig_2.pdf}
        \caption{Untere Schranke nach \cref{satz:untere_schranke_fuer_infsup_nach_andreev}.}
    \end{subfigure}
    \caption[Stabilität der Diskretisierung mit homogenen Randbedingungen, zweites Beispiel.]{%
        Ergebnisse des zweiten Beispiels mit homogenen Randbedingungen.
        Erneut repräsentieren die Farben die verschiedenen Werte für $\mathcal J$ und die horizontale Achse die Dimension $\mathcal K$.
        }
    \label{figure:infsup_homogen_zwei_felder}
\end{figure}

\paragraph{Periodische Randbedingungen.} % (fold)
\label{par:periodische_randbedingungen_}

Auch auf die periodischen Randbedingungen wollen wir an dieser Stelle kurz eingehen und das zweite Beispiel nun in diesem Setting noch einmal betrachten.
Dabei übernehmen wir die dortigen Gegebenheiten und passen lediglich die verwendete Diskretisierung $V_{\mathcal J}$ an.
Diese wählen wir erneut in Form eines Fourier-Ansatzes, diesmal als
\begin{equation}
    V_{\mathcal J} := \spn \Set{ v_{j} \given j = 1, \dots, \mathcal J}
\end{equation}
mit den Basisfunktionen
\begin{equation}
    v_{j} := \begin{cases}
        \cos(\pi (j - 1) \blank / L), & \text{falls } j \text{ ungerade},\\
        \sin(\pi j \blank / L), & \text{falls } j \text{ gerade}.
    \end{cases}
\end{equation}

Nach den Aussagen in \cref{section:periodische_randbedingungen} können wir ohne Transformation im Sinne von \cref{lemma:transformation_zu_elliptischem_operator} die G\aa{}rding-Ungleichung im Allgemeinen nicht mit $\lambda = 0$ erfüllen.
Wir verzichten an dieser Stelle darauf, da diese im Wesentlichen nur für die Invertierbarkeit des Operators $A(t)$ benötigt wird und diese hier, wie bei der numerischen Umsetzung deutlich wurde, bereits gegeben ist.

\begin{figure}[p]
    \centering
    \centering
    \begin{subfigure}[b]{0.495\textwidth}
        \centering
        \includegraphics[width=1\textwidth]{figures/chapter4/stability_fourier_dataset1_fig_1.pdf}
        \caption{Exakte inf-sup-Konstante $\beta_{\mathcal N}$.}
    \end{subfigure}
    \begin{subfigure}[b]{0.495\textwidth}
        \centering
        \includegraphics[width=1\textwidth]{figures/chapter4/stability_fourier_dataset1_fig_2.pdf}
        \caption{Untere Schranke nach \cref{satz:untere_schranke_fuer_infsup_nach_andreev}.}
    \end{subfigure}
    \caption[Stabilität der Diskretisierung mit periodischen Randbedingungen.]{%
        Ergebnisse des zweiten Beispiels im Falle periodischer Randbedingungen.
        }
    \label{figure:infsup_periodisch_zwei_felder}
\end{figure}

Die Ergebnisse sind in \cref{figure:infsup_periodisch_zwei_felder} zu sehen.
Diese weisen im Vergleich zum Fall homogener Randbedingungen nur kleine Differenzen auf und liefern sogar leicht bessere, das heißt größere Werte bei der exakt bestimmten diskreten inf-sup-Konstante.

\paragraph{}
Sowohl im Fall homogener als auch periodischer Randbedingungen bekräftigen die vorliegenden Ergebnisse die Wahl der Diskretisierung.
Zwar handelt es sich dabei um relativ simple Modelle, allerdings legen diese bereits nahe, dass mit der Anpassung der Verschiebung $\mu$ der Felder ein Instrument zur Verfügung steht, mit dem auch in komplizierteren Fällen gute Stabilitätsergebnisse erzielt werden können.

\end{document}

    % % -*- root: ../main.tex -*-

\documentclass[../main.tex]{subfiles}
\begin{document}

\chapter{Reduzierte-Basis-Methode} % (fold)
\label{chapter:rbm}

Nun wird aufbauend auf das Petrov"=Galerkin"=Verfahren des vorherigen Kapitels die Reduzierte"=Basis"=Methode eingeführt und auf die vorliegenden Gegebenheiten angepasst.
Diese Methode wird anschließend numerisch umgesetzt, exemplarisch auf unsere Problemstellung angewandt und die resultierenden Ergebnisse werden diskutiert.

\section{Grundlagen} % (fold)
\label{sub:grb:rb:grundlagen}

Wir beginnen mit einer kurz gehaltenen Motivation und einer Einführung in die Reduzierte"=Basis"=Methoden.
Dabei handelt es sich um ein relativ junges Verfahren, welches besondere in den letzten zehn Jahren viel Aufmerksamkeit und Weiterentwicklung erfahren hat.
Wir orientieren uns hierbei an den Arbeiten von \textcite{Rozza2008,Patera:2007un}, welche auch eine deutlich umfassendere Einführung bieten.

Bevor wir die Idee hinter der Reduzierte"=Basis"=Methode angehen, wollen wir erneut die Rahmenbedingungen in Form eines abstrakten Variationsproblems festlegen.
Seien dazu zwei Hilberträume $\mathcal X$ und $\mathcal Y$ und eine abgeschlossene, konvexe Parametermenge $\mathcal P \subset \mathbb{R}^{p}$, für ein $p \in \mathbb{N}$, gegeben.
Weiter seien $b \colon \mathcal X \times \mathcal Y \times \mathcal P \to \mathbb{R}$ eine parametrische stetige Bilinearform und $f \colon \mathcal Y \to \mathbb{R}$ ein stetiges lineares Funktional.
Wir betrachten das abstrakte parametrische Variationsproblem:
\begin{equation}
\label{eq:abstraktes_parametrische_vp}
    \text{Sei } \bm \sigma \in \mathcal P, \text{ finde } u(\bm \sigma) \in \mathcal X \colon \quad b(u(\bm \sigma), v; \bm \sigma) = f(v) \quad \fa v \in \mathcal Y.
\end{equation}

Reduzierte"=Basis"=Methoden bieten sich vor allem dann an, wenn das obige Variationsproblem in Echtzeit für einen gegebenen Parameter oder immer wieder für viele verschiedene Parameter gelöst werden muss.
Um dies auf eine effiziente Weise bewerkstelligen zu können, konzentriert man sich auf die Lösungsmenge
\begin{equation}
\label{eq:stetige_loesungsmenge}
    \mathcal M := \Set{u(\bm \sigma) \in \mathcal X \given \bm \sigma \in \mathcal P}.
\end{equation}
Diese bildet oftmals, je nachdem welche Regularität die Abbildung $\mathcal P \ni \bm \sigma \mapsto u(\bm \sigma) \in \mathcal X$ aufweist, eine Mannigfaltigkeit mit gewissen Regularitätseigenschaften und vergleichsweise niedriger Dimension.

Um diese Eigenschaften ausnutzen zu können, muss diese Lösungsmenge zunächst durch eine diskretes Analogon ersetzt werden.
Hierzu kann beispielsweise das im vorherigen Kapitel beschriebene Galerkin"=Verfahren verwendet werden.
Auf diesen Fall beschränken wir uns im Folgenden und führen daher das diskrete abstrakte parametrische, durch eine Petrov"=Galerkin"=Diskretisierung erhaltene, Variationsproblem ein.
Seien dazu $\mathcal X_{\mathcal N} \subset \mathcal N$ und $\mathcal Y_{\mathcal N} \subset \mathcal Y$ Unterräume der Dimension $\mathcal N \in \mathbb{N}$.
Weiter nehmen wir an, dass diese Diskretisierung für alle $\bm \sigma \in \mathcal P$ zu einem korrekt gestellten Variationsproblem führt, welches gegeben sei durch:
\begin{equation}
\label{eq:diskretes_abstraktes_parametrisches_vp}
    \text{Sei } \bm \sigma \in \mathcal P, \text{ finde } u_{\mathcal N}(\bm \sigma) \in \mathcal X_{\mathcal N} \colon \quad b(u_{\mathcal N}(\bm \sigma), v; \bm \sigma) = f(v) \quad \fa v \in \mathcal Y_{\mathcal N}.
\end{equation}
Diese Petrov"=Galerkin"=Diskretisierung wird als Grundlage für die Reduzierte"=Basis"=Methode dienen und wird im Folgenden mit dem Präfix \emph{Truth} kenntlich gemacht, während wir \emph{RB} als Abkürzung für Reduzierte"=Basis verwenden.
Obige Diskretisierung erlaubt nun die Definition der Truth-Lösungsmenge durch
\begin{equation}
\label{eq:truth_loesungsmenge}
    \mathcal M_{\mathcal N} := \Set{u_{\mathcal N}(\bm \sigma) \in \mathcal X_{\mathcal N} \given \bm \sigma \in \mathcal P}.
\end{equation}
Aufbauend auf diese kann nun eine endlichdimensionale Teilmenge $\mathcal X_{N} \subset \mathcal M_{\mathcal N} \subset \mathcal X_{\mathcal N}$ konstruiert werden, welche wiederum als Ansatzraum eines weiteren Galerkin"=Verfahrens verwendet werden kann,
wobei dieses zu deutlich niedrigdimensionaleren Systemen führt.
Dies wird in \cref{figure:rbm_loesungsmenge} noch einmal veranschaulicht.

\begin{figure}[tb]
    \centering
    \includegraphics[width=0.6\textwidth]{figures/rb.pdf}
    \caption[%
    Skizze zur Motivation der Reduzierte-Basis-Methode.
    ]{
        Beispiel der Funktionsweise der Reduzierte"=Basis"=Methode im Falle eines eindimensionalen Parameters $\sigma$.
        Die Reduzierte"=Basis"=Lösungen $u_{N}(\sigma)$ ergeben sich als Linearkombinationen der Truth"=Lösungen $u_{\mathcal N}(\sigma_{i})$ für gewisse $\sigma_{i} \in \mathcal P$.
        }
    \label{figure:rbm_loesungsmenge}
\end{figure}

RB"=Methoden der beschriebenen Art lassen sich im Allgemeinen als zweistufiges Verfahren auffassen.
Die erste Stufe, die sogenannte \emph{Offline-Phase}, dient dazu, die niedrigdimensionalen Systeme zu konstruieren.
Hierzu gibt es verschiedene Ansätze, die meist die Minimierung des maximal möglichen Fehlers zwischen Truth- und RB"=Lösung verfolgen und dementsprechend einen hohen Aufwand darstellen.
In der zweiten Stufe, der \emph{Online-Phase}, wird dieses niedrigdimensionale System verwendet um Lösungen zu gegebenen Parametern zu bestimmen.
Das Lösen mit diesem System hat, alleine aus dimensionstechnischen Gründen, einen deutlich geringeren Aufwand als das bestimmen der Truth-Lösung.

Dieser zweistufige Aufbau führt schließlich auch zu einem wichtigen Kriterium für respektive gegen die Verwendung der RB"=Methoden, die Anzahl der zu bestimmenden Lösungen in der Online-Phase, da die Laufzeit der aufwändigen Offline-Phase zunächst amortisiert werden muss.

Wir beenden an dieser Stelle die Motivation, widmen uns nun einer rigorosen Einführung der RB"=Methode und beginnen diese mit der Definition der grundlegenden Begriffe.

\begin{Definition}
\label{definition:rb_variationsproblem}
    Seien $N \in \mathbb{N}$ sowie $\mathcal S_{N} := \Set{ \bm \sigma_{1}, \dots, \bm \sigma_{N} } \subset \mathcal P$ eine Teilmenge der Parameter und $\Phi_{N} := \Set{ u_{\mathcal N}(\bm \sigma_{1}), \dots, u_{\mathcal N}(\bm \sigma_{N})} \subset \mathcal X_{\mathcal N}$ eine Basis, die durch die zugehörigen Truth-Lösungen gegeben sei.
    Seien weiter $\mathcal X_{N} := \spn \Phi_{N}$ und $\mathcal Y_{N} \subset \mathcal Y_{\mathcal N}$ ein $N$-dimensionaler Unterraum.
    Als \emph{Reduzierte"=Basis"=Variationsformulierung} von \cref{eq:diskretes_abstraktes_parametrisches_vp} bezeichnen wir das folgende Variationsproblem:
    \begin{equation}
    \label{eq:rb_abstraktes_parametriches_vp}
        \text{Sei } \bm \sigma \in \mathcal P, \text{ finde } u_{N}(\bm \sigma) \in \mathcal X_{N} \colon \quad b(u_{N}(\bm \sigma), v; \bm \sigma) = f(v) \quad \fa v \in \mathcal Y_{N}.
    \end{equation}
    Fernen nennen wir $u_{N}(\bm \sigma)$ \emph{Reduzierte"=Basis"=Lösung}.
\end{Definition}

Wie auch bei den Petrov"=Galerkin"=Verfahren muss sichergestellt werden, dass es sich bei dieser Diskretisierung tatsächlich um ein korrekt gestelltes Problem handelt.
Dies kann hauptsächlich durch die Wahl des Testraumes $\mathcal Y_{N}$ gesteuert werden.
An dieser Stelle nehmen wir an, dass dies stets bewerkstelligt werden kann.
Ein konkretes Verfahren, um diese zu wählen, geben wir an späterer Stelle an, wobei die Korrektheit im Rahmen dieser Arbeit nicht bewiesen werden kann und nur numerisch bekräftigt werden wird.

Weiter wollen wir begründen, warum wir eine Galerkin"=Diskretisierung als Grundlage verwenden und nicht das stetige Variationsproblem \cref{eq:abstraktes_parametrische_vp}.
Die zugrundeliegende Prämisse ist, neben der Berechenbarkeit, dass nach \cref{satz:lemma_von_cea}, unter gewissen Annahmen, die Petrov"=Galerkin"=Lösung $u_{\mathcal N}(\bm \sigma)$ die exakte Lösung $u(\bm \sigma)$ durch Verfeinerung der Diskretisierung beliebig gut approximieren kann.
Betrachten wir die Abschätzung
\begin{equation}
    \norm{u(\bm \sigma) - u_{N}(\bm \sigma)}_{\mathcal X} \leq \norm{u(\bm \sigma) - u_{\mathcal N}(\bm \sigma)}_{\mathcal X} + \norm{u_{\mathcal N}(\bm \sigma) - u_{N}(\bm \sigma)}_{\mathcal X},
\end{equation}
dann sagt dies gerade aus, dass der erste Summand beliebig klein gehalten werden kann, weswegen wir uns in diesem Kapitel lediglich für den zweiten Summanden, also den Fehler zwischen Truth- und RB"=Lösung, interessieren.

Um die notationelle Wiederholung möglichst gering zu halten, seien für den Rest dieses Abschnitts stets durch $u_{N}(\bm \sigma) \in \mathcal X_{N}$ die Reduzierte-Basis-Lösung und durch $u_{\mathcal N}(\bm \sigma) \in \mathcal X_{\mathcal N}$ die Truth-Lösung zu einem Parameter $\bm \sigma \in \mathcal P$ gegeben.

Da es sich bei der RB"=Diskretisierung um ein Galerkin"=Verfahren handelt, erhalten wir Analoga der Aussagen \cref{satz:galerkin_stabilitaet}, \cref{lemma:galerkin_orthogonalitaet} und \cref{satz:lemma_von_cea}, wobei die Truth-Räume $\mathcal X_{\mathcal N}, \mathcal Y_{\mathcal N}$ die Rolle der Räume $\mathcal X, \mathcal Y$ einnehmen.
An dieser Stelle wiederholen wir nur die Galerkin"=Orthogonalität ohne Beweis, da diese nachfolgend verwendet wird.

\begin{Lemma}
    \label{lemma:rb_galerkin_orthogonalitaet}
    Es gilt $b(u_{N}(\bm \sigma) - u_{\mathcal N}(\bm \sigma), v) = 0$ für alle $v \in \mathcal Y_{N}$, das heißt, der Fehler $u_{N}(\bm \sigma) - u_{\mathcal N}(\bm \sigma)$ ist orthogonal zu $\mathcal Y_{N}$.
\end{Lemma}

Den Fehler des vorherigen Lemmas finden wir auch in der nachfolgenden Definition, welche einen weiteren Baustein für die RB"=Methode liefert.

\begin{Definition}
\label{definition:rbm_fehler_und_residuum}
    Als \emph{Fehler} $e_{N}(\bm \sigma) \in \mathcal X_{\mathcal N}$ bezeichnen wir
    \begin{equation}
        e_{N}(\bm \sigma) := u_{\mathcal N}(\bm \sigma) - u_{N}(\bm \sigma).
    \end{equation}
    Weiter sei das \emph{Residuum} $r_{N}(\blank; \bm \sigma) \colon \mathcal Y_{\mathcal N} \to \mathbb{R}$ festgelegt als
    \begin{equation}
    \label{eq:variationsproblem_residuum}
        r_{N}(v; \bm \sigma) := b(e_{N}(\bm \sigma), v; \bm \sigma), \quad v \in \mathcal Y_{\mathcal N}.
    \end{equation}
\end{Definition}

\begin{Lemma}
\label{lemma:rbm_residuum_ist_funktional}
    Das Residuum $r_{N}(\blank; \bm \sigma)$ ist für alle $\bm \sigma \in \mathcal P$ ein stetiges lineares Funktional, kurz also $r_{N}(\blank; \bm \sigma) \in \mathcal Y_{\mathcal N}'$.

    \begin{Beweis}
        Sowohl Linearität als auch Stetigkeit sind direkt ersichtlich, denn das Residuum kann nach Definition geschrieben werden als
        \begin{equation}
            \begin{aligned}
            r_{N}(v; \bm \sigma)
            &= b(e_{N}(\bm \sigma), v; \bm \sigma)
            = b(u_{\mathcal N}(\bm \sigma), v; \bm \sigma) - b(u_{N}(\bm \sigma), v; \bm \sigma)
            \\&= f(v) - b(u_{N}(\bm \sigma), v; \bm \sigma).
            \end{aligned}
        \end{equation}
    \end{Beweis}
\end{Lemma}

Das Residuum lässt sich nun verwenden, um eine der wichtigsten Eigenschaften der RB"=Methoden herzuleiten, die Zertifizierung der Lösung durch eine a posteriori-Fehlerabschätzung.
Dazu benötigen wir eine untere Schranke $\beta_{\mathrm{LB}}(\bm \sigma)$ der inf-sup-Konstante $\beta_{\mathcal N}(\bm \sigma)$, da die exakte Konstante im Allgemeinen nicht bestimmt werden kann.

\begin{Lemma}
\label{lemma:rbm_fehler_schranke}
    Sei $\beta_{\mathrm{LB}}(\bm \sigma) > 0$ eine berechenbare untere Schranke für $\beta_{\mathcal N}(\bm \sigma)$.
    Dann gilt
    \begin{equation}
        \norm{u_{\mathcal N}(\bm \sigma) - u_{N}(\bm \sigma)}_{\mathcal X} \leq \Delta_{N}(\bm \sigma) := \frac{\norm{r_{N}(\blank; \bm \sigma)}_{\mathcal Y_{\mathcal N}'}}{\beta_{\mathrm{LB}}(\bm \sigma)},
    \end{equation}
    wobei wir $\Delta_{N}(\bm \sigma)$ als \emph{a posteriori-Fehlerschätzer} bezeichnen.

    \begin{Beweis}
        Wir können \cref{eq:variationsproblem_residuum} als Variationsproblem \cref{eq:diskretes_abstraktes_parametrisches_vp} auffassen, welches unter den getroffenen Annahmen die eindeutige Lösung $e_{N}(\bm \sigma)$ besitzt.
        Die Abschätzung folgt nun aus \cref{satz:galerkin_stabilitaet}.
    \end{Beweis}
\end{Lemma}

Dieser a posteriori-Fehlerschätzer bildet das Herzstück der RB"=Methode, da er sowohl für die Konstruktion der niedrigdimensionalen Räume $\mathcal X_{N}$ und $\mathcal Y_{N}$, als auch für die bereits angesprochene Zertifizierung verwendet wird, welche eine garantierte obere Schranke zwischen Truth- und RB"=Lösung liefert.

Wir sind bisher nicht darauf eingegangen, wie dieser Fehlerschätzer numerisch bestimmt werden kann.
Auf den ersten Blick scheint dies schwierig, da sowohl die Norm des Residuums als auch eine untere Schranke für die inf-sup-Konstante im Allgemeinen nur unter hohem Aufwand bestimmt werden können.
Diesem Punkt widmen wir uns im nächsten Abschnitt, vorher wollen wir aber noch ein Maß für die Güte des Fehlerschätzers einführen.

\begin{Lemma}
\label{lemma:effektivitaet}
    Die \emph{Effektivität} $\eta_{N}(\bm \sigma)$ des a posteriori-Fehlerschätzers ist beschränkt durch
    \begin{equation}
        1 \leq \eta_{N}(\bm \sigma) := \frac{\Delta_{N}(\bm \sigma)}{\norm{u_{\mathcal N}(\bm \sigma) - u_{N}(\bm \sigma)}_{\mathcal X}} \leq \frac{\gamma_{b}(\bm \sigma)}{\beta_{\mathrm{LB}}(\bm \sigma)},
    \end{equation}
    wobei $\gamma_{b}(\bm \sigma)$ die Stetigkeitskonstante der Bilinearform $b(\blank, \blank; \bm \sigma)$ sei.

    \begin{Beweis}
        Die Abschätzung nach unten ergibt sich direkt aus \cref{lemma:rbm_fehler_schranke}.

        Für die andere Abschätzung sei $\bm \sigma \in \mathcal P$ beliebig.
        Anwendung des Rieszschen Darstellungssatz auf das Residuum $r_{N}(\blank; \bm \sigma)$ liefert ein $v_{r} \in \mathcal Y_{\mathcal N}$ mit $r_{N}(v; \bm \sigma) = \skp{v_{r}}{v}{\mathcal Y_{\mathcal N}}$, für alle $v \in \mathcal Y_{\mathcal N}$, und $\norm{r_{N}(\blank; \bm \sigma)}_{\mathcal Y_{\mathcal N}'} = \norm{v_{r}}_{\mathcal Y_{\mathcal N}}$.
        Damit erhalten wir mit der Stetigkeit der Bilinearform $b$ die Abschätzung
        \begin{equation}
            \norm{v_{r}}_{\mathcal Y}^{2}
            = \skp{v_{r}}{v_{r}}{\mathcal Y}
            = r_{N}(v_{r}; \bm \sigma)
            = b(e_{N}(\bm \sigma), v_{r}; \bm \sigma)
            \leq \gamma_{b}(\bm \sigma) \norm{e_{N}(\bm \sigma)}_{\mathcal X} \norm{v_{r}}_{\mathcal Y},
        \end{equation}
        oder auch kurz
        \begin{equation}
            \norm{v_{r}}_{\mathcal Y} \leq \gamma_{b}(\bm \sigma) \norm{e_{N}(\bm \sigma)}_{\mathcal X}.
        \end{equation}
        Zusammen mit der Definition des a posteriori-Fehlerschätzers $\Delta_{N}(\bm \sigma)$ liefert dies nun
        \begin{equation}
            \eta_{N}(\bm \sigma)
            = \frac{\Delta_{N}(\bm \sigma)}{\norm{u_{\mathcal N}(\bm \sigma) - u_{N}(\bm \sigma)}_{\mathcal X}}
            \leq \frac{\gamma_{b}(\bm \sigma) \norm{e_{N}(\bm \sigma)}_{\mathcal X}}{\beta_{\mathrm{LB}}(\bm \sigma)\norm{u_{\mathcal N}(\bm \sigma) - u_{N}(\bm \sigma)}_{\mathcal X}}
            = \frac{\gamma_{b}(\bm \sigma)}{\beta_{\mathrm{LB}}(\bm \sigma)}.
        \end{equation}
    \end{Beweis}
\end{Lemma}


\section{Numerische Umsetzung} % (fold)
\label{sub:grb:rb:numerische_umsetzung}

Wir richten unser Augenmerk nun auf die numerische Umsetzung der RB"=Methode und führen an den entsprechenden Stellen weitere, hierfür benötigte Grundlagen ein.
Im Rahmen dieser Arbeit beschränken wir uns auf die Betrachtung parametrische-affiner Variationsprobleme, so wie sie in \cref{lemma:bilinearform_affin_parametrisch} bereits erwähnt wurden, führen sie an dieser Stelle noch einmal in allgemeinerer Form ein.

\begin{Definition}
\label{definition:parametrisch_affine_bf_fuer_rbm}
    Wir nennen eine parametrische Bilinearform $b \colon \mathcal X \times \mathcal Y \times \mathcal P \to \mathbb{R}$ \emph{parametrisch affin}, wenn sie die Form
    \begin{equation}
        b(u, v; \bm \sigma) = \sum_{q = 1}^{Q_{b}} \theta_{q}^{b}(\bm \sigma) b_{q}(u, v), \quad \text{für } u \in \mathcal X, v \in \mathcal Y, \bm \sigma \in \mathcal P,
    \end{equation}
    hat, wobei für $q = 1, \dots, Q_{b}$ durch $\theta_{q}^{b} \colon \mathcal P \to \mathbb{R}$ Abbildungen und durch $b_{q} \colon \mathcal X \times \mathcal Y \to \mathbb{R}$ parameterunabhängige Bilinearformen gegeben seien.
\end{Definition}

Weiter abstrahieren wir an dieser Stelle die Notationen aus \cref{section:galerkin_numerische_umsetzung}.
Die diskreten Räume $\mathcal X_{\mathcal N}$ und $\mathcal Y_{\mathcal N}$ fassen wir als Span entsprechender Basisfunktionen auf und schreiben
\begin{equation}
    \mathcal X_{\mathcal N} = \spn \Set{ \phi_{i} \given i = 1, \dots, \mathcal N },
    \quad
    \mathcal Y_{\mathcal N} = \spn \Set{ \psi_{j} \given j = 1, \dots, \mathcal N }.
\end{equation}
Ebenso schreiben wir die Reduzierte"=Basis"=Räume $\mathcal X_{N}$ und $\mathcal Y_{N}$ als Span der Form
\begin{equation}
    \mathcal X_{N} = \spn \Set{ \zeta_{i} \given i = 1, \dots, N},
    \quad
    \mathcal Y_{N} = \spn \Set{ \eta_{j} \given j = 1, \dots, N},
\end{equation}
wobei $\zeta_{i} \in \mathcal X_{\mathcal N}$ und $\eta_{j} \in \mathcal Y_{\mathcal N}$ für $i, j = 1, \dots, N$ Basisfunktionen seien.
Die zur Bestimmung der Petrov"=Galerkin"=Lösung verwendeten Strukturen schreiben wir ferner als
\begin{equation}
    \begin{aligned}
        \mat{X} &= \left[ \skp{\phi_{i}}{\phi_{j}}{\mathcal X} \right]_{i,j = 1, \dots, \mathcal N},
        &\qquad
        \mat{Y} &= \left[ \skp{\psi_{i}}{\psi_{j}}{\mathcal Y} \right]_{i,j = 1, \dots, \mathcal N},
        \\
        \vec{f} &= \left[ f(\psi_{j}) \right]_{j = 1, \dots, \mathcal N},
        &\qquad
        \mat{B}_{q} &= \left[ b_{q}(\phi_{i}, \psi_{j}) \right]_{j,i = 1, \dots, \mathcal N}, \quad q = 1, \dots, Q_{b}.
    \end{aligned}
\end{equation}

Zunächst wollen wir eine einfaches Verfahren angeben, mit dem wir aus den zugrundeliegenden Petrov"=Galerkin"=Strukturen die für die RB"=Methode benötigten berechnen können.

\begin{Definition}
    Als \emph{Reduzierte"=Basis"=Ansatzfunktion} $u_{N} \in \mathcal X_{N}$ respektive \emph{Testfunktion} $v_{N} \in \mathcal Y_{N}$ bezeichnen wir
    \begin{equation}
        u_{N} := \sum_{i = 1}^{N} u_{N i} \zeta_{i},
        \quad
        v_{N} := \sum_{j = 1}^{N} v_{N j} \eta_{j},
    \end{equation}
    mit den Koeffizientenvektoren $\vec{u}_{N} := [u_{Ni}]_{i = 1, \dots, N} \in \mathbb{R}^{N}$ und $\vec{v}_{N} := [v_{Nj}]_{j = 1, \dots, N} \in \mathbb{R}^{N}$.
\end{Definition}

Schreiben wir für $i,j = 1, \dots, N$ die Koeffizientenvektoren der Basisfunktionen $\zeta_{i} \in \mathcal X_{N}$ und $\eta_{j} \in \mathcal Y_{N}$ bezüglich der Basen von $\mathcal X_{\mathcal N}$ und $\mathcal Y_{\mathcal N}$  als $\bm \zeta_{i}, \bm \eta_{j} \in \mathbb{R}^{\mathcal N}$, dann können wir ferner die Matrizen
\begin{equation}
  \mat{Z} := [\bm \zeta_{i}]_{i = 1, \dots, N} \in \mathbb{R}^{\mathcal N \times N}, \quad \mat{H} := [\bm \eta_{i}]_{i = 1, \dots, N} \in \mathbb{R}^{\mathcal N \times N}
\end{equation}
definieren.
Mit Hilfe dieser beiden Matrizen definieren wir nun die Reduzierte"=Basis"=Matrizen $\mat{B}_{N, q} \in \mathbb{R}^{N \times N}$, $q = 1, \dots, Q_{b}$, und den Reduzierte"=Basis"=Lastvektor $\vec{f}_{N} \in \mathbb{R}^{N}$ definieren via
\begin{equation}
    \begin{aligned}
        \mat{B}_{N,q} &:= \left[ b_{q}(\zeta_{i}, \eta_{j}) \right]_{j, i = 1, \dots, N} = \mat{H}\tran \mat{B}_{q} \mat{Z}, \quad q = 1, \dots, Q_{b}, \\
        \vec{f}_{N} &:= \left[ f(\eta_{j}) \right]_{j = 1, \dots, N} = \mat{H}\tran \vec{f}
    \end{aligned}
\end{equation}
und analog auch die entsprechenden Norminduzierenden Matrizen $\mat{X}_{N}, \mat{Y}_{N} \in \mathbb{R}^{N \times N}$ durch
\begin{equation}
    \begin{aligned}
    \mat{X}_{N} &:= \left[ \skp{\zeta_{i}}{\zeta_{j}}{\mathcal X} \right]_{i,j = 1, \dots, N} = \mat{Z}\tran \mat{X} \mat{Z}, \\
    \mat{Y}_{N} &:= \left[ \skp{\eta_{i}}{\eta_{j}}{\mathcal Y} \right]_{i,j = 1, \dots, N} = \mat{H}\tran \mat{Y} \mat{H}.
    \end{aligned}
\end{equation}

Um nun die RB"=Lösung $u_{N}(\bm \sigma)$ zu einem Parameter $\bm \sigma \in \mathcal P$ zu bestimmen, reicht es, das lineare Gleichungssystem
\begin{equation}
\label{eq:rbm_gleichungssystem}
    \mat{B}_{N}(\bm \sigma) \vec{u}_{N}(\bm \sigma) = \vec{f}_{N}
\end{equation}
zu lösen, wobei
\begin{equation}
\label{eq:affine_rbm_systemmatrix}
    \mat{B}_{N}(\bm \sigma) = \sum_{q = 1}^{Q_b} \theta_{q}^{b}(\bm \sigma) \mat{B}_{N, q}
\end{equation}
die Systemmatrix und $\vec{u}_{N}(\bm \sigma)$ der Koeffizientenvektor der Lösung ist.

\subsection{Zerlegung in Offline- und Online-Phase} % (fold)
\label{sub:zerlegung_in_offline_und_online_phase}

Wir gehen nun detailliert auf die beiden Phasen der Reduzierte"=Basis"=Methode ein und beginnen mit der einfacheren Online-Phase, bevor wir zur aufwendigeren Offline-Phase übergehen.

\subsubsection{Online-Phase} % (fold)
\label{ssub:online_phase}

In der Online-Phase muss zu einem gegebenen Parameter $\bm \sigma \in \mathcal P$ lediglich die Systemmatrix $\mat{B}_{N}(\bm \sigma)$ anhand der parametrisch-affinen Darstellung \cref{eq:affine_rbm_systemmatrix} berechnet werden, um anschließend die RB"=Lösung $u_{N}(\bm \sigma)$ über das Gleichungssystem \cref{eq:rbm_gleichungssystem} bestimmen zu können.

Ferner bestimmen wir den a posteriori-Fehlerschätzer $\Delta_{N}(\bm \sigma)$ zur Zertifizierung des Fehlers.

% subsubsection online_phase (end)

\subsubsection{Offline-Phase} % (fold)
\label{ssub:offline_phase}

Ziel der Offline-Phase ist die Bestimmung der für die Online-Phase benötigten Strukturen unter dem Gesichtspunkt den maximalen Fehler möglichst zu minimieren.
Dies wird oft auf Basis heuristisch motivierter Strategien getan, da die theoretisch optimale Auswahl der Snapshot-Parameter numerisch nicht umsetzbar ist.
Weiterführend findet man hierzu bei \cite{??} eine umfassende Einführung in die sogenannten \emph{Kolmogorow-N-Weiten}.

Als praktisch umsetzbare Verfahren erfreuen vor allem \emph{Greedy}-Strategien und auf die \emph{Proper Orthogonal Decomposition} aufbauende Algorithmen großer Beliebtheit.
Da wir uns auf einen Greedy-Ansatz beschränken werden, sei zu letzterem an dieser Stelle \cite{} als weitereführende Literatur erwähnt.

Der hier zum Einsatz kommende Ansatz ist unter dem Namen \emph{Greedy Sampling Scheme} bekannt, und lässt sich wie folgt zusammenfassen.
Iterativ wird der Parameter $\bm \sigma^{*} \in \mathcal P$ bestimmt, welcher in der aktuellen reduzierten Basis den Fehler $\norm{e_{N}(\bm \sigma)}_{\mathcal X}$ maximiert.
Durch Hinzunahme der zugehörigen Truth"=Lösung $u_{\mathcal N}(\bm \sigma^{*})$ zur reduzierten Basis erreichen wir somit in jedem Schritt eine maximale Fehlerreduktion.
Da die Auswertung des exakten Fehlers $\norm{e_{N}(\bm \sigma)}_{\mathcal X}$ einen zu hohen Aufwand mit sich liefert, kommt hier der a posteriori-Fehlerschätzer $\Delta_{N}(\bm \sigma)$ aus \cref{lemma:rbm_fehler_schranke} zum Einsatz.
Das daraus resultierende Verfahren findet sich formal genauer in \cref{algorithm:greedy_training}.

Ein offenbar kritischer Punkt des Algorithmus ist die Auswertung des a posteriori-Fehlerschätzers.
Diese setzt sich zum einen aus der Berechnung der Norm des Residuums und aus der Bestimmung einer Schranke für die inf-sup-Konstante.
Wir gehen zunächst auf das einfacher zu handhabende Residuum ein.

\paragraph{Berechnung der Norm des Residuums.} % (fold)
\label{par:berechnung_der_norm_des_residuum}

Um diese zu bestimmen, greifen wir erneut auf den Rieszschen Darstellungssatz zurück und erhalten damit für jedes $\bm \sigma \in \mathcal P$ ein $\hat{e}_{N}(\bm \sigma) \in \mathcal Y_{\mathcal N}$ mit
\begin{equation}
    \norm{\hat{e}_{N}(\bm \sigma)}_{\mathcal Y} = \norm{r_{N}(\blank; \bm \sigma)}_{\mathcal Y_{\mathcal N}'}, \qquad
    r_{N}(v; \bm \sigma) = \skp{\hat{e}_{N}(\bm \sigma)}{v}{\mathcal Y_{\mathcal N}}, \quad \fa v \in \mathcal Y.
\end{equation}
Aufgrund der Definition des Residuums \cref{eq:variationsproblem_residuum} und der parametrisch-affinen Darstellung \cref{definition:parametrisch_affine_bf_fuer_rbm} der Bilinearform $b$ können wir dies verwursten zu
\begin{equation}
    \skp{\hat{e}_{N}(\bm \sigma)}{v}{\mathcal Y}
    = r_{N}(v; \bm \sigma)
    % = f(v) - \sum_{q = 1}^{Q_b} \theta_{q}^{b}(\bm \sigma) b_{q}(u_{N}(\bm \sigma), v)
    = f(v) - \sum_{q = 1}^{Q_b} \sum_{n = 1}^{N} \theta_{q}^{b}(\bm \sigma) u^{(n)}_{N}(\bm \sigma) b_{q}(\zeta_{n}, v).
\end{equation}
Fassen wir nun weiter die Abbildungen $b_{q}(\zeta_{n}, \blank) \colon \mathcal Y_{\mathcal N} \to \mathbb{R}$ als Funktionale auf, dann sind diese stetig und linear und wir können erneut den Rieszschen Darstellungssatz anwenden.
Dies liefert die Existenz von $\hat{f}, \hat{b}^{n}_{q} \in \mathcal Y_{\mathcal N}$, $n = 1, \dots, N,$, $q = 1, \dots, Q_b$, mit
\begin{equation}
    \begin{aligned}
        f(v) &= \skp{\hat{f}}{v}{\mathcal Y}, \quad \fa v \in \mathcal Y_{\mathcal N},
        \\
        b_{q}(\zeta_{n}, v) &= \skp{\hat{b}^{n}_{q}}{v}{\mathcal Y_{\mathcal N}}, \quad \fa v \in \mathcal \mathcal Y_{\mathcal N}.
    \end{aligned}
\end{equation}
Zusammen mit obiger Darstellung erhalten wir damit
\begin{equation}
    \hat{r}_{N}(\bm \sigma) = \hat{f} - \sum_{q = 1}^{Q_b} \sum_{n = 1}^{N} \theta_{q}^{b}(\bm \sigma) u_{N}^{(n)}(\bm \sigma) \hat{b}^{n}_{q}.
\end{equation}
Die Norm des Residuums kann damit über
\begin{equation}
    \begin{aligned}
        \norm{\hat{r}_{N}(\bm \sigma)}_{\mathcal Y}^{2}
        &= \skp{\hat{r}_{N}(\bm \sigma)}{\hat{r}_{N}(\bm \sigma)}{\mathcal Y}
        \\&= \begin{multlined}[t]
            \skp{\hat{f}}{\hat{f}}{\mathcal Y}
                - 2 \sum_{q = 1}^{Q_b} \sum_{n = 1}^{N} \theta_{q}^{b}(\bm \sigma) u_{N}^{(n)}(\bm \sigma) \skp{\hat{b}^{n}_{q}}{\hat{f}}{\mathcal Y}
                \\+ \sum_{q, p = 1}^{Q_b} \sum_{n, m = 1}^{N} \theta_{q}^{b}(\bm \sigma) \theta_{p}^{b}(\bm \sigma) u_{N}^{(n)}(\bm \sigma) u_{N}^{(m)}(\bm \sigma) \skp{\hat{b}^{n}_{q}}{\hat{b}^{m}_{p}}{\mathcal Y}.
        \end{multlined}
    \end{aligned}
\end{equation}

Diese Berechnung kann nun effizient in eine Offline- und eine Online-Phase zerlegt werden.

\begin{description}[font=\normalfont\itshape]
    \item[Offline:] Berechne und speichere $\skp{\hat{f}}{\hat{f}}{\mathcal Y}$, $\skp{\hat{b}^{n}_{q}}{\hat{f}}{\mathcal Y}$ und $\skp{\hat{b}^{n}_{q}}{\hat{b}^{m}_{p}}{\mathcal Y}$.
    \item[Online:] Werte obige Darstellung aus.
\end{description}

TODO: Alternative Berechnung über Matrixschreibweise?

% paragraph berechnung_der_norm_des_residuum (end)

\paragraph{Berechnung der unteren Schranke für die inf-sup-Konstante.} % (fold)
\label{par:berechnung_der_unteren_schranke_f_r_die_inf_sup_konstante_}

Da die exakte Bestimmung der inf-sup-Konstante $\beta_{\mathcal N}(\bm \sigma)$ für jeden Parameter $\bm \sigma \in \Xi_{\mathrm{train}}$ laufzeittechnisch nicht vertretbar ist, wird ein Verfahren benötigt, welches auf effiziente Weise eine möglichst genaue untere Schranke $\beta_{\mathrm{LB}}(\bm \sigma)$ liefert.
Hierfür verwenden wir die zu diesem Zweck entwickelte \ac{scm} und orientieren uns an der Originalarbeit von \textcite{Huynh2007}, wobei wir an dieser Stelle nur die Idee des Verfahrens wiederholen wollen.
Weiter wurden bei der Implementierung die Verbesserungen von \textcite{Chen2009} berücksichtigt, die wir aber nicht weiter ausführen werden.

Da die \ac{scm} auf die Berechnung der Koerzivitätskonstante einer symmetrischen parametrischen Bilinearform ausgelegt ist, müssen wir zunächst die nötigen Rahmenbedingungen schaffen.
Wir werden später darauf eingehen, wie wir mit der \ac{scm} die gewünschte inf-sup-Schranke bestimmen können.

Es sei eine abstrakte symmetrische, parametrisch-affine Bilinearform $a \colon \mathcal X \times \mathcal X \times \mathcal P \to \mathbb{R}$ durch
\begin{equation}
    a(u, v; \bm \sigma) = \sum_{q = 1}^{Q_a} \theta_{q}^{a} a_{q}(u, v), \quad u,v \in \mathcal X,
\end{equation}
gegeben, wobei $\theta_{q}^{a} \colon \mathcal P \to \mathbb{R}$ stetige Funktionen und $a_{q} \colon \mathcal X \times \mathcal X \to \mathbb{R}$ symmetrische, stetige und parameterunabhängige Bilinearformen seien.
Wie zuvor sei $\mathcal X_{\mathcal N} \subset \mathcal X$ eine Diskretisierung eines Galerkin"=Verfahrens.
Wir wollen nun eine Schranke $\alpha_{\mathrm{LB}}(\bm \sigma)$ für die \emph{Koerzivitätskonstante} $\alpha_{\mathcal N}(\bm \sigma)$ bestimmen, welche als
\begin{equation}
    \alpha_{\mathcal N}(\bm \sigma) := \inf_{u \in \mathcal X_{\mathcal N}} \frac{a(u, u; \bm \sigma)}{\norm{u}_{\mathcal X}^{2}}
\end{equation}
definiert ist.

Die Idee hinter der \ac{scm} ist nun, die Bestimmung der obigen Koerzivitätskonstante als lineares Programm aufzufassen.
Dazu sei zunächst die tatsächliche zulässige Menge $\mathcal Z$ definiert als
\begin{equation}
    \mathcal Z := \Set[\bigg]{ \vec{z} = (z_{1}, \dots, z_{Q_a}) \in \mathbb{R}^{Q_a} \given \exists u_{z} \in \mathcal X_{\mathcal N} \colon z_{q} = \frac{a_{q}(u_{z}, u_{z})}{\norm{u_{z}}_{\mathcal X}^{2}}, q = 1, \dots, Q_a}.
\end{equation}
Weiter definieren wir die Zielfunktion
\begin{equation}
    \mathcal J \colon \mathbb{R}^{Q_a} \times \mathcal P \to \mathbb{R}, \quad \mathcal J(\vec{z} ;\bm \sigma) := \sum_{q = 1}^{Q_a} \theta_{q}^{a}(\bm \sigma) z_{q}.
\end{equation}
Mit dieser Konstruktion gilt nun $\alpha_{\mathcal N}(\bm \sigma) = \min_{\vec{z} \in \mathcal Z} \mathcal J(\vec{z}; \bm \sigma)$, es handelt sich aber noch nicht um ein lineares Programm.

Hierfür beschränken wir zunächst die einzelnen Variablen $z_{q}$ von oben und unten durch die Wahl von
\begin{equation}
    \alpha_{q}^{-} := \inf_{u \in \mathcal X_{\mathcal{N}}} \frac{a_{q}(u, u)}{\norm{u}^{2}_{\mathcal X}},
    \quad
    \alpha_{q}^{+} := \sup_{u \in \mathcal X_{\mathcal{N}}} \frac{a_{q}(u, u)}{\norm{u}^{2}_{\mathcal X}},
    \qquad \fa q = 1, \dots, Q_a,
\end{equation}
und des Quaders $B_{Q_a} := \prod_{q = 1}^{Q_a} [\alpha^{-}_{q}, \alpha^{+}_{q}] \subset \mathbb{R}^{Q_a}$.

Weiter benötigen wir eine endliche Teilmenge $\Xi_{\mathrm{train}} \subset \mathcal P$ als Trainingsparameter und definieren hier bereits die Notation $\mathcal C_{K} := \Set{ \bm \sigma_{1}, \dots, \bm \sigma_{K} } \subset \mathcal P$ für eine weitere, noch zu konstruierende Parameterteilmenge.
Ferner sei durch $P_{M}(\bm \sigma; E)$ eine Abbildung gegeben, welche für ein $M \in \mathbb{N}_{0}$ und eine Teilmenge $E \subset \mathcal P$ die bezüglich der euklidischen Norm $M$ nächsten Punkte zu $\bm \sigma$ aus $E$ liefert.
Schließlich seien durch $M_{\alpha}, M_{+} \in \mathbb{N}$ die Anzahl der Stabilitätsbedingungen respektive Positivitätsbedingungen festgelegt.

Mit dieser Vorarbeit lassen sich nun eine Untermenge $\mathcal Z_{\mathrm{LB}}(\bm \sigma; \mathcal C_{K}) \subset \mathcal Z$ und eine Obermenge $\mathcal Z_{\mathrm{UB}}(\mathcal C_{K}) \supset \mathcal Z$ durch
\begin{equation}
    \begin{aligned}
        \mathcal Z_{\mathrm{LB}}(\bm \sigma; \mathcal C_{K}) &:=
        \begin{multlined}[t]
        \Set*{\vec{z} \in B_{Q_a} \given
        \textstyle\sum_{q = 1}^{Q_a} \theta_{q}^{a}(\hat{\bm \sigma})z_{q} \geq \alpha_{\mathcal N}(\bm \sigma)~\fa \hat{\bm \sigma} \in P_{M_{\alpha}}(\bm \sigma; \mathcal C_{K})
        }
        \\ \cap
        \Set*{\vec{z} \in B_{Q_a} \given
        \textstyle\sum_{q = 1}^{Q_a} \theta_{q}^{a}(\hat{\bm \sigma}) z_{q} \geq 0~\fa \hat{\bm \sigma} \in P_{M_{+}}(\bm \sigma; \Xi_{\mathrm{train}})
        },
        \end{multlined}\\
        \mathcal Z_{\mathrm{UB}}(\mathcal C_{K}) &:= \Set[\Big]{ \vec{z}^{*}(\bm \sigma_{k}) := \arg \min_{\vec z \in \mathcal Z} \mathcal J(\vec{z}; \bm \sigma_{k}) \given k = 1, \dots, K}.
    \end{aligned}
\end{equation}
Nach Konstruktion handelt es sich dabei um konvexe Polyeder, sodass wir nun die beiden linearen Programme
\begin{equation}
    \alpha_{\mathrm{LB}}(\bm \sigma; \mathcal C_{K}) := \min_{\vec{z} \in \mathcal Z_{\mathrm{LB}}(\bm \sigma; \mathcal C_{K})} \mathcal J(\vec{z}; \bm \sigma),
    \quad
    \alpha_{\mathrm{UB}}(\bm \sigma; \mathcal C_{K}) := \min_{\vec{z} \in \mathcal Z_{\mathrm{UB}}(\mathcal C_{K})} \mathcal J(\vec{z}; \bm \sigma)
\end{equation}
definieren können.
Es lässt sich nachweisen \cite[Proposition 1]{Huynh2007}, dass diese für beliebige $\mathcal C_{K}, M_{\alpha}$ und $M_{+}$ die Abschätzung
\begin{equation}
    \alpha_{\mathrm{LB}}(\bm \sigma; \mathcal C_{K}) \leq \alpha_{\mathcal N}(\bm \sigma) \leq \alpha_{\mathrm{UB}}(\bm \sigma; \mathcal C_{K}), \quad \fa \bm \sigma \in \mathcal P,
\end{equation}
erfüllen.

Es bleibt lediglich die Konstruktion der Teilmengen $\mathcal C_{K} \subset \mathcal P$ zu klären.
Dazu wird erneut ein Greedy-Verfahren eingesetzt, siehe \cref{algorithm:scm_greedy}.
Dieses wird in der Offline-Phase der \ac{scm} ausgeführt, während in der Online-Phase zur Bestimmung der gesuchten Schranke $\alpha_{\mathrm{LB}}(\bm \sigma)$ ein relativ kleines lineares Programm gelöst werden muss.

\begin{algorithm}[tb]
    \DontPrintSemicolon
    \SetKwInOut{Input}{Eingabe}\SetKwInOut{Output}{Ausgabe}
    \SetKwProg{Proc}{Prozedur}{}{}
    %
    \Input{Menge $\Xi_{\mathrm{train}} \subset \mathcal P$ der Trainingsparameter und Fehlertoleranz $\epsilon_{\mathrm{tol}} > 0$}
    \Output{Reduzierte-Basis-Ansatzraum $\mathcal X_{N}$}
    \BlankLine
    Setze $K = 1$, $\mathcal C_{1} = \Set{\bm \sigma_{1}}$ mit zufälligem $\bm \sigma_{1} \in \Xi_{\mathrm{train}}$\;
    \While{$\max_{\bm \sigma \in \Xi_{\mathrm{train}}} [(\alpha_{\mathrm{UB}}(\bm \sigma; \mathcal C_{K}) - \alpha_{\mathrm{LB}}(\bm \sigma; \mathcal C_{K})) / {\alpha_{\mathrm{UB}}(\bm \sigma; \mathcal C_{K})}] > \epsilon_{\mathrm{tol}}$}{
        $\bm \sigma_{K + 1} \leftarrow \arg \max_{\bm \sigma \in \Xi_{\mathrm{train}}} [(\alpha_{\mathrm{UB}}(\bm \sigma; \mathcal C_{K}) - \alpha_{\mathrm{LB}}(\bm \sigma; \mathcal C_{K})) / {\alpha_{\mathrm{UB}}(\bm \sigma; \mathcal C_{K})}]$\;
        $\mathcal C_{K + 1} \leftarrow \mathcal C_{K} \cup \Set{\bm \sigma_{K + 1}}$\;
        $K \leftarrow K + 1$\;
    }
    \caption{Successive Constraint Method}
    \label{algorithm:scm_greedy}
\end{algorithm}

Das selbe Verfahren lässt schlussendlich auch für die Bestimmung einer inf-sup-Konstante verwenden.
Dazu betrachten wir die bekannte Bilinearform $b$ aus \cref{definition:parametrisch_affine_bf_fuer_rbm} und definieren für jedes $q = 1, \dots, Q_{b}$ mittels Rieszschen Darstellungssatzes einen Operator $T_{q} \colon \mathcal X_{\mathcal N} \to \mathcal Y_{\mathcal N}$ mittels
\begin{equation}
    \skp{T_{q} u}{v}{\mathcal Y} = b_{q}(u, v), \quad u \in \mathcal X_{\mathcal N}, v \in \mathcal Y_{\mathcal N},
\end{equation}
sowie den Operator $T(\bm \sigma) \colon \mathcal X_{\mathcal N} \to \mathcal Y_{\mathcal N}$ mittels
\begin{equation}
    T(\bm \sigma) := \sum_{q = 1}^{Q_b} \theta_{q}^{b}(\bm \sigma) T_{q}.
\end{equation}
Analog zu \cref{par:berechnung_der_inf_sup_konstante} geht ab
Es lässt sich zeigen \cite[476]{Huynh2007}, dass für die inf-sup-Konstante $\beta_{\mathcal N}(\bm \sigma)$ nun die Beziehung
\begin{equation}
    (\beta_{\mathcal N}(\bm \sigma))^{2}
    = \left( \inf_{u \in \mathcal X_{\mathcal N}} \sup_{v \in \mathcal Y_{\mathcal N}} \frac{b(u, v; \bm \sigma)}{\norm{u}_{\mathcal X} \norm{v}_{\mathcal Y}} \right)^{2}
    = \inf_{u \in \mathcal X_{\mathcal N}} \frac{\skp{T(\bm \sigma)u}{T(\bm \sigma)u}{\mathcal Y}}{\norm{u}_{\mathcal X}^{2}}
\end{equation}
gilt, wobei $\skp{T(\bm \sigma)\blank}{T(\bm \sigma)\blank}{\mathcal Y} \colon \mathcal X_{\mathcal N} \times \mathcal X_{\mathcal N} \to \mathbb{R}$ eine symmetrische, parametrische, stetige Bilinearform definiert.
Auf diese kann nun die beschriebene \acl{scm} angewandt werden, um die inf-sup-Konstante abzuschätzen.


% paragraph berechnung_der_unteren_schranke_f_r_die_inf_sup_konstante_ (end)

% subsubsection offline_phase (end)

\clearpage

\begin{algorithm}[tb]
    \DontPrintSemicolon
    \SetKwInOut{Input}{Eingabe}\SetKwInOut{Output}{Ausgabe}
    \SetKwProg{Proc}{Prozedur}{}{}
    %
    \Input{Menge $\Xi_{\mathrm{train}} \subset \mathcal P$ der Trainingsparameter und Fehlertoleranz $\epsilon_{\mathrm{tol}} > 0$}
    \Output{Reduzierte-Basis-Ansatzraum $\mathcal X_{N}$ der Dimension $N$}
    \BlankLine
    Setze $N = 0$, $\mathcal P_{0} = \Set{}$, $\Phi_{0} = \Set{}$, $\mathcal X_{0} = \Set{0}$\;
    \While{$\max_{\bm \sigma \in \Xi_{\mathrm{train}}} \Delta_{N}(\bm \sigma) > \epsilon_{\mathrm{tol}}$}{
        $\bm \sigma_{N + 1} \leftarrow \arg \max_{\bm \sigma \in \Xi_{\mathrm{train}}} \Delta_{N}(\bm \sigma)$\;
        $\mathcal P_{N + 1} \leftarrow \mathcal P_{N} \cup \Set{\bm \sigma_{N + 1}}$\;
        $\Phi_{N + 1} \leftarrow \Phi_{N} \cup \Set{u_{\mathcal N}(\bm \sigma_{N + 1})}$\;
        $\mathcal X_{N + 1} \leftarrow \spn \Phi_{N + 1}$\;
        $N \leftarrow N + 1$\;
    }
    %
    \caption{Greedy Training}
    \label{algorithm:greedy_training}
\end{algorithm}


\paragraph{Konstruktion des Testraumes.} % (fold)
\label{par:konstruktion_des_testraumes_}

Wir haben bisher nicht geklärt, wie der Reduzierte"=Basis"=Testraum $\mathcal Y_{N} \subset \mathcal Y_{\mathcal N}$ bestimmt wird.
Hierzu greifen wir erneut auf den Supremizing-Operator aus \cref{definition:supremizing_operator} zurück und definieren damit einen parameterabhängigen Testraum $\mathcal Y_{N}^{\bm \sigma}$ via
\begin{equation}
    \label{eq:rb_testraum_parameterabhaengig}
    \mathcal Y_{N}^{\bm \sigma} := \Set{ T_{\bm \sigma} u_{\mathcal N}(\bm \sigma_{n}) \given n = 1, \dots, N },
\end{equation}
wobei $T_{\bm \sigma}$ der Supremizing-Operator der Bilinearform $b(\blank, \blank; \bm \sigma) \colon \mathcal X_{\mathcal N} \times \mathcal Y_{\mathcal N} \to \mathbb{R}$ ist.
Diese Wahl des Testraumes ist vorteilhaft, wie die folgende Aussage nachweist.

\begin{Satz}
\label{satz:rb_testraum_liefert_korrekt_gestelltes_problem}
    Das RB"=Variationsproblem \cref{eq:rb_abstraktes_parametriches_vp} mit den Räumen $\mathcal X_{N}$ und $\mathcal Y_{N}^{\bm \sigma}$ ist für alle $\bm \sigma \in \mathcal P$ korrekt gestellt.

    \begin{Beweis}
        Wir weisen die Bedingungen des \acl{bnb}s nach, wobei hier nur die inf"=sup"=Bedingung von Interesse ist.
        Nach Annahme gilt für die Truth"=inf"=sup"=Konstante $\beta_{\mathcal N}(\bm \sigma) > 0$.
        Da nach Definition ferner für $u \in \mathcal X_{N}$ auch $T_{\bm \sigma}u \in \mathcal Y_{N}^{\bm \sigma}$ gilt, erhalten wir damit
        \begin{equation}
            \begin{aligned}
                \beta_{N}(\bm \sigma)
                &= \inf_{u \in \mathcal X_{N}} \sup_{v \in \mathcal Y_{N}^{\bm \sigma}} \frac{b(u, v; \bm \sigma)}{\norm{u}_{\mathcal X} \norm{v}_{\mathcal Y}}
                = \inf_{u \in \mathcal X_{N}} \frac{b(u, T_{\bm \sigma}u; \bm \sigma)}{\norm{u}_{\mathcal X} \norm{T_{\bm \sigma} u}_{\mathcal Y}}
                \\&= \inf_{u \in \mathcal X_{N}} \frac{\norm{T_{\bm \sigma}u}_{\mathcal Y}}{\norm{u}_{\mathcal X}}
                \geq \inf_{u \in \mathcal X_{\mathcal N}} \frac{\norm{T_{\bm \sigma}u}_{\mathcal Y}}{\norm{u}_{\mathcal X}}
                \\&= \beta_{\mathcal N}(\bm \sigma) > 0.
            \end{aligned}
        \end{equation}
    \end{Beweis}
\end{Satz}

Da $\mathcal Y_{N}^{\bm \sigma}$ für jeden Parameter neu aufgebaut werden muss, ist auch hierfür eine Zerlegung in Offline- und Online-Phase erwünscht.
Dazu nutzen wir die parametrisch-affine Darstellung der Bilinearform und definieren für jede Bilinearform $b_{q}$, $q = 1, \dots, Q_b$, den zugehörigen Supremizing-Operatoren als $T_{q}$ und es gilt
\begin{equation}
    T_{\bm \sigma} = \sum_{q = 1}^{Q_b} \theta_{q}^{b}(\bm \sigma) T_{q}.
\end{equation}
Die diskrete Darstellung $\mat{T}_{q} \in \mathbb{R}^{\mathcal N \times \mathcal N}$ der Operatoren $T_q$ ist nach Rieszschem Darstellungssatzes durch $\mat{T}_{q} = \mat{Y}^{-1} \mat{B}_{q}$ gegeben.
Dies erlaubt die folgende Zerlegung:

\begin{description}[font=\normalfont\itshape]
    \item[Offline:]Berechne und speichere für jedes $u_{\mathcal N}(\bm \sigma_{n}) \in \mathcal X_{N}$ die Vektoren $\vec{v}_{q}^{n} := \mat{T}_{q} \vec{u}_{\mathcal N}(\bm \sigma_{n}) = \mat{Y}^{-1} \mat{B}_{q} \vec{u}_{\mathcal N}(\bm \sigma_{n})$.
    \item[Online:] Bestimme $\mathcal Y_{N}^{\bm \sigma}$ durch Auswertung von $\mat{T}_{\bm \sigma} \vec{u}_{\mathcal N}(\bm \sigma_{n}) = \sum_{q = 1}^{Q_b} \theta_{q}^{b}(\bm \sigma) \vec{v}_{q}^{n}$.
\end{description}

% paragraph konstruktion_des_testraumes_ (end)

\section{Experimente} % (fold)
\label{sec:experimente}


% section experimente (end)

\end{document}

    % %!TEX root = ../main.tex

\chapter{Ausblick} % (fold)
\label{chapter:ausblick}

% chapter ausblick (end)


    % % \input{chapters/cha5_eindim}
    % Alles, was noch unbedingt rein muss
    % %!TEX root = ../main.tex

\setchapterpreamble[ul][0.6\textwidth]{%
    \dictum[Terry Pratchett]{\enquote{Coffee is a way of stealing time that should by rights belong to your older self.}}
    \vspace*{2\baselineskip}
}
\chapter{Funktionalanalytische Grundlagen} % (fold)
\label{cha:funktionalanalytische_grundlagen}

\todo[inline]{Ständig: ordnen, sortieren, aufräumen, erweitern.}

\section{Orthogonale Funktionen und Polynome}
\label{sec:orthogonale_funktionen_und_polynome}

\begin{Satz}[Orthogonalität trigonometrischer Funktionen]
\label{satz:trigonometrische_funktionen_orthogonal}
    Seien $k, l \in \mathbb{N}$.
    Dann gilt
    \begin{align}
        \skprod{\sin(\pi k x)}{\sin(\pi l x)}_{L_{2}([0, 1])} &= \frac{1}{2} \delta_{kl},
        % \quad\text{und}\quad
        \\\skprod{\cos(\pi k x)}{\cos(\pi l x)}_{L_{2}([0, 1])} &= \frac{1}{2} \delta_{kl},
        \\\skprod{\sin(\pi k x)}{\cos(\pi l x)}_{L_{2}([0, 1])} &= 0.
    \end{align}
\end{Satz}

\begin{Definition}[Legendre-Polynome]
\label{definition:legendre_polynome}
    Sei $I = [-1, 1]$.
    Die Legendre-Polynome $L_{n} \in \Pi_{n}$ sind definiert durch
    \begin{equation}
        L_{n}(x) = \frac{1}{2^{n}n!}\frac{\diff^{n}}{\diff x^{n}} (x^{2} - 1)^{n}.
    \end{equation}
    Durch die Transformation $x \mapsto 2x - 1$ erhält man die auf das Interval $[0, 1]$ geshifteten Legendre-Polynome $\tilde L_{n}$.
\end{Definition}

\begin{Satz}[Orthogonalität der Legendre-Polynome]
\label{satz:legendre_polynome_orthogonal}
    Die Legendre-Polynome $L_{n}$ sind orthogonal bezüglich der $L_{2}([-1, 1])$-Norm, denn es gilt
    \begin{equation}
        \skprod{L_{n}}{L_{m}}_{L_{2}([-1, 1])} = \frac{2}{2n + 1} \delta_{n m}.
    \end{equation}
    Auch für die geshifteten Legendre-Polynome $\tilde L_{n}$ gilt Orthogonalität, denn es ist
    \begin{equation}
        \skprod{\tilde L_{n}}{\tilde L_{m}}_{L_{2}([0, 1])} = \frac{1}{2n + 1} \delta_{n m}.
    \end{equation}
\end{Satz}

\begin{Bemerkung}
\label{satz:legendre_polynome_rekursion}
    Die Legendre-Polynome $L_{n}$ erfüllen die Rekursionsformel
    \begin{equation}
        n L_{n}(x) = (2n - 1) x L_{n-1}(x) - (n - 1) L_{n-2}(x), \quad L_{0}(x) = 1, L_{1}(x) = x.
    \end{equation}
    Analog gilt für die erste Ableitung $L_{n}'$ die Rekursionsformel
    \begin{equation}
        (n - 1) L_{n}'(x) = (2n -1) x L_{n-1}'(x) - n L_{n-2}'(x), \quad L_{0}'(x) = 0, L_{1}'(x) = 1.
    \end{equation}
\end{Bemerkung}

\section{Sonstiges} % (fold)
\label{sec:sonstiges}

% \begin{Lemma}
%     $\mathcal C^{0}([a, b]; X)$ liegt dicht in $L_{p}(a, b; X)$ für $1 \leq p < \infty$.
% \end{Lemma}

% TODO: zitieren
\begin{Satz}[Poincaré-Friedrichs-Ungleichung, vgl. {{\cite[Lemma 89.4]{HankeBourgeois:2009fk}}}]
\label{satz:grundlagen:poincare_friedrichs_ungleichung}
    Sei $\Omega \subset \mathbb{R}^{n}$ offen, beschränkt und mit Lipschitz-Rand.
    Dann existiert eine Konstante $\gamma_{\Omega} > 0$ mit
    \begin{equation}
        \label{eq:grundlagen:poincare_friedrichs_ungleichung}
        \norm{\grad u}_{L_{2}(\Omega)} \geq \gamma_{\Omega} \norm{u}_{H^{1}(\Omega)} \quad \fa u \in H^{1}_{0}(\Omega).
    \end{equation}
\end{Satz}

\begin{Satz}[Poincaré-Friedrichs-Ungleichung, vgl. {{\cite[Theorem II.1.7]{Braess:2007wm}}}]
    Es sei $\Omega \subset \mathbb{R}^{n}$ beschränkt und in einem $n$-dimensionalen Würfel mit Seitenlänge $s$ enthalten.
    Dann gilt
    \begin{equation}
        (1 + s)^{m} \abs{u}_{H^{m}} \geq \norm{u}_{H^{m}} \geq \abs{u}_{H^{m}} \quad \text{für alle}~u \in H^{m}_{0}(\Omega).
    \end{equation}
\end{Satz}

\begin{Lemma}[{{\cite[Remark 2.1.48]{Sauter:9_WoPZ0Y}}}]
\label{lemma:sauter:2.1.48}
    Seien $X$ und $Y$ zwei reflexive Banachräume und $a \colon X \times Y \to \mathbb{R}$ eine Bilinearform.
    Finden wir für jedes $x \in X$ ein $y_{x} \in Y$, so dass
    \begin{equation}
        \label{eq:lemma:sauter:2.1.48:eq1}
        \abs{a(x, y_{x})} \geq C_{1} \norm{x}_{X}^{2} \quad \text{und} \quad \norm{y_{x}}_{Y} \leq C_{2} \norm{x}_{X}
    \end{equation}
    mit von $x$ und $y_{x}$ unabhängigen Konstanten $C_{1}, C_{2} > 0$ gilt, dann folgt daraus die inf-sup-Bedingung
    \begin{equation}
    \label{eq:lemma:sauter:2.1.48:inf_sup}
        \inf_{0 \neq x \in X} \sup_{0 \neq y \in Y} \frac{a(x, y)}{\norm{x}_{X}\norm{y}_{Y}} \geq \gamma > 0
    \end{equation}
    mit $\gamma = \frac{C_{1}}{C_{2}}$.

    \begin{Beweis}
        Seien $x \in X$ und $y_{x} \in Y$ so, dass \cref{eq:lemma:sauter:2.1.48:eq1} erfüllt ist.
        Dann gilt
        \begin{align}
            \inf_{0 \neq x \in X} \sup_{0 \neq y \in Y} \frac{\abs{a(x, y)}}{\norm{x}_{X} \norm{y}_{Y}}
            &\geq
            \inf_{0 \neq x \in X} \frac{\abs{a(x, y_{x})}}{\norm{x}_{X} \norm{y_{x}}_{Y}}
            \\&\geq
            \inf_{0 \neq x \in X} \frac{C_{1} \norm{x}^{2}_{X}}{\norm{x}_{X} C_{2} \norm{x}_{X}}
            =
            \frac{C_{1}}{C_{2}}
            > 0.
        \end{align}
    \end{Beweis}
\end{Lemma}

% section sonstiges (end)

    % %!TEX root = ../main.tex

\chapter{Notizen} % (fold)
\label{cha:notizen}

% section zum_petrov_galerkin_verfahren (end)

\section{Reduzierte-Basis-Methode} % (fold)
\label{sec:reduzierte_basis_methode}

Bei der Reduzierte-Basis-Methode für die Raum-Zeit-Variationsformulierung parabolischer partieller Differentialgleichungen haben sich folgende Punkte ergeben, welche eine Anmerkung verdienen.

Seien dazu $\mathcal X^{\mathcal N} = \spn\Set{\phi_{n}}_{n=1}^{\mathcal N}$ und $\mathcal Y^{\mathcal M} = \spn\Set{\psi_{m}}_{m = 1}^{\mathcal M}$ endlichdimensionale Hilberträume, beispielsweise aus dem Petrov-Galerkin-Verfahren.
Im Allgeimeinen muss hier nicht $\mathcal N = \mathcal M$ gelten.
Weiter bezeichnen wir mit $\mathcal X_{N} \subset \mathcal X^{\mathcal N}$ und $\mathcal Y_{\mathcal M} \subset \mathcal Y_{M}$ die Reduzierte-Basis-Räume.

Wir betrachten das abstrakte Variationsproblem
\begin{equation}
    b(u, v; \mu) = f(v; \mu) \qquad \text{mit}~u \in \mathcal X,~v \in \mathcal Y,
\end{equation}
wobei $b \colon \mathcal X \times \mathcal Y \times \mathcal P \to \mathbb{R}$ eine affin parametrische Bilinearform und $f \colon \mathcal Y \times \mathcal P \to \mathbb{R}$ ein affin parametrisches lineares stetiges Funktional sei,
das heißt, es gelte
\begin{equation}
    b(u, v; \mu) = \sum_{q = 1}^{Q_b} \theta^{b}_{q}(\mu) b_{q}(u, v)
    \qquad \text{und} \qquad
    f(v; \mu) = \sum_{q = 1}^{Q_f} \theta^{f}_{q}(\mu) f_{q}(v).
\end{equation}

\paragraph{A posteriori Fehlerschätzer} % (fold)
\label{par:a_posteriori_fehlersch_tzer}

Der A-posteriori-Fehlerschätzer ergibt sich wie bei Reduzierte-Basis-Methoden üblich folgendermaßen:

Sei $\mu \in \mathcal P$ ein Parameter, $u^{\mathcal N}(\mu) \in \mathcal X^{\mathcal N}$ die Truth-Lösung des Variationsproblem für $\mu$ und $u_{N}(\mu) \in \mathcal X_{N}$ die entsprechende Reduzierte-Basis-Lösung.
Definiere den Fehler
\begin{equation}
    e_{N}(\mu) := u^{\mathcal N}(\mu) - u_{N}(\mu) \in \mathcal X^{\mathcal N}.
\end{equation}
Weiter wird das Residuum für alle $v \in \mathcal Y^{\mathcal M}$ definiert als
\begin{equation}
    r_{N}(v; \mu) = b(e_{N}(\mu), v; \mu) = f(v; \mu) - b(u_{N}(\mu), v; \mu).
\end{equation}
Fasst man das Residuum als rechte Seite des obigen Variationsproblems auf, dann ist $e_{N}(\mu)$ die zugehörige Lösung.
Weiter können wir die übliche Abschätzung (Lemma von Cea bzw. ähnliche Aussage) verwenden und erhalten die Ungleichung
\begin{equation}
    \norm{e_{N}(\mu)}_{\mathcal X} \leq \frac{1}{\beta^{\mathcal N}(\mu)} \norm{r_{N}(\blank; \mu)}_{\mathcal Y^{\mathcal M}'}.
\end{equation}
Dabei ist $\beta^{\mathcal N}(\mu)$ die inf-sup-Konstante des Truth-Probelms für den Parameter $\mu$.

\paragraph{Berechnung der inf-sup-Konstante} % (fold)
\label{par:berechnung_der_inf_sup_konstante}

Die obige inf-sup-Konstante $\beta^{\mathcal N}(\mu)$ wird in der Offline-Phase der Reduzierte-Basis-Methode für jeden Parameter $\mu$ des verwendeten Trainingsraums benötigt, muss also effizient auswertbar sein.

Zunächst eine Erklärung, wie man $\beta^{\mathcal N}(\mu)$ grundsätzlich berechnen kann.
Dazu greift man auf den sogenannten \emph{Supremizing Operator} zurück.
Dieser ist eine Abbildung $T_{\mu} \colon \mathcal X^{\mathcal N} \to \mathcal Y^{\mathcal M}$ definiert durch
\begin{equation}
    \skp{T_{\mu} u}{v}{\mathcal Y^{\mathcal M}} = b(u, v; \mu) \quad \fa v \in \mathcal Y^{\mathcal M}.
\end{equation}
Weiter gilt
\begin{equation}
    T_{\mu}u = \arg \sup_{v \in \mathcal{Y}^{\mathcal M}}  \frac{b(u, v; \mu)}{\norm{v}_{\mathcal Y^{\mathcal M}}}
\end{equation}
und damit
\begin{equation}
    \beta^{\mathcal N}(\mu) = \inf_{u \in \mathcal X^{\mathcal N}} \frac{\norm{T_{\mu}u}_{\mathcal Y^{\mathcal M}}}{\norm{u}_{\mathcal X^{\mathcal N}}}.
\end{equation}
Mittels des Rieszschen Darstellungssatzes lässt sich der Operator $T_{\mu}$ berechnen.

Sei dazu $\mat{Y} = [\skp{\psi_{m}}{\psi_{m'}}{\mathcal Y^{\mathcal M}}]_{m, m'}$, $\mat{X} = [\skp{\phi_{n}}{\phi_{n'}}{\mathcal X^{\mathcal N}}]_{n, n'}$ und $\mat{B}_{\mu} = [b(\phi_{n}, \psi_{m}; \mu)]_{m, n}$.
Dann gilt
\begin{equation}
    \mat{Y} \vec{T}_{\mu} = \mat{B}_{\mu}.
\end{equation}
Eingesetzt ergibt sich dann das Quadrat der inf-sup-Konstante dann als
\begin{equation}
    (\beta^{\mathcal N}(\mu))^{2} = \inf_{\vec{u} \in \mathbb{R}^{\mathcal N}} \frac{\vec{u}\Transp \mat{B}_{\mu}\tranps \mat{Y}^{-1} \mat{B}_{\mu} \vec{u}}{\vec{u}\Transp \mat{X} \vec{u}}
\end{equation}
und lässt sich als kleinster Eigenwert $\lambda$ des verallgemeinerten Eigenwertproblems
\begin{equation}
    \mat{B}_{\mu}\tranps \mat{Y}^{-1} \mat{B}_{\mu} \vec{x} = \lambda \mat{X} \vec{x}
\end{equation}
bestimmen.

Für die Successive Constraint Method, siehe \textcite{Huynh2007}.


% paragraph berechnung_der_inf_sup_konstante (end)

\paragraph{Stabiler Reduzierte-Basis-Testraum} % (fold)
\label{par:stabiler_reduzierte_basis_testraum}

Bei der Reduzierte-Basis-Methode für parabolische Probleme ergibt sich das Problem, dass die Lösungen nur den Reduzierte-Basis-Ansatzraum aufspannen. Der Testraum muss dagegen anderweitig konstruiert werden.
Hier scheint es verschiedene, hauptsächlich heuristisch motivierte Ansätze zu geben, die wiederum den obigen Supremizing Operator verwenden.
Beispiele sind \textcite[Abschnitt 4.2]{Mayerhofer:2014vx} beziehungsweise \textcite{Dahmen:2014cl}.

Momentan implementiert ist \textcite[Abschnitt 4.2]{Mayerhofer:2014vx}.
% paragraph stabiler_reduzierte_basis_testraum (end)


\section{Petrov-Galerkin} % (fold)
\label{sec:petrov_galerkin}

Die Zeitdiskretisierung wurde ausgetauscht. Statt Legendre-Polynomen werden nun, da es weit verbreitet zu sein scheint und laut \textcite{Andreev:2012uh,Andreev:2012ep,Andreev:2013gk} mit guten Stabilitätsergebnissen, nodale Hutfunktionen für den Ansatzraum und Indikatorfunktionen für den Testraum verwendet. Dabei wird im Allgeimeinen die Zeitdiskretisierung des Testraumes um den Faktor 2 verfeinert, wodurch sich die inf-sup-Stabilität ohne Beachtung einer CFL-Bedingung ergibt. (Die räumliche Diskretisierung muss ein, zwei Bedingungen erfüllen, mal checken).

Durch den größeren Testraum ergibt sich ein überbestimmtes System, welches im Sinne einer Residuum-Minimierung gelöst werden muss. Es lässt sich zeigen, dass dieses \emph{Minimales Residuum Petrov-Galerkin-Verfahren} ähnliche Aussagen wie das übliche Petrov-Galerkin-Verfahren erfüllt.

% section petrov_galerkin (end)

\section{Weiteres} % (fold)
\label{sec:weiteres}

\begin{itemize}
    \item Im Moment ist die periodische Randbedingung am laufen. Wird durch Fourier-Disrektisierung mit Konstanter Basisfunktion geregelt. Die Theorie wird aber für $H^{1}_{0, per}(\Omega) := H^{1}_{per}(\Omega) / \mathbb{R}$ angeregt. Schlimm? Wie anders lösbar?
    \item Wie könnte man die Anzahl der verwendeten Feld-Entwicklungsfunktionen einschränken?
    \item
\end{itemize}

% section weiteres (end)

    % %!TEX root = ../main.tex

\pagestyle{plain}

\section*{Eindimensionaler Fall mit $\omega \in \mathbb{R}$ und ohne Quellterm}

Sei $I := [0, \hat t]$ für ein $0 < \hat t < \infty$ und $\Omega := [0, 1]$.
Betrachte folgende parametrisierte PDE
\begin{align}
    \begin{cases}
    u_{t}(t, x) = \sigma u_{xx}(t, x) - \omega u(t, x), & (t, x) \in I \times \Omega\\
    u(0, x) = g(x), & x \in \Omega \\
    u(t, 0) = u(t, 1) = 0, & t \in I
    \end{cases}
\end{align}
mit Konstanten $\sigma, \omega \in \mathbb{R}$.

Ein Separation der Variablen Ansatz $u(t, x) = X(x) T(t)$ liefert
\begin{equation}
    X(x)T'(t) = \sigma X''(x) T(t) - \omega X(x) T(t)
\end{equation}
oder auch
\begin{equation}
    \frac{T'(t)}{T(t)} = \sigma \frac{X''(x)}{X(x)} - \omega = \lambda
\end{equation}
mit $\lambda \in \mathbb{R}$.

Ohne Einschränkung sei $\lambda \neq 0$, dann erhalten wir zum einen die Dgl.
    $T'(t) = \lambda T(t)$,
deren Lösung
\begin{equation}
    T(t) = d_{3} e^{\lambda t}
\end{equation}
ist, und zum anderen die Dgl.
    $X''(x) =  \frac{\lambda + \omega}{\sigma} X(x)$
mit der Lösung
\begin{equation}
    X(x) = d_{1} e^{\sqrt{\frac{\lambda + \omega}{\sigma}} x} + d_{2} e^{-\sqrt{\frac{\lambda + \omega}{\sigma}}x},
\end{equation}
wobei $d_{1}, d_{2}, d_{3} \in \mathbb{R}$.

Als nächstes Verwenden wir die Anfangs- und Randbedingungen um die Konstanten $d_{i}$ zu bestimmen.
Sei
\begin{equation}
    u(t, x) = \left( d_{1} e^{\sqrt{\frac{\lambda + \omega}{\sigma}} x} + d_{2} e^{-\sqrt{\frac{\lambda + \omega}{\sigma}}x} \right) \left( d_{3} e^{\lambda t} \right),
\end{equation}
Betrachten wir zunächst die Randbedingung $u(t, 0) = u(t, 1) = 0$, dann erhalten wir aus
\begin{equation}
    0 = u(t, 0) = \left( d_{1} + d_{2} \right) \left( d_{3} e^{\lambda t} \right),
\end{equation}
oder äquivalent $d_{1} = - d_{2}$, und aus 
\begin{equation}
    0 = u(t, 1) = \left( d_{1} e^{\sqrt{\frac{\lambda + \omega}{\sigma}}} + d_{2} e^{-\sqrt{\frac{\lambda + \omega}{\sigma}}} \right) \left( d_{3} e^{\lambda t} \right) = 
    d_{1} \left( e^{\sqrt{\frac{\lambda + \omega}{\sigma}}} - e^{-\sqrt{\frac{\lambda + \omega}{\sigma}}} \right) \left( d_{3} e^{\lambda t} \right),
\end{equation}
ohne Einschränkung $d_{1} \neq 0$, die Gleichung
\begin{equation}
    0 = e^{\sqrt{\frac{\lambda + \omega}{\sigma}}} - e^{-\sqrt{\frac{\lambda + \omega}{\sigma}}}.
\end{equation}
Aus dieser erhalten wir durch Äquivalenzumformungen
\begin{align}
    e^{\sqrt{\frac{\lambda + \omega}{\sigma}}} - e^{-\sqrt{\frac{\lambda + \omega}{\sigma}}} = 0
    &\quad \iff \quad
    e^{2\sqrt{\frac{\lambda + \omega}{\sigma}}} = 1
    \quad \iff \quad
    \sqrt{\tfrac{\lambda + \omega}{\sigma}} = k \pi i
    \\&\quad \iff \quad
    \tfrac{\lambda + \omega}{\sigma} = -k^2 \pi^2
    \quad \iff \quad
    \lambda = -k^2 \pi^2 \sigma - \omega,
\end{align}
mit $k \in \mathbb{Z}$ beliebig.
Einsetzen liefert nun
\begin{align}
    u_{k}(t, x) &= d_{1} \left( e^{k \pi i x} - e^{-k \pi i x} \right) \left( d_{3} e^{- (k^2 \pi^2 \sigma + \omega) t} \right)
    \\&= 2 d_{1} d_{3} i \sin(k \pi x) e^{-(k^2 \pi^2 \sigma + \omega)t},
\end{align}
wobei wir $\beta_{k} := 2 d_{1} d_{3} i$ setzen.

Da jedes $u_{k}$, $k \in \mathbb{Z}$, eine Lösung ist, erhalten wir durch
\begin{equation}
    u(t, x) = \sum_{k = 1}^{\infty} u_{k}(t, x) = \sum_{k = 1}^{\infty} \beta_{k} \sin(k \pi x) e^{-(k^2 \pi^2 \sigma + \omega)t}}
\end{equation}
ebenfalls eine Lösung. 
Damit die Anfangsbedingung erfüllt wird, muss
\begin{equation}
    g(x) = u(x, 0) = \sum_{k = 1}^{\infty} \beta_{k} \sin(k \pi x)
\end{equation}
gelten, was genau dann der Fall ist, wenn
\begin{equation}
    \beta_{k} = 2 \int_{0}^{1} g(x) \sin(k \pi x) \diff x.
\end{equation}

Da $u_{k}$ analytisch in $\omega$ für alle $k \in \mathbb{Z}$, ist auch $u$ analytisch in $\omega$ und es gilt
\begin{equation}
    \frac{\partial^{j} u(t, x; \omega)}{\partial \omega^{j}} = \sum_{k = 1}^{\infty} (-t)^{j} \beta_{k} \sin(k \pi x) e^{-(k^{2} \pi^{2} \sigma + \omega)t}.
\end{equation}


    %%% Anhang
    \appendix{}

    % DVD Inhalt
    % %!TEX root = ../main.tex

\chapter{Inhalt der Begleit-DVD} % (fold)
\label{cha:inhalt_der_begleit_dvd}

Dieser Arbeit liegt eine DVD bei, welche die Implementierungen der beschriebenen Verfahren, sowie die in \autoref{cha:Beispiele} angeführten Beispiele enthält.
Weiterhin findet man den Inhalt der DVD identisch auch als Git-Repository unter \url{https://github.com/nobbs/thesis}.

Die Implementierungen sind vollständig in \textcite{Matlab} gehalten und daher plattformunabhängig.
Neben dem eigentlichen MATLAB-Softwarepaket wird nur die \emph{Optimization Toolbox} benötigt für die in \autoref{scm} beschrieben Successive Constraint Method benötigt.

\begin{figure}[tb]
    \dirtree{%
    .1 /.
    .2 code.
    .3 examples.
    .4 datasets.
    .3 lib.
    .3 src.
    .3 test.
    .2 config.
    .2 doc.
    .3 index.html.
    .2 tex.
    .2 README.md.
    }
    \caption{Ausschnitt der Verzeichnisstruktur}
    \label{fig:dvd}
\end{figure}

Ein Ausschnitt der wichtigsten Elemente der Verzeichnisstruktur findet sich in \autoref{fig:dvd}.



% chapter inhalt_der_begleit_dvd (end)


    %% Los geht's mit den Verzeichnissen

    % Abbildungsverzeichnis
    \listoffigures

    % Tabellenverzeichnis
    \listoftables

    % Symbolverzeichnis
    % \printnomenclature{}
    \glsaddall[]
    \printglossary[type=symbolslist, nonumberlist, style=long]

    % Akronymverzeichnis
    % \printacronyms[heading=chapter*]

    % Literaturverzeichnis
    \printbibliography

    % Eidesstattliche Erklärung
    % %!TEX root = ../main.tex

\chapter*{Eidesstattliche Erklärung}

Ich versichere hiermit, dass ich die vorliegende Masterarbeit selbständig
verfasst und keine anderen als die angegebenen Quellen und Hilfsmittel benutzt
habe, wobei ich alle wörtlichen und sinngemäßen Zitate als solche gekennzeichnet
habe. Die Arbeit wurde bisher keiner anderen Prüfungsbehörde vorgelegt und auch
nicht veröffentlicht.\\[6ex]

\begin{flushright}
\ort, den \today

\color{jgu_hellgrau}\hdashrule[-0.5cm]{5cm}{0.5pt}{1pt}
\end{flushright}

\end{document}
