% -*- root: main.tex -*-

% griechisches Alphabet anpassen
\renewcommand{\epsilon}{\varepsilon}
\renewcommand{\theta}{\vartheta}


%%% Operatoren und anderes mathematisches Geplänkel

% Beträge und Normen
\DeclarePairedDelimiter{\abs}{\lvert}{\rvert}
\DeclarePairedDelimiter{\norm}{\lVert}{\rVert}

% Gauß-Klammern
\DeclarePairedDelimiter{\ceil}{\lceil}{\rceil}
\DeclarePairedDelimiter{\floor}{\lfloor}{\rfloor}

% Skalarprodukte und Duale Paarungen
\newcommand{\skprod}[2]{\left\langle#1,#2\right\rangle}
\newcommand{\Skp}[3]{\left\langle#1,#2\right\rangle_{#3}}
\newcommand{\skp}[3]{\langle#1,#2\rangle_{#3}}
\newcommand{\Dual}[2]{\left(#1,#2\right)}
\newcommand{\dual}[2]{(#1,#2)}

% Inneres und Äußeres
\newcommand{\Int}[1]{#1^\circ}
\newcommand{\Ext}[1]{\overline{#1}}

% Diverse Operatoren
\DeclareMathOperator{\spn}{span}
% \DeclareMathOperator{\ee}{e}
% \DeclareMathOperator{\ii}{i}
\newcommand{\ee}{\mathrm{e}}
\newcommand{\ii}{\mathrm{i}}
\newcommand*{\tran}{^{\mkern-1.5mu\mathsf{T}}}
\newcommand{\Stern}{^{*}}
\newcommand{\grad}{\nabla}
\DeclareMathOperator{\divergenz}{div}
\newcommand{\hesse}{\nabla^2}

% Einschränkung einer Funktion
\newcommand\restr[2]{\ensuremath{\left.#1\right|_{#2}}}

% Ordentlicher Befehl um Mengen zu setzen
\providecommand\given{} % so it exists
\newcommand\SetSymbol[1][]{
   \nonscript\,#1\vert\nonscript\,\mathopen{}\allowbreak}
\DeclarePairedDelimiterX\Set[1]{\lbrace}{\rbrace}{ \renewcommand\given{\SetSymbol[\delimsize]} #1 }

% Vektoren und Matrizen anders setzen
\renewcommand{\vec}[1]{\mathbf{#1}}
\newcommand{\mat}[1]{\mathbf{#1}}

% Differential-d für Integral ordentlich setzen
\newcommand*\diff{\mathop{}\!\mathrm{d}}
\newcommand*\Diff[1]{\mathop{}\!\mathrm{d^#1}}

% wesentliches Supremum
\DeclareMathOperator*{\esssup}{ess\,sup}
\DeclareMathOperator*{\argsup}{arg\,sup}
\DeclareMathOperator*{\argmax}{arg\,max}
\newcommand\blank{{\mkern2mu\cdot\mkern2mu}}
\newcommand\infsup[2]{\adjustlimits\inf_{#1}\sup_{#2}}
\newcommand\supsup[2]{\adjustlimits\sup_{#1}\sup_{#2}}

% "Definition-Gleich"
% FIXME: deprecated
\newcommand\deq{\coloneqq}

% Schrift einfach anpassen
\newcommand{\changefont}[3]{
\fontfamily{#1} \fontseries{#2} \fontshape{#3} \selectfont}


%%% Textbausteine

\newcommand{\fa}{\text{für alle}~}

% um fremde Begriffe einheitlich mit Übersetzung zu setzen
\newcommand\foreign[2]{#1 \textit{#2}}

% disjunkte vereinigung
\makeatletter
\providecommand*{\cupdot}{%
  \mathbin{%
    \mathpalette\@cupdot{}%
  }%
}
\newcommand*{\@cupdot}[2]{%
  \ooalign{%
    $\m@th#1\cup$\cr
    \hidewidth$\m@th#1\cdot$\hidewidth
  }%
}
\makeatother

% even und odd als text
\newcommand{\even}{\text{g}}
\newcommand{\odd}{\text{u}}


% aligned substack
\makeatletter
\newcommand{\subalign}[1]{%
  \vcenter{%
    \Let@ \restore@math@cr \default@tag
    \baselineskip\fontdimen10 \scriptfont\tw@
    \advance\baselineskip\fontdimen12 \scriptfont\tw@
    \lineskip\thr@@\fontdimen8 \scriptfont\thr@@
    \lineskiplimit\lineskip
    \ialign{\hfil$\m@th\scriptstyle##$&$\m@th\scriptstyle{}##$\crcr
      #1\crcr
    }%
  }
}
\makeatother
