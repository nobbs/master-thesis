%!TEX root = ../main.tex

\chapter{Einführung} % (fold)
\label{cha:einfuehrung}

Wir führen nun zunächst die parabolische partielle Differentialgleichung ganz allgemein, also noch ohne konkrete Ausrichtung auf die Ziele dieser Arbeit, ein.
Anschließend leiten wir für diese eine Raum-Zeit-Variationsformulierung und behandeln Existenz und Eindeutigkeit von Lösungen des daraus gewonnenen Variationsproblems.
Dieses Kapitel orientiert sich an den Ausführungen in \cite{Schwab:2009ec, Urban:2014kg}.

\section{Allgemeine Problemstellung} % (fold)
\label{sec:allgemeine_problemstellung}

Zunächst einige notationelle Vorbereitungen.
Seien $V$ und $H$ zwei separable Hilberträume, wobei eine dichte stetige Einbettung $V \hookrightarrow H$ existiere.
Durch Identifikation von $H$ mit seinem Dualraum $H'$ erhalten wir ein Gelfand-Tripel $V \hookrightarrow H \hookrightarrow V'$.
Wir verwenden die Schreibweise $\skprod{\cdot}{\cdot}$ mit entsprechendem Index sowohl für die inneren Produkte der Hilberträume, als auch für die duale Paarung auf $V' \times V$.

Es sei $0 < T < \infty$ und $I = [0, T]$.
Weiterhin sei $A \in \mathcal L(V, V')$ ein stetiger linearer Operator und $a(\cdot, \cdot) \colon V \times V \to \mathbb{R}$ die zugehörige Bilinearform, das heißt es gilt $\skprod{A \eta}{\zeta}_{V' \times V} = a(\eta, \zeta)$ für $\eta, \zeta \in V$.
Seien außerdem $g \in L_{2}(I; V')$ und $u_{0} \in H$ gegeben.
Wir sind nun an Lösungen der parabolischen partiellen Differentialgleichung
\begin{equation}
    \label{eq:allgemeine_parabolische_pde}
    u_{t}(t) + A u(t) = g(t) \quad \text{in}~V',
    \qquad
    u(0) = u_{0} \quad \text{in}~H
\end{equation}
interessiert.

Von der Bilinearform $a(\cdot, \cdot)$ fordern wir außerdem Stetigkeit, das heißt die Existenz einer Konstante $0 < M_{a} < \infty$, so dass für alle $\eta, \zeta \in V$ die Ungleichung
\begin{equation}
    \label{eq:allgemeine_parabolische_pde:bf_stetig}
    \abs{a(\eta, \zeta)} \leq M_{a} \norm{\eta}_{V} \norm{\zeta}_{V}
\end{equation}
erfüllt ist, und eine G\aa rding-Ungleichung, das heißt es existieren Konstanten $\alpha > 0$ und $\lambda \geq 0$ mit
\begin{equation}
    \label{eq:allgemeine_parabolische_pde:bf_garding}
    a(\eta, \eta) + \lambda \norm{\eta}_{H}^{2} \geq \alpha \norm{\eta}_{V}^{2}
\end{equation}
für alle $\eta \in V$.

% section allgemeine_problemstellung (end)

\section{Raum-Zeit-Variationsformulierung} % (fold)
\label{sec:raum_zeit_variationsformulierung}

Seien $V$ und $H$ Hilberträume mti $V \hookrightarrow H$.
Dann bildet $V \hookrightarrow H \cong H' \hookrightarrow V'$ ein sogenanntes
Gelfand-Triple.
Wir schreiben $\skprod{\cdot}{\cdot}_{V}$, $\skprod{\cdot}{\cdot}_{H}$ für das
Skalarprodukt auf $V$ und $\skprod{\cdot}{\cdot}_{V' \times V}$ für das
\emph{Duality Pairing} auf $V' \times V$.:w

Sei $0 < T < \infty$ und $I \coloneqq [0, T] \subset \mathbb{R}$.
Für fast alle $t \in I$ sei $A(t) \in \mathcal L(V, V')$ und
\begin{equation}
    a(t; \cdot, \cdot) \colon V \times V \to \mathbb{R}
\end{equation}
die zugehörige Bilinearform, das heißt es gilt
\begin{equation}
    \skprod{A(t)u}{v}_{H} = a(t; u, v)
\end{equation}

Wir fordern von der Bilinearform $a(\cdot; \cdot, \cdot)$, dass Konstanten $M_a,
\alpha > 0$ und $\lambda \in \mathbb{R}$ existieren, so dass für fast alle $t
\in I$ gilt
\begin{itemize}
    \item \emph{Stetigkeit:} für alle $u, v \in V$ gilt
        \begin{equation}
            \label{eq:stetigkeit}
            \abs{a(t; u, v)} \leq M_a \norm{u}_V \norm{v}_V,
        \end{equation}
    \item \emph{Garding-Ungleichung:} für alle $u \in V$ gilt
        \begin{equation}
            \label{eq:garding-inequality}
            a(t; u, u) + \lambda \norm{u}^2_H \geq \alpha \norm{u}^2_V.
        \end{equation}
\end{itemize}

Wir wollen nun die parabolische partielle Differentialgleichung
\begin{equation}
    \label{eq:}
    \begin{aligned}
        u_t(t) + A(t) u(t) &= g(t) &\qquad \text{in}~V'\\
        u(0) &= u_0 \qquad \text{in}~H.
    \end{aligned}
\end{equation}

Nach Multiplikation mit einer Testfunktion $v \coloneqq (v_1, v_2) \in \mathcal{Y}$ und Integration nach Ort und Zeit erhalten wir
\begin{equation}
    \label{eq:bilinearform}
    b(u, v) = F(v),
\end{equation}
wobei
\begin{equation}
    b(u,v) = \int_{I} \skprod{u_{t}(t)}{v_{1}(t)}_{H} + a(t; u(t), v_{1}(t)) \diff t + \skprod{u(0)}{v_{2}}_{H}
\end{equation}
und
\begin{equation}
    F(v) = \int_{I} \skprod{g(t)}{v_{1}(t)}_{H} \diff t + \skprod{u_{0}}{v_{2}}_{H}
\end{equation}

% section raum_zeit_variationsformulierung (end)

% chapter einfuehrung (end)
