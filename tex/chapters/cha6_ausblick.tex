% -*- root: ../main.tex -*-

\documentclass[../main.tex]{subfiles}
\begin{document}

\chapter{Fazit \& Ausblick} % (fold)
\label{chapter:ausblick}

In diesem abschließenden Kapitel wollen wir die Ergebnisse dieser Arbeit noch einmal diskutieren.
Dabei betrachten wir diese hauptsächlich unter dem Aspekt der Anwendbarkeit auf die aus der \nameref{chapter:einleitung} bekannte selbstkonsistente Feldtheorie.

Analog zum Aufbau der Arbeit beginnen wir mit den funktionalanalytischen Ergebnissen aus \cref{chapter:propagator_differentialgleichung}, bevor wir weiter auf die zugrundeliegende Diskretisierung mittels \nameref{chapter:galerkin} und zuletzt auf die darauf aufbauende \nameref{chapter:rbm} eingehen.

\paragraph{Funktionalanalytische Ergebnisse.} % (fold)
\label{par:funktionalanalytische_ergebnisse}

\cref{chapter:propagator_differentialgleichung} lässt sich, neben der Herleitung der parametrischen Raum"=Zeit"=Variationsformulierung für die Propagator"=Differentialgleichung, zu den folgenden beiden Hauptergebnissen zusammenfassen:

\begin{enumerate}[label={\itshape\roman*.},ref={\itshape\roman*}]
    \item
    Der Nachweis, dass die hergeleitete Raum"=Zeit"=Variationsformulierung ein im Sinne von Hadamard korrekt gestelltes Problem darstellt.
    \item
    Die Untersuchung der Regularität der Lösungen der parametrischen Raum"=Zeit"=Variationsformulierung bezüglich der Parameter.
    Dabei wurde insbesondere eine hinreichende Bedingung dafür angegeben, wann diese Abhängigkeit analytisch ist.
\end{enumerate}

Vor allem der zweite Punkt bietet noch Potenzial für weitere Untersuchungen.
Die im Rahmen dieser Arbeit hergeleitete hinreichende Bedingung (siehe \cref{satz:loesungen_analytisch}) erweist sich als äußerst restriktiv, da sie maßgeblich festlegt, wie groß die Amplitude der verwendeten Felder sein darf und die zu erfüllende Schranke relativ niedrig ist.

Inwiefern dieses Ergebnis mit dem Ziel der analytischen Regularität verbessert werden kann, ist an dieser Stelle unklar.
Im Rahmen der Forschungsphase für das erzielte Ergebnis wurden verschiedene Ansätze verfolgt, die im Endeffekt alle zu einer ähnlich einschränkenden hinreichenden Bedingung führten.

Alternativ kann der Anspruch an die Regularität verringert werden, da Analytizität für die Anwendung der Reduzierte"=Basis"=Methode zwar gut, aber nicht notwendig ist.
Aufgrund der erzielten Ergebnisse liegt es nahe, dass schwächere Regularitätsergebnisse unter deutlich angenehmeren Bedingungen erreicht werden können.

\paragraph{Petrov-Galerkin-Diskretisierung.} % (fold)
\label{par:petrov_galerkin_diskretisierung}

In \cref{chapter:galerkin} wurde eine Diskretisierung durchgeführt, die als Grundlage für die Reduzierte"=Basis"=Methode verwendbar ist.
Weiter wurde eine hinreichende CFL"=Bedingung angegeben, unter welcher diese Diskretisierung stabil ist.

Auch hier bietet sich weiteres Verbesserungspotenzial.
Als erste einfache Verbesserung bietet es sich an, bei der numerischen Umsetzung die Raum"=Zeit"=Struktur stärker zu nutzen.
So müssen die vorkommenden Kronecker"=Produkte nicht zu $\mathcal N$-dimensionalen Objekten ausgewertet werden, da viele Berechnungen, die auf diesen Objekten basieren, durch Äquivalenzaussagen zu Berechnungen in den Dimensionen $\mathcal K$ und $\mathcal J$ überführt werden können.

Eine deutlich weiter ausholende Option ist, ein bedingungslos stabiles Verfahren zu konstruieren.
\textcite[Section 5.2]{Andreev:2012ep} hat gezeigt, dass die CFL"=Bedingung für das verwendete Verfahren möglicherweise verbessert, aber im Allgemeinen nicht entfernt werden kann.
Eine einfache Variante, um aus dem vorliegenden Verfahren ein bedingungslos stabiles zu erhalten, ist es, für den Testraum eine verfeinerte Zeitdiskretisierung zu verwenden.
Teilt man die Zeitgitterintervalle des Ansatzraumes für den Testraum jeweils in der Mitte auf, so führt dies zu bedingungsloser Stabilität.
Als Problem erweist sich nun aber, dass das resultierende System in Ansatz- und Testraum nicht mehr die gleiche Dimension besitzt und dementsprechend nur noch als Residuum"=minimierendes Variationsproblem aufgefasst werden kann.
Dieses kann aber nicht mehr als Grundlage für die Reduzierte"=Basis"=Methode verwendet werden.

Es bleibt zu untersuchen, ob und wie das verwendete Verfahren optimiert oder möglicherweise durch ein \enquote{besseres} Petrov"=Galerkin"=Verfahren ersetzt werden kann.

\paragraph{Reduzierte"=Basis"=Methode.} % (fold)
\label{par:reduzierte_basis_methode}

Als nächster Schritt wurde in \cref{chapter:rbm} die Reduzierte"=Basis"=Methode eingeführt.
Die theoretischen Grundlagen können zwar kurz gefasst werden, nichtsdestotrotz weist dieses Verfahren verschiedene verbesserungswürdige Punkte auf.

Wir beschränken uns auf den nach den Beispielen in \cref{sec:cha5_rbm:beispiele} offensichtlichen Ansatzpunkt, die Berechnung einer unteren Schranke für die parameterabhängige inf"=sup"=Konstante.
Dies stellt ein Kernstück der Reduzierte"=Basis"=Methode dar und wurde hier mit der \acl{scm} angegangen.
Dabei hat sich herausgestellt, dass diese stark unter dem Fluch der Dimensionalität leidet.
Ist die \ac{scm} bei einem Parameter noch in wenigen Minuten ausführbar, so benötigt sie bereits bei einer Parameterzahl im mittleren einstelligen Bereich mehrere Stunden.


Um dies zu verbessern, bietet sich zum einen die Verwendung einer optimierten \ac{scm}-Variante an wie beispielsweise \cite{Huynh2010}.
Zum anderen können möglicherweise die Schranken nach \cite{Schwab:2009ec} (siehe \cref{korrolar:ss09:theorem51_abschaetzungen}) als Ersatz zur \ac{scm} verwendet werden.
Für Letzteres bleibt aber zunächst die Exaktheit dieser Schranken zu untersuchen.
Spiegelt die Abschätzung das Verhalten der exakten inf"=sup"=Konstanten nur unzureichend wider, dann verfälscht dies insbesondere die Resultate des verwendeten Greedy"=Verfahrens.

Auch das Greedy-Iterationsverfahren der Reduzierte"=Basis"=Methode scheint stark unter der Erhöhung der Parameterdimension zu leiden, liefert hier aber nur bedingt aussagekräftige Ergebnisse, da die Bestimmung der inf-sup-Konstanten die weitere Analyse für höhere Dimensionen erschwert.

\paragraph{Selbstkonsistente Feldtheorie.} % (fold)
\label{par:selbstkonsistente_feldtheorie}

Als größter offener Punkt verbleibt die Frage, wie die in dieser Arbeit vorgestellte Modellreduktion verwendet werden kann, um die Berechnungen im Rahmen der selbstkonsistenten Feldtheorie zu beschleunigen.

Die vergleichsweise einfachste Option stellt die Substitution des in der \nameref{chapter:einleitung} beschriebenen Differentialgleichung"=Lösers im iterativen Verfahren durch die konstruierte Reduzierte"=Basis"=Methode dar.
Diese erfordert in der aktuellen Umsetzung a priori relativ tiefgehende Informationen, beispielsweise Symmetrien, über die resultierenden stabilen Felder, da sonst die Dimension des verwendeten Parameterraumes unpraktikabel groß wird.
Weiter muss hierfür bei Änderung der Modellparameter vor dem iterativen Verfahren stets zunächst der Offline-Anteil der Reduzierte"=Basis"=Methode ausgeführt werden, was bei aktueller Umsetzung jegliche mögliche Zeiteinsparung durch die Modellreduktion zerstört.

Um auf solches a priori-Wissen verzichten zu können, ist eine dynamische Reduzierte"=Basis"=Methode gefragt, die nicht einfach nur als Löser"=Ersatz für die Propagator"=Differentialgleichungen verwendet wird.
Diese sollte nach Möglichkeit direkt mit dem iterativen Verfahren gekoppelt sein und während der Ausführung \enquote{lernen}, welche Funktionen ein zur Modellreduktion der Felder geeignetes System darstellen.

Unseres Wissens wurden dynamische Reduzierte"=Basis"=Methoden dieser Art bisher nicht untersucht und stellen damit einen größeren, noch unerforschten Bereich dar.

\end{document}
