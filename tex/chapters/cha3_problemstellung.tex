%!TEX root = ../main.tex

\setchapterpreamble[ul][0.6\textwidth]{%
    \dictum[Andy Weir, \textit{The Martian}]{\enquote{I guess you could call it a \enquote{failure}, but I prefer the term \enquote{learning
    experience}.}}
    \vspace*{2\baselineskip}
}
\chapter{Problemstellung} % (fold)
\label{cha:ps:problemstellung}

In diesem Kapitel greifen wir die in \cref{cha:el:einleitung} als Propagatoren bezeichneten parabolischen partiellen Differentialgleichungen \cref{eq:el:forward_propagator,eq:el:backward_propagator} auf, konkretisieren die Rahmenbedingungen und reinterpretieren diese Propagatoren anschließend als parametrische Differentialgleichungen.

Zudem leiten wir Variationsformulierungen her und weisen anhand dieser nach, dass diese sachgemäß gestellt sind und dass die Lösungen der parametrischen partiellen Differentialgleichung eine gewisse Regularität bezüglich der Parameter aufweisen.

\mdo{Quellen}

\section{Die parabolische partielle Differentialgleichung} % (fold)
\label{sec:ps:pde:die_parabolische_partielle_differentialgleichung}

Ohne Einschränkung reicht es, nur den Vorwärts-Propagator \cref{eq:el:forward_propagator} zu betrachten, da der Rückwärts-Propagator \cref{eq:el:backward_propagator} durch die simple Transformation $s \mapsto 1 - s$ auf die selbe Form, lediglich mit vertauschten Rollen bei den Feldern $\omega_{\mathrm{A}}$ und $\omega_{\mathrm{B}}$, gebracht werden kann.

Wir wollen nun zunächst das Setting festlegen und dabei die aus der Einleitung bekannte partielle Differentialgleichung etwas allgemeiner auffassen.

Seien also $0 < T < \infty$ eine Konstante und $I = [0, T]$ ein reelles Intervall.
Weiter sei $\Omega \subset \mathbb{R}^{n}$ ein Gebiet, das heißt, eine offene, nichtleere, zusammenhängende Teilmenge, mit Lipschitz-Rand.
Wir betrachten nun die partielle Differentialgleichung
\begin{equation}
\label{eq:ps:pde:pde_erste_erwaehnung}
    u_{t}(t, \vec{x}) = c \Delta_{\vec{x}} u(t, \vec{x}) - \omega(t, \vec{x}) u(t, \vec{x}) - \mu u(t, \vec{x}) \quad \text{auf}~I \times \Omega,
\end{equation}
wobei $c \in \mathbb{R}_{+}$ und $\mu \in \mathbb{R}$ Konstanten seien.
% TODO: eventuell sollte man das hier anders machen...
% Weiter sei $\omega \colon \Omega \to \mathbb{R}$ eine $L_{\infty}$-Abbildung, das heißt, wir verzichten an dieser Stelle zunächst auf die zeitliche Abhängigkeit, die in \cref{cha:el:einleitung} noch in Form eines Wechsels zwischen zwei Feldern $\omega_{1}, \omega_{2}$ zu finden war.
Weiter sei
\begin{equation}
\label{eq:ps:pde:omega_definition}
    \omega \colon I \times \Omega \to \mathbb{R}, \quad (t, \vec{x}) \mapsto
    \begin{cases}
        \omega_{1}(\vec{x}), & t \leq f, \\
        \omega_{2}(\vec{x}), & t > f,
    \end{cases}
\end{equation}
mit $L_{\infty}(\Omega)$-Abbildungen $\omega_{1}, \omega_{2}$ und einer Konstante $f \in I$.

Da, wie in \cref{cha:el:einleitung} erwähnt, konstante Verschiebungen der Felder $\omega_{i}$ keinen Einfluss auf das Ergebnis des dort beschriebenen Iterationsverfahrens haben führen wir den zusätzlichen Term $\mu u(t, x)$ ein.
Dies wird sich später noch als hilfreich beim Nachweis einiger Eigenschaften der Differentialgleichung erweisen.

Weiter beschränken wir uns an dieser Stelle auf den Fall homogener Randbedingungen in $\Omega$, das heißt, es gilt $\restr{u(t, \blank)}{\partial \Omega} = 0$ für alle $t \in I$.

Unter diesen Gegebenheiten können wir die obige partielle Differentialgleichung \cref{eq:ps:pde:pde_erste_erwaehnung} weiter konkretisieren; es bietet sich an, dazu die Räume $V = H^{1}_{0}(\Omega)$ und $H = L_{2}(\Omega)$ zu verwenden.
Dabei handelt es sich jeweils um einen separablen Hilbertraum und es existiert eine dichte stetige Einbettung $H^{1}_{0}(\Omega) \denseinclusion L_{2}(\Omega)$.
Nach \cref{def:gl:br:gelfand_tripel} erhalten wir daraus das Gelfand-Tripel
\begin{equation}
\label{eq:ps:pde:gelfand_triple}
    H^{1}_{0}(\Omega) \denseinclusion L_{2}(\Omega) \denseinclusion (H^{1}_{0}(\Omega))' = H^{-1}(\Omega).
\end{equation}

Mit diesen Vorbemerkungen können wir unter Hinzunahme eines Quellterms und einer Anfangsbedingung unser Problem wie folgt definieren:

\mwarn{Sind die Voraussetzungen so passend?}
\begin{Definition}[Starke Formulierung]
\label{def:ps:pde:starke_formulierung}
    Seien eine \emph{Anfangsbedingung} $u_{0} \in L_{2}(\Omega)$, ein \emph{Quellterm} $g \in L_{2}(I) \otimes H^{-1}(\Omega)$ und $\omega$ wie in \cref{eq:ps:pde:omega_definition}, sowie Konstanten $c \in \mathbb{R}_{+}$ und $\mu \in \mathbb{R}$ gegeben.
    Als \emph{starke Formulierung} bezeichnen wir die parabolische partielle Differentialgleichung
    \begin{equation}
    \label{eq:ps:pde:starke_formulierung}
        \left\{
        \begin{aligned}
            u_{t}(t, x) - c \Delta u(t, x) + \omega(t, x) u(t, x) + \mu u(t, x) &= g(t, x) \quad &&\text{auf}~I \times \Omega,\\
            u(0, x) &= u_{0}(x) \quad &&\text{auf}~\Omega.
        \end{aligned}
        \right.
    \end{equation}
\end{Definition}

\begin{Bemerkung}
\label{bem:ps:pde:definition_operator_A}
    Der notationellen Einfachheit halber definieren wir für $t \in I$ den Operator
    \begin{equation}
    \label{eq:ps:pde:definition_operator_A}
        A(t) \colon H^{1}_{0}(\Omega) \to H^{-1}(\Omega), \quad \eta \mapsto A(t) \eta = - c \Delta \eta + \omega(t, \blank) \eta + \mu \eta.
    \end{equation}
    Damit lässt sich die starke Formulierung \cref{eq:ps:pde:starke_formulierung} auch schreiben als
    \begin{equation}
        \left\{
        \begin{aligned}
            u_t(t) + A(t)u(t) &= g(t) \quad &&\text{in}~H^{-1}(\Omega)~\fa t \in I,\\
            u(0) &= u_{0} \quad &&\text{in}~L_{2}(\Omega).
        \end{aligned}
        \right.
    \end{equation}
\end{Bemerkung}

\subsection{Raum-Zeit-Variationsformulierung} % (fold)
\label{sub:ps:rzvp:raum_zeit_variationsformulierung}

% subsection variationsformulierungen (end)

Als nächsten Schritt wollen wir nun eine Raum-Zeit-Variationsformulierung, auch schwache Formulierung genannt, für das Problem  \cref{eq:ps:pde:starke_formulierung} herleiten.
Dazu ist es nützlich, den Operator $A(t)$ in Form einer Bilinearform zu schreiben.
Nach dem Rieszschen Darstellungssatz, vergleiche \cite[Theorem \S{}22.1]{Halmos:1957vd}, existiert eine Bilinearform
\begin{equation}
    a \colon H^{1}_{0}(\Omega) \times H^{1}_{0}(\Omega) \to \mathbb{R},
\end{equation}
so dass
\begin{equation}
    (A(t)\eta)\zeta = \skp{A(t)\eta}{\zeta}{H^{-1}(\Omega) \times H^{1}_{0}(\Omega)} = a(\eta, \zeta; t) \quad \fa \eta, \zeta \in H^{1}_{0}(\Omega)
\end{equation}
gilt.

\mfix{Nochmal genauer anschauen, ob das mit der dualen Paarung und dem Skalarprodukt auch so einfach geht.}

Für den gegebenen Fall können wir diese Bilinearform explizit angeben, denn es gilt für $\eta, \zeta \in H^{1}_{0}(\Omega)$ nach dem Gaußschen Integralsatz die Gleichheit
\begin{equation}
    \begin{aligned}
        a(\eta, \zeta; t)
        &=    \skp{A(t)\eta}{\zeta}{H^{-1}(\Omega) \times H^{1}_{0}(\Omega)}
        =  \skp{A(t)\eta}{\zeta}{L_{2}(\Omega)}
        \\&= - c \skp{\Delta \eta}{\zeta}{L_{2}(\Omega)}
                + \skp{\omega(t, \blank) \eta}{\zeta}{L_{2}(\Omega)}
                + \mu \skp{\eta}{\zeta}{L_{2}(\Omega)}
        \\&= c \skp{\grad \eta}{\grad \zeta}{L_{2}(\Omega)}
                + \skp{\omega(t, \blank) \eta}{\zeta}{L_{2}(\Omega)}
                + \mu \skp{\eta}{\zeta}{L_{2}(\Omega)}.
    \end{aligned}
\end{equation}

Mit Hilfe dieser Darstellung erhalten wir nun für die Bilinearform $a(\blank, \blank; t)$ und damit auch für den Operator $A(t)$ die folgenden Aussagen für ein festes $t \in I$.
%
\begin{Satz}
\label{satz:ps:rzvp:bilinearform_a_eigenschaften}
    Seien $c \in \mathbb{R}_{+}$, $\mu \in \mathbb{R}$, $\omega \in L_{\infty}(\Omega)$ und
    \begin{equation}
    \label{eq:ps:rzvp:bilinearform_a}
        \begin{aligned}
            &a(\blank, \blank) \colon H^{1}_{0}(\Omega) \times H^{1}_{0}(\Omega) \to \mathbb{R}, \\
            &(\eta, \zeta) \mapsto c\skp{\grad \eta}{\grad \zeta}{L_{2}(\Omega)} + \skp{\omega \eta}{\zeta}{L_{2}(\Omega)} + \mu \skp{\eta}{\zeta}{L_{2}(\Omega)}.
        \end{aligned}
    \end{equation}
    Dann erfüllt $a$ die folgenden Eigenschaften:
    \begin{thmenumerate}
        \item\label{satz:ps:rzvp:bilinearform_a_eigenschaften_stetig}
        \emph{Stetigkeit:} es gilt
        \begin{equation}
            \abs{a(\eta, \zeta)} \leq M_{a} \norm{\eta}_{H^{1}(\Omega)} \norm{\zeta}_{H^{1}(\Omega)} \quad \text{für alle}~\eta, \zeta \in H^{1}_{0}(\Omega)
        \end{equation}
        mit $M_{a} = \max\Set{c, \norm{\omega}_{L_{\infty}(\Omega)} + \abs{\mu}} \geq 0$.
        \item\label{satz:ps:rzvp:bilinearform_a_eigenschaften_garding}
        \emph{G\aa{}rding-Ungleichung:} es gilt
        \begin{equation}
                a(\eta, \eta) + \lambda \norm{\eta}_{L_{2}(\Omega)}^{2} \geq \alpha \norm{\eta}_{H^{1}(\Omega)}^{2} \quad \text{für alle}~\eta \in H^{1}_{0}(\Omega)
        \end{equation}
        mit $\alpha = c \gamma_{\Omega}^{2} > 0$ und $\lambda = \min\Set{\norm{\omega}_{L_{\infty}(\Omega)} - \mu, 0} \geq 0$, wobei $\gamma_{\Omega}$ die Poincaré-Friedrichs-Konstante ist.
    \end{thmenumerate}

    \begin{Beweis}
    Zunächst zeigen wir die Stetigkeit.
    Seien dazu $\eta, \zeta \in H^{1}_{0}(\Omega)$ beliebig.
    Unter Verwendung der Dreiecks- und der Cauchy-Schwarz-Ungleichung erhalten wir
    \begin{align}
        \abs{a(\eta, \zeta)}
        &= \abs{c \skprod{\grad \eta}{\grad \zeta}_{L_{2}(\Omega)} + \skprod{\omega \eta}{\zeta}_{L_{2}(\Omega)} + \mu \skp{\eta}{\zeta}{L_{2}(\Omega)} }
        \\&\leq c \abs{\skprod{\grad \eta}{\grad \zeta}_{L_{2}(\Omega)}} + \abs{\skprod{\omega \eta}{\zeta}_{L_{2}(\Omega)}} + \abs{\mu} \abs{\skp{\eta}{\zeta}{L_{2}(\Omega)}}
        \\&\leq c \norm{\grad \eta}_{L_{2}(\Omega)} \norm{\grad \zeta}_{L_{2}(\Omega)} + (\norm{\omega}_{L_{\infty}(\Omega)} + \abs{\mu}) \norm{\eta}_{L_{2}(\Omega)} \norm{\zeta}_{L_{2}(\Omega)}
        \\&\leq \max \Set{ c, \norm{\omega}_{L_{\infty}(\Omega)} + \abs{\mu}} \norm{\eta}_{H^{1}(\Omega)} \norm{\zeta}_{H^{1}(\Omega)}.
    \end{align}
    Für die G\aa{}rding-Ungleichung seien $\eta \in H^{1}_{0}(\Omega)$ und $\lambda \in \mathbb{R}$.
    Betrachte
    \begin{align}
        a(\eta, \eta) + \lambda \norm{\eta}^{2}_{L_{2}(\Omega)}
        &= c \norm{\grad \eta}^{2}_{L_{2}(\Omega)} + \skprod{\omega \eta}{\eta}_{L_{2}(\Omega)} + \mu \skprod{\eta}{\eta}_{L_{2}(\Omega)} + \lambda \skprod{\eta}{\eta}_{L_{2}(\Omega)}
        \\&= c \norm{\grad \eta}^{2}_{L_{2}(\Omega)} + \skprod{(\omega + \mu + \lambda) \eta}{\eta}_{L_{2}(\Omega)}.
    \end{align}
    Wählen wir nun $\lambda = \min\Set{\norm{\omega}_{L_{\infty}(\Omega)} - \mu, 0} \geq 0$, dann gilt $\omega + \mu + \lambda \geq 0$ fast überall in $\Omega$ und wir erhalten die Abschätzung
    \begin{align}
        a(\eta, \eta) + \lambda \norm{\eta}^{2}_{L_{2}(\Omega)}
        &\geq c \norm{\grad \eta}^{2}_{L_{2}(\Omega)},
        \intertext{woraus wir durch Anwenden der Poincaré-Friedrichs-Ungleichung (\cref{satz:gl:poincare_friedrichs_ungleichung})}
        a(\eta, \eta) + \lambda \norm{\eta}^{2}_{L_{2}(\Omega)}
        &\geq c \gamma_{\Omega}^{2} \norm{\eta}^{2}_{H^{1}(\Omega)}
    \end{align}
    folgern.
    \end{Beweis}
\end{Satz}

\begin{Korollar}
\label{kor:ps:rzvp:bilinearform_elliptisch}
    Ist $\mu \geq \norm{\omega}_{L_{\infty}(\Omega)}$, dann ist die Bilinearform $a$ aus \cref{satz:ps:rzvp:bilinearform_a_eigenschaften} elliptisch.
\end{Korollar}

Unter diesen Gegebenheiten können wir nun mit den theoretischen Grundlagen aus \cref{sec:gl:le:lineare_evolutionsgleichungen} eine sachgemäß gestellte Raum-Zeit-Variationsformulierung herleiten.
Als Ansatz- und Testfunktionenraum erhalten wir mit den konkret gewählten Hilberträumen
\begin{equation}
    \label{eq:ps:rzvp:ansatzraum_testraum}
    \mathcal X = L_{2}(I; H^{1}_{0}(\Omega)) \cap H^{1}(I; H^{-1}(\Omega))
    \quad \text{und} \quad
    \mathcal Y = L_{2}(I; H^{1}_{0}(\Omega)) \times L_{2}(\Omega).
\end{equation}
Das Raum-Zeit-Variationsproblem lautet damit:
\begin{Definition}[Schwache Formulierung]
\label{def:ps:rzvp:schwache_formulierung}
    Als \emph{schwache Formulierung}, oder auch \emph{Raum"=Zeit"=Va"-ri"-a"-ti"-ons"-for"-mu"-lie"-rung}, der parabolischen partiellen Differentialgleichung aus \cref{def:ps:pde:starke_formulierung} bezeichnen wir das folgende Variationsproblem:
    Gegeben einen Quellterm $g \in L_{2}(I; H^{-1}(\Omega))$ und eine Anfangsbedingung $u_{0} \in L_{2}(\Omega)$.
    Finde ein $u \in \mathcal X$ mit
    \begin{equation}
        \label{eq:ps:rzvp:schwache_formulierung}
        b(u, v) = f(v) \quad \text{für alle}~v = (v_{1}, v_{2}) \in \mathcal Y,
    \end{equation}
    wobei $b(\blank, \blank) \colon \mathcal X \times \mathcal Y \to \mathbb{R}$ die durch
    \begin{equation}
        \label{eq:ps:rzvp:schwache_formulierung_lhs_b}
        b(u, v)
            = \int_{I} \skprod{u_{t}(t)}{v_{1}(t)}_{L_{2}(\Omega)} + a(u(t), v_{1}(t); t) \diff t + \skprod{u(0)}{v_{2}}_{L_{2}(\Omega)}
    \end{equation}
    gegebene Bilinearform und $f(\blank) \colon \mathcal Y \to \mathbb{R}$ definiert ist durch
    \begin{equation}
        \label{eq:ps:rzvp:schwache_formulierung_rhs_f}
        f(v) = \int_{I} \skprod{g(t)}{v_{1}(t)}_{L_{2}(\Omega)} \diff t + \skprod{u_{0}}{v_{2}}_{L_{2}(\Omega)}.
    \end{equation}
\end{Definition}

Diese Variationsformulierung wird uns in den nachfolgenden Kapiteln als Ausgangspunkt für die von uns angestrebten numerischen Verfahren dienen.

\subsection{Existenz und Eindeutigkeit von Lösungen} % (fold)
\label{sub:ps:eel:existenz_und_eindeutigkeit_von_loesungen}

Wir weisen nun aufbauend auf \cref{sec:gl:le:lineare_evolutionsgleichungen} nach, dass das Raum-Zeit-Variationsproblem aus \cref{def:ps:rzvp:schwache_formulierung} sachgemäß gestellt ist, und bestimmen zudem Abschätzungen für die Norm des Operators und dessen Inverse.
Die Hauptarbeit dazu wurde bereits durch das \cref{satz:gl:le:bnb_theorem} und \cref{satz:gl:le:ss09_theorem51} geleistet und muss hier nur noch angewandt werden.

\begin{Satz}
\label{satz:ps:eel:schwache_formulierung_sachgemaess_gestellt}
    Seien $\mathcal X$ und $\mathcal Y$ gegeben wie in \cref{eq:ps:rzvp:ansatzraum_testraum} und sei $B \colon \mathcal X \to \mathcal Y'$ definiert durch
    \begin{equation}
        \skprod{Bu}{v}_{\mathcal Y' \times \mathcal Y}  = b(u, v), \quad u \in \mathcal X,~ v \in \mathcal Y,
    \end{equation}
    mit $b(\blank, \blank)$ wie in \cref{eq:ps:rzvp:schwache_formulierung_lhs_b}.
    Dann ist $B$ stetig invertierbar.
    %  und es gilt
    % \begin{equation}
    %     \norm{B}_{\mathcal L(\mathcal X, \mathcal Y')}
    %     \leq
    %     \frac{\sqrt{2 \max\Set{1, c^{2}, \norm{\omega}_{L_{\infty}(\Omega)}^{2}} + M_{e}^{2}}}{\max\Set{\sqrt{1 + 2 \norm{\omega}_{L_{\infty}(\Omega)}^{2} \rho^{4}}, \sqrt{2} }}
    % \end{equation}
    % und
    % \begin{equation}
    %     \norm{B^{-1}}_{\mathcal L( \mathcal Y', \mathcal X)}
    %     \leq \frac{e^{2 T \norm{\omega}_{L_{\infty}(\Omega)}} \max\Set{\sqrt{1 + 2 \norm{\omega}_{L_{\infty}(\Omega)}^{2} \rho^{4}}, \sqrt{2}} \sqrt{2 \max\Set{c^{-2} \gamma_{\Omega}^{-4}, 1} + M_{e}^{2}}}{\min\Set{c^{-1} \gamma_{\Omega}^{2}, c \gamma_{\Omega}^{2} \norm{\omega}_{L_{\infty}(\Omega)}^{-2}, c \gamma_{\Omega}^{2} }}.
    %     % \leq
    %     % \frac{\max\{\sqrt{ 1 + 2 \norm{\omega}_{L_{\infty}(\Omega)} \rho^{4}}, \sqrt{2} \}}{e^{-2 \norm{\omega}_{L_{\infty}(\Omega)} T}}
    %     % \frac{\sqrt{2 \max\{ 1, \sigma^{-2} \gamma_{\Omega}^{-4} \} + M_{e}^{2}}}{\min\{ \sigma \gamma_{\Omega}^{2} \norm{\omega}_{L_{\infty}(\Omega)}^{-2}, \sigma \gamma_{\Omega}^{2} \}}
    % \end{equation}
    % mit $M_{e}$ und $\rho$ wie in \cref{kor:gl:br:einbettungskonstante_M_e} respektive \cref{kor:gl:le:einbettungskonstante_rho}.
\end{Satz}

\mfix{Den Teil hier mal ordentlich formulieren und formatieren.}
\begin{Korollar}
\label{kor:ps:eel:schwache_formulierung_operator_schranken}
    Unter den selben Gegebenheiten wie in \cref{satz:ps:eel:schwache_formulierung_sachgemaess_gestellt} gilt konkret
    \begin{equation}
        \norm{B}_{\mathcal L(\mathcal X, \mathcal Y')} \leq \sqrt{2 \max\Set{1, c, \norm{\omega}_{L_{\infty}(\Omega)} + \abs{\mu}}^{2} + M_{e}^{2}}
    \end{equation}
    und
    \begin{equation}
        \norm{B^{-1}}_{\mathcal L(\mathcal Y', \mathcal X)} \leq \frac{\sqrt{2 \max \Set{(c \gamma_{\Omega}^{2})^{-2}, 1} + M_{e}^{2}}}{\gamma_{\Omega}^{2} \min\Set{1, c , c  (\norm{\omega}_{L_{\infty}(\Omega)} + \abs{\mu})^{-1}}}.
    \end{equation}

    Ist $\lambda \neq 0$, dann gilt
    \begin{equation}
        \norm{B}_{\mathcal L(\mathcal X, \mathcal Y')} \leq \frac{\sqrt{2 \max\Set{1, c, \norm{\omega}_{L_{\infty}(\Omega)} + \abs{\mu}}^{2} + M_{e}^{2}}}{\max\Set{\sqrt{1 + 2 \lambda^{2} \rho^{4}}, \sqrt{2}}}
    \end{equation}
    und
    \begin{equation}
        \norm{B^{-1}}_{\mathcal L(\mathcal Y', \mathcal X)} \leq e^{2 \lambda T} \max\Set{\sqrt{1 + 2 \lambda^{2} \rho^{4}}, \sqrt{2}} \frac{\sqrt{2 \max \Set{(c \gamma_{\Omega}^{2})^{-2}, 1} + M_{e}^{2}}}{\gamma_{\Omega}^{2} \min\Set{1, c , c  (\norm{\omega}_{L_{\infty}(\Omega)} + \abs{\mu})^{-1}}}.
    \end{equation}
\end{Korollar}



% subsection existenz_und_eindeutigkeit_von_l_sungen (end)

% section die_parabolische_partielle_differentialgleichung (end)

\section{Parametrische Formulierung} % (fold)
\label{sec:ps:pf:parametrische_formulierung}

Nachdem wir nun die Rahmenbedingungen geklärt haben, fahren wir fort mit der Einführung einer parametrischen Variante der in diesem Kapitel bisher betrachteten parabolischen partiellen Differentialgleichung.
Diese ermöglicht es uns in den folgenden Kapiteln erst, eine Reduzierte-Basis-Methode zu entwickeln.

Dazu greifen wir den aus \cref{bem:ps:pde:definition_operator_A} bekannten Operator $A(t)$ auf und betrachten diesen zunächst zwar für einen festen Zeitpunkt $t \in I$, dafür aber in Abhängigkeit von einem $\omega \in L_{\infty}(\Omega)$.
Sei also $\omega \in L_{\infty}(\Omega)$, dann definieren wir eine Abbildung
\begin{equation}
    \label{eq:ps:pf:operator_A}
    A(\omega) \colon H^{1}_{0}(\Omega) \to H^{-1}(\Omega), \quad A(\omega) \eta = - c \Delta \eta + \omega \eta + \mu \eta
\end{equation}
und erhalten wie zuvor die zugehörige Bilinearform $a(\blank, \blank; \omega)$ durch
\begin{equation}
    \label{eq:ps:pf:bilinearform_a}
    \begin{aligned}
        &a(\blank, \blank; \omega) \colon H^{1}_{0}(\Omega) \times H^{1}_{0}(\Omega) \to \mathbb{R}, \\
        &(\eta, \zeta) \mapsto c\skp{\grad \eta}{\grad \zeta}{L_{2}(\Omega)} + \skp{\omega \eta}{\zeta}{L_{2}(\Omega)} + \mu \skp{\eta}{\zeta}{L_{2}(\Omega)}.
    \end{aligned}
\end{equation}

Als nächstes konkretisieren wir die Abhängigkeit des Operators $A(\omega)$ beziehungsweise der Bilinearform $a(\blank, \blank; \omega)$ von $\omega$.
Dazu stellen wir die folgende Bedingung an den Parameter $\omega$, welche sich später wiederum bei der Reduzierte-Basis-Methode als nützlich erweisen wird.

\begin{Definition}
\label{def:ps:pf:omega_affin}
    Die Funktionen $\omega \in L_{\infty}(\Omega)$ seien \emph{affin} darstellbar.
    Genauer sei $\mathcal S \subset \mathbb{R}^{\mathbb{N}}$ ein Parameterraum und $\Set{ \varphi_{j} }_{j \in \mathbb{N}} \in L_{\infty}(\Omega)$ eine Folge von Funktion, so dass $\omega$ sich für $\sigma \in \mathcal S$ als Reihe der Form
    \begin{equation}
    \label{eq:ps:pf:omega_affine_zerlegung}
        w(\blank; \sigma) \colon \Omega \to \mathbb{R}, \quad w(x; \sigma) = \sum_{j = 1}^{\infty} \sigma_{j} \varphi_{j}(x)
    \end{equation}
    schreiben lässt.
\end{Definition}

\begin{Bemerkung}
    Wir wählen für den Rest der Arbeit $\mathcal S = [-1, 1]^{\mathbb{N}}$.
    Dies stellt keine Einschränkung dar, da die Funktionen $\Set{ \varphi_{j} }_{j \in \mathbb{N}}$ beliebig umskaliert werden können.
\end{Bemerkung}

Setzen wir diese affine Darstellung nun zunächst in den Operator $A(\omega)$ ein, dann erhalten wir die Darstellung
\begin{equation}
\label{eq:ps:pf:operator_A_sigma}
    A(\omega(\sigma)) \colon H^{1}_{0}(\Omega) \to H^{-1}(\Omega), \quad A(\omega(\sigma)) u = -c \Delta u + \sum_{j = 1}^{\infty} \sigma_{j} \varphi_{j} u + \mu u,
\end{equation}
und als zugehörige Bilinearform $a(\blank, \blank; \omega(\sigma))$ ergibt sich wie zuvor
\begin{equation}
\label{eq:ps:pf:bilinearform_a_sigma}
    \begin{aligned}
    &a(\blank, \blank; \omega(\sigma)) \colon H^{1}_{0}(\Omega) \times H^{1}_{0}(\Omega) \to \mathbb{R}, \\
    &(u, v) \mapsto c\skp{\grad u}{\grad v}{L_{2}(\Omega)} + \sum_{j = 1}^{\infty} \sigma_{j} \skp{\varphi_{j} u}{v}{L_{2}(\Omega)} + \mu \skp{u}{v}{L_{2}(\Omega)}.
    \end{aligned}
\end{equation}

\begin{Bemerkung}
    Um die Schreibweisen kurz zu halten, schreiben wir meist $\omega(\sigma)$ statt $\omega(\blank; \sigma)$.
    Weiter schreiben wir, vor allem um zwischen dem allgemeinen parametrischen und dem affinen Fall zu unterscheiden, kurz $A(\sigma)$ statt $A(\omega(\sigma))$ für \cref{eq:ps:pf:operator_A_sigma} und $a(\blank, \blank; \sigma)$ statt $a(\blank, \blank; \omega(\sigma))$ für \cref{eq:ps:pf:bilinearform_a_sigma}.
\end{Bemerkung}

Damit der Operator $A(\sigma)$ sowie die Bilinearform $a(\blank, \blank; \sigma)$ für $\sigma \in \mathcal S$ wohldefiniert sind, müssen wir Wohldefiniertheit, dass heißt gleichmäßige Konvergenz, der obigen affinen Zerlegung \cref{eq:ps:pf:omega_affine_zerlegung} von $\omega$ aus \cref{def:ps:pf:omega_affin} fordern.
Dies wird durch folgende Bedingung sichergestellt.
%%
\begin{Annahme}
    Das Funktionensystem $\Set{ \varphi_{j} }_{j \in \mathbb{N}} \in L_{\infty}(\Omega)$ sei einfach summierbar in der $L_{\infty}$-Norm, das heißt es gelte
    \begin{equation}
        \Set{ \norm{\varphi_{j}}_{L_{\infty}(\Omega) } }_{j \in \mathbb{N}} \in \ell_{1}(\mathbb{N}).
    \end{equation}
\end{Annahme}

Hieraus folgt wegen $\mathcal S = [-1, 1]^{\mathbb{N}}$ insbesondere
\begin{equation}
    \sup_{\sigma \in \mathcal S} \norm{\omega(\sigma)}_{L_{\infty}(\Omega)} \leq \sum_{j = 1}^{\infty} \norm{\varphi_{j}}_{L_{\infty}(\Omega)} < \infty.
\end{equation}

Mit dieser Vorarbeit können wir die schwache Formulierung aus \cref{def:ps:rzvp:schwache_formulierung} nun zu der folgenden parametrischen schwachen Formulierung ausweiten.

\mfix{Die Zeitabhängigkeit des Operators $A$ und der Bilinearform $a$ muss noch eingebaut werden!}

\begin{Definition}[Parametrische schwache Formulierung]
\label{def:ps:pf:schwache_formulierung}
    Als \emph{parametrisches schwache Formulierung}, oder auch \emph{parametrische Raum"=Zeit"=Va"-ri"-a"-ti"-ons"-for"-mu"-lie"-rung} von \cref{eq:ps:pde:starke_formulierung} bezeichnen wir das folgende Variationsproblem:
    Gegeben einen Quellterm $g \in L_{2}(I; H^{-1}(\Omega))$ und eine Anfangsbedingung $u_{0} \in L_{2}(\Omega)$.
    Finde für alle $\sigma \in \mathcal S$ ein $u(\sigma) \in \mathcal X$ mit
    \begin{equation}
        \label{eq:ps:pf:schwache_formulierung}
        b(u(\sigma), v; \sigma) = f(v) \quad \text{für alle}~v = (v_{1}, v_{2}) \in \mathcal Y,
    \end{equation}
    wobei $b(\blank, \blank; \sigma) \colon \mathcal X \times \mathcal Y \to \mathbb{R}$ die durch
    \begin{equation}
        \label{eq:ps:pf:schwache_formulierung_lhs_b}
        b(u, v; \sigma)
            = \int_{I} \skprod{u_{t}(t)}{v_{1}(t)}_{L_{2}(\Omega)} + a(u(t), v_{1}(t); t, \sigma) \diff t + \skprod{u(0)}{v_{2}}_{L_{2}(\Omega)}
    \end{equation}
    gegebene Bilinearform und $f(\blank) \colon \mathcal Y \to \mathbb{R}$ definiert ist durch
    \begin{equation}
        \label{eq:ps:pf:schwache_formulierung_rhs_f}
        f(v) = \int_{I} \skprod{g(t)}{v_{1}(t)}_{L_{2}(\Omega)} \diff t + \skprod{u_{0}}{v_{2}}_{L_{2}(\Omega)}.
    \end{equation}
\end{Definition}

% section parametrische_formulierung (end)


\section{Regularität bezüglich der Parameter} % (fold)
\label{sec:ps:rg:regularitaet_bezueglich_der_parameter}

In diesem Abschnitt wollen wir nun nachweisen, dass die im vorherigen Abschnitt hergeleitete parametrische schwache Formulierung eine Regularität bezüglich des Parameters aufweist, konkret werden wir nachweisen, dass die Lösung $u(\sigma)$ analytisch im Parameter $\sigma$ ist.
Dabei orientieren wir uns an den Arbeiten von \textcite{Cohen:2010kz,Kunoth:2013ef} und weisen diese Eigenschaften zunächst für den stationären Fall nach, das heißt, wir verlieren die Zeitabhängigkeit und betrachten stattdessen die parametrische Operatorgleichung
\begin{equation}
\label{eq:ps:rg:operatorgleichung_parametrisch}
    A(\omega) \eta(\omega) = g \quad \text{in}~H^{-1}(\Omega)
\end{equation}
mit dem parametrischen Operator $A(\omega)$ aus Gleichung \cref{eq:ps:pf:operator_A}, beziehungsweise die zugehörige Variationsformulierung
\begin{equation}
\label{eq:ps:rg:variationsformulierung_parametrisch}
    a(\eta, \zeta; \omega) = \skp{g}{\zeta}{H^{-1}(\Omega) \times H^{1}_{0}(\Omega)}
\end{equation}
mit der parametrischen Bilinearform $a(\blank, \blank; \omega)$ aus Gleichung \cref{eq:ps:pf:bilinearform_a}.

Wir beginnen damit, die Existenz der partiellen Ableitung $\partial^{\nu}_{\sigma} \eta$ in jedem Punkt $\sigma \in \mathcal S$ nachzuweisen.

Zunächst einige notationelle Vorbemerkungen.
\begin{Bemerkung}
    Bezeichne mit $\mathfrak F = \Set{ \nu \in \mathbb{N}^{\mathbb{N}}_{0} \given \norm{\nu}_{\ell_{1}(\mathbb{N})} < \infty }$ die Menge aller Folgen nichtnegativer ganzer Zahlen, wobei
    \begin{equation}
        \norm{\nu}_{\ell_{1}(\mathbb{N})} = \sum_{k = 1}^{\infty} \abs{\nu_{k}}
    \end{equation}
    die $\ell_{1}(\mathbb{N})$-Norm sei.
    Damit ergibt sich, dass $\mathfrak F$ gerade diejenigen Folgen enthält, die nur endlich viele Einträgen ungleich Null enthalten.

    Sei $\nu \in \mathfrak F$ und $b \in \ell_{p}(\mathbb{N})$, $p > 0$, dann schreiben wir
    \begin{equation}
        b^{\nu} = \prod_{j = 1}^{\infty} b_{j}^{\nu_{j}}
    \end{equation}
    mit der Konvention $0^{0} = 1$.
    Wegen $\norm{\nu}_{\ell_{1}(\mathbb{N})} < \infty$ ist dieses Produkt stets endlich.
\end{Bemerkung}

Bevor wir uns an die partiellen Ableitungen wagen, beweisen wir eine Stabilitätsaussage für die obige parametrische Operatorgleichung \cref{eq:ps:rg:operatorgleichung_parametrisch}
Um diese nachzuweisen, benötigen wir zunächst, dass die obige Operatorgleichung sachgemäß gestellt ist.

\begin{Satz}
\label{satz:ps:rg:lax_milgram_anwendung}
    Seien $\omega \in L_{\infty}(\Omega)$, $\mu \geq \norm{\omega}_{L_{\infty}(\Omega)}$, $g \in H^{-1}(\Omega)$ und weiter $A(\omega)$ wie in \cref{eq:ps:pf:operator_A}, dann besitzt die Operatorgleichung
    \begin{equation}
        \tag*{\cref{eq:ps:rg:operatorgleichung_parametrisch}}
        A(\omega) \eta(\omega) = g \quad \text{in}~H^{-1}(\Omega)
    \end{equation}
    eine eindeutige Lösung $\eta(\omega) \in H^{1}_{0}(\Omega)$ und diese erfüllt
    \begin{equation}
        \norm{\eta(\omega)}_{H^{1}(\Omega)} \leq \frac{\norm{g}_{H^{-1}(\Omega)}}{\alpha}
    \end{equation}
    mit $\alpha$ aus \cref{satz:ps:rzvp:bilinearform_a_eigenschaften}.

    \begin{Beweis}
        Folgt direkt aus dem Lemma vom Lax-Milgram, \cref{lem:gl:le:lax_milgram}.
    \end{Beweis}
\end{Satz}

\begin{Lemma}
\label{lem:ps:rg:norm_abschaetzung}
    Seien $\omega_{1}, \omega_{2} \in L_{\infty}(\Omega)$ und $\eta_{1}, \eta_{2}$ die zugehörigen Lösungen von \cref{eq:ps:rg:operatorgleichung_parametrisch}, dann gilt
    \begin{equation}
        \norm{\eta_{1} - \eta_{2}}_{H^{1}(\Omega)} \leq \frac{\norm{g}_{H^{-1}(\Omega)}}{\alpha^{2}} \norm{\omega_{1} - \omega_{2}}_{L_{\infty}(\Omega)}.
    \end{equation}

    \begin{Beweis}
        Durch Subtraktion der Variationsformulierungen \cref{eq:ps:rg:variationsformulierung_parametrisch} für $\eta_{1}$ und $\eta_{2}$ erhalten wir für alle $\zeta \in H^{1}_{0}(\Omega)$ die Gleichung
        \begin{align}
            0 &= a(\eta_{1}, \zeta; \omega_{1}) - a(\eta_{2}, \zeta; \omega_{2})
            \\&= c \skp{\grad \eta_{1} - \grad \eta_{2}}{\grad \zeta}{L_{2}(\Omega)} + \skp{\omega_{1}\eta_{1} - \omega_{2} \eta_{2}}{\zeta}{L_{2}(\Omega)} + \mu \skp{\eta_{1} - \eta_{2}}{\zeta}{L_{2}(\Omega)},
            \intertext{durch setzen von $\theta = \eta_{1} - \eta_{2}$ erhalten wir weiter}
            0 &= c \skp{\grad \theta}{\grad \zeta}{L_{2}(\Omega)} + \skp{\omega_{1} \theta}{\zeta}{L_{2}(\Omega)} + \mu \skp{\theta}{\zeta}{L_{2}(\Omega)} + \skp{(\omega_{1} - \omega_{2}) \eta_{2}}{\zeta}{L_{2}(\Omega)}
            \\&= a(\theta, \zeta; \omega_{1}) + \skp{(\omega_{1} - \omega_{2}) \eta_{2}}{\zeta}{L_{2}(\Omega)}.
        \end{align}
        Dies lässt sich nun in der Form
        \begin{equation}
            a(\theta, \zeta; \omega_{1}) = - \skp{(\omega_{1} - \omega_{2}) \eta_{2}}{\zeta}{L_{2}(\Omega)} =: h(\zeta)
        \end{equation}
        wieder als Variationsproblems \cref{eq:ps:rg:variationsformulierung_parametrisch} interpretieren.
        Da es sich bei $h$ um ein stetiges lineares Funktional auf $H^{1}_{0}(\Omega)$ handelt, liefert nun \cref{satz:ps:rg:lax_milgram_anwendung}, dass $\theta = \eta_{1} - \eta_{2}$ die eindeutige Lösung dieses Variationsproblems ist und weiterhin die Ungleichung
        \begin{equation}
            \norm{\theta}_{H^{1}(\Omega)} \leq \frac{\norm{h}_{H^{-1}(\Omega)}}{\alpha}
        \end{equation}
        erfüllt.
        Die Operatornorm von $h$ lässt sich mittels der Cauchy-Schwarz-Ungleichung und \cref{satz:ps:rg:lax_milgram_anwendung} bestimmen zu
        \begin{equation}
            \begin{aligned}
                \norm{h}_{H^{-1}(\Omega)}
                  &=    \sup_{\norm{\zeta}_{H^{1}(\Omega)} = 1} \abs{h(\zeta)}
                \\&=    \sup_{\norm{\zeta}_{H^{1}(\Omega)} = 1} \abs{\skp{(\omega_{1} - \omega_{2}) \eta_{2}}{\zeta}{L_{2}(\Omega)}}
                \\&\leq \sup_{\norm{\zeta}_{H^{1}(\Omega)} = 1} \norm{\omega_{1} - \omega_{2}}_{L_{\infty}(\Omega)} \norm{\eta_{1}}_{L_{2}(\Omega)} \norm{\zeta}_{L_{2}(\Omega)}
                \\&\leq \sup_{\norm{\zeta}_{H^{1}(\Omega)} = 1} \norm{\omega_{1} - \omega_{2}}_{L_{\infty}(\Omega)} \norm{\eta_{1}}_{H^{1}(\Omega)} \norm{\zeta}_{H^{1}(\Omega)}
                \\&=    \norm{\omega_{1} - \omega_{2}}_{L_{\infty}(\Omega)} \norm{\eta_{1}}_{H^{1}(\Omega)}
                \\&\leq \norm{\omega_{1} - \omega_{2}}_{L_{\infty}(\Omega)} \frac{\norm{g}_{H^{-1}(\Omega)}}{\alpha}.
            \end{aligned}
        \end{equation}
        Zusammen liefert dies die Ungleichung
        \begin{equation}
            \norm{\eta_{1} - \eta_{2}}_{H^{1}(\Omega)}
            = \norm{\theta}_{H^{1}(\Omega)} \leq \frac{\norm{g}_{H^{-1}(\Omega)}}{\alpha^{2}} \norm{\omega_{1} - \omega_{2}}_{L_{\infty}(\Omega)}
        \end{equation}
        und damit die Behauptung.
    \end{Beweis}
\end{Lemma}

\begin{Satz}
\label{satz:ps:rg:existenz_partieller_ableitungen}
    Die Abbildung $\mathcal S \ni \sigma \mapsto \eta(\sigma) \in H^{1}_{0}(\Omega)$, welche einem Parameter $\sigma$ die zugehörige Lösung $\eta(\sigma)$ des Variationsproblems \cref{eq:ps:rg:variationsformulierung_parametrisch} zuordnet, besitzt für alle $\nu \in \mathfrak F$ eine partielle Ableitung $\partial^{\nu}_{\sigma} \eta(\sigma)$.

    \begin{Beweis}
        Wir beschränken uns darauf die Behauptung exemplarisch für die partiellen Ableitungen erster Ordnung für ein festes $\sigma \in \mathcal S$ nachzuweisen.
        Seien dazu $\nu = e_{j}$ für ein $j \in \mathbb{N}$ und sei weiter $h \in \mathbb{R} \setminus \Set{ 0 }$.
        Definiere $\sigma_{h} = \sigma + h e_{j}$ und folglich $\sigma_{0} = \sigma$.
        Weiter setzen wir
        \begin{equation}
            \theta_{h} = \frac{\eta(\sigma_{h}) - \eta(\sigma)}{h},
        \end{equation}
        wobei $\eta(\blank)$ die Lösungen des Variationsproblems zu den entsprechenden Parametern seien.
        Ist $\abs{h}$ klein genug, so dass $\sigma_{h} = \sigma + h e_{j} \in \mathcal S$ gilt, dann existieren die eindeutigen $\eta(\blank)$ nach \cref{satz:ps:rg:lax_milgram_anwendung}, das heißt, der obige Ausdruck für $\theta_{h}$ ist für diese $h$ wohldefiniert.

        Betrachte unter diesen Gegebenheiten nun die Differenz der zu $\eta(\sigma_{h})$ und $\eta(\sigma)$ zugehörigen Variationsprobleme \cref{eq:ps:rg:variationsformulierung_parametrisch}, dann gilt für alle $\zeta \in H^{1}_{0}(\Omega)$ die Gleichung
        \begin{align}
            0 &= a(\eta(\sigma_{h}), \zeta; \sigma_{h}) - a(\eta(\sigma), \zeta; \sigma)
            %
            \\ &= c \skp{\grad \eta(\sigma_{h}) - \grad \eta(\sigma)}{\grad \zeta}{L_{2}(\Omega)}
                    + \skp{\omega(\sigma_{h})\eta(\sigma_{h}) - \omega(\sigma) \eta(\sigma)}{\zeta}{L_{2}(\Omega)}
                    \\&\qquad+ \mu \skp{\eta(\sigma_{h}) - \eta(\sigma)}{\zeta}{L_{2}(\Omega)}
            \\ &= h a(\theta_{h}, \zeta; \sigma) + \skp{(\omega(\sigma_{h}) - \omega(\sigma))\eta(\sigma_{h})}{\zeta}{L_{2}(\Omega)}
        \end{align}
        Erneut können wir diese Gleichung in die Form des Variationsproblems \cref{eq:ps:rg:variationsformulierung_parametrisch} bringen, konkret also
        \begin{equation}
            a(\theta_{h}, \zeta; \sigma) = F_{h}(\zeta) \quad \fa \zeta \in H^{1}_{0}(\Omega).
        \end{equation}
        Dabei lässt sich das stetige lineare Funktional $F_{h} \in H^{-1}(\Omega)$ wegen der affinen Darstellung \cref{eq:ps:pf:omega_affine_zerlegung} von $\omega(\blank)$ schreiben als
        \begin{equation}
            F_{h} \colon H^{1}_{0}(\Omega) \to \mathbb{R},
            \quad \zeta \mapsto - h^{-1} \skp{(\omega(\sigma_{h}) - \omega(\sigma))\eta(\sigma_{h})}{\zeta}{L_{2}(\Omega)}
            = - \skp{\varphi_{j} \eta(\sigma_{h})}{\zeta}{L_{2}(\Omega)}.
        \end{equation}
        Weiter ist $F_{h}(\blank)$ stetig in $h = 0$, denn für festes $\zeta \in H^{1}_{0}(\Omega)$ gilt unter Verwendung der Cauchy-Schwarz-Ungleichung die Abschätzung
        \begin{align}
            \abs{F_{h}(\zeta) - F_{0}(\zeta)}
            &= \abs{\skp{\varphi_{j} (\eta(\sigma_{h}) - \eta(\sigma))}{\zeta}{L_{2}(\Omega)}}
            \\&\leq \norm{\varphi_{j}}_{L_{\infty}(\Omega)} \abs{\skp{\eta(\sigma_{h}) - \eta(\sigma)}{\zeta}{L_{2}(\Omega)}}
            \\&\leq \norm{\varphi_{j}}_{L_{\infty}(\Omega)} \norm{\eta(\sigma_{h}) - \eta(\sigma)}_{L_{2}(\Omega)} \norm{\zeta}_{L_{2}(\Omega)}
            \\&\leq \norm{\varphi_{j}}_{L_{\infty}(\Omega)} \norm{\eta(\sigma_{h}) - \eta(\sigma)}_{H^{1}(\Omega)} \norm{\zeta}_{H^{1}(\Omega)}.
        \end{align}
        Weiter können wir die Stabilitätsaussage aus \cref{lem:ps:rg:norm_abschaetzung} für die hier auftretenden Parameter vereinfachen zu
        \begin{equation}
            \norm{\eta(\sigma_{h}) - \eta(\sigma)}_{H^{1}(\Omega)}
            \leq \frac{\norm{g}_{H^{-1}(\Omega)}}{\alpha^{2}} \norm{\omega(\sigma_{h}) - \omega(\sigma)}_{L_{\infty}(\Omega)}
            = \abs{h} \norm{\varphi_{j}}_{L_{\infty}(\Omega)} \frac{\norm{g}_{H^{-1}(\Omega)}}{\alpha^{2}}.
        \end{equation}
        Zusammen liefern die beiden obigen Abschätzungen
        \begin{equation}
            \abs{F_{h}(\zeta) - F_{0}(\zeta)} \leq \abs{h} \norm{\varphi_{j}}^{2}_{L_{\infty}(\Omega)} \norm{\zeta}_{H^{1}(\Omega)} \frac{\norm{g}_{H^{-1}(\Omega)}}{\gamma_{0}^{2}} \to 0 \quad \text{für}~h \to 0,
        \end{equation}
        das heißt, es gilt $F_{h} \to F_{0}$ in $H^{-1}(\Omega)$ für $h \to 0$.
        \mfix{Genauer ausführen, warum. (vergleiche $\eta(\sigma_{h}) = A(\sigma_{h})^{-1} F_{h}$)}
        Dies impliziert insbesondere $\theta_{h} \to \theta_{0}$ in $H^{1}_{0}(\Omega)$ für $h \to 0$.
        Weiter erfüllt $\theta_{0}$ die Gleichung
        \begin{equation}
            a(\theta_{0}, \zeta; \sigma) = F_{0}(\zeta) \quad \fa \zeta \in H^{1}_{0}(\Omega).
        \end{equation}
        Damit existiert $\partial_{\sigma_{j}} \eta(\sigma) = \theta_{0}$ in $H^{1}_{0}(\Omega)$ und ist die eindeutige Lösung des Variationsproblems
        \begin{equation}
        \label{eq:ps:rg:variationsproblem_partielle_ableitung}
            a(\partial_{\sigma_{j}} \eta(\sigma), \zeta; \sigma) = - \skp{\varphi_{j} \eta(\sigma)}{\zeta}{L_{2}(\Omega)} \quad \fa \zeta \in H^{1}_{0}(\Omega).
        \end{equation}

        Analog kann man auch die Existenz der partiellen Ableitungen höherer Ordnung nachweisen, indem man die in diesem Beweis verwendeten Schritte nun auf das neue Variationsproblem \cref{eq:ps:rg:variationsproblem_partielle_ableitung} anwendet.
    \end{Beweis}
\end{Satz}

\begin{Bemerkung}
\label{bem:ps:rg:partielle_ableitungen_alternativ_ueber_ableitung_des_operators}
    Alternativ erhält man das Variationsproblem \cref{eq:ps:rg:variationsproblem_partielle_ableitung} auch durch formales Differenzieren der Variationsformulierung \cref{eq:ps:rg:variationsformulierung_parametrisch} nach $\sigma_{j}$.
    Denn es gilt
    \begin{align}
        \partial_{\sigma_{j}} a(\eta(\sigma), \zeta; \sigma)
        &= \partial_{\sigma_{j}} \left( c \skp{\grad \eta(\sigma)}{\grad \zeta}{L_{2}(\Omega)} + \skp{\omega(\sigma)\eta(\sigma)}{\zeta}{L_{2}(\Omega)} + \mu \skp{\eta(\sigma)}{\zeta}{L_{2}(\Omega)} \right)
        \\&= c \skp{\grad \partial_{\sigma_{j}} \eta(\sigma)}{\grad \zeta}{L_{2}(\Omega)}
                + \skp{\partial_{\sigma_{j}} \omega(\sigma) \eta(\sigma) + \omega(\sigma) \partial_{\sigma_{j}} \eta(\sigma)}{\zeta}{L_{2}(\Omega)}
            \\&\qquad + \mu \skp{\partial_{\sigma_{j}} \eta(\sigma)}{\zeta}{L_{2}(\Omega)}
        \\&= a(\partial_{\sigma_{j}} \eta(\sigma), \zeta; \sigma) + \skp{\partial_{\sigma_{j}} \omega(\sigma) \eta(\sigma)}{\zeta}{L_{2}(\Omega)}
        \\&= a(\partial_{\sigma_{j}} \eta(\sigma), \zeta; \sigma) + \skp{\varphi_{j} \eta(\sigma)}{\zeta}{L_{2}(\Omega)}
    \end{align}
    und
    \begin{equation}
        \partial_{\sigma_{j}} \skp{g}{\zeta}{H^{-1}(\Omega) \times H^{1}_{0}(\Omega)} = 0,
    \end{equation}
    woraus man insgesamt erneut das Variationsproblem \cref{eq:ps:rg:variationsproblem_partielle_ableitung},
    \begin{equation}
        a(\partial_{\sigma_{j}} \eta(\sigma), \zeta; \sigma) = - \skp{\varphi_{j} \eta(\sigma)}{\zeta}{L_{2}(\Omega)} \quad \fa \zeta \in H^{1}_{0}(\Omega),
    \end{equation}
    erhält.
\end{Bemerkung}

Nach der Existenz der partiellen Ableitungen beliebiger Ordnung weisen wir nun weiter nach, dass diese jeweils gleichmäßig in $\sigma \in \mathcal S$ beschränkt sind.

\begin{Satz}
\label{satz:ps:rg:partielle_ableitungen_schranke}
    Sei $b := (b_{j})_{j \in \mathbb{N}} \in \mathbb{R}^{\mathbb{N}}$ mit $b_{j} := \alpha^{-1} \norm{\varphi_{j}}_{L_{\infty}(\Omega)}$, dann gilt
    \begin{equation}
    \label{eq:ps:rg:partielle_ableitungen_schranke}
        \sup_{\sigma \in \mathcal S} \norm{\partial^{\nu}_{\sigma} \eta(\sigma)} \leq \frac{\norm{g}_{H^{-1}(\Omega)}}{\alpha} \norm{\nu}_{\ell_{1}(\mathbb{N})}! b^{\nu}.
    \end{equation}

    \begin{Beweis}
        Betrachte die Variationsprobleme, welche von den partiellen Ableitungen $\partial^{\nu}_{\sigma} \eta(\sigma)$ erfüllt werden.
        Wir zeigen zunächst, dass diese durch
        \begin{equation}
        \label{eq:ps:rg:partielle_ableitungen_rekursive_ableitungen}
            a(\partial^{\nu}_{\sigma} \eta(\sigma), \zeta; \sigma)
            = - \sum_{\Set{j \given \nu_{j} \neq 0}} \nu_{j} \skp{\varphi_{j} \partial^{\nu - e_{j}}_{\sigma} \eta(\sigma)}{\zeta}{L_{2}(\Omega)}.
        \end{equation}
        rekursiv dargestellt werden können.
        Die Gültigkeit dieser Darstellung zeigen wir induktiv.

        Den Fall $\norm{\nu}_{\ell_{1}(\mathbb{N})} = 1$ haben wir in \cref{bem:ps:rg:partielle_ableitungen_alternativ_ueber_ableitung_des_operators} bereits gezeigt.
        Betrachte also $\norm{\nu}_{\ell_{1}(\mathbb{N})} > 1$.
        Sei $k \in \mathbb{N}$ ein Index mit $\nu_{k} > 0$, dann definieren wir $\tilde{\nu} := \nu - e_{k}$ und es gilt offenbar $\norm{\tilde\nu}_{\ell_{1}(\mathbb{N})} = \norm{\nu}_{\ell_{1}(\mathbb{N})} - 1$.
        Nach Induktionsvoraussetzung gilt damit
        \begin{equation}
            a(\partial^{\tilde{\nu}}_{\sigma} \eta(\sigma), \zeta; \sigma) + \sum_{\Set{j \given \tilde{\nu}_{j} \neq 0}} \tilde{\nu}_{j} \skp{\varphi_{j} \partial^{\tilde{\nu} - e_{j}}_{\sigma} \eta(\sigma)}{\zeta}{L_{2}(\Omega)} = 0,
        \end{equation}
        wobei nach Definition $\nu_{j} = \tilde{\nu}_{j}$ für $j \neq k$ und $\tilde{\nu}_{k} = \nu_{k} - 1$ ist.
        Partielles Differenzieren dieser Gleichung nach $\sigma_{k}$ analog zu \cref{bem:ps:rg:partielle_ableitungen_alternativ_ueber_ableitung_des_operators} liefert dann die Gleichung
        \begin{align}
            0 &=
                a(\partial^{\nu}_{\sigma} \eta(\sigma), \zeta; \sigma)
                + \skp{\varphi_{k} \partial^{\nu - e_{k}}_{\sigma} \eta(\sigma)}{\zeta}{L_{2}(\Omega)}
                + (\nu_{k} - 1) \skp{\varphi_{k} \partial^{\nu - e_{k}}_{\sigma} \eta(\sigma) }{\zeta}{L_{2}(\Omega)}
           \\&\qquad     + \sum_{\Set{j \neq k \given \nu_{j} \neq 0}} \nu_{j} \skp{\varphi_{j} \partial^{\nu - e_{j}}_{\sigma} \eta(\sigma)}{\zeta}{L_{2}(\Omega)},
        \end{align}
        welche nach Zusammenfassen der Gleichung \cref{eq:ps:rg:partielle_ableitungen_rekursive_ableitungen} entspricht.

        Wählt man nun $v = \partial^{\nu}_{\sigma} \eta(\sigma)$, dann gilt einerseits aufgrund der Koerzivität von $a(\blank, \blank; \sigma)$ die Ungleichung
        \begin{equation}
            a(\partial^{\nu}_{\sigma} \eta(\sigma), \partial^{\nu}_{\sigma} \eta(\sigma); \sigma) \geq \alpha \norm{\partial^{\nu}_{\sigma} \eta(\sigma)}_{H^{1}(\Omega)}^{2}.
        \end{equation}
        Andererseits erhalten wir aus der rekursiven Darstellung \cref{eq:ps:rg:partielle_ableitungen_rekursive_ableitungen} mit Hilfe er Cauchy-Schwarz-Ungleichung die Abschätzung
        \begin{align}
            a(\partial^{\nu}_{\sigma} \eta(\sigma), \partial^{\nu}_{\sigma} \eta(\sigma); \sigma)
            &= - \sum_{\Set{j \given \nu_{j} \neq 0}} \nu_{j} \skp{\varphi_{j} \partial^{\nu - e_{j}}_{\sigma} \eta(\sigma) }{\partial^{\nu}_{\sigma} \eta(\sigma)}{L_{2}(\Omega)}
            \\&\leq \sum_{\Set{j \given \nu_{j} \neq 0}} \nu_{j} \norm{\varphi_{j}}_{L_{\infty}(\Omega)} \norm{\partial^{\nu - e_{j}}_{\sigma} \eta(\sigma)}_{H^{1}(\Omega)} \norm{\partial^{\nu}_{\sigma} \eta(\sigma)}_{H^{1}(\Omega)}.
        \end{align}
        Beide Ungleichungen zusammen ergeben
        \begin{equation}
        \label{eq:ps:rg:partielle_ableitungen_schranke_rekursiv}
            \norm{\partial^{\nu}_{\sigma} \eta(\sigma)}_{H^{1}(\Omega)} \leq \sum_{\Set{j \given \nu_{j} \neq 0}} \nu_{j} \frac{\norm{\varphi_{j}}_{L_{\infty}(\Omega)}}{\alpha} \norm{\partial^{\nu - e_{j}}_{\sigma} \eta(\sigma)}_{H^{1}(\Omega)}.
        \end{equation}

        Um nun die eigentliche Behauptung zu beweisen, verfolgen wir erneut einen Induktionsansatz.
        Sei zunächst $\norm{\nu}_{\ell_{1}(\mathbb{N})} = 0$, dann entspricht
        \begin{equation}
            \norm{\eta(\sigma)}_{H^{1}(\Omega)} \leq \frac{\norm{g}_{H^{-1}(\Omega)}}{\alpha},
        \end{equation}
        der Ungleichung \cref{eq:ps:rg:partielle_ableitungen_schranke} und ist nach \cref{satz:ps:rg:lax_milgram_anwendung} erfüllt.
        Sei also weiter $\norm{\nu}_{\ell_{1}(\mathbb{N})} > 0$, dann gilt für die rekursive Darstellung \cref{eq:ps:rg:partielle_ableitungen_schranke_rekursiv} unter Verwendung der Induktionsvoraussetzung für $\norm{\partial^{\nu - e_{j}}_{\sigma} \eta(\sigma)}_{H^{1}(\Omega)}$ die Abschätzung
        \begin{align}
            \norm{\partial^{\nu}_{\sigma} \eta(\sigma)}_{H^{1}(\Omega)}
            &\leq
            \sum_{\Set{j \given \nu_{j} \neq 0}} \nu_{j} \frac{\norm{\varphi_{j}}_{L_{\infty}(\Omega)}}{\alpha} \norm{\partial^{\nu - e_{j}}_{\sigma} \eta(\sigma)}_{H^{1}(\Omega)}
            \\&\leq
            \sum_{\Set{j \given \nu_{j} \neq 0}} \nu_{j} \frac{\norm{\varphi_{j}}_{L_{\infty}(\Omega)}}{\alpha} \frac{\norm{g}_{H^{-1}(\Omega)}}{\alpha} \norm{\nu - e_{j}}_{\ell_{1}(\mathbb{N})}! b^{\nu - e_{j}}
            \\&=
            \Bigg( \sum_{\Set{j \given \nu_{j} \neq 0}} \nu_{j} \Bigg) \Bigg( \frac{\norm{g}_{H^{-1}(\Omega)}}{\alpha} (\norm{\nu}_{\ell_{1}(\mathbb{N})} - 1)! b^{\nu} \Bigg)
            \\&=
            \frac{\norm{g}_{H^{-1}(\Omega)}}{\alpha} \norm{\nu}_{\ell_{1}(\mathbb{N})}! b^{\nu}
         \end{align}
         und damit die Behauptung.
    \end{Beweis}
\end{Satz}

Zusammenfassend erhalten wir damit die folgende Aussage.
\mfix{Sauber formulieren.}

\begin{Satz}
\label{satz:ps:rg:operatorgleichung_zusammenfassung}
    Sei $\mu \geq \sum_{j = 1}^{\infty} \norm{\varphi_{j}}_{L_{\infty}(\Omega)}$ und $A(\sigma)$ wie in \cref{eq:ps:pf:operator_A_sigma}.
    Dann existiert für jedes $g \in H^{-1}(\Omega)$ und jedes $\sigma \in \mathcal S$ eine eindeutige Lösung $\eta(\sigma)$ der parametrischen Operatorgleichung
    \begin{equation}
        A(\sigma) \eta(\sigma) = g \quad \text{in}~H^{-1}(\Omega).
    \end{equation}
    Weiter hängt diese Lösung $\eta(\sigma)$ analytisch vom Parameter $\sigma$ ab und es gilt die Abschätzung
    \begin{equation}
        \sup_{\sigma \in \mathcal S} \norm{\partial^{\nu}_{\sigma} \eta(\sigma)}_{H^{1}_{0}(\Omega)} \leq \frac{\norm{g}_{H^{-1}(\Omega)}}{\alpha} \norm{\nu}_{\ell_{1}(\mathbb{N})}! b^{\nu},
    \end{equation}
    mit $b = (b_{j})_{j \in \mathbb{N}} \in \ell_{1}(\mathbb{N})$ und $b_{j} = \alpha^{-1} \norm{\varphi_{j}}_{L_{\infty}(\Omega)}$.

    \begin{Beweis}
        Existenz und Eindeutigkeit folgen analog zu \cref{satz:ps:rg:lax_milgram_anwendung} aus dem Lemma von Lax-Milgram, \cref{lem:gl:le:lax_milgram}.
        Die Abschätzung wurde bereits in \cref{satz:ps:rg:partielle_ableitungen_schranke} gezeigt.
        \mdo{Daraus folgern, dass die Lösung analytisch im Parameter ist.}
    \end{Beweis}
\end{Satz}

\mdo{Daraus folgern, dass es für den parabolischen Fall auch gilt.}

\Cref{satz:ps:rg:operatorgleichung_zusammenfassung} wurde bei \textcite[Theorem 4]{Kunoth:2013ef}, unter Verweisung auf die Beweise von \cref{satz:ps:rg:existenz_partieller_ableitungen} und \cref{satz:ps:rg:partielle_ableitungen_schranke} bei \textcite[Theorem 4.2, 4.3]{Cohen:2010kz}, für allgemeinere parametrische Operatoren respektive Bilinearformen nachgewiesen.
Dafür wurde im Wesentlichen gefordert, dass die partiellen Ableitungen der Operatoren beziehungsweise Bilinearformen wohldefiniert sind.
Dies ist bei uns beispielsweise der Fall, wie in den Beweisen von \cref{satz:ps:rg:existenz_partieller_ableitungen} und \cref{satz:ps:rg:partielle_ableitungen_schranke} und an \cref{bem:ps:rg:partielle_ableitungen_alternativ_ueber_ableitung_des_operators} zu sehen ist.

Aufbauend auf das Ergebnis in \cref{satz:ps:rg:operatorgleichung_zusammenfassung} können wir nun, analog zu \textcite[Section 4]{Kunoth:2013ef}, ein ähnliches Ergebnis für das parabolische Variationsproblem nachweisen.
Dies wollen wir anhand der folgenden, aus \cite{Kunoth:2013ef} entnommenen, Punkte motivieren.

\begin{Annahme}[{{\cite[Assumption 1]{Kunoth:2013ef}}}]
\label{ann:ps:rg:kunoth13_assumption1}
    Seien $X$ und $Y$ zwei reflexive Banachräume.
    Die parametrische Familie von Operatoren
    $\Set{ A(\sigma) \in \mathcal L(X, Y') \given \sigma \in \mathcal S }$ sei eine $\mathfrak p$-reguläre Operatorfamilie für ein $0 < \mathfrak p \leq 1$, das heißt,
    \begin{thmenumerate}
        \item $A(\sigma) \in \mathcal L(X, Y')$ sei stetig invertierbar für alle $\sigma \in \mathcal S$ mit gleichmäßig beschränktem Inversen $A{(\sigma)}^{-1} \in \mathcal L(Y', X)$, das heißt, es existiert ein $C_{0} > 0$ mit
        \begin{equation}
            \sup_{\sigma \in \mathcal S} \norm{A{(\sigma)}^{-1}}_{\mathcal L(Y', X)} \leq C_{0},
        \end{equation}
        \item für jedes feste $\sigma \in \mathcal S$ seien die Operatoren $A(\sigma)$ analytisch bezüglich $\sigma$.
        Konkret existiert eine nichtnegative Folge $b = (b_{j})_{j \in \mathbb{N}} \in \ell_{\mathfrak p}(\mathbb{N})$, so dass
        \begin{equation}
            \sup_{\sigma \in \mathcal S} \norm{(A{(0)})^{-1}(\partial^{\nu}_{\sigma} A(\sigma))}_{\mathcal L(X, X)} \leq C_{0} b^{\nu}
        \end{equation}
        für alle $\nu \in \mathfrak F \setminus \{ 0 \}$ gilt.
    \end{thmenumerate}
\end{Annahme}
%
Diese Annahme alleine reicht bereits aus, um ein Analogon zu \cref{satz:ps:rg:operatorgleichung_zusammenfassung} zu erhalten, vergleiche \cite[Theorem 4]{Kunoth:2013ef}.

Weiter wird die obige Annahme von unserem Variationsproblem für $\mathfrak p = 1$ erfüllt, denn in unserem Falle ist $C_{0} = \alpha^{-1}$ und $b_{j} = \alpha^{-1} \norm{\varphi_{j}}_{L_{\infty}(\Omega)}$ und nach Konstruktion insbesondere $b \in \ell_{1}(\mathbb{N})$.\mfix{referenzieren.}

% \begin{Satz}[{{\cite[Theorem 4]{Kunoth:2013ef}}}]
% \label{satz:ps:rg:kunoth13_theorem4}
%     Die parametrische Familie von Operatoren $\Set{ A(\sigma) \in \mathcal L(X, Y') \given \sigma \in \mathcal S }$ erfülle \cref{ann:ps:rg:kunoth13_assumption1} für ein $0 \leq \mathfrak p \leq 1$.
%     Dann existiert für jedes $g \in Y'$ und jedes $\sigma \in \mathcal S$ eine eindeutige Lösung $\eta(\sigma)$ der parametrischen Operatorgleichung
%     \begin{equation}
%         A(\sigma) \eta(\sigma) = g \quad \text{in}~Y'.
%     \end{equation}
%     Weiter hängt die Lösung $\eta(\sigma)$ analytisch vom Parameter $\sigma$ ab und es gilt
%     \begin{equation}
%     \label{eq:ps:rg:kunoth13_theorem4_abschaetzung}
%         \sup_{\sigma \in \mathcal S} \norm{\partial^{\nu}_{\sigma} \eta(\sigma)}_{X} \leq C_{0} \norm{g}_{Y'} \norm{\nu}_{\ell_{1}(\mathbb{N})} ! \tilde{b}^{\nu},
%     \end{equation}
%     wobei $\tilde{b} \in \ell_{\mathfrak p}(\mathbb{N})}$ durch
%     \begin{equation}
%         \tilde{b}_{j} = \frac{b_{j}}{\ln 2} \quad \fa j \in \mathbb{N}
%     \end{equation}
%     definiert ist.
% \end{Satz}

% Bis auf einen konstanten Faktor bei der Abschätzung \cref{eq:ps:rg:kunoth13_theorem4_abschaetzung} entspricht dies den von uns nachgewiesenen Aussagen.

% Der Grund für die Einführung der obigen Ergebnisse von \textcite{Kunoth:2013ef} ist der folgende:
% Aus den Eigenschaften der Familie von parametrischen Operatoren $\Set{ A(\sigma, t) \in \mathcal L(V, V') \given \sigma \in \mathcal S, t \in [0, T] }$ können wir Rückschlüsse auf die parametrische Regularität der Lösungen des parabolischen Variationsproblems ziehen.

Der Grund für die Einführung von \cref{ann:ps:rg:kunoth13_assumption1} ist der folgende:
Erfüllt die Familie von parametrischen Operatoren $\Set{ A(\sigma, t) \in \mathcal L(V, V') \given \sigma \in \mathcal S, t \in [0, T] }$ \cref{ann:ps:rg:kunoth13_assumption1}, dann lässt sich zeigen, dass dies auch für die parametrische Raum-Zeit-Variationsformulierung aus \cref{def:ps:pf:schwache_formulierung} gilt.
Zusammenfassend ergibt sich damit der folgende Satz, welcher eine auf unser Problem abgewandelte Version von \cite[Theorem 21]{Kunoth:2013ef} darstellt.

\begin{Satz}
\label{satz:ps:rg:kunoth13_theorem21}
    Seien $\mathcal X$ und $\mathcal Y$ gegeben wie in \cref{eq:ps:rzvp:ansatzraum_testraum}.
    Sei weiter für jedes $\sigma \in \mathcal S$ der Operator $B(\sigma) \in \mathcal L(\mathcal X, \mathcal Y')$ durch
    \begin{equation}
        \skp{B(\sigma) u}{v}{\mathcal Y' \times \mathcal Y} = b(u, v; \sigma), \quad u \in \mathcal X,~y \in \mathcal Y
    \end{equation}
    für die Bilinearform $b(\blank, \blank; \sigma)$ aus \cref{eq:ps:pf:schwache_formulierung_lhs_b} gegeben.
    Dann ist $B(\sigma)$ für jedes $\sigma \in \mathcal S$ stetig invertierbar und es existieren Konstanten $0 < \beta_{1} \leq \beta_{2} < \infty$ mit
    \begin{equation}
        \label{eq:ps:rg:theorem21_abschaetzungen_normen_B_und_B_inv_parametrisch}
        \sup_{\sigma \in \mathcal S} \norm{B(\sigma)}_{\mathcal L(\mathcal X, \mathcal Y')} \leq \beta_{2} \quad \text{und} \quad  \sup_{\sigma \in \mathcal S} \norm{B(\sigma)^{-1}}_{\mathcal L(\mathcal Y', \mathcal X)} \leq \frac{1}{\beta_{1}}.
    \end{equation}

    Ferner hängen die Lösungen $u(\sigma)$ des parametrischen Raum-Zeit-Variationsproblems \cref{eq:ps:pf:schwache_formulierung} analytisch vom Parameter $\sigma$ ab und es gilt die Abschätzung
    \begin{equation}
        \label{eq:ps:rg:theorem21_apriori_schranke}
        \sup_{\sigma \in \mathcal S} \norm{(\partial^{\nu}_{\sigma} u)(\sigma)}_{\mathcal X} \leq \frac{\norm{f}_{\mathcal Y'}}{\beta_{1}} \norm{\nu}_{\ell_{1}(\mathbb{N})}! b^{\nu}
    \end{equation}
    für alle $\nu \in \mathfrak F$, wobei $f$ wie in~\cref{eq:ps:pf:schwache_formulierung_rhs_f} und $b = (b_{j})_{j \in \mathbb{N}} \in \ell_{1}(\mathbb{N})$ durch $b_{j} = \beta_{1}^{-1} \norm{\varphi_{j}}_{L_{\infty}(\Omega)}$ gegeben sind.

    \begin{Beweis}
        Wir beginnen mit der stetigen Invertierbarkeit von $B(\sigma)$.
        Es bietet sich an, dies durch Anwendung von \cref{satz:gl:le:ss09_theorem51} nachzuweisen, das bedeutet, wir müssen dementsprechend die Voraussetzungen, welche in \cref{ann:gl:le:bilinearform_eigenschaften} zu finden sind, überprüfen.

        Dies wurde in \cref{satz:ps:rzvp:bilinearform_a_eigenschaften} und \cref{kor:ps:rzvp:bilinearform_elliptisch} bereits getan.
        Da wir weiter vom Fall $\mu \geq \sum_{j = 1}^{\infty} \norm{\varphi_{j}}_{L_{\infty}(\Omega)}$ ausgehen, liefert dies von $\omega$ respektive $\sigma$ unabhängige Konstanten $M_{a} = \max\Set{c, 2 \mu}$ und $\alpha = c \gamma_{\Omega}^{2}$, sowie $\lambda = 0$.
        \Cref{satz:gl:le:ss09_theorem51} liefert damit die stetige Invertierbarkeit von $B(\sigma)$ für jedes $\sigma \in \mathcal S$.
        Weiter erhalten wir damit durch \cref{kor:gl:le:ss09_theorem51_ungleichungen} auch die Konstanten $\beta_{1}$ und $\beta_{2}$, welche durch die obigen $M_{a}$ und $\alpha$ wiederum unabhängig von $\omega$ beziehungsweise $\sigma$ sind.

        Die analytische Abhängigkeit und die zugehörige Abschätzung \cref{eq:ps:rg:theorem21_apriori_schranke} lassen sich nun vollkommen analog zu \cref{lem:ps:rg:norm_abschaetzung}, \cref{satz:ps:rg:existenz_partieller_ableitungen} und \cref{satz:ps:rg:partielle_ableitungen_schranke} nachweisen, weswegen wir dies an dieser Stelle auch nicht weiter ausführen wollen.
    \end{Beweis}
\end{Satz}
