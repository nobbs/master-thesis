%!TEX root = ../main.tex

\subsection{Zu klärende Fragen} % (fold)
\label{sub:zu_kl_rende_fragen}

\begin{enumerate}
    \item Well-posedness von
    \begin{equation}
        u_{t} = \Delta u - \omega u + f, \quad u(0) = u_{0}
    \end{equation}
    für $\omega \in L_{\infty}(\Omega)$, $\Omega$ beschränkt mit Lipschitz-Rand.
    Genauer: \cite[Theorem 5.1]{Schwab:2009ec}  nachweisen für obiges Problem, d.h.
    \begin{itemize}
        \item inf-sup-Bedingungen und so weiter nachrechnen
        \item Konstanten (in Abhängigkeit von $\omega$) bestimmen
    \end{itemize}
    \item Ab hier Bezug auf \cite{Kunoth:2013ef}. Entwicklung des Parameters
    \begin{equation}
        \omega = \sum_{j = 0}^{\infty} \sigma_{j} \phi_{j},
    \end{equation}
    Zu klären:
    \begin{itemize}
        \item Konvergenz? In welchem Raum? Zum Beispiel $Z \hookrightarrow L_{\infty}(\Omega)$, $Z = H^{1}(\Omega)$.
        \item Bedingungen an $\sigma_{j}$? Zum Beispiel $\abs{\sigma_{j}} \leq 1$.
        \item Welche $\phi_{j}$? Möglich mit $\phi_{j} = \sin(j \pi \cdot)$ oder $\sin(j \pi \cdot) / j \pi$ oder $\sin(j \pi \cdot) / (j \pi)^{1 + \epsilon}$? Je nach verwendetem Raum anders normalisiert etc.
        \item Randwerte? Homogene machbar mit obigen $\sin$-Funktionen und $\sigma_{0} = 0$. Periodische mit passender Wahl von $\sigma_{0}$. Sonst?
    \end{itemize}
    \item Wie sieht der Operator $G$ aus und wie sieht eine brauchbare Zerlegung in $G = G_{0} + \sum_{j} \sigma_{j} G_{j}$ aus?
    \item Nachweisen, dass $G_{j}$ beschränkt bzw. die Konvergenz erfüllt.
    \item Ist \cite[Assumption 2]{Kunoth:2013ef} erfüllt?
\end{enumerate}

\subsection{Well-posedness} % (fold)
\label{sub:well_posedness}

Seien $V$, $H$ zwei separable Hilberträume mit der Einbettung $V \hookrightarrow H$.
Daraus ergibt sich der Gelfand-Dreier $V \hookrightarrow H \hookrightarrow V'$.
Es sei weiterhin $a \colon V \times V \to \mathbb{R}$ eine Bilinearform, welche die folgenden Eigenschaften:
\begin{itemize}
    \item Beschränkheit: es existiert ein $M_{a} > 0$ mit
    \begin{equation}
        \abs{a(\eta, \zeta)} \leq M_{a} \norm{\eta}_{V} \norm{\zeta}_{V}, \quad \eta, \zeta \in V.
    \end{equation}
    \item G\aa{}rding-Ungleichung: es existieren $\alpha > 0$, $\lambda \in \mathbb{R}$ mit
    \begin{equation}
        a(\eta, \eta) + \lambda \norm{\eta}^{2}_{H} \geq \alpha \norm{\eta}^{2}_{V}, \quad \eta \in V.
    \end{equation}
\end{itemize}
Wir bezeichnen mit $A \in \mathcal L(V, V')$ den zugehörigen Operator, das heißt es gilt
\begin{equation}
    \skprod{A \eta}{\zeta}_{H} = a(\eta, \zeta).
\end{equation}

Unser konkretes Setting ist folgendermaßen:
Sei $\Omega \subset \mathbb{R}^{n}$ beschränkt mit Lipschitz-Rand.
Wir setzen $V = H^{1}_{0}(\Omega)$ und $H = L_{2}(\Omega)$, das heißt es ist $V' = (H^{1}_{0}(\Omega))' = H^{-1}(\Omega)$.
Der Operator $A$ ist gegeben durch
\begin{equation}
    A \eta = - \sigma \Delta \eta + \omega \eta,
\end{equation}
wobei $\sigma \in \mathbb{R}$, $\omega \in L_{\infty}(\Omega)$.

Die zugehörige Bilinearform $a$ wird damit zu
\begin{equation}
    a(\eta, \zeta) =
\end{equation}

% subsection well_posedness (end)

\clearpage

\subsection{Problemstellung} % (fold)
\label{sub:problemstellung}

Es sei $0 < T < \infty$ und $I = [0, T]$.
Wir bezeichnen mit $\Omega$ eine beschränkte Teilmenge des $\mathbb{R}^{n}$ mit Lipschitz-Rand.

Weiterhin schreiben wir $V = H^{1}_{0}(\Omega)$ und $H = L_{2}(\Omega)$.
Da es sich dabei separable Hilberträume handelt, erhalten wir ein Gelfand-Tripel $H^{1}_{0}(\Omega) \hookrightarrow L_{2}(\Omega) \hookrightarrow H^{-1}(\Omega)$.
Wir verwenden die Notation $\skprod{\cdot}{\cdot}$ sowohl für die inneren Produkte als auch für das \emph{duality Pairing} von $H^{-1}(\Omega)$ und $H^{1}_{0}(\Omega)$.

Seien nun $g \in L_{2}(0, T; H^{-1}(\Omega))$, $u_{0} \in L_{2}(\Omega)$, $\sigma \in \mathbb{R_{+}}$ und $\omega \in L_{\infty}(\Omega)$ gegeben.
Wir wollen nun das parabolische Problem
\begin{equation}
    \begin{cases}
        u_{t}(t) - \sigma \Delta u(t) + \omega u(t) = g(t) & \text{in}~H^{-1}(\Omega)\\
        u(0) = u_{0} & \text{in}~L_{2}(\Omega)
    \end{cases}
\end{equation}
lösen.

\subsubsection{Herleitung einer Space-Time-Variationsformulierung} % (fold)

Wir betrachten zunächst nur den \emph{Raum}-Differentialoperator $A \in \mathcal L(H^{1}_{0}(\Omega), H^{-1}(\Omega))$ der durch
\begin{equation}
    A \eta = - \sigma \Delta \eta + \omega \eta
\end{equation}
gegeben ist und bestimmen die zugehörige Bilinearform $a \colon H^{1}_{0}(\Omega) \times H^{1}_{0}(\Omega) \to \mathbb{R}$ durch Multiplikation mit einer Testfunktion $\zeta \in H^{1}_{0}(\Omega)$ und anschließender Integration.
Unter Verwendung der Greenschen Formel erhalten wir
\begin{equation}
    \label{eq:a_bf}
    \begin{aligned}
        a(\eta, \zeta)
        :&= \skprod{A \eta}{\zeta}_{V' \times V}
        = \int_{\Omega} A \eta \cdot \zeta \diff x
        % = \int_{\Omega} \left( - \sigma \Delta \eta + \omega \eta \right) \zeta \diff x
        \\&= - \sigma \int_{\Omega} \Delta \eta \zeta \diff x + \int_{\Omega} \omega \eta \zeta \diff x
        \\ &= \sigma \int_{\Omega} \skprod{\grad \eta}{\grad \zeta}_{\mathbb{R}^{n}} \diff x + \int_{\Omega} \omega \eta \zeta \diff x
        \\&= \sigma \skprod{\grad \eta}{\grad \zeta}_{L_{2(\Omega, \mathbb{R}^{n})}} + \skprod{\omega \eta}{\zeta}_{L_{2}(\Omega)}.
    \end{aligned}
\end{equation}

Um nun eine Space-Time-Variationsformulierung für unser parabolisches Problem zu erhalten, definieren wir zunächst die benötigten Räume für \emph{Ansatz-} und \emph{Testfunktionen}
\begin{equation}
    \mathcal X = L_{2}(0, T; H^{1}_{0}(\Omega)) \cap H^{1}(0, T; H^{-1}(\Omega))
\end{equation}
und
\begin{equation}
    \mathcal Y = L_{2}(0, T; H^{1}_{0}(\Omega)) \times L_{2}(\Omega).
\end{equation}

Multiplikation mit $v = (v_{1}, v_{2}) \in \mathcal Y$ und Integration über die Zeit $t$ liefert uns nun das Variationsproblem: finde $u \in \mathcal X$ mit
\begin{equation}
     b(u, v) = F(v) \qquad \text{für alle}~v \in \mathcal Y.
\end{equation}
Dabei ist $b \colon \mathcal X \times \mathcal Y \to \mathbb{R}$ eine Bilinearform, gegeben durch
\begin{equation}
    b(u, v) = \int_{0}^{T} \skprod{u_{t}(t)}{v_{1}(t)}_{L_{2}(\Omega)} + a(u(t), v_{1}(t)) \diff t + \skprod{u(0)}{v_{2}}_{L_{2}(\Omega)}
\end{equation}
und $F \colon \mathcal Y \to \mathbb{R}$ das Funktional
\begin{equation}
    F(v) = \int_{0}^{T} \skprod{g(t)}{v_{1}(t)}_{L_{2}(\Omega)} \diff t + \skprod{u_{0}}{v_{2}}_{L_{2}(\Omega)}.
\end{equation}


\subsubsection{Well-posedness} % (fold)
\label{ssub:well_posedness}
Um zu klären, ob und unter welchen Bedingungen das obige Variationsproblem \emph{well-posed} ist, das heißt eine eindeutige Lösung existiert, verweisen wir auf \cite{Schwab:2009ec}.
Es reicht aus, Beschränktheit und eine G\aa{}rding-Ungleichung für die Bilinearform $a(\cdot, \cdot)$ nachzuweisen.

\begin{Lemma}
    \label{lemma:a_bf_bounded_garding}
    Sei $a \colon H^{1}_{0}(\Omega) \times H^{1}_{0}(\Omega) \to \mathbb{R}$ wie in \eqref{eq:a_bf}, also
    \begin{equation}
        a(\eta, \zeta) = \sigma \skprod{\grad \eta}{\grad \zeta}_{L_{2}(\Omega, \mathbb{R}^{n})} + \skprod{\omega \eta}{\zeta}_{L_{2}(\Omega)}
    \end{equation}
    mit $\sigma \in \mathbb{R}$ und $\omega \in L_{\infty}(\Omega)$.
    Dann erfüllt $a(\cdot, \cdot)$ die folgenden Eigenschaften:
    \begin{enumerate}[label=(\arabic*),ref=(\arabic*)]
        \item \emph{Beschränktheit:} es existiert eine Konstante $M_{a} > 0$ mit
        \begin{equation}
            \abs{a(\eta, \zeta)} \leq M_{a} \norm{\eta}_{H^{1}_{0}(\Omega)} \norm{\zeta}_{H^{1}_{0}(\Omega)}
        \end{equation}
        für alle $\eta, \zeta \in H^{1}_{0}(\Omega)$.
        \label{lemma:a_bf_bounded_garding:1}
        \item \emph{G\aa{}rding-Ungleichung:} es existieren Konstanten $\alpha > 0$ und $\lambda \in \mathbb{R}$ mit
        \begin{equation}
                a(\eta, \eta) + \lambda \norm{\eta}_{L_{2}(\Omega)}^{2} \geq \alpha \norm{\eta}_{H^{1}_{0}(\Omega)}^{2}
        \end{equation}
        für alle $\eta \in H^{1}_{0}(\Omega)$.
        \label{lemma:a_bf_bounded_garding:2}
    \end{enumerate}

    \begin{Beweis}
    Wir zeigen zunächst die Beschränktheit.
    Seien dazu $\eta, \zeta \in H^{1}_{0}(\Omega)$ beliebig.
    Unter Verwendung der Dreiecks- und der Cauchy-Schwarz-Ungleichung erhalten wir
    \begin{align}
        \abs{a(\eta, \zeta)}
        &= \abs{\sigma \skprod{\grad \eta}{\grad \zeta}_{L_{2(\Omega, \mathbb{R}^{n})}} + \skprod{\omega \eta}{\zeta}_{L_{2}(\Omega)}}
        \\&\leq \abs{\sigma} \abs{\skprod{\grad \eta}{\grad \zeta}_{L_{2(\Omega, \mathbb{R}^{n})}}} + \abs{\skprod{\omega \eta}{\zeta}_{L_{2}(\Omega)}}
        \\&\leq \abs{\sigma} \norm{\grad \eta}_{L_{2}(\Omega)} \norm{\grad \zeta}_{L_{2}(\Omega)} + \norm{\omega}_{L_{\infty}(\Omega)} \norm{\eta}_{L_{2}(\Omega)} \norm{\zeta}_{L_{2}(\Omega)}
        \\&\leq \max \left\{ \abs{\sigma}, \norm{\omega}_{L_{\infty}(\Omega)} \right\} \norm{\eta}_{H^{1}_{0}(\Omega)} \norm{\zeta}_{H^{1}_{0}(\Omega)}
        \\&= M_{a} \norm{\eta}_{H^{1}_{0}(\Omega)} \norm{\zeta}_{H^{1}_{0}(\Omega)}
    \end{align}
    mit $M_{a} = \max \left\{ \abs{\sigma}, \norm{\omega}_{L_{\infty}(\Omega)} \right\}$.

    Für die G\aa{}rding-Ungleichung seien nun $\eta \in H^{1}_{0}(\Omega)$ und $\lambda \in \mathbb{R}$.
    Wir betrachten
    \begin{align}
        a(\eta, \eta) + \lambda \norm{\eta}^{2}_{L_{2}(\Omega)}
        &= \sigma \norm{\grad \eta}^{2}_{L_{2}(\Omega)} + \skprod{\omega \eta}{\eta}_{L_{2}(\Omega)} + \lambda \skprod{\eta}{\eta}_{L_{2}(\Omega)}
        \\&= \sigma \norm{\grad \eta}^{2}_{L_{2}(\Omega)} + \skprod{(\omega + \lambda) \eta}{\eta}_{L_{2}(\Omega)},
        \intertext{wählen wir nun $\lambda = \norm{\omega}_{L_{\infty}(\Omega)}$, dann gilt $\omega + \lambda \geq 0$ fast überall in $\Omega$ und wir erhalten die Abschätzung}
        a(\eta, \eta) + \lambda \norm{\eta}^{2}_{L_{2}(\Omega)}
        &\geq \sigma \norm{\grad \eta}^{2}_{L_{2}(\Omega)},
        \intertext{woraus wir durch Anwenden der Poincaré-Friedrichs-Ungleichung}
        a(\eta, \eta) + \lambda \norm{\eta}^{2}_{L_{2}(\Omega)}
        &\geq \sigma \gamma_{\Omega}^{2} \norm{\eta}^{2}_{H^{1}_{0}(\Omega)}
        = \alpha \norm{\eta}^{2}_{H^{1}_{0}(\Omega)},
    \end{align}
    mit $\alpha = \sigma \gamma_{\Omega}^{2}$, gewinnen.
    \end{Beweis}
\end{Lemma}



% subsubsection well_posedness (end)

% subsection problemstellung (end)

% subsection zu_kl_rende_fragen (end)
