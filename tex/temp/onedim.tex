%!TEX root = ../main.tex

\pagestyle{plain}

\section*{Eindimensionaler Fall mit $\omega \in \mathbb{R}$ und ohne Quellterm}

Sei $I := [0, \hat t]$ für ein $0 < \hat t < \infty$ und $\Omega := [0, 1]$.
Betrachte folgende parametrisierte PDE
\begin{align}
    \begin{cases}
    u_{t}(t, x) = \sigma u_{xx}(t, x) - \omega u(t, x), & (t, x) \in I \times \Omega\\
    u(0, x) = g(x), & x \in \Omega \\
    u(t, 0) = u(t, 1) = 0, & t \in I
    \end{cases}
\end{align}
mit Konstanten $\sigma, \omega \in \mathbb{R}$.

Ein Separation der Variablen Ansatz $u(t, x) = X(x) T(t)$ liefert
\begin{equation}
    X(x)T'(t) = \sigma X''(x) T(t) - \omega X(x) T(t)
\end{equation}
oder auch
\begin{equation}
    \frac{T'(t)}{T(t)} = \sigma \frac{X''(x)}{X(x)} - \omega = \lambda
\end{equation}
mit $\lambda \in \mathbb{R}$.

Ohne Einschränkung sei $\lambda \neq 0$, dann erhalten wir zum einen die Dgl.
    $T'(t) = \lambda T(t)$,
deren Lösung
\begin{equation}
    T(t) = d_{3} e^{\lambda t}
\end{equation}
ist, und zum anderen die Dgl.
    $X''(x) =  \frac{\lambda + \omega}{\sigma} X(x)$
mit der Lösung
\begin{equation}
    X(x) = d_{1} e^{\sqrt{\frac{\lambda + \omega}{\sigma}} x} + d_{2} e^{-\sqrt{\frac{\lambda + \omega}{\sigma}}x},
\end{equation}
wobei $d_{1}, d_{2}, d_{3} \in \mathbb{R}$.

Als nächstes Verwenden wir die Anfangs- und Randbedingungen um die Konstanten $d_{i}$ zu bestimmen.
Sei
\begin{equation}
    u(t, x) = \left( d_{1} e^{\sqrt{\frac{\lambda + \omega}{\sigma}} x} + d_{2} e^{-\sqrt{\frac{\lambda + \omega}{\sigma}}x} \right) \left( d_{3} e^{\lambda t} \right),
\end{equation}
Betrachten wir zunächst die Randbedingung $u(t, 0) = u(t, 1) = 0$, dann erhalten wir aus
\begin{equation}
    0 = u(t, 0) = \left( d_{1} + d_{2} \right) \left( d_{3} e^{\lambda t} \right),
\end{equation}
oder äquivalent $d_{1} = - d_{2}$, und aus 
\begin{equation}
    0 = u(t, 1) = \left( d_{1} e^{\sqrt{\frac{\lambda + \omega}{\sigma}}} + d_{2} e^{-\sqrt{\frac{\lambda + \omega}{\sigma}}} \right) \left( d_{3} e^{\lambda t} \right) = 
    d_{1} \left( e^{\sqrt{\frac{\lambda + \omega}{\sigma}}} - e^{-\sqrt{\frac{\lambda + \omega}{\sigma}}} \right) \left( d_{3} e^{\lambda t} \right),
\end{equation}
ohne Einschränkung $d_{1} \neq 0$, die Gleichung
\begin{equation}
    0 = e^{\sqrt{\frac{\lambda + \omega}{\sigma}}} - e^{-\sqrt{\frac{\lambda + \omega}{\sigma}}}.
\end{equation}
Aus dieser erhalten wir durch Äquivalenzumformungen
\begin{align}
    e^{\sqrt{\frac{\lambda + \omega}{\sigma}}} - e^{-\sqrt{\frac{\lambda + \omega}{\sigma}}} = 0
    &\quad \iff \quad
    e^{2\sqrt{\frac{\lambda + \omega}{\sigma}}} = 1
    \quad \iff \quad
    \sqrt{\tfrac{\lambda + \omega}{\sigma}} = k \pi i
    \\&\quad \iff \quad
    \tfrac{\lambda + \omega}{\sigma} = -k^2 \pi^2
    \quad \iff \quad
    \lambda = -k^2 \pi^2 \sigma - \omega,
\end{align}
mit $k \in \mathbb{Z}$ beliebig.
Einsetzen liefert nun
\begin{align}
    u_{k}(t, x) &= d_{1} \left( e^{k \pi i x} - e^{-k \pi i x} \right) \left( d_{3} e^{- (k^2 \pi^2 \sigma + \omega) t} \right)
    \\&= 2 d_{1} d_{3} i \sin(k \pi x) e^{-(k^2 \pi^2 \sigma + \omega)t},
\end{align}
wobei wir $\beta_{k} := 2 d_{1} d_{3} i$ setzen.

Da jedes $u_{k}$, $k \in \mathbb{Z}$, eine Lösung ist, erhalten wir durch
\begin{equation}
    u(t, x) = \sum_{k = 1}^{\infty} u_{k}(t, x) = \sum_{k = 1}^{\infty} \beta_{k} \sin(k \pi x) e^{-(k^2 \pi^2 \sigma + \omega)t}}
\end{equation}
ebenfalls eine Lösung. 
Damit die Anfangsbedingung erfüllt wird, muss
\begin{equation}
    g(x) = u(x, 0) = \sum_{k = 1}^{\infty} \beta_{k} \sin(k \pi x)
\end{equation}
gelten, was genau dann der Fall ist, wenn
\begin{equation}
    \beta_{k} = 2 \int_{0}^{1} g(x) \sin(k \pi x) \diff x.
\end{equation}

Da $u_{k}$ analytisch in $\omega$ für alle $k \in \mathbb{Z}$, ist auch $u$ analytisch in $\omega$ und es gilt
\begin{equation}
    \frac{\partial^{j} u(t, x; \omega)}{\partial \omega^{j}} = \sum_{k = 1}^{\infty} (-t)^{j} \beta_{k} \sin(k \pi x) e^{-(k^{2} \pi^{2} \sigma + \omega)t}.
\end{equation}
