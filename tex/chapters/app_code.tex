% -*- root: ../main.tex -*-

\documentclass[../main.tex]{subfiles}
\begin{document}

\chapter{Inhalt der Begleit-DVD} % (fold)
\label{cha:inhalt_der_begleit_dvd}

Dieser Arbeit liegt eine DVD bei, welche die Implementierungen der beschriebenen Verfahren sowie die in den \cref{sec:cha4_galerkin:beispiele,sec:cha5_rbm:beispiele} aufgeführten Beispiele enthält.
Ferner findet sich der Inhalt der DVD auch als Git"=Repository unter \url{https://github.com/nobbs/thesis}.

Die relevanten Implementierungen und Skripte sind vollständig in \textcite{Matlab} gehalten.
Weiter sind diese bis mindestens Version 2013a abwärtskompatibel und plattformunabhängig ausführbar.
Neben dem eigentlichen MATLAB-Softwarepaket wird die \emph{Optimization Toolbox} für die linearen Programme der \acl{scm} benötigt.

Als Hardware für die durchgeführten Simulationen diente vor allem ein MacBook Pro aus dem Jahre 2010 mit dem Betriebssystem Mac OS X 10.11 Beta.
Die hier nennenswerte Hardware sind eine Intel Core 2 Duo CPU mit 2.4 GHz und 8 GB Arbeitsspeicher.

Bevor der Inhalt der DVD genauer beleuchtet wird, soll an dieser Stelle durch ein Überblick über die wichtigsten Verzeichnisse geboten werden.

\dirtree{%
.1 /.
.2 code.
.3 examples\DTcomment{Beispiele}.
.4 chapter4.
.4 chapter5.
.3 lib\DTcomment{Externe MatLab-Skripte und Libraries}.
.3 src\DTcomment{Hauptteil der Implementierung}.
.3 test.
.2 config.
.2 doc\DTcomment{Automatisch generierte Dokumentation der Implementierung}.
.3 index.html.
.2 tex\DTcomment{\LaTeX-Dateien dieser Thesis}.
.2 README.md.
}

\end{document}
