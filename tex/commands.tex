%!TEX root = main.tex

% griechisches Alphabet
\renewcommand{\epsilon}{\varepsilon}
\renewcommand{\theta}{\vartheta}

%%% Operatoren und Funktionen
\DeclarePairedDelimiter{\abs}{\lvert}{\rvert}
\DeclarePairedDelimiter{\norm}{\lVert}{\rVert}

\DeclarePairedDelimiter{\ceil}{\lceil}{\rceil}
\DeclarePairedDelimiter{\floor}{\lfloor}{\rfloor}

\newcommand{\skprod}[2]{\left\langle#1,#2\right\rangle}
\newcommand{\fracpart}[2]{\frac{\partial#1}{\partial#2}}

\newcommand{\Transp}{^{\mathrm{T}}}
\newcommand{\Stern}{^{*}}
\newcommand{\Int}[1]{#1^\circ}
\newcommand{\Ext}[1]{\overline{#1}}

\DeclareMathOperator{\spn}{span}
\DeclareMathOperator{\ee}{e}
\DeclareMathOperator{\ii}{i}

\newcommand{\grad}{\nabla}
\DeclareMathOperator{\divergenz}{div}
\newcommand{\hesse}{\nabla^2}

\newcommand\restr[2]{\ensuremath{\left.#1\right|_{#2}}}

\newcommand{\fa}{\text{für alle}~}

% fetter Vektor
\renewcommand{\vec}[1]{\mathbf{#1}}
\newcommand{\mat}[1]{\mathbf{#1}}

% Differential-d
\newcommand*\diff{\mathop{}\!\mathrm{d}}
\newcommand*\Diff[1]{\mathop{}\!\mathrm{d^#1}}

\DeclareMathOperator*{\esssup}{ess\,sup}
\newcommand\blank{{\mkern2mu\cdot\mkern2mu}}
