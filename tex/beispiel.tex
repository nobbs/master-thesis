%!TEX root = main.tex

\section{Numerische Beispiele} % (fold)
\label{sec:beispiel}

Der in diesem Abschnitt verwendete \texttt{MatLab}-Sourcecode kann unter \url{https://github.com/nobbs/hs-rb-ws13} eingesehen werden.

\subsection{Durchschnittliche und maximale Effektivität} % (fold)
\label{sub:durchschnittliche_und_maximale_effektivitaet}

Bevor wir uns den Beispielen widmen, wollen wir zunächst einige Maße einführen, mit denen man die Güte unserer Fehlerschätzer bewerten kann.
Dazu betrachten wir vor allem die durchschnittliche und die maximale Effektivität des Fehlerschätzers $\Delta^s_N$.

% Um Aussagen über die Güte der Fehlerschätzer zu treffen, ist es sinnvoll sich die durchschnittlichen beziehungsweise maximalen Effektivitäten anzusehen.
Sei dazu $\Xi_{\text{test}} \subset \mathcal D$ eine endliche Teilmenge des Parameterraumes mit $n_\text{test} \in \mathbb{N}$ Elementen.
Wir nennen, wobei $\bullet$ ein Platzhalter für \glqq{}$\text{en}$\grqq{}, \glqq{}$s$\grqq{}, \glqq{} \grqq{} sei,
\begin{equation}
    \eta^\bullet_{N,\max} := \max_{\mu \in \Xi_\text{test}} \eta^\bullet_N(\mu), \qquad
    \eta^\bullet_{N,\text{ave}} := \frac{1}{n_\text{test}} \sum_{\mu \in \Xi_\text{test}} \eta^\bullet_N(\mu)
\end{equation}
maximale respektive durchschnittliche Effektivität des Fehlerschätzers $\Delta^\bullet_N$ über $\Xi_\text{test}$.

Wir beschränken uns nun im Wesentlichen auf die Effektivität von $\Delta^s_N$. Korollar \ref{korollar:effektivitaeten} liefert uns
\begin{equation}
    \eta^s_{N,\max} \leq \max_{\mu \in \Xi_\text{test}} \frac{\gamma(\mu)}{\alpha_{\text{LB}}(\mu)} \leq \max_{\mu \in \mathcal D} \frac{\gamma(\mu)}{\alpha_{\text{LB}}(\mu)} =: \eta^s_{\max,\text{UB}}.
\end{equation}
Diese obere Schranke ist sowohl unabhängig von der Dimension $N$ des Reduzierte-Basis-Ansatzraumes als auch von der Dimension $\mathcal N$ des Galerkin-Ansatzraumes und bestärkt uns damit in der Wahl unserer Fehlerschätzer. Allerdings kann $\eta^s_{\max,\text{UB}}$ recht groß werden, da hierbei vom \emph{worst-case} ausgegangen wird.

Ein weiteres Maß für die Güte der Fehlerschätzer liefert der Quotient des maximalen geschätzten Fehlers und des maximalen tatsächlichen Fehlers (statt wie oben das Maximum des Quotienten dieser Werte), gegeben durch
\begin{equation}
     \rho^s_{\text{err},N} := \frac{\max\limits_{\mu \in \Xi_\text{test}} \Delta_N^s(\mu)}{\max\limits_{\mu \in \Xi_\text{test}} (s(\mu) - s_N(\mu))}.
\end{equation}

% subsection durchschnittliche_und_maximale_effektivitaet   (end)

\subsection{Eindimensionaler Parameterraum} % (fold)
\label{sub:eindimensionaler_parameterraum}

% subsection eindimensionaler_parameterraum (end)

Wir betrachten nun das Thermal-Block-Beispiel (siehe \cite[2.2.1]{Paro}), allerdings nur für einen eindimensionalen Parameter.

Es sei $\Omega = [ 0, 1 ]^2 \subset \mathbb{R}^2$. Wir teilen den Definitionsbereich in zwei Rechtecke $\Omega_1 = [0, \frac{1}{2}] \times [ 0, 1 ] $ und $\Omega_2 = [ \frac{1}{2}, 1 ] \times [ 0, 1 ]$ auf und unterteilen den Rand von $\Omega$ in $\Gamma_1 = [0, 1] \times \{ 0 \}$, $\Gamma_2 = \{ 1 \} \times [0, 1]$, $\Gamma_1 = [0, 1] \times \{ 1 \}$ und $\Gamma_4 = \{ 0 \} \times [0, 1]$.

Sei der Parameterbereich gegeben durch $\mathcal D = [0.1, 10]$. Als Referenzparameter wählen wir $\bar \mu = 1$.

Die zugrundeliegende partielle Differentialgleichung lautet
\begin{equation}
    \divergenz{(\hat \mu \grad u)}  = 0 \quad \text{in}~\Omega,
\end{equation}
mit den Nebenbedingungen
\begin{equation}
        \frac{\partial u}{\partial \nu} = 0 \quad \text{auf}~\Gamma_2, \Gamma_4, \qquad
    \frac{\partial u}{\partial \nu} = 1 \quad \text{auf}~\Gamma_1, \qquad
    u                               = 0 \quad \text{auf}~\Gamma_3,
\end{equation}
wobei die Abbildung $\hat \mu \colon \Omega \to \mathbb{R}$ die Eigenschaft $\restr{\hat \mu}{\Omega_1} = \mu$ und $\restr{\hat \mu}{\Omega_2} = 1$ erfüllt.

Das zugehörige Variatonsproblem lautet nun:
\begin{addmargin}[2em]{2em}
Sei $\mu \in \mathcal D$. Gesucht ist ein $u(\mu) \in X := \{ v \in H^1(\Omega) \mid \restr{v}{\Gamma_3} = 0 \}$, sodass gilt
\begin{equation}
    \mu \int_{\Omega_1} \nabla u(\mu) \nabla v \, \mathrm d x + \int_{\Omega_2} \nabla u(\mu) \nabla v \, \mathrm d x = \int_{\Gamma_1} v \, \mathrm{d} x \quad \fa v \in X.
\end{equation}
Wir interessieren uns nun für
\begin{equation}
    s(\mu) = \int_{\Gamma_1} u(\mu) \, \mathrm{d}x.
\end{equation}
\end{addmargin}

Wir schreiben dieses Variatonsproblem nun als
\begin{equation}
    a(w, v;\mu) = \Theta_a^1(\mu) a^1(w, v) + \Theta_a^2(\mu) a^2(w, v) = \Theta_f^1(\mu) f^1(v),
\end{equation}
wobei
\begin{align}
    a^1(w, v) &= \int_{\Omega_1} \nabla w \nabla v \, \mathrm{d}x,
    \quad
    &a^2(w, v) &= \int_{\Omega_2} \nabla w \nabla v \, \mathrm{d}x,
    \quad
    &f^1(v) &= \int_{\Gamma_1} v \, \mathrm{d}x,
    \\
    \Theta_a^1(\mu) &= \mu,
    &\Theta_a^2(\mu) &= 1,
    &\Theta_f^1(\mu) &= 1.
\end{align}

Damit erhalten wir für unsere untere Schranke der Koerzivitätskonstante
\begin{equation}
    \alpha_\text{LB}(\mu) = \min_{k = 1 \ldots 2} \frac{\Theta_a^k(\mu)}{\Theta_a^k(\bar \mu)} = \min \left(  \frac{\mu}{\bar \mu}, 1  \right)  = \min(\mu, 1).
\end{equation}

Analog zu $\alpha_\text{LB}$ aus Abschnitt \ref{sub:untere_schranke_f_r_die_koerzivit_tskonstante} lässt sich für die Stetigkeitskonstante eine obere Schranke bestimmen mittels
\begin{equation}
    \gamma(\mu) \leq \gamma_\text{UB}(\mu) := \Theta_a^{\max, \bar \mu}(\mu) = \max_{k = 1 \ldots 2} \frac{\Theta_a^k(\mu)}{\Theta_a^k(\bar \mu)} = \max(\mu, 1), \quad \fa \mu \in \mathcal D.
\end{equation}
Damit können wir nun $\eta^s_{\max,\text{UB}}$ nach oben abschätzen und erhalten für unsere Wahl von $\mathcal D$
\begin{equation}
    \eta^s_{\max,\text{UB}} = \max_{\mu \in D} \frac{\gamma(\mu)}{\alpha_{\text{LB}}(\mu)} \leq \max_{\mu \in D} \frac{\gamma_\text{UB}(\mu)}{\alpha_{\text{LB}}(\mu)} \leq 10.
\end{equation}

Um das Variationsproblem für festes $\mu \in \mathcal D$ zu lösen, verwenden wir die Finite-Elemente-Methode mit $\mathcal N = 314$.

Wir verwenden $\Xi_\text{train} = \Xi_\text{test} \subset \mathcal D$ mit $10^4$ Elementen, welche zufällig logarithmisch gleichverteilt aus $\mathcal D$ gewählt werden.
Als ersten Snapshot-Parameter wählen wir $\mu_1 = \bar \mu$, alle weiteren werden mittels Greedy-Verfahren bestimmt. Als Fehlertoleranz für $\Delta^s_N$ wird $10^{-8}$ verwendet.

Wie man an den Ergebnissen in Tabelle \ref{tab:eindim} und Abbildung \ref{fig:plot_s_fehler} sehen kann, wird diese Toleranz bereits für $N = 4$ erreicht.
Zudem sieht man in Abbildung \ref{fig:plot_s_fehler}, wie sich der tatsächliche Fehler $s - s_N$ und der Fehlerschätzer $\Delta^s_N$ verhalten.

\begin{table}[h!]
    \begin{center}
    \small
        \begin{tabular}{r|llll}
        $N$ & $\Delta^s_{N,\max}$ & $\eta^s_{N,\text{ave}}$ & $\eta^s_{N,\max}$ & $\rho^s_{\text{err}, N}$ \\
        \hline
            $1$ & $1.7856 \cdot 10^{0}$ & $1.4812$ & $1.8186$ & $1.8186$ \\
            $2$ & $9.9302 \cdot 10^{-2}$ & $2.2520$ & $8.1952$ & $1.1818$ \\
            $3$ & $2.7852 \cdot 10^{-7}$ & $2.4487$ & $5.9672$ & $2.7117$ \\
            $4$ & $4.3904 \cdot 10^{-11}$ & $2.3990$ & $57.6900$ & $2.4440$ \\
        \end{tabular}
        \caption{Der hohe Wert von $\eta^s_{\max,\text{UB}}$ für $N = 4$ lässt sich durch Rechenungenauigkeiten erklären, da der Fehlerschätzer bereits im Bereich $10^{-11}$ liegt und der tatsächliche Fehler noch kleiner ist.}
        \label{tab:eindim}
    \end{center}
\end{table}

\begin{figure}[h!]
    \begin{center}
        \tiny
        \newlength\figureheight
        \newlength\figurewidth
        \setlength\figureheight{4cm}
        \setlength\figurewidth{0.4\textwidth}
        \begin{subfigure}[b]{0.45\textwidth}
            ~
            % This file was created by matlab2tikz v0.4.6 running on MATLAB 8.1.
% Copyright (c) 2008--2014, Nico Schlömer <nico.schloemer@gmail.com>
% All rights reserved.
% Minimal pgfplots version: 1.3
%
% The latest updates can be retrieved from
%   http://www.mathworks.com/matlabcentral/fileexchange/22022-matlab2tikz
% where you can also make suggestions and rate matlab2tikz.
%
\begin{tikzpicture}

\begin{axis}[%
width=10cm,
height=7cm,
scale only axis,
xmin=0,
xmax=10,
ymode=log,
ymin=1e-16,
ymax=1,
yminorticks=true,
ultra thick,
xlabel={One dimensional parameter space},
legend style={at={(1,0.03)},anchor=south east,legend cell align=left,align=left,fill=none,draw=none}
]
\addplot [color=matlab1,solid]
  table[row sep=crcr]{
0.100015	1.78556	\\
0.100336	1.77755	\\
0.100561	1.77196	\\
0.101092	1.75887	\\
0.101691	1.74429	\\
0.102174	1.73265	\\
0.10289	1.71562	\\
0.103285	1.70633	\\
0.103801	1.69432	\\
0.104376	1.68106	\\
0.104902	1.66907	\\
0.10552	1.65517	\\
0.106405	1.63553	\\
0.106778	1.62735	\\
0.107259	1.61691	\\
0.107686	1.60772	\\
0.107963	1.6018	\\
0.108685	1.5865	\\
0.109214	1.57545	\\
0.109707	1.56524	\\
0.110134	1.55648	\\
0.111103	1.53686	\\
0.111734	1.52429	\\
0.112186	1.51537	\\
0.112688	1.50555	\\
0.113033	1.49887	\\
0.113665	1.48672	\\
0.114368	1.47337	\\
0.11488	1.46375	\\
0.115329	1.45542	\\
0.115513	1.45201	\\
0.116102	1.4412	\\
0.116665	1.43098	\\
0.116962	1.42563	\\
0.117579	1.41459	\\
0.11802	1.40679	\\
0.118636	1.396	\\
0.119443	1.38204	\\
0.119963	1.37313	\\
0.120924	1.35693	\\
0.121181	1.35264	\\
0.122195	1.3359	\\
0.122727	1.32724	\\
0.12329	1.31815	\\
0.123732	1.31109	\\
0.12409	1.30542	\\
0.124442	1.29986	\\
0.125119	1.28928	\\
0.125727	1.27987	\\
0.126295	1.27118	\\
0.127179	1.25782	\\
0.127657	1.25067	\\
0.128238	1.24207	\\
0.128554	1.23742	\\
0.129049	1.2302	\\
0.129915	1.2177	\\
0.130325	1.21184	\\
0.130921	1.20341	\\
0.131481	1.19555	\\
0.132223	1.18526	\\
0.132941	1.17541	\\
0.133901	1.16243	\\
0.134572	1.15348	\\
0.135143	1.14594	\\
0.135771	1.13772	\\
0.136557	1.12756	\\
0.137199	1.11934	\\
0.137877	1.11077	\\
0.138381	1.10445	\\
0.139077	1.09581	\\
0.139834	1.08652	\\
0.140459	1.07892	\\
0.141017	1.07222	\\
0.141424	1.06735	\\
0.142416	1.05563	\\
0.143505	1.04298	\\
0.144129	1.03582	\\
0.144945	1.02656	\\
0.145646	1.01869	\\
0.146226	1.01225	\\
0.146843	1.00546	\\
0.147435	0.999	\\
0.148121	0.991587	\\
0.148733	0.985035	\\
0.1495	0.976919	\\
0.150498	0.966479	\\
0.151159	0.959653	\\
0.151942	0.95165	\\
0.152293	0.948096	\\
0.153102	0.939965	\\
0.154286	0.928237	\\
0.154867	0.922558	\\
0.155517	0.916259	\\
0.156005	0.911572	\\
0.156597	0.905919	\\
0.157325	0.899044	\\
0.158119	0.891616	\\
0.159341	0.880356	\\
0.159964	0.874682	\\
0.160553	0.869367	\\
0.161612	0.859924	\\
0.162586	0.851358	\\
0.163148	0.846476	\\
0.163583	0.842718	\\
0.164439	0.835382	\\
0.165021	0.830447	\\
0.165753	0.824294	\\
0.166827	0.815381	\\
0.167842	0.807068	\\
0.168846	0.798959	\\
0.170115	0.788866	\\
0.170973	0.782139	\\
0.171892	0.775012	\\
0.172206	0.772598	\\
0.173045	0.766202	\\
0.173717	0.761124	\\
0.174618	0.754385	\\
0.175431	0.748381	\\
0.175763	0.745938	\\
0.17639	0.741369	\\
0.176848	0.738054	\\
0.177562	0.732919	\\
0.178582	0.725669	\\
0.179392	0.719976	\\
0.180059	0.715337	\\
0.180937	0.709285	\\
0.181749	0.703752	\\
0.182615	0.697909	\\
0.183481	0.692124	\\
0.184193	0.687424	\\
0.184908	0.682742	\\
0.185697	0.677622	\\
0.186241	0.674123	\\
0.187253	0.66767	\\
0.188144	0.662062	\\
0.189283	0.654971	\\
0.189833	0.651585	\\
0.190917	0.64498	\\
0.191531	0.641276	\\
0.192787	0.633781	\\
0.193147	0.631657	\\
0.194284	0.624997	\\
0.195764	0.616467	\\
0.196783	0.610683	\\
0.197564	0.606294	\\
0.198592	0.60058	\\
0.199754	0.594203	\\
0.200825	0.5884	\\
0.201526	0.584646	\\
0.202452	0.579728	\\
0.203152	0.576044	\\
0.203499	0.574231	\\
0.204247	0.570345	\\
0.204955	0.566697	\\
0.205628	0.563255	\\
0.206233	0.560189	\\
0.206993	0.556362	\\
0.207526	0.553695	\\
0.208113	0.550783	\\
0.209284	0.545023	\\
0.211021	0.536625	\\
0.212628	0.528994	\\
0.213591	0.524488	\\
0.214658	0.519552	\\
0.215148	0.517305	\\
0.216583	0.510794	\\
0.217329	0.507451	\\
0.218106	0.503995	\\
0.218764	0.501092	\\
0.219373	0.498424	\\
0.220098	0.495268	\\
0.221431	0.489537	\\
0.221823	0.487867	\\
0.22244	0.48525	\\
0.223583	0.480453	\\
0.224506	0.476622	\\
0.225701	0.471716	\\
0.226956	0.466634	\\
0.227942	0.462687	\\
0.229057	0.458272	\\
0.230206	0.453778	\\
0.231229	0.449821	\\
0.232482	0.445036	\\
0.234038	0.439176	\\
0.234983	0.435663	\\
0.236193	0.431216	\\
0.237082	0.427985	\\
0.238066	0.424441	\\
0.238912	0.421424	\\
0.239986	0.41763	\\
0.241948	0.410806	\\
0.242831	0.407778	\\
0.243463	0.405628	\\
0.244878	0.400866	\\
0.245524	0.398714	\\
0.246599	0.39516	\\
0.24792	0.390849	\\
0.24935	0.386245	\\
0.250667	0.382065	\\
0.251826	0.378427	\\
0.252669	0.375811	\\
0.253519	0.373191	\\
0.254469	0.370292	\\
0.25498	0.368743	\\
0.25552	0.367114	\\
0.256499	0.364184	\\
0.257579	0.360983	\\
0.258834	0.357306	\\
0.260244	0.353228	\\
0.261773	0.348866	\\
0.262476	0.346883	\\
0.263972	0.342706	\\
0.265131	0.339513	\\
0.266681	0.335294	\\
0.267848	0.332159	\\
0.269029	0.329021	\\
0.270687	0.324676	\\
0.272064	0.321118	\\
0.273021	0.318671	\\
0.274058	0.316046	\\
0.274715	0.314395	\\
0.275995	0.311209	\\
0.277002	0.308727	\\
0.279084	0.303671	\\
0.280201	0.300998	\\
0.280852	0.299453	\\
0.281772	0.297286	\\
0.282503	0.295574	\\
0.283784	0.292606	\\
0.284679	0.290551	\\
0.285512	0.288655	\\
0.286971	0.285368	\\
0.287939	0.283211	\\
0.290064	0.278542	\\
0.291414	0.275621	\\
0.293254	0.271695	\\
0.294542	0.268986	\\
0.296175	0.265595	\\
0.297304	0.26328	\\
0.298542	0.260767	\\
0.299978	0.257888	\\
0.301643	0.254592	\\
0.303399	0.251169	\\
0.304693	0.248679	\\
0.306072	0.246059	\\
0.307168	0.243998	\\
0.308907	0.240766	\\
0.310409	0.238013	\\
0.311453	0.236122	\\
0.312886	0.233552	\\
0.314384	0.2309	\\
0.315802	0.228421	\\
0.316826	0.226649	\\
0.318472	0.223832	\\
0.319978	0.221289	\\
0.320952	0.219662	\\
0.322137	0.2177	\\
0.323487	0.215489	\\
0.32546	0.2123	\\
0.326333	0.210907	\\
0.327803	0.208583	\\
0.328939	0.206808	\\
0.330106	0.205	\\
0.331217	0.203295	\\
0.332388	0.201516	\\
0.334033	0.199044	\\
0.336997	0.194676	\\
0.337996	0.193227	\\
0.339107	0.191629	\\
0.34047	0.18969	\\
0.341418	0.188353	\\
0.343424	0.185559	\\
0.345429	0.18281	\\
0.346538	0.181309	\\
0.347824	0.179586	\\
0.349127	0.177857	\\
0.351381	0.17491	\\
0.352489	0.173479	\\
0.353752	0.171865	\\
0.355398	0.169786	\\
0.357443	0.167241	\\
0.358614	0.165801	\\
0.360395	0.163637	\\
0.362943	0.160594	\\
0.364721	0.158507	\\
0.367066	0.155798	\\
0.368198	0.154508	\\
0.370096	0.15237	\\
0.371913	0.150354	\\
0.374146	0.147913	\\
0.375713	0.146226	\\
0.378326	0.143457	\\
0.379935	0.14178	\\
0.381667	0.139998	\\
0.383385	0.138253	\\
0.385675	0.135963	\\
0.387083	0.134575	\\
0.38852	0.133173	\\
0.391974	0.129865	\\
0.39389	0.128067	\\
0.395107	0.126939	\\
0.396493	0.125667	\\
0.397647	0.124617	\\
0.399095	0.123313	\\
0.400141	0.12238	\\
0.401839	0.12088	\\
0.403771	0.119197	\\
0.405512	0.117701	\\
0.407462	0.116048	\\
0.409876	0.114036	\\
0.411737	0.112508	\\
0.413276	0.111261	\\
0.414509	0.110271	\\
0.416468	0.108718	\\
0.417323	0.108047	\\
0.418981	0.106758	\\
0.420591	0.105521	\\
0.421779	0.104618	\\
0.423482	0.103336	\\
0.425064	0.102161	\\
0.426785	0.100896	\\
0.428318	0.0997841	\\
0.431548	0.0974795	\\
0.434767	0.0952365	\\
0.437359	0.0934679	\\
0.439772	0.0918507	\\
0.441974	0.090399	\\
0.44365	0.0893096	\\
0.445904	0.0878655	\\
0.447634	0.0867722	\\
0.450898	0.0847465	\\
0.452192	0.0839565	\\
0.454729	0.0824281	\\
0.456512	0.0813705	\\
0.459154	0.0798278	\\
0.461828	0.0782956	\\
0.463977	0.0770847	\\
0.466542	0.0756639	\\
0.469142	0.074249	\\
0.47098	0.0732649	\\
0.472419	0.0725029	\\
0.473777	0.0717909	\\
0.475743	0.0707721	\\
0.477949	0.0696455	\\
0.480112	0.0685575	\\
0.4835	0.0668862	\\
0.485685	0.0658289	\\
0.488222	0.0646214	\\
0.490474	0.0635674	\\
0.491845	0.0629339	\\
0.494792	0.0615919	\\
0.497299	0.0604716	\\
0.49979	0.0593778	\\
0.501809	0.0585049	\\
0.503775	0.0576665	\\
0.50673	0.0564274	\\
0.508105	0.0558593	\\
0.511606	0.0544371	\\
0.515601	0.0528556	\\
0.519691	0.0512806	\\
0.521657	0.0505391	\\
0.523934	0.0496928	\\
0.525638	0.0490679	\\
0.527003	0.0485727	\\
0.529262	0.0477629	\\
0.531221	0.047071	\\
0.53368	0.0462151	\\
0.536445	0.0452701	\\
0.538675	0.0445206	\\
0.541318	0.0436469	\\
0.543193	0.0430368	\\
0.544877	0.0424951	\\
0.548326	0.041405	\\
0.551087	0.0405508	\\
0.552196	0.0402122	\\
0.554199	0.039607	\\
0.5554	0.0392479	\\
0.558038	0.0384695	\\
0.559977	0.0379063	\\
0.563032	0.0370337	\\
0.566192	0.03615	\\
0.568973	0.0353879	\\
0.57462	0.0338835	\\
0.577324	0.0331833	\\
0.581472	0.032134	\\
0.583344	0.0316699	\\
0.586459	0.0309107	\\
0.588681	0.0303789	\\
0.592427	0.0295001	\\
0.595686	0.0287537	\\
0.596761	0.0285111	\\
0.600326	0.0277189	\\
0.601606	0.0274391	\\
0.604555	0.0268037	\\
0.607517	0.0261781	\\
0.610228	0.0256166	\\
0.612805	0.0250922	\\
0.614009	0.0248505	\\
0.617021	0.0242541	\\
0.619689	0.023736	\\
0.62227	0.0232437	\\
0.626615	0.0224343	\\
0.629554	0.0219001	\\
0.632247	0.02142	\\
0.638017	0.0204209	\\
0.639473	0.020175	\\
0.643873	0.0194464	\\
0.648235	0.0187457	\\
0.651095	0.0182975	\\
0.654668	0.01775	\\
0.657656	0.0173023	\\
0.659734	0.0169965	\\
0.663918	0.0163938	\\
0.666508	0.0160294	\\
0.670846	0.0154337	\\
0.674249	0.0149789	\\
0.677272	0.0145838	\\
0.68107	0.0140992	\\
0.683865	0.0137508	\\
0.686305	0.0134522	\\
0.691086	0.0128819	\\
0.692885	0.0126722	\\
0.695898	0.012327	\\
0.69955	0.0119186	\\
0.702505	0.0115958	\\
0.705434	0.0112825	\\
0.706966	0.0111213	\\
0.710002	0.010807	\\
0.71249	0.0105545	\\
0.718108	0.010001	\\
0.722177	0.009614	\\
0.724212	0.00942478	\\
0.727772	0.00910047	\\
0.730907	0.00882189	\\
0.733312	0.00861257	\\
0.737613	0.00824741	\\
0.743549	0.00776245	\\
0.746625	0.00751953	\\
0.748683	0.00736019	\\
0.750398	0.0072293	\\
0.754213	0.00694429	\\
0.757265	0.00672218	\\
0.760278	0.00650799	\\
0.762993	0.00631929	\\
0.768141	0.00597231	\\
0.771412	0.005759	\\
0.774243	0.00557888	\\
0.778214	0.00533295	\\
0.781509	0.00513481	\\
0.785039	0.00492833	\\
0.789806	0.00465879	\\
0.793167	0.00447508	\\
0.796088	0.00431952	\\
0.80221	0.0040057	\\
0.805098	0.00386321	\\
0.80767	0.0037393	\\
0.809874	0.00363529	\\
0.812891	0.00349609	\\
0.816306	0.00334292	\\
0.818988	0.00322587	\\
0.821809	0.00310574	\\
0.824666	0.00298712	\\
0.828553	0.0028307	\\
0.832199	0.00268897	\\
0.834358	0.00260727	\\
0.837803	0.00248033	\\
0.844224	0.00225468	\\
0.848839	0.002101	\\
0.854473	0.00192272	\\
0.858483	0.00180193	\\
0.865074	0.00161404	\\
0.869705	0.00148972	\\
0.873873	0.00138309	\\
0.879156	0.00125495	\\
0.882406	0.00117989	\\
0.885564	0.00110965	\\
0.889228	0.00103144	\\
0.891719	0.000980231	\\
0.893937	0.000935952	\\
0.898794	0.00084326	\\
0.905655	0.000722031	\\
0.909293	0.00066221	\\
0.914879	0.000576221	\\
0.91945	0.000511	\\
0.922132	0.000474815	\\
0.924317	0.000446468	\\
0.930693	0.000369392	\\
0.935614	0.000315508	\\
0.940388	0.000267762	\\
0.943663	0.000237514	\\
0.947772	0.000202393	\\
0.955624	0.000143746	\\
0.958188	0.000126944	\\
0.965368	8.58126e-05	\\
0.969607	6.55185e-05	\\
0.974504	4.56459e-05	\\
0.980515	2.63376e-05	\\
0.98462	1.62717e-05	\\
0.98935	7.7282e-06	\\
0.99532	1.47478e-06	\\
0.998514	1.47676e-07	\\
1.00389	1.00756e-06	\\
1.00875	5.06325e-06	\\
1.01152	8.74697e-06	\\
1.01834	2.20362e-05	\\
1.02222	3.21981e-05	\\
1.02639	4.52538e-05	\\
1.02968	5.70428e-05	\\
1.03472	7.76656e-05	\\
1.03708	8.83737e-05	\\
1.0432	0.000119286	\\
1.04803	0.000146707	\\
1.05248	0.000174425	\\
1.05556	0.000194894	\\
1.06137	0.000236453	\\
1.06618	0.00027374	\\
1.07213	0.000323286	\\
1.07635	0.000360743	\\
1.08382	0.000431664	\\
1.08761	0.00046983	\\
1.09427	0.000540597	\\
1.10094	0.000615889	\\
1.10428	0.000655211	\\
1.11107	0.000738565	\\
1.11445	0.000781666	\\
1.1179	0.000826726	\\
1.12264	0.000890578	\\
1.12783	0.000962847	\\
1.13184	0.00102034	\\
1.13883	0.001124	\\
1.14441	0.00120985	\\
1.14871	0.00127787	\\
1.15328	0.00135193	\\
1.15726	0.00141767	\\
1.1602	0.00146724	\\
1.16443	0.00153977	\\
1.16996	0.00163672	\\
1.17577	0.00174106	\\
1.17954	0.00181035	\\
1.1832	0.00187856	\\
1.18791	0.00196793	\\
1.1929	0.00206443	\\
1.19916	0.00218798	\\
1.20767	0.00236069	\\
1.21307	0.00247294	\\
1.21808	0.00257903	\\
1.22241	0.00267184	\\
1.2294	0.00282473	\\
1.23769	0.00301021	\\
1.24156	0.00309816	\\
1.24652	0.00321252	\\
1.25395	0.00338664	\\
1.26076	0.00354931	\\
1.26748	0.00371248	\\
1.27252	0.00383674	\\
1.27741	0.00395849	\\
1.28349	0.0041118	\\
1.2914	0.00431466	\\
1.29571	0.00442644	\\
1.30192	0.00458956	\\
1.30717	0.00472897	\\
1.31215	0.00486247	\\
1.31742	0.00500534	\\
1.32457	0.00520114	\\
1.32922	0.00532997	\\
1.33684	0.00554319	\\
1.34381	0.00574056	\\
1.34972	0.00590989	\\
1.35563	0.00608073	\\
1.36125	0.00624455	\\
1.36887	0.00646884	\\
1.37549	0.00666579	\\
1.38533	0.0069619	\\
1.39279	0.00718918	\\
1.39763	0.00733768	\\
1.4055	0.00758111	\\
1.41028	0.00773037	\\
1.41439	0.00785904	\\
1.41835	0.00798385	\\
1.42524	0.00820214	\\
1.42924	0.00832967	\\
1.43495	0.00851274	\\
1.44057	0.00869377	\\
1.44784	0.00892992	\\
1.46077	0.00935386	\\
1.46666	0.00954891	\\
1.48032	0.0100048	\\
1.4846	0.0101488	\\
1.49294	0.0104312	\\
1.49869	0.010627	\\
1.50401	0.0108087	\\
1.51435	0.0111642	\\
1.52143	0.0114096	\\
1.53151	0.0117606	\\
1.53696	0.0119516	\\
1.54633	0.0122816	\\
1.54953	0.0123946	\\
1.55879	0.0127232	\\
1.56552	0.0129631	\\
1.57553	0.013322	\\
1.58196	0.0135535	\\
1.5874	0.0137503	\\
1.59513	0.0140306	\\
1.59983	0.0142014	\\
1.60538	0.014404	\\
1.61302	0.0146835	\\
1.61949	0.0149211	\\
1.62707	0.0152005	\\
1.63344	0.0154357	\\
1.64378	0.0158196	\\
1.6555	0.0162563	\\
1.66625	0.0166585	\\
1.67451	0.0169689	\\
1.68825	0.0174874	\\
1.69695	0.0178164	\\
1.7046	0.0181068	\\
1.7117	0.0183771	\\
1.72369	0.0188346	\\
1.72862	0.0190233	\\
1.73709	0.0193476	\\
1.7441	0.0196168	\\
1.75046	0.0198614	\\
1.75613	0.02008	\\
1.76015	0.0202348	\\
1.76765	0.0205246	\\
1.77208	0.0206958	\\
1.78519	0.0212034	\\
1.79739	0.0216771	\\
1.80889	0.0221247	\\
1.81709	0.0224441	\\
1.82862	0.0228944	\\
1.83824	0.0232705	\\
1.84642	0.0235906	\\
1.85506	0.0239292	\\
1.86419	0.0242873	\\
1.87	0.0245157	\\
1.87543	0.024729	\\
1.88149	0.024967	\\
1.88978	0.0252933	\\
1.89788	0.0256121	\\
1.90767	0.0259974	\\
1.91344	0.0262252	\\
1.92585	0.0267144	\\
1.93516	0.0270815	\\
1.94096	0.0273106	\\
1.95279	0.0277777	\\
1.96239	0.0281571	\\
1.97105	0.028499	\\
1.98137	0.0289067	\\
1.9862	0.0290978	\\
1.99367	0.029393	\\
2.00146	0.0297008	\\
2.00946	0.0300171	\\
2.01415	0.0302025	\\
2.02919	0.0307966	\\
2.03299	0.0309469	\\
2.04513	0.0314264	\\
2.06157	0.0320758	\\
2.07762	0.0327093	\\
2.08203	0.0328829	\\
2.09592	0.0334306	\\
2.09902	0.0335529	\\
2.1106	0.034009	\\
2.11849	0.0343198	\\
2.12568	0.0346027	\\
2.13566	0.0349949	\\
2.14361	0.0353075	\\
2.15612	0.0357986	\\
2.16166	0.0360158	\\
2.17471	0.0365275	\\
2.18536	0.0369446	\\
2.19598	0.0373599	\\
2.20653	0.0377724	\\
2.22139	0.0383519	\\
2.22653	0.0385523	\\
2.23919	0.0390456	\\
2.24707	0.0393518	\\
2.25385	0.0396152	\\
2.26292	0.0399674	\\
2.27041	0.0402577	\\
2.27992	0.0406262	\\
2.29442	0.0411868	\\
2.30608	0.0416369	\\
2.31854	0.0421174	\\
2.3339	0.0427081	\\
2.35015	0.0433315	\\
2.35749	0.0436126	\\
2.3632	0.0438308	\\
2.36934	0.0440656	\\
2.38516	0.0446691	\\
2.39573	0.0450719	\\
2.40215	0.0453157	\\
2.41704	0.0458809	\\
2.43092	0.0464061	\\
2.44218	0.0468315	\\
2.45123	0.0471729	\\
2.46866	0.0478285	\\
2.47758	0.0481633	\\
2.4849	0.0484373	\\
2.49979	0.0489941	\\
2.5111	0.0494159	\\
2.52495	0.0499314	\\
2.53541	0.0503197	\\
2.54724	0.0507576	\\
2.55612	0.0510858	\\
2.57322	0.0517163	\\
2.58161	0.0520248	\\
2.59033	0.0523449	\\
2.60481	0.0528749	\\
2.61999	0.0534291	\\
2.63512	0.0539793	\\
2.64529	0.0543485	\\
2.65716	0.0547781	\\
2.66849	0.0551874	\\
2.68218	0.0556805	\\
2.69601	0.0561767	\\
2.70992	0.0566746	\\
2.71965	0.0570222	\\
2.73612	0.0576083	\\
2.7522	0.0581784	\\
2.77408	0.0589511	\\
2.79011	0.0595146	\\
2.80601	0.0600714	\\
2.82521	0.0607409	\\
2.83653	0.0611344	\\
2.85437	0.0617525	\\
2.86762	0.0622095	\\
2.88104	0.0626712	\\
2.88872	0.0629344	\\
2.90365	0.0634455	\\
2.91603	0.0638676	\\
2.93079	0.0643692	\\
2.94329	0.0647925	\\
2.96475	0.0655166	\\
2.97502	0.0658614	\\
2.98864	0.0663176	\\
3.00315	0.0668021	\\
3.01437	0.0671754	\\
3.02547	0.0675438	\\
3.03958	0.0680106	\\
3.05821	0.0686242	\\
3.07135	0.0690554	\\
3.07922	0.0693129	\\
3.09105	0.0696989	\\
3.09501	0.0698281	\\
3.11529	0.0704864	\\
3.12956	0.0709478	\\
3.1364	0.0711683	\\
3.14312	0.0713848	\\
3.15484	0.0717611	\\
3.16956	0.0722322	\\
3.18858	0.0728381	\\
3.206	0.0733905	\\
3.21873	0.0737926	\\
3.23985	0.0744567	\\
3.2588	0.0750495	\\
3.27349	0.0755071	\\
3.29053	0.0760357	\\
3.30883	0.0766006	\\
3.32326	0.0770441	\\
3.34255	0.0776347	\\
3.35805	0.0781069	\\
3.37076	0.0784925	\\
3.38273	0.0788548	\\
3.39877	0.0793381	\\
3.40947	0.0796596	\\
3.42309	0.0800672	\\
3.44559	0.0807375	\\
3.4579	0.0811027	\\
3.47785	0.0816917	\\
3.49421	0.0821725	\\
3.50401	0.0824595	\\
3.51176	0.082686	\\
3.52382	0.0830372	\\
3.53656	0.0834074	\\
3.55629	0.0839778	\\
3.5692	0.0843496	\\
3.58819	0.0848939	\\
3.6019	0.0852852	\\
3.62605	0.0859711	\\
3.64093	0.0863914	\\
3.66119	0.0869612	\\
3.67574	0.0873683	\\
3.69356	0.087865	\\
3.70987	0.0883175	\\
3.7214	0.088636	\\
3.73114	0.0889045	\\
3.75029	0.0894302	\\
3.76804	0.0899149	\\
3.78452	0.0903631	\\
3.81006	0.0910538	\\
3.82486	0.0914518	\\
3.84733	0.0920529	\\
3.86475	0.0925167	\\
3.89033	0.0931936	\\
3.9009	0.093472	\\
3.92121	0.0940045	\\
3.93057	0.0942491	\\
3.94864	0.0947193	\\
3.96642	0.0951799	\\
3.98334	0.0956163	\\
3.9953	0.0959233	\\
4.0092	0.0962793	\\
4.01971	0.0965473	\\
4.04036	0.097072	\\
4.06313	0.0976473	\\
4.08001	0.0980713	\\
4.10293	0.0986443	\\
4.12102	0.0990941	\\
4.13415	0.099419	\\
4.15229	0.0998663	\\
4.17638	0.100457	\\
4.19415	0.10089	\\
4.20661	0.101193	\\
4.21883	0.101488	\\
4.24168	0.102039	\\
4.27087	0.102738	\\
4.2816	0.102993	\\
4.30361	0.103514	\\
4.33567	0.104269	\\
4.35606	0.104746	\\
4.36785	0.10502	\\
4.39144	0.105567	\\
4.41208	0.106042	\\
4.42695	0.106383	\\
4.43477	0.106561	\\
4.44879	0.106881	\\
4.472	0.107408	\\
4.51114	0.108288	\\
4.52652	0.108632	\\
4.54622	0.10907	\\
4.56374	0.109458	\\
4.58013	0.109819	\\
4.60286	0.110317	\\
4.62445	0.110788	\\
4.64708	0.111278	\\
4.67292	0.111835	\\
4.68998	0.1122	\\
4.71447	0.112722	\\
4.73037	0.113059	\\
4.74845	0.113441	\\
4.7763	0.114025	\\
4.81018	0.114731	\\
4.84206	0.115389	\\
4.8553	0.11566	\\
4.87923	0.11615	\\
4.8989	0.116549	\\
4.92062	0.116988	\\
4.94125	0.117403	\\
4.96225	0.117823	\\
4.97405	0.118058	\\
4.99798	0.118533	\\
5.01152	0.1188	\\
5.02698	0.119104	\\
5.04296	0.119417	\\
5.07515	0.120044	\\
5.10643	0.120648	\\
5.12456	0.120997	\\
5.13305	0.121159	\\
5.16698	0.121805	\\
5.1908	0.122255	\\
5.20705	0.122561	\\
5.24076	0.123191	\\
5.24932	0.12335	\\
5.27866	0.123894	\\
5.30212	0.124325	\\
5.32995	0.124834	\\
5.35464	0.125282	\\
5.38854	0.125894	\\
5.41724	0.126408	\\
5.436	0.126741	\\
5.48011	0.12752	\\
5.51535	0.128137	\\
5.54338	0.128624	\\
5.55911	0.128895	\\
5.58382	0.12932	\\
5.62636	0.130046	\\
5.64758	0.130405	\\
5.67398	0.130849	\\
5.70173	0.131313	\\
5.74764	0.132075	\\
5.77259	0.132485	\\
5.7944	0.132841	\\
5.8163	0.133198	\\
5.84153	0.133606	\\
5.87032	0.134069	\\
5.89715	0.134498	\\
5.92661	0.134965	\\
5.96002	0.135492	\\
5.988	0.13593	\\
6.0035	0.136171	\\
6.02802	0.136551	\\
6.07394	0.137257	\\
6.09769	0.13762	\\
6.12448	0.138026	\\
6.15224	0.138444	\\
6.19105	0.139025	\\
6.22736	0.139564	\\
6.25917	0.140032	\\
6.31269	0.140813	\\
6.35032	0.141356	\\
6.39267	0.141962	\\
6.40727	0.14217	\\
6.46447	0.142977	\\
6.48122	0.143211	\\
6.519	0.143737	\\
6.53461	0.143953	\\
6.5647	0.144367	\\
6.5933	0.144758	\\
6.64122	0.145408	\\
6.68137	0.145947	\\
6.70655	0.146283	\\
6.73582	0.146671	\\
6.76645	0.147075	\\
6.78333	0.147297	\\
6.80961	0.147639	\\
6.85015	0.148165	\\
6.87144	0.148439	\\
6.89538	0.148746	\\
6.9168	0.149019	\\
6.95005	0.149441	\\
6.99506	0.150007	\\
7.02438	0.150373	\\
7.04214	0.150593	\\
7.07241	0.150967	\\
7.10173	0.151327	\\
7.1255	0.151618	\\
7.17333	0.152198	\\
7.20271	0.152551	\\
7.21908	0.152747	\\
7.26382	0.153279	\\
7.29549	0.153653	\\
7.32505	0.154	\\
7.3725	0.154553	\\
7.40642	0.154944	\\
7.45068	0.155451	\\
7.49681	0.155975	\\
7.54219	0.156486	\\
7.58626	0.156977	\\
7.60773	0.157215	\\
7.63563	0.157523	\\
7.67934	0.158001	\\
7.7144	0.158382	\\
7.74902	0.158756	\\
7.81312	0.159441	\\
7.84339	0.159762	\\
7.87181	0.160061	\\
7.89968	0.160353	\\
7.94355	0.160809	\\
7.99478	0.161337	\\
8.0338	0.161736	\\
8.05909	0.161993	\\
8.08868	0.162292	\\
8.14235	0.16283	\\
8.18281	0.163232	\\
8.22709	0.163668	\\
8.25756	0.163967	\\
8.29441	0.164325	\\
8.33781	0.164744	\\
8.3573	0.164931	\\
8.38705	0.165216	\\
8.43292	0.165651	\\
8.45956	0.165902	\\
8.47307	0.166029	\\
8.5	0.166281	\\
8.52038	0.166471	\\
8.56203	0.166857	\\
8.59445	0.167155	\\
8.62651	0.167448	\\
8.67078	0.167851	\\
8.71021	0.168206	\\
8.74704	0.168536	\\
8.78105	0.168839	\\
8.81942	0.169178	\\
8.84083	0.169367	\\
8.88987	0.169795	\\
8.91925	0.17005	\\
8.98304	0.1706	\\
9.02012	0.170916	\\
9.06756	0.171318	\\
9.10711	0.171651	\\
9.13123	0.171852	\\
9.16562	0.172138	\\
9.18945	0.172336	\\
9.24534	0.172795	\\
9.30029	0.173243	\\
9.33588	0.17353	\\
9.37337	0.173831	\\
9.43709	0.174339	\\
9.47973	0.174675	\\
9.54244	0.175166	\\
9.57435	0.175413	\\
9.61281	0.17571	\\
9.64683	0.175971	\\
9.6953	0.17634	\\
9.77374	0.176931	\\
9.81293	0.177223	\\
9.8489	0.17749	\\
9.90572	0.177908	\\
9.932	0.178101	\\
};
\addlegendentry{$\Delta_{N}(\sigma)$};


\addplot [color=matlab2,densely dotted]
  table[row sep=crcr]{
0.100015	0.981815	\\
0.100336	0.977692	\\
0.100561	0.974817	\\
0.101092	0.968087	\\
0.101691	0.960581	\\
0.102174	0.954593	\\
0.10289	0.945825	\\
0.103285	0.941036	\\
0.103801	0.934853	\\
0.104376	0.928021	\\
0.104902	0.921842	\\
0.10552	0.914675	\\
0.106405	0.904547	\\
0.106778	0.900326	\\
0.107259	0.894939	\\
0.107686	0.890193	\\
0.107963	0.88714	\\
0.108685	0.87924	\\
0.109214	0.873533	\\
0.109707	0.868257	\\
0.110134	0.863728	\\
0.111103	0.853583	\\
0.111734	0.847084	\\
0.112186	0.842472	\\
0.112688	0.837392	\\
0.113033	0.833931	\\
0.113665	0.827643	\\
0.114368	0.820727	\\
0.11488	0.815748	\\
0.115329	0.811427	\\
0.115513	0.80966	\\
0.116102	0.804061	\\
0.116665	0.79876	\\
0.116962	0.795983	\\
0.117579	0.790258	\\
0.11802	0.786211	\\
0.118636	0.780611	\\
0.119443	0.773361	\\
0.119963	0.768734	\\
0.120924	0.760314	\\
0.121181	0.758086	\\
0.122195	0.74938	\\
0.122727	0.744879	\\
0.12329	0.740149	\\
0.123732	0.736472	\\
0.12409	0.73352	\\
0.124442	0.730628	\\
0.125119	0.725115	\\
0.125727	0.720212	\\
0.126295	0.715686	\\
0.127179	0.708718	\\
0.127657	0.704989	\\
0.128238	0.7005	\\
0.128554	0.698073	\\
0.129049	0.694306	\\
0.129915	0.68778	\\
0.130325	0.684717	\\
0.130921	0.680311	\\
0.131481	0.676205	\\
0.132223	0.670825	\\
0.132941	0.665673	\\
0.133901	0.658881	\\
0.134572	0.654195	\\
0.135143	0.650244	\\
0.135771	0.645935	\\
0.136557	0.640611	\\
0.137199	0.636304	\\
0.137877	0.631804	\\
0.138381	0.628489	\\
0.139077	0.623953	\\
0.139834	0.619077	\\
0.140459	0.615084	\\
0.141017	0.611561	\\
0.141424	0.609004	\\
0.142416	0.602839	\\
0.143505	0.596183	\\
0.144129	0.592413	\\
0.144945	0.587534	\\
0.145646	0.58339	\\
0.146226	0.579997	\\
0.146843	0.576415	\\
0.147435	0.573007	\\
0.148121	0.569096	\\
0.148733	0.565637	\\
0.1495	0.561351	\\
0.150498	0.555835	\\
0.151159	0.552226	\\
0.151942	0.547993	\\
0.152293	0.546113	\\
0.153102	0.54181	\\
0.154286	0.5356	\\
0.154867	0.532591	\\
0.155517	0.529253	\\
0.156005	0.526767	\\
0.156597	0.523769	\\
0.157325	0.520122	\\
0.158119	0.516178	\\
0.159341	0.510198	\\
0.159964	0.507182	\\
0.160553	0.504357	\\
0.161612	0.499334	\\
0.162586	0.494774	\\
0.163148	0.492175	\\
0.163583	0.490173	\\
0.164439	0.486264	\\
0.165021	0.483633	\\
0.165753	0.480351	\\
0.166827	0.475595	\\
0.167842	0.471157	\\
0.168846	0.466824	\\
0.170115	0.461427	\\
0.170973	0.457828	\\
0.171892	0.454012	\\
0.172206	0.45272	\\
0.173045	0.449293	\\
0.173717	0.446571	\\
0.174618	0.442957	\\
0.175431	0.439736	\\
0.175763	0.438425	\\
0.17639	0.435972	\\
0.176848	0.434191	\\
0.177562	0.431432	\\
0.178582	0.427535	\\
0.179392	0.424472	\\
0.180059	0.421976	\\
0.180937	0.418717	\\
0.181749	0.415737	\\
0.182615	0.412587	\\
0.183481	0.409468	\\
0.184193	0.406932	\\
0.184908	0.404404	\\
0.185697	0.401639	\\
0.186241	0.399748	\\
0.187253	0.39626	\\
0.188144	0.393226	\\
0.189283	0.389388	\\
0.189833	0.387554	\\
0.190917	0.383975	\\
0.191531	0.381967	\\
0.192787	0.377901	\\
0.193147	0.376748	\\
0.194284	0.373131	\\
0.195764	0.368495	\\
0.196783	0.365349	\\
0.197564	0.36296	\\
0.198592	0.359848	\\
0.199754	0.356372	\\
0.200825	0.353207	\\
0.201526	0.351159	\\
0.202452	0.348474	\\
0.203152	0.346461	\\
0.203499	0.34547	\\
0.204247	0.343345	\\
0.204955	0.34135	\\
0.205628	0.339467	\\
0.206233	0.337788	\\
0.206993	0.335692	\\
0.207526	0.33423	\\
0.208113	0.332634	\\
0.209284	0.329475	\\
0.211021	0.324864	\\
0.212628	0.32067	\\
0.213591	0.318191	\\
0.214658	0.315474	\\
0.215148	0.314236	\\
0.216583	0.310647	\\
0.217329	0.308804	\\
0.218106	0.306896	\\
0.218764	0.305294	\\
0.219373	0.30382	\\
0.220098	0.302076	\\
0.221431	0.298907	\\
0.221823	0.297983	\\
0.22244	0.296534	\\
0.223583	0.293877	\\
0.224506	0.291754	\\
0.225701	0.289033	\\
0.226956	0.286212	\\
0.227942	0.284019	\\
0.229057	0.281565	\\
0.230206	0.279064	\\
0.231229	0.276861	\\
0.232482	0.274194	\\
0.234038	0.270926	\\
0.234983	0.268965	\\
0.236193	0.26648	\\
0.237082	0.264674	\\
0.238066	0.262691	\\
0.238912	0.261002	\\
0.239986	0.258877	\\
0.241948	0.25505	\\
0.242831	0.25335	\\
0.243463	0.252143	\\
0.244878	0.249466	\\
0.245524	0.248255	\\
0.246599	0.246256	\\
0.24792	0.243827	\\
0.24935	0.241231	\\
0.250667	0.238872	\\
0.251826	0.236817	\\
0.252669	0.235339	\\
0.253519	0.233857	\\
0.254469	0.232216	\\
0.25498	0.231339	\\
0.25552	0.230416	\\
0.256499	0.228755	\\
0.257579	0.226939	\\
0.258834	0.224852	\\
0.260244	0.222535	\\
0.261773	0.220053	\\
0.262476	0.218925	\\
0.263972	0.216545	\\
0.265131	0.214724	\\
0.266681	0.212316	\\
0.267848	0.210525	\\
0.269029	0.20873	\\
0.270687	0.206243	\\
0.272064	0.204204	\\
0.273021	0.2028	\\
0.274058	0.201294	\\
0.274715	0.200345	\\
0.275995	0.198515	\\
0.277002	0.197087	\\
0.279084	0.194175	\\
0.280201	0.192634	\\
0.280852	0.191743	\\
0.281772	0.190492	\\
0.282503	0.189504	\\
0.283784	0.187788	\\
0.284679	0.186599	\\
0.285512	0.185502	\\
0.286971	0.183598	\\
0.287939	0.182347	\\
0.290064	0.179637	\\
0.291414	0.177939	\\
0.293254	0.175655	\\
0.294542	0.174077	\\
0.296175	0.172099	\\
0.297304	0.170747	\\
0.298542	0.169279	\\
0.299978	0.167595	\\
0.301643	0.165665	\\
0.303399	0.163659	\\
0.304693	0.162198	\\
0.306072	0.160658	\\
0.307168	0.159446	\\
0.308907	0.157544	\\
0.310409	0.155921	\\
0.311453	0.154805	\\
0.312886	0.153288	\\
0.314384	0.15172	\\
0.315802	0.150253	\\
0.316826	0.149204	\\
0.318472	0.147534	\\
0.319978	0.146024	\\
0.320952	0.145058	\\
0.322137	0.143891	\\
0.323487	0.142575	\\
0.32546	0.140675	\\
0.326333	0.139843	\\
0.327803	0.138456	\\
0.328939	0.137395	\\
0.330106	0.136314	\\
0.331217	0.135293	\\
0.332388	0.134227	\\
0.334033	0.132745	\\
0.336997	0.13012	\\
0.337996	0.129248	\\
0.339107	0.128286	\\
0.34047	0.127117	\\
0.341418	0.12631	\\
0.343424	0.124623	\\
0.345429	0.12296	\\
0.346538	0.122051	\\
0.347824	0.121006	\\
0.349127	0.119957	\\
0.351381	0.118167	\\
0.352489	0.117296	\\
0.353752	0.116314	\\
0.355398	0.115046	\\
0.357443	0.113493	\\
0.358614	0.112613	\\
0.360395	0.111289	\\
0.362943	0.109424	\\
0.364721	0.108143	\\
0.367066	0.106477	\\
0.368198	0.105683	\\
0.370096	0.104366	\\
0.371913	0.103121	\\
0.374146	0.101612	\\
0.375713	0.100568	\\
0.378326	0.0988509	\\
0.379935	0.0978096	\\
0.381667	0.0967015	\\
0.383385	0.0956151	\\
0.385675	0.0941867	\\
0.387083	0.0933198	\\
0.38852	0.0924433	\\
0.391974	0.0903715	\\
0.39389	0.0892433	\\
0.395107	0.0885342	\\
0.396493	0.0877341	\\
0.397647	0.0870731	\\
0.399095	0.0862515	\\
0.400141	0.0856629	\\
0.401839	0.0847158	\\
0.403771	0.0836512	\\
0.405512	0.0827042	\\
0.407462	0.0816557	\\
0.409876	0.0803773	\\
0.411737	0.0794052	\\
0.413276	0.0786106	\\
0.414509	0.0779794	\\
0.416468	0.0769874	\\
0.417323	0.0765586	\\
0.418981	0.0757337	\\
0.420591	0.0749416	\\
0.421779	0.0743621	\\
0.423482	0.0735392	\\
0.425064	0.0727834	\\
0.426785	0.0719694	\\
0.428318	0.0712526	\\
0.431548	0.0697644	\\
0.434767	0.0683125	\\
0.437359	0.067165	\\
0.439772	0.0661138	\\
0.441974	0.0651685	\\
0.44365	0.064458	\\
0.445904	0.0635147	\\
0.447634	0.0627995	\\
0.450898	0.0614718	\\
0.452192	0.0609531	\\
0.454729	0.0599481	\\
0.456512	0.0592515	\\
0.459154	0.0582336	\\
0.461828	0.0572206	\\
0.463977	0.0564185	\\
0.466542	0.0554757	\\
0.469142	0.0545349	\\
0.47098	0.0538794	\\
0.472419	0.0533712	\\
0.473777	0.0528959	\\
0.475743	0.0522148	\\
0.477949	0.0514605	\\
0.480112	0.0507307	\\
0.4835	0.0496073	\\
0.485685	0.0488951	\\
0.488222	0.0480802	\\
0.490474	0.0473676	\\
0.491845	0.0469387	\\
0.494792	0.0460285	\\
0.497299	0.0452672	\\
0.49979	0.0445224	\\
0.501809	0.0439269	\\
0.503775	0.0433541	\\
0.50673	0.042506	\\
0.508105	0.0421165	\\
0.511606	0.0411395	\\
0.515601	0.0400499	\\
0.519691	0.0389614	\\
0.521657	0.0384477	\\
0.523934	0.0378605	\\
0.525638	0.0374262	\\
0.527003	0.0370816	\\
0.529262	0.0365174	\\
0.531221	0.0360345	\\
0.53368	0.0354362	\\
0.536445	0.0347742	\\
0.538675	0.0342481	\\
0.541318	0.0336337	\\
0.543193	0.0332039	\\
0.544877	0.0328218	\\
0.548326	0.0320512	\\
0.551087	0.031446	\\
0.552196	0.0312057	\\
0.554199	0.0307757	\\
0.5554	0.0305203	\\
0.558038	0.0299658	\\
0.559977	0.0295638	\\
0.563032	0.0289399	\\
0.566192	0.0283065	\\
0.568973	0.0277589	\\
0.57462	0.0266745	\\
0.577324	0.0261682	\\
0.581472	0.0254074	\\
0.583344	0.02507	\\
0.586459	0.0245173	\\
0.588681	0.0241292	\\
0.592427	0.0234865	\\
0.595686	0.0229391	\\
0.596761	0.0227608	\\
0.600326	0.0221778	\\
0.601606	0.0219716	\\
0.604555	0.0215023	\\
0.607517	0.0210392	\\
0.610228	0.0206227	\\
0.612805	0.0202329	\\
0.614009	0.020053	\\
0.617021	0.0196082	\\
0.619689	0.0192211	\\
0.62227	0.0188523	\\
0.626615	0.0182446	\\
0.629554	0.0178424	\\
0.632247	0.0174801	\\
0.638017	0.0167237	\\
0.639473	0.016537	\\
0.643873	0.0159826	\\
0.648235	0.0154476	\\
0.651095	0.0151045	\\
0.654668	0.0146842	\\
0.657656	0.0143397	\\
0.659734	0.0141039	\\
0.663918	0.0136381	\\
0.666508	0.0133557	\\
0.670846	0.0128929	\\
0.674249	0.0125384	\\
0.677272	0.0122298	\\
0.68107	0.0118502	\\
0.683865	0.0115765	\\
0.686305	0.0113416	\\
0.691086	0.0108916	\\
0.692885	0.0107257	\\
0.695898	0.0104521	\\
0.69955	0.0101276	\\
0.702505	0.00987037	\\
0.705434	0.00962023	\\
0.706966	0.00949132	\\
0.710002	0.00923952	\\
0.71249	0.00903679	\\
0.718108	0.00859096	\\
0.722177	0.00827808	\\
0.724212	0.00812474	\\
0.727772	0.00786138	\\
0.730907	0.00763455	\\
0.733312	0.00746377	\\
0.737613	0.00716505	\\
0.743549	0.00676679	\\
0.746625	0.0065666	\\
0.748683	0.00643503	\\
0.750398	0.00632679	\\
0.754213	0.00609061	\\
0.757265	0.00590607	\\
0.760278	0.00572769	\\
0.762993	0.0055702	\\
0.768141	0.00527972	\\
0.771412	0.00510057	\\
0.774243	0.00494895	\\
0.778214	0.00474138	\\
0.781509	0.00457367	\\
0.785039	0.00439847	\\
0.789806	0.00416901	\\
0.793167	0.00401213	\\
0.796088	0.00387898	\\
0.80221	0.00360943	\\
0.805098	0.00348661	\\
0.80767	0.0033796	\\
0.809874	0.00328959	\\
0.812891	0.00316891	\\
0.816306	0.00303579	\\
0.818988	0.00293382	\\
0.821809	0.00282894	\\
0.824666	0.00272517	\\
0.828553	0.00258796	\\
0.832199	0.00246329	\\
0.834358	0.00239126	\\
0.837803	0.00227912	\\
0.844224	0.00207901	\\
0.848839	0.00194215	\\
0.854473	0.00178277	\\
0.858483	0.00167439	\\
0.865074	0.00150512	\\
0.869705	0.00139264	\\
0.873873	0.00129584	\\
0.879156	0.0011791	\\
0.882406	0.00111049	\\
0.885564	0.00104614	\\
0.889228	0.000974295	\\
0.891719	0.000927144	\\
0.893937	0.000886302	\\
0.898794	0.000800575	\\
0.905655	0.00068796	\\
0.909293	0.000632167	\\
0.914879	0.000551689	\\
0.91945	0.000490412	\\
0.922132	0.000456323	\\
0.924317	0.000429567	\\
0.930693	0.000356587	\\
0.935614	0.000305347	\\
0.940388	0.000259779	\\
0.943663	0.000230822	\\
0.947772	0.000197106	\\
0.955624	0.000140556	\\
0.958188	0.000124289	\\
0.965368	8.43261e-05	\\
0.969607	6.45225e-05	\\
0.974504	4.50638e-05	\\
0.980515	2.60809e-05	\\
0.98462	1.61466e-05	\\
0.98935	7.68703e-06	\\
0.99532	1.47133e-06	\\
0.998514	1.47566e-07	\\
1.00389	1.00561e-06	\\
1.00875	5.04129e-06	\\
1.01152	8.69719e-06	\\
1.01834	2.18378e-05	\\
1.02222	3.18484e-05	\\
1.02639	4.46722e-05	\\
1.02968	5.6221e-05	\\
1.03472	7.63631e-05	\\
1.03708	8.67945e-05	\\
1.0432	0.000116817	\\
1.04803	0.000143347	\\
1.05248	0.000170078	\\
1.05556	0.000189767	\\
1.06137	0.00022962	\\
1.06618	0.000265246	\\
1.07213	0.000312414	\\
1.07635	0.000347952	\\
1.08382	0.000414977	\\
1.08761	0.000450914	\\
1.09427	0.000517317	\\
1.10094	0.000587663	\\
1.10428	0.000624283	\\
1.11107	0.00070166	\\
1.11445	0.000741541	\\
1.1179	0.000783145	\\
1.12264	0.000841949	\\
1.12783	0.000908299	\\
1.13184	0.000960935	\\
1.13883	0.00105551	\\
1.14441	0.00113354	\\
1.14871	0.00119518	\\
1.15328	0.00126211	\\
1.15726	0.00132138	\\
1.1602	0.00136597	\\
1.16443	0.00143109	\\
1.16996	0.00151787	\\
1.17577	0.00161097	\\
1.17954	0.00167261	\\
1.1832	0.00173317	\\
1.18791	0.00181233	\\
1.1929	0.00189757	\\
1.19916	0.00200635	\\
1.20767	0.00215778	\\
1.21307	0.00225583	\\
1.21808	0.00234823	\\
1.22241	0.00242886	\\
1.2294	0.00256127	\\
1.23769	0.00272126	\\
1.24156	0.00279687	\\
1.24652	0.00289496	\\
1.25395	0.00304382	\\
1.26076	0.00318237	\\
1.26748	0.00332087	\\
1.27252	0.00342603	\\
1.27741	0.0035288	\\
1.28349	0.00365785	\\
1.2914	0.00382802	\\
1.29571	0.0039215	\\
1.30192	0.00405756	\\
1.30717	0.00417352	\\
1.31215	0.00428429	\\
1.31742	0.00440253	\\
1.32457	0.00456411	\\
1.32922	0.00467012	\\
1.33684	0.00484506	\\
1.34381	0.00500645	\\
1.34972	0.00514449	\\
1.35563	0.00528339	\\
1.36125	0.00541622	\\
1.36887	0.00559754	\\
1.37549	0.00575625	\\
1.38533	0.00599399	\\
1.39279	0.00617577	\\
1.39763	0.00629422	\\
1.4055	0.00648786	\\
1.41028	0.00660626	\\
1.41439	0.00670813	\\
1.41835	0.00680678	\\
1.42524	0.00697892	\\
1.42924	0.00707925	\\
1.43495	0.00722299	\\
1.44057	0.00736479	\\
1.44784	0.00754927	\\
1.46077	0.00787909	\\
1.46666	0.00803026	\\
1.48032	0.00838216	\\
1.4846	0.00849298	\\
1.49294	0.00870964	\\
1.49869	0.00885947	\\
1.50401	0.00899819	\\
1.51435	0.00926885	\\
1.52143	0.00945506	\\
1.53151	0.00972051	\\
1.53696	0.00986451	\\
1.54633	0.0101127	\\
1.54953	0.0101974	\\
1.55879	0.0104434	\\
1.56552	0.0106225	\\
1.57553	0.0108896	\\
1.58196	0.0110613	\\
1.5874	0.011207	\\
1.59513	0.011414	\\
1.59983	0.01154	\\
1.60538	0.011689	\\
1.61302	0.0118942	\\
1.61949	0.0120682	\\
1.62707	0.0122723	\\
1.63344	0.0124437	\\
1.64378	0.0127227	\\
1.6555	0.013039	\\
1.66625	0.0133291	\\
1.67451	0.0135524	\\
1.68825	0.0139239	\\
1.69695	0.0141589	\\
1.7046	0.0143658	\\
1.7117	0.0145578	\\
1.72369	0.014882	\\
1.72862	0.0150154	\\
1.73709	0.0152441	\\
1.7441	0.0154335	\\
1.75046	0.0156053	\\
1.75613	0.0157585	\\
1.76015	0.0158668	\\
1.76765	0.0160693	\\
1.77208	0.0161887	\\
1.78519	0.0165419	\\
1.79739	0.0168702	\\
1.80889	0.0171795	\\
1.81709	0.0173995	\\
1.82862	0.0177088	\\
1.83824	0.0179665	\\
1.84642	0.0181852	\\
1.85506	0.0184161	\\
1.86419	0.0186596	\\
1.87	0.0188146	\\
1.87543	0.0189592	\\
1.88149	0.0191203	\\
1.88978	0.0193406	\\
1.89788	0.0195555	\\
1.90767	0.0198146	\\
1.91344	0.0199674	\\
1.92585	0.020295	\\
1.93516	0.0205401	\\
1.94096	0.0206927	\\
1.95279	0.0210033	\\
1.96239	0.0212549	\\
1.97105	0.0214811	\\
1.98137	0.0217503	\\
1.9862	0.0218762	\\
1.99367	0.0220704	\\
2.00146	0.0222726	\\
2.00946	0.0224799	\\
2.01415	0.0226013	\\
2.02919	0.0229892	\\
2.03299	0.0230871	\\
2.04513	0.023399	\\
2.06157	0.02382	\\
2.07762	0.0242291	\\
2.08203	0.024341	\\
2.09592	0.0246932	\\
2.09902	0.0247718	\\
2.1106	0.0250641	\\
2.11849	0.0252628	\\
2.12568	0.0254434	\\
2.13566	0.0256934	\\
2.14361	0.0258922	\\
2.15612	0.026204	\\
2.16166	0.0263416	\\
2.17471	0.0266651	\\
2.18536	0.0269282	\\
2.19598	0.0271896	\\
2.20653	0.0274487	\\
2.22139	0.0278118	\\
2.22653	0.027937	\\
2.23919	0.0282449	\\
2.24707	0.0284356	\\
2.25385	0.0285994	\\
2.26292	0.0288182	\\
2.27041	0.0289982	\\
2.27992	0.0292263	\\
2.29442	0.0295726	\\
2.30608	0.0298499	\\
2.31854	0.0301452	\\
2.3339	0.0305074	\\
2.35015	0.0308886	\\
2.35749	0.0310601	\\
2.3632	0.031193	\\
2.36934	0.031336	\\
2.38516	0.0317027	\\
2.39573	0.0319468	\\
2.40215	0.0320944	\\
2.41704	0.0324358	\\
2.43092	0.0327524	\\
2.44218	0.0330082	\\
2.45123	0.0332131	\\
2.46866	0.0336059	\\
2.47758	0.033806	\\
2.4849	0.0339696	\\
2.49979	0.0343014	\\
2.5111	0.0345522	\\
2.52495	0.0348581	\\
2.53541	0.035088	\\
2.54724	0.0353469	\\
2.55612	0.0355407	\\
2.57322	0.0359121	\\
2.58161	0.0360935	\\
2.59033	0.0362814	\\
2.60481	0.0365921	\\
2.61999	0.0369162	\\
2.63512	0.0372373	\\
2.64529	0.0374523	\\
2.65716	0.0377021	\\
2.66849	0.0379397	\\
2.68218	0.0382255	\\
2.69601	0.0385125	\\
2.70992	0.0387999	\\
2.71965	0.0390002	\\
2.73612	0.0393373	\\
2.7522	0.0396645	\\
2.77408	0.0401069	\\
2.79011	0.0404286	\\
2.80601	0.0407459	\\
2.82521	0.0411265	\\
2.83653	0.0413498	\\
2.85437	0.0416998	\\
2.86762	0.041958	\\
2.88104	0.0422185	\\
2.88872	0.0423669	\\
2.90365	0.0426544	\\
2.91603	0.0428916	\\
2.93079	0.0431728	\\
2.94329	0.0434098	\\
2.96475	0.0438144	\\
2.97502	0.0440067	\\
2.98864	0.0442608	\\
3.00315	0.0445301	\\
3.01437	0.0447373	\\
3.02547	0.0449416	\\
3.03958	0.0452	\\
3.05821	0.0455391	\\
3.07135	0.0457769	\\
3.07922	0.0459188	\\
3.09105	0.0461312	\\
3.09501	0.0462023	\\
3.11529	0.0465638	\\
3.12956	0.0468167	\\
3.1364	0.0469374	\\
3.14312	0.0470558	\\
3.15484	0.0472615	\\
3.16956	0.0475186	\\
3.18858	0.0478487	\\
3.206	0.0481491	\\
3.21873	0.0483674	\\
3.23985	0.0487273	\\
3.2588	0.0490479	\\
3.27349	0.049295	\\
3.29053	0.04958	\\
3.30883	0.049884	\\
3.32326	0.0501223	\\
3.34255	0.0504391	\\
3.35805	0.0506919	\\
3.37076	0.0508981	\\
3.38273	0.0510917	\\
3.39877	0.0513495	\\
3.40947	0.0515208	\\
3.42309	0.0517377	\\
3.44559	0.0520939	\\
3.4579	0.0522877	\\
3.47785	0.0525996	\\
3.49421	0.0528539	\\
3.50401	0.0530056	\\
3.51176	0.0531251	\\
3.52382	0.0533103	\\
3.53656	0.0535053	\\
3.55629	0.0538054	\\
3.5692	0.0540007	\\
3.58819	0.0542863	\\
3.6019	0.0544913	\\
3.62605	0.0548501	\\
3.64093	0.0550696	\\
3.66119	0.0553667	\\
3.67574	0.0555787	\\
3.69356	0.055837	\\
3.70987	0.056072	\\
3.7214	0.0562372	\\
3.73114	0.0563764	\\
3.75029	0.0566486	\\
3.76804	0.0568992	\\
3.78452	0.0571306	\\
3.81006	0.0574866	\\
3.82486	0.0576915	\\
3.84733	0.0580004	\\
3.86475	0.0582385	\\
3.89033	0.0585853	\\
3.9009	0.0587278	\\
3.92121	0.059	\\
3.93057	0.0591249	\\
3.94864	0.0593648	\\
3.96642	0.0595995	\\
3.98334	0.0598215	\\
3.9953	0.0599776	\\
4.0092	0.0601584	\\
4.01971	0.0602944	\\
4.04036	0.0605604	\\
4.06313	0.0608516	\\
4.08001	0.0610659	\\
4.10293	0.0613552	\\
4.12102	0.0615819	\\
4.13415	0.0617456	\\
4.15229	0.0619706	\\
4.17638	0.0622673	\\
4.19415	0.0624847	\\
4.20661	0.0626364	\\
4.21883	0.0627845	\\
4.24168	0.0630601	\\
4.27087	0.063409	\\
4.2816	0.0635363	\\
4.30361	0.0637963	\\
4.33567	0.0641718	\\
4.35606	0.0644085	\\
4.36785	0.0645448	\\
4.39144	0.0648158	\\
4.41208	0.0650512	\\
4.42695	0.0652198	\\
4.43477	0.0653081	\\
4.44879	0.065466	\\
4.472	0.0657259	\\
4.51114	0.0661597	\\
4.52652	0.0663287	\\
4.54622	0.0665441	\\
4.56374	0.0667344	\\
4.58013	0.0669116	\\
4.60286	0.0671557	\\
4.62445	0.0673861	\\
4.64708	0.0676258	\\
4.67292	0.0678976	\\
4.68998	0.0680758	\\
4.71447	0.06833	\\
4.73037	0.0684941	\\
4.74845	0.0686796	\\
4.7763	0.0689634	\\
4.81018	0.0693055	\\
4.84206	0.0696241	\\
4.8553	0.0697555	\\
4.87923	0.0699918	\\
4.8989	0.0701847	\\
4.92062	0.0703965	\\
4.94125	0.0705963	\\
4.96225	0.0707984	\\
4.97405	0.0709114	\\
4.99798	0.0711394	\\
5.01152	0.0712677	\\
5.02698	0.0714137	\\
5.04296	0.0715637	\\
5.07515	0.0718639	\\
5.10643	0.0721529	\\
5.12456	0.0723192	\\
5.13305	0.0723968	\\
5.16698	0.0727049	\\
5.1908	0.0729194	\\
5.20705	0.0730649	\\
5.24076	0.0733646	\\
5.24932	0.0734402	\\
5.27866	0.073698	\\
5.30212	0.0739027	\\
5.32995	0.0741436	\\
5.35464	0.0743558	\\
5.38854	0.0746447	\\
5.41724	0.0748872	\\
5.436	0.0750446	\\
5.48011	0.0754116	\\
5.51535	0.0757015	\\
5.54338	0.0759301	\\
5.55911	0.0760576	\\
5.58382	0.0762568	\\
5.62636	0.0765966	\\
5.64758	0.0767647	\\
5.67398	0.0769723	\\
5.70173	0.077189	\\
5.74764	0.077544	\\
5.77259	0.077735	\\
5.7944	0.077901	\\
5.8163	0.0780667	\\
5.84153	0.0782564	\\
5.87032	0.0784713	\\
5.89715	0.0786701	\\
5.92661	0.0788868	\\
5.96002	0.0791305	\\
5.988	0.0793329	\\
6.0035	0.0794444	\\
6.02802	0.0796199	\\
6.07394	0.0799456	\\
6.09769	0.0801125	\\
6.12448	0.0802996	\\
6.15224	0.0804921	\\
6.19105	0.0807589	\\
6.22736	0.0810061	\\
6.25917	0.0812208	\\
6.31269	0.0815783	\\
6.35032	0.0818267	\\
6.39267	0.0821035	\\
6.40727	0.0821982	\\
6.46447	0.0825661	\\
6.48122	0.0826729	\\
6.519	0.082912	\\
6.53461	0.0830103	\\
6.5647	0.0831984	\\
6.5933	0.0833761	\\
6.64122	0.0836708	\\
6.68137	0.0839152	\\
6.70655	0.0840673	\\
6.73582	0.0842429	\\
6.76645	0.0844253	\\
6.78333	0.0845254	\\
6.80961	0.0846802	\\
6.85015	0.0849172	\\
6.87144	0.0850407	\\
6.89538	0.0851789	\\
6.9168	0.0853019	\\
6.95005	0.0854916	\\
6.99506	0.0857461	\\
7.02438	0.0859104	\\
7.04214	0.0860093	\\
7.07241	0.0861771	\\
7.10173	0.0863385	\\
7.1255	0.0864686	\\
7.17333	0.0867281	\\
7.20271	0.0868862	\\
7.21908	0.0869738	\\
7.26382	0.0872115	\\
7.29549	0.0873784	\\
7.32505	0.0875331	\\
7.3725	0.0877792	\\
7.40642	0.0879536	\\
7.45068	0.088179	\\
7.49681	0.0884118	\\
7.54219	0.0886384	\\
7.58626	0.0888564	\\
7.60773	0.0889618	\\
7.63563	0.089098	\\
7.67934	0.0893099	\\
7.7144	0.0894783	\\
7.74902	0.0896434	\\
7.81312	0.0899459	\\
7.84339	0.0900873	\\
7.87181	0.0902192	\\
7.89968	0.0903478	\\
7.94355	0.0905488	\\
7.99478	0.090781	\\
8.0338	0.0909563	\\
8.05909	0.0910691	\\
8.08868	0.0912004	\\
8.14235	0.0914364	\\
8.18281	0.0916126	\\
8.22709	0.0918038	\\
8.25756	0.0919344	\\
8.29441	0.0920912	\\
8.33781	0.0922744	\\
8.3573	0.0923561	\\
8.38705	0.0924802	\\
8.43292	0.0926702	\\
8.45956	0.0927798	\\
8.47307	0.0928351	\\
8.5	0.0929449	\\
8.52038	0.0930276	\\
8.56203	0.0931957	\\
8.59445	0.0933255	\\
8.62651	0.0934531	\\
8.67078	0.093628	\\
8.71021	0.0937824	\\
8.74704	0.0939256	\\
8.78105	0.094057	\\
8.81942	0.0942041	\\
8.84083	0.0942857	\\
8.88987	0.0944714	\\
8.91925	0.0945818	\\
8.98304	0.0948195	\\
9.02012	0.0949563	\\
9.06756	0.0951299	\\
9.10711	0.0952735	\\
9.13123	0.0953605	\\
9.16562	0.0954839	\\
9.18945	0.0955689	\\
9.24534	0.0957669	\\
9.30029	0.0959596	\\
9.33588	0.0960834	\\
9.37337	0.0962128	\\
9.43709	0.0964309	\\
9.47973	0.0965754	\\
9.54244	0.0967858	\\
9.57435	0.096892	\\
9.61281	0.0970192	\\
9.64683	0.097131	\\
9.6953	0.097289	\\
9.77374	0.0975419	\\
9.81293	0.097667	\\
9.8489	0.097781	\\
9.90572	0.0979596	\\
9.932	0.0980416	\\
};
\addlegendentry{$\norm{u_{N}(\sigma) - u_{\mathcal N}(\sigma)}_{\mathcal X}$};

\end{axis}
\end{tikzpicture}%

        \end{subfigure}
        \hfill
        \begin{subfigure}[b]{0.45\textwidth}
            ~
            % This file was created by matlab2tikz v0.4.6 running on MATLAB 8.1.
% Copyright (c) 2008--2014, Nico Schlömer <nico.schloemer@gmail.com>
% All rights reserved.
% Minimal pgfplots version: 1.3
%
% The latest updates can be retrieved from
%   http://www.mathworks.com/matlabcentral/fileexchange/22022-matlab2tikz
% where you can also make suggestions and rate matlab2tikz.
%
\begin{tikzpicture}

\begin{axis}[%
width=10cm,
height=7cm,
scale only axis,
xmin=0,
xmax=10,
ymode=log,
ymin=1e-16,
ymax=1,
yminorticks=true,
ultra thick,
xlabel={One dimensional parameter space},
legend style={at={(1,0.03)},anchor=south east,legend cell align=left,align=left,fill=none,draw=none}
]
\addplot [color=matlab1,solid]
  table[row sep=crcr]{
0.100015	1.74311e-06	\\
0.100336	8.39316e-07	\\
0.100561	6.10431e-06	\\
0.101092	3.71296e-05	\\
0.101691	0.000102166	\\
0.102174	0.000176628	\\
0.10289	0.000321033	\\
0.103285	0.000417309	\\
0.103801	0.000559459	\\
0.104376	0.000739463	\\
0.104902	0.000922604	\\
0.10552	0.00115868	\\
0.106405	0.00153453	\\
0.106778	0.00170541	\\
0.107259	0.00193537	\\
0.107686	0.0021488	\\
0.107963	0.00229142	\\
0.108685	0.00267927	\\
0.109214	0.00297602	\\
0.109707	0.00326239	\\
0.110134	0.00351726	\\
0.111103	0.00411764	\\
0.111734	0.00452307	\\
0.112186	0.00482032	\\
0.112688	0.00515671	\\
0.113033	0.00539106	\\
0.113665	0.00582763	\\
0.114368	0.00632321	\\
0.11488	0.00668968	\\
0.115329	0.00701412	\\
0.115513	0.00714846	\\
0.116102	0.00758051	\\
0.116665	0.00799815	\\
0.116962	0.00822021	\\
0.117579	0.00868481	\\
0.11802	0.00901875	\\
0.118636	0.00948797	\\
0.119443	0.0101074	\\
0.119963	0.0105095	\\
0.120924	0.0112541	\\
0.121181	0.0114538	\\
0.122195	0.0122444	\\
0.122727	0.0126592	\\
0.12329	0.0130994	\\
0.123732	0.0134445	\\
0.12409	0.0137233	\\
0.124442	0.0139979	\\
0.125119	0.0145253	\\
0.125727	0.0149984	\\
0.126295	0.0154384	\\
0.127179	0.0161214	\\
0.127657	0.0164896	\\
0.128238	0.0169351	\\
0.128554	0.0171769	\\
0.129049	0.0175536	\\
0.129915	0.0182095	\\
0.130325	0.0185188	\\
0.130921	0.018965	\\
0.131481	0.0193824	\\
0.132223	0.019931	\\
0.132941	0.0204579	\\
0.133901	0.0211548	\\
0.134572	0.0216366	\\
0.135143	0.0220434	\\
0.135771	0.0224875	\\
0.136557	0.0230367	\\
0.137199	0.0234811	\\
0.137877	0.0239453	\\
0.138381	0.0242872	\\
0.139077	0.0247546	\\
0.139834	0.0252566	\\
0.140459	0.0256669	\\
0.141017	0.0260285	\\
0.141424	0.0262905	\\
0.142416	0.0269209	\\
0.143505	0.0275987	\\
0.144129	0.027981	\\
0.144945	0.0284741	\\
0.145646	0.0288912	\\
0.146226	0.0292313	\\
0.146843	0.0295891	\\
0.147435	0.0299281	\\
0.148121	0.0303155	\\
0.148733	0.0306563	\\
0.1495	0.0310765	\\
0.150498	0.0316133	\\
0.151159	0.0319619	\\
0.151942	0.0323681	\\
0.152293	0.0325476	\\
0.153102	0.032956	\\
0.154286	0.0335395	\\
0.154867	0.0338195	\\
0.155517	0.034128	\\
0.156005	0.0343562	\\
0.156597	0.0346298	\\
0.157325	0.03496	\\
0.158119	0.0353135	\\
0.159341	0.0358428	\\
0.159964	0.0361064	\\
0.160553	0.0363513	\\
0.161612	0.0367815	\\
0.162586	0.0371662	\\
0.163148	0.037383	\\
0.163583	0.0375487	\\
0.164439	0.0378689	\\
0.165021	0.0380819	\\
0.165753	0.0383447	\\
0.166827	0.0387197	\\
0.167842	0.0390632	\\
0.168846	0.0393924	\\
0.170115	0.0397936	\\
0.170973	0.0400557	\\
0.171892	0.0403286	\\
0.172206	0.0404199	\\
0.173045	0.0406589	\\
0.173717	0.0408456	\\
0.174618	0.0410893	\\
0.175431	0.0413023	\\
0.175763	0.0413879	\\
0.17639	0.0415461	\\
0.176848	0.0416595	\\
0.177562	0.0418327	\\
0.178582	0.042072	\\
0.179392	0.0422557	\\
0.180059	0.0424025	\\
0.180937	0.0425901	\\
0.181749	0.0427577	\\
0.182615	0.0429306	\\
0.183481	0.0430974	\\
0.184193	0.0432298	\\
0.184908	0.0433589	\\
0.185697	0.0434967	\\
0.186241	0.0435888	\\
0.187253	0.0437544	\\
0.188144	0.0438936	\\
0.189283	0.0440634	\\
0.189833	0.0441419	\\
0.190917	0.0442903	\\
0.191531	0.0443708	\\
0.192787	0.0445273	\\
0.193147	0.0445702	\\
0.194284	0.0447	\\
0.195764	0.0448563	\\
0.196783	0.0449558	\\
0.197564	0.0450277	\\
0.198592	0.0451166	\\
0.199754	0.0452095	\\
0.200825	0.0452882	\\
0.201526	0.045336	\\
0.202452	0.0453951	\\
0.203152	0.0454366	\\
0.203499	0.0454561	\\
0.204247	0.0454961	\\
0.204955	0.0455312	\\
0.205628	0.0455621	\\
0.206233	0.0455879	\\
0.206993	0.0456176	\\
0.207526	0.0456368	\\
0.208113	0.0456562	\\
0.209284	0.0456899	\\
0.211021	0.0457279	\\
0.212628	0.0457507	\\
0.213591	0.0457589	\\
0.214658	0.0457632	\\
0.215148	0.0457635	\\
0.216583	0.0457588	\\
0.217329	0.045753	\\
0.218106	0.0457446	\\
0.218764	0.0457357	\\
0.219373	0.0457259	\\
0.220098	0.0457124	\\
0.221431	0.0456825	\\
0.221823	0.0456724	\\
0.22244	0.0456555	\\
0.223583	0.0456206	\\
0.224506	0.0455891	\\
0.225701	0.0455441	\\
0.226956	0.0454917	\\
0.227942	0.045447	\\
0.229057	0.0453928	\\
0.230206	0.0453331	\\
0.231229	0.0452765	\\
0.232482	0.0452032	\\
0.234038	0.0451061	\\
0.234983	0.0450439	\\
0.236193	0.0449609	\\
0.237082	0.0448975	\\
0.238066	0.044825	\\
0.238912	0.0447608	\\
0.239986	0.0446769	\\
0.241948	0.0445167	\\
0.242831	0.0444418	\\
0.243463	0.0443872	\\
0.244878	0.0442619	\\
0.245524	0.0442033	\\
0.246599	0.0441039	\\
0.24792	0.0439788	\\
0.24935	0.0438396	\\
0.250667	0.0437082	\\
0.251826	0.04359	\\
0.252669	0.0435027	\\
0.253519	0.0434133	\\
0.254469	0.0433122	\\
0.25498	0.0432572	\\
0.25552	0.0431987	\\
0.256499	0.0430914	\\
0.257579	0.0429714	\\
0.258834	0.0428298	\\
0.260244	0.0426683	\\
0.261773	0.04249	\\
0.262476	0.0424071	\\
0.263972	0.0422286	\\
0.265131	0.0420886	\\
0.266681	0.0418988	\\
0.267848	0.0417542	\\
0.269029	0.0416065	\\
0.270687	0.0413968	\\
0.272064	0.0412206	\\
0.273021	0.0410971	\\
0.274058	0.0409624	\\
0.274715	0.0408766	\\
0.275995	0.0407085	\\
0.277002	0.0405752	\\
0.279084	0.0402974	\\
0.280201	0.0401471	\\
0.280852	0.0400592	\\
0.281772	0.0399344	\\
0.282503	0.0398348	\\
0.283784	0.0396596	\\
0.284679	0.0395365	\\
0.285512	0.0394217	\\
0.286971	0.0392196	\\
0.287939	0.039085	\\
0.290064	0.0387878	\\
0.291414	0.038598	\\
0.293254	0.038338	\\
0.294542	0.0381554	\\
0.296175	0.0379229	\\
0.297304	0.0377618	\\
0.298542	0.0375846	\\
0.299978	0.0373787	\\
0.301643	0.0371392	\\
0.303399	0.036886	\\
0.304693	0.036699	\\
0.306072	0.0364996	\\
0.307168	0.0363409	\\
0.308907	0.0360886	\\
0.310409	0.0358705	\\
0.311453	0.0357188	\\
0.312886	0.0355105	\\
0.314384	0.0352926	\\
0.315802	0.0350864	\\
0.316826	0.0349375	\\
0.318472	0.034698	\\
0.319978	0.0344789	\\
0.320952	0.0343374	\\
0.322137	0.0341651	\\
0.323487	0.033969	\\
0.32546	0.0336826	\\
0.326333	0.033556	\\
0.327803	0.0333431	\\
0.328939	0.0331788	\\
0.330106	0.0330101	\\
0.331217	0.0328496	\\
0.332388	0.0326808	\\
0.334033	0.0324438	\\
0.336997	0.0320183	\\
0.337996	0.0318752	\\
0.339107	0.0317163	\\
0.34047	0.0315218	\\
0.341418	0.0313867	\\
0.343424	0.0311017	\\
0.345429	0.0308176	\\
0.346538	0.030661	\\
0.347824	0.0304797	\\
0.349127	0.0302966	\\
0.351381	0.0299808	\\
0.352489	0.029826	\\
0.353752	0.0296501	\\
0.355398	0.0294216	\\
0.357443	0.0291389	\\
0.358614	0.0289775	\\
0.360395	0.028733	\\
0.362943	0.028385	\\
0.364721	0.0281435	\\
0.367066	0.0278266	\\
0.368198	0.0276743	\\
0.370096	0.02742	\\
0.371913	0.0271778	\\
0.374146	0.0268818	\\
0.375713	0.0266752	\\
0.378326	0.0263326	\\
0.379935	0.0261231	\\
0.381667	0.0258987	\\
0.383385	0.0256772	\\
0.385675	0.0253838	\\
0.387083	0.0252044	\\
0.38852	0.0250222	\\
0.391974	0.0245875	\\
0.39389	0.0243485	\\
0.395107	0.0241975	\\
0.396493	0.0240264	\\
0.397647	0.0238843	\\
0.399095	0.0237071	\\
0.400141	0.0235795	\\
0.401839	0.0233734	\\
0.403771	0.0231403	\\
0.405512	0.0229317	\\
0.407462	0.0226995	\\
0.409876	0.0224144	\\
0.411737	0.0221962	\\
0.413276	0.0220169	\\
0.414509	0.0218739	\\
0.416468	0.0216482	\\
0.417323	0.0215502	\\
0.418981	0.021361	\\
0.420591	0.0211785	\\
0.421779	0.0210445	\\
0.423482	0.0208534	\\
0.425064	0.0206772	\\
0.426785	0.0204865	\\
0.428318	0.0203178	\\
0.431548	0.0199656	\\
0.434767	0.019619	\\
0.437359	0.0193432	\\
0.439772	0.0190889	\\
0.441974	0.018859	\\
0.44365	0.0186854	\\
0.445904	0.0184539	\\
0.447634	0.0182776	\\
0.450898	0.0179484	\\
0.452192	0.0178191	\\
0.454729	0.0175677	\\
0.456512	0.0173926	\\
0.459154	0.0171355	\\
0.461828	0.0168783	\\
0.463977	0.0166737	\\
0.466542	0.016432	\\
0.469142	0.0161896	\\
0.47098	0.01602	\\
0.472419	0.0158881	\\
0.473777	0.0157644	\\
0.475743	0.0155866	\\
0.477949	0.0153889	\\
0.480112	0.015197	\\
0.4835	0.0149	\\
0.485685	0.0147108	\\
0.488222	0.0144935	\\
0.490474	0.0143027	\\
0.491845	0.0141875	\\
0.494792	0.0139423	\\
0.497299	0.0137362	\\
0.49979	0.0135338	\\
0.501809	0.0133714	\\
0.503775	0.0132148	\\
0.50673	0.012982	\\
0.508105	0.0128747	\\
0.511606	0.0126047	\\
0.515601	0.012302	\\
0.519691	0.0119979	\\
0.521657	0.0118538	\\
0.523934	0.0116887	\\
0.525638	0.0115662	\\
0.527003	0.0114688	\\
0.529262	0.011309	\\
0.531221	0.0111719	\\
0.53368	0.0110015	\\
0.536445	0.0108124	\\
0.538675	0.0106616	\\
0.541318	0.0104851	\\
0.543193	0.0103613	\\
0.544877	0.010251	\\
0.548326	0.0100279	\\
0.551087	0.00985208	\\
0.552196	0.00978214	\\
0.554199	0.00965675	\\
0.5554	0.00958215	\\
0.558038	0.00941984	\\
0.559977	0.00930192	\\
0.563032	0.00911841	\\
0.566192	0.00893153	\\
0.568973	0.00876951	\\
0.57462	0.00844737	\\
0.577324	0.00829636	\\
0.581472	0.00806877	\\
0.583344	0.00796758	\\
0.586459	0.00780142	\\
0.588681	0.00768451	\\
0.592427	0.00749038	\\
0.595686	0.00732455	\\
0.596761	0.00727048	\\
0.600326	0.00709325	\\
0.601606	0.00703042	\\
0.604555	0.00688729	\\
0.607517	0.0067457	\\
0.610228	0.00661808	\\
0.612805	0.00649843	\\
0.614009	0.00644312	\\
0.617021	0.00630622	\\
0.619689	0.0061868	\\
0.62227	0.00607287	\\
0.626615	0.00588466	\\
0.629554	0.0057598	\\
0.632247	0.00564714	\\
0.638017	0.0054113	\\
0.639473	0.00535295	\\
0.643873	0.00517941	\\
0.648235	0.00501153	\\
0.651095	0.00490364	\\
0.654668	0.00477126	\\
0.657656	0.00466257	\\
0.659734	0.00458806	\\
0.663918	0.00444066	\\
0.666508	0.00435115	\\
0.670846	0.0042042	\\
0.674249	0.00409143	\\
0.677272	0.00399312	\\
0.68107	0.003872	\\
0.683865	0.00378458	\\
0.686305	0.00370942	\\
0.691086	0.00356526	\\
0.692885	0.00351206	\\
0.695898	0.00342421	\\
0.69955	0.00331988	\\
0.702505	0.00323709	\\
0.705434	0.00315648	\\
0.706966	0.0031149	\\
0.710002	0.00303363	\\
0.71249	0.00296813	\\
0.718108	0.0028239	\\
0.722177	0.00272252	\\
0.724212	0.00267279	\\
0.727772	0.00258729	\\
0.730907	0.00251359	\\
0.733312	0.00245805	\\
0.737613	0.00236082	\\
0.743549	0.002231	\\
0.746625	0.00216567	\\
0.748683	0.00212271	\\
0.750398	0.00208734	\\
0.754213	0.00201013	\\
0.757265	0.00194975	\\
0.760278	0.00189135	\\
0.762993	0.00183976	\\
0.768141	0.00174451	\\
0.771412	0.00168572	\\
0.774243	0.00163594	\\
0.778214	0.00156774	\\
0.781509	0.0015126	\\
0.785039	0.00145496	\\
0.789806	0.00137942	\\
0.793167	0.00132774	\\
0.796088	0.00128386	\\
0.80221	0.00119497	\\
0.805098	0.00115444	\\
0.80767	0.00111912	\\
0.809874	0.0010894	\\
0.812891	0.00104954	\\
0.816306	0.00100555	\\
0.818988	0.000971848	\\
0.821809	0.000937174	\\
0.824666	0.000902855	\\
0.828553	0.000857466	\\
0.832199	0.00081621	\\
0.834358	0.000792369	\\
0.837803	0.000755241	\\
0.844224	0.000688968	\\
0.848839	0.000643627	\\
0.854473	0.00059081	\\
0.858483	0.000554886	\\
0.865074	0.00049877	\\
0.869705	0.000461472	\\
0.873873	0.000429373	\\
0.879156	0.000390656	\\
0.882406	0.000367901	\\
0.885564	0.000346559	\\
0.889228	0.000322729	\\
0.891719	0.000307091	\\
0.893937	0.000293545	\\
0.898794	0.000265114	\\
0.905655	0.000227769	\\
0.909293	0.00020927	\\
0.914879	0.000182589	\\
0.91945	0.000162278	\\
0.922132	0.000150979	\\
0.924317	0.000142113	\\
0.930693	0.000117933	\\
0.935614	0.000100962	\\
0.940388	8.58729e-05	\\
0.943663	7.6287e-05	\\
0.947772	6.51283e-05	\\
0.955624	4.6421e-05	\\
0.958188	4.1042e-05	\\
0.965368	2.78327e-05	\\
0.969607	2.12901e-05	\\
0.974504	1.48644e-05	\\
0.980515	8.5991e-06	\\
0.98462	5.32201e-06	\\
0.98935	2.53278e-06	\\
0.99532	4.84556e-07	\\
0.998514	4.85857e-08	\\
1.00389	3.32235e-07	\\
1.00875	1.67292e-06	\\
1.01152	2.89334e-06	\\
1.01834	7.30951e-06	\\
1.02222	1.0697e-05	\\
1.02639	1.50598e-05	\\
1.02968	1.90079e-05	\\
1.03472	2.59318e-05	\\
1.03708	2.95346e-05	\\
1.0432	3.99616e-05	\\
1.04803	4.92404e-05	\\
1.05248	5.86445e-05	\\
1.05556	6.56041e-05	\\
1.06137	7.97705e-05	\\
1.06618	9.25184e-05	\\
1.07213	0.000109508	\\
1.07635	0.000122389	\\
1.08382	0.000146856	\\
1.08761	0.000160063	\\
1.09427	0.00018462	\\
1.10094	0.00021084	\\
1.10428	0.00022457	\\
1.11107	0.000253752	\\
1.11445	0.00026888	\\
1.1179	0.000284725	\\
1.12264	0.000307224	\\
1.12783	0.000332754	\\
1.13184	0.000353112	\\
1.13883	0.000389916	\\
1.14441	0.000420495	\\
1.14871	0.000444778	\\
1.15328	0.000471278	\\
1.15726	0.000494848	\\
1.1602	0.000512652	\\
1.16443	0.000538747	\\
1.16996	0.000573707	\\
1.17577	0.000611435	\\
1.17954	0.000636543	\\
1.1832	0.000661305	\\
1.18791	0.000693812	\\
1.1929	0.000728991	\\
1.19916	0.000774144	\\
1.20767	0.000837476	\\
1.21307	0.000878767	\\
1.21808	0.000917881	\\
1.22241	0.000952171	\\
1.2294	0.0010088	\\
1.23769	0.00107772	\\
1.24156	0.00111049	\\
1.24652	0.00115318	\\
1.25395	0.00121835	\\
1.26076	0.00127941	\\
1.26748	0.00134083	\\
1.27252	0.00138772	\\
1.27741	0.00143376	\\
1.28349	0.00149185	\\
1.2914	0.00156895	\\
1.29571	0.00161153	\\
1.30192	0.00167381	\\
1.30717	0.00172715	\\
1.31215	0.00177833	\\
1.31742	0.00183321	\\
1.32457	0.00190861	\\
1.32922	0.00195833	\\
1.33684	0.00204081	\\
1.34381	0.00211738	\\
1.34972	0.00218322	\\
1.35563	0.00224979	\\
1.36125	0.00231377	\\
1.36887	0.00240157	\\
1.37549	0.00247887	\\
1.38533	0.00259543	\\
1.39279	0.00268517	\\
1.39763	0.00274393	\\
1.4055	0.00284047	\\
1.41028	0.00289979	\\
1.41439	0.00295101	\\
1.41835	0.00300076	\\
1.42524	0.00308794	\\
1.42924	0.00313896	\\
1.43495	0.00321232	\\
1.44057	0.00328501	\\
1.44784	0.00338002	\\
1.46077	0.00355117	\\
1.46666	0.00363015	\\
1.48032	0.00381532	\\
1.4846	0.00387401	\\
1.49294	0.00398927	\\
1.49869	0.00406937	\\
1.50401	0.00414382	\\
1.51435	0.00428987	\\
1.52143	0.00439096	\\
1.53151	0.00453591	\\
1.53696	0.00461495	\\
1.54633	0.00475185	\\
1.54953	0.00479881	\\
1.55879	0.00493565	\\
1.56552	0.00503578	\\
1.57553	0.00518594	\\
1.58196	0.00528301	\\
1.5874	0.00536567	\\
1.59513	0.00548362	\\
1.59983	0.00555567	\\
1.60538	0.00564119	\\
1.61302	0.00575941	\\
1.61949	0.00586008	\\
1.62707	0.00597873	\\
1.63344	0.0060788	\\
1.64378	0.00624246	\\
1.6555	0.00642923	\\
1.66625	0.00660174	\\
1.67451	0.0067352	\\
1.68825	0.00695877	\\
1.69695	0.00710107	\\
1.7046	0.00722692	\\
1.7117	0.00734425	\\
1.72369	0.00754338	\\
1.72862	0.00762568	\\
1.73709	0.00776732	\\
1.7441	0.00788515	\\
1.75046	0.00799238	\\
1.75613	0.00808836	\\
1.76015	0.00815639	\\
1.76765	0.00828394	\\
1.77208	0.00835938	\\
1.78519	0.00858361	\\
1.79739	0.00879342	\\
1.80889	0.00899226	\\
1.81709	0.00913446	\\
1.82862	0.00933537	\\
1.83824	0.00950363	\\
1.84642	0.00964705	\\
1.85506	0.00979912	\\
1.86419	0.00996021	\\
1.87	0.0100631	\\
1.87543	0.0101594	\\
1.88149	0.0102669	\\
1.88978	0.0104145	\\
1.89788	0.010559	\\
1.90767	0.010734	\\
1.91344	0.0108375	\\
1.92585	0.0110605	\\
1.93516	0.0112281	\\
1.94096	0.0113329	\\
1.95279	0.011547	\\
1.96239	0.0117212	\\
1.97105	0.0118785	\\
1.98137	0.0120665	\\
1.9862	0.0121547	\\
1.99367	0.0122911	\\
2.00146	0.0124337	\\
2.00946	0.0125803	\\
2.01415	0.0126664	\\
2.02919	0.0129427	\\
2.03299	0.0130127	\\
2.04513	0.0132364	\\
2.06157	0.0135403	\\
2.07762	0.0138376	\\
2.08203	0.0139192	\\
2.09592	0.0141771	\\
2.09902	0.0142348	\\
2.1106	0.0144503	\\
2.11849	0.0145973	\\
2.12568	0.0147313	\\
2.13566	0.0149174	\\
2.14361	0.015066	\\
2.15612	0.0152998	\\
2.16166	0.0154034	\\
2.17471	0.0156478	\\
2.18536	0.0158474	\\
2.19598	0.0160465	\\
2.20653	0.0162446	\\
2.22139	0.0165235	\\
2.22653	0.0166201	\\
2.23919	0.0168583	\\
2.24707	0.0170064	\\
2.25385	0.0171339	\\
2.26292	0.0173046	\\
2.27041	0.0174456	\\
2.27992	0.0176247	\\
2.29442	0.0178977	\\
2.30608	0.0181174	\\
2.31854	0.0183523	\\
2.3339	0.0186417	\\
2.35015	0.0189479	\\
2.35749	0.0190862	\\
2.3632	0.0191937	\\
2.36934	0.0193095	\\
2.38516	0.0196075	\\
2.39573	0.0198067	\\
2.40215	0.0199275	\\
2.41704	0.0202078	\\
2.43092	0.020469	\\
2.44218	0.0206808	\\
2.45123	0.0208511	\\
2.46866	0.0211787	\\
2.47758	0.0213462	\\
2.4849	0.0214836	\\
2.49979	0.021763	\\
2.5111	0.0219751	\\
2.52495	0.0222348	\\
2.53541	0.0224306	\\
2.54724	0.0226519	\\
2.55612	0.0228179	\\
2.57322	0.0231375	\\
2.58161	0.0232941	\\
2.59033	0.0234567	\\
2.60481	0.0237265	\\
2.61999	0.0240091	\\
2.63512	0.0242902	\\
2.64529	0.0244791	\\
2.65716	0.0246993	\\
2.66849	0.0249093	\\
2.68218	0.0251627	\\
2.69601	0.0254182	\\
2.70992	0.0256749	\\
2.71965	0.0258544	\\
2.73612	0.0261575	\\
2.7522	0.0264529	\\
2.77408	0.0268542	\\
2.79011	0.0271475	\\
2.80601	0.0274378	\\
2.82521	0.0277876	\\
2.83653	0.0279936	\\
2.85437	0.0283176	\\
2.86762	0.0285575	\\
2.88104	0.0288003	\\
2.88872	0.0289389	\\
2.90365	0.0292083	\\
2.91603	0.0294312	\\
2.93079	0.0296963	\\
2.94329	0.0299204	\\
2.96475	0.0303044	\\
2.97502	0.0304876	\\
2.98864	0.0307302	\\
3.00315	0.0309883	\\
3.01437	0.0311873	\\
3.02547	0.0313841	\\
3.03958	0.0316336	\\
3.05821	0.0319622	\\
3.07135	0.0321934	\\
3.07922	0.0323317	\\
3.09105	0.0325391	\\
3.09501	0.0326086	\\
3.11529	0.032963	\\
3.12956	0.0332118	\\
3.1364	0.0333309	\\
3.14312	0.0334478	\\
3.15484	0.0336513	\\
3.16956	0.0339063	\\
3.18858	0.0342348	\\
3.206	0.0345348	\\
3.21873	0.0347535	\\
3.23985	0.0351152	\\
3.2588	0.0354386	\\
3.27349	0.0356886	\\
3.29053	0.0359779	\\
3.30883	0.0362875	\\
3.32326	0.0365309	\\
3.34255	0.0368555	\\
3.35805	0.0371154	\\
3.37076	0.0373279	\\
3.38273	0.0375278	\\
3.39877	0.0377947	\\
3.40947	0.0379725	\\
3.42309	0.0381981	\\
3.44559	0.0385696	\\
3.4579	0.0387723	\\
3.47785	0.0390997	\\
3.49421	0.0393673	\\
3.50401	0.0395272	\\
3.51176	0.0396534	\\
3.52382	0.0398494	\\
3.53656	0.0400562	\\
3.55629	0.0403752	\\
3.5692	0.0405833	\\
3.58819	0.0408885	\\
3.6019	0.0411081	\\
3.62605	0.0414937	\\
3.64093	0.0417303	\\
3.66119	0.0420514	\\
3.67574	0.0422812	\\
3.69356	0.0425618	\\
3.70987	0.0428178	\\
3.7214	0.0429982	\\
3.73114	0.0431503	\\
3.75029	0.0434485	\\
3.76804	0.0437239	\\
3.78452	0.0439788	\\
3.81006	0.0443721	\\
3.82486	0.0445991	\\
3.84733	0.0449424	\\
3.86475	0.0452076	\\
3.89033	0.0455952	\\
3.9009	0.0457548	\\
3.92121	0.0460605	\\
3.93057	0.0462009	\\
3.94864	0.0464713	\\
3.96642	0.0467364	\\
3.98334	0.0469879	\\
3.9953	0.047165	\\
4.0092	0.0473705	\\
4.01971	0.0475253	\\
4.04036	0.0478288	\\
4.06313	0.0481619	\\
4.08001	0.0484077	\\
4.10293	0.0487403	\\
4.12102	0.0490017	\\
4.13415	0.0491907	\\
4.15229	0.0494512	\\
4.17638	0.0497955	\\
4.19415	0.0500484	\\
4.20661	0.0502252	\\
4.21883	0.0503981	\\
4.24168	0.0507204	\\
4.27087	0.0511297	\\
4.2816	0.0512795	\\
4.30361	0.0515859	\\
4.33567	0.0520297	\\
4.35606	0.0523104	\\
4.36785	0.0524722	\\
4.39144	0.0527947	\\
4.41208	0.0530756	\\
4.42695	0.0532772	\\
4.43477	0.0533829	\\
4.44879	0.0535721	\\
4.472	0.0538842	\\
4.51114	0.0544069	\\
4.52652	0.0546111	\\
4.54622	0.0548719	\\
4.56374	0.0551028	\\
4.58013	0.0553181	\\
4.60286	0.0556154	\\
4.62445	0.0558966	\\
4.64708	0.0561899	\\
4.67292	0.0565231	\\
4.68998	0.0567422	\\
4.71447	0.0570552	\\
4.73037	0.0572576	\\
4.74845	0.0574869	\\
4.7763	0.0578386	\\
4.81018	0.0582636	\\
4.84206	0.0586607	\\
4.8553	0.0588248	\\
4.87923	0.0591205	\\
4.8989	0.0593623	\\
4.92062	0.0596283	\\
4.94125	0.0598799	\\
4.96225	0.0601347	\\
4.97405	0.0602774	\\
4.99798	0.0605659	\\
5.01152	0.0607284	\\
5.02698	0.0609135	\\
5.04296	0.0611041	\\
5.07515	0.0614863	\\
5.10643	0.0618552	\\
5.12456	0.062068	\\
5.13305	0.0621674	\\
5.16698	0.0625626	\\
5.1908	0.0628386	\\
5.20705	0.0630259	\\
5.24076	0.0634129	\\
5.24932	0.0635106	\\
5.27866	0.0638446	\\
5.30212	0.0641102	\\
5.32995	0.0644236	\\
5.35464	0.0647001	\\
5.38854	0.0650775	\\
5.41724	0.0653951	\\
5.436	0.0656016	\\
5.48011	0.0660842	\\
5.51535	0.0664666	\\
5.54338	0.0667689	\\
5.55911	0.0669378	\\
5.58382	0.067202	\\
5.62636	0.0676537	\\
5.64758	0.0678776	\\
5.67398	0.0681548	\\
5.70173	0.0684447	\\
5.74764	0.0689206	\\
5.77259	0.0691774	\\
5.7944	0.0694009	\\
5.8163	0.0696243	\\
5.84153	0.0698805	\\
5.87032	0.0701713	\\
5.89715	0.0704407	\\
5.92661	0.0707349	\\
5.96002	0.0710666	\\
5.988	0.0713426	\\
6.0035	0.0714949	\\
6.02802	0.0717348	\\
6.07394	0.072181	\\
6.09769	0.0724103	\\
6.12448	0.0726675	\\
6.15224	0.0729327	\\
6.19105	0.0733011	\\
6.22736	0.0736432	\\
6.25917	0.0739409	\\
6.31269	0.0744377	\\
6.35032	0.0747839	\\
6.39267	0.0751705	\\
6.40727	0.0753031	\\
6.46447	0.0758189	\\
6.48122	0.0759689	\\
6.519	0.0763054	\\
6.53461	0.0764439	\\
6.5647	0.0767094	\\
6.5933	0.0769604	\\
6.64122	0.0773778	\\
6.68137	0.0777247	\\
6.70655	0.0779409	\\
6.73582	0.078191	\\
6.76645	0.0784511	\\
6.78333	0.0785939	\\
6.80961	0.0788152	\\
6.85015	0.0791546	\\
6.87144	0.0793317	\\
6.89538	0.0795302	\\
6.9168	0.0797069	\\
6.95005	0.07998	\\
6.99506	0.0803469	\\
7.02438	0.0805842	\\
7.04214	0.0807274	\\
7.07241	0.0809703	\\
7.10173	0.0812044	\\
7.1255	0.0813932	\\
7.17333	0.0817706	\\
7.20271	0.0820009	\\
7.21908	0.0821286	\\
7.26382	0.0824758	\\
7.29549	0.0827199	\\
7.32505	0.0829465	\\
7.3725	0.0833077	\\
7.40642	0.083564	\\
7.45068	0.0838961	\\
7.49681	0.0842394	\\
7.54219	0.0845745	\\
7.58626	0.0848973	\\
7.60773	0.0850536	\\
7.63563	0.0852559	\\
7.67934	0.0855709	\\
7.7144	0.0858217	\\
7.74902	0.0860679	\\
7.81312	0.0865199	\\
7.84339	0.0867316	\\
7.87181	0.0869294	\\
7.89968	0.0871223	\\
7.94355	0.0874242	\\
7.99478	0.0877738	\\
8.0338	0.0880381	\\
8.05909	0.0882084	\\
8.08868	0.0884068	\\
8.14235	0.088764	\\
8.18281	0.0890312	\\
8.22709	0.0893215	\\
8.25756	0.08952	\\
8.29441	0.0897587	\\
8.33781	0.090038	\\
8.3573	0.0901628	\\
8.38705	0.0903524	\\
8.43292	0.090643	\\
8.45956	0.0908107	\\
8.47307	0.0908954	\\
8.5	0.0910639	\\
8.52038	0.0911908	\\
8.56203	0.091449	\\
8.59445	0.0916487	\\
8.62651	0.0918452	\\
8.67078	0.0921148	\\
8.71021	0.0923532	\\
8.74704	0.0925746	\\
8.78105	0.0927779	\\
8.81942	0.0930058	\\
8.84083	0.0931324	\\
8.88987	0.0934207	\\
8.91925	0.0935923	\\
8.98304	0.0939622	\\
9.02012	0.0941755	\\
9.06756	0.0944465	\\
9.10711	0.0946709	\\
9.13123	0.0948071	\\
9.16562	0.0950003	\\
9.18945	0.0951336	\\
9.24534	0.0954443	\\
9.30029	0.0957471	\\
9.33588	0.0959418	\\
9.37337	0.0961457	\\
9.43709	0.0964897	\\
9.47973	0.096718	\\
9.54244	0.0970509	\\
9.57435	0.0972191	\\
9.61281	0.0974208	\\
9.64683	0.0975982	\\
9.6953	0.0978493	\\
9.77374	0.0982519	\\
9.81293	0.0984512	\\
9.8489	0.0986331	\\
9.90572	0.0989185	\\
9.932	0.0990496	\\
};
\addlegendentry{$\Delta_{N}(\sigma)$};

\addplot [color=matlab2, densely dotted]
  table[row sep=crcr]{
0.100015	2.127e-07	\\
0.100336	1.02715e-07	\\
0.100561	7.48563e-07	\\
0.101092	4.57499e-06	\\
0.101691	1.26564e-05	\\
0.102174	2.19753e-05	\\
0.10289	4.01961e-05	\\
0.103285	5.24334e-05	\\
0.103801	7.0613e-05	\\
0.104376	9.38035e-05	\\
0.104902	0.000117573	\\
0.10552	0.000148449	\\
0.106405	0.000198103	\\
0.106778	0.000220867	\\
0.107259	0.000251677	\\
0.107686	0.000280446	\\
0.107963	0.00029976	\\
0.108685	0.000352636	\\
0.109214	0.000393429	\\
0.109707	0.000433064	\\
0.110134	0.000468554	\\
0.111103	0.000552938	\\
0.111734	0.000610524	\\
0.112186	0.000653047	\\
0.112688	0.000701475	\\
0.113033	0.000735402	\\
0.113665	0.000799014	\\
0.114368	0.000871862	\\
0.11488	0.000926164	\\
0.115329	0.000974544	\\
0.115513	0.00099466	\\
0.116102	0.00105969	\\
0.116665	0.00112302	\\
0.116962	0.00115689	\\
0.117579	0.00122817	\\
0.11802	0.00127977	\\
0.118636	0.00135277	\\
0.119443	0.00145005	\\
0.119963	0.00151374	\\
0.120924	0.00163286	\\
0.121181	0.00166507	\\
0.122195	0.00179363	\\
0.122727	0.00186177	\\
0.12329	0.00193461	\\
0.123732	0.00199209	\\
0.12409	0.00203877	\\
0.124442	0.00208498	\\
0.125119	0.0021743	\\
0.125727	0.00225512	\\
0.126295	0.00233087	\\
0.127179	0.0024496	\\
0.127657	0.00251418	\\
0.128238	0.00259287	\\
0.128554	0.00263585	\\
0.129049	0.00270314	\\
0.129915	0.00282139	\\
0.130325	0.00287762	\\
0.130921	0.00295931	\\
0.131481	0.0030363	\\
0.132223	0.00313838	\\
0.132941	0.0032374	\\
0.133901	0.00336985	\\
0.134572	0.00346243	\\
0.135143	0.00354126	\\
0.135771	0.00362803	\\
0.136557	0.00373635	\\
0.137199	0.00382487	\\
0.137877	0.00391816	\\
0.138381	0.00398744	\\
0.139077	0.00408294	\\
0.139834	0.00418654	\\
0.140459	0.00427204	\\
0.141017	0.004348	\\
0.141424	0.00440342	\\
0.142416	0.00453805	\\
0.143505	0.00468495	\\
0.144129	0.00476883	\\
0.144945	0.00487812	\\
0.145646	0.00497157	\\
0.146226	0.00504848	\\
0.146843	0.00513009	\\
0.147435	0.0052081	\\
0.148121	0.00529806	\\
0.148733	0.00537798	\\
0.1495	0.00547751	\\
0.150498	0.00560633	\\
0.151159	0.00569104	\\
0.151942	0.00579082	\\
0.152293	0.00583529	\\
0.153102	0.00593737	\\
0.154286	0.00608546	\\
0.154867	0.0061575	\\
0.155517	0.00623765	\\
0.156005	0.00629746	\\
0.156597	0.00636979	\\
0.157325	0.00645799	\\
0.158119	0.0065536	\\
0.159341	0.00669909	\\
0.159964	0.00677265	\\
0.160553	0.00684169	\\
0.161612	0.00696467	\\
0.162586	0.00707655	\\
0.163148	0.00714043	\\
0.163583	0.00718967	\\
0.164439	0.00728592	\\
0.165021	0.00735076	\\
0.165753	0.0074317	\\
0.166827	0.0075491	\\
0.167842	0.00765874	\\
0.168846	0.00776581	\\
0.170115	0.00789918	\\
0.170973	0.00798811	\\
0.171892	0.00808235	\\
0.172206	0.00811427	\\
0.173045	0.00819883	\\
0.173717	0.00826595	\\
0.174618	0.00835499	\\
0.175431	0.00843426	\\
0.175763	0.0084665	\\
0.17639	0.00852676	\\
0.176848	0.00857047	\\
0.177562	0.0086381	\\
0.178582	0.00873347	\\
0.179392	0.00880825	\\
0.180059	0.00886909	\\
0.180937	0.00894836	\\
0.181749	0.00902068	\\
0.182615	0.00909689	\\
0.183481	0.00917217	\\
0.184193	0.0092332	\\
0.184908	0.00929387	\\
0.185697	0.00936006	\\
0.186241	0.00940518	\\
0.187253	0.00948819	\\
0.188144	0.00956007	\\
0.189283	0.00965059	\\
0.189833	0.00969366	\\
0.190917	0.0097774	\\
0.191531	0.00982416	\\
0.192787	0.00991839	\\
0.193147	0.00994499	\\
0.194284	0.0100281	\\
0.195764	0.0101337	\\
0.196783	0.0102049	\\
0.197564	0.0102586	\\
0.198592	0.0103281	\\
0.199754	0.0104052	\\
0.200825	0.0104748	\\
0.201526	0.0105195	\\
0.202452	0.0105778	\\
0.203152	0.0106212	\\
0.203499	0.0106425	\\
0.204247	0.0106879	\\
0.204955	0.0107303	\\
0.205628	0.0107701	\\
0.206233	0.0108054	\\
0.206993	0.0108491	\\
0.207526	0.0108795	\\
0.208113	0.0109124	\\
0.209284	0.0109771	\\
0.211021	0.0110702	\\
0.212628	0.0111534	\\
0.213591	0.011202	\\
0.214658	0.0112546	\\
0.215148	0.0112783	\\
0.216583	0.0113464	\\
0.217329	0.011381	\\
0.218106	0.0114164	\\
0.218764	0.0114459	\\
0.219373	0.0114728	\\
0.220098	0.0115043	\\
0.221431	0.0115609	\\
0.221823	0.0115772	\\
0.22244	0.0116026	\\
0.223583	0.0116486	\\
0.224506	0.0116848	\\
0.225701	0.0117305	\\
0.226956	0.0117769	\\
0.227942	0.0118125	\\
0.229057	0.0118515	\\
0.230206	0.0118906	\\
0.231229	0.0119244	\\
0.232482	0.0119644	\\
0.234038	0.0120123	\\
0.234983	0.0120404	\\
0.236193	0.0120751	\\
0.237082	0.0120999	\\
0.238066	0.0121265	\\
0.238912	0.0121488	\\
0.239986	0.0121762	\\
0.241948	0.0122238	\\
0.242831	0.0122443	\\
0.243463	0.0122585	\\
0.244878	0.0122893	\\
0.245524	0.0123028	\\
0.246599	0.0123247	\\
0.24792	0.0123503	\\
0.24935	0.0123766	\\
0.250667	0.0123994	\\
0.251826	0.0124185	\\
0.252669	0.0124318	\\
0.253519	0.0124447	\\
0.254469	0.0124585	\\
0.25498	0.0124657	\\
0.25552	0.0124731	\\
0.256499	0.012486	\\
0.257579	0.0124994	\\
0.258834	0.0125141	\\
0.260244	0.0125294	\\
0.261773	0.0125445	\\
0.262476	0.0125509	\\
0.263972	0.0125636	\\
0.265131	0.0125725	\\
0.266681	0.0125832	\\
0.267848	0.0125902	\\
0.269029	0.0125965	\\
0.270687	0.0126041	\\
0.272064	0.0126091	\\
0.273021	0.012612	\\
0.274058	0.0126146	\\
0.274715	0.0126159	\\
0.275995	0.0126178	\\
0.277002	0.0126187	\\
0.279084	0.0126189	\\
0.280201	0.0126181	\\
0.280852	0.0126174	\\
0.281772	0.0126159	\\
0.282503	0.0126145	\\
0.283784	0.0126113	\\
0.284679	0.0126087	\\
0.285512	0.0126058	\\
0.286971	0.0126001	\\
0.287939	0.0125958	\\
0.290064	0.0125848	\\
0.291414	0.0125767	\\
0.293254	0.0125645	\\
0.294542	0.0125551	\\
0.296175	0.0125422	\\
0.297304	0.0125326	\\
0.298542	0.0125215	\\
0.299978	0.0125079	\\
0.301643	0.0124911	\\
0.303399	0.0124723	\\
0.304693	0.0124577	\\
0.306072	0.0124414	\\
0.307168	0.0124281	\\
0.308907	0.0124059	\\
0.310409	0.012386	\\
0.311453	0.0123717	\\
0.312886	0.0123515	\\
0.314384	0.0123297	\\
0.315802	0.0123083	\\
0.316826	0.0122926	\\
0.318472	0.0122665	\\
0.319978	0.012242	\\
0.320952	0.0122257	\\
0.322137	0.0122056	\\
0.323487	0.0121822	\\
0.32546	0.0121471	\\
0.326333	0.0121312	\\
0.327803	0.0121041	\\
0.328939	0.0120827	\\
0.330106	0.0120604	\\
0.331217	0.0120388	\\
0.332388	0.0120157	\\
0.334033	0.0119828	\\
0.336997	0.0119218	\\
0.337996	0.0119007	\\
0.339107	0.0118771	\\
0.34047	0.0118478	\\
0.341418	0.0118271	\\
0.343424	0.0117828	\\
0.345429	0.0117376	\\
0.346538	0.0117123	\\
0.347824	0.0116827	\\
0.349127	0.0116523	\\
0.351381	0.0115991	\\
0.352489	0.0115726	\\
0.353752	0.0115421	\\
0.355398	0.0115019	\\
0.357443	0.0114514	\\
0.358614	0.0114222	\\
0.360395	0.0113773	\\
0.362943	0.0113123	\\
0.364721	0.0112663	\\
0.367066	0.011205	\\
0.368198	0.0111751	\\
0.370096	0.0111247	\\
0.371913	0.011076	\\
0.374146	0.0110155	\\
0.375713	0.0109728	\\
0.378326	0.0109008	\\
0.379935	0.0108561	\\
0.381667	0.0108077	\\
0.383385	0.0107594	\\
0.385675	0.0106946	\\
0.387083	0.0106545	\\
0.38852	0.0106133	\\
0.391974	0.0105138	\\
0.39389	0.0104581	\\
0.395107	0.0104226	\\
0.396493	0.010382	\\
0.397647	0.0103481	\\
0.399095	0.0103055	\\
0.400141	0.0102746	\\
0.401839	0.0102243	\\
0.403771	0.0101668	\\
0.405512	0.0101149	\\
0.407462	0.0100564	\\
0.409876	0.00998384	\\
0.411737	0.00992767	\\
0.413276	0.0098811	\\
0.414509	0.0098437	\\
0.416468	0.00978416	\\
0.417323	0.00975813	\\
0.418981	0.00970757	\\
0.420591	0.0096584	\\
0.421779	0.00962204	\\
0.423482	0.00956983	\\
0.425064	0.00952128	\\
0.426785	0.00946836	\\
0.428318	0.00942118	\\
0.431548	0.00932155	\\
0.434767	0.00922209	\\
0.437359	0.00914188	\\
0.439772	0.00906712	\\
0.441974	0.00899884	\\
0.44365	0.00894685	\\
0.445904	0.00887693	\\
0.447634	0.00882323	\\
0.450898	0.00872193	\\
0.452192	0.00868177	\\
0.454729	0.00860304	\\
0.456512	0.00854774	\\
0.459154	0.00846584	\\
0.461828	0.00838302	\\
0.463977	0.0083165	\\
0.466542	0.00823722	\\
0.469142	0.00815693	\\
0.47098	0.00810028	\\
0.472419	0.00805595	\\
0.473777	0.00801416	\\
0.475743	0.00795373	\\
0.477949	0.00788604	\\
0.480112	0.00781977	\\
0.4835	0.00771622	\\
0.485685	0.00764961	\\
0.488222	0.00757245	\\
0.490474	0.00750414	\\
0.491845	0.00746264	\\
0.494792	0.00737362	\\
0.497299	0.00729813	\\
0.49979	0.00722337	\\
0.501809	0.00716296	\\
0.503775	0.00710427	\\
0.50673	0.00701635	\\
0.508105	0.00697556	\\
0.511606	0.00687208	\\
0.515601	0.00675467	\\
0.519691	0.00663522	\\
0.521657	0.00657809	\\
0.523934	0.00651216	\\
0.525638	0.00646299	\\
0.527003	0.0064237	\\
0.529262	0.00635888	\\
0.531221	0.00630291	\\
0.53368	0.00623289	\\
0.536445	0.00615459	\\
0.538675	0.00609172	\\
0.541318	0.00601756	\\
0.543193	0.00596521	\\
0.544877	0.00591833	\\
0.548326	0.00582283	\\
0.551087	0.0057469	\\
0.552196	0.00571653	\\
0.554199	0.00566184	\\
0.5554	0.00562916	\\
0.558038	0.00555769	\\
0.559977	0.00550543	\\
0.563032	0.00542355	\\
0.566192	0.00533946	\\
0.568973	0.00526597	\\
0.57462	0.00511822	\\
0.577324	0.0050482	\\
0.581472	0.00494172	\\
0.583344	0.004894	\\
0.586459	0.00481515	\\
0.588681	0.00475929	\\
0.592427	0.00466584	\\
0.595686	0.00458531	\\
0.596761	0.00455891	\\
0.600326	0.00447188	\\
0.601606	0.00444084	\\
0.604555	0.00436977	\\
0.607517	0.00429895	\\
0.610228	0.00423468	\\
0.612805	0.00417404	\\
0.614009	0.00414589	\\
0.617021	0.00407584	\\
0.619689	0.00401433	\\
0.62227	0.00395529	\\
0.626615	0.00385695	\\
0.629554	0.00379116	\\
0.632247	0.00373142	\\
0.638017	0.00360514	\\
0.639473	0.00357364	\\
0.643873	0.00347934	\\
0.648235	0.00338722	\\
0.651095	0.00332754	\\
0.654668	0.0032538	\\
0.657656	0.00319282	\\
0.659734	0.00315079	\\
0.663918	0.00306707	\\
0.666508	0.00301587	\\
0.670846	0.00293118	\\
0.674249	0.00286566	\\
0.677272	0.00280815	\\
0.68107	0.00273679	\\
0.683865	0.00268494	\\
0.686305	0.00264011	\\
0.691086	0.0025535	\\
0.692885	0.00252133	\\
0.695898	0.00246793	\\
0.69955	0.0024041	\\
0.702505	0.0023531	\\
0.705434	0.00230316	\\
0.706966	0.00227729	\\
0.710002	0.00222649	\\
0.71249	0.00218532	\\
0.718108	0.00209395	\\
0.722177	0.00202912	\\
0.724212	0.00199713	\\
0.727772	0.00194185	\\
0.730907	0.00189388	\\
0.733312	0.00185754	\\
0.737613	0.00179352	\\
0.743549	0.00170723	\\
0.746625	0.00166343	\\
0.748683	0.0016345	\\
0.750398	0.0016106	\\
0.754213	0.00155815	\\
0.757265	0.00151687	\\
0.760278	0.00147673	\\
0.762993	0.00144108	\\
0.768141	0.00137481	\\
0.771412	0.00133359	\\
0.774243	0.00129849	\\
0.778214	0.00125013	\\
0.781509	0.00121077	\\
0.785039	0.00116939	\\
0.789806	0.00111476	\\
0.793167	0.00107712	\\
0.796088	0.00104499	\\
0.80221	0.000979388	\\
0.805098	0.000949249	\\
0.80767	0.000922857	\\
0.809874	0.000900565	\\
0.812891	0.000870532	\\
0.816306	0.000837213	\\
0.818988	0.000811553	\\
0.821809	0.000785032	\\
0.824666	0.00075866	\\
0.828553	0.000723586	\\
0.832199	0.000691508	\\
0.834358	0.000672883	\\
0.837803	0.000643744	\\
0.844224	0.000591318	\\
0.848839	0.00055513	\\
0.854473	0.000512629	\\
0.858483	0.000483497	\\
0.865074	0.000437613	\\
0.869705	0.000406845	\\
0.873873	0.000380183	\\
0.879156	0.000347788	\\
0.882406	0.000328624	\\
0.885564	0.00031056	\\
0.889228	0.000290285	\\
0.891719	0.000276918	\\
0.893937	0.000265297	\\
0.898794	0.000240777	\\
0.905655	0.000208285	\\
0.909293	0.000192062	\\
0.914879	0.000168504	\\
0.91945	0.000150434	\\
0.922132	0.000140329	\\
0.924317	0.000132371	\\
0.930693	0.000110532	\\
0.935614	9.5077e-05	\\
0.940388	8.12399e-05	\\
0.943663	7.2398e-05	\\
0.947772	6.2051e-05	\\
0.955624	4.45581e-05	\\
0.958188	3.94903e-05	\\
0.965368	2.69614e-05	\\
0.969607	2.07054e-05	\\
0.974504	1.4522e-05	\\
0.980515	8.44777e-06	\\
0.98462	5.24811e-06	\\
0.98935	2.50843e-06	\\
0.99532	4.8251e-07	\\
0.998514	4.85205e-08	\\
1.00389	3.32109e-07	\\
1.00875	1.6715e-06	\\
1.01152	2.8901e-06	\\
1.01834	7.29653e-06	\\
1.02222	1.06741e-05	\\
1.02639	1.50215e-05	\\
1.02968	1.89537e-05	\\
1.03472	2.58454e-05	\\
1.03708	2.94298e-05	\\
1.0432	3.97968e-05	\\
1.04803	4.90152e-05	\\
1.05248	5.83521e-05	\\
1.05556	6.52583e-05	\\
1.06137	7.93075e-05	\\
1.06618	9.19407e-05	\\
1.07213	0.000108766	\\
1.07635	0.000121512	\\
1.08382	0.000145705	\\
1.08761	0.000158755	\\
1.09427	0.000183002	\\
1.10094	0.000208868	\\
1.10428	0.000222403	\\
1.11107	0.000251153	\\
1.11445	0.000266049	\\
1.1179	0.000281641	\\
1.12264	0.000303771	\\
1.12783	0.000328866	\\
1.13184	0.000348865	\\
1.13883	0.000384996	\\
1.14441	0.00041499	\\
1.14871	0.000438795	\\
1.15328	0.000464758	\\
1.15726	0.000487839	\\
1.1602	0.000505266	\\
1.16443	0.000530796	\\
1.16996	0.000564979	\\
1.17577	0.000601843	\\
1.17954	0.000626362	\\
1.1832	0.000650531	\\
1.18791	0.000682245	\\
1.1929	0.000716544	\\
1.19916	0.000760539	\\
1.20767	0.000822192	\\
1.21307	0.000862355	\\
1.21808	0.000900377	\\
1.22241	0.000933692	\\
1.2294	0.00098867	\\
1.23769	0.00105553	\\
1.24156	0.0010873	\\
1.24652	0.00112865	\\
1.25395	0.00119175	\\
1.26076	0.00125082	\\
1.26748	0.00131019	\\
1.27252	0.00135549	\\
1.27741	0.00139994	\\
1.28349	0.001456	\\
1.2914	0.00153032	\\
1.29571	0.00157136	\\
1.30192	0.00163132	\\
1.30717	0.00168265	\\
1.31215	0.00173188	\\
1.31742	0.00178464	\\
1.32457	0.00185707	\\
1.32922	0.0019048	\\
1.33684	0.00198393	\\
1.34381	0.00205733	\\
1.34972	0.00212041	\\
1.35563	0.00218415	\\
1.36125	0.00224536	\\
1.36887	0.00232932	\\
1.37549	0.00240317	\\
1.38533	0.00251445	\\
1.39279	0.00260005	\\
1.39763	0.00265606	\\
1.4055	0.00274803	\\
1.41028	0.00280451	\\
1.41439	0.00285325	\\
1.41835	0.00290057	\\
1.42524	0.00298345	\\
1.42924	0.00303193	\\
1.43495	0.00310161	\\
1.44057	0.00317061	\\
1.44784	0.00326074	\\
1.46077	0.00342294	\\
1.46666	0.00349773	\\
1.48032	0.0036729	\\
1.4846	0.00372837	\\
1.49294	0.00383725	\\
1.49869	0.00391287	\\
1.50401	0.00398312	\\
1.51435	0.00412082	\\
1.52143	0.00421606	\\
1.53151	0.0043525	\\
1.53696	0.00442686	\\
1.54633	0.00455556	\\
1.54953	0.00459968	\\
1.55879	0.00472817	\\
1.56552	0.00482213	\\
1.57553	0.00496294	\\
1.58196	0.00505389	\\
1.5874	0.00513131	\\
1.59513	0.00524171	\\
1.59983	0.00530911	\\
1.60538	0.00538908	\\
1.61302	0.00549956	\\
1.61949	0.00559359	\\
1.62707	0.00570434	\\
1.63344	0.00579771	\\
1.64378	0.00595028	\\
1.6555	0.00612424	\\
1.66625	0.00628476	\\
1.67451	0.00640885	\\
1.68825	0.00661655	\\
1.69695	0.00674863	\\
1.7046	0.00686536	\\
1.7117	0.00697413	\\
1.72369	0.00715859	\\
1.72862	0.00723477	\\
1.73709	0.00736582	\\
1.7441	0.00747477	\\
1.75046	0.00757388	\\
1.75613	0.00766254	\\
1.76015	0.00772536	\\
1.76765	0.00784309	\\
1.77208	0.0079127	\\
1.78519	0.00811943	\\
1.79739	0.00831269	\\
1.80889	0.00849569	\\
1.81709	0.00862646	\\
1.82862	0.0088111	\\
1.83824	0.0089656	\\
1.84642	0.00909722	\\
1.85506	0.00923669	\\
1.86419	0.00938433	\\
1.87	0.0094786	\\
1.87543	0.00956675	\\
1.88149	0.00966515	\\
1.88978	0.00980019	\\
1.89788	0.00993232	\\
1.90767	0.0100922	\\
1.91344	0.0101868	\\
1.92585	0.0103903	\\
1.93516	0.0105432	\\
1.94096	0.0106387	\\
1.95279	0.0108337	\\
1.96239	0.0109924	\\
1.97105	0.0111355	\\
1.98137	0.0113064	\\
1.9862	0.0113866	\\
1.99367	0.0115105	\\
2.00146	0.0116399	\\
2.00946	0.011773	\\
2.01415	0.0118511	\\
2.02919	0.0121016	\\
2.03299	0.012165	\\
2.04513	0.0123676	\\
2.06157	0.0126426	\\
2.07762	0.0129113	\\
2.08203	0.012985	\\
2.09592	0.0132179	\\
2.09902	0.01327	\\
2.1106	0.0134643	\\
2.11849	0.0135968	\\
2.12568	0.0137175	\\
2.13566	0.0138851	\\
2.14361	0.0140189	\\
2.15612	0.0142292	\\
2.16166	0.0143223	\\
2.17471	0.0145419	\\
2.18536	0.0147211	\\
2.19598	0.0148998	\\
2.20653	0.0150775	\\
2.22139	0.0153274	\\
2.22653	0.015414	\\
2.23919	0.0156271	\\
2.24707	0.0157597	\\
2.25385	0.0158737	\\
2.26292	0.0160264	\\
2.27041	0.0161523	\\
2.27992	0.0163123	\\
2.29442	0.016556	\\
2.30608	0.0167519	\\
2.31854	0.0169613	\\
2.3339	0.0172191	\\
2.35015	0.0174916	\\
2.35749	0.0176146	\\
2.3632	0.0177101	\\
2.36934	0.017813	\\
2.38516	0.0180777	\\
2.39573	0.0182546	\\
2.40215	0.0183617	\\
2.41704	0.0186104	\\
2.43092	0.0188418	\\
2.44218	0.0190294	\\
2.45123	0.0191801	\\
2.46866	0.01947	\\
2.47758	0.0196181	\\
2.4849	0.0197395	\\
2.49979	0.0199863	\\
2.5111	0.0201735	\\
2.52495	0.0204025	\\
2.53541	0.0205752	\\
2.54724	0.0207702	\\
2.55612	0.0209164	\\
2.57322	0.0211977	\\
2.58161	0.0213355	\\
2.59033	0.0214785	\\
2.60481	0.0217156	\\
2.61999	0.0219638	\\
2.63512	0.0222106	\\
2.64529	0.0223763	\\
2.65716	0.0225693	\\
2.66849	0.0227533	\\
2.68218	0.0229753	\\
2.69601	0.0231989	\\
2.70992	0.0234235	\\
2.71965	0.0235804	\\
2.73612	0.0238453	\\
2.7522	0.0241033	\\
2.77408	0.0244535	\\
2.79011	0.0247092	\\
2.80601	0.0249621	\\
2.82521	0.0252667	\\
2.83653	0.0254459	\\
2.85437	0.0257277	\\
2.86762	0.0259362	\\
2.88104	0.0261471	\\
2.88872	0.0262674	\\
2.90365	0.0265012	\\
2.91603	0.0266946	\\
2.93079	0.0269244	\\
2.94329	0.0271186	\\
2.96475	0.0274512	\\
2.97502	0.0276097	\\
2.98864	0.0278196	\\
3.00315	0.0280427	\\
3.01437	0.0282147	\\
3.02547	0.0283847	\\
3.03958	0.0286001	\\
3.05821	0.0288837	\\
3.07135	0.0290831	\\
3.07922	0.0292023	\\
3.09105	0.029381	\\
3.09501	0.0294409	\\
3.11529	0.0297462	\\
3.12956	0.0299603	\\
3.1364	0.0300627	\\
3.14312	0.0301633	\\
3.15484	0.0303383	\\
3.16956	0.0305574	\\
3.18858	0.0308396	\\
3.206	0.0310972	\\
3.21873	0.0312848	\\
3.23985	0.031595	\\
3.2588	0.0318721	\\
3.27349	0.0320863	\\
3.29053	0.0323339	\\
3.30883	0.0325988	\\
3.32326	0.0328069	\\
3.34255	0.0330844	\\
3.35805	0.0333064	\\
3.37076	0.0334878	\\
3.38273	0.0336584	\\
3.39877	0.0338861	\\
3.40947	0.0340377	\\
3.42309	0.03423	\\
3.44559	0.0345465	\\
3.4579	0.0347192	\\
3.47785	0.0349977	\\
3.49421	0.0352254	\\
3.50401	0.0353614	\\
3.51176	0.0354687	\\
3.52382	0.0356352	\\
3.53656	0.0358109	\\
3.55629	0.0360818	\\
3.5692	0.0362585	\\
3.58819	0.0365173	\\
3.6019	0.0367036	\\
3.62605	0.0370303	\\
3.64093	0.0372307	\\
3.66119	0.0375026	\\
3.67574	0.037697	\\
3.69356	0.0379344	\\
3.70987	0.0381509	\\
3.7214	0.0383033	\\
3.73114	0.0384318	\\
3.75029	0.0386837	\\
3.76804	0.0389161	\\
3.78452	0.0391312	\\
3.81006	0.0394629	\\
3.82486	0.0396543	\\
3.84733	0.0399434	\\
3.86475	0.0401667	\\
3.89033	0.0404929	\\
3.9009	0.0406272	\\
3.92121	0.0408842	\\
3.93057	0.0410023	\\
3.94864	0.0412294	\\
3.96642	0.0414521	\\
3.98334	0.0416633	\\
3.9953	0.0418119	\\
4.0092	0.0419843	\\
4.01971	0.0421141	\\
4.04036	0.0423686	\\
4.06313	0.0426477	\\
4.08001	0.0428536	\\
4.10293	0.0431321	\\
4.12102	0.0433508	\\
4.13415	0.043509	\\
4.15229	0.0437267	\\
4.17638	0.0440145	\\
4.19415	0.0442258	\\
4.20661	0.0443735	\\
4.21883	0.0445178	\\
4.24168	0.0447868	\\
4.27087	0.0451282	\\
4.2816	0.0452531	\\
4.30361	0.0455085	\\
4.33567	0.0458782	\\
4.35606	0.0461118	\\
4.36785	0.0462465	\\
4.39144	0.0465148	\\
4.41208	0.0467484	\\
4.42695	0.0469159	\\
4.43477	0.0470038	\\
4.44879	0.047161	\\
4.472	0.0474202	\\
4.51114	0.0478541	\\
4.52652	0.0480235	\\
4.54622	0.0482397	\\
4.56374	0.0484311	\\
4.58013	0.0486096	\\
4.60286	0.0488558	\\
4.62445	0.0490887	\\
4.64708	0.0493314	\\
4.67292	0.0496071	\\
4.68998	0.0497883	\\
4.71447	0.050047	\\
4.73037	0.0502143	\\
4.74845	0.0504037	\\
4.7763	0.0506941	\\
4.81018	0.0510448	\\
4.84206	0.0513723	\\
4.8553	0.0515077	\\
4.87923	0.0517513	\\
4.8989	0.0519506	\\
4.92062	0.0521696	\\
4.94125	0.0523767	\\
4.96225	0.0525865	\\
4.97405	0.0527039	\\
4.99798	0.0529411	\\
5.01152	0.0530747	\\
5.02698	0.0532269	\\
5.04296	0.0533836	\\
5.07515	0.0536975	\\
5.10643	0.0540004	\\
5.12456	0.054175	\\
5.13305	0.0542566	\\
5.16698	0.0545808	\\
5.1908	0.054807	\\
5.20705	0.0549606	\\
5.24076	0.0552776	\\
5.24932	0.0553577	\\
5.27866	0.0556311	\\
5.30212	0.0558485	\\
5.32995	0.0561049	\\
5.35464	0.0563311	\\
5.38854	0.0566395	\\
5.41724	0.0568991	\\
5.436	0.0570677	\\
5.48011	0.0574617	\\
5.51535	0.0577737	\\
5.54338	0.0580202	\\
5.55911	0.0581579	\\
5.58382	0.0583732	\\
5.62636	0.0587413	\\
5.64758	0.0589236	\\
5.67398	0.0591493	\\
5.70173	0.0593851	\\
5.74764	0.0597722	\\
5.77259	0.059981	\\
5.7944	0.0601626	\\
5.8163	0.0603441	\\
5.84153	0.0605522	\\
5.87032	0.0607883	\\
5.89715	0.061007	\\
5.92661	0.0612457	\\
5.96002	0.0615147	\\
5.988	0.0617384	\\
6.0035	0.0618618	\\
6.02802	0.0620562	\\
6.07394	0.0624176	\\
6.09769	0.0626032	\\
6.12448	0.0628114	\\
6.15224	0.063026	\\
6.19105	0.0633239	\\
6.22736	0.0636005	\\
6.25917	0.063841	\\
6.31269	0.0642423	\\
6.35032	0.0645217	\\
6.39267	0.0648337	\\
6.40727	0.0649406	\\
6.46447	0.0653565	\\
6.48122	0.0654774	\\
6.519	0.0657485	\\
6.53461	0.06586	\\
6.5647	0.0660738	\\
6.5933	0.0662759	\\
6.64122	0.0666118	\\
6.68137	0.0668908	\\
6.70655	0.0670646	\\
6.73582	0.0672657	\\
6.76645	0.0674747	\\
6.78333	0.0675894	\\
6.80961	0.0677672	\\
6.85015	0.0680397	\\
6.87144	0.0681819	\\
6.89538	0.0683411	\\
6.9168	0.0684829	\\
6.95005	0.0687019	\\
6.99506	0.0689961	\\
7.02438	0.0691864	\\
7.04214	0.0693011	\\
7.07241	0.0694957	\\
7.10173	0.0696832	\\
7.1255	0.0698344	\\
7.17333	0.0701365	\\
7.20271	0.0703207	\\
7.21908	0.0704229	\\
7.26382	0.0707006	\\
7.29549	0.0708958	\\
7.32505	0.0710769	\\
7.3725	0.0713655	\\
7.40642	0.0715702	\\
7.45068	0.0718353	\\
7.49681	0.0721094	\\
7.54219	0.0723767	\\
7.58626	0.0726341	\\
7.60773	0.0727588	\\
7.63563	0.0729201	\\
7.67934	0.073171	\\
7.7144	0.0733709	\\
7.74902	0.073567	\\
7.81312	0.0739268	\\
7.84339	0.0740953	\\
7.87181	0.0742526	\\
7.89968	0.0744061	\\
7.94355	0.0746462	\\
7.99478	0.0749241	\\
8.0338	0.0751341	\\
8.05909	0.0752694	\\
8.08868	0.075427	\\
8.14235	0.0757107	\\
8.18281	0.0759228	\\
8.22709	0.0761531	\\
8.25756	0.0763106	\\
8.29441	0.0764999	\\
8.33781	0.0767214	\\
8.3573	0.0768202	\\
8.38705	0.0769705	\\
8.43292	0.0772008	\\
8.45956	0.0773336	\\
8.47307	0.0774008	\\
8.5	0.0775342	\\
8.52038	0.0776347	\\
8.56203	0.0778391	\\
8.59445	0.0779972	\\
8.62651	0.0781526	\\
8.67078	0.0783659	\\
8.71021	0.0785545	\\
8.74704	0.0787295	\\
8.78105	0.0788902	\\
8.81942	0.0790703	\\
8.84083	0.0791703	\\
8.88987	0.0793981	\\
8.91925	0.0795336	\\
8.98304	0.0798257	\\
9.02012	0.079994	\\
9.06756	0.0802079	\\
9.10711	0.0803849	\\
9.13123	0.0804923	\\
9.16562	0.0806447	\\
9.18945	0.0807498	\\
9.24534	0.0809946	\\
9.30029	0.0812332	\\
9.33588	0.0813866	\\
9.37337	0.0815473	\\
9.43709	0.0818181	\\
9.47973	0.0819978	\\
9.54244	0.0822598	\\
9.57435	0.0823921	\\
9.61281	0.0825507	\\
9.64683	0.0826902	\\
9.6953	0.0828877	\\
9.77374	0.083204	\\
9.81293	0.0833607	\\
9.8489	0.0835035	\\
9.90572	0.0837277	\\
9.932	0.0838307	\\
};
\addlegendentry{$\norm{u_{N}(\sigma) - u_{\mathcal N}(\sigma)}_{\mathcal X}$};

\end{axis}
\end{tikzpicture}%

        \end{subfigure}

        \begin{subfigure}[b]{0.45\textwidth}
            % This file was created by matlab2tikz v0.4.6 running on MATLAB 8.1.
% Copyright (c) 2008--2014, Nico Schlömer <nico.schloemer@gmail.com>
% All rights reserved.
% Minimal pgfplots version: 1.3
%
% The latest updates can be retrieved from
%   http://www.mathworks.com/matlabcentral/fileexchange/22022-matlab2tikz
% where you can also make suggestions and rate matlab2tikz.
%
\begin{tikzpicture}

\begin{axis}[%
width=10cm,
height=7cm,
scale only axis,
xmin=0,
xmax=10,
ymode=log,
ymin=1e-16,
ymax=1,
yminorticks=true,
ultra thick,
xlabel={One dimensional parameter space},
legend style={at={(1,0.03)},anchor=south east,legend cell align=left,align=left,fill=none,draw=none}
]
\addplot [color=matlab1,solid]
  table[row sep=crcr]{
0.100015	1.22584e-11	\\
0.100336	5.89461e-12	\\
0.100561	4.28909e-11	\\
0.101092	2.60629e-10	\\
0.101691	7.16305e-10	\\
0.102174	1.23719e-09	\\
0.10289	2.24553e-09	\\
0.103285	2.9167e-09	\\
0.103801	3.90633e-09	\\
0.104376	5.15746e-09	\\
0.104902	6.42832e-09	\\
0.10552	8.06375e-09	\\
0.106405	1.06616e-08	\\
0.106778	1.18406e-08	\\
0.107259	1.34252e-08	\\
0.107686	1.48939e-08	\\
0.107963	1.58744e-08	\\
0.108685	1.85367e-08	\\
0.109214	2.057e-08	\\
0.109707	2.25293e-08	\\
0.110134	2.42707e-08	\\
0.111103	2.83644e-08	\\
0.111734	3.11224e-08	\\
0.112186	3.31413e-08	\\
0.112688	3.54229e-08	\\
0.113033	3.70105e-08	\\
0.113665	3.99638e-08	\\
0.114368	4.33098e-08	\\
0.11488	4.57796e-08	\\
0.115329	4.79632e-08	\\
0.115513	4.88666e-08	\\
0.116102	5.17685e-08	\\
0.116665	5.45689e-08	\\
0.116962	5.6056e-08	\\
0.117579	5.91633e-08	\\
0.11802	6.13933e-08	\\
0.118636	6.45219e-08	\\
0.119443	6.86435e-08	\\
0.119963	7.13136e-08	\\
0.120924	7.62477e-08	\\
0.121181	7.75689e-08	\\
0.122195	8.27887e-08	\\
0.122727	8.55213e-08	\\
0.12329	8.84162e-08	\\
0.123732	9.06826e-08	\\
0.12409	9.25113e-08	\\
0.124442	9.43111e-08	\\
0.125119	9.77611e-08	\\
0.125727	1.00851e-07	\\
0.126295	1.03719e-07	\\
0.127179	1.08162e-07	\\
0.127657	1.10552e-07	\\
0.128238	1.13439e-07	\\
0.128554	1.15004e-07	\\
0.129049	1.17439e-07	\\
0.129915	1.2167e-07	\\
0.130325	1.23661e-07	\\
0.130921	1.2653e-07	\\
0.131481	1.29208e-07	\\
0.132223	1.32721e-07	\\
0.132941	1.36088e-07	\\
0.133901	1.40529e-07	\\
0.134572	1.43591e-07	\\
0.135143	1.46172e-07	\\
0.135771	1.48984e-07	\\
0.136557	1.52453e-07	\\
0.137199	1.55254e-07	\\
0.137877	1.58173e-07	\\
0.138381	1.60319e-07	\\
0.139077	1.63247e-07	\\
0.139834	1.66384e-07	\\
0.140459	1.68943e-07	\\
0.141017	1.71192e-07	\\
0.141424	1.7282e-07	\\
0.142416	1.76726e-07	\\
0.143505	1.80912e-07	\\
0.144129	1.83265e-07	\\
0.144945	1.86293e-07	\\
0.145646	1.88847e-07	\\
0.146226	1.90924e-07	\\
0.146843	1.93105e-07	\\
0.147435	1.95167e-07	\\
0.148121	1.97517e-07	\\
0.148733	1.9958e-07	\\
0.1495	2.02116e-07	\\
0.150498	2.05345e-07	\\
0.151159	2.07435e-07	\\
0.151942	2.09863e-07	\\
0.152293	2.10933e-07	\\
0.153102	2.13363e-07	\\
0.154286	2.1682e-07	\\
0.154867	2.18472e-07	\\
0.155517	2.20288e-07	\\
0.156005	2.21628e-07	\\
0.156597	2.2323e-07	\\
0.157325	2.25158e-07	\\
0.158119	2.27215e-07	\\
0.159341	2.3028e-07	\\
0.159964	2.31799e-07	\\
0.160553	2.33206e-07	\\
0.161612	2.35668e-07	\\
0.162586	2.37858e-07	\\
0.163148	2.39087e-07	\\
0.163583	2.40024e-07	\\
0.164439	2.41827e-07	\\
0.165021	2.43022e-07	\\
0.165753	2.4449e-07	\\
0.166827	2.46574e-07	\\
0.167842	2.4847e-07	\\
0.168846	2.50276e-07	\\
0.170115	2.5246e-07	\\
0.170973	2.53876e-07	\\
0.171892	2.5534e-07	\\
0.172206	2.55828e-07	\\
0.173045	2.57099e-07	\\
0.173717	2.58086e-07	\\
0.174618	2.59365e-07	\\
0.175431	2.60476e-07	\\
0.175763	2.60919e-07	\\
0.17639	2.61736e-07	\\
0.176848	2.62319e-07	\\
0.177562	2.63203e-07	\\
0.178582	2.64414e-07	\\
0.179392	2.65335e-07	\\
0.180059	2.66065e-07	\\
0.180937	2.66989e-07	\\
0.181749	2.67806e-07	\\
0.182615	2.6864e-07	\\
0.183481	2.69435e-07	\\
0.184193	2.70059e-07	\\
0.184908	2.70661e-07	\\
0.185697	2.71295e-07	\\
0.186241	2.71714e-07	\\
0.187253	2.72457e-07	\\
0.188144	2.73071e-07	\\
0.189283	2.73802e-07	\\
0.189833	2.74134e-07	\\
0.190917	2.74749e-07	\\
0.191531	2.75075e-07	\\
0.192787	2.75691e-07	\\
0.193147	2.75855e-07	\\
0.194284	2.76339e-07	\\
0.195764	2.7689e-07	\\
0.196783	2.7722e-07	\\
0.197564	2.77447e-07	\\
0.198592	2.77709e-07	\\
0.199754	2.7796e-07	\\
0.200825	2.78149e-07	\\
0.201526	2.7825e-07	\\
0.202452	2.78359e-07	\\
0.203152	2.78423e-07	\\
0.203499	2.78448e-07	\\
0.204247	2.7849e-07	\\
0.204955	2.78512e-07	\\
0.205628	2.7852e-07	\\
0.206233	2.78514e-07	\\
0.206993	2.78491e-07	\\
0.207526	2.78465e-07	\\
0.208113	2.78427e-07	\\
0.209284	2.7832e-07	\\
0.211021	2.78091e-07	\\
0.212628	2.77806e-07	\\
0.213591	2.77603e-07	\\
0.214658	2.77351e-07	\\
0.215148	2.77226e-07	\\
0.216583	2.76826e-07	\\
0.217329	2.766e-07	\\
0.218106	2.76349e-07	\\
0.218764	2.76127e-07	\\
0.219373	2.75913e-07	\\
0.220098	2.75647e-07	\\
0.221431	2.7513e-07	\\
0.221823	2.74971e-07	\\
0.22244	2.74714e-07	\\
0.223583	2.74218e-07	\\
0.224506	2.73799e-07	\\
0.225701	2.73232e-07	\\
0.226956	2.7261e-07	\\
0.227942	2.72101e-07	\\
0.229057	2.71505e-07	\\
0.230206	2.7087e-07	\\
0.231229	2.70286e-07	\\
0.232482	2.69549e-07	\\
0.234038	2.686e-07	\\
0.234983	2.68007e-07	\\
0.236193	2.6723e-07	\\
0.237082	2.66646e-07	\\
0.238066	2.65987e-07	\\
0.238912	2.65411e-07	\\
0.239986	2.64667e-07	\\
0.241948	2.63272e-07	\\
0.242831	2.6263e-07	\\
0.243463	2.62165e-07	\\
0.244878	2.61109e-07	\\
0.245524	2.6062e-07	\\
0.246599	2.59796e-07	\\
0.24792	2.58769e-07	\\
0.24935	2.57639e-07	\\
0.250667	2.56583e-07	\\
0.251826	2.5564e-07	\\
0.252669	2.54949e-07	\\
0.253519	2.54245e-07	\\
0.254469	2.53452e-07	\\
0.25498	2.53023e-07	\\
0.25552	2.52567e-07	\\
0.256499	2.51736e-07	\\
0.257579	2.50811e-07	\\
0.258834	2.49727e-07	\\
0.260244	2.48497e-07	\\
0.261773	2.4715e-07	\\
0.262476	2.46526e-07	\\
0.263972	2.4519e-07	\\
0.265131	2.44148e-07	\\
0.266681	2.42743e-07	\\
0.267848	2.41678e-07	\\
0.269029	2.40595e-07	\\
0.270687	2.39065e-07	\\
0.272064	2.37786e-07	\\
0.273021	2.36893e-07	\\
0.274058	2.35923e-07	\\
0.274715	2.35307e-07	\\
0.275995	2.34102e-07	\\
0.277002	2.33151e-07	\\
0.279084	2.31177e-07	\\
0.280201	2.30113e-07	\\
0.280852	2.29493e-07	\\
0.281772	2.28614e-07	\\
0.282503	2.27914e-07	\\
0.283784	2.26686e-07	\\
0.284679	2.25826e-07	\\
0.285512	2.25025e-07	\\
0.286971	2.2362e-07	\\
0.287939	2.22687e-07	\\
0.290064	2.20634e-07	\\
0.291414	2.19328e-07	\\
0.293254	2.17547e-07	\\
0.294542	2.16299e-07	\\
0.296175	2.14716e-07	\\
0.297304	2.13621e-07	\\
0.298542	2.12421e-07	\\
0.299978	2.1103e-07	\\
0.301643	2.09417e-07	\\
0.303399	2.07717e-07	\\
0.304693	2.06466e-07	\\
0.306072	2.05134e-07	\\
0.307168	2.04077e-07	\\
0.308907	2.024e-07	\\
0.310409	2.00955e-07	\\
0.311453	1.99952e-07	\\
0.312886	1.98578e-07	\\
0.314384	1.97144e-07	\\
0.315802	1.9579e-07	\\
0.316826	1.94814e-07	\\
0.318472	1.93248e-07	\\
0.319978	1.91819e-07	\\
0.320952	1.90898e-07	\\
0.322137	1.89779e-07	\\
0.323487	1.88507e-07	\\
0.32546	1.86653e-07	\\
0.326333	1.85836e-07	\\
0.327803	1.84464e-07	\\
0.328939	1.83407e-07	\\
0.330106	1.82324e-07	\\
0.331217	1.81294e-07	\\
0.332388	1.80214e-07	\\
0.334033	1.787e-07	\\
0.336997	1.7599e-07	\\
0.337996	1.75081e-07	\\
0.339107	1.74073e-07	\\
0.34047	1.72842e-07	\\
0.341418	1.71987e-07	\\
0.343424	1.70188e-07	\\
0.345429	1.684e-07	\\
0.346538	1.67416e-07	\\
0.347824	1.66279e-07	\\
0.349127	1.65132e-07	\\
0.351381	1.63159e-07	\\
0.352489	1.62194e-07	\\
0.353752	1.61098e-07	\\
0.355398	1.59678e-07	\\
0.357443	1.57925e-07	\\
0.358614	1.56926e-07	\\
0.360395	1.55414e-07	\\
0.362943	1.53269e-07	\\
0.364721	1.51784e-07	\\
0.367066	1.49839e-07	\\
0.368198	1.48907e-07	\\
0.370096	1.47353e-07	\\
0.371913	1.45875e-07	\\
0.374146	1.44073e-07	\\
0.375713	1.42818e-07	\\
0.378326	1.40742e-07	\\
0.379935	1.39475e-07	\\
0.381667	1.3812e-07	\\
0.383385	1.36785e-07	\\
0.385675	1.3502e-07	\\
0.387083	1.33944e-07	\\
0.38852	1.32851e-07	\\
0.391974	1.30253e-07	\\
0.39389	1.28828e-07	\\
0.395107	1.27928e-07	\\
0.396493	1.26911e-07	\\
0.397647	1.26067e-07	\\
0.399095	1.25016e-07	\\
0.400141	1.2426e-07	\\
0.401839	1.23041e-07	\\
0.403771	1.21664e-07	\\
0.405512	1.20435e-07	\\
0.407462	1.19068e-07	\\
0.409876	1.17394e-07	\\
0.411737	1.16115e-07	\\
0.413276	1.15066e-07	\\
0.414509	1.14231e-07	\\
0.416468	1.12913e-07	\\
0.417323	1.12342e-07	\\
0.418981	1.11241e-07	\\
0.420591	1.1018e-07	\\
0.421779	1.09402e-07	\\
0.423482	1.08295e-07	\\
0.425064	1.07274e-07	\\
0.426785	1.06172e-07	\\
0.428318	1.05199e-07	\\
0.431548	1.0317e-07	\\
0.434767	1.01179e-07	\\
0.437359	9.95993e-08	\\
0.439772	9.81459e-08	\\
0.441974	9.68343e-08	\\
0.44365	9.58458e-08	\\
0.445904	9.45296e-08	\\
0.447634	9.35287e-08	\\
0.450898	9.16644e-08	\\
0.452192	9.09337e-08	\\
0.454729	8.95145e-08	\\
0.456512	8.85281e-08	\\
0.459154	8.70827e-08	\\
0.461828	8.56395e-08	\\
0.463977	8.44935e-08	\\
0.466542	8.31428e-08	\\
0.469142	8.17909e-08	\\
0.47098	8.08468e-08	\\
0.472419	8.01134e-08	\\
0.473777	7.94265e-08	\\
0.475743	7.84406e-08	\\
0.477949	7.73463e-08	\\
0.480112	7.62854e-08	\\
0.4835	7.46478e-08	\\
0.485685	7.3607e-08	\\
0.488222	7.24134e-08	\\
0.490474	7.13675e-08	\\
0.491845	7.0737e-08	\\
0.494792	6.93967e-08	\\
0.497299	6.82729e-08	\\
0.49979	6.71715e-08	\\
0.501809	6.62894e-08	\\
0.503775	6.54397e-08	\\
0.50673	6.41791e-08	\\
0.508105	6.35993e-08	\\
0.511606	6.21427e-08	\\
0.515601	6.05142e-08	\\
0.519691	5.88833e-08	\\
0.521657	5.81122e-08	\\
0.523934	5.72298e-08	\\
0.525638	5.65764e-08	\\
0.527003	5.60576e-08	\\
0.529262	5.52072e-08	\\
0.531221	5.44786e-08	\\
0.53368	5.35748e-08	\\
0.536445	5.25737e-08	\\
0.538675	5.17772e-08	\\
0.541318	5.0846e-08	\\
0.543193	5.0194e-08	\\
0.544877	4.96139e-08	\\
0.548326	4.84428e-08	\\
0.551087	4.7522e-08	\\
0.552196	4.71562e-08	\\
0.554199	4.65012e-08	\\
0.5554	4.61119e-08	\\
0.558038	4.52662e-08	\\
0.559977	4.46527e-08	\\
0.563032	4.36998e-08	\\
0.566192	4.27315e-08	\\
0.568973	4.18938e-08	\\
0.57462	4.02333e-08	\\
0.577324	3.94572e-08	\\
0.581472	3.82904e-08	\\
0.583344	3.77727e-08	\\
0.586459	3.69241e-08	\\
0.588681	3.63281e-08	\\
0.592427	3.53407e-08	\\
0.595686	3.44992e-08	\\
0.596761	3.42252e-08	\\
0.600326	3.33288e-08	\\
0.601606	3.30115e-08	\\
0.604555	3.22897e-08	\\
0.607517	3.15772e-08	\\
0.610228	3.09363e-08	\\
0.612805	3.03364e-08	\\
0.614009	3.00595e-08	\\
0.617021	2.93751e-08	\\
0.619689	2.87793e-08	\\
0.62227	2.82119e-08	\\
0.626615	2.72767e-08	\\
0.629554	2.66578e-08	\\
0.632247	2.61005e-08	\\
0.638017	2.49371e-08	\\
0.639473	2.46501e-08	\\
0.643873	2.37978e-08	\\
0.648235	2.29758e-08	\\
0.651095	2.24488e-08	\\
0.654668	2.18036e-08	\\
0.657656	2.1275e-08	\\
0.659734	2.09133e-08	\\
0.663918	2.01992e-08	\\
0.666508	1.97665e-08	\\
0.670846	1.90578e-08	\\
0.674249	1.85154e-08	\\
0.677272	1.80435e-08	\\
0.68107	1.74634e-08	\\
0.683865	1.70457e-08	\\
0.686305	1.66871e-08	\\
0.691086	1.60011e-08	\\
0.692885	1.57484e-08	\\
0.695898	1.53319e-08	\\
0.69955	1.48383e-08	\\
0.702505	1.44475e-08	\\
0.705434	1.40677e-08	\\
0.706966	1.38721e-08	\\
0.710002	1.34903e-08	\\
0.71249	1.31832e-08	\\
0.718108	1.25087e-08	\\
0.722177	1.2036e-08	\\
0.724212	1.18046e-08	\\
0.727772	1.14076e-08	\\
0.730907	1.1066e-08	\\
0.733312	1.08091e-08	\\
0.737613	1.03603e-08	\\
0.743549	9.76306e-09	\\
0.746625	9.46339e-09	\\
0.748683	9.26662e-09	\\
0.750398	9.10488e-09	\\
0.754213	8.75234e-09	\\
0.757265	8.47727e-09	\\
0.760278	8.21171e-09	\\
0.762993	7.97754e-09	\\
0.768141	7.54636e-09	\\
0.771412	7.28093e-09	\\
0.774243	7.05658e-09	\\
0.778214	6.74993e-09	\\
0.781509	6.50259e-09	\\
0.785039	6.24458e-09	\\
0.789806	5.90736e-09	\\
0.793167	5.67725e-09	\\
0.796088	5.48224e-09	\\
0.80221	5.08835e-09	\\
0.805098	4.90928e-09	\\
0.80767	4.75347e-09	\\
0.809874	4.62258e-09	\\
0.812891	4.44732e-09	\\
0.816306	4.25431e-09	\\
0.818988	4.10672e-09	\\
0.821809	3.95513e-09	\\
0.824666	3.80537e-09	\\
0.828553	3.60771e-09	\\
0.832199	3.42849e-09	\\
0.834358	3.3251e-09	\\
0.837803	3.16439e-09	\\
0.844224	2.8784e-09	\\
0.848839	2.68343e-09	\\
0.854473	2.45703e-09	\\
0.858483	2.30351e-09	\\
0.865074	2.06451e-09	\\
0.869705	1.90621e-09	\\
0.873873	1.77035e-09	\\
0.879156	1.60697e-09	\\
0.882406	1.5112e-09	\\
0.885564	1.42156e-09	\\
0.889228	1.32169e-09	\\
0.891719	1.25627e-09	\\
0.893937	1.19969e-09	\\
0.898794	1.0812e-09	\\
0.905655	9.26127e-10	\\
0.909293	8.49563e-10	\\
0.914879	7.39453e-10	\\
0.91945	6.55898e-10	\\
0.922132	6.09526e-10	\\
0.924317	5.7319e-10	\\
0.930693	4.7436e-10	\\
0.935614	4.05238e-10	\\
0.940388	3.4397e-10	\\
0.943663	3.05145e-10	\\
0.947772	2.60054e-10	\\
0.955624	1.84737e-10	\\
0.958188	1.63153e-10	\\
0.965368	1.10306e-10	\\
0.969607	8.42257e-11	\\
0.974504	5.86832e-11	\\
0.980515	3.38627e-11	\\
0.98462	2.09215e-11	\\
0.98935	9.93702e-12	\\
0.99532	1.89643e-12	\\
0.998514	1.89998e-13	\\
1.00389	1.29573e-12	\\
1.00875	6.51065e-12	\\
1.01152	1.1247e-11	\\
1.01834	2.83331e-11	\\
1.02222	4.13972e-11	\\
1.02639	5.81801e-11	\\
1.02968	7.33333e-11	\\
1.03472	9.98382e-11	\\
1.03708	1.13599e-10	\\
1.0432	1.53317e-10	\\
1.04803	1.88543e-10	\\
1.05248	2.24142e-10	\\
1.05556	2.50426e-10	\\
1.06137	3.03782e-10	\\
1.06618	3.51637e-10	\\
1.07213	4.15206e-10	\\
1.07635	4.6325e-10	\\
1.08382	5.54179e-10	\\
1.08761	6.03093e-10	\\
1.09427	6.9375e-10	\\
1.10094	7.90153e-10	\\
1.10428	8.40479e-10	\\
1.11107	9.4711e-10	\\
1.11445	1.00222e-09	\\
1.1179	1.05982e-09	\\
1.12264	1.1414e-09	\\
1.12783	1.23369e-09	\\
1.13184	1.30708e-09	\\
1.13883	1.43931e-09	\\
1.14441	1.54876e-09	\\
1.14871	1.63541e-09	\\
1.15328	1.72971e-09	\\
1.15726	1.81338e-09	\\
1.1602	1.87644e-09	\\
1.16443	1.96867e-09	\\
1.16996	2.09186e-09	\\
1.17577	2.22435e-09	\\
1.17954	2.31226e-09	\\
1.1832	2.39877e-09	\\
1.18791	2.51205e-09	\\
1.1929	2.63428e-09	\\
1.19916	2.79063e-09	\\
1.20767	3.00895e-09	\\
1.21307	3.15069e-09	\\
1.21808	3.28454e-09	\\
1.22241	3.40154e-09	\\
1.2294	3.5941e-09	\\
1.23769	3.82741e-09	\\
1.24156	3.93792e-09	\\
1.24652	4.0815e-09	\\
1.25395	4.29987e-09	\\
1.26076	4.5036e-09	\\
1.26748	4.7077e-09	\\
1.27252	4.86296e-09	\\
1.27741	5.01494e-09	\\
1.28349	5.2061e-09	\\
1.2914	5.45868e-09	\\
1.29571	5.5977e-09	\\
1.30192	5.80033e-09	\\
1.30717	5.97331e-09	\\
1.31215	6.13876e-09	\\
1.31742	6.31563e-09	\\
1.32457	6.55772e-09	\\
1.32922	6.71678e-09	\\
1.33684	6.97969e-09	\\
1.34381	7.22264e-09	\\
1.34972	7.43077e-09	\\
1.35563	7.64047e-09	\\
1.36125	7.84127e-09	\\
1.36887	8.11576e-09	\\
1.37549	8.35637e-09	\\
1.38533	8.7174e-09	\\
1.39279	8.99389e-09	\\
1.39763	9.17428e-09	\\
1.4055	9.46949e-09	\\
1.41028	9.65019e-09	\\
1.41439	9.80579e-09	\\
1.41835	9.95656e-09	\\
1.42524	1.02199e-08	\\
1.42924	1.03735e-08	\\
1.43495	1.05937e-08	\\
1.44057	1.08111e-08	\\
1.44784	1.10942e-08	\\
1.46077	1.16011e-08	\\
1.46666	1.18336e-08	\\
1.48032	1.23756e-08	\\
1.4846	1.25464e-08	\\
1.49294	1.28807e-08	\\
1.49869	1.31119e-08	\\
1.50401	1.33262e-08	\\
1.51435	1.37444e-08	\\
1.52143	1.40323e-08	\\
1.53151	1.4443e-08	\\
1.53696	1.46659e-08	\\
1.54633	1.50502e-08	\\
1.54953	1.51815e-08	\\
1.55879	1.55627e-08	\\
1.56552	1.58402e-08	\\
1.57553	1.62543e-08	\\
1.58196	1.65206e-08	\\
1.5874	1.67465e-08	\\
1.59513	1.70676e-08	\\
1.59983	1.7263e-08	\\
1.60538	1.74941e-08	\\
1.61302	1.78124e-08	\\
1.61949	1.80822e-08	\\
1.62707	1.83988e-08	\\
1.63344	1.86647e-08	\\
1.64378	1.90972e-08	\\
1.6555	1.95874e-08	\\
1.66625	2.0037e-08	\\
1.67451	2.03827e-08	\\
1.68825	2.09579e-08	\\
1.69695	2.13214e-08	\\
1.7046	2.16412e-08	\\
1.7117	2.19379e-08	\\
1.72369	2.24385e-08	\\
1.72862	2.26443e-08	\\
1.73709	2.29969e-08	\\
1.7441	2.32888e-08	\\
1.75046	2.35533e-08	\\
1.75613	2.37891e-08	\\
1.76015	2.39558e-08	\\
1.76765	2.4267e-08	\\
1.77208	2.44504e-08	\\
1.78519	2.49924e-08	\\
1.79739	2.54954e-08	\\
1.80889	2.59684e-08	\\
1.81709	2.63045e-08	\\
1.82862	2.67763e-08	\\
1.83824	2.71687e-08	\\
1.84642	2.75012e-08	\\
1.85506	2.78517e-08	\\
1.86419	2.82209e-08	\\
1.87	2.84556e-08	\\
1.87543	2.86743e-08	\\
1.88149	2.89176e-08	\\
1.88978	2.92501e-08	\\
1.89788	2.95738e-08	\\
1.90767	2.99634e-08	\\
1.91344	3.01928e-08	\\
1.92585	3.06836e-08	\\
1.93516	3.10499e-08	\\
1.94096	3.12777e-08	\\
1.95279	3.17402e-08	\\
1.96239	3.2114e-08	\\
1.97105	3.24493e-08	\\
1.98137	3.28473e-08	\\
1.9862	3.30331e-08	\\
1.99367	3.33193e-08	\\
2.00146	3.36166e-08	\\
2.00946	3.39208e-08	\\
2.01415	3.40987e-08	\\
2.02919	3.46655e-08	\\
2.03299	3.48082e-08	\\
2.04513	3.52615e-08	\\
2.06157	3.58709e-08	\\
2.07762	3.64603e-08	\\
2.08203	3.6621e-08	\\
2.09592	3.71252e-08	\\
2.09902	3.72373e-08	\\
2.1106	3.76537e-08	\\
2.11849	3.79358e-08	\\
2.12568	3.81915e-08	\\
2.13566	3.85445e-08	\\
2.14361	3.88243e-08	\\
2.15612	3.92613e-08	\\
2.16166	3.94537e-08	\\
2.17471	3.99043e-08	\\
2.18536	4.02691e-08	\\
2.19598	4.063e-08	\\
2.20653	4.09863e-08	\\
2.22139	4.1483e-08	\\
2.22653	4.16538e-08	\\
2.23919	4.20717e-08	\\
2.24707	4.23296e-08	\\
2.25385	4.25503e-08	\\
2.26292	4.28441e-08	\\
2.27041	4.3085e-08	\\
2.27992	4.33891e-08	\\
2.29442	4.38482e-08	\\
2.30608	4.42137e-08	\\
2.31854	4.46008e-08	\\
2.3339	4.50724e-08	\\
2.35015	4.55649e-08	\\
2.35749	4.57852e-08	\\
2.3632	4.59554e-08	\\
2.36934	4.61379e-08	\\
2.38516	4.66034e-08	\\
2.39573	4.69111e-08	\\
2.40215	4.70963e-08	\\
2.41704	4.75224e-08	\\
2.43092	4.79143e-08	\\
2.44218	4.82288e-08	\\
2.45123	4.84794e-08	\\
2.46866	4.89559e-08	\\
2.47758	4.91968e-08	\\
2.4849	4.93927e-08	\\
2.49979	4.97876e-08	\\
2.5111	5.00837e-08	\\
2.52495	5.0442e-08	\\
2.53541	5.07094e-08	\\
2.54724	5.10082e-08	\\
2.55612	5.12303e-08	\\
2.57322	5.16525e-08	\\
2.58161	5.1857e-08	\\
2.59033	5.20675e-08	\\
2.60481	5.24128e-08	\\
2.61999	5.27693e-08	\\
2.63512	5.31187e-08	\\
2.64529	5.33505e-08	\\
2.65716	5.36177e-08	\\
2.66849	5.38696e-08	\\
2.68218	5.41696e-08	\\
2.69601	5.44677e-08	\\
2.70992	5.4763e-08	\\
2.71965	5.49669e-08	\\
2.73612	5.53064e-08	\\
2.7522	5.56314e-08	\\
2.77408	5.60637e-08	\\
2.79011	5.6373e-08	\\
2.80601	5.66736e-08	\\
2.82521	5.70285e-08	\\
2.83653	5.72337e-08	\\
2.85437	5.75509e-08	\\
2.86762	5.77815e-08	\\
2.88104	5.80109e-08	\\
2.88872	5.81402e-08	\\
2.90365	5.83879e-08	\\
2.91603	5.85892e-08	\\
2.93079	5.88246e-08	\\
2.94329	5.902e-08	\\
2.96475	5.93473e-08	\\
2.97502	5.95e-08	\\
2.98864	5.9699e-08	\\
3.00315	5.99065e-08	\\
3.01437	6.00636e-08	\\
3.02547	6.02164e-08	\\
3.03958	6.04066e-08	\\
3.05821	6.06509e-08	\\
3.07135	6.08188e-08	\\
3.07922	6.09174e-08	\\
3.09105	6.10632e-08	\\
3.09501	6.11114e-08	\\
3.11529	6.13525e-08	\\
3.12956	6.1517e-08	\\
3.1364	6.15942e-08	\\
3.14312	6.16693e-08	\\
3.15484	6.17977e-08	\\
3.16956	6.1955e-08	\\
3.18858	6.21514e-08	\\
3.206	6.23248e-08	\\
3.21873	6.24476e-08	\\
3.23985	6.26439e-08	\\
3.2588	6.28124e-08	\\
3.27349	6.29381e-08	\\
3.29053	6.30786e-08	\\
3.30883	6.3223e-08	\\
3.32326	6.33323e-08	\\
3.34255	6.34721e-08	\\
3.35805	6.35793e-08	\\
3.37076	6.36637e-08	\\
3.38273	6.37404e-08	\\
3.39877	6.3839e-08	\\
3.40947	6.39022e-08	\\
3.42309	6.39794e-08	\\
3.44559	6.40996e-08	\\
3.4579	6.41614e-08	\\
3.47785	6.42559e-08	\\
3.49421	6.4328e-08	\\
3.50401	6.43689e-08	\\
3.51176	6.44001e-08	\\
3.52382	6.44465e-08	\\
3.53656	6.44928e-08	\\
3.55629	6.4559e-08	\\
3.5692	6.45987e-08	\\
3.58819	6.46519e-08	\\
3.6019	6.46866e-08	\\
3.62605	6.474e-08	\\
3.64093	6.47682e-08	\\
3.66119	6.48008e-08	\\
3.67574	6.48201e-08	\\
3.69356	6.48392e-08	\\
3.70987	6.48522e-08	\\
3.7214	6.48589e-08	\\
3.73114	6.4863e-08	\\
3.75029	6.48668e-08	\\
3.76804	6.48652e-08	\\
3.78452	6.48596e-08	\\
3.81006	6.48428e-08	\\
3.82486	6.48287e-08	\\
3.84733	6.48012e-08	\\
3.86475	6.47749e-08	\\
3.89033	6.47285e-08	\\
3.9009	6.47066e-08	\\
3.92121	6.46603e-08	\\
3.93057	6.46371e-08	\\
3.94864	6.45888e-08	\\
3.96642	6.45371e-08	\\
3.98334	6.44839e-08	\\
3.9953	6.44441e-08	\\
4.0092	6.43954e-08	\\
4.01971	6.4357e-08	\\
4.04036	6.42774e-08	\\
4.06313	6.41834e-08	\\
4.08001	6.41095e-08	\\
4.10293	6.40036e-08	\\
4.12102	6.39155e-08	\\
4.13415	6.38492e-08	\\
4.15229	6.37541e-08	\\
4.17638	6.3622e-08	\\
4.19415	6.35202e-08	\\
4.20661	6.34467e-08	\\
4.21883	6.33729e-08	\\
4.24168	6.32305e-08	\\
4.27087	6.30403e-08	\\
4.2816	6.29681e-08	\\
4.30361	6.28161e-08	\\
4.33567	6.25857e-08	\\
4.35606	6.24337e-08	\\
4.36785	6.23439e-08	\\
4.39144	6.216e-08	\\
4.41208	6.19947e-08	\\
4.42695	6.18731e-08	\\
4.43477	6.18083e-08	\\
4.44879	6.16906e-08	\\
4.472	6.14918e-08	\\
4.51114	6.11454e-08	\\
4.52652	6.10055e-08	\\
4.54622	6.08233e-08	\\
4.56374	6.06585e-08	\\
4.58013	6.05018e-08	\\
4.60286	6.0281e-08	\\
4.62445	6.00672e-08	\\
4.64708	5.98391e-08	\\
4.67292	5.95736e-08	\\
4.68998	5.93955e-08	\\
4.71447	5.91361e-08	\\
4.73037	5.89651e-08	\\
4.74845	5.87685e-08	\\
4.7763	5.8461e-08	\\
4.81018	5.80795e-08	\\
4.84206	5.77135e-08	\\
4.8553	5.75594e-08	\\
4.87923	5.7278e-08	\\
4.8989	5.7044e-08	\\
4.92062	5.67827e-08	\\
4.94125	5.65319e-08	\\
4.96225	5.62739e-08	\\
4.97405	5.61278e-08	\\
4.99798	5.5829e-08	\\
5.01152	5.56586e-08	\\
5.02698	5.54625e-08	\\
5.04296	5.52586e-08	\\
5.07515	5.48436e-08	\\
5.10643	5.44351e-08	\\
5.12456	5.4196e-08	\\
5.13305	5.40835e-08	\\
5.16698	5.36304e-08	\\
5.1908	5.33089e-08	\\
5.20705	5.30883e-08	\\
5.24076	5.26264e-08	\\
5.24932	5.25084e-08	\\
5.27866	5.21014e-08	\\
5.30212	5.17734e-08	\\
5.32995	5.13814e-08	\\
5.35464	5.10312e-08	\\
5.38854	5.05467e-08	\\
5.41724	5.0133e-08	\\
5.436	4.98613e-08	\\
5.48011	4.92175e-08	\\
5.51535	4.86989e-08	\\
5.54338	4.82836e-08	\\
5.55911	4.80496e-08	\\
5.58382	4.76806e-08	\\
5.62636	4.70418e-08	\\
5.64758	4.67213e-08	\\
5.67398	4.63213e-08	\\
5.70173	4.58991e-08	\\
5.74764	4.51971e-08	\\
5.77259	4.48139e-08	\\
5.7944	4.44779e-08	\\
5.8163	4.41398e-08	\\
5.84153	4.37493e-08	\\
5.87032	4.33024e-08	\\
5.89715	4.2885e-08	\\
5.92661	4.24256e-08	\\
5.96002	4.19032e-08	\\
5.988	4.1465e-08	\\
6.0035	4.12219e-08	\\
6.02802	4.08369e-08	\\
6.07394	4.01146e-08	\\
6.09769	3.97404e-08	\\
6.12448	3.93181e-08	\\
6.15224	3.88801e-08	\\
6.19105	3.82673e-08	\\
6.22736	3.76938e-08	\\
6.25917	3.71914e-08	\\
6.31269	3.63461e-08	\\
6.35032	3.57522e-08	\\
6.39267	3.50845e-08	\\
6.40727	3.48544e-08	\\
6.46447	3.39545e-08	\\
6.48122	3.36914e-08	\\
6.519	3.30988e-08	\\
6.53461	3.28542e-08	\\
6.5647	3.23836e-08	\\
6.5933	3.19371e-08	\\
6.64122	3.11913e-08	\\
6.68137	3.05685e-08	\\
6.70655	3.01791e-08	\\
6.73582	2.97275e-08	\\
6.76645	2.92566e-08	\\
6.78333	2.89975e-08	\\
6.80961	2.85954e-08	\\
6.85015	2.79773e-08	\\
6.87144	2.7654e-08	\\
6.89538	2.72914e-08	\\
6.9168	2.6968e-08	\\
6.95005	2.64678e-08	\\
6.99506	2.57944e-08	\\
7.02438	2.53583e-08	\\
7.04214	2.50951e-08	\\
7.07241	2.46481e-08	\\
7.10173	2.42173e-08	\\
7.1255	2.38696e-08	\\
7.17333	2.31745e-08	\\
7.20271	2.27505e-08	\\
7.21908	2.25153e-08	\\
7.26382	2.18763e-08	\\
7.29549	2.14275e-08	\\
7.32505	2.10112e-08	\\
7.3725	2.03487e-08	\\
7.40642	1.98793e-08	\\
7.45068	1.92727e-08	\\
7.49681	1.86471e-08	\\
7.54219	1.80388e-08	\\
7.58626	1.7455e-08	\\
7.60773	1.71731e-08	\\
7.63563	1.68093e-08	\\
7.67934	1.62451e-08	\\
7.7144	1.57978e-08	\\
7.74902	1.53607e-08	\\
7.81312	1.45639e-08	\\
7.84339	1.41933e-08	\\
7.87181	1.38488e-08	\\
7.89968	1.35142e-08	\\
7.94355	1.29942e-08	\\
7.99478	1.23973e-08	\\
8.0338	1.19505e-08	\\
8.05909	1.16644e-08	\\
8.08868	1.13332e-08	\\
8.14235	1.07429e-08	\\
8.18281	1.03065e-08	\\
8.22709	9.83785e-09	\\
8.25756	9.52072e-09	\\
8.29441	9.14316e-09	\\
8.33781	8.70692e-09	\\
8.3573	8.51404e-09	\\
8.38705	8.2233e-09	\\
8.43292	7.78356e-09	\\
8.45956	7.53311e-09	\\
8.47307	7.40749e-09	\\
8.5	7.15983e-09	\\
8.52038	6.97491e-09	\\
8.56203	6.60374e-09	\\
8.59445	6.32106e-09	\\
8.62651	6.04707e-09	\\
8.67078	5.6777e-09	\\
8.71021	5.35767e-09	\\
8.74704	5.06643e-09	\\
8.78105	4.80405e-09	\\
8.81942	4.51579e-09	\\
8.84083	4.35854e-09	\\
8.88987	4.00813e-09	\\
8.91925	3.80471e-09	\\
8.98304	3.38011e-09	\\
9.02012	3.14415e-09	\\
9.06756	2.85399e-09	\\
9.10711	2.62227e-09	\\
9.13123	2.48556e-09	\\
9.16562	2.29658e-09	\\
9.18945	2.16984e-09	\\
9.24534	1.88609e-09	\\
9.30029	1.62571e-09	\\
9.33588	1.46701e-09	\\
9.37337	1.30837e-09	\\
9.43709	1.05891e-09	\\
9.47973	9.06305e-10	\\
9.54244	7.02942e-10	\\
9.57435	6.09171e-10	\\
9.61281	5.04858e-10	\\
9.64683	4.20627e-10	\\
9.6953	3.13699e-10	\\
9.77374	1.73451e-10	\\
9.81293	1.187e-10	\\
9.8489	7.75055e-11	\\
9.90572	3.01682e-11	\\
9.932	1.56689e-11	\\
};
\addlegendentry{$\norm{u_{N}(\sigma) - u_{\mathcal N}(\sigma)}_{\mathcal X}$};

\addplot [color=matlab2, densely dotted]
  table[row sep=crcr]{
0.100015	2.24398e-12	\\
0.100336	1.06848e-12	\\
0.100561	7.8253e-12	\\
0.101092	4.77778e-11	\\
0.101691	1.32002e-10	\\
0.102174	2.28966e-10	\\
0.10289	4.18234e-10	\\
0.103285	5.45107e-10	\\
0.103801	7.3337e-10	\\
0.104376	9.73124e-10	\\
0.104902	1.21843e-09	\\
0.10552	1.53655e-09	\\
0.106405	2.04698e-09	\\
0.106778	2.28054e-09	\\
0.107259	2.59627e-09	\\
0.107686	2.89065e-09	\\
0.107963	3.0881e-09	\\
0.108685	3.62779e-09	\\
0.109214	4.04338e-09	\\
0.109707	4.44653e-09	\\
0.110134	4.80704e-09	\\
0.111103	5.66235e-09	\\
0.111734	6.24463e-09	\\
0.112186	6.67393e-09	\\
0.112688	7.16209e-09	\\
0.113033	7.50366e-09	\\
0.113665	8.14311e-09	\\
0.114368	8.87394e-09	\\
0.11488	9.4177e-09	\\
0.115329	9.90145e-09	\\
0.115513	1.01024e-08	\\
0.116102	1.07512e-08	\\
0.116665	1.1382e-08	\\
0.116962	1.17189e-08	\\
0.117579	1.24269e-08	\\
0.11802	1.29386e-08	\\
0.118636	1.36613e-08	\\
0.119443	1.46224e-08	\\
0.119963	1.52503e-08	\\
0.120924	1.6422e-08	\\
0.121181	1.67382e-08	\\
0.122195	1.79979e-08	\\
0.122727	1.8664e-08	\\
0.12329	1.93747e-08	\\
0.123732	1.99347e-08	\\
0.12409	2.0389e-08	\\
0.124442	2.08381e-08	\\
0.125119	2.17049e-08	\\
0.125727	2.24876e-08	\\
0.126295	2.32199e-08	\\
0.127179	2.4365e-08	\\
0.127657	2.49864e-08	\\
0.128238	2.57425e-08	\\
0.128554	2.61547e-08	\\
0.129049	2.67995e-08	\\
0.129915	2.793e-08	\\
0.130325	2.84664e-08	\\
0.130921	2.92446e-08	\\
0.131481	2.99766e-08	\\
0.132223	3.0945e-08	\\
0.132941	3.18823e-08	\\
0.133901	3.31325e-08	\\
0.134572	3.4004e-08	\\
0.135143	3.47446e-08	\\
0.135771	3.55581e-08	\\
0.136557	3.65714e-08	\\
0.137199	3.73974e-08	\\
0.137877	3.82661e-08	\\
0.138381	3.89099e-08	\\
0.139077	3.97955e-08	\\
0.139834	4.07539e-08	\\
0.140459	4.1543e-08	\\
0.141017	4.22427e-08	\\
0.141424	4.27523e-08	\\
0.142416	4.39873e-08	\\
0.143505	4.533e-08	\\
0.144129	4.60944e-08	\\
0.144945	4.70879e-08	\\
0.145646	4.79351e-08	\\
0.146226	4.86307e-08	\\
0.146843	4.93673e-08	\\
0.147435	5.00699e-08	\\
0.148121	5.08782e-08	\\
0.148733	5.15947e-08	\\
0.1495	5.24846e-08	\\
0.150498	5.36328e-08	\\
0.151159	5.43855e-08	\\
0.151942	5.52697e-08	\\
0.152293	5.56629e-08	\\
0.153102	5.65637e-08	\\
0.154286	5.78654e-08	\\
0.154867	5.84966e-08	\\
0.155517	5.91971e-08	\\
0.156005	5.97187e-08	\\
0.156597	6.03481e-08	\\
0.157325	6.11137e-08	\\
0.158119	6.19411e-08	\\
0.159341	6.31951e-08	\\
0.159964	6.38268e-08	\\
0.160553	6.44182e-08	\\
0.161612	6.54681e-08	\\
0.162586	6.64192e-08	\\
0.163148	6.69605e-08	\\
0.163583	6.73769e-08	\\
0.164439	6.81886e-08	\\
0.165021	6.87337e-08	\\
0.165753	6.94123e-08	\\
0.166827	7.03927e-08	\\
0.167842	7.1304e-08	\\
0.168846	7.219e-08	\\
0.170115	7.3288e-08	\\
0.170973	7.40165e-08	\\
0.171892	7.47854e-08	\\
0.172206	7.5045e-08	\\
0.173045	7.5731e-08	\\
0.173717	7.62735e-08	\\
0.174618	7.69905e-08	\\
0.175431	7.76261e-08	\\
0.175763	7.78839e-08	\\
0.17639	7.83646e-08	\\
0.176848	7.87123e-08	\\
0.177562	7.92487e-08	\\
0.178582	8.00017e-08	\\
0.179392	8.05894e-08	\\
0.180059	8.10656e-08	\\
0.180937	8.16835e-08	\\
0.181749	8.22446e-08	\\
0.182615	8.28333e-08	\\
0.183481	8.34119e-08	\\
0.184193	8.38789e-08	\\
0.184908	8.43412e-08	\\
0.185697	8.48434e-08	\\
0.186241	8.51844e-08	\\
0.187253	8.58088e-08	\\
0.188144	8.63463e-08	\\
0.189283	8.7019e-08	\\
0.189833	8.73374e-08	\\
0.190917	8.79531e-08	\\
0.191531	8.82951e-08	\\
0.192787	8.89797e-08	\\
0.193147	8.9172e-08	\\
0.194284	8.97692e-08	\\
0.195764	9.05219e-08	\\
0.196783	9.10239e-08	\\
0.197564	9.14004e-08	\\
0.198592	9.18844e-08	\\
0.199754	9.24163e-08	\\
0.200825	9.28923e-08	\\
0.201526	9.31961e-08	\\
0.202452	9.35892e-08	\\
0.203152	9.38798e-08	\\
0.203499	9.40216e-08	\\
0.204247	9.43229e-08	\\
0.204955	9.46022e-08	\\
0.205628	9.48626e-08	\\
0.206233	9.5092e-08	\\
0.206993	9.53747e-08	\\
0.207526	9.55695e-08	\\
0.208113	9.57799e-08	\\
0.209284	9.61892e-08	\\
0.211021	9.6769e-08	\\
0.212628	9.72777e-08	\\
0.213591	9.75698e-08	\\
0.214658	9.78823e-08	\\
0.215148	9.80221e-08	\\
0.216583	9.84176e-08	\\
0.217329	9.86153e-08	\\
0.218106	9.88156e-08	\\
0.218764	9.89807e-08	\\
0.219373	9.91299e-08	\\
0.220098	9.9303e-08	\\
0.221431	9.96083e-08	\\
0.221823	9.9695e-08	\\
0.22244	9.98288e-08	\\
0.223583	1.00067e-07	\\
0.224506	1.00252e-07	\\
0.225701	1.00479e-07	\\
0.226956	1.00705e-07	\\
0.227942	1.00872e-07	\\
0.229057	1.01053e-07	\\
0.230206	1.01227e-07	\\
0.231229	1.01374e-07	\\
0.232482	1.01542e-07	\\
0.234038	1.01734e-07	\\
0.234983	1.01842e-07	\\
0.236193	1.0197e-07	\\
0.237082	1.02057e-07	\\
0.238066	1.02146e-07	\\
0.238912	1.02218e-07	\\
0.239986	1.02301e-07	\\
0.241948	1.02432e-07	\\
0.242831	1.02483e-07	\\
0.243463	1.02516e-07	\\
0.244878	1.0258e-07	\\
0.245524	1.02604e-07	\\
0.246599	1.0264e-07	\\
0.24792	1.02673e-07	\\
0.24935	1.02697e-07	\\
0.250667	1.02708e-07	\\
0.251826	1.0271e-07	\\
0.252669	1.02705e-07	\\
0.253519	1.02697e-07	\\
0.254469	1.02683e-07	\\
0.25498	1.02673e-07	\\
0.25552	1.02661e-07	\\
0.256499	1.02636e-07	\\
0.257579	1.02601e-07	\\
0.258834	1.02553e-07	\\
0.260244	1.0249e-07	\\
0.261773	1.02409e-07	\\
0.262476	1.02368e-07	\\
0.263972	1.02273e-07	\\
0.265131	1.02192e-07	\\
0.266681	1.02073e-07	\\
0.267848	1.01977e-07	\\
0.269029	1.01873e-07	\\
0.270687	1.01716e-07	\\
0.272064	1.01577e-07	\\
0.273021	1.01476e-07	\\
0.274058	1.01362e-07	\\
0.274715	1.01287e-07	\\
0.275995	1.01137e-07	\\
0.277002	1.01014e-07	\\
0.279084	1.00748e-07	\\
0.280201	1.00598e-07	\\
0.280852	1.00509e-07	\\
0.281772	1.0038e-07	\\
0.282503	1.00275e-07	\\
0.283784	1.00088e-07	\\
0.284679	9.99532e-08	\\
0.285512	9.98256e-08	\\
0.286971	9.95966e-08	\\
0.287939	9.94408e-08	\\
0.290064	9.90884e-08	\\
0.291414	9.88572e-08	\\
0.293254	9.85334e-08	\\
0.294542	9.83008e-08	\\
0.296175	9.79993e-08	\\
0.297304	9.77866e-08	\\
0.298542	9.75494e-08	\\
0.299978	9.72694e-08	\\
0.301643	9.69378e-08	\\
0.303399	9.6581e-08	\\
0.304693	9.63133e-08	\\
0.306072	9.60239e-08	\\
0.307168	9.57908e-08	\\
0.308907	9.54154e-08	\\
0.310409	9.50859e-08	\\
0.311453	9.48543e-08	\\
0.312886	9.45328e-08	\\
0.314384	9.41924e-08	\\
0.315802	9.38664e-08	\\
0.316826	9.36287e-08	\\
0.318472	9.32424e-08	\\
0.319978	9.2885e-08	\\
0.320952	9.2652e-08	\\
0.322137	9.23663e-08	\\
0.323487	9.2038e-08	\\
0.32546	9.15532e-08	\\
0.326333	9.13369e-08	\\
0.327803	9.09702e-08	\\
0.328939	9.0685e-08	\\
0.330106	9.039e-08	\\
0.331217	9.01074e-08	\\
0.332388	8.98081e-08	\\
0.334033	8.93845e-08	\\
0.336997	8.86136e-08	\\
0.337996	8.83516e-08	\\
0.339107	8.80589e-08	\\
0.34047	8.76984e-08	\\
0.341418	8.74463e-08	\\
0.343424	8.69104e-08	\\
0.345429	8.63709e-08	\\
0.346538	8.60712e-08	\\
0.347824	8.57224e-08	\\
0.349127	8.53676e-08	\\
0.351381	8.47509e-08	\\
0.352489	8.44463e-08	\\
0.353752	8.40982e-08	\\
0.355398	8.36432e-08	\\
0.357443	8.30756e-08	\\
0.358614	8.27495e-08	\\
0.360395	8.2252e-08	\\
0.362943	8.1538e-08	\\
0.364721	8.10379e-08	\\
0.367066	8.03767e-08	\\
0.368198	8.00566e-08	\\
0.370096	7.95194e-08	\\
0.371913	7.9004e-08	\\
0.374146	7.83695e-08	\\
0.375713	7.79235e-08	\\
0.378326	7.71787e-08	\\
0.379935	7.67199e-08	\\
0.381667	7.62255e-08	\\
0.383385	7.57348e-08	\\
0.385675	7.50804e-08	\\
0.387083	7.46781e-08	\\
0.38852	7.42673e-08	\\
0.391974	7.32802e-08	\\
0.39389	7.2733e-08	\\
0.395107	7.23856e-08	\\
0.396493	7.19903e-08	\\
0.397647	7.1661e-08	\\
0.399095	7.12485e-08	\\
0.400141	7.09506e-08	\\
0.401839	7.04672e-08	\\
0.403771	6.99179e-08	\\
0.405512	6.94239e-08	\\
0.407462	6.88709e-08	\\
0.409876	6.81881e-08	\\
0.411737	6.76626e-08	\\
0.413276	6.72289e-08	\\
0.414509	6.68817e-08	\\
0.416468	6.63313e-08	\\
0.417323	6.60915e-08	\\
0.418981	6.56272e-08	\\
0.420591	6.51775e-08	\\
0.421779	6.4846e-08	\\
0.423482	6.43718e-08	\\
0.425064	6.39326e-08	\\
0.426785	6.34556e-08	\\
0.428318	6.3032e-08	\\
0.431548	6.21425e-08	\\
0.434767	6.12609e-08	\\
0.437359	6.05546e-08	\\
0.439772	5.98999e-08	\\
0.441974	5.9305e-08	\\
0.44365	5.8854e-08	\\
0.445904	5.825e-08	\\
0.447634	5.77879e-08	\\
0.450898	5.6921e-08	\\
0.452192	5.65791e-08	\\
0.454729	5.59112e-08	\\
0.456512	5.54441e-08	\\
0.459154	5.47555e-08	\\
0.461828	5.40629e-08	\\
0.463977	5.35093e-08	\\
0.466542	5.28526e-08	\\
0.469142	5.21908e-08	\\
0.47098	5.17259e-08	\\
0.472419	5.13632e-08	\\
0.473777	5.10223e-08	\\
0.475743	5.05308e-08	\\
0.477949	4.99825e-08	\\
0.480112	4.94479e-08	\\
0.4835	4.86169e-08	\\
0.485685	4.8085e-08	\\
0.488222	4.74716e-08	\\
0.490474	4.6931e-08	\\
0.491845	4.66036e-08	\\
0.494792	4.59041e-08	\\
0.497299	4.53139e-08	\\
0.49979	4.47319e-08	\\
0.501809	4.42635e-08	\\
0.503775	4.381e-08	\\
0.50673	4.31336e-08	\\
0.508105	4.2821e-08	\\
0.511606	4.20313e-08	\\
0.515601	4.11411e-08	\\
0.519691	4.02416e-08	\\
0.521657	3.98135e-08	\\
0.523934	3.93214e-08	\\
0.525638	3.89555e-08	\\
0.527003	3.8664e-08	\\
0.529262	3.81844e-08	\\
0.531221	3.77717e-08	\\
0.53368	3.72574e-08	\\
0.536445	3.66846e-08	\\
0.538675	3.62266e-08	\\
0.541318	3.56885e-08	\\
0.543193	3.53101e-08	\\
0.544877	3.49721e-08	\\
0.548326	3.42866e-08	\\
0.551087	3.37442e-08	\\
0.552196	3.35279e-08	\\
0.554199	3.31395e-08	\\
0.5554	3.29079e-08	\\
0.558038	3.2403e-08	\\
0.559977	3.20352e-08	\\
0.563032	3.14611e-08	\\
0.566192	3.08745e-08	\\
0.568973	3.03642e-08	\\
0.57462	2.93448e-08	\\
0.577324	2.88647e-08	\\
0.581472	2.81386e-08	\\
0.583344	2.78147e-08	\\
0.586459	2.72815e-08	\\
0.588681	2.69053e-08	\\
0.592427	2.62787e-08	\\
0.595686	2.57416e-08	\\
0.596761	2.55661e-08	\\
0.600326	2.49896e-08	\\
0.601606	2.47847e-08	\\
0.604555	2.4317e-08	\\
0.607517	2.3853e-08	\\
0.610228	2.34337e-08	\\
0.612805	2.30397e-08	\\
0.614009	2.28572e-08	\\
0.617021	2.24046e-08	\\
0.619689	2.20089e-08	\\
0.62227	2.16305e-08	\\
0.626615	2.10034e-08	\\
0.629554	2.05861e-08	\\
0.632247	2.02086e-08	\\
0.638017	1.94156e-08	\\
0.639473	1.92188e-08	\\
0.643873	1.86322e-08	\\
0.648235	1.80627e-08	\\
0.651095	1.76956e-08	\\
0.654668	1.72441e-08	\\
0.657656	1.68725e-08	\\
0.659734	1.66173e-08	\\
0.663918	1.61111e-08	\\
0.666508	1.5803e-08	\\
0.670846	1.52959e-08	\\
0.674249	1.49056e-08	\\
0.677272	1.45646e-08	\\
0.68107	1.41435e-08	\\
0.683865	1.38389e-08	\\
0.686305	1.35765e-08	\\
0.691086	1.30721e-08	\\
0.692885	1.28855e-08	\\
0.695898	1.2577e-08	\\
0.69955	1.22098e-08	\\
0.702505	1.19177e-08	\\
0.705434	1.16328e-08	\\
0.706966	1.14857e-08	\\
0.710002	1.11977e-08	\\
0.71249	1.09652e-08	\\
0.718108	1.0452e-08	\\
0.722177	1.00903e-08	\\
0.724212	9.91248e-09	\\
0.727772	9.60638e-09	\\
0.730907	9.34201e-09	\\
0.733312	9.14246e-09	\\
0.737613	8.7925e-09	\\
0.743549	8.32396e-09	\\
0.746625	8.08761e-09	\\
0.748683	7.93196e-09	\\
0.750398	7.80373e-09	\\
0.754213	7.52336e-09	\\
0.757265	7.30373e-09	\\
0.760278	7.09096e-09	\\
0.762993	6.90273e-09	\\
0.768141	6.55462e-09	\\
0.771412	6.3393e-09	\\
0.774243	6.15671e-09	\\
0.778214	5.90618e-09	\\
0.781509	5.70331e-09	\\
0.785039	5.49091e-09	\\
0.789806	5.21204e-09	\\
0.793167	5.02094e-09	\\
0.796088	4.85843e-09	\\
0.80221	4.52864e-09	\\
0.805098	4.378e-09	\\
0.80767	4.24656e-09	\\
0.809874	4.13587e-09	\\
0.812891	3.98726e-09	\\
0.816306	3.82306e-09	\\
0.818988	3.6971e-09	\\
0.821809	3.56739e-09	\\
0.824666	3.43886e-09	\\
0.828553	3.26867e-09	\\
0.832199	3.11376e-09	\\
0.834358	3.02415e-09	\\
0.837803	2.88445e-09	\\
0.844224	2.6347e-09	\\
0.848839	2.46352e-09	\\
0.854473	2.26377e-09	\\
0.858483	2.12769e-09	\\
0.865074	1.9148e-09	\\
0.869705	1.77304e-09	\\
0.873873	1.6509e-09	\\
0.879156	1.50337e-09	\\
0.882406	1.41656e-09	\\
0.885564	1.33508e-09	\\
0.889228	1.244e-09	\\
0.891719	1.18419e-09	\\
0.893937	1.13235e-09	\\
0.898794	1.02344e-09	\\
0.905655	8.80162e-10	\\
0.909293	8.09101e-10	\\
0.914879	7.06504e-10	\\
0.91945	6.28303e-10	\\
0.922132	5.84766e-10	\\
0.924317	5.50592e-10	\\
0.930693	4.57286e-10	\\
0.935614	3.91725e-10	\\
0.940388	3.33372e-10	\\
0.943663	2.96278e-10	\\
0.947772	2.53057e-10	\\
0.955624	1.80534e-10	\\
0.958188	1.59657e-10	\\
0.965368	1.08353e-10	\\
0.969607	8.29194e-11	\\
0.974504	5.79228e-11	\\
0.980515	3.35252e-11	\\
0.98462	2.07607e-11	\\
0.98935	9.88276e-12	\\
0.99532	1.88738e-12	\\
0.998514	1.8674e-13	\\
1.00389	1.28819e-12	\\
1.00875	6.48193e-12	\\
1.01152	1.11796e-11	\\
1.01834	2.80665e-11	\\
1.02222	4.09396e-11	\\
1.02639	5.74188e-11	\\
1.02968	7.22593e-11	\\
1.03472	9.81352e-11	\\
1.03708	1.11538e-10	\\
1.0432	1.50087e-10	\\
1.04803	1.84132e-10	\\
1.05248	2.18424e-10	\\
1.05556	2.43671e-10	\\
1.06137	2.94763e-10	\\
1.06618	3.40394e-10	\\
1.07213	4.00789e-10	\\
1.07635	4.46259e-10	\\
1.08382	5.31939e-10	\\
1.08761	5.77844e-10	\\
1.09427	6.62595e-10	\\
1.10094	7.52281e-10	\\
1.10428	7.98933e-10	\\
1.11107	8.974e-10	\\
1.11445	9.48104e-10	\\
1.1179	1.00096e-09	\\
1.12264	1.07561e-09	\\
1.12783	1.15976e-09	\\
1.13184	1.22644e-09	\\
1.13883	1.34612e-09	\\
1.14441	1.44471e-09	\\
1.14871	1.5225e-09	\\
1.15328	1.60688e-09	\\
1.15726	1.6815e-09	\\
1.1602	1.73762e-09	\\
1.16443	1.81946e-09	\\
1.16996	1.92841e-09	\\
1.17577	2.04509e-09	\\
1.17954	2.12225e-09	\\
1.1832	2.19796e-09	\\
1.18791	2.29682e-09	\\
1.1929	2.40311e-09	\\
1.19916	2.53851e-09	\\
1.20767	2.72658e-09	\\
1.21307	2.84807e-09	\\
1.21808	2.96237e-09	\\
1.22241	3.06194e-09	\\
1.2294	3.22515e-09	\\
1.23769	3.42181e-09	\\
1.24156	3.51456e-09	\\
1.24652	3.63466e-09	\\
1.25395	3.81654e-09	\\
1.26076	3.98535e-09	\\
1.26748	4.15366e-09	\\
1.27252	4.28114e-09	\\
1.27741	4.40548e-09	\\
1.28349	4.56128e-09	\\
1.2914	4.76609e-09	\\
1.29571	4.87833e-09	\\
1.30192	5.04131e-09	\\
1.30717	5.17986e-09	\\
1.31215	5.3119e-09	\\
1.31742	5.45255e-09	\\
1.32457	5.64418e-09	\\
1.32922	5.76956e-09	\\
1.33684	5.9759e-09	\\
1.34381	6.16556e-09	\\
1.34972	6.32731e-09	\\
1.35563	6.48958e-09	\\
1.36125	6.64432e-09	\\
1.36887	6.85485e-09	\\
1.37549	7.03845e-09	\\
1.38533	7.31233e-09	\\
1.39279	7.52078e-09	\\
1.39763	7.65617e-09	\\
1.4055	7.87674e-09	\\
1.41028	8.01115e-09	\\
1.41439	8.1265e-09	\\
1.41835	8.23796e-09	\\
1.42524	8.43187e-09	\\
1.42924	8.54454e-09	\\
1.43495	8.7055e-09	\\
1.44057	8.86378e-09	\\
1.44784	9.06893e-09	\\
1.46077	9.43353e-09	\\
1.46666	9.59969e-09	\\
1.48032	9.98422e-09	\\
1.4846	1.01046e-08	\\
1.49294	1.03392e-08	\\
1.49869	1.05006e-08	\\
1.50401	1.06496e-08	\\
1.51435	1.09388e-08	\\
1.52143	1.11366e-08	\\
1.53151	1.1417e-08	\\
1.53696	1.15684e-08	\\
1.54633	1.18279e-08	\\
1.54953	1.19162e-08	\\
1.55879	1.21712e-08	\\
1.56552	1.23559e-08	\\
1.57553	1.26297e-08	\\
1.58196	1.28047e-08	\\
1.5874	1.29526e-08	\\
1.59513	1.31617e-08	\\
1.59983	1.32883e-08	\\
1.60538	1.34376e-08	\\
1.61302	1.36422e-08	\\
1.61949	1.38147e-08	\\
1.62707	1.40161e-08	\\
1.63344	1.41843e-08	\\
1.64378	1.44564e-08	\\
1.6555	1.47621e-08	\\
1.66625	1.50402e-08	\\
1.67451	1.52526e-08	\\
1.68825	1.5603e-08	\\
1.69695	1.58225e-08	\\
1.7046	1.60145e-08	\\
1.7117	1.61917e-08	\\
1.72369	1.64884e-08	\\
1.72862	1.66096e-08	\\
1.73709	1.68162e-08	\\
1.7441	1.69862e-08	\\
1.75046	1.71395e-08	\\
1.75613	1.72755e-08	\\
1.76015	1.73713e-08	\\
1.76765	1.75493e-08	\\
1.77208	1.76538e-08	\\
1.78519	1.79604e-08	\\
1.79739	1.82421e-08	\\
1.80889	1.85046e-08	\\
1.81709	1.86897e-08	\\
1.82862	1.89475e-08	\\
1.83824	1.91601e-08	\\
1.84642	1.9339e-08	\\
1.85506	1.95264e-08	\\
1.86419	1.97223e-08	\\
1.87	1.9846e-08	\\
1.87543	1.99609e-08	\\
1.88149	2.0088e-08	\\
1.88978	2.02608e-08	\\
1.89788	2.04278e-08	\\
1.90767	2.06275e-08	\\
1.91344	2.07442e-08	\\
1.92585	2.09922e-08	\\
1.93516	2.11756e-08	\\
1.94096	2.1289e-08	\\
1.95279	2.15174e-08	\\
1.96239	2.17003e-08	\\
1.97105	2.18632e-08	\\
1.98137	2.20549e-08	\\
1.9862	2.21438e-08	\\
1.99367	2.228e-08	\\
2.00146	2.24206e-08	\\
2.00946	2.25635e-08	\\
2.01415	2.26465e-08	\\
2.02919	2.29089e-08	\\
2.03299	2.29744e-08	\\
2.04513	2.31811e-08	\\
2.06157	2.34553e-08	\\
2.07762	2.37165e-08	\\
2.08203	2.37871e-08	\\
2.09592	2.40066e-08	\\
2.09902	2.40551e-08	\\
2.1106	2.42336e-08	\\
2.11849	2.43535e-08	\\
2.12568	2.44614e-08	\\
2.13566	2.46091e-08	\\
2.14361	2.47251e-08	\\
2.15612	2.49046e-08	\\
2.16166	2.49828e-08	\\
2.17471	2.51645e-08	\\
2.18536	2.53097e-08	\\
2.19598	2.54519e-08	\\
2.20653	2.55907e-08	\\
2.22139	2.57817e-08	\\
2.22653	2.58466e-08	\\
2.23919	2.6004e-08	\\
2.24707	2.61001e-08	\\
2.25385	2.61816e-08	\\
2.26292	2.62892e-08	\\
2.27041	2.63766e-08	\\
2.27992	2.64859e-08	\\
2.29442	2.66486e-08	\\
2.30608	2.67761e-08	\\
2.31854	2.69092e-08	\\
2.3339	2.70687e-08	\\
2.35015	2.7232e-08	\\
2.35749	2.73039e-08	\\
2.3632	2.73591e-08	\\
2.36934	2.74177e-08	\\
2.38516	2.75651e-08	\\
2.39573	2.76609e-08	\\
2.40215	2.77178e-08	\\
2.41704	2.78469e-08	\\
2.43092	2.79632e-08	\\
2.44218	2.80548e-08	\\
2.45123	2.81267e-08	\\
2.46866	2.82607e-08	\\
2.47758	2.8327e-08	\\
2.4849	2.83803e-08	\\
2.49979	2.84856e-08	\\
2.5111	2.85629e-08	\\
2.52495	2.86544e-08	\\
2.53541	2.87211e-08	\\
2.54724	2.87943e-08	\\
2.55612	2.88475e-08	\\
2.57322	2.89463e-08	\\
2.58161	2.89929e-08	\\
2.59033	2.904e-08	\\
2.60481	2.91154e-08	\\
2.61999	2.91907e-08	\\
2.63512	2.92619e-08	\\
2.64529	2.93078e-08	\\
2.65716	2.93591e-08	\\
2.66849	2.9406e-08	\\
2.68218	2.946e-08	\\
2.69601	2.95116e-08	\\
2.70992	2.95605e-08	\\
2.71965	2.9593e-08	\\
2.73612	2.96447e-08	\\
2.7522	2.96914e-08	\\
2.77408	2.97489e-08	\\
2.79011	2.97867e-08	\\
2.80601	2.98206e-08	\\
2.82521	2.9857e-08	\\
2.83653	2.98761e-08	\\
2.85437	2.99029e-08	\\
2.86762	2.992e-08	\\
2.88104	2.99351e-08	\\
2.88872	2.99427e-08	\\
2.90365	2.99553e-08	\\
2.91603	2.99638e-08	\\
2.93079	2.99714e-08	\\
2.94329	2.99758e-08	\\
2.96475	2.9979e-08	\\
2.97502	2.99787e-08	\\
2.98864	2.99764e-08	\\
3.00315	2.99717e-08	\\
3.01437	2.99665e-08	\\
3.02547	2.996e-08	\\
3.03958	2.99498e-08	\\
3.05821	2.99331e-08	\\
3.07135	2.99192e-08	\\
3.07922	2.991e-08	\\
3.09105	2.9895e-08	\\
3.09501	2.98897e-08	\\
3.11529	2.986e-08	\\
3.12956	2.98367e-08	\\
3.1364	2.98249e-08	\\
3.14312	2.98128e-08	\\
3.15484	2.97908e-08	\\
3.16956	2.97613e-08	\\
3.18858	2.97203e-08	\\
3.206	2.96799e-08	\\
3.21873	2.96488e-08	\\
3.23985	2.95941e-08	\\
3.2588	2.95419e-08	\\
3.27349	2.94994e-08	\\
3.29053	2.94479e-08	\\
3.30883	2.93902e-08	\\
3.32326	2.93429e-08	\\
3.34255	2.92772e-08	\\
3.35805	2.92224e-08	\\
3.37076	2.91763e-08	\\
3.38273	2.91318e-08	\\
3.39877	2.90706e-08	\\
3.40947	2.90288e-08	\\
3.42309	2.89746e-08	\\
3.44559	2.88823e-08	\\
3.4579	2.88304e-08	\\
3.47785	2.87444e-08	\\
3.49421	2.8672e-08	\\
3.50401	2.86279e-08	\\
3.51176	2.85927e-08	\\
3.52382	2.85371e-08	\\
3.53656	2.84775e-08	\\
3.55629	2.83835e-08	\\
3.5692	2.83208e-08	\\
3.58819	2.8227e-08	\\
3.6019	2.81581e-08	\\
3.62605	2.80344e-08	\\
3.64093	2.79568e-08	\\
3.66119	2.78494e-08	\\
3.67574	2.77712e-08	\\
3.69356	2.7674e-08	\\
3.70987	2.75838e-08	\\
3.7214	2.75195e-08	\\
3.73114	2.74646e-08	\\
3.75029	2.73556e-08	\\
3.76804	2.72533e-08	\\
3.78452	2.71571e-08	\\
3.81006	2.70061e-08	\\
3.82486	2.69175e-08	\\
3.84733	2.67816e-08	\\
3.86475	2.66749e-08	\\
3.89033	2.65166e-08	\\
3.9009	2.64505e-08	\\
3.92121	2.63226e-08	\\
3.93057	2.62633e-08	\\
3.94864	2.6148e-08	\\
3.96642	2.60336e-08	\\
3.98334	2.59239e-08	\\
3.9953	2.5846e-08	\\
4.0092	2.57549e-08	\\
4.01971	2.56857e-08	\\
4.04036	2.5549e-08	\\
4.06313	2.5397e-08	\\
4.08001	2.52836e-08	\\
4.10293	2.51286e-08	\\
4.12102	2.50054e-08	\\
4.13415	2.49157e-08	\\
4.15229	2.47911e-08	\\
4.17638	2.46247e-08	\\
4.19415	2.45014e-08	\\
4.20661	2.44145e-08	\\
4.21883	2.43291e-08	\\
4.24168	2.41687e-08	\\
4.27087	2.39627e-08	\\
4.2816	2.38867e-08	\\
4.30361	2.37303e-08	\\
4.33567	2.35013e-08	\\
4.35606	2.33551e-08	\\
4.36785	2.32703e-08	\\
4.39144	2.31003e-08	\\
4.41208	2.29511e-08	\\
4.42695	2.28434e-08	\\
4.43477	2.27867e-08	\\
4.44879	2.26848e-08	\\
4.472	2.25159e-08	\\
4.51114	2.22303e-08	\\
4.52652	2.21178e-08	\\
4.54622	2.19734e-08	\\
4.56374	2.1845e-08	\\
4.58013	2.17246e-08	\\
4.60286	2.15576e-08	\\
4.62445	2.13988e-08	\\
4.64708	2.12322e-08	\\
4.67292	2.10418e-08	\\
4.68998	2.09161e-08	\\
4.71447	2.07356e-08	\\
4.73037	2.06183e-08	\\
4.74845	2.04851e-08	\\
4.7763	2.02797e-08	\\
4.81018	2.003e-08	\\
4.84206	1.97953e-08	\\
4.8553	1.96979e-08	\\
4.87923	1.95219e-08	\\
4.8989	1.93774e-08	\\
4.92062	1.92179e-08	\\
4.94125	1.90666e-08	\\
4.96225	1.89129e-08	\\
4.97405	1.88265e-08	\\
4.99798	1.86517e-08	\\
5.01152	1.85529e-08	\\
5.02698	1.84402e-08	\\
5.04296	1.83239e-08	\\
5.07515	1.809e-08	\\
5.10643	1.78633e-08	\\
5.12456	1.77323e-08	\\
5.13305	1.7671e-08	\\
5.16698	1.74267e-08	\\
5.1908	1.72556e-08	\\
5.20705	1.71392e-08	\\
5.24076	1.68984e-08	\\
5.24932	1.68375e-08	\\
5.27866	1.6629e-08	\\
5.30212	1.64628e-08	\\
5.32995	1.62664e-08	\\
5.35464	1.60929e-08	\\
5.38854	1.58556e-08	\\
5.41724	1.56555e-08	\\
5.436	1.55254e-08	\\
5.48011	1.52207e-08	\\
5.51535	1.49789e-08	\\
5.54338	1.47876e-08	\\
5.55911	1.46807e-08	\\
5.58382	1.45134e-08	\\
5.62636	1.42271e-08	\\
5.64758	1.40851e-08	\\
5.67398	1.39093e-08	\\
5.70173	1.37255e-08	\\
5.74764	1.34237e-08	\\
5.77259	1.32609e-08	\\
5.7944	1.31193e-08	\\
5.8163	1.29777e-08	\\
5.84153	1.28156e-08	\\
5.87032	1.26316e-08	\\
5.89715	1.24612e-08	\\
5.92661	1.22753e-08	\\
5.96002	1.2066e-08	\\
5.988	1.18921e-08	\\
6.0035	1.17962e-08	\\
6.02802	1.16453e-08	\\
6.07394	1.13651e-08	\\
6.09769	1.12215e-08	\\
6.12448	1.10605e-08	\\
6.15224	1.08949e-08	\\
6.19105	1.06654e-08	\\
6.22736	1.04528e-08	\\
6.25917	1.02683e-08	\\
6.31269	9.96154e-09	\\
6.35032	9.74866e-09	\\
6.39267	9.51184e-09	\\
6.40727	9.43085e-09	\\
6.46447	9.11702e-09	\\
6.48122	9.02613e-09	\\
6.519	8.82286e-09	\\
6.53461	8.73951e-09	\\
6.5647	8.5801e-09	\\
6.5933	8.42996e-09	\\
6.64122	8.18151e-09	\\
6.68137	7.97626e-09	\\
6.70655	7.84896e-09	\\
6.73582	7.70228e-09	\\
6.76645	7.55039e-09	\\
6.78333	7.46731e-09	\\
6.80961	7.33899e-09	\\
6.85015	7.14332e-09	\\
6.87144	7.04171e-09	\\
6.89538	6.92832e-09	\\
6.9168	6.82772e-09	\\
6.95005	6.67306e-09	\\
6.99506	6.46671e-09	\\
7.02438	6.33416e-09	\\
7.04214	6.25457e-09	\\
7.07241	6.12015e-09	\\
7.10173	5.99145e-09	\\
7.1255	5.88815e-09	\\
7.17333	5.68326e-09	\\
7.20271	5.55932e-09	\\
7.21908	5.49089e-09	\\
7.26382	5.30621e-09	\\
7.29549	5.17751e-09	\\
7.32505	5.05891e-09	\\
7.3725	4.87166e-09	\\
7.40642	4.74009e-09	\\
7.45068	4.57139e-09	\\
7.49681	4.39901e-09	\\
7.54219	4.23292e-09	\\
7.58626	4.07493e-09	\\
7.60773	3.99913e-09	\\
7.63563	3.90176e-09	\\
7.67934	3.75182e-09	\\
7.7144	3.63385e-09	\\
7.74902	3.51934e-09	\\
7.81312	3.31252e-09	\\
7.84339	3.21719e-09	\\
7.87181	3.12906e-09	\\
7.89968	3.04391e-09	\\
7.94355	2.91243e-09	\\
7.99478	2.76284e-09	\\
8.0338	2.65176e-09	\\
8.05909	2.58106e-09	\\
8.08868	2.49962e-09	\\
8.14235	2.35551e-09	\\
8.18281	2.24989e-09	\\
8.22709	2.13728e-09	\\
8.25756	2.06158e-09	\\
8.29441	1.97198e-09	\\
8.33781	1.86917e-09	\\
8.3573	1.82396e-09	\\
8.38705	1.7561e-09	\\
8.43292	1.65411e-09	\\
8.45956	1.59638e-09	\\
8.47307	1.56752e-09	\\
8.5	1.51082e-09	\\
8.52038	1.46866e-09	\\
8.56203	1.38445e-09	\\
8.59445	1.32071e-09	\\
8.62651	1.25926e-09	\\
8.67078	1.17693e-09	\\
8.71021	1.10609e-09	\\
8.74704	1.04201e-09	\\
8.78105	9.84611e-10	\\
8.81942	9.21918e-10	\\
8.84083	8.8788e-10	\\
8.88987	8.12454e-10	\\
8.91925	7.68935e-10	\\
8.98304	6.7876e-10	\\
9.02012	6.29043e-10	\\
9.06756	5.68302e-10	\\
9.10711	5.20119e-10	\\
9.13123	4.9183e-10	\\
9.16562	4.52898e-10	\\
9.18945	4.26905e-10	\\
9.24534	3.69055e-10	\\
9.30029	3.16412e-10	\\
9.33588	2.84539e-10	\\
9.37337	2.52854e-10	\\
9.43709	2.03396e-10	\\
9.47973	1.73374e-10	\\
9.54244	1.33671e-10	\\
9.57435	1.15491e-10	\\
9.61281	9.53667e-11	\\
9.64683	7.92025e-11	\\
9.6953	5.88023e-11	\\
9.77374	3.22754e-11	\\
9.81293	2.20065e-11	\\
9.8489	1.43213e-11	\\
9.90572	5.54529e-12	\\
9.932	2.87281e-12	\\
};
\addlegendentry{$\Delta_{N}(\sigma)$};

\end{axis}
\end{tikzpicture}%

        \end{subfigure}
        \hfill
        \begin{subfigure}[b]{0.45\textwidth}
            % This file was created by matlab2tikz v0.4.6 running on MATLAB 8.1.
% Copyright (c) 2008--2014, Nico Schlömer <nico.schloemer@gmail.com>
% All rights reserved.
% Minimal pgfplots version: 1.3
%
% The latest updates can be retrieved from
%   http://www.mathworks.com/matlabcentral/fileexchange/22022-matlab2tikz
% where you can also make suggestions and rate matlab2tikz.
%
\begin{tikzpicture}

\begin{axis}[%
width=10cm,
height=7cm,
scale only axis,
xmin=0,
xmax=10,
ymode=log,
ymin=1e-16,
ymax=1,
yminorticks=true,
ultra thick,
xlabel={One dimensional parameter space},
legend style={at={(1,0.03)},anchor=south east,legend cell align=left,align=left,fill=none,draw=none}
]
\addplot [color=matlab1,solid]
  table[row sep=crcr]{
0.100015	7.75678e-15	\\
0.100336	5.12589e-15	\\
0.100561	9.29349e-15	\\
0.101092	3.00772e-15	\\
0.101691	1.01468e-15	\\
0.102174	7.80842e-15	\\
0.10289	1.51626e-14	\\
0.103285	2.0435e-14	\\
0.103801	3.22935e-14	\\
0.104376	4.21669e-14	\\
0.104902	5.40473e-14	\\
0.10552	7.07678e-14	\\
0.106405	9.42049e-14	\\
0.106778	1.03373e-13	\\
0.107259	1.18081e-13	\\
0.107686	1.28375e-13	\\
0.107963	1.37803e-13	\\
0.108685	1.57662e-13	\\
0.109214	1.74327e-13	\\
0.109707	1.90897e-13	\\
0.110134	2.02765e-13	\\
0.111103	2.32068e-13	\\
0.111734	2.5179e-13	\\
0.112186	2.67209e-13	\\
0.112688	2.80916e-13	\\
0.113033	2.9399e-13	\\
0.113665	3.13998e-13	\\
0.114368	3.36522e-13	\\
0.11488	3.49076e-13	\\
0.115329	3.6224e-13	\\
0.115513	3.66649e-13	\\
0.116102	3.82946e-13	\\
0.116665	4.01421e-13	\\
0.116962	4.09564e-13	\\
0.117579	4.24672e-13	\\
0.11802	4.34649e-13	\\
0.118636	4.51374e-13	\\
0.119443	4.70878e-13	\\
0.119963	4.81025e-13	\\
0.120924	5.02437e-13	\\
0.121181	5.07878e-13	\\
0.122195	5.29943e-13	\\
0.122727	5.38295e-13	\\
0.12329	5.50175e-13	\\
0.123732	5.57942e-13	\\
0.12409	5.63094e-13	\\
0.124442	5.68844e-13	\\
0.125119	5.79368e-13	\\
0.125727	5.87532e-13	\\
0.126295	5.95193e-13	\\
0.127179	6.08658e-13	\\
0.127657	6.12491e-13	\\
0.128238	6.18971e-13	\\
0.128554	6.20995e-13	\\
0.129049	6.2637e-13	\\
0.129915	6.31297e-13	\\
0.130325	6.37733e-13	\\
0.130921	6.39162e-13	\\
0.131481	6.43344e-13	\\
0.132223	6.46793e-13	\\
0.132941	6.4896e-13	\\
0.133901	6.52344e-13	\\
0.134572	6.52316e-13	\\
0.135143	6.54101e-13	\\
0.135771	6.54735e-13	\\
0.136557	6.5223e-13	\\
0.137199	6.52714e-13	\\
0.137877	6.50477e-13	\\
0.138381	6.49068e-13	\\
0.139077	6.46681e-13	\\
0.139834	6.44892e-13	\\
0.140459	6.41738e-13	\\
0.141017	6.39165e-13	\\
0.141424	6.36679e-13	\\
0.142416	6.28601e-13	\\
0.143505	6.20956e-13	\\
0.144129	6.14565e-13	\\
0.144945	6.09247e-13	\\
0.145646	6.02658e-13	\\
0.146226	5.9663e-13	\\
0.146843	5.89632e-13	\\
0.147435	5.83769e-13	\\
0.148121	5.75843e-13	\\
0.148733	5.68681e-13	\\
0.1495	5.61187e-13	\\
0.150498	5.47808e-13	\\
0.151159	5.40886e-13	\\
0.151942	5.29758e-13	\\
0.152293	5.26497e-13	\\
0.153102	5.15917e-13	\\
0.154286	4.98938e-13	\\
0.154867	4.91506e-13	\\
0.155517	4.81278e-13	\\
0.156005	4.75005e-13	\\
0.156597	4.66285e-13	\\
0.157325	4.56623e-13	\\
0.158119	4.45032e-13	\\
0.159341	4.26719e-13	\\
0.159964	4.16841e-13	\\
0.160553	4.08762e-13	\\
0.161612	3.93861e-13	\\
0.162586	3.78817e-13	\\
0.163148	3.69632e-13	\\
0.163583	3.64683e-13	\\
0.164439	3.5141e-13	\\
0.165021	3.42922e-13	\\
0.165753	3.32625e-13	\\
0.166827	3.16894e-13	\\
0.167842	3.01214e-13	\\
0.168846	2.87608e-13	\\
0.170115	2.70607e-13	\\
0.170973	2.58415e-13	\\
0.171892	2.4543e-13	\\
0.172206	2.41825e-13	\\
0.173045	2.30015e-13	\\
0.173717	2.22048e-13	\\
0.174618	2.10141e-13	\\
0.175431	1.99126e-13	\\
0.175763	1.94844e-13	\\
0.17639	1.8701e-13	\\
0.176848	1.82052e-13	\\
0.177562	1.73409e-13	\\
0.178582	1.63312e-13	\\
0.179392	1.53806e-13	\\
0.180059	1.45947e-13	\\
0.180937	1.3606e-13	\\
0.181749	1.27638e-13	\\
0.182615	1.18756e-13	\\
0.183481	1.10225e-13	\\
0.184193	1.02979e-13	\\
0.184908	9.64223e-14	\\
0.185697	9.04977e-14	\\
0.186241	8.51583e-14	\\
0.187253	7.72308e-14	\\
0.188144	6.95686e-14	\\
0.189283	6.08244e-14	\\
0.189833	5.62267e-14	\\
0.190917	4.88507e-14	\\
0.191531	4.56245e-14	\\
0.192787	3.74513e-14	\\
0.193147	3.62094e-14	\\
0.194284	2.99786e-14	\\
0.195764	2.24488e-14	\\
0.196783	1.82797e-14	\\
0.197564	1.48287e-14	\\
0.198592	1.10914e-14	\\
0.199754	8.60381e-15	\\
0.200825	5.01676e-15	\\
0.201526	4.75913e-15	\\
0.202452	2.69774e-15	\\
0.203152	1.70647e-15	\\
0.203499	1.5882e-15	\\
0.204247	1.14372e-15	\\
0.204955	7.96746e-16	\\
0.205628	5.6523e-18	\\
0.206233	1.42759e-16	\\
0.206993	1.00905e-15	\\
0.207526	9.87758e-16	\\
0.208113	9.73585e-16	\\
0.209284	3.07008e-15	\\
0.211021	6.28204e-15	\\
0.212628	1.08803e-14	\\
0.213591	1.35309e-14	\\
0.214658	1.76626e-14	\\
0.215148	1.91769e-14	\\
0.216583	2.58342e-14	\\
0.217329	3.04116e-14	\\
0.218106	3.40361e-14	\\
0.218764	3.72629e-14	\\
0.219373	4.1198e-14	\\
0.220098	4.51546e-14	\\
0.221431	5.42284e-14	\\
0.221823	5.61926e-14	\\
0.22244	6.07868e-14	\\
0.223583	6.88712e-14	\\
0.224506	7.64347e-14	\\
0.225701	8.53862e-14	\\
0.226956	9.65672e-14	\\
0.227942	1.05126e-13	\\
0.229057	1.15052e-13	\\
0.230206	1.26263e-13	\\
0.231229	1.36234e-13	\\
0.232482	1.49226e-13	\\
0.234038	1.66566e-13	\\
0.234983	1.77013e-13	\\
0.236193	1.90273e-13	\\
0.237082	2.01022e-13	\\
0.238066	2.13187e-13	\\
0.238912	2.2389e-13	\\
0.239986	2.37074e-13	\\
0.241948	2.62432e-13	\\
0.242831	2.74385e-13	\\
0.243463	2.82747e-13	\\
0.244878	3.02367e-13	\\
0.245524	3.1229e-13	\\
0.246599	3.27106e-13	\\
0.24792	3.4566e-13	\\
0.24935	3.67902e-13	\\
0.250667	3.88021e-13	\\
0.251826	4.05972e-13	\\
0.252669	4.18744e-13	\\
0.253519	4.32507e-13	\\
0.254469	4.47264e-13	\\
0.25498	4.55342e-13	\\
0.25552	4.64286e-13	\\
0.256499	4.80202e-13	\\
0.257579	4.98784e-13	\\
0.258834	5.19498e-13	\\
0.260244	5.44085e-13	\\
0.261773	5.70274e-13	\\
0.262476	5.82755e-13	\\
0.263972	6.08882e-13	\\
0.265131	6.29545e-13	\\
0.266681	6.56794e-13	\\
0.267848	6.78451e-13	\\
0.269029	6.99568e-13	\\
0.270687	7.30077e-13	\\
0.272064	7.56104e-13	\\
0.273021	7.73648e-13	\\
0.274058	7.9265e-13	\\
0.274715	8.0562e-13	\\
0.275995	8.30022e-13	\\
0.277002	8.48683e-13	\\
0.279084	8.88868e-13	\\
0.280201	9.09571e-13	\\
0.280852	9.21967e-13	\\
0.281772	9.4048e-13	\\
0.282503	9.54287e-13	\\
0.283784	9.79201e-13	\\
0.284679	9.96345e-13	\\
0.285512	1.01282e-12	\\
0.286971	1.04019e-12	\\
0.287939	1.05976e-12	\\
0.290064	1.10125e-12	\\
0.291414	1.12737e-12	\\
0.293254	1.16313e-12	\\
0.294542	1.18873e-12	\\
0.296175	1.22043e-12	\\
0.297304	1.24255e-12	\\
0.298542	1.26707e-12	\\
0.299978	1.29515e-12	\\
0.301643	1.32764e-12	\\
0.303399	1.36174e-12	\\
0.304693	1.38703e-12	\\
0.306072	1.41346e-12	\\
0.307168	1.4343e-12	\\
0.308907	1.46875e-12	\\
0.310409	1.49747e-12	\\
0.311453	1.51711e-12	\\
0.312886	1.54467e-12	\\
0.314384	1.57296e-12	\\
0.315802	1.59933e-12	\\
0.316826	1.61914e-12	\\
0.318472	1.64967e-12	\\
0.319978	1.67784e-12	\\
0.320952	1.69572e-12	\\
0.322137	1.71778e-12	\\
0.323487	1.74223e-12	\\
0.32546	1.77765e-12	\\
0.326333	1.79391e-12	\\
0.327803	1.82019e-12	\\
0.328939	1.84001e-12	\\
0.330106	1.86122e-12	\\
0.331217	1.88063e-12	\\
0.332388	1.90089e-12	\\
0.334033	1.92926e-12	\\
0.336997	1.98016e-12	\\
0.337996	1.99698e-12	\\
0.339107	2.01543e-12	\\
0.34047	2.03834e-12	\\
0.341418	2.054e-12	\\
0.343424	2.08665e-12	\\
0.345429	2.11916e-12	\\
0.346538	2.13659e-12	\\
0.347824	2.15673e-12	\\
0.349127	2.17722e-12	\\
0.351381	2.21183e-12	\\
0.352489	2.22891e-12	\\
0.353752	2.24739e-12	\\
0.355398	2.272e-12	\\
0.357443	2.30227e-12	\\
0.358614	2.31884e-12	\\
0.360395	2.34396e-12	\\
0.362943	2.37964e-12	\\
0.364721	2.40424e-12	\\
0.367066	2.43543e-12	\\
0.368198	2.45018e-12	\\
0.370096	2.4748e-12	\\
0.371913	2.49795e-12	\\
0.374146	2.52544e-12	\\
0.375713	2.54455e-12	\\
0.378326	2.57557e-12	\\
0.379935	2.59393e-12	\\
0.381667	2.61328e-12	\\
0.383385	2.63254e-12	\\
0.385675	2.65703e-12	\\
0.387083	2.67202e-12	\\
0.38852	2.68673e-12	\\
0.391974	2.721e-12	\\
0.39389	2.73939e-12	\\
0.395107	2.75089e-12	\\
0.396493	2.76336e-12	\\
0.397647	2.77389e-12	\\
0.399095	2.78653e-12	\\
0.400141	2.79538e-12	\\
0.401839	2.80991e-12	\\
0.403771	2.8255e-12	\\
0.405512	2.8396e-12	\\
0.407462	2.85453e-12	\\
0.409876	2.87215e-12	\\
0.411737	2.88515e-12	\\
0.413276	2.89576e-12	\\
0.414509	2.90396e-12	\\
0.416468	2.91617e-12	\\
0.417323	2.92185e-12	\\
0.418981	2.93182e-12	\\
0.420591	2.94111e-12	\\
0.421779	2.94783e-12	\\
0.423482	2.95694e-12	\\
0.425064	2.96521e-12	\\
0.426785	2.97387e-12	\\
0.428318	2.98122e-12	\\
0.431548	2.99573e-12	\\
0.434767	3.00882e-12	\\
0.437359	3.01819e-12	\\
0.439772	3.02617e-12	\\
0.441974	3.03281e-12	\\
0.44365	3.03767e-12	\\
0.445904	3.04328e-12	\\
0.447634	3.04748e-12	\\
0.450898	3.05397e-12	\\
0.452192	3.05616e-12	\\
0.454729	3.0605e-12	\\
0.456512	3.06248e-12	\\
0.459154	3.06526e-12	\\
0.461828	3.06739e-12	\\
0.463977	3.06811e-12	\\
0.466542	3.06879e-12	\\
0.469142	3.06831e-12	\\
0.47098	3.06764e-12	\\
0.472419	3.06674e-12	\\
0.473777	3.06605e-12	\\
0.475743	3.06429e-12	\\
0.477949	3.0622e-12	\\
0.480112	3.0593e-12	\\
0.4835	3.05398e-12	\\
0.485685	3.04988e-12	\\
0.488222	3.04478e-12	\\
0.490474	3.03956e-12	\\
0.491845	3.03603e-12	\\
0.494792	3.02842e-12	\\
0.497299	3.02124e-12	\\
0.49979	3.01352e-12	\\
0.501809	3.00677e-12	\\
0.503775	2.99996e-12	\\
0.50673	2.98905e-12	\\
0.508105	2.98353e-12	\\
0.511606	2.96946e-12	\\
0.515601	2.95209e-12	\\
0.519691	2.93294e-12	\\
0.521657	2.92317e-12	\\
0.523934	2.91185e-12	\\
0.525638	2.90307e-12	\\
0.527003	2.89583e-12	\\
0.529262	2.88356e-12	\\
0.531221	2.87281e-12	\\
0.53368	2.85872e-12	\\
0.536445	2.84278e-12	\\
0.538675	2.82917e-12	\\
0.541318	2.81295e-12	\\
0.543193	2.80124e-12	\\
0.544877	2.79041e-12	\\
0.548326	2.76799e-12	\\
0.551087	2.7494e-12	\\
0.552196	2.74192e-12	\\
0.554199	2.72794e-12	\\
0.5554	2.71998e-12	\\
0.558038	2.70122e-12	\\
0.559977	2.68772e-12	\\
0.563032	2.66537e-12	\\
0.566192	2.64182e-12	\\
0.568973	2.62084e-12	\\
0.57462	2.57732e-12	\\
0.577324	2.55608e-12	\\
0.581472	2.52293e-12	\\
0.583344	2.50757e-12	\\
0.586459	2.48217e-12	\\
0.588681	2.4637e-12	\\
0.592427	2.43251e-12	\\
0.595686	2.405e-12	\\
0.596761	2.39573e-12	\\
0.600326	2.365e-12	\\
0.601606	2.35425e-12	\\
0.604555	2.32845e-12	\\
0.607517	2.3026e-12	\\
0.610228	2.27881e-12	\\
0.612805	2.25595e-12	\\
0.614009	2.24537e-12	\\
0.617021	2.21832e-12	\\
0.619689	2.19443e-12	\\
0.62227	2.17118e-12	\\
0.626615	2.13186e-12	\\
0.629554	2.10474e-12	\\
0.632247	2.08032e-12	\\
0.638017	2.02727e-12	\\
0.639473	2.01402e-12	\\
0.643873	1.97343e-12	\\
0.648235	1.9331e-12	\\
0.651095	1.90667e-12	\\
0.654668	1.87345e-12	\\
0.657656	1.846e-12	\\
0.659734	1.82662e-12	\\
0.663918	1.78791e-12	\\
0.666508	1.76396e-12	\\
0.670846	1.7239e-12	\\
0.674249	1.69261e-12	\\
0.677272	1.66473e-12	\\
0.68107	1.6299e-12	\\
0.683865	1.60439e-12	\\
0.686305	1.58209e-12	\\
0.691086	1.5387e-12	\\
0.692885	1.52248e-12	\\
0.695898	1.4952e-12	\\
0.69955	1.46278e-12	\\
0.702505	1.43622e-12	\\
0.705434	1.41027e-12	\\
0.706966	1.39658e-12	\\
0.710002	1.36978e-12	\\
0.71249	1.34795e-12	\\
0.718108	1.29893e-12	\\
0.722177	1.26383e-12	\\
0.724212	1.24625e-12	\\
0.727772	1.2161e-12	\\
0.730907	1.18954e-12	\\
0.733312	1.16936e-12	\\
0.737613	1.13331e-12	\\
0.743549	1.08484e-12	\\
0.746625	1.05976e-12	\\
0.748683	1.04308e-12	\\
0.750398	1.02937e-12	\\
0.754213	9.99158e-13	\\
0.757265	9.751e-13	\\
0.760278	9.51637e-13	\\
0.762993	9.30719e-13	\\
0.768141	8.91663e-13	\\
0.771412	8.67217e-13	\\
0.774243	8.46008e-13	\\
0.778214	8.16841e-13	\\
0.781509	7.93198e-13	\\
0.785039	7.6803e-13	\\
0.789806	7.34672e-13	\\
0.793167	7.11541e-13	\\
0.796088	6.91717e-13	\\
0.80221	6.51096e-13	\\
0.805098	6.32133e-13	\\
0.80767	6.15483e-13	\\
0.809874	6.01528e-13	\\
0.812891	5.82625e-13	\\
0.816306	5.61449e-13	\\
0.818988	5.45107e-13	\\
0.821809	5.28293e-13	\\
0.824666	5.11386e-13	\\
0.828553	4.88821e-13	\\
0.832199	4.68169e-13	\\
0.834358	4.56115e-13	\\
0.837803	4.37273e-13	\\
0.844224	4.02939e-13	\\
0.848839	3.79209e-13	\\
0.854473	3.51129e-13	\\
0.858483	3.31998e-13	\\
0.865074	3.01441e-13	\\
0.869705	2.8096e-13	\\
0.873873	2.62965e-13	\\
0.879156	2.41143e-13	\\
0.882406	2.28085e-13	\\
0.885564	2.16002e-13	\\
0.889228	2.02198e-13	\\
0.891719	1.93092e-13	\\
0.893937	1.85193e-13	\\
0.898794	1.68357e-13	\\
0.905655	1.46133e-13	\\
0.909293	1.34924e-13	\\
0.914879	1.18639e-13	\\
0.91945	1.0611e-13	\\
0.922132	9.9058e-14	\\
0.924317	9.35784e-14	\\
0.930693	7.82232e-14	\\
0.935614	6.74373e-14	\\
0.940388	5.78163e-14	\\
0.943663	5.15408e-14	\\
0.947772	4.41649e-14	\\
0.955624	3.18973e-14	\\
0.958188	2.82787e-14	\\
0.965368	1.94459e-14	\\
0.969607	1.49664e-14	\\
0.974504	1.05597e-14	\\
0.980515	6.19301e-15	\\
0.98462	3.9891e-15	\\
0.98935	1.96879e-15	\\
0.99532	4.86183e-16	\\
0.998514	2.29417e-16	\\
1.00389	4.345e-16	\\
1.00875	1.27108e-15	\\
1.01152	2.10825e-15	\\
1.01834	5.30977e-15	\\
1.02222	8.10529e-15	\\
1.02639	1.12856e-14	\\
1.02968	1.4033e-14	\\
1.03472	1.9303e-14	\\
1.03708	2.21281e-14	\\
1.0432	3.02813e-14	\\
1.04803	3.73641e-14	\\
1.05248	4.47556e-14	\\
1.05556	5.01661e-14	\\
1.06137	6.12217e-14	\\
1.06618	7.13467e-14	\\
1.07213	8.49888e-14	\\
1.07635	9.51989e-14	\\
1.08382	1.15172e-13	\\
1.08761	1.25639e-13	\\
1.09427	1.45968e-13	\\
1.10094	1.67675e-13	\\
1.10428	1.79195e-13	\\
1.11107	2.03574e-13	\\
1.11445	2.16305e-13	\\
1.1179	2.2986e-13	\\
1.12264	2.49007e-13	\\
1.12783	2.70824e-13	\\
1.13184	2.8834e-13	\\
1.13883	3.20115e-13	\\
1.14441	3.46802e-13	\\
1.14871	3.68034e-13	\\
1.15328	3.91431e-13	\\
1.15726	4.12372e-13	\\
1.1602	4.28119e-13	\\
1.16443	4.51407e-13	\\
1.16996	4.82644e-13	\\
1.17577	5.16809e-13	\\
1.17954	5.39292e-13	\\
1.1832	5.61685e-13	\\
1.18791	5.91658e-13	\\
1.1929	6.2363e-13	\\
1.19916	6.65229e-13	\\
1.20767	7.2401e-13	\\
1.21307	7.62612e-13	\\
1.21808	7.99506e-13	\\
1.22241	8.31906e-13	\\
1.2294	8.85663e-13	\\
1.23769	9.51402e-13	\\
1.24156	9.8248e-13	\\
1.24652	1.02364e-12	\\
1.25395	1.08695e-12	\\
1.26076	1.14602e-12	\\
1.26748	1.2062e-12	\\
1.27252	1.25217e-12	\\
1.27741	1.2978e-12	\\
1.28349	1.3551e-12	\\
1.2914	1.432e-12	\\
1.29571	1.47464e-12	\\
1.30192	1.5371e-12	\\
1.30717	1.59078e-12	\\
1.31215	1.64277e-12	\\
1.31742	1.69844e-12	\\
1.32457	1.7754e-12	\\
1.32922	1.82608e-12	\\
1.33684	1.91119e-12	\\
1.34381	1.99011e-12	\\
1.34972	2.05838e-12	\\
1.35563	2.12778e-12	\\
1.36125	2.1944e-12	\\
1.36887	2.28679e-12	\\
1.37549	2.36824e-12	\\
1.38533	2.49136e-12	\\
1.39279	2.58709e-12	\\
1.39763	2.64978e-12	\\
1.4055	2.75303e-12	\\
1.41028	2.81679e-12	\\
1.41439	2.87218e-12	\\
1.41835	2.92594e-12	\\
1.42524	3.02023e-12	\\
1.42924	3.07556e-12	\\
1.43495	3.15531e-12	\\
1.44057	3.23476e-12	\\
1.44784	3.33865e-12	\\
1.46077	3.52712e-12	\\
1.46666	3.61418e-12	\\
1.48032	3.81983e-12	\\
1.4846	3.8853e-12	\\
1.49294	4.01386e-12	\\
1.49869	4.10367e-12	\\
1.50401	4.18698e-12	\\
1.51435	4.3517e-12	\\
1.52143	4.46622e-12	\\
1.53151	4.6304e-12	\\
1.53696	4.72019e-12	\\
1.54633	4.87617e-12	\\
1.54953	4.92994e-12	\\
1.55879	5.0866e-12	\\
1.56552	5.20154e-12	\\
1.57553	5.3745e-12	\\
1.58196	5.48658e-12	\\
1.5874	5.58208e-12	\\
1.59513	5.71886e-12	\\
1.59983	5.80243e-12	\\
1.60538	5.90204e-12	\\
1.61302	6.03959e-12	\\
1.61949	6.15697e-12	\\
1.62707	6.29582e-12	\\
1.63344	6.41309e-12	\\
1.64378	6.60541e-12	\\
1.6555	6.82514e-12	\\
1.66625	7.02892e-12	\\
1.67451	7.1867e-12	\\
1.68825	7.45167e-12	\\
1.69695	7.62081e-12	\\
1.7046	7.77041e-12	\\
1.7117	7.91057e-12	\\
1.72369	8.14798e-12	\\
1.72862	8.24636e-12	\\
1.73709	8.41606e-12	\\
1.7441	8.55727e-12	\\
1.75046	8.68592e-12	\\
1.75613	8.80134e-12	\\
1.76015	8.88324e-12	\\
1.76765	9.0363e-12	\\
1.77208	9.12723e-12	\\
1.78519	9.39724e-12	\\
1.79739	9.6503e-12	\\
1.80889	9.89076e-12	\\
1.81709	1.00627e-11	\\
1.82862	1.03058e-11	\\
1.83824	1.05096e-11	\\
1.84642	1.06837e-11	\\
1.85506	1.0868e-11	\\
1.86419	1.10635e-11	\\
1.87	1.11885e-11	\\
1.87543	1.13056e-11	\\
1.88149	1.14363e-11	\\
1.88978	1.16156e-11	\\
1.89788	1.17911e-11	\\
1.90767	1.20041e-11	\\
1.91344	1.213e-11	\\
1.92585	1.24011e-11	\\
1.93516	1.26053e-11	\\
1.94096	1.27327e-11	\\
1.95279	1.29932e-11	\\
1.96239	1.32052e-11	\\
1.97105	1.33966e-11	\\
1.98137	1.3625e-11	\\
1.9862	1.37323e-11	\\
1.99367	1.38984e-11	\\
2.00146	1.40714e-11	\\
2.00946	1.42498e-11	\\
2.01415	1.43545e-11	\\
2.02919	1.469e-11	\\
2.03299	1.47748e-11	\\
2.04513	1.50461e-11	\\
2.06157	1.54145e-11	\\
2.07762	1.57742e-11	\\
2.08203	1.5873e-11	\\
2.09592	1.61847e-11	\\
2.09902	1.62544e-11	\\
2.1106	1.65142e-11	\\
2.11849	1.66916e-11	\\
2.12568	1.68531e-11	\\
2.13566	1.70768e-11	\\
2.14361	1.7255e-11	\\
2.15612	1.75358e-11	\\
2.16166	1.76601e-11	\\
2.17471	1.79528e-11	\\
2.18536	1.81911e-11	\\
2.19598	1.84288e-11	\\
2.20653	1.86648e-11	\\
2.22139	1.89961e-11	\\
2.22653	1.91107e-11	\\
2.23919	1.93931e-11	\\
2.24707	1.95681e-11	\\
2.25385	1.97186e-11	\\
2.26292	1.99201e-11	\\
2.27041	2.00859e-11	\\
2.27992	2.02965e-11	\\
2.29442	2.06166e-11	\\
2.30608	2.08731e-11	\\
2.31854	2.1147e-11	\\
2.3339	2.14835e-11	\\
2.35015	2.18379e-11	\\
2.35749	2.19974e-11	\\
2.3632	2.21212e-11	\\
2.36934	2.22545e-11	\\
2.38516	2.25965e-11	\\
2.39573	2.28245e-11	\\
2.40215	2.29623e-11	\\
2.41704	2.32811e-11	\\
2.43092	2.35769e-11	\\
2.44218	2.3816e-11	\\
2.45123	2.40078e-11	\\
2.46866	2.4375e-11	\\
2.47758	2.45622e-11	\\
2.4849	2.4715e-11	\\
2.49979	2.50252e-11	\\
2.5111	2.52596e-11	\\
2.52495	2.55452e-11	\\
2.53541	2.57599e-11	\\
2.54724	2.60013e-11	\\
2.55612	2.61818e-11	\\
2.57322	2.65277e-11	\\
2.58161	2.66964e-11	\\
2.59033	2.68713e-11	\\
2.60481	2.71596e-11	\\
2.61999	2.74599e-11	\\
2.63512	2.77569e-11	\\
2.64529	2.79555e-11	\\
2.65716	2.81858e-11	\\
2.66849	2.84047e-11	\\
2.68218	2.86671e-11	\\
2.69601	2.89302e-11	\\
2.70992	2.9193e-11	\\
2.71965	2.93758e-11	\\
2.73612	2.96826e-11	\\
2.7522	2.9979e-11	\\
2.77408	3.0379e-11	\\
2.79011	3.0668e-11	\\
2.80601	3.0952e-11	\\
2.82521	3.12912e-11	\\
2.83653	3.14895e-11	\\
2.85437	3.17987e-11	\\
2.86762	3.20259e-11	\\
2.88104	3.22543e-11	\\
2.88872	3.23837e-11	\\
2.90365	3.26339e-11	\\
2.91603	3.28392e-11	\\
2.93079	3.30818e-11	\\
2.94329	3.32849e-11	\\
2.96475	3.36297e-11	\\
2.97502	3.37926e-11	\\
2.98864	3.40068e-11	\\
3.00315	3.42325e-11	\\
3.01437	3.44054e-11	\\
3.02547	3.45744e-11	\\
3.03958	3.47878e-11	\\
3.05821	3.50652e-11	\\
3.07135	3.52585e-11	\\
3.07922	3.53733e-11	\\
3.09105	3.55442e-11	\\
3.09501	3.5601e-11	\\
3.11529	3.58889e-11	\\
3.12956	3.60887e-11	\\
3.1364	3.61833e-11	\\
3.14312	3.62761e-11	\\
3.15484	3.6436e-11	\\
3.16956	3.66344e-11	\\
3.18858	3.68867e-11	\\
3.206	3.71139e-11	\\
3.21873	3.72775e-11	\\
3.23985	3.75441e-11	\\
3.2588	3.77788e-11	\\
3.27349	3.79576e-11	\\
3.29053	3.81612e-11	\\
3.30883	3.83763e-11	\\
3.32326	3.85425e-11	\\
3.34255	3.87609e-11	\\
3.35805	3.89325e-11	\\
3.37076	3.90714e-11	\\
3.38273	3.92001e-11	\\
3.39877	3.93695e-11	\\
3.40947	3.9481e-11	\\
3.42309	3.96207e-11	\\
3.44559	3.9846e-11	\\
3.4579	3.99665e-11	\\
3.47785	4.01584e-11	\\
3.49421	4.03114e-11	\\
3.50401	4.04016e-11	\\
3.51176	4.04723e-11	\\
3.52382	4.05803e-11	\\
3.53656	4.06925e-11	\\
3.55629	4.08625e-11	\\
3.5692	4.09708e-11	\\
3.58819	4.11268e-11	\\
3.6019	4.12366e-11	\\
3.62605	4.14241e-11	\\
3.64093	4.15359e-11	\\
3.66119	4.16838e-11	\\
3.67574	4.17875e-11	\\
3.69356	4.19103e-11	\\
3.70987	4.20195e-11	\\
3.7214	4.20946e-11	\\
3.73114	4.21569e-11	\\
3.75029	4.2276e-11	\\
3.76804	4.23823e-11	\\
3.78452	4.24778e-11	\\
3.81006	4.26191e-11	\\
3.82486	4.26975e-11	\\
3.84733	4.28116e-11	\\
3.86475	4.28962e-11	\\
3.89033	4.30136e-11	\\
3.9009	4.30601e-11	\\
3.92121	4.31453e-11	\\
3.93057	4.31831e-11	\\
3.94864	4.32533e-11	\\
3.96642	4.33184e-11	\\
3.98334	4.33776e-11	\\
3.9953	4.34169e-11	\\
4.0092	4.34611e-11	\\
4.01971	4.34929e-11	\\
4.04036	4.35521e-11	\\
4.06313	4.36116e-11	\\
4.08001	4.36524e-11	\\
4.10293	4.37022e-11	\\
4.12102	4.37379e-11	\\
4.13415	4.37617e-11	\\
4.15229	4.37912e-11	\\
4.17638	4.38252e-11	\\
4.19415	4.38461e-11	\\
4.20661	4.38588e-11	\\
4.21883	4.38699e-11	\\
4.24168	4.38863e-11	\\
4.27087	4.38996e-11	\\
4.2816	4.39024e-11	\\
4.30361	4.3904e-11	\\
4.33567	4.38985e-11	\\
4.35606	4.38893e-11	\\
4.36785	4.38824e-11	\\
4.39144	4.38643e-11	\\
4.41208	4.38438e-11	\\
4.42695	4.38268e-11	\\
4.43477	4.38172e-11	\\
4.44879	4.37979e-11	\\
4.472	4.37624e-11	\\
4.51114	4.36912e-11	\\
4.52652	4.36594e-11	\\
4.54622	4.36158e-11	\\
4.56374	4.3574e-11	\\
4.58013	4.3532e-11	\\
4.60286	4.34704e-11	\\
4.62445	4.34076e-11	\\
4.64708	4.33377e-11	\\
4.67292	4.32525e-11	\\
4.68998	4.31932e-11	\\
4.71447	4.31038e-11	\\
4.73037	4.30433e-11	\\
4.74845	4.29719e-11	\\
4.7763	4.28569e-11	\\
4.81018	4.27087e-11	\\
4.84206	4.25611e-11	\\
4.8553	4.24975e-11	\\
4.87923	4.23793e-11	\\
4.8989	4.2279e-11	\\
4.92062	4.2165e-11	\\
4.94125	4.20535e-11	\\
4.96225	4.19369e-11	\\
4.97405	4.18699e-11	\\
4.99798	4.17314e-11	\\
5.01152	4.16512e-11	\\
5.02698	4.15582e-11	\\
5.04296	4.14602e-11	\\
5.07515	4.12579e-11	\\
5.10643	4.10549e-11	\\
5.12456	4.09347e-11	\\
5.13305	4.08771e-11	\\
5.16698	4.06441e-11	\\
5.1908	4.04764e-11	\\
5.20705	4.03598e-11	\\
5.24076	4.01135e-11	\\
5.24932	4.00496e-11	\\
5.27866	3.98282e-11	\\
5.30212	3.9648e-11	\\
5.32995	3.94301e-11	\\
5.35464	3.92329e-11	\\
5.38854	3.89572e-11	\\
5.41724	3.87191e-11	\\
5.436	3.85614e-11	\\
5.48011	3.81832e-11	\\
5.51535	3.78743e-11	\\
5.54338	3.7625e-11	\\
5.55911	3.74834e-11	\\
5.58382	3.72589e-11	\\
5.62636	3.68655e-11	\\
5.64758	3.66666e-11	\\
5.67398	3.64166e-11	\\
5.70173	3.61507e-11	\\
5.74764	3.57044e-11	\\
5.77259	3.54588e-11	\\
5.7944	3.52419e-11	\\
5.8163	3.50225e-11	\\
5.84153	3.4768e-11	\\
5.87032	3.44748e-11	\\
5.89715	3.4199e-11	\\
5.92661	3.38937e-11	\\
5.96002	3.35446e-11	\\
5.988	3.32495e-11	\\
6.0035	3.30852e-11	\\
6.02802	3.28236e-11	\\
6.07394	3.23305e-11	\\
6.09769	3.20729e-11	\\
6.12448	3.1781e-11	\\
6.15224	3.14768e-11	\\
6.19105	3.1049e-11	\\
6.22736	3.06455e-11	\\
6.25917	3.02903e-11	\\
6.31269	2.96885e-11	\\
6.35032	2.92627e-11	\\
6.39267	2.87809e-11	\\
6.40727	2.86144e-11	\\
6.46447	2.79588e-11	\\
6.48122	2.7766e-11	\\
6.519	2.73303e-11	\\
6.53461	2.715e-11	\\
6.5647	2.68021e-11	\\
6.5933	2.64704e-11	\\
6.64122	2.59136e-11	\\
6.68137	2.54459e-11	\\
6.70655	2.51524e-11	\\
6.73582	2.48108e-11	\\
6.76645	2.44535e-11	\\
6.78333	2.42566e-11	\\
6.80961	2.39496e-11	\\
6.85015	2.34761e-11	\\
6.87144	2.32277e-11	\\
6.89538	2.29483e-11	\\
6.9168	2.26988e-11	\\
6.95005	2.23111e-11	\\
6.99506	2.17875e-11	\\
7.02438	2.14471e-11	\\
7.04214	2.1241e-11	\\
7.07241	2.08905e-11	\\
7.10173	2.05513e-11	\\
7.1255	2.02772e-11	\\
7.17333	1.9727e-11	\\
7.20271	1.939e-11	\\
7.21908	1.92028e-11	\\
7.26382	1.86926e-11	\\
7.29549	1.8333e-11	\\
7.32505	1.79985e-11	\\
7.3725	1.74646e-11	\\
7.40642	1.70851e-11	\\
7.45068	1.65928e-11	\\
7.49681	1.6083e-11	\\
7.54219	1.5586e-11	\\
7.58626	1.51072e-11	\\
7.60773	1.48753e-11	\\
7.63563	1.45757e-11	\\
7.67934	1.41092e-11	\\
7.7144	1.37387e-11	\\
7.74902	1.33756e-11	\\
7.81312	1.27115e-11	\\
7.84339	1.24014e-11	\\
7.87181	1.21124e-11	\\
7.89968	1.18317e-11	\\
7.94355	1.13938e-11	\\
7.99478	1.08898e-11	\\
8.0338	1.05113e-11	\\
8.05909	1.02685e-11	\\
8.08868	9.98705e-12	\\
8.14235	9.48372e-12	\\
8.18281	9.11088e-12	\\
8.22709	8.70945e-12	\\
8.25756	8.4367e-12	\\
8.29441	8.11184e-12	\\
8.33781	7.73552e-12	\\
8.3573	7.56893e-12	\\
8.38705	7.3172e-12	\\
8.43292	6.93596e-12	\\
8.45956	6.71808e-12	\\
8.47307	6.60871e-12	\\
8.5	6.39316e-12	\\
8.52038	6.23218e-12	\\
8.56203	5.90769e-12	\\
8.59445	5.66026e-12	\\
8.62651	5.42025e-12	\\
8.67078	5.09606e-12	\\
8.71021	4.81391e-12	\\
8.74704	4.55756e-12	\\
8.78105	4.32546e-12	\\
8.81942	4.07059e-12	\\
8.84083	3.93126e-12	\\
8.88987	3.62006e-12	\\
8.91925	3.43921e-12	\\
8.98304	3.06089e-12	\\
9.02012	2.84996e-12	\\
9.06756	2.58973e-12	\\
9.10711	2.38241e-12	\\
9.13123	2.25948e-12	\\
9.16562	2.08958e-12	\\
9.18945	1.97551e-12	\\
9.24534	1.7198e-12	\\
9.30029	1.48391e-12	\\
9.33588	1.34059e-12	\\
9.37337	1.1967e-12	\\
9.43709	9.69937e-13	\\
9.47973	8.30739e-13	\\
9.54244	6.45374e-13	\\
9.57435	5.59254e-13	\\
9.61281	4.63896e-13	\\
9.64683	3.8688e-13	\\
9.6953	2.88424e-13	\\
9.77374	1.59516e-13	\\
9.81293	1.08784e-13	\\
9.8489	7.08517e-14	\\
9.90572	2.74351e-14	\\
9.932	1.3639e-14	\\
};
\addlegendentry{$\norm{u_{N}(\sigma) - u_{\mathcal N}(\sigma)}_{\mathcal X}$};

\addplot [color=matlab2, densely dotted]
  table[row sep=crcr]{
0.100015	1.82077e-14	\\
0.100336	5.32907e-15	\\
0.100561	4.44089e-16	\\
0.101092	7.99361e-15	\\
0.101691	7.54952e-15	\\
0.102174	5.32907e-15	\\
0.10289	2.4869e-14	\\
0.103285	4.44089e-15	\\
0.103801	1.55431e-14	\\
0.104376	2.66454e-14	\\
0.104902	1.59872e-14	\\
0.10552	2.13163e-14	\\
0.106405	1.42109e-14	\\
0.106778	1.15463e-14	\\
0.107259	2.53131e-14	\\
0.107686	1.9984e-14	\\
0.107963	3.68594e-14	\\
0.108685	3.4639e-14	\\
0.109214	4.04121e-14	\\
0.109707	3.68594e-14	\\
0.110134	4.75175e-14	\\
0.111103	5.4623e-14	\\
0.111734	4.39648e-14	\\
0.112186	6.70575e-14	\\
0.112688	5.90639e-14	\\
0.113033	6.66134e-14	\\
0.113665	6.08402e-14	\\
0.114368	6.88338e-14	\\
0.11488	7.23865e-14	\\
0.115329	8.21565e-14	\\
0.115513	7.81597e-14	\\
0.116102	7.01661e-14	\\
0.116665	8.92619e-14	\\
0.116962	8.61533e-14	\\
0.117579	9.54792e-14	\\
0.11802	9.59233e-14	\\
0.118636	9.23706e-14	\\
0.119443	9.32587e-14	\\
0.119963	1.07025e-13	\\
0.120924	1.10578e-13	\\
0.121181	1.14131e-13	\\
0.122195	1.2168e-13	\\
0.122727	1.21236e-13	\\
0.12329	1.18128e-13	\\
0.123732	1.3145e-13	\\
0.12409	1.31006e-13	\\
0.124442	1.2923e-13	\\
0.125119	1.3145e-13	\\
0.125727	1.35003e-13	\\
0.126295	1.37668e-13	\\
0.127179	1.41664e-13	\\
0.127657	1.36335e-13	\\
0.128238	1.50546e-13	\\
0.128554	1.38112e-13	\\
0.129049	1.55431e-13	\\
0.129915	1.52323e-13	\\
0.130325	1.44773e-13	\\
0.130921	1.53655e-13	\\
0.131481	1.61648e-13	\\
0.132223	1.57652e-13	\\
0.132941	1.51879e-13	\\
0.133901	1.55875e-13	\\
0.134572	1.54543e-13	\\
0.135143	1.5099e-13	\\
0.135771	1.52323e-13	\\
0.136557	1.53211e-13	\\
0.137199	1.59872e-13	\\
0.137877	1.52767e-13	\\
0.138381	1.62981e-13	\\
0.139077	1.66978e-13	\\
0.139834	1.51879e-13	\\
0.140459	1.57208e-13	\\
0.141017	1.54099e-13	\\
0.141424	1.57208e-13	\\
0.142416	1.58984e-13	\\
0.143505	1.43885e-13	\\
0.144129	1.4877e-13	\\
0.144945	1.58984e-13	\\
0.145646	1.43441e-13	\\
0.146226	1.57208e-13	\\
0.146843	1.54987e-13	\\
0.147435	1.47438e-13	\\
0.148121	1.49214e-13	\\
0.148733	1.49214e-13	\\
0.1495	1.40332e-13	\\
0.150498	1.52767e-13	\\
0.151159	1.42109e-13	\\
0.151942	1.35891e-13	\\
0.152293	1.26121e-13	\\
0.153102	1.42997e-13	\\
0.154286	1.32339e-13	\\
0.154867	1.22125e-13	\\
0.155517	1.43441e-13	\\
0.156005	1.29674e-13	\\
0.156597	1.27898e-13	\\
0.157325	1.24789e-13	\\
0.158119	1.30118e-13	\\
0.159341	1.12355e-13	\\
0.159964	1.29674e-13	\\
0.160553	1.14575e-13	\\
0.161612	1.05693e-13	\\
0.162586	9.4591e-14	\\
0.163148	1.06137e-13	\\
0.163583	1.00364e-13	\\
0.164439	9.81437e-14	\\
0.165021	1.02141e-13	\\
0.165753	9.4591e-14	\\
0.166827	8.43769e-14	\\
0.167842	8.65974e-14	\\
0.168846	8.34888e-14	\\
0.170115	7.32747e-14	\\
0.170973	7.41629e-14	\\
0.171892	6.70575e-14	\\
0.172206	5.90639e-14	\\
0.173045	6.21725e-14	\\
0.173717	6.4837e-14	\\
0.174618	5.50671e-14	\\
0.175431	6.03961e-14	\\
0.175763	5.81757e-14	\\
0.17639	5.59552e-14	\\
0.176848	6.17284e-14	\\
0.177562	5.50671e-14	\\
0.178582	4.66294e-14	\\
0.179392	4.30767e-14	\\
0.180059	4.35207e-14	\\
0.180937	4.21885e-14	\\
0.181749	3.59712e-14	\\
0.182615	3.68594e-14	\\
0.183481	3.10862e-14	\\
0.184193	3.28626e-14	\\
0.184908	3.10862e-14	\\
0.185697	3.24185e-14	\\
0.186241	2.79776e-14	\\
0.187253	1.9984e-14	\\
0.188144	2.4869e-14	\\
0.189283	2.17604e-14	\\
0.189833	9.32587e-15	\\
0.190917	1.06581e-14	\\
0.191531	1.06581e-14	\\
0.192787	5.77316e-15	\\
0.193147	6.21725e-15	\\
0.194284	3.55271e-15	\\
0.195764	2.22045e-15	\\
0.196783	1.33227e-15	\\
0.197564	5.77316e-15	\\
0.198592	3.10862e-15	\\
0.199754	5.32907e-15	\\
0.200825	4.44089e-15	\\
0.201526	2.22045e-15	\\
0.202452	7.99361e-15	\\
0.203152	1.02141e-14	\\
0.203499	2.22045e-15	\\
0.204247	8.88178e-15	\\
0.204955	6.21725e-15	\\
0.205628	4.44089e-16	\\
0.206233	8.88178e-16	\\
0.206993	3.77476e-15	\\
0.207526	6.66134e-16	\\
0.208113	7.32747e-15	\\
0.209284	5.77316e-15	\\
0.211021	2.44249e-15	\\
0.212628	2.22045e-15	\\
0.213591	1.28786e-14	\\
0.214658	1.22125e-14	\\
0.215148	1.04361e-14	\\
0.216583	4.44089e-15	\\
0.217329	7.77156e-15	\\
0.218106	9.99201e-15	\\
0.218764	2.22045e-15	\\
0.219373	2.68674e-14	\\
0.220098	1.55431e-14	\\
0.221431	1.37668e-14	\\
0.221823	1.46549e-14	\\
0.22244	1.53211e-14	\\
0.223583	2.4647e-14	\\
0.224506	1.13243e-14	\\
0.225701	2.77556e-14	\\
0.226956	3.5083e-14	\\
0.227942	3.88578e-14	\\
0.229057	3.57492e-14	\\
0.230206	3.5083e-14	\\
0.231229	3.9968e-14	\\
0.232482	5.17364e-14	\\
0.234038	6.03961e-14	\\
0.234983	6.08402e-14	\\
0.236193	7.01661e-14	\\
0.237082	7.43849e-14	\\
0.238066	7.4607e-14	\\
0.238912	9.03722e-14	\\
0.239986	8.52651e-14	\\
0.241948	8.9706e-14	\\
0.242831	1.05915e-13	\\
0.243463	1.06359e-13	\\
0.244878	1.27232e-13	\\
0.245524	1.25455e-13	\\
0.246599	1.24567e-13	\\
0.24792	1.36557e-13	\\
0.24935	1.37446e-13	\\
0.250667	1.50102e-13	\\
0.251826	1.53655e-13	\\
0.252669	1.672e-13	\\
0.253519	1.78302e-13	\\
0.254469	1.85851e-13	\\
0.25498	1.8141e-13	\\
0.25552	1.74749e-13	\\
0.256499	1.96954e-13	\\
0.257579	1.99618e-13	\\
0.258834	2.17826e-13	\\
0.260244	2.21823e-13	\\
0.261773	2.31815e-13	\\
0.262476	2.42251e-13	\\
0.263972	2.59348e-13	\\
0.265131	2.61791e-13	\\
0.266681	2.66454e-13	\\
0.267848	2.83107e-13	\\
0.269029	2.89546e-13	\\
0.270687	3.13527e-13	\\
0.272064	3.22853e-13	\\
0.273021	3.2685e-13	\\
0.274058	3.35065e-13	\\
0.274715	3.41283e-13	\\
0.275995	3.5616e-13	\\
0.277002	3.71925e-13	\\
0.279084	3.80584e-13	\\
0.280201	3.99236e-13	\\
0.280852	4.05453e-13	\\
0.281772	4.1056e-13	\\
0.282503	4.22773e-13	\\
0.283784	4.2566e-13	\\
0.284679	4.34319e-13	\\
0.285512	4.34541e-13	\\
0.286971	4.54303e-13	\\
0.287939	4.82725e-13	\\
0.290064	4.92051e-13	\\
0.291414	5.07594e-13	\\
0.293254	5.24469e-13	\\
0.294542	5.43565e-13	\\
0.296175	5.52447e-13	\\
0.297304	5.69544e-13	\\
0.298542	5.81313e-13	\\
0.299978	5.94635e-13	\\
0.301643	6.01963e-13	\\
0.303399	6.33493e-13	\\
0.304693	6.5592e-13	\\
0.306072	6.59028e-13	\\
0.307168	6.73461e-13	\\
0.308907	6.93667e-13	\\
0.310409	7.02771e-13	\\
0.311453	7.1676e-13	\\
0.312886	7.363e-13	\\
0.314384	7.52065e-13	\\
0.315802	7.65832e-13	\\
0.316826	7.73825e-13	\\
0.318472	7.86038e-13	\\
0.319978	8.08908e-13	\\
0.320952	8.14682e-13	\\
0.322137	8.35332e-13	\\
0.323487	8.4599e-13	\\
0.32546	8.71747e-13	\\
0.326333	8.7752e-13	\\
0.327803	8.89955e-13	\\
0.328939	9.10383e-13	\\
0.330106	9.1549e-13	\\
0.331217	9.38583e-13	\\
0.332388	9.47242e-13	\\
0.334033	9.59677e-13	\\
0.336997	9.93872e-13	\\
0.337996	1.00675e-12	\\
0.339107	1.02585e-12	\\
0.34047	1.03495e-12	\\
0.341418	1.04428e-12	\\
0.343424	1.05782e-12	\\
0.345429	1.08646e-12	\\
0.346538	1.08336e-12	\\
0.347824	1.10423e-12	\\
0.349127	1.12954e-12	\\
0.351381	1.14553e-12	\\
0.352489	1.15508e-12	\\
0.353752	1.16152e-12	\\
0.355398	1.18949e-12	\\
0.357443	1.21148e-12	\\
0.358614	1.21569e-12	\\
0.360395	1.23834e-12	\\
0.362943	1.255e-12	\\
0.364721	1.2903e-12	\\
0.367066	1.30673e-12	\\
0.368198	1.30917e-12	\\
0.370096	1.32694e-12	\\
0.371913	1.35381e-12	\\
0.374146	1.37179e-12	\\
0.375713	1.38489e-12	\\
0.378326	1.40199e-12	\\
0.379935	1.43086e-12	\\
0.381667	1.44174e-12	\\
0.383385	1.44373e-12	\\
0.385675	1.47149e-12	\\
0.387083	1.48659e-12	\\
0.38852	1.49303e-12	\\
0.391974	1.53189e-12	\\
0.39389	1.53477e-12	\\
0.395107	1.55564e-12	\\
0.396493	1.57119e-12	\\
0.397647	1.56586e-12	\\
0.399095	1.58096e-12	\\
0.400141	1.59073e-12	\\
0.401839	1.60161e-12	\\
0.403771	1.6267e-12	\\
0.405512	1.63847e-12	\\
0.407462	1.64913e-12	\\
0.409876	1.66223e-12	\\
0.411737	1.68199e-12	\\
0.413276	1.68798e-12	\\
0.414509	1.69753e-12	\\
0.416468	1.71507e-12	\\
0.417323	1.72484e-12	\\
0.418981	1.72773e-12	\\
0.420591	1.73883e-12	\\
0.421779	1.75326e-12	\\
0.423482	1.75726e-12	\\
0.425064	1.76881e-12	\\
0.426785	1.78368e-12	\\
0.428318	1.77902e-12	\\
0.431548	1.79945e-12	\\
0.434767	1.81766e-12	\\
0.437359	1.83298e-12	\\
0.439772	1.84386e-12	\\
0.441974	1.85896e-12	\\
0.44365	1.86362e-12	\\
0.445904	1.87539e-12	\\
0.447634	1.88249e-12	\\
0.450898	1.88716e-12	\\
0.452192	1.89537e-12	\\
0.454729	1.90759e-12	\\
0.456512	1.92513e-12	\\
0.459154	1.92446e-12	\\
0.461828	1.93112e-12	\\
0.463977	1.95111e-12	\\
0.466542	1.94311e-12	\\
0.469142	1.96265e-12	\\
0.47098	1.9591e-12	\\
0.472419	1.95932e-12	\\
0.473777	1.97242e-12	\\
0.475743	1.97509e-12	\\
0.477949	1.98153e-12	\\
0.480112	1.98086e-12	\\
0.4835	1.99707e-12	\\
0.485685	1.99218e-12	\\
0.488222	1.99663e-12	\\
0.490474	1.99307e-12	\\
0.491845	1.99862e-12	\\
0.494792	1.99485e-12	\\
0.497299	1.99663e-12	\\
0.49979	2.00795e-12	\\
0.501809	2.0095e-12	\\
0.503775	2.0095e-12	\\
0.50673	2.0064e-12	\\
0.508105	1.99973e-12	\\
0.511606	2.00528e-12	\\
0.515601	2.00684e-12	\\
0.519691	2.00484e-12	\\
0.521657	2.00062e-12	\\
0.523934	1.99663e-12	\\
0.525638	1.99041e-12	\\
0.527003	1.99551e-12	\\
0.529262	1.9913e-12	\\
0.531221	1.99551e-12	\\
0.53368	1.98819e-12	\\
0.536445	1.97864e-12	\\
0.538675	1.97287e-12	\\
0.541318	1.9702e-12	\\
0.543193	1.96843e-12	\\
0.544877	1.95999e-12	\\
0.548326	1.96065e-12	\\
0.551087	1.94755e-12	\\
0.552196	1.948e-12	\\
0.554199	1.93778e-12	\\
0.5554	1.9349e-12	\\
0.558038	1.92779e-12	\\
0.559977	1.93601e-12	\\
0.563032	1.91647e-12	\\
0.566192	1.90226e-12	\\
0.568973	1.89937e-12	\\
0.57462	1.88383e-12	\\
0.577324	1.85807e-12	\\
0.581472	1.85207e-12	\\
0.583344	1.84541e-12	\\
0.586459	1.84275e-12	\\
0.588681	1.81566e-12	\\
0.592427	1.807e-12	\\
0.595686	1.79146e-12	\\
0.596761	1.78457e-12	\\
0.600326	1.78146e-12	\\
0.601606	1.76925e-12	\\
0.604555	1.75215e-12	\\
0.607517	1.73039e-12	\\
0.610228	1.72107e-12	\\
0.612805	1.71596e-12	\\
0.614009	1.7093e-12	\\
0.617021	1.68798e-12	\\
0.619689	1.67155e-12	\\
0.62227	1.66156e-12	\\
0.626615	1.64513e-12	\\
0.629554	1.62492e-12	\\
0.632247	1.61249e-12	\\
0.638017	1.57208e-12	\\
0.639473	1.56208e-12	\\
0.643873	1.54232e-12	\\
0.648235	1.5139e-12	\\
0.651095	1.5028e-12	\\
0.654668	1.47904e-12	\\
0.657656	1.4595e-12	\\
0.659734	1.45683e-12	\\
0.663918	1.42242e-12	\\
0.666508	1.40599e-12	\\
0.670846	1.37246e-12	\\
0.674249	1.36158e-12	\\
0.677272	1.34226e-12	\\
0.68107	1.31406e-12	\\
0.683865	1.30584e-12	\\
0.686305	1.27809e-12	\\
0.691086	1.2581e-12	\\
0.692885	1.24589e-12	\\
0.695898	1.22835e-12	\\
0.69955	1.20726e-12	\\
0.702505	1.18305e-12	\\
0.705434	1.15885e-12	\\
0.706966	1.14975e-12	\\
0.710002	1.13087e-12	\\
0.71249	1.11688e-12	\\
0.718108	1.08336e-12	\\
0.722177	1.05693e-12	\\
0.724212	1.04539e-12	\\
0.727772	1.01741e-12	\\
0.730907	1.00586e-12	\\
0.733312	9.83214e-13	\\
0.737613	9.59233e-13	\\
0.743549	9.23039e-13	\\
0.746625	8.99725e-13	\\
0.748683	8.94618e-13	\\
0.750398	8.76632e-13	\\
0.754213	8.59979e-13	\\
0.757265	8.41771e-13	\\
0.760278	8.1446e-13	\\
0.762993	7.99583e-13	\\
0.768141	7.74047e-13	\\
0.771412	7.49845e-13	\\
0.774243	7.36966e-13	\\
0.778214	7.09877e-13	\\
0.781509	6.94333e-13	\\
0.785039	6.76792e-13	\\
0.789806	6.41931e-13	\\
0.793167	6.28608e-13	\\
0.796088	6.06626e-13	\\
0.80221	5.74651e-13	\\
0.805098	5.53779e-13	\\
0.80767	5.48894e-13	\\
0.809874	5.278e-13	\\
0.812891	5.16476e-13	\\
0.816306	5.03375e-13	\\
0.818988	4.865e-13	\\
0.821809	4.73621e-13	\\
0.824666	4.58078e-13	\\
0.828553	4.41647e-13	\\
0.832199	4.21663e-13	\\
0.834358	4.13225e-13	\\
0.837803	3.85025e-13	\\
0.844224	3.65485e-13	\\
0.848839	3.51497e-13	\\
0.854473	3.24629e-13	\\
0.858483	3.05089e-13	\\
0.865074	2.81331e-13	\\
0.869705	2.52687e-13	\\
0.873873	2.4114e-13	\\
0.879156	2.22045e-13	\\
0.882406	2.06724e-13	\\
0.885564	2.01839e-13	\\
0.889228	1.87184e-13	\\
0.891719	1.82965e-13	\\
0.893937	1.75193e-13	\\
0.898794	1.61204e-13	\\
0.905655	1.36335e-13	\\
0.909293	1.27454e-13	\\
0.914879	1.15241e-13	\\
0.91945	9.99201e-14	\\
0.922132	8.88178e-14	\\
0.924317	9.05942e-14	\\
0.930693	7.19425e-14	\\
0.935614	6.52811e-14	\\
0.940388	5.12923e-14	\\
0.943663	4.95159e-14	\\
0.947772	3.59712e-14	\\
0.955624	2.95319e-14	\\
0.958188	2.37588e-14	\\
0.965368	1.42109e-14	\\
0.969607	1.15463e-14	\\
0.974504	9.54792e-15	\\
0.980515	2.66454e-15	\\
0.98462	5.55112e-15	\\
0.98935	1.11022e-15	\\
0.99532	4.21885e-15	\\
0.998514	2.44249e-15	\\
1.00389	4.66294e-15	\\
1.00875	9.99201e-16	\\
1.01152	1.11022e-15	\\
1.01834	4.10783e-15	\\
1.02222	3.9968e-15	\\
1.02639	5.55112e-15	\\
1.02968	9.65894e-15	\\
1.03472	1.5099e-14	\\
1.03708	2.30926e-14	\\
1.0432	3.25295e-14	\\
1.04803	3.58602e-14	\\
1.05248	4.21885e-14	\\
1.05556	4.4964e-14	\\
1.06137	6.23945e-14	\\
1.06618	6.09512e-14	\\
1.07213	8.19345e-14	\\
1.07635	9.21485e-14	\\
1.08382	1.06803e-13	\\
1.08761	1.16906e-13	\\
1.09427	1.35669e-13	\\
1.10094	1.57319e-13	\\
1.10428	1.73528e-13	\\
1.11107	1.91069e-13	\\
1.11445	2.02394e-13	\\
1.1179	2.1394e-13	\\
1.12264	2.30704e-13	\\
1.12783	2.55795e-13	\\
1.13184	2.65676e-13	\\
1.13883	2.9754e-13	\\
1.14441	3.17413e-13	\\
1.14871	3.4206e-13	\\
1.15328	3.62488e-13	\\
1.15726	3.76921e-13	\\
1.1602	3.92908e-13	\\
1.16443	4.14779e-13	\\
1.16996	4.48641e-13	\\
1.17577	4.71068e-13	\\
1.17954	4.9194e-13	\\
1.1832	5.10703e-13	\\
1.18791	5.35239e-13	\\
1.1929	5.73319e-13	\\
1.19916	6.03184e-13	\\
1.20767	6.58251e-13	\\
1.21307	6.88338e-13	\\
1.21808	7.22422e-13	\\
1.22241	7.45959e-13	\\
1.2294	7.88702e-13	\\
1.23769	8.48877e-13	\\
1.24156	8.77187e-13	\\
1.24652	9.05165e-13	\\
1.25395	9.63007e-13	\\
1.26076	1.00864e-12	\\
1.26748	1.06115e-12	\\
1.27252	1.0999e-12	\\
1.27741	1.13598e-12	\\
1.28349	1.18527e-12	\\
1.2914	1.24534e-12	\\
1.29571	1.28597e-12	\\
1.30192	1.33715e-12	\\
1.30717	1.38212e-12	\\
1.31215	1.41664e-12	\\
1.31742	1.47138e-12	\\
1.32457	1.52789e-12	\\
1.32922	1.56675e-12	\\
1.33684	1.64024e-12	\\
1.34381	1.69542e-12	\\
1.34972	1.75149e-12	\\
1.35563	1.80644e-12	\\
1.36125	1.85918e-12	\\
1.36887	1.92624e-12	\\
1.37549	1.98896e-12	\\
1.38533	2.08933e-12	\\
1.39279	2.16471e-12	\\
1.39763	2.20524e-12	\\
1.4055	2.28961e-12	\\
1.41028	2.33757e-12	\\
1.41439	2.37488e-12	\\
1.41835	2.41462e-12	\\
1.42524	2.49134e-12	\\
1.42924	2.53253e-12	\\
1.43495	2.58826e-12	\\
1.44057	2.64733e-12	\\
1.44784	2.73093e-12	\\
1.46077	2.86626e-12	\\
1.46666	2.9301e-12	\\
1.48032	3.08065e-12	\\
1.4846	3.12428e-12	\\
1.49294	3.22065e-12	\\
1.49869	3.28215e-12	\\
1.50401	3.337e-12	\\
1.51435	3.46545e-12	\\
1.52143	3.54272e-12	\\
1.53151	3.66274e-12	\\
1.53696	3.71947e-12	\\
1.54633	3.83071e-12	\\
1.54953	3.86879e-12	\\
1.55879	3.97582e-12	\\
1.56552	4.05342e-12	\\
1.57553	4.16822e-12	\\
1.58196	4.2526e-12	\\
1.5874	4.31566e-12	\\
1.59513	4.40903e-12	\\
1.59983	4.46587e-12	\\
1.60538	4.52949e-12	\\
1.61302	4.6233e-12	\\
1.61949	4.69791e-12	\\
1.62707	4.78939e-12	\\
1.63344	4.86977e-12	\\
1.64378	4.99945e-12	\\
1.6555	5.14166e-12	\\
1.66625	5.27234e-12	\\
1.67451	5.37648e-12	\\
1.68825	5.53979e-12	\\
1.69695	5.65303e-12	\\
1.7046	5.74962e-12	\\
1.7117	5.83411e-12	\\
1.72369	5.98388e-12	\\
1.72862	6.03817e-12	\\
1.73709	6.15119e-12	\\
1.7441	6.24034e-12	\\
1.75046	6.32039e-12	\\
1.75613	6.38467e-12	\\
1.76015	6.43408e-12	\\
1.76765	6.52789e-12	\\
1.77208	6.59084e-12	\\
1.78519	6.74649e-12	\\
1.79739	6.89959e-12	\\
1.80889	7.04692e-12	\\
1.81709	7.14762e-12	\\
1.82862	7.28828e-12	\\
1.83824	7.40674e-12	\\
1.84642	7.50877e-12	\\
1.85506	7.62146e-12	\\
1.86419	7.73237e-12	\\
1.87	7.80032e-12	\\
1.87543	7.8697e-12	\\
1.88149	7.9392e-12	\\
1.88978	8.04112e-12	\\
1.89788	8.1426e-12	\\
1.90767	8.26128e-12	\\
1.91344	8.32923e-12	\\
1.92585	8.47877e-12	\\
1.93516	8.58869e-12	\\
1.94096	8.6654e-12	\\
1.95279	8.80263e-12	\\
1.96239	8.91864e-12	\\
1.97105	9.02478e-12	\\
1.98137	9.14047e-12	\\
1.9862	9.19775e-12	\\
1.99367	9.28702e-12	\\
2.00146	9.37961e-12	\\
2.00946	9.47342e-12	\\
2.01415	9.52727e-12	\\
2.02919	9.7008e-12	\\
2.03299	9.75264e-12	\\
2.04513	9.88887e-12	\\
2.06157	1.00776e-11	\\
2.07762	1.0257e-11	\\
2.08203	1.03031e-11	\\
2.09592	1.04583e-11	\\
2.09902	1.04953e-11	\\
2.1106	1.06229e-11	\\
2.11849	1.07094e-11	\\
2.12568	1.07909e-11	\\
2.13566	1.08958e-11	\\
2.14361	1.09838e-11	\\
2.15612	1.11177e-11	\\
2.16166	1.11812e-11	\\
2.17471	1.13199e-11	\\
2.18536	1.14299e-11	\\
2.19598	1.15421e-11	\\
2.20653	1.16497e-11	\\
2.22139	1.17981e-11	\\
2.22653	1.1853e-11	\\
2.23919	1.198e-11	\\
2.24707	1.20605e-11	\\
2.25385	1.21249e-11	\\
2.26292	1.2216e-11	\\
2.27041	1.22886e-11	\\
2.27992	1.23856e-11	\\
2.29442	1.25236e-11	\\
2.30608	1.2636e-11	\\
2.31854	1.27551e-11	\\
2.3339	1.28932e-11	\\
2.35015	1.30456e-11	\\
2.35749	1.31148e-11	\\
2.3632	1.31657e-11	\\
2.36934	1.32179e-11	\\
2.38516	1.33591e-11	\\
2.39573	1.34542e-11	\\
2.40215	1.35071e-11	\\
2.41704	1.36356e-11	\\
2.43092	1.37519e-11	\\
2.44218	1.3848e-11	\\
2.45123	1.39235e-11	\\
2.46866	1.40626e-11	\\
2.47758	1.41392e-11	\\
2.4849	1.41948e-11	\\
2.49979	1.43119e-11	\\
2.5111	1.43959e-11	\\
2.52495	1.45036e-11	\\
2.53541	1.4587e-11	\\
2.54724	1.46699e-11	\\
2.55612	1.47364e-11	\\
2.57322	1.48601e-11	\\
2.58161	1.49224e-11	\\
2.59033	1.49821e-11	\\
2.60481	1.50808e-11	\\
2.61999	1.51836e-11	\\
2.63512	1.52827e-11	\\
2.64529	1.53504e-11	\\
2.65716	1.54274e-11	\\
2.66849	1.54962e-11	\\
2.68218	1.55823e-11	\\
2.69601	1.56695e-11	\\
2.70992	1.57512e-11	\\
2.71965	1.58058e-11	\\
2.73612	1.59048e-11	\\
2.7522	1.59925e-11	\\
2.77408	1.61099e-11	\\
2.79011	1.61938e-11	\\
2.80601	1.628e-11	\\
2.82521	1.63773e-11	\\
2.83653	1.64336e-11	\\
2.85437	1.65208e-11	\\
2.86762	1.65715e-11	\\
2.88104	1.66377e-11	\\
2.88872	1.66687e-11	\\
2.90365	1.67345e-11	\\
2.91603	1.67888e-11	\\
2.93079	1.68469e-11	\\
2.94329	1.68972e-11	\\
2.96475	1.69831e-11	\\
2.97502	1.70201e-11	\\
2.98864	1.7069e-11	\\
3.00315	1.71229e-11	\\
3.01437	1.71561e-11	\\
3.02547	1.7194e-11	\\
3.03958	1.7239e-11	\\
3.05821	1.72987e-11	\\
3.07135	1.73376e-11	\\
3.07922	1.7361e-11	\\
3.09105	1.7396e-11	\\
3.09501	1.74056e-11	\\
3.11529	1.7461e-11	\\
3.12956	1.7497e-11	\\
3.1364	1.751e-11	\\
3.14312	1.75285e-11	\\
3.15484	1.75566e-11	\\
3.16956	1.75937e-11	\\
3.18858	1.76281e-11	\\
3.206	1.7669e-11	\\
3.21873	1.76911e-11	\\
3.23985	1.77311e-11	\\
3.2588	1.77602e-11	\\
3.27349	1.778e-11	\\
3.29053	1.78098e-11	\\
3.30883	1.78311e-11	\\
3.32326	1.78516e-11	\\
3.34255	1.78724e-11	\\
3.35805	1.78882e-11	\\
3.37076	1.78991e-11	\\
3.38273	1.79103e-11	\\
3.39877	1.79199e-11	\\
3.40947	1.79295e-11	\\
3.42309	1.79334e-11	\\
3.44559	1.79489e-11	\\
3.4579	1.79499e-11	\\
3.47785	1.79599e-11	\\
3.49421	1.79616e-11	\\
3.50401	1.79623e-11	\\
3.51176	1.79599e-11	\\
3.52382	1.79624e-11	\\
3.53656	1.79626e-11	\\
3.55629	1.79583e-11	\\
3.5692	1.79544e-11	\\
3.58819	1.79507e-11	\\
3.6019	1.79451e-11	\\
3.62605	1.79314e-11	\\
3.64093	1.79248e-11	\\
3.66119	1.79095e-11	\\
3.67574	1.7895e-11	\\
3.69356	1.78801e-11	\\
3.70987	1.78675e-11	\\
3.7214	1.78541e-11	\\
3.73114	1.78442e-11	\\
3.75029	1.78204e-11	\\
3.76804	1.7801e-11	\\
3.78452	1.77779e-11	\\
3.81006	1.77426e-11	\\
3.82486	1.7722e-11	\\
3.84733	1.769e-11	\\
3.86475	1.76598e-11	\\
3.89033	1.76134e-11	\\
3.9009	1.75946e-11	\\
3.92121	1.75542e-11	\\
3.93057	1.75393e-11	\\
3.94864	1.75027e-11	\\
3.96642	1.74669e-11	\\
3.98334	1.74304e-11	\\
3.9953	1.74092e-11	\\
4.0092	1.73759e-11	\\
4.01971	1.73533e-11	\\
4.04036	1.73033e-11	\\
4.06313	1.72529e-11	\\
4.08001	1.72095e-11	\\
4.10293	1.71486e-11	\\
4.12102	1.71014e-11	\\
4.13415	1.70706e-11	\\
4.15229	1.70208e-11	\\
4.17638	1.69564e-11	\\
4.19415	1.69076e-11	\\
4.20661	1.68702e-11	\\
4.21883	1.68374e-11	\\
4.24168	1.67676e-11	\\
4.27087	1.66828e-11	\\
4.2816	1.66502e-11	\\
4.30361	1.65805e-11	\\
4.33567	1.64805e-11	\\
4.35606	1.64109e-11	\\
4.36785	1.63738e-11	\\
4.39144	1.62947e-11	\\
4.41208	1.6226e-11	\\
4.42695	1.61757e-11	\\
4.43477	1.61475e-11	\\
4.44879	1.60986e-11	\\
4.472	1.60181e-11	\\
4.51114	1.58782e-11	\\
4.52652	1.58219e-11	\\
4.54622	1.57497e-11	\\
4.56374	1.56858e-11	\\
4.58013	1.56237e-11	\\
4.60286	1.55395e-11	\\
4.62445	1.54575e-11	\\
4.64708	1.53703e-11	\\
4.67292	1.52712e-11	\\
4.68998	1.52047e-11	\\
4.71447	1.51056e-11	\\
4.73037	1.50438e-11	\\
4.74845	1.49733e-11	\\
4.7763	1.48616e-11	\\
4.81018	1.47259e-11	\\
4.84206	1.45925e-11	\\
4.8553	1.45376e-11	\\
4.87923	1.44392e-11	\\
4.8989	1.43547e-11	\\
4.92062	1.42647e-11	\\
4.94125	1.41769e-11	\\
4.96225	1.40888e-11	\\
4.97405	1.40381e-11	\\
4.99798	1.39354e-11	\\
5.01152	1.38783e-11	\\
5.02698	1.38112e-11	\\
5.04296	1.37438e-11	\\
5.07515	1.36017e-11	\\
5.10643	1.34694e-11	\\
5.12456	1.33892e-11	\\
5.13305	1.335e-11	\\
5.16698	1.3204e-11	\\
5.1908	1.30964e-11	\\
5.20705	1.30239e-11	\\
5.24076	1.28746e-11	\\
5.24932	1.28367e-11	\\
5.27866	1.27091e-11	\\
5.30212	1.26023e-11	\\
5.32995	1.24777e-11	\\
5.35464	1.23669e-11	\\
5.38854	1.22157e-11	\\
5.41724	1.20867e-11	\\
5.436	1.2002e-11	\\
5.48011	1.18058e-11	\\
5.51535	1.16482e-11	\\
5.54338	1.15198e-11	\\
5.55911	1.14485e-11	\\
5.58382	1.13383e-11	\\
5.62636	1.11446e-11	\\
5.64758	1.10474e-11	\\
5.67398	1.09308e-11	\\
5.70173	1.0807e-11	\\
5.74764	1.06018e-11	\\
5.77259	1.04877e-11	\\
5.7944	1.0391e-11	\\
5.8163	1.02915e-11	\\
5.84153	1.01805e-11	\\
5.87032	1.00536e-11	\\
5.89715	9.93183e-12	\\
5.92661	9.80371e-12	\\
5.96002	9.65672e-12	\\
5.988	9.53215e-12	\\
6.0035	9.46393e-12	\\
6.02802	9.35468e-12	\\
6.07394	9.15729e-12	\\
6.09769	9.0532e-12	\\
6.12448	8.93674e-12	\\
6.15224	8.81617e-12	\\
6.19105	8.65197e-12	\\
6.22736	8.49565e-12	\\
6.25917	8.35937e-12	\\
6.31269	8.13305e-12	\\
6.35032	7.97495e-12	\\
6.39267	7.79699e-12	\\
6.40727	7.73792e-12	\\
6.46447	7.50405e-12	\\
6.48122	7.43483e-12	\\
6.519	7.28179e-12	\\
6.53461	7.21961e-12	\\
6.5647	7.09721e-12	\\
6.5933	6.98391e-12	\\
6.64122	6.79434e-12	\\
6.68137	6.6373e-12	\\
6.70655	6.54032e-12	\\
6.73582	6.4268e-12	\\
6.76645	6.30723e-12	\\
6.78333	6.24378e-12	\\
6.80961	6.1417e-12	\\
6.85015	5.98993e-12	\\
6.87144	5.91371e-12	\\
6.89538	5.8244e-12	\\
6.9168	5.74601e-12	\\
6.95005	5.62334e-12	\\
6.99506	5.46274e-12	\\
7.02438	5.35472e-12	\\
7.04214	5.29088e-12	\\
7.07241	5.18435e-12	\\
7.10173	5.08193e-12	\\
7.1255	5.00028e-12	\\
7.17333	4.83513e-12	\\
7.20271	4.73599e-12	\\
7.21908	4.68048e-12	\\
7.26382	4.53354e-12	\\
7.29549	4.42635e-12	\\
7.32505	4.33187e-12	\\
7.3725	4.17927e-12	\\
7.40642	4.07196e-12	\\
7.45068	3.93302e-12	\\
7.49681	3.79252e-12	\\
7.54219	3.65624e-12	\\
7.58626	3.52712e-12	\\
7.60773	3.46151e-12	\\
7.63563	3.38113e-12	\\
7.67934	3.25728e-12	\\
7.7144	3.15842e-12	\\
7.74902	3.06449e-12	\\
7.81312	2.89019e-12	\\
7.84339	2.8107e-12	\\
7.87181	2.73526e-12	\\
7.89968	2.66254e-12	\\
7.94355	2.55324e-12	\\
7.99478	2.42667e-12	\\
8.0338	2.33163e-12	\\
8.05909	2.26963e-12	\\
8.08868	2.20129e-12	\\
8.14235	2.07739e-12	\\
8.18281	1.98719e-12	\\
8.22709	1.89032e-12	\\
8.25756	1.82637e-12	\\
8.29441	1.74682e-12	\\
8.33781	1.66001e-12	\\
8.3573	1.61982e-12	\\
8.38705	1.56103e-12	\\
8.43292	1.47127e-12	\\
8.45956	1.42142e-12	\\
8.47307	1.39749e-12	\\
8.5	1.34709e-12	\\
8.52038	1.31078e-12	\\
8.56203	1.23646e-12	\\
8.59445	1.1825e-12	\\
8.62651	1.12704e-12	\\
8.67078	1.0566e-12	\\
8.71021	9.93261e-13	\\
8.74704	9.37139e-13	\\
8.78105	8.85347e-13	\\
8.81942	8.30447e-13	\\
8.84083	8.00471e-13	\\
8.88987	7.34135e-13	\\
8.91925	6.93667e-13	\\
8.98304	6.12621e-13	\\
9.02012	5.69877e-13	\\
9.06756	5.13867e-13	\\
9.10711	4.72178e-13	\\
9.13123	4.45921e-13	\\
9.16562	4.10949e-13	\\
9.18945	3.87856e-13	\\
9.24534	3.3612e-13	\\
9.30029	2.89602e-13	\\
9.33588	2.58515e-13	\\
9.37337	2.30205e-13	\\
9.43709	1.86517e-13	\\
9.47973	1.5743e-13	\\
9.54244	1.21125e-13	\\
9.57435	1.05027e-13	\\
9.61281	8.58202e-14	\\
9.64683	7.18869e-14	\\
9.6953	5.4734e-14	\\
9.77374	2.89768e-14	\\
9.81293	1.88183e-14	\\
9.8489	1.19904e-14	\\
9.90572	4.44089e-15	\\
9.932	1.83187e-15	\\
};
\addlegendentry{$\Delta_{N}(\sigma)$};

\end{axis}
\end{tikzpicture}%

        \end{subfigure}
    \end{center}
    \caption{
    Vergleich des tatsächlichen Fehlers $s - s_N$ (gestrichelt) und Fehlerschätzer $\Delta^s_N$ über $\Xi_\text{test} \subset \mathcal D$ mit $10^4$ Elementen für $N = 1 \ldots 4$.}
    \label{fig:plot_s_fehler}
\end{figure}

\subsection{Mehrdimensionale Parameterräume} % (fold)
\label{sub:mehrdimensionale_parameterr_ume}

Wir können das oben betrachtete Problem auch auf höherdimensionale Parameterräume verallgemeinern. Teilen wir $\Omega$ in vier gleichgroße Quadrate $\Omega_1 = [0, \frac 12] \times [0, \frac 12]$, $\Omega_2 = [\frac 12, 1] \times [0, \frac 12]$, $\Omega_3 = [0, \frac 12] \times [\frac 12, 1]$ und $\Omega_4 = [\frac 12, 1] \times [\frac 12, 1]$ statt den zwei Rechtecken ein, und ordnen $\Omega_1$, $\Omega_2$ und $\Omega_3$ jeweils ein $\mu_i$ zu, dann erhalten wir damit einen dreidimensionalen Parameterraum $\mathcal D \subset \mathbb{R}^3$. Analog zum eindimensionalen Beispiel wählen wir $\mathcal D = [0.1, 10]^3 \subset \mathbb{R}^3$.

Zur Lösung des Variatonsproblems für ein festes $\mu \in \mathcal D$ verwenden wir erneut die Finite-Elemente-Methode, diesmal mit $\mathcal N = 319$.

Als obere Schranke für die maximale Effektivität erhält man in diesem Fall $\eta^s_{\max,\text{UB}}  \leq 100$.

Um ein besseres Bild des Fehlers über $\mathcal D$ zu erhalten, vergrößeren wir unsere diskreten Teilmengen $\Xi_\text{train} = \Xi_\text{test} \subset \mathcal D$ auf $10^5$ Elemente, welche wieder zufällig logarithmisch gleichverteilt gewählt werden.
Tabelle \ref{tab:dreidim} und Abbildung \ref{fig:3par_plot_fehler_nach_N} zeigen die Entwicklung des Fehlers für dieses Beispiel. Wie man sieht, benötigen wir für diesen dreidimensionalen Parameterraum bereits ein deutlich größeres $N$ um die Fehlertoleranz von $10^{-8}$ zu erreichen, aber nach wie vor ist $N \ll \mathcal N$.

\begin{figure}[h!]
    \begin{center}
        \small
        % \newlength\figureheight
        % \newlength\figurewidth
        \setlength\figureheight{5cm}
        \setlength\figurewidth{0.6\textwidth}
        % This file was created by matlab2tikz v0.4.6 running on MATLAB 8.1.
% Copyright (c) 2008--2014, Nico Schlömer <nico.schloemer@gmail.com>
% All rights reserved.
% Minimal pgfplots version: 1.3
%
% The latest updates can be retrieved from
%   http://www.mathworks.com/matlabcentral/fileexchange/22022-matlab2tikz
% where you can also make suggestions and rate matlab2tikz.
%
\begin{tikzpicture}

\begin{axis}[%
width=10cm,
height=7cm,
scale only axis,
xmin=0,
xmax=30,
xlabel={reduced basis dimension $N$},
ymode=log,
ymin=1e-10,
ymax=100,
yminorticks=true,
ultra thick,
ylabel={$\max_{\bm \sigma \in \mathcal P_{\mathrm{train}}} \Delta_{N}(\bm \sigma)$},
% title={$\text{Entwicklung von }\Delta{}^\text{s}_{\text{N,max}}\text{ in Abhängigkeit von N}$}
]
\addplot [color=matlab1,solid,mark size=1.4pt,mark=*,mark options={solid},forget plot]
  table[row sep=crcr]{
1	16.5737	\\
2	4.37484	\\
3	6.0695	\\
4	6.15463	\\
5	1.17623	\\
6	1.12893	\\
7	0.44049	\\
8	0.120821	\\
9	0.0109261	\\
10	0.00458001	\\
11	0.00172374	\\
12	0.000516839	\\
13	0.000163186	\\
14	6.59753e-05	\\
15	1.37859e-05	\\
16	5.48762e-06	\\
17	4.57878e-06	\\
18	2.18171e-06	\\
19	9.46407e-07	\\
20	5.66519e-07	\\
21	3.89545e-07	\\
22	3.57025e-07	\\
23	2.25267e-07	\\
24	1.22227e-07	\\
25	3.26813e-08	\\
26	2.48763e-08	\\
27	8.18716e-09	\\
};
\end{axis}
\end{tikzpicture}%

    \end{center}
    \caption{Entwicklung von $\Delta{}^\text{s}_{\text{N,max}}$ in Abhängigkeit von $N$ beim dreidimensionalen Parameterraum. Bei $N = 27$ wird die geforderte Fehlertoleranz von $10^{-8}$ erstmals erreicht.}
    \label{fig:3par_plot_fehler_nach_N}
\end{figure}

\begin{table}[h!]
    \begin{center}
        \small
        \begin{tabular}{r|llll}
        $N$ & $\Delta^s_{N,\max}$ & $\eta^s_{N,\text{ave}}$ & $\eta^s_{N,\max}$ & $\rho^s_{\text{err}, N}$ \\
        \hline
            $1$ & $1.6574 \cdot 10^{1}$ & $2.5379$ & $29.7780$ & $3.6535$ \\
            $2$ & $4.3748 \cdot 10^{0}$ & $2.2085$ & $19.8043$ & $3.5034$ \\
            $3$ & $6.0695 \cdot 10^{0}$ & $2.7280$ & $45.2241$ & $5.7009$ \\
            $4$ & $6.1546 \cdot 10^{0}$ & $2.2541$ & $45.5506$ & $6.5551$ \\
            $5$ & $1.1762 \cdot 10^{0}$ & $2.2814$ & $46.7323$ & $8.5654$ \\
            $6$ & $1.1289 \cdot 10^{0}$ & $4.2444$ & $62.2367$ & $8.7753$ \\
            $7$ & $4.4049 \cdot 10^{-1}$ & $4.7197$ & $64.1045$ & $15.1326$ \\
            $8$ & $1.2082 \cdot 10^{-1}$ & $4.5329$ & $42.8482$ & $6.8982$ \\
            $9$ & $1.0926 \cdot 10^{-2}$ & $3.8287$ & $57.1067$ & $5.2397$ \\
            $10$ & $4.5800 \cdot 10^{-3}$ & $4.9657$ & $51.0632$ & $29.2355$ \\
            $11$ & $1.7237 \cdot 10^{-3}$ & $4.9454$ & $51.4003$ & $13.6833$ \\
            $12$ & $5.1684 \cdot 10^{-4}$ & $4.6870$ & $53.3714$ & $9.7341$ \\
            $13$ & $1.6319 \cdot 10^{-4}$ & $4.4641$ & $55.5663$ & $3.0734$ \\
            $14$ & $6.5975 \cdot 10^{-5}$ & $4.0050$ & $58.5909$ & $4.0218$ \\
            $15$ & $1.3786 \cdot 10^{-5}$ & $4.1535$ & $63.5388$ & $3.0056$ \\
            $16$ & $5.4876 \cdot 10^{-6}$ & $3.9654$ & $49.5149$ & $5.7093$ \\
            $17$ & $4.5788 \cdot 10^{-6}$ & $3.9664$ & $49.8546$ & $7.6339$ \\
            $18$ & $2.1817 \cdot 10^{-6}$ & $5.0265$ & $56.0621$ & $7.1431$ \\
            $19$ & $9.4641 \cdot 10^{-7}$ & $4.5956$ & $54.2310$ & $3.1401$ \\
            $20$ & $5.6652 \cdot 10^{-7}$ & $4.7061$ & $59.4396$ & $9.4472$ \\
            $21$ & $3.8955 \cdot 10^{-7}$ & $4.6886$ & $57.4457$ & $7.6115$ \\
            $22$ & $3.5702 \cdot 10^{-7}$ & $4.7276$ & $56.2459$ & $8.4688$ \\
            $23$ & $2.2527 \cdot 10^{-7}$ & $4.8083$ & $58.0783$ & $11.1384$ \\
            $24$ & $1.2223 \cdot 10^{-7}$ & $4.8561$ & $56.4697$ & $6.5937$ \\
            $25$ & $3.2681 \cdot 10^{-8}$ & $4.4157$ & $59.7853$ & $2.9218$ \\
            $26$ & $2.4876 \cdot 10^{-8}$ & $3.8898$ & $48.0801$ & $2.3823$ \\
            $27$ & $8.1872 \cdot 10^{-9}$ & $4.0380$ & $53.6857$ & $5.3194$
        \end{tabular}
        \caption{Es reichen $N = 27$ Snapshot-Parameter, um die Fehlertoleranz von $10^{-8}$ zu erreichen.}
        \label{tab:dreidim}
    \end{center}
\end{table}

% subsection mehrdimensionale_parameterr_ume (end)

% paragraph numerische_ergebnisse (end)

% section beispiel (end)
