% -*- root: ../main.tex -*-

\documentclass[../main.tex]{subfiles}
\begin{document}

\chapter{Sonstiger Kram}

\begin{Lemma}[Berechnung der Rieszchen Darstellung]
\label{lemma:berechnung_rieszsche_darstellung}
    Sei $X$ ein endlichdimensionaler Hilbertraum mit Basis ${\phi_i}_{i=1}^{N}$.
    Sei weiter $g \in X'$.
    Der Koeffizientenvektor $\vec{v} \in \mathbb{R}^{N}$ der Rieszschen Darstellung $v_g = \sum_{i=1}^{N} v_{i} \phi_{i} \in X$ von $g$, das heißt, es gilt $\skp{v_g}{w}{X} = \skp{g}{w}{X' \times X}$ für alle $w \in X$, lässt sich durch das Gleichungssystem $\mat{K}\vec{v} = \vec{g}$ berechnen, wobei $\mat{K} = [\skp{\phi_{i}}{\phi_{j}}{X}]_{i,j}$ die Massematrix zum inneren Produkt auf $X$ sei und weiter $\vec{g} = [\skp{g}{\phi_{i}}{X' \times X}]_{i}$ sei.
\end{Lemma}

% section sonstiger_kram (end)

\end{document}
