%!TEX root = main.tex

%%%%%%%%%%%%%%%%%%%%%%%%%%%%%%%%%%%%%%%%%%%%%%%%%%%%%%%%%%%%%%%%%%%%%%%%%%%%%%%
%%% Allgemeines

% Mehr Speicher für latex
\usepackage{etex}

% Encoding und Schriftsystem
\usepackage[utf8]{inputenc}
\usepackage[T1]{fontenc}

% Deutsche Silbentrennung
\usepackage[ngerman]{babel}

% Deutsche Anführungszeichen
\usepackage[babel,german=quotes]{csquotes}

% Bessere Standard-Schriftart
\usepackage{lmodern}

% Fraktur-Schriftsatz
\usepackage{eufrak}

% Typographische Kleinigkeiten
\usepackage{microtype}

% Graphiken und Farben
\usepackage{graphicx, xcolor}
\definecolor{jgu:rot}{RGB}{193,0,42}
\definecolor{jgu:hgrau}{RGB}{172,172,172}
\definecolor{jgu:dgrau}{RGB}{99,99,99}

% PDF-Verlinkungen und Metadaten
\usepackage[
	colorlinks=true,
	linkcolor=jgu:rot,          % color of internal links (change box color with linkbordercolor)
	citecolor=jgu:rot,        % color of links to bibliography
	filecolor=jgu:rot,      % color of file links
	urlcolor=jgu:rot
]{hyperref}

\newcommand{\changefont}[3]{
\fontfamily{#1} \fontseries{#2} \fontshape{#3} \selectfont}

% Andere Titles
% \usepackage{titlesec}

% \usepackage{nameref}

% Bibliographie-Stil
\usepackage[%
    backend=biber,
    style=alphabetic,
    backref=true,
]{biblatex}
\addbibresource{literature.bib}

% Fancyhdr-Ersatz für scrbook
\usepackage[
    automark,
    % headsepline
]{scrlayer-scrpage}

% enums
\usepackage{enumitem}
\newlist{thmenumerate}{enumerate}{1}
\setlist[thmenumerate]{label={\upshape(\roman*)}, align=left, widest=iii, leftmargin=*}

% Dashed Linien...
\usepackage{dashrule}

% Symbolverzeichnis
\usepackage[german,intoc]{nomencl}
\setlength{\nomitemsep}{-\parsep}
\makenomenclature

%%%%%%%%%%%%%%%%%%%%%%%%%%%%%%%%%%%%%%%%%%%%%%%%%%%%%%%%%%%%%%%%%%%%%%%%%%%%%%%
%%% Mathematik

% Standardpackages
\usepackage{amsmath, amssymb, stmaryrd, mathtools}

% "Bessere" Theorem-Umgebungen
\usepackage[amsmath, thmmarks, hyperref, thref]{ntheorem}

% Setze Label-Nummern nur, wenn diese auch referenziert werden
\mathtoolsset{showonlyrefs, showmanualtags}

\providecommand\given{} % so it exists
\newcommand\SetSymbol[1][]{
   \nonscript\,#1\vert\nonscript\,\mathopen{}\allowbreak}
\DeclarePairedDelimiterX\Set[1]{\lbrace}{\rbrace}{ \renewcommand\given{\SetSymbol[\delimsize]} #1 }

%%%%%%%%%%%%%%%%%%%%%%%%%%%%%%%%%%%%%%%%%%%%%%%%%%%%%%%%%%%%%%%%%%%%%%%%%%%%%%%
%%% Nebensächliches

\newcommand{\altfont}{\fontfamily{qpl}}
% \newcommand{\altsffont}{\fontfamily{phv}}
\newcommand{\altsffont}{\fontfamily{pfr}\fontseries{l}}

% Querformatseiten
\usepackage{pdflscape}

% Labels am Seitenrand anzeigen
% \usepackage{showlabels}

% todos
\usepackage[%
    german,
    % disable
]{todonotes}

% Blindtext
\usepackage{blindtext}
% \blindmathtrue{}


\usepackage{tocstyle}
\usetocstyle{KOMAlike}

%%% Chapter-Style und mehr
% Überschriften in Dunkelgrau
% \addtokomafont{disposition}{\color{jgu:dgrau}}
% Überschriften in TeX Gyre Pagella
\addtokomafont{disposition}{\normalfont\altfont\selectfont}

% Chapterstyle
\renewcommand*{\chapterheadstartvskip}{\vspace*{5\baselineskip}}
\renewcommand*{\chapterheadendvskip}{\vspace*{2\baselineskip}}
\renewcommand*{\chapterformat}{%
				\raggedright
				\color{jgu:rot}
                \altfont\fontsize{60}{30}\selectfont\thechapter
                \fontsize{20}{30}\scshape\selectfont\enskip\chapappifchapterprefix
                }
\renewcommand*{\raggedchapter}{\raggedleft}

% Schriften für Überschriften (einzeln)
% \addtokomafont{chapter}{\normalfont\altfont\selectfont}
% \addtokomafont{chapter}{\color{black}}
% \addtokomafont{section}{\normalfont\altfont\selectfont}
% \addtokomafont{subsection}{\normalfont\altfont\selectfont}
% \addtokomafont{subsubsection}{\normalfont\altfont\selectfont}
% \addtokomafont{minisec}{\normalfont\altfont\selectfont}
% \addtokomafont{paragraph}{\normalfont\altfont\selectfont}
\addtokomafont{paragraph}{\bfseries}

\addtokomafont{pageheadfoot}{\color{jgu:dgrau}\normalfont\altfont\selectfont}% Fußzeile normale Schrift

\renewcommand*\dictumwidth{.95\linewidth}
\renewcommand*\dictumrule{}
\renewcommand*\dictumauthorformat[1]{--- #1}
\addtokomafont{dictumtext}{\color{jgu:dgrau}\normalfont\altsffont\fontsize{9}{12}\selectfont\itshape}
\addtokomafont{dictumauthor}{\color{jgu:dgrau}\normalfont\altsffont\fontsize{8}{12}\selectfont}

\addtokomafont{pagenumber}{\color{jgu:dgrau}\normalfont\altfont\selectfont}

% \addtokomafont{sectionentrypagenumber}{\color{black}}
\addtokomafont{chapterentrypagenumber}{\color{black}\rmfamily}
\addtokomafont{chapterentry}{\rmfamily\bfseries}

\usepackage[style=base,font+=small,labelfont+=bf,margin=1em]{caption}
\usepackage{subcaption}

\usepackage{tikz}
\usepackage[subpreambles]{standalone}

% Akronymverzeichnis
\usepackage{acro}
\acsetup{
    page-ref=paren,
    pages=first,
    page-name={siehe S.\@\,},
}
