% -*- root: main.tex -*-

\RequirePackage[ngerman=ngerman-x-latest]{hyphsubst}
\documentclass[
  % draft,
  % final,
  a4paper,
  BCOR=1cm,
  twoside,
  % twoside=semi,
  % oneside,
  toc=bibliography,
  toc=listof,
  chapterprefix=true,
  version=last,
  cleardoublepage=empty,
  pagesize=auto,
]{scrbook}

%!TEX root = main.tex

%%%%%%%%%%%%%%%%%%%%%%%%%%%%%%%%%%%%%%%%%%%%%%%%%%%%%%%%%%%%%%%%%%%%%%%%%%%%%%%
%%% Allgemeines

% Mehr Speicher für latex
\usepackage{etex}

% Encoding und Schriftsystem
\usepackage[utf8]{inputenc}
\usepackage[T1]{fontenc}

% Deutsche Silbentrennung
\usepackage[ngerman]{babel}

% Bessere Standard-Schriftart
\usepackage{lmodern}

% Typographische Kleinigkeiten
\usepackage{microtype}

% Graphiken und Farben
\usepackage{graphicx, color}

% PDF-Verlinkungen und Metadaten
\usepackage[colorlinks=false]{hyperref}

% Bibliographie-Stil
\bibliographystyle{amsplain}

% Blindtext
\usepackage{blindtext}
\blindmathtrue

% Dashed Linien...
\usepackage{dashrule}

%%%%%%%%%%%%%%%%%%%%%%%%%%%%%%%%%%%%%%%%%%%%%%%%%%%%%%%%%%%%%%%%%%%%%%%%%%%%%%%
%%% Mathematik

% Standardpackages
\usepackage{amsmath, amssymb, stmaryrd, mathtools}

% "Bessere" Theorem-Umgebungen
\usepackage[standard,amsmath,thmmarks,hyperref]{ntheorem}

% Setze Label-Nummern nur, wenn diese auch referenziert werden
\mathtoolsset{showonlyrefs=true}

%!TEX root = main.tex

% griechisches Alphabet
\renewcommand{\epsilon}{\varepsilon}
\renewcommand{\theta}{\vartheta}

%%% Operatoren und Funktionen
\DeclarePairedDelimiter{\abs}{\lvert}{\rvert}
\DeclarePairedDelimiter{\norm}{\lVert}{\rVert}

\DeclarePairedDelimiter{\ceil}{\lceil}{\rceil}
\DeclarePairedDelimiter{\floor}{\lfloor}{\rfloor}

\newcommand{\skprod}[2]{\left\langle#1,#2\right\rangle}
\newcommand{\fracpart}[2]{\frac{\partial#1}{\partial#2}}

\newcommand{\Transp}{^{\mathrm{T}}}
\newcommand{\Stern}{^{*}}
\newcommand{\Int}[1]{#1^\circ}
\newcommand{\Ext}[1]{\overline{#1}}

\DeclareMathOperator{\spn}{span}
\DeclareMathOperator{\ee}{e}
\DeclareMathOperator{\ii}{i}

\newcommand{\grad}{\nabla}
\DeclareMathOperator{\divergenz}{div}
\newcommand{\hesse}{\nabla^2}

\newcommand\restr[2]{\ensuremath{\left.#1\right|_{#2}}}

\newcommand{\fa}{\text{für alle}~}

% fetter Vektor
\renewcommand{\vec}[1]{\mathbf{#1}}
\newcommand{\mat}[1]{\mathbf{#1}}

% Differential-d
\newcommand*\diff{\mathop{}\!\mathrm{d}}
\newcommand*\Diff[1]{\mathop{}\!\mathrm{d^#1}}

\DeclareMathOperator*{\esssup}{ess\,sup}
\newcommand\blank{{\mkern2mu\cdot\mkern2mu}}

% -*- root: main.tex -*-

\newcommand{\titel}{Numerische Behandlung von Self-Consistent Field Theory-Modellen mittels Reduzierter-Basis-Methoden}
\newcommand{\art}{Masterarbeit}
\newcommand{\autor}{Alexej Disterhoft}
\newcommand{\fach}{Mathematik}
\newcommand{\matrikelnr}{2669611}
\newcommand{\erstgutachter}{Prof. Dr. Thorsten Raasch}
\newcommand{\zweitgutachter}{Prof. Dr. Martin Hanke-Bourgeois}
% \newcommand{\monat}{Irgendwann}
% \newcommand{\jahr}{2015}
\newcommand{\ort}{Mainz}
\newcommand{\logo}{figures/title/logo_gross.eps}

\hypersetup{pdfinfo={
  Title=\titel,
  Author=\autor}
}


%!TEX root = main.tex

\DeclareAcronym{scft}{
	short            = SCFT,
	long             = selbstkonsistente Feldtheorie,
	long-plural-form = selbstkonsistenten Feldtheorie,
	short-plural     = {},
	foreign          = \foreign{engl.}{self-consistent field theory}
}

\DeclareAcronym{rbm}{
	short = RBM,
	long  = Reduzierte-Basis-Methode
}

\DeclareAcronym{fem}{
	short = FEM,
	long = Finite-Elemente-Methode
}

\DeclareAcronym{bnb}{
	short = BNB,
	long = Banach-Ne{\v c}as-Babu{\v s}ka-Theorem
}

% -*- root: main.tex -*-

\newglossaryentry{symb:stetige_inklusion} {
    name={\ensuremath{\hookrightarrow}},
    description={Stetige Einbettung}
}
\newglossaryentry{symb:bochner_skp} {
    name={\ensuremath{[f, g]}},
    description={Kurzschreibweise für \ensuremath{\int_{T} \skprod{f(t)}{g(t)} \diff t}}
}


\addbibresource{literature.bib}
% \addbibresource{literature_old.bib}

% Zeige overfull boxes in nicht-Draft Mode
% \overfullrule=1mm

\begin{document}
    %%% Frontmatter-Teil
    \frontmatter{}

    % Deckblack / Titelseite
    %!TEX root = ../main.tex

\thispagestyle{plain}
\begin{titlepage}

\begin{center}

\huge{\textbf{\titel}}\\[1.5ex]
\LARGE{\textbf{\untertitel}}\\[6ex]
\LARGE{\textbf{\art}}\\[1.5ex]
\Large{im Fach \fach}\\[18ex]

\includegraphics[scale=0.5]{\logo}\\[6ex]

\normalsize
\begin{tabular}{p{5.4cm}p{6cm}}\\
vorgelegt von:  & \quad \autor\\[1.2ex]
Matrikelnummer: & \quad \matrikelnr\\[1.2ex]
Erstgutachter:  & \quad \erstgutachter\\[1.2ex]
Zweitgutachter: & \quad \zweitgutachter\\[3ex]
\end{tabular}

\end{center}
\end{titlepage}


    % Danksagung
    % -*- root: ../main.tex -*-

\thispagestyle{empty}
\vspace*{0.2\textheight}
\noindent\enquote{I guess you could call it a \enquote{failure}, but I prefer the term \enquote{learning
experience}.}\bigbreak

\hfill Andy Weir, \textit{The Martian}

\vfill{}
\begin{flushright}
\emph{Für meine Eltern.}
\end{flushright}
\cleardoublepage


    % Inhaltsverzeichnis
    \tableofcontents

    %%% Hauptteil
    \mainmatter{}

    % Kapitel einbinden
    \subfile{chapters/cha1_einleitung}
    \subfile{chapters/cha2_grundlagen}
    \subfile{chapters/cha3_propagator_dgl}
    \subfile{chapters/cha4_galerkin}
    \subfile{chapters/cha5_rbm}
    \subfile{chapters/cha6_ausblick}

    %%% Anhang
    \appendix{}

    % DVD Inhalt
    \subfile{chapters/chaA_dvd.tex}

    %% Los geht's mit den Verzeichnissen

    % Abbildungsverzeichnis
    \listoffigures

    % Tabellenverzeichnis
    \listoftables

    % Literaturverzeichnis
    \printbibliography

    % Symbolverzeichnis
    % \glsaddall[]
    % \printglossary[type=symbolslist, nonumberlist, style=long]

    % Eidesstattliche Erklärung
    % \cleardoublepage
    %!TEX root = ../main.tex

\chapter*{Eidesstattliche Erklärung}

Ich versichere hiermit, dass ich die vorliegende Masterarbeit selbständig
verfasst und keine anderen als die angegebenen Quellen und Hilfsmittel benutzt
habe, wobei ich alle wörtlichen und sinngemäßen Zitate als solche gekennzeichnet
habe. Die Arbeit wurde bisher keiner anderen Prüfungsbehörde vorgelegt und auch
nicht veröffentlicht.\\[6ex]

\begin{flushright}
\ort, den \today

\color{jgu_hellgrau}\hdashrule[-0.5cm]{5cm}{0.5pt}{1pt}
\end{flushright}

\end{document}
