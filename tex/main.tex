% -*- root: main.tex -*-

\documentclass[
  draft,
  % final,
  a4paper,
  % BCOR=1cm,
  % twoside,
  twoside=semi,
  % oneside,
  toc=bibliography,
  toc=listof,
  chapterprefix=true,
  version=last,
  cleardoublepage=empty,
]{scrbook}

%!TEX root = main.tex

%%%%%%%%%%%%%%%%%%%%%%%%%%%%%%%%%%%%%%%%%%%%%%%%%%%%%%%%%%%%%%%%%%%%%%%%%%%%%%%
%%% Allgemeines

% Mehr Speicher für latex
\usepackage{etex}

% Encoding und Schriftsystem
\usepackage[utf8]{inputenc}
\usepackage[T1]{fontenc}

% Deutsche Silbentrennung
\usepackage[ngerman]{babel}

% Bessere Standard-Schriftart
\usepackage{lmodern}

% Typographische Kleinigkeiten
\usepackage{microtype}

% Graphiken und Farben
\usepackage{graphicx, color}

% PDF-Verlinkungen und Metadaten
\usepackage[colorlinks=false]{hyperref}

% Bibliographie-Stil
\bibliographystyle{amsplain}

% Blindtext
\usepackage{blindtext}
\blindmathtrue

% Dashed Linien...
\usepackage{dashrule}

%%%%%%%%%%%%%%%%%%%%%%%%%%%%%%%%%%%%%%%%%%%%%%%%%%%%%%%%%%%%%%%%%%%%%%%%%%%%%%%
%%% Mathematik

% Standardpackages
\usepackage{amsmath, amssymb, stmaryrd, mathtools}

% "Bessere" Theorem-Umgebungen
\usepackage[standard,amsmath,thmmarks,hyperref]{ntheorem}

% Setze Label-Nummern nur, wenn diese auch referenziert werden
\mathtoolsset{showonlyrefs=true}

%!TEX root = main.tex

% griechisches Alphabet
\renewcommand{\epsilon}{\varepsilon}
\renewcommand{\theta}{\vartheta}

%%% Operatoren und Funktionen
\DeclarePairedDelimiter{\abs}{\lvert}{\rvert}
\DeclarePairedDelimiter{\norm}{\lVert}{\rVert}

\DeclarePairedDelimiter{\ceil}{\lceil}{\rceil}
\DeclarePairedDelimiter{\floor}{\lfloor}{\rfloor}

\newcommand{\skprod}[2]{\left\langle#1,#2\right\rangle}
\newcommand{\fracpart}[2]{\frac{\partial#1}{\partial#2}}

\newcommand{\Transp}{^{\mathrm{T}}}
\newcommand{\Stern}{^{*}}
\newcommand{\Int}[1]{#1^\circ}
\newcommand{\Ext}[1]{\overline{#1}}

\DeclareMathOperator{\spn}{span}
\DeclareMathOperator{\ee}{e}
\DeclareMathOperator{\ii}{i}

\newcommand{\grad}{\nabla}
\DeclareMathOperator{\divergenz}{div}
\newcommand{\hesse}{\nabla^2}

\newcommand\restr[2]{\ensuremath{\left.#1\right|_{#2}}}

\newcommand{\fa}{\text{für alle}~}

% fetter Vektor
\renewcommand{\vec}[1]{\mathbf{#1}}
\newcommand{\mat}[1]{\mathbf{#1}}

% Differential-d
\newcommand*\diff{\mathop{}\!\mathrm{d}}
\newcommand*\Diff[1]{\mathop{}\!\mathrm{d^#1}}

\DeclareMathOperator*{\esssup}{ess\,sup}
\newcommand\blank{{\mkern2mu\cdot\mkern2mu}}

% -*- root: main.tex -*-

\newcommand{\titel}{Numerische Behandlung von Self-Consistent Field Theory-Modellen mittels Reduzierter-Basis-Methoden}
\newcommand{\art}{Masterarbeit}
\newcommand{\autor}{Alexej Disterhoft}
\newcommand{\fach}{Mathematik}
\newcommand{\matrikelnr}{2669611}
\newcommand{\erstgutachter}{Prof. Dr. Thorsten Raasch}
\newcommand{\zweitgutachter}{Prof. Dr. Martin Hanke-Bourgeois}
% \newcommand{\monat}{Irgendwann}
% \newcommand{\jahr}{2015}
\newcommand{\ort}{Mainz}
\newcommand{\logo}{figures/title/logo_gross.eps}

\hypersetup{pdfinfo={
  Title=\titel,
  Author=\autor}
}


%!TEX root = main.tex

\DeclareAcronym{scft}{
	short            = SCFT,
	long             = selbstkonsistente Feldtheorie,
	long-plural-form = selbstkonsistenten Feldtheorie,
	short-plural     = {},
	foreign          = \foreign{engl.}{self-consistent field theory}
}

\DeclareAcronym{rbm}{
	short = RBM,
	long  = Reduzierte-Basis-Methode
}

\DeclareAcronym{fem}{
	short = FEM,
	long = Finite-Elemente-Methode
}

\DeclareAcronym{bnb}{
	short = BNB,
	long = Banach-Ne{\v c}as-Babu{\v s}ka-Theorem
}

% -*- root: main.tex -*-

\newglossaryentry{symb:stetige_inklusion} {
    name={\ensuremath{\hookrightarrow}},
    description={Stetige Einbettung}
}
\newglossaryentry{symb:bochner_skp} {
    name={\ensuremath{[f, g]}},
    description={Kurzschreibweise für \ensuremath{\int_{T} \skprod{f(t)}{g(t)} \diff t}}
}


\addbibresource{literature.bib}
\addbibresource{literature_old.bib}

\begin{document}
    %%% Frontmatter-Teil
    \frontmatter{}

    % Deckblack / Titelseite
    %!TEX root = ../main.tex

\thispagestyle{plain}
\begin{titlepage}

\begin{center}

\huge{\textbf{\titel}}\\[1.5ex]
\LARGE{\textbf{\untertitel}}\\[6ex]
\LARGE{\textbf{\art}}\\[1.5ex]
\Large{im Fach \fach}\\[18ex]

\includegraphics[scale=0.5]{\logo}\\[6ex]

\normalsize
\begin{tabular}{p{5.4cm}p{6cm}}\\
vorgelegt von:  & \quad \autor\\[1.2ex]
Matrikelnummer: & \quad \matrikelnr\\[1.2ex]
Erstgutachter:  & \quad \erstgutachter\\[1.2ex]
Zweitgutachter: & \quad \zweitgutachter\\[3ex]
\end{tabular}

\end{center}
\end{titlepage}


    % Danksagung
    % % -*- root: ../main.tex -*-

\thispagestyle{empty}
\vspace*{0.2\textheight}
\noindent\enquote{I guess you could call it a \enquote{failure}, but I prefer the term \enquote{learning
experience}.}\bigbreak

\hfill Andy Weir, \textit{The Martian}

\vfill{}
\begin{flushright}
\emph{Für meine Eltern.}
\end{flushright}
\cleardoublepage


    % Abstrakt
    % %!TEX root = ../main.tex

\chapter{Zusammenfassung} % (fold)
\label{cha:Zusammenfassung}

\blindtext

% chapter Zusammenfassung (end)


    % Inhaltsverzeichnis
    \tableofcontents

    % TODO-Liste
    % \listoftodos

    %%% Hauptteil
    \mainmatter{}

    % Kapitel einbinden
    \subfile{chapters/cha1_einleitung}
    \subfile{chapters/cha2_grundlagen}
    \subfile{chapters/cha3_propagator_dgl}
    \subfile{chapters/cha4_galerkin}
    \subfile{chapters/cha5_rbm}
    \subfile{chapters/cha6_ausblick}

    % % \input{chapters/cha5_eindim}
    % Alles, was noch unbedingt rein muss
    % %!TEX root = ../main.tex

\setchapterpreamble[ul][0.6\textwidth]{%
    \dictum[Terry Pratchett]{\enquote{Coffee is a way of stealing time that should by rights belong to your older self.}}
    \vspace*{2\baselineskip}
}
\chapter{Funktionalanalytische Grundlagen} % (fold)
\label{cha:funktionalanalytische_grundlagen}

\todo[inline]{Ständig: ordnen, sortieren, aufräumen, erweitern.}

\section{Orthogonale Funktionen und Polynome}
\label{sec:orthogonale_funktionen_und_polynome}

\begin{Satz}[Orthogonalität trigonometrischer Funktionen]
\label{satz:trigonometrische_funktionen_orthogonal}
    Seien $k, l \in \mathbb{N}$.
    Dann gilt
    \begin{align}
        \skprod{\sin(\pi k x)}{\sin(\pi l x)}_{L_{2}([0, 1])} &= \frac{1}{2} \delta_{kl},
        % \quad\text{und}\quad
        \\\skprod{\cos(\pi k x)}{\cos(\pi l x)}_{L_{2}([0, 1])} &= \frac{1}{2} \delta_{kl},
        \\\skprod{\sin(\pi k x)}{\cos(\pi l x)}_{L_{2}([0, 1])} &= 0.
    \end{align}
\end{Satz}

\begin{Definition}[Legendre-Polynome]
\label{definition:legendre_polynome}
    Sei $I = [-1, 1]$.
    Die Legendre-Polynome $L_{n} \in \Pi_{n}$ sind definiert durch
    \begin{equation}
        L_{n}(x) = \frac{1}{2^{n}n!}\frac{\diff^{n}}{\diff x^{n}} (x^{2} - 1)^{n}.
    \end{equation}
    Durch die Transformation $x \mapsto 2x - 1$ erhält man die auf das Interval $[0, 1]$ geshifteten Legendre-Polynome $\tilde L_{n}$.
\end{Definition}

\begin{Satz}[Orthogonalität der Legendre-Polynome]
\label{satz:legendre_polynome_orthogonal}
    Die Legendre-Polynome $L_{n}$ sind orthogonal bezüglich der $L_{2}([-1, 1])$-Norm, denn es gilt
    \begin{equation}
        \skprod{L_{n}}{L_{m}}_{L_{2}([-1, 1])} = \frac{2}{2n + 1} \delta_{n m}.
    \end{equation}
    Auch für die geshifteten Legendre-Polynome $\tilde L_{n}$ gilt Orthogonalität, denn es ist
    \begin{equation}
        \skprod{\tilde L_{n}}{\tilde L_{m}}_{L_{2}([0, 1])} = \frac{1}{2n + 1} \delta_{n m}.
    \end{equation}
\end{Satz}

\begin{Bemerkung}
\label{satz:legendre_polynome_rekursion}
    Die Legendre-Polynome $L_{n}$ erfüllen die Rekursionsformel
    \begin{equation}
        n L_{n}(x) = (2n - 1) x L_{n-1}(x) - (n - 1) L_{n-2}(x), \quad L_{0}(x) = 1, L_{1}(x) = x.
    \end{equation}
    Analog gilt für die erste Ableitung $L_{n}'$ die Rekursionsformel
    \begin{equation}
        (n - 1) L_{n}'(x) = (2n -1) x L_{n-1}'(x) - n L_{n-2}'(x), \quad L_{0}'(x) = 0, L_{1}'(x) = 1.
    \end{equation}
\end{Bemerkung}

\section{Sonstiges} % (fold)
\label{sec:sonstiges}

% \begin{Lemma}
%     $\mathcal C^{0}([a, b]; X)$ liegt dicht in $L_{p}(a, b; X)$ für $1 \leq p < \infty$.
% \end{Lemma}

% TODO: zitieren
\begin{Satz}[Poincaré-Friedrichs-Ungleichung, vgl. {{\cite[Lemma 89.4]{HankeBourgeois:2009fk}}}]
\label{satz:grundlagen:poincare_friedrichs_ungleichung}
    Sei $\Omega \subset \mathbb{R}^{n}$ offen, beschränkt und mit Lipschitz-Rand.
    Dann existiert eine Konstante $\gamma_{\Omega} > 0$ mit
    \begin{equation}
        \label{eq:grundlagen:poincare_friedrichs_ungleichung}
        \norm{\grad u}_{L_{2}(\Omega)} \geq \gamma_{\Omega} \norm{u}_{H^{1}(\Omega)} \quad \fa u \in H^{1}_{0}(\Omega).
    \end{equation}
\end{Satz}

\begin{Satz}[Poincaré-Friedrichs-Ungleichung, vgl. {{\cite[Theorem II.1.7]{Braess:2007wm}}}]
    Es sei $\Omega \subset \mathbb{R}^{n}$ beschränkt und in einem $n$-dimensionalen Würfel mit Seitenlänge $s$ enthalten.
    Dann gilt
    \begin{equation}
        (1 + s)^{m} \abs{u}_{H^{m}} \geq \norm{u}_{H^{m}} \geq \abs{u}_{H^{m}} \quad \text{für alle}~u \in H^{m}_{0}(\Omega).
    \end{equation}
\end{Satz}

\begin{Lemma}[{{\cite[Remark 2.1.48]{Sauter:9_WoPZ0Y}}}]
\label{lemma:sauter:2.1.48}
    Seien $X$ und $Y$ zwei reflexive Banachräume und $a \colon X \times Y \to \mathbb{R}$ eine Bilinearform.
    Finden wir für jedes $x \in X$ ein $y_{x} \in Y$, so dass
    \begin{equation}
        \label{eq:lemma:sauter:2.1.48:eq1}
        \abs{a(x, y_{x})} \geq C_{1} \norm{x}_{X}^{2} \quad \text{und} \quad \norm{y_{x}}_{Y} \leq C_{2} \norm{x}_{X}
    \end{equation}
    mit von $x$ und $y_{x}$ unabhängigen Konstanten $C_{1}, C_{2} > 0$ gilt, dann folgt daraus die inf-sup-Bedingung
    \begin{equation}
    \label{eq:lemma:sauter:2.1.48:inf_sup}
        \inf_{0 \neq x \in X} \sup_{0 \neq y \in Y} \frac{a(x, y)}{\norm{x}_{X}\norm{y}_{Y}} \geq \gamma > 0
    \end{equation}
    mit $\gamma = \frac{C_{1}}{C_{2}}$.

    \begin{Beweis}
        Seien $x \in X$ und $y_{x} \in Y$ so, dass \cref{eq:lemma:sauter:2.1.48:eq1} erfüllt ist.
        Dann gilt
        \begin{align}
            \inf_{0 \neq x \in X} \sup_{0 \neq y \in Y} \frac{\abs{a(x, y)}}{\norm{x}_{X} \norm{y}_{Y}}
            &\geq
            \inf_{0 \neq x \in X} \frac{\abs{a(x, y_{x})}}{\norm{x}_{X} \norm{y_{x}}_{Y}}
            \\&\geq
            \inf_{0 \neq x \in X} \frac{C_{1} \norm{x}^{2}_{X}}{\norm{x}_{X} C_{2} \norm{x}_{X}}
            =
            \frac{C_{1}}{C_{2}}
            > 0.
        \end{align}
    \end{Beweis}
\end{Lemma}

% section sonstiges (end)

    % %!TEX root = ../main.tex

\chapter{Notizen} % (fold)
\label{cha:notizen}

% section zum_petrov_galerkin_verfahren (end)

\section{Reduzierte-Basis-Methode} % (fold)
\label{sec:reduzierte_basis_methode}

Bei der Reduzierte-Basis-Methode für die Raum-Zeit-Variationsformulierung parabolischer partieller Differentialgleichungen haben sich folgende Punkte ergeben, welche eine Anmerkung verdienen.

Seien dazu $\mathcal X^{\mathcal N} = \spn\Set{\phi_{n}}_{n=1}^{\mathcal N}$ und $\mathcal Y^{\mathcal M} = \spn\Set{\psi_{m}}_{m = 1}^{\mathcal M}$ endlichdimensionale Hilberträume, beispielsweise aus dem Petrov-Galerkin-Verfahren.
Im Allgeimeinen muss hier nicht $\mathcal N = \mathcal M$ gelten.
Weiter bezeichnen wir mit $\mathcal X_{N} \subset \mathcal X^{\mathcal N}$ und $\mathcal Y_{\mathcal M} \subset \mathcal Y_{M}$ die Reduzierte-Basis-Räume.

Wir betrachten das abstrakte Variationsproblem
\begin{equation}
    b(u, v; \mu) = f(v; \mu) \qquad \text{mit}~u \in \mathcal X,~v \in \mathcal Y,
\end{equation}
wobei $b \colon \mathcal X \times \mathcal Y \times \mathcal P \to \mathbb{R}$ eine affin parametrische Bilinearform und $f \colon \mathcal Y \times \mathcal P \to \mathbb{R}$ ein affin parametrisches lineares stetiges Funktional sei,
das heißt, es gelte
\begin{equation}
    b(u, v; \mu) = \sum_{q = 1}^{Q_b} \theta^{b}_{q}(\mu) b_{q}(u, v)
    \qquad \text{und} \qquad
    f(v; \mu) = \sum_{q = 1}^{Q_f} \theta^{f}_{q}(\mu) f_{q}(v).
\end{equation}

\paragraph{A posteriori Fehlerschätzer} % (fold)
\label{par:a_posteriori_fehlersch_tzer}

Der A-posteriori-Fehlerschätzer ergibt sich wie bei Reduzierte-Basis-Methoden üblich folgendermaßen:

Sei $\mu \in \mathcal P$ ein Parameter, $u^{\mathcal N}(\mu) \in \mathcal X^{\mathcal N}$ die Truth-Lösung des Variationsproblem für $\mu$ und $u_{N}(\mu) \in \mathcal X_{N}$ die entsprechende Reduzierte-Basis-Lösung.
Definiere den Fehler
\begin{equation}
    e_{N}(\mu) := u^{\mathcal N}(\mu) - u_{N}(\mu) \in \mathcal X^{\mathcal N}.
\end{equation}
Weiter wird das Residuum für alle $v \in \mathcal Y^{\mathcal M}$ definiert als
\begin{equation}
    r_{N}(v; \mu) = b(e_{N}(\mu), v; \mu) = f(v; \mu) - b(u_{N}(\mu), v; \mu).
\end{equation}
Fasst man das Residuum als rechte Seite des obigen Variationsproblems auf, dann ist $e_{N}(\mu)$ die zugehörige Lösung.
Weiter können wir die übliche Abschätzung (Lemma von Cea bzw. ähnliche Aussage) verwenden und erhalten die Ungleichung
\begin{equation}
    \norm{e_{N}(\mu)}_{\mathcal X} \leq \frac{1}{\beta^{\mathcal N}(\mu)} \norm{r_{N}(\blank; \mu)}_{\mathcal Y^{\mathcal M}'}.
\end{equation}
Dabei ist $\beta^{\mathcal N}(\mu)$ die inf-sup-Konstante des Truth-Probelms für den Parameter $\mu$.

\paragraph{Berechnung der inf-sup-Konstante} % (fold)
\label{par:berechnung_der_inf_sup_konstante}

Die obige inf-sup-Konstante $\beta^{\mathcal N}(\mu)$ wird in der Offline-Phase der Reduzierte-Basis-Methode für jeden Parameter $\mu$ des verwendeten Trainingsraums benötigt, muss also effizient auswertbar sein.

Zunächst eine Erklärung, wie man $\beta^{\mathcal N}(\mu)$ grundsätzlich berechnen kann.
Dazu greift man auf den sogenannten \emph{Supremizing Operator} zurück.
Dieser ist eine Abbildung $T_{\mu} \colon \mathcal X^{\mathcal N} \to \mathcal Y^{\mathcal M}$ definiert durch
\begin{equation}
    \skp{T_{\mu} u}{v}{\mathcal Y^{\mathcal M}} = b(u, v; \mu) \quad \fa v \in \mathcal Y^{\mathcal M}.
\end{equation}
Weiter gilt
\begin{equation}
    T_{\mu}u = \arg \sup_{v \in \mathcal{Y}^{\mathcal M}}  \frac{b(u, v; \mu)}{\norm{v}_{\mathcal Y^{\mathcal M}}}
\end{equation}
und damit
\begin{equation}
    \beta^{\mathcal N}(\mu) = \inf_{u \in \mathcal X^{\mathcal N}} \frac{\norm{T_{\mu}u}_{\mathcal Y^{\mathcal M}}}{\norm{u}_{\mathcal X^{\mathcal N}}}.
\end{equation}
Mittels des Rieszschen Darstellungssatzes lässt sich der Operator $T_{\mu}$ berechnen.

Sei dazu $\mat{Y} = [\skp{\psi_{m}}{\psi_{m'}}{\mathcal Y^{\mathcal M}}]_{m, m'}$, $\mat{X} = [\skp{\phi_{n}}{\phi_{n'}}{\mathcal X^{\mathcal N}}]_{n, n'}$ und $\mat{B}_{\mu} = [b(\phi_{n}, \psi_{m}; \mu)]_{m, n}$.
Dann gilt
\begin{equation}
    \mat{Y} \vec{T}_{\mu} = \mat{B}_{\mu}.
\end{equation}
Eingesetzt ergibt sich dann das Quadrat der inf-sup-Konstante dann als
\begin{equation}
    (\beta^{\mathcal N}(\mu))^{2} = \inf_{\vec{u} \in \mathbb{R}^{\mathcal N}} \frac{\vec{u}\Transp \mat{B}_{\mu}\tranps \mat{Y}^{-1} \mat{B}_{\mu} \vec{u}}{\vec{u}\Transp \mat{X} \vec{u}}
\end{equation}
und lässt sich als kleinster Eigenwert $\lambda$ des verallgemeinerten Eigenwertproblems
\begin{equation}
    \mat{B}_{\mu}\tranps \mat{Y}^{-1} \mat{B}_{\mu} \vec{x} = \lambda \mat{X} \vec{x}
\end{equation}
bestimmen.

Für die Successive Constraint Method, siehe \textcite{Huynh2007}.


% paragraph berechnung_der_inf_sup_konstante (end)

\paragraph{Stabiler Reduzierte-Basis-Testraum} % (fold)
\label{par:stabiler_reduzierte_basis_testraum}

Bei der Reduzierte-Basis-Methode für parabolische Probleme ergibt sich das Problem, dass die Lösungen nur den Reduzierte-Basis-Ansatzraum aufspannen. Der Testraum muss dagegen anderweitig konstruiert werden.
Hier scheint es verschiedene, hauptsächlich heuristisch motivierte Ansätze zu geben, die wiederum den obigen Supremizing Operator verwenden.
Beispiele sind \textcite[Abschnitt 4.2]{Mayerhofer:2014vx} beziehungsweise \textcite{Dahmen:2014cl}.

Momentan implementiert ist \textcite[Abschnitt 4.2]{Mayerhofer:2014vx}.
% paragraph stabiler_reduzierte_basis_testraum (end)


\section{Petrov-Galerkin} % (fold)
\label{sec:petrov_galerkin}

Die Zeitdiskretisierung wurde ausgetauscht. Statt Legendre-Polynomen werden nun, da es weit verbreitet zu sein scheint und laut \textcite{Andreev:2012uh,Andreev:2012ep,Andreev:2013gk} mit guten Stabilitätsergebnissen, nodale Hutfunktionen für den Ansatzraum und Indikatorfunktionen für den Testraum verwendet. Dabei wird im Allgeimeinen die Zeitdiskretisierung des Testraumes um den Faktor 2 verfeinert, wodurch sich die inf-sup-Stabilität ohne Beachtung einer CFL-Bedingung ergibt. (Die räumliche Diskretisierung muss ein, zwei Bedingungen erfüllen, mal checken).

Durch den größeren Testraum ergibt sich ein überbestimmtes System, welches im Sinne einer Residuum-Minimierung gelöst werden muss. Es lässt sich zeigen, dass dieses \emph{Minimales Residuum Petrov-Galerkin-Verfahren} ähnliche Aussagen wie das übliche Petrov-Galerkin-Verfahren erfüllt.

% section petrov_galerkin (end)

\section{Weiteres} % (fold)
\label{sec:weiteres}

\begin{itemize}
    \item Im Moment ist die periodische Randbedingung am laufen. Wird durch Fourier-Disrektisierung mit Konstanter Basisfunktion geregelt. Die Theorie wird aber für $H^{1}_{0, per}(\Omega) := H^{1}_{per}(\Omega) / \mathbb{R}$ angeregt. Schlimm? Wie anders lösbar?
    \item Wie könnte man die Anzahl der verwendeten Feld-Entwicklungsfunktionen einschränken?
    \item
\end{itemize}

% section weiteres (end)

    % %!TEX root = ../main.tex

\pagestyle{plain}

\section*{Eindimensionaler Fall mit $\omega \in \mathbb{R}$ und ohne Quellterm}

Sei $I := [0, \hat t]$ für ein $0 < \hat t < \infty$ und $\Omega := [0, 1]$.
Betrachte folgende parametrisierte PDE
\begin{align}
    \begin{cases}
    u_{t}(t, x) = \sigma u_{xx}(t, x) - \omega u(t, x), & (t, x) \in I \times \Omega\\
    u(0, x) = g(x), & x \in \Omega \\
    u(t, 0) = u(t, 1) = 0, & t \in I
    \end{cases}
\end{align}
mit Konstanten $\sigma, \omega \in \mathbb{R}$.

Ein Separation der Variablen Ansatz $u(t, x) = X(x) T(t)$ liefert
\begin{equation}
    X(x)T'(t) = \sigma X''(x) T(t) - \omega X(x) T(t)
\end{equation}
oder auch
\begin{equation}
    \frac{T'(t)}{T(t)} = \sigma \frac{X''(x)}{X(x)} - \omega = \lambda
\end{equation}
mit $\lambda \in \mathbb{R}$.

Ohne Einschränkung sei $\lambda \neq 0$, dann erhalten wir zum einen die Dgl.
    $T'(t) = \lambda T(t)$,
deren Lösung
\begin{equation}
    T(t) = d_{3} e^{\lambda t}
\end{equation}
ist, und zum anderen die Dgl.
    $X''(x) =  \frac{\lambda + \omega}{\sigma} X(x)$
mit der Lösung
\begin{equation}
    X(x) = d_{1} e^{\sqrt{\frac{\lambda + \omega}{\sigma}} x} + d_{2} e^{-\sqrt{\frac{\lambda + \omega}{\sigma}}x},
\end{equation}
wobei $d_{1}, d_{2}, d_{3} \in \mathbb{R}$.

Als nächstes Verwenden wir die Anfangs- und Randbedingungen um die Konstanten $d_{i}$ zu bestimmen.
Sei
\begin{equation}
    u(t, x) = \left( d_{1} e^{\sqrt{\frac{\lambda + \omega}{\sigma}} x} + d_{2} e^{-\sqrt{\frac{\lambda + \omega}{\sigma}}x} \right) \left( d_{3} e^{\lambda t} \right),
\end{equation}
Betrachten wir zunächst die Randbedingung $u(t, 0) = u(t, 1) = 0$, dann erhalten wir aus
\begin{equation}
    0 = u(t, 0) = \left( d_{1} + d_{2} \right) \left( d_{3} e^{\lambda t} \right),
\end{equation}
oder äquivalent $d_{1} = - d_{2}$, und aus 
\begin{equation}
    0 = u(t, 1) = \left( d_{1} e^{\sqrt{\frac{\lambda + \omega}{\sigma}}} + d_{2} e^{-\sqrt{\frac{\lambda + \omega}{\sigma}}} \right) \left( d_{3} e^{\lambda t} \right) = 
    d_{1} \left( e^{\sqrt{\frac{\lambda + \omega}{\sigma}}} - e^{-\sqrt{\frac{\lambda + \omega}{\sigma}}} \right) \left( d_{3} e^{\lambda t} \right),
\end{equation}
ohne Einschränkung $d_{1} \neq 0$, die Gleichung
\begin{equation}
    0 = e^{\sqrt{\frac{\lambda + \omega}{\sigma}}} - e^{-\sqrt{\frac{\lambda + \omega}{\sigma}}}.
\end{equation}
Aus dieser erhalten wir durch Äquivalenzumformungen
\begin{align}
    e^{\sqrt{\frac{\lambda + \omega}{\sigma}}} - e^{-\sqrt{\frac{\lambda + \omega}{\sigma}}} = 0
    &\quad \iff \quad
    e^{2\sqrt{\frac{\lambda + \omega}{\sigma}}} = 1
    \quad \iff \quad
    \sqrt{\tfrac{\lambda + \omega}{\sigma}} = k \pi i
    \\&\quad \iff \quad
    \tfrac{\lambda + \omega}{\sigma} = -k^2 \pi^2
    \quad \iff \quad
    \lambda = -k^2 \pi^2 \sigma - \omega,
\end{align}
mit $k \in \mathbb{Z}$ beliebig.
Einsetzen liefert nun
\begin{align}
    u_{k}(t, x) &= d_{1} \left( e^{k \pi i x} - e^{-k \pi i x} \right) \left( d_{3} e^{- (k^2 \pi^2 \sigma + \omega) t} \right)
    \\&= 2 d_{1} d_{3} i \sin(k \pi x) e^{-(k^2 \pi^2 \sigma + \omega)t},
\end{align}
wobei wir $\beta_{k} := 2 d_{1} d_{3} i$ setzen.

Da jedes $u_{k}$, $k \in \mathbb{Z}$, eine Lösung ist, erhalten wir durch
\begin{equation}
    u(t, x) = \sum_{k = 1}^{\infty} u_{k}(t, x) = \sum_{k = 1}^{\infty} \beta_{k} \sin(k \pi x) e^{-(k^2 \pi^2 \sigma + \omega)t}}
\end{equation}
ebenfalls eine Lösung. 
Damit die Anfangsbedingung erfüllt wird, muss
\begin{equation}
    g(x) = u(x, 0) = \sum_{k = 1}^{\infty} \beta_{k} \sin(k \pi x)
\end{equation}
gelten, was genau dann der Fall ist, wenn
\begin{equation}
    \beta_{k} = 2 \int_{0}^{1} g(x) \sin(k \pi x) \diff x.
\end{equation}

Da $u_{k}$ analytisch in $\omega$ für alle $k \in \mathbb{Z}$, ist auch $u$ analytisch in $\omega$ und es gilt
\begin{equation}
    \frac{\partial^{j} u(t, x; \omega)}{\partial \omega^{j}} = \sum_{k = 1}^{\infty} (-t)^{j} \beta_{k} \sin(k \pi x) e^{-(k^{2} \pi^{2} \sigma + \omega)t}.
\end{equation}


    %%% Anhang
    \appendix{}

    % DVD Inhalt
    \subfile{chapters/app_misc}
    \subfile{chapters/app_code}

    %% Los geht's mit den Verzeichnissen

    % Abbildungsverzeichnis
    % \listoffigures

    % Tabellenverzeichnis
    % \listoftables

    % Symbolverzeichnis
    % \glsaddall[]
    % \printglossary[type=symbolslist, nonumberlist, style=long]

    % Literaturverzeichnis
    \printbibliography

    % Eidesstattliche Erklärung
    % %!TEX root = ../main.tex

\chapter*{Eidesstattliche Erklärung}

Ich versichere hiermit, dass ich die vorliegende Masterarbeit selbständig
verfasst und keine anderen als die angegebenen Quellen und Hilfsmittel benutzt
habe, wobei ich alle wörtlichen und sinngemäßen Zitate als solche gekennzeichnet
habe. Die Arbeit wurde bisher keiner anderen Prüfungsbehörde vorgelegt und auch
nicht veröffentlicht.\\[6ex]

\begin{flushright}
\ort, den \today

\color{jgu_hellgrau}\hdashrule[-0.5cm]{5cm}{0.5pt}{1pt}
\end{flushright}

\end{document}
