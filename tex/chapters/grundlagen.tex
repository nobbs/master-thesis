%!TEX root = ../main.tex

\chapter{Grundlagen} % (fold)
\label{cha:grundlagen}

In diesem Kapitel fassen wir die für diese Arbeit benötigten Grundlagen aus verschiedenen mathematischen Bereichen, wie der Funktionalanalysis und der Numerik, zusammen.
Wir orientieren uns dabei maßgeblich an \cite{Dautray:1992by}.

\section{Bochner-Räume} % (fold)
\label{sec:bochner_r_ume}

Wir beginnen mit der Einführung sogenannter Bochner-Räume.
Dabei handelt es sich um Verallgemeinerungen der Lebesgue-Räume $L_{p}$ auf Banachraum-wertige Funktionen.

\begin{Definition}[Bochner-Raum, {{{\cite[Def. XVIII.1.1]{Dautray:1992by}}}}]
\label{definition:gl:bochner_raum}
    Sei $X$ ein Banachraum, $- \infty \leq a < b \leq \infty$ und $1 \leq p < \infty$.
    Als Bochner-Raum $L_{p}(a, b; X)$ bezeichnen wir die Menge (der Äquivalenzklassen) $L_{p}$-integrierbarer Funktionen $f \colon [a, b] \to X$, das heißt aller Lebesgue-messbarer Funktionen auf $[a, b]$ mit
    \begin{equation}
        \norm{f}_{L_{p}(a, b; X)} \deq \left( \int_{a}^{b} \norm{f(t)}_{X}^{p} \diff t \right)^{\frac{1}{p}} < \infty.
    \end{equation}
    Weiterhin ist der Bochner-Raum $L_{\infty}(a, b; X)$ definiert als die Menge (der Äquivalenzklassen) der für fast alle $t \in [a, b]$ wesentlich beschränkten Funktionen mit
    \begin{equation}
        \norm{f}_{L_{\infty}(a, b; X)} \deq \esssup_{t \in [a, b]} \norm{f(t)}_{X} < \infty.
    \end{equation}
\end{Definition}

\begin{Lemma}[{{\cite[Prop. XVIII.1.1]{Dautray:1992by}}}, {{\cite[Abs. 1.1.3]{Lions:1972tg}}}]
\label{lemma:gl:bochner_ist_banach}
    Für $1 \leq p \leq \infty$ ist $L_{p}(a, b; X)$ ein Banachraum.
    Ist $H$ ein Hilbertraum, dann ist $L_{2}(a, b; H)$ ebenfalls ein Hilbertraum.
\end{Lemma}

\begin{Definition}
\label{definition:gl:stetige_funktionen}
    Mit $\mathcal C^{0}(a, b; X)$ bezeichnen wir die Menge aller bezüglich der Norm $\norm{f} = \sup_{t \in [a, b]} \norm{f(t)}_{X}$ stetigen Funktionen $f \colon [a, b] \to X$.
\end{Definition}

\begin{Definition}[Schwache Zeitableitung, {{{\cite{Dautray:1992by}}}}]
\label{definition:gl:schwache_zeitableitung}
    Seien $X$ und $Y$ Banachräume mit $X \hookrightarrow Y$ und $u \in L_{2}(a, b; X)$.
    Die distributionelle Ableitung $\frac{\partial}{\partial t} u \in L_{2}(a, b; Y)$ sei definiert als das $v \in L_{2}(a, b; Y)$, welches
    \begin{equation}
        \int_{a}^{b} v(t) \varphi(t) \diff t = - \int_{a}^{b} u(t) \frac{\partial}{\partial t} \varphi(t) \diff t \quad \fa \phi \in C^{\infty}_{0}(a, b)
    \end{equation}
    erfüllt, falls ein solches $v$ existiert.
\end{Definition}

\begin{Bemerkung*}
    Je nach Situation werden wir der Einfachheit halber eine der Schreibweisen $\frac{\partial}{\partial t} u = u' = u_{t}$ verwenden.
\end{Bemerkung*}

% section bochner_r_ume (end)

Im Zuge dieser Arbeit werden wir fast immer mit Hilberträumen zu tun haben.
Dabei werden wir oft auf folgendes Konstrukt zurückgreifen.

\begin{Definition}[Gelfand-Tripel]
\label{definition:gl:gelfand_tripel}
    Seien $V$ und $H$ separable Hilberträume mit den Dualräumen $V'$ und $H'$.
    Weiter sei $V$ ein dichter Unterraum von $H$.
    Durch Identifikation von $H$ mit seinem Dualraum $H'$ erhalten wir das sogenannte Gelfand-Tripel
    \begin{equation}
        V \hookrightarrow H \cong H' \hookrightarrow V',
    \end{equation}
    wobei die Inklusionen jeweils dichte stetige Einbettungen sind.
\end{Definition}

\begin{Bemerkung*}
    Mit $\skprod{\blank}{\blank}_{V}$ und $\skprod{\blank}{\blank}_{H}$ bezeichnen wir das Skalarprodukt auf $V$ respektive $H$.
    Weiterhin $\skprod{\blank}{\blank}_{V' \times V}$ wird auch für die duale Paarung auf $V' \times V$, die als die eindeutige stetige Fortsetzung von $\skprod{\blank}{\blank}_{H}$ definiert ist, verwendet.
    Insbesondere gilt für $u \in H$ und $v \in V$ die Gleichheit
    \begin{equation}
        \skprod{u}{v}_{V' \times V} = \skprod{u}{v}_{H}.
    \end{equation}
\end{Bemerkung*}

\begin{Definition}
    Sei $V \hookrightarrow H \hookrightarrow V'$ ein Gelfand-Tripel.
    Definiere den Raum $W(a, b)$ als
    \begin{equation}
        W(a, b) \deq \Set{u \in L_{2}(a, b; V) \given u' \in L_{2}(a, b; V')}
    \end{equation}
    wobei $u'$ im Sinne von \thref{definition:gl:schwache_zeitableitung} zu verstehen ist.
\end{Definition}

\begin{Lemma}
    Zusammen mit der Norm
    \begin{equation}
        \norm{u}_{W(a, b)} \deq \left( \int_{a}^{b} \norm{u'(t)}_{V'}^{2} \diff t + \int_{a}^{b} \norm{u(t)}_{V}^{2} \diff t \right)^{\frac 12}
        = \left( \norm{u'}_{L_{2}(a, b; V')}^{2} + \norm{u}_{L_{2}(a, b; V)}^{2} \right)^{\frac 12},
    \end{equation}
    und dem induzierten Skalarprodukt TODO ist $W(a, b)$ ein Hilbertraum.
\end{Lemma}

% chapter grundlagen (end)


